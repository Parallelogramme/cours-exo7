
\documentclass[11pt, class=report,crop=false]{standalone}
\usepackage[screen]{../exo7book}


\begin{document}

%====================================================================
\chapitre{Difféomorphismes}
%====================================================================


%%%%%%%%%%%%%%%%%%%%%%%%%%%%%%%%%%%%%%%%%%%%%%%%%%%%%
\section{Difféomorphismes}

%----------------------------------------------------
\subsection{Définition}

Les applications de $\Rr^n$ dans $\Rr^n$ qui sont bijectives et de classe $\mathcal{C}^1$ ainsi que leur réciproque sont utilisées comme changements de variables. On les appelle des difféomorphismes.

\begin{definition}
Soient $U$ et $V$ deux ouverts de $\Rr^n$ et $\Phi:U\to V$.
On dit que $\Phi$ est un \defi{difféomorphisme} si :
\begin{enumerate}
    \item $\Phi$ est une bijection de $U$ sur $V$,
    \item $\Phi$ est de classe $\mathcal{C}^1$ sur $U$, c'est-à-dire différentiable sur $U$, de différentielle continue,
    \item et $\Phi^{-1}$ est de classe $\mathcal{C}^1$ sur $V$.
\end{enumerate}
\end{definition}

Remarque : on peut être plus précis en parlant d'un $\mathcal{C}^1$-difféomorphisme.

\begin{exemple}
\sauteligne
\begin{enumerate}
    \item $f : ]0,+\infty[ \to ]0,+\infty[$ avec $f(x) = x^3$ est un difféomorphisme. Son inverse est $f^{-1} : \Rr \to \Rr$ définie par $f^{-1}(x) = x^{\frac13}$. $f$ et $f^{-1}$ sont bien des applications de classe $\mathcal{C}^1$.
    
    \item Un exemple fondamental qu'on développera dans ce chapitre est le passage en coordonnées polaires : on remplace les coordonnées cartésiennes $(x,y)$ par les coordonnées polaires $(r,\theta)$.  Il faut exclure certaines parties du plan afin d'obtenir un difféomorphisme.
    Soit $U = \{ (r,\theta) \in \Rr^2 \mid r>0 \text{ et } 0 < \theta < 2\pi \}$ et $V = \Rr^2 \setminus (\Rr_+ \times \{ 0 \})$. Alors
    $\Phi : U \to V$ défini par
    $$\Phi(r,\theta) = (r\cos\theta, r \sin\theta)$$
    est un difféomorphisme.
    Trouver son inverse revient à exprimer les coordonnées polaires $(r,\theta)$ en fonction des coordonnées cartésiennes $(x,y)$.

\myfigure{0.8}{
    \tikzinput{fig-diffeo-01}
}    
    
\end{enumerate}    
\end{exemple}
    

%----------------------------------------------------
\subsection{Difféomorphisme et jacobienne}

Du théorème donnant la matrice jacobienne d'une composée découle que les matrices jacobiennes de $\Phi$ et $\Phi^{-1}$ sont inverses l'une de l'autre.

Soient $U$ et $V$ deux ouverts de $\Rr^n$ et $\Phi:U\to V$.
Soient $x \in U$ et $y = \Phi(x) \in V$.
Notons $J_\Phi(x)$ la matrice jacobienne de $\Phi$ en $x$ 
et $J_{\Phi^{-1}}(y)$ la matrice jacobienne de $\Phi^{-1}$ en $y$.
Ce sont des matrices carrées $n\times n$.

\begin{proposition}
\label{prop:invjacob}
~
\mybox{$J_{\Phi^{-1}}(y) = \left( J_\Phi(x) \right)^{-1}$}
\end{proposition}
Comme on l'a dit : la matrice jacobienne de l'inverse est l'inverse de la matrice jacobienne !

\begin{proof}
Notons $J = J_\Phi(x)$, $y = \Phi(x)$,  $\Psi = \Phi^{-1}$ et $K = J_\Psi(y)$.
Comme $\Psi$ est l'inverse de $\Phi$ alors :
$$(\Psi \circ \Phi) (x) = x.$$
On applique la formule de la composition des jacobiennes :
$$J_{\Psi \circ \Phi}(x) = J_\Psi(\Phi(x)) \times J_\Phi(x).$$
L'application $\Psi \circ \Phi$ est l'identité, donc sa matrice jacobienne est la matrice identité : $J_{\Psi \circ \Phi}(x) = I$.
Quant au terme de droite, c'est exactement $K \times J$.
Ainsi $K J = I$ et donc $K = J^{-1}$.
\end{proof}


\begin{exemple}[Cas d'une variable]
Soit $f : I  \to J$ une fonction d'une variable où $I$ et $J$ sont des intervalles ouverts de $\Rr$. 
On suppose que $f$ est une fonction de classe $\mathcal{C}^1$, bijective et que la dérivée de $f$ ne s'annule pas sur $I$. Alors, pour $y\in J$, on retrouve la formule connue :
$$\left(f^{-1}\right)'(y) = \frac{1}{f'(x)}.$$

Rappelons que, pour une fonction d'une variable, la matrice jacobienne $J_f(x)$ est la matrice $1 \times 1$ avec pour seul coefficient $f'(x)$.
Par exemple, pour $f(x) = \sin(x)$ en $x_0=\frac\pi3$, on sait que $y_0=\sin(\frac\pi3)=\frac{\sqrt3}2$ et $f'(x_0)=\cos(\frac\pi3)=\frac12$.
La bijection réciproque est $f^{-1}(y)=\arcsin(y)$, et on peut calculer $\arcsin'(\frac{\sqrt3}2)$ directement :
$$\arcsin'\left(\frac{\sqrt3}2\right) = \arcsin'(y_0) = \frac{1}{\sin'(x_0)} = 2.$$
\end{exemple}

\begin{exemple}
Soit $F : \Rr^2 \to \Rr^2$ définie par $F(x,y) = (x-y^2,y+1)$.

\begin{enumerate}
     \item \emph{Matrice jacobienne de $F$.}
Notons $f_1(x,y)=x-y^2$ et $f_2(x,y)=y+1$.
Alors la matrice jacobienne de $F$ est :
$$J_F(x,y) = 
\begin{pmatrix}
\frac{\partial f_1}{\partial x} & \frac{\partial f_1}{\partial y} \\
\frac{\partial f_2}{\partial x} & \frac{\partial f_2}{\partial y} \\
\end{pmatrix}
= 
\begin{pmatrix}
1 & -2y \\
0 & 1 \\
\end{pmatrix}$$

     \item \emph{Inverse de $F$.}
     Notons $F(x,y) = (X,Y)$. Alors $F$ est bijective et son inverse est 
     $F^{-1}(X,Y) = G(X,Y) = (X+(Y-1)^2, Y-1)$ : à vous de vérifier qu'on a bien $(G \circ F)(x,y) = (x,y)$ et  $(F \circ G)(X,Y) = (X,Y)$.
     
     \item \emph{Matrice jacobienne de $F^{-1}$.}
     Par la formule $G(X,Y) = (g_1(X,Y),g_2(X,Y))$ définissant l'inverse, on a directement :
     $$J_{F^{-1}}(X,Y) = 
     \begin{pmatrix}
     \frac{\partial g_1}{\partial X} & \frac{\partial g_1}{\partial Y} \\
     \frac{\partial g_2}{\partial X} & \frac{\partial g_2}{\partial Y} \\
     \end{pmatrix}
     = 
     \begin{pmatrix}
     1 & 2(Y-1) \\
     0 & 1 \\
     \end{pmatrix}$$
     
     \item \emph{Matrice jacobienne de $F^{-1}$ (encore).}    
     Une autre façon d'obtenir la matrice jacobienne de $F^{-1}$ est de d'abord calculer l'inverse de  $J_F(x,y)$ (qui est inversible car de déterminant $1$) puis on applique la proposition \ref{prop:invjacob}, en se souvenant que $F(x,y) = (X,Y)$
     avec $Y=y+1$ et donc $y=Y-1$ :
     
     $$J_{F^{-1}}(X,Y) = \big(J_F(x,y)\big)^{-1} 
     = \begin{pmatrix}
     1 & +2y \\
     0 & 1 \\
     \end{pmatrix}
     = \begin{pmatrix}
     1 & 2(Y-1) \\
     0 & 1 \\
    \end{pmatrix}     
     $$
     On retrouve le résultat précédent.
     \end{enumerate}    

\end{exemple}


%----------------------------------------------------
\subsection{Déterminant jacobien}

Pour un difféomorphisme, le déterminant de la matrice jacobienne joue un rôle particulier.

On appelle \defi{déterminant jacobien} de $\Phi$, en $x \in U$, le déterminant de la matrice jacobienne de $\Phi$ en $x$ :
$$\det \left( J _{\Phi}(x) \right).$$

Il est clair que le déterminant jacobien d'un difféomorphisme ne s'annule pas, puisque la matrice jacobienne est inversible. La réciproque sera donnée par le théorème d'inversion locale.


%----------------------------------------------------
\begin{miniexercices}
    \sauteligne
    \begin{enumerate}
        \item Soit $\Phi : \Rr^n \to \Rr^n$ une application linéaire, de matrice $A$ inversible. Monter que $\Phi$ est un difféomorphisme (on sait qu'une application linéaire est continue). Quelle est la matrice de $\Phi$ ? Et celle de $\Phi^{-1}$ ?
        
        \item Le changement de coordonnées cylindriques est le changement de variables $(x,y,z)=(r\cos\theta,r\sin\theta,z)$. On se limite ici au domaine $U$ défini par $r>0$, $\theta \in [0,\frac\pi2]$, $z\in\Rr$. Calculer l'ensemble $V$ des $(x,y,z)$ correspondants. On définit $\Phi : U \to V$ par $(r,\theta,z) \mapsto (x,y,z)$. Montrer que $\Phi$ est un difféomorphisme. Calculer sa matrice jacobienne et le déterminant jacobien. Calculer la matrice jacobienne de $\Phi^{-1}$.
    \end{enumerate}
\end{miniexercices}



%%%%%%%%%%%%%%%%%%%%%%%%%%%%%%%%%%%%%%%%%%%%%%%%%%%%%
\section{Théorème d'inversion locale}

Le théorème d'inversion locale et celui d'inversion globale permettent de montrer qu'une application est un difféomorphisme.

%----------------------------------------------------
\subsection{Théorème d'inversion locale}

\begin{theoreme}[Théorème d'inversion locale]
Soient $U$ et $V$ deux ouverts de $\Rr^n$ et $\Phi:U\to V$.
Soit $x \in U$.
Si :
\begin{enumerate}
    \item $\Phi:U\to V$ est de classe $\mathcal{C}^1$,
    \item \label{it:invloc} et si la matrice jacobienne $J_\Phi(x)$ est inversible,
\end{enumerate}
alors $\Phi$ est localement inversible, c'est-à-dire il existe un ouvert $U_x$ contenant $x$ et un ouvert $V_y$ contenant $y=\Phi(x)$ tels que $\Phi : U_x \to V_y$ soit un difféomorphisme.
\end{theoreme}

Nous admettons ce théorème dont la preuve est assez technique.

\bigskip

On obtient deux autres formulations équivalentes, si on remplace la condition \ref{it:invloc}. par la condition :
\begin{itemize}
    \item[2'.] le déterminant jacobien $\det J_\Phi(x)$ est non nul,
\end{itemize}
ou, si on préfère le langage des différentielles :
\begin{itemize}
    \item[2''.] la différentielle $\dd F (x) : \Rr^n \to \Rr^n$ est une application linéaire inversible.
\end{itemize}

\bigskip

Conclusion : on sait qu'une application linéaire $L : \Rr^n \to \Rr^n$ est inversible si le déterminant de la matrice $A$ associée est non nul. Le théorème d'inversion locale est donc une extension puissante de ce résultat pour une application non linéaire : si le déterminant jacobien est non nul alors l'application est (localement) inversible.


%----------------------------------------------------
\subsection{Théorème d'inversion globale}

Le théorème d'inversion globale est encore plus intéressant car il conduit à une fonction qui est partout un difféomorphisme.

\begin{theoreme}[Théorème d'inversion globale]
Soient $U$ et $V$ deux ouverts de $\Rr^n$ et $\Phi:U\to V$.
Si :
\begin{enumerate}
     \item $\Phi:U\to V$ est de classe $\mathcal{C}^1$,
     \item $\Phi:U\to V$ est bijective,
     \item et le déterminant jacobien $\det J_\Phi(x)$ est non nul, pour tout $x \in U$, 
    \end{enumerate}
    alors $\Phi : U \to V$ est un difféomorphisme.
\end{theoreme}

C'est une application du théorème d'inversion locale en tout point de l'ensemble de définition.

%----------------------------------------------------
\subsection{Exemple}



Les passages en coordonnées polaires, cylindriques ou sphériques sont très souvent utilisés. Détaillons le premier qui consiste à remplacer les coordonnées cartésiennes $(x,y)$ d'un point du plan par le module $r$ et l'argument $\theta$ du point dans le plan complexe.

  
\textbf{Coordonnées polaires.}

\begin{itemize}
    \item \emph{Définition de $\Phi$.}

$$\begin{array}{ccccl}\Phi &:&U=]0,+\infty [\times ]0,2\pi [
&\to&V=\Rr^2\setminus \left(\Rr_+\times\{0\}\right)\\
&&(r,\theta)&\mapsto &(x,y)=(r\cos\theta, r \sin\theta)
\end{array}$$
  
Clairement, $\Phi$ est une application de classe $\mathcal{C}^1$.
   
\begin{center} 
\begin{minipage}{0.3\textwidth}      
\myfigure{0.8}{
    \tikzinput{fig-diffeo-02}
}
\end{minipage}\qquad\qquad
\begin{minipage}{0.55\textwidth}  
\myfigure{0.7}{
    \tikzinput{fig-diffeo-01}
} 
\end{minipage}  
\end{center}
   
  \item \emph{Inverse de $\Phi$.}    
  Il s'agit d'exprimer $(r,\theta)$ en fonction de $(x,y)$.
  On sait $x^2+y^2 = r^2$, donc $r = \sqrt{x^2+y^2}$.
  
  On note $\theta \in ]0,2\pi[$ l'angle  entre l'axe des abscisses et le point $(x,y)$. Il est défini de manière unique (et pourrait être déterminé par des formules : par exemple, $\theta = \arctan(\frac y x)$ pour $x>0$, $y>0$).
  
  L'inverse de $\Phi$ est alors $\Phi^{-1} : V \to U$ défini par 
  $\Phi^{-1}(x,y) = (r,\theta)$.
  
  \item \emph{Matrice jacobienne de $\Phi$.}

$$J_\Phi(r,\theta)
= 
\begin{pmatrix}\frac{\partial (r\cos\theta)}{\partial r} & \frac{\partial (r\cos\theta)}{\partial \theta} \\
\frac{\partial (r\sin\theta)}{\partial r} & \frac{\partial (r\sin\theta)}{\partial \theta} \\
\end{pmatrix}
= 
\begin{pmatrix}
\cos \theta &-r\sin \theta \\
\sin \theta &r\cos \theta \\
\end{pmatrix}$$
  
    
\item \emph{Déterminant jacobien de $\Phi$.}   
$$\det J_\Phi(r,\theta) = \det \begin{pmatrix}
\cos \theta &-r\sin \theta \\
\sin \theta &r\cos \theta \\
\end{pmatrix}
= r\cos^2\theta + r\sin^2\theta = r$$

    
\item \emph{Théorème d'inversion locale.} 
  Comme $r>0$ sur $U$, le déterminant jacobien ne s'annule jamais donc, dans un voisinage de chaque point $(r,\theta) \in U$, $\Phi$ est localement un difféomorphisme. Ainsi, $\Phi$ est inversible (on l'avait déjà dit), mais en plus $\Phi^{-1}$ est de classe $\mathcal{C}^1$ (on ne le savait pas car on n'avait pas explicité $\Phi^{-1}$).
  


\item \emph{Théorème d'inversion globale.}
  On peut faire mieux en appliquant le théorème d'inversion globale (vérifier les hypothèses !) : $\Phi : U \to V$ est un difféomorphisme. On retrouve que $\Phi$ est inversible d'inverse de classe $\mathcal{C}^1$ sur $V$.
  
  
\item \emph{Matrice jacobienne de $\Phi^{-1}$.}

$$J_{\Phi^{-1}}(x,y) = \big( J_\Phi(r,\theta) \big)^{-1}
= 
\frac1r
\begin{pmatrix}
r\cos \theta &r\sin \theta \\
-\sin \theta &\cos \theta \\
\end{pmatrix}
= 
\begin{pmatrix}
\frac{x}{\sqrt{x^2+y^2}} & \frac{y}{\sqrt{x^2+y^2}} \\
-\frac{y}{x^2+y^2} & \frac{x}{x^2+y^2} \\
\end{pmatrix} 
$$
où on a utilisé $x=r\cos\theta$, $y=r\sin\theta$ et $r=\sqrt{x^2+y^2}$. 

\end{itemize}  
  
  \bigskip
  
\textbf{Application.}


Considérons maintenant une application $(x, y)\mapsto f(x,y)$ de $U$ (défini ci-dessus) dans $\Rr$. 
Pour \og{}passer en coordonnées polaires\fg{}, on doit remplacer les anciennes coordonnées $(x,y)$ par les nouvelles coordonnées $(r,\theta)$. 
On compose par la fonction $\Phi$ pour définir :
$$g(r,\theta) = (f \circ \Phi) (r,\theta)=f\left(r\cos\theta,r\sin\theta\right).$$

Supposons que l'on connaisse les dérivées partielles 
$\frac{\partial f}{\partial x}$ et $\frac{\partial f}{\partial y}$.
Nous avons besoin de calculer les dérivées partielles de la nouvelle fonction $g$ : 
$\frac{\partial g}{\partial r}$ et $\frac{\partial g}{\partial \theta}$.
Elles sont données par les formules :

\mybox{
$\everymath={\displaystyle}
\left\{\begin{array}{rcl}
\frac{\partial g}{\partial r}(r,\theta) & = &
\cos\theta \frac{\partial f}{\partial x}(x,y) + \sin\theta
\frac{\partial f}{\partial y}(x,y) \\[4mm]
\frac{\partial g}{\partial \theta}(r,\theta) & = &
-r\sin\theta \frac{\partial f}{\partial x}(x,y) + r\cos\theta \frac{\partial f}{\partial y}(x,y)
\\
\end{array}\right.$    
}
Dans ces formules, pour n'avoir que des variables $r$ et $\theta$, il faut substituer $x = r\cos\theta$ et $y=r\sin\theta$.

\bigskip

\emph{Preuve.}
D'une part, la matrice jacobienne de $g$ est la matrice ligne :
$$J_g(r,\theta) = 
\begin{pmatrix}
\frac{\partial g}{\partial r}(r,\theta) &
\frac{\partial g}{\partial \theta}(r,\theta) \\
\end{pmatrix}$$

Mais d'autre part, comme $g(r,\theta)= (f \circ \Phi) (r,\theta)$, alors
$$ J_g(r,\theta) = J_f\big( \Phi(r,\theta) \big) \times J_\Phi(r,\theta) .$$
On sait que $\Phi(r,\theta) = (x,y)$, on a déjà calculé $J_\Phi(r,\theta)$ et la matrice jacobienne de $f$ est la matrice ligne :
$$
J_f(x,y) = 
\begin{pmatrix}
\frac{\partial f}{\partial x}(x,y) &
\frac{\partial f}{\partial y}(x,y) \\
\end{pmatrix}
$$

La formule résulte alors du produit de matrices 
$J_g(r,\theta) = J_f(x,y) \times J_\Phi(r,\theta)$.

\bigskip

\emph{Exercice.}
Exprimer $\frac{\partial f}{\partial x}$ et $\frac{\partial f}{\partial y}$ en fonction de $\frac{\partial g}{\partial r}$ et $\frac{\partial g}{\partial \theta}$. 
Indication : $f(x,y) = (g \circ \Phi^{-1}) (x,y)$ et nous avons déjà calculé $J_{\Phi^{-1}}(x,y)$.


\bigskip

%----------------------------------------------------
\begin{miniexercices}
\sauteligne
\begin{enumerate}
  \item Soit $U = \Rr^2 \setminus \{ (0,0) \}$ le plan privé de l'origine et soit $\Phi(x,y)=(x^2-y^2,2xy)$.  
  Montrer que $\Phi$ est un difféomorphisme local au voisinage de
  tout point de $U$. Montrer que $\Phi$ n'est pas un difféomorphisme global (indication : montrer que $\Phi$ n'est pas injective).
  
  \item Soit $U = \Rr \times ]0,2\pi[$
  et soit $\Phi : U \to \Rr^2$ l'application définie
  par $(x,y) \mapsto (e^x\cos y,e^x\sin y)$.
  Montrer que $\Phi$ est un difféomorphisme local au voisinage de
  tout point de $U$.  Montrer que $\Phi$ est injective. Déterminer $\Phi(U)$ et vérifier que c'est un ouvert. En déduire que $\Phi : U \to \Phi(U)$ est un difféomorphisme (global).
\end{enumerate}
\end{miniexercices}




%%%%%%%%%%%%%%%%%%%%%%%%%%%%%%%%%%%%%%%%%%%%%%%%%%%%%
\section{Théorème des fonctions implicites}

%----------------------------------------------------
\subsection{Motivation}

Le théorème des fonctions implicites est assez technique et son énoncé est difficile à comprendre, cependant l'idée  sous-jacente est assez simple.
Comprenons donc d'abord le théorème sur un exemple.
Il s'agit de remplacer l'étude d'une fonction de deux variables par une fonction d'une seule variable.
Partons de la fonction de deux variables $F : \Rr^2 \to \Rr$ définie par $F(x,y) = x^2+y^2-1$.
On souhaite étudier l'ensemble $(F(x,y)=0)$, c'est-à-dire l'ensemble des points $(x,y) \in \Rr^2$ tels que $x^2+y^2=1$.
Géométriquement, c'est donc le cercle $\mathcal{C}$ de rayon $1$, centré à l'origine.

\myfigure{0.8}{
    \tikzinput{fig-diffeo-03}
}


\bigskip

Plaçons-nous autour du point $(0,1) \in \mathcal{C}$. Alors, dans un voisinage de ce point, la portion de cercle $\mathcal{C}$ est le graphe $y = \varphi(x)$ d'une fonction $x \mapsto \varphi(x)$. On ne cherche pas à expliciter cette fonction, mais en fait c'est $\varphi(x) = \sqrt{1-x^2}$.

\myfigure{0.8}{
    \tikzinput{fig-diffeo-04}
}


\bigskip

Est-ce qu'on peut faire ce travail pour tous les points du cercle ? 
Presque ! C'est possible pour tous les points, sauf autour des points $(1,0)$ et $(-1,0)$. Autour de ces deux points, la portion de cercle n'est pas le graphe d'une fonction de $x$. L'explication, c'est qu'en ces points il y a une tangente verticale, et que le graphe d'une fonction de $x$ dérivable n'a jamais de tangente verticale.


\myfigure{0.8}{
    \tikzinput{fig-diffeo-05}
}

\bigskip

Comment détecter les points de tangence verticale d'une courbe quelconque définie par $(F(x,y)=0)$ ?
\mybox{Les points de tangence verticale sur $(F(x,y)=0)$ sont les points où 
$\frac{\partial F}{\partial y}(x,y)=0$ (et $\frac{\partial F}{\partial x}(x,y)\neq0$).}


\myfigure{0.8}{
    \tikzinput{fig-diffeo-06}
}

C'est une formule valable pour n'importe quelle fonction $F :\Rr^2 \to \Rr$.
Faites le calcul pour notre fonction : si $x^2+y^2=1$ et $\frac{\partial F}{\partial y}(x,y)=0$, alors $(x,y)=(\pm1,0)$.

\bigskip

Comment faire pour les deux points spéciaux, par exemple en $(1,0)$ ? On peut tout simplement considérer la portion de cercle $\mathcal{C}$ autour de $(1,0)$ comme le graphe d'une fonction de $y$ : $y \mapsto \tilde\varphi(y)$. C'est un peu déroutant car les rôles de $x$ et de $y$ sont inversés ; c'est comme si on avait tourné la feuille de $90^\circ$.

\begin{center} 
    \begin{minipage}{0.3\textwidth}      
        \myfigure{0.8}{
            \tikzinput{fig-diffeo-07}
        }
    \end{minipage}\qquad\qquad
    \begin{minipage}{0.55\textwidth}  
        \myfigure{0.7}{
            \tikzinput{fig-diffeo-08}
        } 
    \end{minipage}  
\end{center}


En fait, le cercle $\mathcal{C}$ est localement le graphe d'une fonction $y \mapsto \tilde\varphi(y)$, sauf aux points ayant une tangente \emph{horizontale}, qui sont donnés cette fois par l'équation $\frac{\partial F}{\partial x}(x,y)=0$ (et $\frac{\partial F}{\partial y}(x,y)\neq0$).


\bigskip


De façon plus générale, pour $F : \Rr^n \to \Rr$ une fonction de $n$ variables, on remplace l'étude de $(F(x_1,\ldots,x_n)=0)$ par l'étude d'une fonction $\varphi$ de $n-1$ variables, à condition que l'une des dérivées partielles ne s'annule pas.
Le théorème des fonctions implicites est surtout un résultat théorique, car la fonction $\varphi$ n'est pas fournie explicitement.



%----------------------------------------------------
\subsection{Cas $F(x,y) = 0$}

Soit $F : \Rr^2 \to \Rr$. On considère la courbe de niveau 
$\mathcal{C}$ : $\big( F(x,y) = 0 \big)$.

On dit que la fonction $y = \varphi(x)$ est \defi{définie implicitement} par $F(x,y) = 0$ si $F(x,\varphi(x)) = 0$, c'est-à-dire si $(x,\varphi(x)) \in \mathcal{C}$.
On dit alors que $y = \varphi(x)$ est une \defi{fonction implicite} de $F(x,y) = 0$.

    
\begin{theoreme}[des fonctions implicites]
Soient $F : \Rr^2 \to \Rr$ une fonction de classe $\mathcal{C}^1$ et $(x_0,y_0)$ un point tel que $F(x_0,y_0) = 0$.
Si 
\mybox{$\displaystyle \frac{\partial F}{\partial y}(x_0,y_0) \neq 0$}
alors :
\begin{enumerate}
    \item Il existe une fonction $\varphi : I \to J$ de classe $\mathcal{C}^1$ définissant une fonction implicite $y = \varphi(x)$, où $I$ est un intervalle ouvert contenant $x_0$, $J$ est un intervalle ouvert contenant $y_0$ et $y_0 = \varphi(x_0)$.
    Plus précisément, pour tout $(x,y) \in I \times J$, on a :
    \mybox{$F(x,y) = 0 \iff y = \varphi(x)$}
    En particulier : 
    \mybox{pour tout $x \in I$,\quad  $F(x,\varphi(x)) = 0$}
    
    
    \item De plus, pour tout $x \in I$, on a $\frac{\partial F}{\partial y}(x,\varphi(x)) \neq 0$, et la dérivée de $\varphi$ est donnée par :
    \mybox{$\displaystyle \varphi'(x) = \frac{-\frac{\partial F}{\partial x}(x, \varphi(x))}{\frac{\partial F}{\partial y}(x, \varphi(x))}$}

\end{enumerate}
\end{theoreme}


On a un énoncé symétrique en échangeant $x$ et $y$ si c'est l'autre dérivée partielle qui ne s'annule pas.
Si $\frac{\partial F}{\partial x}(x_0,y_0) \neq 0$, 
il existe une fonction $\tilde\varphi : J \to I$ 
de la variable $y$, définie autour de $y_0$, qui définit la fonction implicite $\tilde\varphi(y)$, de sorte que dans un voisinage ouvert de $(x_0,y_0)$ :
$$F(x,y) = 0 \iff x = \tilde\varphi(y)$$
(ce qui implique ici $F(\tilde\varphi(y),y)=0$).



Si les deux dérivées partielles $\frac{\partial F}{\partial x}(x_0,y_0)$ et $\frac{\partial F}{\partial y}(x_0,y_0)$ s'annulent simultanément, le point $(x_0,y_0)$ s'appelle un \defi{point singulier} et le théorème des fonctions implicites ne s'applique pas.

%----------------------------------------------------
\subsection{Exemple}

Soit 
$$F(x,y) = x(x^2+y^2)-(x^2-y^2).$$

\textbf{Courbe $(F(x,y)=0)$.}

Commençons par tracer la courbe $\mathcal{C}$ d'équation $F(x,y)=0$, ce qui nous permettra d'anticiper les calculs.
La courbe forme une boucle. On note qu'il y a une tangente verticale au point $(1,0)$.
Le point $(0,0)$ est un point double, et on verra que les deux dérivées partielles s'y annulent en même temps : c'est un point singulier. En dehors de ces deux points, on peut donc exprimer localement la courbe $\mathcal{C}$ comme le graphe d'une fonction d'une variable $y = \varphi(x)$.

\myfigure{1}{
   \tikzinput{fig-diffeo-09}
} 



Il y a deux points $P_1$ et $P_2$ où la tangente est horizontale. En dehors des trois points $P_1$, $P_2$ et $(0,0)$, on pourrait aussi exprimer localement la courbe $\mathcal{C}$ comme le graphe d'une fonction d'une variable $x = \tilde\varphi(y)$.

\bigskip

\textbf{Dérivées partielles.}

On a $F(x,y) = x^3 + xy^2 -x^2+y^2$. Donc
$$\frac{\partial F}{\partial x}(x,y) = 3x^2+y^2-2x 
\qquad \text{ et } \qquad 
\frac{\partial F}{\partial y}(x,y) = 2xy+2y.$$

\bigskip

\textbf{Tangence verticale.}

Pour chercher les points de tangence verticale, on résout d'abord l'équation $\frac{\partial F}{\partial y}(x,y) = 0$ :
$$\frac{\partial F}{\partial y}(x,y) = 0 \iff  2y(x+1)=0 \iff y=0 \text{ ou } x=-1.$$

En plus, on cherche un point $(x,y) \in \mathcal{C}$.

\emph{Cas $y=0$.} Alors, comme on veut en plus $F(x,y)=0$, on a $x^3-x^2=0$, donc $x=0$ ou $x=1$.
Ainsi, il y a deux solutions : $(0,0)$ et $(1,0)$.

\emph{Cas $x=-1$.} On veut en plus $F(x,y)=0$, mais l'équation $F(-1,y)=0$ équivaut à 
$-2+y^2-y^2=0$, qui n'a pas de solution.

Conclusion :
\begin{itemize}
     \item $(1,0) \in \mathcal{C}$ est un point de tangence verticale : la dérivée $\frac{\partial F}{\partial y}$ s'y annule mais pas  $\frac{\partial F}{\partial x}$ (faites le calcul).
     
     \item $(0,0) \in \mathcal{C}$ est un point singulier : les deux dérivées partielles s'y annulent.
\end{itemize}

\bigskip

\textbf{Théorème des fonctions implicites.}

Ainsi, pour n'importe quel point $(x_0,y_0)$ de $\mathcal{C}$ autre que $(0,0)$ et $(1,0)$, il existe $\varphi :I \to J$ (où $I$ est un intervalle ouvert contenant $x_0$ et $J$ un intervalle ouvert contenant $y_0$) telle que 
$$F(x,y) = 0 \iff y = \varphi(x)$$
(pour $x \in I$, $y \in J$).


\bigskip

\textbf{Théorème des fonctions implicites (bis).}

On peut aussi appliquer le théorème des fonctions implicites là où la courbe n'a pas de tangente horizontale (ni de point singulier).

La courbe $\mathcal{C}$ admet deux tangentes horizontales : en $P_1=(x_1,y_1)$ et $P_2=(x_1,-y_1)$ où $x_1 = \frac{\sqrt5-1}{2}$ et $y_1 = \sqrt{\frac{5\sqrt{5}-11}{2}}$.
Pour tout $(x_0,y_0) \in \mathcal{C}$ autre que $P_1$, $P_2$ et $(0,0)$, il existe $\tilde\varphi : J \to I$ de la variable $y$ (où $I$ est encore un intervalle ouvert contenant $x_0$ et $J$ un intervalle ouvert contenant $y_0$) telle que 
$$F(x,y) = 0 \iff x = \tilde\varphi(y).$$


\bigskip

\textbf{\'Equation d'une tangente.}
  
Même si on ne connaît pas l'expression de $\varphi$, le théorème des fonctions implicites permet de calculer des valeurs de la dérivée de $\varphi(x)$ (à condition de connaître la valeur $\varphi(x)$ correspondante).

Considérons le point $Q=(x_2,y_2) \in \mathcal{C}$ pour lequel $x_2 = \frac12$ et $y_2>0$.
En résolvant l'équation $F(\frac12,y)=0$, on trouve la solution positive $y_2 = \frac{\sqrt3}{6}$. Calculons l'équation de la tangente à $\mathcal{C}$ en $Q$.
   
\myfigure{1}{
    \tikzinput{fig-diffeo-10}
}   
   
Appliquons le théorème des fonctions implicites autour de $Q$ : il existe une fonction $y = \varphi(x)$ dont le graphe (autour de $x=\frac12$) est exactement la courbe $\mathcal{C}$ (autour de $Q$).
Ainsi, la tangente $T$ à $\mathcal{C}$ en $Q$ est aussi la tangente au graphe de $\varphi$ en $x_2=\frac12$. 
On sait comment calculer l'équation d'une tangente au graphe de la fonction d'une variable $\varphi$ : il suffit de calculer son coefficient directeur donné par
$\varphi'(x_2)$.

Tout d'abord, par définition de $\varphi$, en $x_2=\frac12$, on a $\varphi(x_2)=y_2$.
La seconde partie du théorème des fonctions implicites nous permet de calculer $\varphi'(x_2)$ :

$$
	\varphi'(x_2) 
	=    -\frac{\frac{\partial F}{\partial x}(x_2, \varphi(x_2))}{\frac{\partial F}{\partial y}(x_2, \varphi(x_2))} 
	= -\frac{\frac{\partial F}{\partial x}(x_2, y_2)}{\frac{\partial F}{\partial y}(x_2, y_2)} 
	= -\frac{\frac{\partial F}{\partial x}( \frac12,\frac{\sqrt3}{6} )}{\frac{\partial F}{\partial y}( \frac12,\frac{\sqrt3}{6}) }   
	= - \frac{-\frac16}{\frac{\sqrt3}{2}} 
	= \frac{\sqrt3}{9} 
$$

Conclusion : l'équation de la tangente en $Q$ est $y=(x-x_2)\varphi'(x_2)+ y_2$, c'est-à-dire
$y= \frac{\sqrt3}{9}(x-\frac12) + \frac{\sqrt3}{6}$,
que l'on peut aussi écrire $y=\frac{\sqrt3}{9}(x+1)$.

%----------------------------------------------------
\subsection{Preuve du théorème des fonctions implicites}

\textbf{Théorème d'inversion locale.}

Soit $F$ une fonction  de deux variables de classe $\mathcal{C}^1$ et soit $(x_0,y_0) \in \Rr^2$ tel que $F(x_0,y_0)=0$ et $\frac{\partial F}{\partial y}(x_0,y_0)\neq 0$. Considérons la fonction $\Phi : \Rr^2 \to \Rr^2$ définie par
$$
\Phi(x,y)=(x,F(x,y)).
$$
La matrice jacobienne de $\Phi$ en $(x_0,y_0)$ est
$$
J_\Phi(x_0,y_0) = 
\begin{pmatrix}  1 & 0 \\
\frac{\partial F}{\partial x}(x_0,y_0) & \frac{\partial F}{\partial y}(x_0,y_0) \\ \end{pmatrix}.
$$
Par hypothèse, $\frac{\partial F}{\partial y}$ ne s'annule pas en $(x_0,y_0)$. La matrice $J_\Phi(x_0,y_0)$ est donc inversible et, d'après le théorème d'inversion locale, $\Phi$ est localement inversible en $(x_0,y_0)$ : il existe un ouvert $U$ contenant $(x_0,y_0)$ et un ouvert $V$ contenant $(x_0,0)$ tels que $\Phi : U \to V$ soit un difféomorphisme. Notons $\Psi=\Phi^{-1} : V \to U$ son inverse et décomposons $\Psi$ en $\Psi(X,Y) = (\psi_1(X,Y),\psi_2(X,Y))$.

\bigskip

\textbf{Expression de $\Psi$.}
Comme $\Psi$ est l'inverse de $\Phi$ alors, d'une part, 
$$(\Phi \circ \Psi) (X,Y) = (X,Y),$$
mais, d'autre part,
$$(\Phi \circ \Psi) (X,Y)
= \Phi\big( \psi_1(X,Y), \psi_2(X,Y) \big) 
= \big(\psi_1(X,Y), F ( \psi_1(X,Y), \psi_2(X,Y)) \big).$$
Ainsi :
$$\psi_1(X,Y) = X \qquad \text{ et } \qquad F \big( X, \psi_2(X,Y) \big) = Y.$$

\bigskip

\textbf{\'Equation $F(x,y)=0$.}
Soit $(x,y) \in U$ un point vérifiant $F(x,y)=0$.
Alors son image par le difféomorphisme $\Phi$ est 
le point $(X,Y) \in V$ tel que $(X,Y)=\Phi(x,y)=(x,F(x,y))=(x,0)$. Donc $X=x$ et $Y=0$.

Ainsi, les points de $U$ vérifiant $F(x,y)=0$ correspondent aux points de la forme $(X,0)$ de $V$. Autrement dit, le difféomorphisme transforme localement la courbe $(F(x,y)=0)$ en un segment horizontal.

\myfigure{1}{
    \tikzinput{fig-diffeo-11}
}

Retenons que $X=x$ et $Y=0$ pour les points qui nous intéressent. 

Définissons alors un intervalle ouvert $I$ contenant $x_0$ et un intervalle ouvert $J$ contenant $y_0$ tels que $I \times J \subset U$.
On peut alors définir $\varphi : I \subset \Rr \to J \subset \Rr$ par $\varphi(x) = \psi_2(x,0)$. 
C'est une fonction de classe  $\mathcal{C}^1$ (car $\Psi$ l'est), définie sur un intervalle ouvert $I$ contenant $x_0$ et à valeurs dans un intervalle ouvert $J$ contenant $y_0$.

Reprenons le calcul avec $(x,y) \in I \times J \subset U$ :
\begin{align*}
F(x,y) = 0
&\iff \Phi(x,y) = (x,0) \\
&\iff (x,y) = \Psi(x,0) \\
&\iff (x,y) = (x, \psi_2(x,0) ) \\
&\iff y = \varphi(x) \\
\end{align*}   

Ainsi, l'équation $F(x,y)=0$ équivaut à $y=\varphi(x)$ (dans un voisinage de $(x_0,y_0)$). En particulier, on a $y_0 = \varphi(x_0)$.

\bigskip

\textbf{Dérivée.}

On rappelle que, par la formule de la matrice jacobienne d'une composition (voir le chapitre \og{}Matrice jacobienne\fg{}), la dérivée de $F\big( u(x), v(x) \big)$ est
$$\frac{d}{dx}F\big( u(x), v(x) \big)
= u'(x) \frac{\partial F}{\partial x} (u(x), v(x)) + v'(x) \frac{\partial F}{\partial y}(u(x), v(x)).$$

Partons de l'équation :
$$F\big( x, \varphi(x) \big) = 0, \quad \text{ pour tout } x \in I.$$
Dérivons cette équation par rapport à la variable $x$ :
$$\frac{\partial F}{\partial x} (x, \varphi(x)) + \varphi'(x) \frac{\partial F}{\partial y} (x, \varphi(x)) = 0.$$

Comme $F$ et $\varphi$ sont de classe $\mathcal{C}^1$, alors $x \mapsto \frac{\partial F}{\partial y}(x, \varphi(x))$ est continue sur $I$. Comme de plus elle ne s'annule pas en $x_0$ puisque $\frac{\partial F}{\partial y}(x_0, y_0) \neq 0$ par hypothèse, on en déduit qu'elle ne s'annule pas sur un voisinage de $x_0$. Ainsi, en réduisant éventuellement $I$, on a $\frac{\partial F}{\partial y}(x, \varphi(x)) \neq 0$ pour tout $x \in I$, et donc 
$$\varphi'(x) = \frac{-\frac{\partial F}{\partial x}(x, \varphi(x))}{\frac{\partial F}{\partial y}(x, \varphi(x))}.$$

%----------------------------------------------------
\subsection{Cas $F(x_1,\ldots,x_n) = 0$}

L'étude est similaire pour les hypersurfaces de niveau en plusieurs variables, où on va pouvoir exprimer une variable en fonction des autres si la dérivée partielle correspondante n'est pas nulle. 

\begin{theoreme}
Si $F : \Rr^n\to \Rr$ est de classe $\mathcal{C}^1$ et si $\displaystyle \frac{\partial F}{\partial x_{n}}  (a_1,\ldots,a_n) \neq 0$ alors : 
\begin{enumerate}
    \item  La fonction implicite $x_n = \varphi (x_1, \ldots, x_{n-1})$ existe sur une boule ouverte $B \subset \Rr^{n-1}$ centrée en $(a_1, \ldots,  a_{n-1})$  et un intervalle ouvert $J$ contenant $a_n$ avec $\varphi : B \to J$ et on a 
$$F(x_1,\ldots,x_{n-1},x_n)=0 \iff x_{n} = \varphi (x_{1},   \ldots,  x_{n-1}).$$
En particulier, $F(x_{1}, \ldots, x_{n-1}, \varphi(x_{1}, \ldots, x_{n-1})) = 0$.

    \item $\displaystyle \frac{\partial \varphi}{\partial x_{i}} (x_1,\ldots,x_{n-1})
     =  \frac{- \frac{\partial f}{\partial x_{i}} (x_{1},  \ldots,  x_{n-1} , \varphi(x_{1},  \ldots,  x_{n-1}))}{\frac{\partial f}{\partial x_{n}} (x_{1},  \ldots,  x_{n-1} , \varphi(x_{1},  \ldots,  x_{n-1}))}$
     pour tout $1 \le i \le n-1$.
\end{enumerate}
\end{theoreme}

Bien sûr, il existe un énoncé similaire, obtenu en remplaçant $n$ par $k$, pour chaque situation où $\displaystyle \frac{\partial F}{\partial x_{k}}  (a_1,\ldots,a_n) \neq 0$.

Prenons comme exemple le cas de $F : \Rr^3 \to \Rr$.
Soit $(x_0,y_0,z_0)$ un point où $\frac{\partial F}{\partial z} (x_0,y_0,z_0)  \neq 0$.
Alors il existe un disque ouvert $D \subset \Rr^2$ centré en $(x_0,y_0)$, un intervalle ouvert $J$ contenant $z_0$ et une unique fonction $\varphi : D \to J$,  $z=\varphi(x,y)$ telle que $F(x,y, \varphi(x,y)) = 0$ (pour tout $(x,y)\in D$ et $z\in J$).
 
%----------------------------------------------------
\begin{miniexercices}
\sauteligne
\begin{enumerate}
    
  \item Soit $F(x,y)=xy^2+y^2+x-6y+1$ et $\mathcal{C} : (F(x,y)=0)$. 
  Calculer $\frac{\partial F}{\partial x}$ et $\frac{\partial F}{\partial y}$.
  Montrer que $P_1 = (-4,-1)$ et $P_2 = (2,1)$ sont des points de $\mathcal{C}$ de tangence verticale. (Bonus : montrer que ce sont les seuls.) 
  Pour $(x_0,y_0)$ un point de $\mathcal{C}$ autre que ces deux-là, énoncer le théorème des fonctions implicites.
  Montrer que $Q = (1,\frac{3-\sqrt5}{2}) \in \mathcal{C}$ et calculer une équation de la tangente à $\mathcal{C}$ en ce point.
  
  \item Soit $F(x,y)=xy^2-x^2+y^2-2x+3$ et $\mathcal{C} : (F(x,y)=0)$. 
  Calculer $\frac{\partial F}{\partial x}$ et $\frac{\partial F}{\partial y}$.
  Trouver les points $P_1$ et $P_2$ de $\mathcal{C}$ de tangence verticale.
  % P : (-3,0) et (-1,0)
  Pour $(x_0,y_0)$ un point de $\mathcal{C}$ autre que ces deux-là, énoncer le théorème des fonctions implicites.
  Trouver les deux points $Q_1$ et $Q_2$ de $\mathcal{C}$ ayant pour abscisse $x=3$. Calculer  
   une équation de la tangente à $\mathcal{C}$ en chacun de ces points.
  
  \item Soit $F(x,y,z) = \sqrt{x^2+y^2+z^2}-\cos(y)$.
  Soit $(x_0,y_0,z_0)=(0,0,1)$. 
  Montrer que $(x_0,y_0,z_0)$ est une solution de $F(x,y,z)=0$.
  Calculer $\frac{\partial F}{\partial z}(x,y,z)$ et sa valeur en $(x_0,y_0,z_0)$.
  Montrer que, pour $(x,y)$ proche de $(x_0,y_0)=(0,0)$, il existe une fonction $\varphi$
  telle que, en posant $z=\varphi(x,y)$, on ait $F(x,y,z)=0$.
  
\end{enumerate}
\end{miniexercices}

\auteurs{
    \\
    Arnaud Bodin.
    D'après des cours de Abdellah Hanani (Lille), 
    Goulwen Fichou et Stéphane Leborgne (Rennes),
    Laurent Pujo-Menjouet (Lyon). 
    Relu par Anne Bauval et Vianney Combet.
}




\finchapitre 
\end{document}


