
\documentclass[12pt, class=report,crop=false]{standalone}
\usepackage[screen]{../exo7book}


\begin{document}

%====================================================================
\chapitre{Intégration des fonctions de plusieurs variables}
%====================================================================

% Commandes à virer
\newcommand{\ou}{\mathscr{O}}
\newcommand{\f}{\mathscr{F}}
\newcommand{\mat}{\mathscr{M}}
\newcommand{\co}{\mathscr{C}}




%%%%%%%%%%%%%%%%%%%%%%%%%%%%%%%%%%%%%%%%%%%%%%%%%%%%%
\section{}

%----------------------------------------------------
\subsection{}


%----------------------------------------------------
\subsection{}


%----------------------------------------------------
\subsection{}


%----------------------------------------------------
\subsection{}
 
 
%----------------------------------------------------
\begin{miniexercices}
\sauteligne
\begin{enumerate}
  \item 
\end{enumerate}
\end{miniexercices}


%%%%%%%%%%%%%%%%%%%%%%%%%%%%%%%%%%%%%%%%%%%%%%%%%%%%%
\section{}

%----------------------------------------------------
\subsection{}


%----------------------------------------------------
\subsection{}


%----------------------------------------------------
\subsection{}


%----------------------------------------------------
\subsection{}
 
 
%----------------------------------------------------
\begin{miniexercices}
\sauteligne
\begin{enumerate}
  \item 
\end{enumerate}
\end{miniexercices}

\subsection{Fichou : Intégration des fonctions de $\mathbb{R}^n$ dans $\mathbb{R}$}


\subsection{Intégration des fonctions d'une variable}
Au lycée on définit l'intégrale d'une fonction (continue) positive comme étant l'aire sous la courbe. Pour cela, il faut savoir ce qu'est l'aire... On rappelle en quelques mots ci-dessous la définition de l'intégrale selon Riemann.


\noindent{Soit $f : [a\,,\,b] \rightarrow \mathbb{R}$ bornée.
\begin{enumerate}
\item[a)] Cas o\`u $f$ est en escalier.\\
Il existe une partition $a = t_{0} < t_{1} < t_{2} < \dots < t_{n} = b$ telle que $f$ est constante sur chaque intervalle \, $]t_{i} \,,\, t_{i+1}[ \;\; (f(x) = C_{i})$.\\
Alors $\displaystyle \int_{a}^b\, f(t)\, dt = \sum_{i = 0}^{n - 1}\, C_{i}\, (t_{i+1} - t_{i})$.
\item[b)] Si $f$ est bornée on l'approche par des fonctions en escalier.
\end{enumerate}

\begin{center}
%\includegraphics[scale=.7]{MethRect.pdf}
\end{center}

 
\begin{definition}
$f$ est {\bf intégrable} sur $[a\,,\,b]$ s'il existe un nombre unique $I$ tel que pour toutes fonctions en escalier $u\,,\,v$ sur $[a\,,\,b]$ vérifiant $u(x) \leqslant f(x) \leqslant v(x)$ on a : 
$$
 \int_{a}^b\, u(x)\, dx \;\; \leqslant \;\; I \;\; \leqslant \;\;  \int_{a}^b\, v(x) dx
 $$
 et si pour tout $\varepsilon > 0$ il existe des fonctions en escalier $u_{\varepsilon}$ et $v_{\varepsilon}$ vérifiant :
 $$
 u_{\varepsilon} (x) \;\; \leqslant \;\; f(x) \;\; \leqslant \;\; v_{\varepsilon} (x)
 $$
 et
 $$
 0 \;\; \leqslant \;\;  \int_{a}^b\, v_{\varepsilon}(x)\, dx - \int_{a}^b\, u_{\varepsilon} (x)\, dx  \;\; < \;\;  \varepsilon
 $$
\end{definition}


\noindent{Notation : $I$ s'appelle l'intégrale de $f$ sur $[a\,,\,b]$ et se note $\displaystyle \int_{a}^b\, f(x)\, dx$\,.

\begin{theoreme}
Si $f$ est continue sur $[a,b]$ alors $f$ est intégrable sur $[a,b]$.
\end{theoreme}
\begin{proof} \rm : Soit $f$ une fonction continue sur $[a,b]$. Soit $\epsilon$ un nombre réel strictement positif. Alors la fonction $f$ est uniformément continue sur $[a,b]$ : il existe $\eta>0$ tel que, si $|x-y|<\eta$ alors $|f(x)-f(y)|<\epsilon$. Prenons $n$ tel que $(b-a)/n<\eta$ et divisons $[a,b]$ en $n$ intervalles de longueur $(b-a)/n<\eta$ dont les extrémités sont les points $x_k=a+k(b-a)/n$, $k$ variant de 0 \`a $n$ ($k$ prenant $n+1$ valeurs détermine $n$ intervalles $[x_k,x_{k+1}]$ $k$ variant de $0$ \`a $n-1$ ; on a $x_0=a$ et $x_n=b$). Définissons deux fonctions en escalier
$$
f^-(x)=\min\{f(t),t\in[x_k,x_{k+1}]\}\ {\rm si}\  x\in [x_k,x_{k+1}]
$$
$$
f^+(x)=\max\{f(t),t\in[x_k,x_{k+1}]\}\ {\rm si}\ x\in [x_k,x_{k+1}]
$$
Par définition on a $f^-\leq f\leq f^+$ et on a
\begin{eqnarray*}
\int_a^bf^+(x)dx-\int_a^bf^-(x)dx&=&\sum_{k=0}^{n-1}(x_{k+1}-x_k)\max\{f(t),t\in[x_k,x_{k+1}]\}\\
& & \ \ \ \ \ \ \ \ -\sum_{k=0}^{n-1}(x_{k+1}-x_k)\min\{f(t),t\in[x_k,x_{k+1}]\}\\
&=&\sum_{k=0}^{n-1}(x_{k+1}-x_k)(\max_{t\in[x_k,x_{k+1}]}f(t)-\min_{t\in[x_k,x_{k+1}]}f(t))\\
&\leq&\sum_{k=0}^{n-1}(x_{k+1}-x_k)\epsilon\\
&=&\epsilon \sum_{k=0}^{n-1}(x_{k+1}-x_k)\\
&=&\epsilon(b-a)
\end{eqnarray*}
\end{proof}

\subsection{Volume de parties bornées de $\R^d$}
On cherche \`a définir le volume d'une partie. Le volume d'un pavé est donné par le produit des longueurs de ses c\^otés.
Si une partie est une réunion disjointe de pavés alors son volume est la somme des volumes des pavés intervenant dans la réunion.
Que le pavé soit ouvert, fermé, ni ouvert ni fermé ne change rien \`a son volume.
On vérifie que ces r\`egles permettent de définir sans ambiguïté et de mani\`ere cohérente le volume de toute réunion disjointe finie de pavés. Il faut montrer que le résultat ne dépend pas de la fa\c con de découper en pavés.

On souhaite évidemment définir le volume de parties plus générales.

Pour définir le volume d'une partie donnée $E$ on peut ensuite procéder de la fa\c con suivante. Supposons qu'il existe deux suites d'ensembles $(E_n^-)_{n\in \Nn}$ et $(E_n^+)_{n\in \Nn}$ telles que, pour tout $n$, $E_n^-$ et $E_n^+$ soit des réunions finies de pavés, et on ait :
$$
E_n^-\subset E_{n+1}^-\subset E\subset E_{n+1}^+\subset E_n^+,
$$
et $\lim vol(E_n^+)-vol(E_n^-)=0$. Alors les deux suites $vol(E_n^-)$ et $vol(E_n^+)$ sont adjacentes. Elles sont donc convergentes et ont m\^eme limite. On définit le volume de $E$ comme étant cette limite commune.

\subsection{Intégration des fonctions de deux variables}


\noindent{Soit $f : [a  \,,\, b ] \times [c  \,,\, d ] \rightarrow \mathbb{R}$. On note $R$ le rectangle $[a  \,,\, b ] \times [c  \,,\, d ]  \subset \mathbb{R}^2$. On va définir l'intégrale d'une fonction sur $R$ en deux temps: d'abord dans le cas o\`u la fonction est en escalier, puis en approchant la fonction considérée (si c'est possible) par des fonctions en escalier. 

\begin{enumerate}
\item[a)] $f$ est en escalier.\\
Il existe une partition de $[a  \,,\, b ] \times [c  \,,\, d ]$ : \\
$a = s_{0} < s_{1} < s_{2} < \, \dots \, < s_{m} = b$\\
$c = t_{0} < t_{1}  < \, \dots \, < t_{n} = d$\\
telle que $f$ est constante \`a l'intérieur de chaque rectangle\, $]s_{i} \,,\, s_{i+1}[ \, \times \, ]t_{j} \,,\, t_{j+1}[$\; (o\`u elle vaut $C_{ij}$).\\
On définit $\displaystyle \iint_{R}\, f(x\,,\,y)\, dx\,dy = \sum_{i\,,\,j}\, C_{ij}\, (s_{i+1} - s_{i})\, (t_{j+1} - t_{j})$.\\
On remarque en particulier que la valeur de l'intégrale ne dépend pas des valeurs de $f$ sur les bords des petits rectangles.

\item[b)] $f$ est bornée sur $R = [a  \,,\, b ] \times [c  \,,\, d ]$.\\
On approche $f$ par des fonctions en escalier.
\end{enumerate}


\begin{definition}
$f$ est {\bf intégrable} sur $R$ s'il existe un nombre unique $I$ tel que pour toutes fonctions en escalier $u(x\,,\,y)$ et $v(x\,,\,y)$, telles que $u(x\,,\,y) \leqslant f(x\,,\,y) \leqslant v(x\,,\,y)$, on a :
$$
\iint_{R}\, u(x\,,\,y)\, dx\, dy \;\; \leqslant \;\;  I \;\; \leqslant \;\; \iint_{R}\, v(x\,,\,y)\, dx\, dy
$$
et si, pour tout $\varepsilon > 0$, il existe des fonctions en escalier $u_{\varepsilon}$ et $v_{\varepsilon}$ telles que :
$$
u_{\varepsilon} (x\,,\,y) \;\; \leqslant \;\; f(x\,,\,y) \;\; \leqslant \;\; v_{\varepsilon} (x\,,\,y)
$$
et
$$
0  \;\; \leqslant \;\;  \int_{R}\, v_{\varepsilon} (x\,,\,y)\, dx\, dy - \int_{R}\, u_{\varepsilon} (x\,,\,y)\, dx\, dy \;\; < \;\; \varepsilon
$$
\end{definition}


\noindent{Notation :  $I$ s'appelle l'intégrale de $f$ sur $R$ et se note $\displaystyle \iint_{R}\, f(x\,,\,y)\, dx\,dy$\,.

\begin{theoreme} 
\begin{enumerate}
\item Si $f$ est continue sur $R$ alors $f$ est intégrable.
\item Si $f$ est positive sur $R$ alors $\displaystyle \iint_{R}\, f(x\,,\,y)\, dx\, dy$ est le volume sous le graphe de $f$ au-dessus de $R$.
\end{enumerate}
\end{theoreme}

\begin{proposition} (Propriétés de l'intégrale double)
\begin{enumerate}
\item $\displaystyle \iint_{R}\, (\alpha f + \beta g)\, (x\,,\,y)\, dx\, dy = \alpha\, \iint_{R}\, f(x\,,\,y)\, dx\, dy + \beta\, \iint_{R}\, g(x\,,\,y)\, dx\, dy$.
\item Si $R = R_{1} \cup R_{2}$ avec $R_{1} \cap R_{2} = \emptyset$ alors :
$$
\iint_{R}\, f(x\,,\,y)\, dx\, dy = \iint_{R_{1}}\, f(x\,,\,y)\, dx\, dy + \iint_{R_{2}}\, f(x\,,\,y)\, dx\, dy
$$
\end{enumerate}
\end{proposition}

\begin{theoreme}
Si $f$ est continue sur $R$ alors $\displaystyle \iint_{R}\, f(x\,,\,y)\, dx\, dy$ existe.
\end{theoreme}



\subsection{Calcul des intégrales doubles}

\begin{theoreme}(Fubini)
Si $f$ est continue sur $R = [a  \,,\, b ] \times [c  \,,\, d ]$ alors :
$$
 \iint_{R}\, f(x\,,\,y)\, dx\, dy \;\; = \;\; \int_{a}^b\, \left( \int_{c}^d\,\, f(x\,,\,y)\, dy \right)\, dx \;\; = \;\; \int_{c}^d\, \left( \int_{a}^b\,\, f(x\,,\,y)\, dx \right)\, dy
 $$
 \end{theoreme}


\noindent{\bf Exemple}
\begin{enumerate}
\item[(1)] $\displaystyle  \iint_{R}\, (x^2 + y^2)\, dx\, dy  \;\;\;\;\;\;\;\;\;  R = [0\,,\,1] \times [0\,,\,1]$
\item[(2)] $\displaystyle  \iint_{R}\, (1 + x + y)\, dx\, dy     \;\;\;\;\;\; R = [0\,,\,1] \times [0\,,\,1]$
\end{enumerate}



\begin{corollaire}
Si la fonction $f$ est le produit de deux fonctions $g$ et $h$ d'une variable, c'est-\`a-dire $f(x,y)=g(x)h(y)$, alors
$$ \iint_{R}\, f(x,y)\, dx\, dy =(\int _a^bg(x)dx)(\int_c^dh(y)dy).$$
\end{corollaire}

\noindent{\bf Exemple:}
 $\displaystyle  \iint_{[0,1]\times [0,1]}\, e^{x+y}\, dx\, dy $



\subsection{Intégration sur les régions bornées de $\mathbb{R}^2$}


\noindent{Soit $f : D \rightarrow \mathbb{R}$ o\`u $D \subset \mathbb{R}^2$ est non rectangulaire.\\
On consid\`ere un rectangle $R$ tel que $D \subset R$ et on définit $\overline{f}$ sur $R$ avec :
$$
\overline{f} (x\,,\,y) = \left \lbrace 
\begin{array}{cl}
f(x\,,\,y) & \textrm{si} \; (x\,,\,y) \in D \\
0 & \textrm{si} \; (x\,,\,y) \notin D
\end{array}
\right.
$$



Avec les notations précédentes on pose, si cela a un sens :
$$
\iint_{D}\, f(x\,,\,y)\, dx\, dy \;\; = \;\; \iint_{R}\, \overline{f}(x\,,\,y)\, dx\, dy
$$
On peut se ramener \`a deux types de domaine $D$ :

\vskip10pt
\noindent{{\it Type 1} : $D = \left \lbrace (x\,,\,y)\;\;\; / \;\;\; \begin{array}{c} a \leqslant x \leqslant b \\ g_{1}(x) \leqslant y \leqslant g_{2}(x) \end{array} \right \rbrace$ o\`u $g_{1}$ et $g_{2}$ sont continues.

\vskip10pt
\noindent{{\it Type 2} : $D = \left \lbrace (x\,,\,y)\;\;\; / \;\;\; \begin{array}{c} c \leqslant y \leqslant d \\ h_{1}(y) \leqslant x \leqslant h_{2}(y) \end{array} \right \rbrace$.


\begin{theoreme}(Fubini)
\begin{enumerate}
\item[a)] Si $f$ est continue sur $D$ de type 1, alors $f$ est intégrable et on a :
$$
\iint_{D}\, f(x\,,\,y)\, dx\, dy \;\; = \;\; \int_{a}^b\;\; \left( \int_{g_{1}(x)}^{g_{2}(x)}\, f(x\,,\,y)\, dy \right)\, dx
$$
\item[b)] Si $f$ est continue sur $D$ de type 2, alors $f$ est intégrable et on a :
$$
\iint_{D}\, f(x\,,\,y)\, dx\, dy \;\; = \;\; \int_{c}^d\;\; \left( \int_{h_{1}(y)}^{h_{2}(y)}\, f(x\,,\,y)\, dx \right)\, dy
$$
\end{enumerate}
\end{theoreme}

\noindent{\bf Exemples}
\begin{enumerate}
\item[(1)] $\displaystyle \iint_{D}\, (x + 2y)\, dx\, dy$ : $D$ est la région entre les deux paraboles $y = 2x^2$ et $y = 1 + x^2$.
\item[(2)] $\displaystyle \iint_{D}\, e^{x^2}\, dx\, dy$\, sur le triangle $D = \left \lbrace (x\,,\,y)\;\;\; / \;\;\; \begin{array}{c} 0 \leqslant x \leqslant 1 \\ 0 \leqslant y \leqslant x \end{array} \right \rbrace$\,. 
\item[(3)] Le choix d'intégrer d'abord par rapport \`a $x$ ou $y$ peut amener des calculs plus ou mois long. Par exemple avec $\displaystyle \iint_{D}\, xy\, dx\, dy$ o\`u $D$ est le trap\`eze délimité par $y=0$, $y=1$ et les droites d'équation $y = 2-x$ et $y = 1 + x/2$.
\end{enumerate}



\begin{definition}
Si $D$ est un domaine borné, on appelle {\bf aire de $D$} : aire$(D) = \displaystyle \iint_{D}\, 1\, dx\, dy$.
\end{definition}




\subsection{Intégrale double et changement de variables}


\noindent{Rappel \`a une variable :
$$
\int_{a}^b\, f(x)\, dx \;\; = \;\; \int_{c}^d\, f(g(t))\, g'(t)\, dt
$$
o\`u $g$ est bijection de $[c\,,\,d]$ sur $[a\,,\,b]$.

\begin{proof} \rm : Soit $F$ une primitive de $f$. 

On a d'une part 
$$
F(g(b))-F(g(a))=\int_{g(a)}^{g(b)}F'(t)dt=\int_{g(a)}^{g(b)}f(t)dt,
$$
d'autre part 
$$
F(g(b))-F(g(a))=\int_a^b(F\circ g)'(s) ds=\int_a^bF'(g(s))g'(s) ds=\int_a^bf(g(s))g'(s) ds.
$$ 
\end{proof}

\begin{theoreme}
Si $G(u\,,\,v) = (x(u\,,\,v) \,,\, y(u\,,\,v))$ :
$$
\iint_{G(S)}\, f(x\,,\,y)\, dx \, dy \;\; = \;\; \iint_{S}\, f(x(u\,,\,v) \,,\,y(u\,,\,v)) \,\, | \det \textrm{Jac}(G(u\,,\,v))|\,\, du\, dv
$$
avec Jac$(G(u\,,\,v)) = \left( \begin{array}{ccc}
\displaystyle \frac{\partial x}{\partial u} &  & \displaystyle \frac{\partial x}{\partial v} \\\\
\displaystyle \frac{\partial y}{\partial u} &  & \displaystyle \frac{\partial y}{\partial v}
\end{array}
\right)$.
\end{theoreme}

\begin{remarque*} Ce n'est pas tr\`es étonnant de trouver l\`a le déterminant. Par exemple, la valeur absolue du déterminant d'une matrice $2\times 2$ calcule l'aire du parallélogramme engendré par les vecteurs colonnes (par exemple).
\end{remarque*}

\noindent{\bf Cas des coordonnées polaires}\\
$x = x(r\,,\,\theta) = r \cos \theta$\\
$y = y(r\,,\,\theta) = r \sin \theta$
$$
J(G) = \left( \begin{array}{ccc}
\cos \theta  &  & -\, r \sin \theta \\
\sin \theta &   & r \cos \theta
\end{array}
\right)
$$
d'o\`u $\displaystyle \iint_{R = G(S)}\, f(x\,,\,y)\, dx\, dy \; = \; \iint_{S}\, f(r \cos \theta \,,\, r \sin \theta)\, r\, dr\, d\theta$.



\noindent{\bf Exemples}
\begin{enumerate}
\item[(1)] Calcul de l'aire d'un disque.
\item[(2)] Calcul de l'aire \`a l'intérieur d'une ellipse.
\item[(3)] $\displaystyle \int_{-\, \infty}^{+\, \infty}\, e^{-\, x^2}\, dx = \sqrt{\pi}$\,.
\item[(4)]  $\displaystyle \iint_{D}\,(x-y)^2 \, dx\, dy$ o\`u $D$ est le morceau du disque unité compris entre l'axe des $x$ et la demi-droite $y=x$.
\end{enumerate}



\subsection{Intégrales triples}
On définit et calcule de mani\`ere totalement similaire les intégrales pour les fonctions de trois variables (et plus...). On va voir quelques exemples...

\noindent{$\displaystyle \iiint_{R}\, f(x\,,\,y\,,\,z)\, dx\, dy\, dz$


\noindent{\bf Exemple:}
 $\displaystyle \iiint_{D}\, (x+y+z)^2 \, dx\, dy\, dz=1/10$ o\`u $D$ est le domaine délimité par les plans d'équations $x=0,y=0,z=0$ et $x+y+z=1$.


\noindent{\bf Changement de variables} en dimension 3. Cas des coordonnées sphériques.

$\displaystyle \iiint_{R = G(S)}\, f(x,y,z)\, dx\, dy \, dz\; = \; \iiint_{S}\, f(r \cos \theta \sin \phi\,,\, r \sin \theta \sin \phi, r \cos \phi) \, r^2 sin \phi\, dr\, d\theta \, d\phi$.

\noindent{\bf Exemple:}
 $\displaystyle \iiint_{D}\, (x^2+y^2+z^2) \, dx\, dy\, dz=4\pi/5$ o\`u $D$  est la boule centrée en l'origine et de rayon 1.
 
\subsection{Quelques calculs classiques}

\subsubsection{L'aire d'un disque}
On peut trouver l'aire d'un disque gr\^ace au calcul intégral. En intégrant par tranche :
\begin{eqnarray*}
Aire&=&4\int_0^R\sqrt{R^2-x^2}dx\\
&=&4\int_0^{\pi/2}R\cos(t)R\cos(t)\ dt\\
&=&4R^2\int_0^{\pi/2}{{1+\cos(2t)}\over{2}} \ dt \\
&=& \pi R^2,
\end{eqnarray*}
calculée gr\^ace au changement de variable $x=R\sin(t)$. 
On peut aussi (et c'est plus simple quand on conna\^ \it l'expression du jacobien du passage en polaires) utiliser les coordonnées polaires
\begin{eqnarray*}
Aire&=&\int_0^R\int_0^{2\pi}r\ dr\ d\theta\\
&=&2\pi. R^2/2\\
&=& \pi R^2.
\end{eqnarray*}


\subsubsection{Le volume de la boule}
On peut calculer le volume de la boule par un changement de variables en coordonnées sphériques.

$\displaystyle \iiint_{B(0,R)}\,  \, dx\, dy\, dz= \iiint_{[0,R]\times [0,2\pi]\times [0,\pi]}\,  r^2 sin \phi\, dr\, d\theta\, d\phi=\int_0^Rr^2dr \int_0^{2\pi}d\theta \int_0^{\pi}\sin \phi d\phi$

et donc le volume vaut $4\pi R^3/3$.

\subsubsection{Le volume d'une pyramide}
Cas d'une pyramide $P$ de base carrée de c\^oté de longueur $a$ et de hauteur $h$. On pose la pyramide sur le plan $z=0$, centrée sur l'axe des $z$. La longueur du c\^oté du carré $C_z$ \`a la hauteur $z$ est donc $a(1-z/h)$. Ainsi par le théor\`eme de Fubini
$$ \iiint_Pdxdydz=\int_0^h( \iint_{C_z}dxdy)dz=\int_0^h( a^2(1-z/h)^2)dz=a^2h/3=Sh/3$$
o\`u $S$ est l'aire de la base.

\subsubsection{Solides de révolution}
Soit $f$ une fonction positive ou nulle définie sur un intervalle $[a,b]$. Considérons la partie de l'espace définie de la façon suivante :
$$
V=\{(x,y,z)\ /\ x\in[a,b], \sqrt{y^2+z^2}\leq f(x)\}.
$$
C'est le solide obtenu en faisant tourner le graphe de $f$ autour de l'axe des $x$.
Le volume de $V$ est donné par l'intégrale triple
$$
\iiint_Vdxdydz.
$$
En intégrant par tranche (d'abord en $y$, $z$, puis en $x$) on obtient :
$$
\iiint_Vdxdydz=\int_a^b\left(\iint_{\{(y,z)\ /\  \sqrt{y^2+z^2}\leq f(x)\}}dydz\right)dx=\int_a^b\pi f(x)^2dx.
$$
Le calcul de l'intégrale triple se ram\`ene donc \`a un calcul d'intégrale simple.

\noindent{\bf Exemples}
\begin{enumerate}
\item[(1)] Cas d'un c\^one: on prend pour $f$ la fonction définie sur $[0,h]$ par $f(x)=ax$ pour $a>0$. Ici la base du c\^one a pour aire $S=\pi(ah)^2$, et donc le volume du c\^one est égal \`a
$$\int_0^h\pi (ax)^2dx=\pi a^2h^3/3=Sh/3.$$
\item[(2)] Cas d'un cylindre de rayon $R$ et de hauteur $h$:
$$\int_0^h\pi R^2dx=\pi R^2h=Sh$$
o\`u $S$ est l'aire de la base.
\end{enumerate}


\auteurs{

}


\finchapitre 
\end{document}


