\documentclass[class=report,crop=false]{standalone}
\usepackage[screen]{../exo7book}

\begin{document}

\newcommand*{\longhookrightarrow}{\ensuremath{\lhook\joinrel%
\relbar\joinrel\relbar\joinrel\relbar\joinrel\relbar%
\joinrel\relbar\joinrel\rightarrow}}
    
%====================================================================
\chapitre{Nombres complexes}
%====================================================================

\insertvideo{utABzdEXLuE}{partie 1. Les nombres complexes, définitions et opérations}

\insertvideo{KmPyB3Twjio}{partie 2. Racines carrées, équation du second degré}

\insertvideo{k9eqlVv535o}{partie 3. Argument et trigonométrie}

\insertvideo{ej9zpQYsQs8}{partie 4. Nombres complexes et géométrie}

\insertfiche{fic00001.pdf}{Nombres complexes}



%%%%%%%%%%%%%%%%%%%%%%%%%%%%%%%%%%%%%%%%%%%%%%%%%%%%%%%%%%%%%%%%
\section*{Préambule}

L'équation $x+5 = 2$ a ses coefficients dans $\Nn$ mais pourtant sa solution
$x=-3$ n'est pas un entier naturel. Il faut ici considérer l'ensemble plus grand $\Zz$
des entiers relatifs.
{\large
$$\Nn \stackrel{x+5=2}{\longhookrightarrow} \Zz \stackrel{2x=-3}\longhookrightarrow \Qq
\stackrel{x^2=\frac12}\longhookrightarrow \Rr \stackrel{x^2=-\sqrt2}\longhookrightarrow \Cc$$
}


De même l'équation $2x=-3$ a ses coefficients dans $\Zz$ mais sa solution
$x = - \frac{3}{2}$ est dans l'ensemble plus grand des rationnels $\Qq$.
Continuons ainsi, l'équation $x^2=\frac12$ à coefficients dans $\Qq$, a ses solutions
$x_1=+1/\sqrt 2$ et $x_2=-1/\sqrt 2$ dans l'ensemble des réels $\Rr$.
Ensuite l'équation $x^2 = -\sqrt{2}$ à ses coefficients dans $\Rr$ et ses solutions
 $x_1 = +\ii\sqrt{\sqrt{2}}$ et $x_2 = -\ii\sqrt{\sqrt{2}}$ dans l'ensemble
des nombres complexes $\Cc$. Ce processus est-il sans fin ?
Non ! Les nombres complexes sont en quelque sorte le bout de la chaîne car nous
avons le théorème de d'Alembert-Gauss suivant  :
\emph{\og Pour n'importe quelle équation polynomiale
$a_nx^n+a_{n-1}x^{n-1}+\cdots + a_2 x^2 + a_1x+a_0 = 0$
où les coefficients $a_i$ sont des complexes (ou bien des réels),
alors les solutions $x_1,\ldots,x_n$ sont dans l'ensemble des nombres complexes \fg.}

\bigskip


Outre la résolution d'équations, les nombres complexes s'appliquent
à la trigonométrie, à la géométrie (comme nous le verrons dans ce chapitre)
 mais aussi à l'électronique, à la mécanique quantique, etc.


%%%%%%%%%%%%%%%%%%%%%%%%%%%%%%%%%%%%%%%%%%%%%%%%%%%%%%%%%%%%%%%%
\section{Les nombres complexes}

%--------------------------------------------------------------
\subsection{Définition}

\begin{definition}
  Un \defi{nombre complexe}\index{nombre complexe} est un couple $(a, b) \in \Rr^2$ que l'on notera $a + \ii b$
\vspace{-0.5em}
\end{definition}
\myfigure{1}{
  \tikzinput{fig_complexes03}
}
Cela revient à identifier $1$ avec le vecteur $(1,0)$ de $\Rr^2$, et $\ii$ avec le vecteur $(0,1)$.
On note $\Cc$ l'ensemble des nombres complexes. Si $b = 0$, alors $z = a$ est situé sur l'axe des abscisses,
que l'on identifie à $\Rr$.
Dans ce cas on dira que $z$ est \defi{réel},
et $\Rr$ apparaît comme un sous-ensemble de $\Cc$, appelé \defi{axe réel}.
Si $b\neq 0$, $z$ est dit \defi{imaginaire}\index{nombre!imaginaire} et si $b \neq 0$ et $a=0$, $z$ est dit \defi{imaginaire pur}.


%--------------------------------------------------------------
\subsection{Opérations}

Si $z = a + \ii b$ et $z' = a' + \ii b'$ sont deux nombres
complexes, alors on définit les opérations suivantes :
\begin{itemize}
  \item \evidence{addition} : $(a + \ii b) + (a' + \ii b') =
(a + a') + \ii  (b + b')$
\myfigure{1}{
\tikzinput{fig_complexes05}
}
  \item \evidence{multiplication} : $(a + \ii b) \times (a' + \ii b')
  = (aa' - bb') + \ii  (ab' + ba')$. On développe en suivant les règles de la multiplication usuelle 
  avec la convention suivante : \mybox{$\ii ^2 = - 1$}
\end{itemize}


%--------------------------------------------------------------
\subsection{Partie réelle et imaginaire}

Soit $z = a + \ii b$ un nombre complexe, sa \defi{partie réelle}\index{partie reelle@partie réelle} est le réel $a$ et on
la note $\Re(z)$ ;
sa \defi{partie imaginaire}\index{partie imaginaire} est le réel $b$ et on la note $\Im(z)$.
\myfigure{1}{
\tikzinput{fig_complexes04}
}
Par identification de $\Cc$ \`a $\Rr^2$,
l'écriture $z = \Re(z) + \ii  \Im(z)$
est unique :
\[ z = z' \quad \Longleftrightarrow \quad \left\{ \begin{array}{c}
     \Re (z) = \Re (z')\\
     \text{et}\\
     \Im (z) = \Im (z')
   \end{array} \right. \]

En particulier un nombre complexe est réel si et seulement si sa partie imaginaire
est nulle. Un nombre complexe est nul si et et seulement si sa partie réelle et sa partie imaginaire sont nuls.

%--------------------------------------------------------------
\subsection{Calculs}

Quelques définitions et calculs sur les nombres complexes.
\myfigure{1}{\tikzinput{fig_complexes06} }

\begin{itemize}
  \item L'\defi{\,opposé} de $z = a + \ii b$ est $- z = (- a) + \ii (-b) = - a - \ii b$.

  \item La \defi{multiplication par un scalaire} $\lambda \in \Rr$ : $\lambda \cdot
  z = (\lambda a) + \ii(\lambda b)$.

  \item L'\defi{\,inverse}: si $z \neq 0$, il existe un unique $z' \in \Cc$ tel
  que $zz' = 1$ (o\`u $1 = 1 + \ii  \times 0$).

Pour la preuve et le calcul on écrit $z=a+\ii b$ puis on cherche $z'=a'+\ii b'$ tel que $zz'=1$.
Autrement dit $(a+ \ii b)(a'+\ii b') =1$. En développant et identifiant les parties
réelles et imaginaires on obtient les \'equations
  \[ \left\{ \begin{array}{ll}
       aa' - bb' = 1 & (L_1)\\
       ab' + ba' = 0 & (L_2)
     \end{array} \right. \]
En écrivant $a L_1 + b L_2$ (on multiplie la ligne ($L_1$) par $a$, la ligne ($L_2$) par $b$ et on additionne)
 et $- bL_1 + aL_2$ on en d\'eduit
  \[ \left\{ \begin{array}{l}
       a'  \left( a^2 + b^2 \right) = a\\
       b'  \left( a^2 + b^2 \right) = - b
     \end{array} \right. \quad \text{ donc }\quad \left\{ \begin{array}{l}
       a' = \frac{a}{a^2 + b^2}\\
       b' = - \frac{b}{a^2 + b^2}
     \end{array} \right. \]
  L'inverse de $z$, noté $\frac1z$, est donc
  \[ z' = \frac{1}{z} = \frac{a}{a^2 + b^2} + \ii  \frac{- b}{a^2 + b^2} =
     \frac{a - \ii b}{a^2 + b^2} . \]

  \item La \defi{division} : $\frac{z}{z'}$ est le nombre complexe $z \times \frac{1}{z'}$.

  \item Propriété d'intégrité : si $zz' = 0$ alors $z = 0$ ou $z' = 0$.

  \item Puissances : $z^2 = z \times z$, $z^n = z \times \cdots \times z$ ($n$
  fois, $n\in \Nn$). Par convention $z^0 = 1$ et $z^{- n} = \left( \frac{1}{z} \right)^n = \frac{1}{z^n}$.
\end{itemize}


\begin{proposition}
\label{prop:somme}
Pour tout $z \in \Cc$ différent de $1$
\mybox{$1 + z + z^2 + \cdots + z^n = \dfrac{1 - z^{n + 1}}{1 - z} .$}
\vspace{-0.5em}
\end{proposition}

La preuve est simple : notons $S=1 + z + z^2 + \cdots + z^n$, alors en développant
$S \cdot (1 - z)$ presque tous les termes se télescopent et l'on trouve $S \cdot (1 - z) = 1 - z^{n + 1}$.


\begin{remarque*}
  Il n'y pas d'ordre naturel sur $\Cc$, il ne faut donc
  jamais écrire $z \geqslant 0$ ou $z \leqslant z'$.
\end{remarque*}


%------------------------------------------------------------------
\subsection{Conjugué, module}

  Le \defi{conjugué}\index{conjugue@conjugué} de $z = a + \ii b$ est $\bar{z} = a - \ii b$, autrement dit
  $\Re(\bar{z}) = \Re(z)$ et
  $\Im(\bar{z}) = - \Im (z)$.
  Le point $\bar{z}$ est le symétrique du point $z$ par rapport à l'axe réel.

  Le \defi{module}\index{module} de $z = a + \ii b$ est le réel positif $|z| = \sqrt{a^2 + b^2}$.
Comme $z \times \bar z = (a+\ii b)(a-\ii b) = a^2+b^2$ alors  le module vaut aussi $|z| = \sqrt{z\bar z}$.
\myfigure{1}{
\tikzinput{fig_complexes07} \qquad
\tikzinput{fig_complexes08}
}

Quelques formules :
\begin{itemize}
  \item $\overline{z + z'} = \bar{z} + \overline{z'}$,\quad $\overline{\bar{z}} =
  z$,\quad $\overline{zz'} = \bar{z}  \overline{z'}$

  \item $z = \bar{z} \Longleftrightarrow z \in \Rr$

  \item $\left| z \right|^2 = z \times \bar{z}$,\quad $\left| \bar{z} \right| =
  \left| z \right|$,\quad $\left| zz' \right| = |z| |z'|$

  \item $\left| z \right| = 0 \Longleftrightarrow z = 0$
\end{itemize}

\begin{proposition}[L'inégalité triangulaire]\index{inegalite triangulaire@inégalité triangulaire}\ 
\mybox{$\left| z + z' \right| \leqslant \left| z  \right| + \left| z' \right|$}
\end{proposition}
\myfigure{1}{
\tikzinput{fig_complexes10bis}
}
Avant de faire la preuve voici deux remarques utiles. Soit $z = a +\ii b \in \Cc$ avec $a,b\in\Rr$ :
\begin{itemize}
  \item $|\Re(z)| \le |z|$ (et aussi $|\Im(z)| \le |z|$). Cela vient du fait que
  $|a| \le \sqrt{a^2+b^2}$. Noter que pour un réel $|a|$ est à la fois le module et la valeur absolue. 
  \item \myboxinline{$z+\bar{z} = 2\Re(z)$} et $z-\bar{z} = 2\ii\Im(z)$. Preuve :
  $z+\bar{z} = (a+\ii b)+ (a-\ii b) = 2a = 2\Re(z)$.
\end{itemize}

\begin{proof}
Pour la preuve on calcule $|z+z'|^2$ :
\begin{eqnarray*}
|z+z'|^2 
 &=& \left( z + z'\right)  \overline{\left( z + z' \right)} \\
 &=& z \bar{z} + z'  \overline{z'} + z \overline{z'} + z'  \bar{z}  \\
 &=& |z|^2 + |z'|^2 + 2\Re(z'  \bar{z}) \\
 &\le& |z|^2 + |z'|^2 + 2|z'\bar{z}| \\
 &\le& |z|^2 + |z'|^2 + 2|zz'| \\ 
 &\le& (|z| + |z'|)^2 \\  
\end{eqnarray*}
\end{proof}


\begin{exemple}
  Dans un parallélogramme, la somme des carrés des diagonales égale la
  somme des carrés des c\^otés.
\end{exemple}


Si les longueurs des côtés sont notées $L$ et $\ell$ et les longueurs des diagonales sont $D$ et $d$
alors il s'agit de montrer l'\'egalit\'e
$$D^2+d^2 = 2\ell^2+2L^2.$$
\myfigure{1}{
\tikzinput{fig_complexes10} \qquad
\tikzinput{fig_complexes09}
}

\begin{proof}
Cela devient simple si l'on considère que notre parallélogramme a pour sommets
$0$, $z$, $z'$ et le dernier sommet est donc $z+z'$.
La longueur du grand côté est ici $|z|$, celle du petit côté est $|z'|$.
La longueur de la grande diagonale est $|z+z'|$. Enfin il faut se convaincre que
la longueur de la petite diagonale est $|z-z'|$.



  \begin{eqnarray*}
    D^2 + d^2 = \left| z + z' \right|^2 + \left| z - z' \right|^2 & = & \left( z + z'
    \right)  \overline{\left( z + z' \right)} + \left( z - z' \right)
    \overline{\left( z - z' \right)}\\
    & = & z \bar{z} + z \overline{z'} + z'  \bar{z} + z'  \overline{z'} + z
    \bar{z} - z \overline{z'} - z'  \bar{z} + z'  \overline{z'}\\
    & = & 2 z \bar{z} + 2 z'  \overline{z'} = 2 \left| z \right|^2 + 2 \left|    z' \right|^2 \\
    & = & 2\ell^2+2L^2 \\
  \end{eqnarray*}
\end{proof}


%------------------------------------------------------------------
%\subsection{Mini-exercices}

\begin{miniexercices}
\sauteligne
\begin{enumerate}
  \item Calculer $1 - 2\ii + \frac{\ii}{1 - 2\ii}$.
  \item \'Ecrire sous la forme $a+\ii b$ les nombres complexes $(1+\ii)^2$,
$(1+\ii)^3$, $(1+\ii)^4$, $(1+\ii)^8$.
  \item En déduire $1+(1+\ii)+(1+\ii)^2+\cdots +(1+\ii)^7$.
  \item Soit $z\in \Cc$ tel que $|1+ \ii z| = |1-\ii z|$, montrer que $z \in \Rr$.
  \item Montrer que si $|\Re z| \le |\Re z'|$ et $|\Im z| \le |\Im z'|$ alors
$|z| \le |z'|$, mais que la réciproque est fausse.
  \item Montrer que $1 / \bar{z} = z/\left| z \right|^2$ (pour $z\neq 0$).
\end{enumerate}
\end{miniexercices}

%%%%%%%%%%%%%%%%%%%%%%%%%%%%%%%%%%%%%%%%%%%%%%%%%%%%%%%%%%%%%%%%
\section{Racines carrées, équation du second degré}

%---------------------------------------------------------------
\subsection{Racines carrées d'un nombre complexe}

Pour $z \in \Cc$, une \defi{racine carrée} est un nombre complexe $\omega$
tel que $\omega^2 = z$.

Par exemple si $x \in \Rr_+$, on connaît
deux racines carrées : $\sqrt{x}, - \sqrt{x}$. Autre exemple : les racines carrées de
$-1$ sont $\ii $ et $-\ii$.

\begin{proposition}
Soit $z$ un nombre complexe, alors $z$ admet deux racines carrées, $\omega$ et $-\omega$.
\end{proposition}

Attention ! Contrairement au cas réel, il n'y a pas de façon privilégiée de choisir une racine plutôt
que l'autre, donc pas de fonction racine. On ne dira donc jamais « soit $\omega$ la racine de $z$ ». 

Si $z\neq 0$ ces deux racines carrées sont distinctes.
Si $z=0$ alors $\omega=0$ est une racine double.

Pour $z=a + \ii b$ nous allons calculer $\omega$ et $-\omega$ en fonction de $a$ et $b$.


\begin{proof}
Nous écrivons $\omega= x + \ii y$, nous cherchons $x,y$ tels que $\omega^2=z$.

\begin{eqnarray*}
  \omega^2 = z & \Longleftrightarrow & \left( x + \ii y \right)^2 = a + \ii b\\
  & \Longleftrightarrow & \left\{ \begin{array}{lcc}
    x^2 - y^2 = a&\quad&\text{en identifiant parties}\\
    2 xy = b     &\quad&\text{et parties imaginaires.}
  \end{array} \right.
\end{eqnarray*}

Petite astuce ici : nous rajoutons l'équation $\left| \omega \right|^2 = \left| z \right|$ (qui se
déduit bien sûr de $\omega^2=z$) qui s'écrit aussi $x^2+y^2 = \sqrt{a^2+b^2}$.
Nous obtenons des systèmes équivalents aux précédents :
{\small
\[ \left\{ \begin{array}{l}
     x^2 - y^2 = a\\
     2 xy = b\\
     x^2 + y^2 = \sqrt{a^2 + b^2}
   \end{array} \right. \Longleftrightarrow \left\{ \begin{array}{l}
     2 x^2 = \sqrt{a^2 + b^2} + a\\
     2 y^2 = \sqrt{a^2 + b^2} - a\\
     2 xy = b
   \end{array} \right. \Longleftrightarrow \left\{ \begin{array}{l}
     x = \pm \frac{1}{\sqrt{2}}  \sqrt{\sqrt{a^2 + b^2} + a}\\
     y = \pm \frac{1}{\sqrt{2}}  \sqrt{\sqrt{a^2 + b^2} - a}\\
     2 xy = b
   \end{array} \right. \]
   }
Discutons suivant le signe du r\'eel $b$. Si $b \geqslant 0$, $x$ et $y$ sont de
m\^eme signe ou nuls (car $2xy=b \ge 0$) donc
\[ \omega = \pm \frac{1}{\sqrt{2}}  \left( \sqrt{\sqrt{a^2 + b^2} + a} + \ii
   \sqrt{\sqrt{a^2 + b^2} - a} \right), \]
et si $b \leqslant 0$
\[ \omega = \pm \frac{1}{\sqrt{2}}  \left( \sqrt{\sqrt{a^2 + b^2} + a} - \ii
   \sqrt{\sqrt{a^2 + b^2} - a} \right). \]
En particulier si $b = 0$ le résultat dépend du signe de $a$, si $a
\geqslant 0$, $\sqrt{a^2} = a$ et par conséquent $\omega = \pm \sqrt{a}$,
tandis que si $a < 0$, $\sqrt{a^2} = - a$ et donc $\omega = \pm \ii  \sqrt{- a} =
\pm \ii  \sqrt{\left| a \right|}$.
\end{proof}

Il n'est pas nécessaire d'apprendre ces formules mais il est indispensable
de savoir refaire les calculs.
\begin{exemple}
Les racines carrées de $\ii $ sont $+\frac{\sqrt{2}}{2}(1 + \ii)$ et $-\frac{\sqrt{2}}{2}(1 + \ii)$.

En effet :
\begin{eqnarray*}
  \omega^2 = \ii & \Longleftrightarrow & \left( x + \ii y \right)^2 = \ii \\
  & \Longleftrightarrow & \left\{ \begin{array}{l}
    x^2 - y^2 = 0\\
    2 xy = 1
  \end{array} \right.
\end{eqnarray*}
Rajoutons la conditions $|\omega|^2 = | \ii |$ pour obtenir le système équivalent au précédent :
\[ \left\{ \begin{array}{l}
     x^2 - y^2 = 0\\
     2 xy = 1\\
     x^2 + y^2 = 1
   \end{array} \right. \Longleftrightarrow \left\{ \begin{array}{l}
     2 x^2 = 1\\
     2 y^2 = 1\\
     2 xy = 1
   \end{array} \right. \Longleftrightarrow \left\{ \begin{array}{l}
     x = \pm \frac{1}{\sqrt{2}} \\
     y = \pm \frac{1}{\sqrt{2}} \\
     2 xy = 1
   \end{array} \right. \]

Les réels $x$ et $y$ sont donc de même signe, nous trouvons bien deux solutions :
$$x+\ii y = \frac{1}{\sqrt{2}} + \ii\frac{1}{\sqrt{2}}
\quad \text{ ou } \quad x+\ii y = -\frac{1}{\sqrt{2}} - \ii\frac{1}{\sqrt{2}}$$
\end{exemple}

%---------------------------------------------------------------
\subsection{\'Equation du second degré}

\begin{proposition}
  L'équation du second degré $az^2 + bz + c = 0$, o\`u $a, b, c \in
  \Cc$ et $a \neq 0$, possède deux solutions $z_1, z_2 \in \Cc$
  éventuellement confondues.

  Soit $\Delta = b^2 - 4 ac$ le discriminant et
  $\delta \in \Cc$ une racine carrée de $\Delta$. Alors les
  solutions sont
 \mybox{
  $ z_1 = \dfrac{- b + \delta}{2 a} \quad \text{ et } \quad z_2 = \dfrac{- b - \delta}{2
     a} . $
}
\vspace{-0.5em}
\end{proposition}

Et si $\Delta = 0$ alors la solution $z = z_1 = z_2 = - b / 2 a$ est unique (elle est dite double).
Si on s'autorisait à écrire $\delta = \sqrt{\Delta}$, on obtiendrait
la même formule que celle que vous connaissez lorsque $a,b,c$ sont réels.

\begin{exemple}
\sauteligne
  \begin{itemize}
    \item $z^2 + z + 1 = 0$, $\Delta = - 3$, $\delta = \ii  \sqrt{3}$,
    les solutions sont $z = \dfrac{- 1 \pm \ii  \sqrt{3}}{2}$.

    \item $z^2 + z + \frac{1 - \ii }{4}=0$, $\Delta = \ii $, $\delta = \frac{\sqrt2}{2}(1 +
    \ii)$, les solutions sont $z = \dfrac{- 1 \pm \frac{\sqrt2}{2}(1 +
    \ii)}{2} = -\frac12 \pm \frac{\sqrt2}{4}(1 +
    \ii)$.
  \end{itemize}
\end{exemple}

On retrouve aussi le résultat bien connu pour le cas des équations à coefficients réels:
\begin{corollaire}
  Si les coefficients $a, b, c$ sont réels alors $\Delta \in \Rr$ et les solutions sont de
  trois types :
  \begin{itemize}
    \item si $\Delta = 0$, la racine double est réelle et vaut $-\dfrac{b}{2 a}$,

    \item si $\Delta > 0$, on a deux solutions réelles $\dfrac{- b \pm\sqrt{\Delta}}{2 a}$,

    \item si $\Delta < 0$, on a deux solutions complexes, mais non réelles, $\dfrac{- b \pm \ii  \sqrt{- \Delta}}{2 a}$.
  \end{itemize}
\end{corollaire}

\begin{proof}
  On écrit la factorisation
  \begin{eqnarray*}
    az^2 + bz + c & = & a \left( z^2 + \frac{b}{a} z + \frac{c}{a} \right) = a
    \left( \left( z + \frac{b}{2 a} \right)^2 - \frac{b^2}{4 a^2} +
    \frac{c}{a} \right) \\
    & = & a \left( \left( z + \frac{b}{2 a} \right)^2 -
    \frac{\Delta}{4 a^2} \right)
    = a \left( \left( z + \frac{b}{2 a} \right)^2 - \frac{\delta^2}{4
    a^2} \right) \\
    & = & a \left( \left( z + \frac{b}{2 a} \right) - \frac{\delta}{2
    a^{}} \right)  \left( \left( z + \frac{b}{2 a} \right) + \frac{\delta}{2
    a^{}} \right)\\
    & = & a \left( z - \frac{- b + \delta}{2 a} \right)  \left( z - \frac{- b
    - \delta}{2 a} \right) = a \left( z - z_1 \right)  \left( z - z_2 \right)
  \end{eqnarray*}
Donc le binôme s'annule si et seulement si $z=z_1$ ou $z=z_2$.
\end{proof}



%---------------------------------------------------------------
\subsection{Théorème fondamental de l'algèbre}

\begin{theoreme}[d'Alembert--Gauss]
\index{theoreme@théorème!de d'Alembert--Gauss}
Soit $P (z) = a_n z^n + a_{n - 1} z^{n - 1}
  + \cdots + a_1 z + a_0$ un polyn\^ome \`a coefficients complexes
  et de degré $n$. Alors l'équation $P(z) = 0$ admet exactement $n$
  solutions complexes comptées avec leur multiplicité.

  En d'autres termes il existe des nombres complexes $z_1, \ldots, z_n$ (dont certains sont
  éventuellement confondus) tels que
  \[ P (z) = a_n  \left( z - z_1 \right) \left( z - z_2 \right)
     \cdots \left( z - z_n \right) . \]
\end{theoreme}

Nous admettons ce théorème.


%------------------------------------------------------------------
%\subsection{Mini-exercices}

\begin{miniexercices}
\sauteligne
\begin{enumerate}
  \item Calculer les racines carrées de $-\ii$, $3-4\ii$.
  \item Résoudre les équations : $z^2+z-1=0$, $2z^2 + (-10-10\ii)z+24-10\ii = 0$.
  \item Résoudre l'équation $z^2+(\ii-\sqrt 2)z-\ii\sqrt 2=0$, puis l'équation $Z^4+(\ii-\sqrt 2)Z^2-\ii\sqrt 2=0$.
  \item Montrer que si $P(z)=z^2+bz+c$ possède pour racines $z_1, z_2 \in \Cc$ alors
$z_1+z_2=-b$ et $z_1 \cdot z_2 = c$.
  \item Trouver les paires de nombres dont la somme vaut $\ii$ et le produit $1$.
  \item Soit $P (z) = a_n z^n + a_{n - 1} z^{n - 1} + \cdots  + a_0$ avec $a_i \in \Rr$ pour tout $i$.
Montrer que si $z$ est racine de $P$ alors $\bar z$ aussi.
\end{enumerate}
\end{miniexercices}


%%%%%%%%%%%%%%%%%%%%%%%%%%%%%%%%%%%%%%%%%%%%%%%%%%%%%%%%%%%%%%%%
\section{Argument et trigonométrie}

%------------------------------------------------------------------
\subsection{Argument}

Si $z=x+iy$ est de module $1$, alors $x^2+y^2=|z|^2=1$. Par conséquent le point $(x,y)$ est sur le cercle unité du plan,
et son abscisse $x$ est notée $\cos \theta$, son ordonnée $y$ est $\sin \theta$, où $\theta$ est
(une mesure de) l'angle entre l'axe réel et $z$. Plus généralement, si $z\neq 0$, $z/|z|$ est de module $1$, et cela amène à :

\begin{definition}
  Pour tout $z \in \Cc^{\ast} =\Cc \setminus \left\{ 0 \right\}$,
  un nombre $\theta \in \Rr$ tel que $z = \left| z
  \right|  \left( \cos \theta + \ii  \sin \theta \right)$ est appelé un \defi{argument}\index{argument}
  de $z$ et noté $\theta = \arg (z)$.
\end{definition}

\myfigure{1}{
\tikzinput{fig_complexes11}
}


Cet argument est défini modulo $2\pi$. On peut imposer à cet argument d'être unique si on
rajoute la condition $\theta \in]-\pi,+\pi]$.

\begin{remarque*}
  \[ \theta \equiv \theta' \pmod {2\pi}\quad  \Longleftrightarrow \quad  \exists k \in
     \Zz, \, \theta = \theta' + 2 k \pi \quad  \Longleftrightarrow \quad
     \left\{ \begin{array}{l}
       \cos \theta = \cos \theta'\\
       \sin \theta = \sin \theta'
     \end{array} \right. \]
\end{remarque*}


\begin{proposition}
  L'argument satisfait les propriétés suivantes :
  \begin{itemize}
    \item $\arg \left( zz' \right) \equiv \arg (z) + \arg \left(
    z' \right) \pmod {2\pi}$

    \item $\arg \left( z^n \right) \equiv n \arg (z) \pmod {2\pi}$

    \item $\arg \left( 1 / z \right) \equiv - \arg (z) \pmod {2\pi}$

    \item $\arg (\bar{z}) \equiv - \arg z \pmod{2 \pi}$
  \end{itemize}
\end{proposition}


\begin{proof}
  \begin{eqnarray*}
    zz' & = & \left| z \right|  \left( \cos \theta + \ii  \sin \theta \right)
    \left| z' \right|  \left( \cos \theta' + \ii  \sin \theta' \right)\\
    & = & \left| zz' \right|  \left( \cos \theta \cos \theta' - \sin \theta
    \sin \theta' + \ii  \left( \cos \theta \sin \theta' + \sin \theta \cos
    \theta' \right) \right)\\
    & = & \left| zz' \right|  \left( \cos \left( \theta + \theta' \right) + \ii
    \sin \left( \theta + \theta' \right) \right)
  \end{eqnarray*}
  donc $\arg \left( zz' \right) \equiv \arg (z) + \arg \left( z'
  \right) \pmod {2\pi}$. On en déduit les deux autres propriétés,
  dont la deuxième par récurrence.
\end{proof}

%------------------------------------------------------------------
\subsection{Formule de Moivre, notation exponentielle}


La \defi{formule de Moivre}\index{formule!de Moivre} est :
\mybox{$
  \left( \cos \theta + \ii \sin \theta \right)^n = \cos \left( n \theta \right)
  + \ii  \sin \left( n \theta \right)$}

\begin{proof}
  Par récurrence, on montre que
  \begin{eqnarray*}
    \left( \cos \theta + \ii  \sin \theta \right)^n & = & ( \cos \theta + \ii \sin \theta)^{n-1}
 \times \left( \cos \theta + \ii  \sin \theta \right)
    \\
  & = & \left( \cos \left(
    \left( n - 1 \right) \theta \right) + \ii  \sin \left( \left( n - 1 \right)
    \theta \right) \right) \times \left( \cos \theta + \ii  \sin \theta \right)
    \\
    & = & \left( \cos \left( \left( n - 1 \right) \theta \right) \cos \theta
    - \sin \left( \left( n - 1 \right) \theta \right) \sin \theta \right)
    \\
    && + \ii
    \left( \cos \left( \left( n - 1 \right) \theta \right) \sin \theta + \sin
    \left( \left( n - 1 \right) \theta \right) \cos \theta \right)
    \\
    & = & \cos n \theta + \ii  \sin n \theta
  \end{eqnarray*}
\end{proof}


Nous définissons la \defi{notation exponentielle}\index{exponentielle complexe} par
\mybox{$e^{\ii  \theta} = \cos \theta + \ii  \sin \theta$}
 et donc tout nombre complexe s'écrit
\mybox{$z = \rho e^{\ii  \theta}$}
o\`u $\rho = \left| z \right|$ est le module et $\theta = \arg (z)$ est un argument.

\bigskip

Avec la notation exponentielle, on peut écrire pour $z = \rho e^{\ii  \theta}$
et $z' = \rho' e^{\ii  \theta'}$
\[ \left\{ \begin{array}{l}
     zz' = \rho \rho' e^{\ii  \theta} e^{\ii  \theta'} = \rho \rho' e^{\ii  (\theta + \theta')}\\
     z^n = \left( \rho e^{\ii  \theta} \right)^n = \rho^n  \left( e^{\ii  \theta}
     \right)^n = \rho^n e^{\ii n \theta}\\
     1 / z = 1 / \left( \rho e^{\ii  \theta} \right) = \frac{1}{\rho} e^{- \ii
     \theta} \\
     \bar{z} = \rho e^{-\ii \theta}
   \end{array} \right. \]

La formule de Moivre se r\'eduit \`a l'égalité : \myboxinline{$\left(e^{\ii\theta}\right)^n = e^{\ii n \theta}$}.

Et nous avons aussi : $\rho e^{\ii \theta} = \rho' e^{\ii \theta'}$ (avec $\rho, \rho' > 0$)
si et seulement si $\rho = \rho'$ et $\theta \equiv \theta' \pmod{2\pi}$.


%------------------------------------------------------------------
\subsection{Racines $n$-ième}

\begin{definition}
  Pour $z \in \Cc$ et $n \in \Nn$, une \defi{racine $n$-ième} est
  un nombre $\omega \in \Cc$ tel que $\omega^n = z$.
\end{definition}

\begin{proposition}
Il y a $n$ racines $n$-ièmes $\omega_0, \omega_1, \ldots, \omega_{n - 1}$ de $z=\rho e^{\ii  \theta}$, ce sont :
\mybox{$\omega_k = \rho^{1 / n} e^{\frac{\ii\theta + 2 \ii k \pi}{n}} \ , \quad k = 0,1,\ldots, n - 1$}
     \vspace{-0.5em}
\end{proposition}

\begin{proof}
\'Ecrivons $z= \rho e^{\ii \theta}$ et cherchons $\omega$ sous la forme
$\omega=re^{\ii t}$ tel que $z=\omega^n$.
Nous obtenons donc $ \rho e^{\ii \theta} =\omega ^n = \left(re^{\ii t}\right)^n
= r^n e^{\ii nt}$.
Prenons tout d'abord le module : $ \rho = \left|\rho e^{\ii \theta} \right|
=  \left|r^n e^{\ii nt}\right| = r^n$ et donc $r = \rho^{1/n}$ (il s'agit ici de nombres réels).
Pour les arguments nous avons $e^{\ii nt} = e^{\ii \theta}$ et donc
$nt \equiv \theta \pmod {2\pi}$ (n'oubliez surtout pas le modulo $2\pi$ !).
Ainsi on résout $nt = \theta + 2k\pi$ (pour $k\in\Zz$) et donc $t = \frac{\theta}{n} + \frac{2k\pi}{n}$.
Les solutions de l'équation $\omega^n=z$ sont donc les $\omega_k = \rho^{1/n} e^{\frac{\ii\theta + 2\ii k\pi}{n}}$.
Mais en fait il n'y a que $n$ solutions distinctes car $\omega_n=\omega_0$, $\omega_{n+1}=\omega_1$, \ldots
Ainsi les $n$ solutions sont $\omega_0,\omega_1,\ldots,\omega_{n-1}$.

\end{proof}


Par exemple pour $z = 1$, on obtient les $n$ \defi{racines $n$-ièmes de l'unité}\index{racine de l unite@racine de l'unité}
$e^{2 \ii k \pi / n}$, $k=0,\ldots,n-1$ qui forment un groupe multiplicatif.

\begin{minipage}{0.5\textwidth}
\myfigure{0.65}{
\tikzinput{fig_complexes14}
}
\vspace*{-2ex}
\centerline{Racine $3$-ième de l'unité ($z=1$, $n=3$)}
\end{minipage}
\begin{minipage}{0.5\textwidth}
\myfigure{0.65}{
\tikzinput{fig_complexes15}
}
\vspace*{-2ex}
\centerline{Racine $3$-ième de $-1$ ($z=-1$, $n=3$)}
\end{minipage}

\bigskip
Les racines $5$-ième de l'unité ($z=1$, $n=5$)
forment un pentagone régulier :
\myfigure{0.8}{
\tikzinput{fig_complexes12}
}


%------------------------------------------------------------------
\subsection{Applications à la trigonométrie}

Voici les \defi{formules d'Euler}\index{formule!d'Euler}, pour $\theta \in \Rr$ :
\mybox{$ \cos \theta = \dfrac{e^{\ii  \theta} + e^{- \ii  \theta}}{2} \quad , \quad
   \sin \theta = \dfrac{e^{\ii  \theta} - e^{- \ii  \theta}}{2 \ii }$}
Ces formules s'obtiennent facilement en utilisant la définition de la notation exponentielle.
Nous les appliquons dans la suite à deux problèmes : le développement et la linéarisation.

\bigskip




\defi{Développement.}\index{developpement@développement} On exprime $\sin n \theta$ ou $\cos n \theta$ en
fonction des puissances de $\cos \theta$ et $\sin \theta$.

\medskip

\emph{Méthode :} on utilise la formule de Moivre pour écrire $\cos \left( n
\theta \right) + \ii  \sin \left( n \theta \right) = \left( \cos \theta + \ii  \sin
\theta \right)^n$ que l'on développe avec la formule du bin\^ome de Newton.

\begin{exemple}
\begin{eqnarray*}
  \cos 3 \theta + \ii  \sin 3 \theta & = & \left( \cos \theta + \ii  \sin \theta
  \right)^3\\
  & = & \cos^3 \theta + 3 \ii  \cos^2 \theta \sin \theta - 3 \cos \theta \sin^2
  \theta - \ii  \sin^3 \theta\\
  & = & \left( \cos^3 \theta - 3 \cos \theta \sin^2 \theta \right) + \ii  \left(
  3 \cos^2 \theta \sin \theta - \sin^3 \theta \right)
\end{eqnarray*}
En identifiant les parties réelles et imaginaires, on déduit que
\[ \cos 3 \theta = \cos^3 \theta - 3 \cos \theta \sin^2 \theta \quad \text{ et } \quad
   \sin 3 \theta = 3 \cos^2 \theta \sin \theta - \sin^3 \theta . \]
\end{exemple}

\bigskip

\defi{Linéarisation.}\index{linearisation@linéarisation}  On exprime
$\cos^n \theta$ ou $\sin^n \theta$ en fonction des $\cos k \theta$ et $\sin k
\theta$ pour $k$ allant de $0$ \`a~$n$.

\medskip

\emph{Méthode :} avec la formule d'Euler on écrit $\sin^n \theta = \left( \frac{e^{\ii
\theta} - e^{- \ii  \theta}}{2 \ii } \right)^n$. On développe à l'aide du binôme de Newton
puis on regroupe les termes par paires conjugu\'ees.

\begin{exemple}
\begin{eqnarray*}
  \sin^3 \theta & = & \left( \frac{e^{\ii  \theta} - e^{- \ii  \theta}}{2 \ii }
  \right)^3 \\
 & = & \frac{1}{- 8 \ii }  \left( (e^{\ii  \theta})^3 - 3 (e^{\ii  \theta})^2e^{- \ii  \theta}
+ 3 e^{\ii  \theta}(e^{-\ii  \theta})^2 - (e^{- \ii  \theta})^3 \right)\\
& = & \frac{1}{- 8 \ii }  \left( e^{3 \ii  \theta} - 3 e^{\ii  \theta} + 3 e^{-
  \ii  \theta} - e^{- 3 \ii  \theta} \right)\\
  & = & - \frac{1}{4} \left( \frac{e^{3 \ii  \theta} -  e^{-3 \ii \theta}}{2 \ii } - 3
  \frac{e^{\ii  \theta} - e^{-\ii  \theta}}{2 \ii } \right) \\
& = & - \frac{\sin 3 \theta}{4}
  + \frac{3 \sin \theta}{4}
\end{eqnarray*}
\end{exemple}

%------------------------------------------------------------------
%\subsection{Mini-exercices}

\begin{miniexercices}
\sauteligne
\begin{enumerate}
  \item Mettre les nombres suivants sont la forme module-argument (avec la notation exponentielle) :
$1$, $\ii$, $-1$, $-\ii$, $3\ii$, $1+\ii$, $\sqrt{3}-\ii$, $\overline{\sqrt{3}-\ii}$, $\frac{1}{\sqrt{3}-\ii}$, $(\sqrt{3}-\ii)^{20xx}$
où $20xx$ est l'année en cours.
  \item Calculer les racines $5$-ième de $\ii$.
  \item Calculer les racines carrées de $\frac{\sqrt{3}}{2}+\frac{\ii}{2}$ de deux façons différentes.
En déduire les valeurs de $\cos \frac{\pi}{12}$ et $\sin \frac{\pi}{12}$.
  \item Donner sans calcul la valeur de $\omega_0 + \omega_1 + \cdots + \omega_{n-1}$,
où les $\omega_i$ sont les racines $n$-ième de $1$.
  \item Développer $\cos (4\theta)$ ; linéariser $\cos^4 \theta$ ; calculer une primitive de $\theta\mapsto\cos^4 \theta$.
\end{enumerate}
\end{miniexercices}


%%%%%%%%%%%%%%%%%%%%%%%%%%%%%%%%%%%%%%%%%%%%%%%%%%%%%%%%%%%%%%%%
\section{Nombres complexes et géométrie}

On associe bijectivement à tout point $M$ du plan affine $\Rr^2$
%ou à un vecteur $\vec{u}$ du plan vectoriel $\Rr^2$
de coordonnées $(x,y)$, le nombre complexe $z=x+\ii y$ appelé son \defi{affixe}\index{affixe}.
% et parfois noté $z_M$.
% ou $z_{\vec{u}}$.

%---------------------------------------------------------------
\subsection{\'Equation complexe d'une droite}

Soit
$$ax+by=c$$
l'équation réelle d'une droite $\mathcal{D}$ : $a,b,c$ sont des nombres réels
($a$ et $b$ n'étant pas tous les deux nuls) d'inconnues $(x,y) \in \Rr^2$.



\'Ecrivons $z=x+\ii  y \in \Cc$, alors
$$x = \frac{z+\bar z}{2}, \quad y = \frac{z - \bar z}{2 \ii },$$
donc $\mathcal{D}$ a aussi pour équation
$a(z+\bar z) -\ii  b(z-\bar z)=2c$ ou encore $(a-\ii  b)z+(a+\ii  b)\bar z = 2c$.
Posons $\omega = a+\ii  b \in \Cc^*$ et $k=2c \in \Rr$ alors
l'équation complexe d'une droite est :
\mybox{$\bar \omega z + \omega \bar z = k$}
où $\omega \in \Cc^*$ et $k\in \Rr$.

\myfigure{1}{
\tikzinput{fig_complexes01} \qquad
\tikzinput{fig_complexes02}
}

%---------------------------------------------------------------
\subsection{\'Equation complexe d'un cercle}

Soit $\mathcal{C}(\Omega,r)$ le cercle de centre $\Omega$ et de rayon $r$.
C'est l'ensemble des points $M$ tel que $\mathrm{dist}(\Omega,M)=r$. Si l'on note $\omega$ l'affixe
de $\Omega$ et $z$ l'affixe de $M$. Nous obtenons:
$$\mathrm{dist}(\Omega,M) = r \iff  |z-\omega|=r \iff |z-\omega|^2=r^2 \iff (z-\omega)\overline{(z-\omega)}=r^2$$
et en développant nous trouvons que l'équation complexe du cercle centré en un point d'affixe $\omega$
et de rayon $r$ est :
\mybox{$z\bar z - \bar \omega z - \omega \bar z = r^2-|\omega|^2$}
où  $\omega \in \Cc$ et $r\in \Rr$.


%---------------------------------------------------------------
\subsection{\'Equation $\frac{|z-a|}{|z-b|}=k$}

\begin{proposition}
Soit $A,B$ deux points du plan et $k\in \Rr_+$.
L'ensemble des points $M$ tel que $\frac{MA}{MB}=k$
est
\begin{itemize}
  \item une droite qui est la médiatrice de $[AB]$, si $k=1$,
  \item un cercle, sinon.
\end{itemize}
\end{proposition}

\begin{exemple}
Prenons $A$ le point d'affixe $+1$,$B$ le point d'affixe $-1$.
Voici les figures pour plusieurs valeurs de $k$.

Par exemple pour $k=2$ le point $M$ dessiné vérifie bien $MA=2MB$.

\myfigure{1.3}{
\tikzinput{fig_complexes13bis}
}
\end{exemple}




\begin{proof}
Si les affixes de $A,B,M$
sont respectivement $a,b,z$, cela revient
à résoudre l'équation $\frac{|z-a|}{|z-b|}=k$.
\begin{align*}
\frac{|z-a|}{|z-b|}=k
  & \Longleftrightarrow |z-a|^2 = k^2 |z-b|^2 \\
  & \Longleftrightarrow (z-a)\overline{(z-a)} = k^2 (z-b)\overline{(z-b)} \\
  & \Longleftrightarrow (1-k^2) z \bar z -z(\bar a - k^2 \bar b) - \bar z(a-k^2b) + |a|^2-k^2|b|^2=0
\end{align*}
Donc si $k=1$, on pose $\omega = a-k^2b$ et  l'équation obtenue
$z\bar \omega  + \bar z \omega = |a|^2-k^2|b|^2$ est bien celle d'une droite.
Et bien sûr l'ensemble des points qui vérifient $MA=MB$ est la médiatrice de $[AB]$.
Si $k\neq 1$ on pose $\omega = \frac{a-k^2b}{1-k^2}$ alors l'équation obtenue est
$z \bar z -z \bar \omega - \bar z \omega = \frac{-|a|^2+k^2|b|^2}{1-k^2}$.
C'est l'équation d'un cercle de centre $\omega$ et de rayon $r$ satisfaisant
$r^2-|\omega|^2= \frac{-|a|^2+k^2|b|^2}{1-k^2}$,
soit $r^2 = \frac{|a-k^2b|^2}{(1-k^2)^2}+\frac{-|a|^2+k^2|b|^2}{1-k^2}$.
\end{proof}

Ces calculs se refont au cas par cas, il n'est pas nécessaire d'apprendre les formules.

%------------------------------------------------------------------
%\subsection{Mini-exercices}

\begin{miniexercices}
\sauteligne
\begin{enumerate}
  \item Calculer l'équation complexe de la droite passant par $1$ et $\ii$.
  \item Calculer l'équation complexe du cercle de centre $1+2\ii$ passant par $\ii$.
  \item Calculer l'équation complexe des solutions de $\dfrac{|z-\ii|}{|z-1|}=1$, puis dessiner les solutions.
  \item Même question avec $\dfrac{|z-\ii|}{|z-1|}=2$.
\end{enumerate}
\end{miniexercices}

\auteurs{
Arnaud Bodin,
Benjamin Boutin,
Pascal Romon
}

\finchapitre
\end{document}
