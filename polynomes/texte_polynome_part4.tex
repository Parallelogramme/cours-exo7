
%%%%%%%%%%%%%%%%%% PREAMBULE %%%%%%%%%%%%%%%%%%


\documentclass[12pt]{article}

\usepackage{amsfonts,amsmath,amssymb,amsthm}
\usepackage[utf8]{inputenc}
\usepackage[T1]{fontenc}
\usepackage[francais]{babel}


% packages
\usepackage{amsfonts,amsmath,amssymb,amsthm}
\usepackage[utf8]{inputenc}
\usepackage[T1]{fontenc}
%\usepackage{lmodern}

\usepackage[francais]{babel}
\usepackage{fancybox}
\usepackage{graphicx}

\usepackage{float}

%\usepackage[usenames, x11names]{xcolor}
\usepackage{tikz}
\usepackage{datetime}

\usepackage{mathptmx}
%\usepackage{fouriernc}
%\usepackage{newcent}
\usepackage[mathcal,mathbf]{euler}

%\usepackage{palatino}
%\usepackage{newcent}


% Commande spéciale prompteur

%\usepackage{mathptmx}
%\usepackage[mathcal,mathbf]{euler}
%\usepackage{mathpple,multido}

\usepackage[a4paper]{geometry}
\geometry{top=2cm, bottom=2cm, left=1cm, right=1cm, marginparsep=1cm}

\newcommand{\change}{{\color{red}\rule{\textwidth}{1mm}\\}}

\newcounter{mydiapo}

\newcommand{\diapo}{\newpage
\hfill {\normalsize  Diapo \themydiapo \quad \texttt{[\jobname]}} \\
\stepcounter{mydiapo}}


%%%%%%% COULEURS %%%%%%%%%%

% Pour blanc sur noir :
%\pagecolor[rgb]{0.5,0.5,0.5}
% \pagecolor[rgb]{0,0,0}
% \color[rgb]{1,1,1}



%\DeclareFixedFont{\myfont}{U}{cmss}{bx}{n}{18pt}
\newcommand{\debuttexte}{
%%%%%%%%%%%%% FONTES %%%%%%%%%%%%%
\renewcommand{\baselinestretch}{1.5}
\usefont{U}{cmss}{bx}{n}
\bfseries

% Taille normale : commenter le reste !
%Taille Arnaud
%\fontsize{19}{19}\selectfont

% Taille Barbara
%\fontsize{21}{22}\selectfont

%Taille François
\fontsize{25}{30}\selectfont

%Taille Pascal
%\fontsize{25}{30}\selectfont

%Taille Laura
%\fontsize{30}{35}\selectfont


%\myfont
%\usefont{U}{cmss}{bx}{n}

%\Huge
%\addtolength{\parskip}{\baselineskip}
}


% \usepackage{hyperref}
% \hypersetup{colorlinks=true, linkcolor=blue, urlcolor=blue,
% pdftitle={Exo7 - Exercices de mathématiques}, pdfauthor={Exo7}}


%section
% \usepackage{sectsty}
% \allsectionsfont{\bf}
%\sectionfont{\color{Tomato3}\upshape\selectfont}
%\subsectionfont{\color{Tomato4}\upshape\selectfont}

%----- Ensembles : entiers, reels, complexes -----
\newcommand{\Nn}{\mathbb{N}} \newcommand{\N}{\mathbb{N}}
\newcommand{\Zz}{\mathbb{Z}} \newcommand{\Z}{\mathbb{Z}}
\newcommand{\Qq}{\mathbb{Q}} \newcommand{\Q}{\mathbb{Q}}
\newcommand{\Rr}{\mathbb{R}} \newcommand{\R}{\mathbb{R}}
\newcommand{\Cc}{\mathbb{C}} 
\newcommand{\Kk}{\mathbb{K}} \newcommand{\K}{\mathbb{K}}

%----- Modifications de symboles -----
\renewcommand{\epsilon}{\varepsilon}
\renewcommand{\Re}{\mathop{\text{Re}}\nolimits}
\renewcommand{\Im}{\mathop{\text{Im}}\nolimits}
%\newcommand{\llbracket}{\left[\kern-0.15em\left[}
%\newcommand{\rrbracket}{\right]\kern-0.15em\right]}

\renewcommand{\ge}{\geqslant}
\renewcommand{\geq}{\geqslant}
\renewcommand{\le}{\leqslant}
\renewcommand{\leq}{\leqslant}

%----- Fonctions usuelles -----
\newcommand{\ch}{\mathop{\mathrm{ch}}\nolimits}
\newcommand{\sh}{\mathop{\mathrm{sh}}\nolimits}
\renewcommand{\tanh}{\mathop{\mathrm{th}}\nolimits}
\newcommand{\cotan}{\mathop{\mathrm{cotan}}\nolimits}
\newcommand{\Arcsin}{\mathop{\mathrm{Arcsin}}\nolimits}
\newcommand{\Arccos}{\mathop{\mathrm{Arccos}}\nolimits}
\newcommand{\Arctan}{\mathop{\mathrm{Arctan}}\nolimits}
\newcommand{\Argsh}{\mathop{\mathrm{Argsh}}\nolimits}
\newcommand{\Argch}{\mathop{\mathrm{Argch}}\nolimits}
\newcommand{\Argth}{\mathop{\mathrm{Argth}}\nolimits}
\newcommand{\pgcd}{\mathop{\mathrm{pgcd}}\nolimits} 

\newcommand{\Card}{\mathop{\text{Card}}\nolimits}
\newcommand{\Ker}{\mathop{\text{Ker}}\nolimits}
\newcommand{\id}{\mathop{\text{id}}\nolimits}
\newcommand{\ii}{\mathrm{i}}
\newcommand{\dd}{\mathrm{d}}
\newcommand{\Vect}{\mathop{\text{Vect}}\nolimits}
\newcommand{\Mat}{\mathop{\mathrm{Mat}}\nolimits}
\newcommand{\rg}{\mathop{\text{rg}}\nolimits}
\newcommand{\tr}{\mathop{\text{tr}}\nolimits}
\newcommand{\ppcm}{\mathop{\text{ppcm}}\nolimits}

%----- Structure des exercices ------

\newtheoremstyle{styleexo}% name
{2ex}% Space above
{3ex}% Space below
{}% Body font
{}% Indent amount 1
{\bfseries} % Theorem head font
{}% Punctuation after theorem head
{\newline}% Space after theorem head 2
{}% Theorem head spec (can be left empty, meaning ‘normal’)

%\theoremstyle{styleexo}
\newtheorem{exo}{Exercice}
\newtheorem{ind}{Indications}
\newtheorem{cor}{Correction}


\newcommand{\exercice}[1]{} \newcommand{\finexercice}{}
%\newcommand{\exercice}[1]{{\tiny\texttt{#1}}\vspace{-2ex}} % pour afficher le numero absolu, l'auteur...
\newcommand{\enonce}{\begin{exo}} \newcommand{\finenonce}{\end{exo}}
\newcommand{\indication}{\begin{ind}} \newcommand{\finindication}{\end{ind}}
\newcommand{\correction}{\begin{cor}} \newcommand{\fincorrection}{\end{cor}}

\newcommand{\noindication}{\stepcounter{ind}}
\newcommand{\nocorrection}{\stepcounter{cor}}

\newcommand{\fiche}[1]{} \newcommand{\finfiche}{}
\newcommand{\titre}[1]{\centerline{\large \bf #1}}
\newcommand{\addcommand}[1]{}
\newcommand{\video}[1]{}

% Marge
\newcommand{\mymargin}[1]{\marginpar{{\small #1}}}



%----- Presentation ------
\setlength{\parindent}{0cm}

%\newcommand{\ExoSept}{\href{http://exo7.emath.fr}{\textbf{\textsf{Exo7}}}}

\definecolor{myred}{rgb}{0.93,0.26,0}
\definecolor{myorange}{rgb}{0.97,0.58,0}
\definecolor{myyellow}{rgb}{1,0.86,0}

\newcommand{\LogoExoSept}[1]{  % input : echelle
{\usefont{U}{cmss}{bx}{n}
\begin{tikzpicture}[scale=0.1*#1,transform shape]
  \fill[color=myorange] (0,0)--(4,0)--(4,-4)--(0,-4)--cycle;
  \fill[color=myred] (0,0)--(0,3)--(-3,3)--(-3,0)--cycle;
  \fill[color=myyellow] (4,0)--(7,4)--(3,7)--(0,3)--cycle;
  \node[scale=5] at (3.5,3.5) {Exo7};
\end{tikzpicture}}
}



\theoremstyle{definition}
%\newtheorem{proposition}{Proposition}
%\newtheorem{exemple}{Exemple}
%\newtheorem{theoreme}{Théorème}
\newtheorem{lemme}{Lemme}
\newtheorem{corollaire}{Corollaire}
%\newtheorem*{remarque*}{Remarque}
%\newtheorem*{miniexercice}{Mini-exercices}
%\newtheorem{definition}{Définition}




%definition d'un terme
\newcommand{\defi}[1]{{\color{myorange}\textbf{\emph{#1}}}}
\newcommand{\evidence}[1]{{\color{blue}\textbf{\emph{#1}}}}



 %----- Commandes divers ------

\newcommand{\codeinline}[1]{\texttt{#1}}

%%%%%%%%%%%%%%%%%%%%%%%%%%%%%%%%%%%%%%%%%%%%%%%%%%%%%%%%%%%%%
%%%%%%%%%%%%%%%%%%%%%%%%%%%%%%%%%%%%%%%%%%%%%%%%%%%%%%%%%%%%%


\begin{document}

\debuttexte

%%%%%%%%%%%%%%%%%%%%%%%%%%%%%%%%%%%%%%%%%%%%%%%%%%%%%%%%%%%
\diapo

\change

Les fractions rationnelles sont le quotient de deux polynômes.

\change


Toute fraction rationnelle se décompose comme une somme de fractions rationnelles
élémentaires que l'on appelle des \og éléments simples \fg. 

\change

Mais les éléments simples 
sont différents sur $\Cc$ 

\change

ou sur $\Rr$.


%%%%%%%%%%%%%%%%%%%%%%%%%%%%%%%%%%%%%%%%%%%%%%%%%%%%%%%%%%%
\diapo

Une \defi{fraction rationnelle} est une expression
de la forme
$$F=\frac{P}{Q}$$


$P,Q \in \Kk[X]$ sont deux polynômes à coefficients dans $\Kk$

avec $Q$ qui n'est pas le polynôme constant $=0$.

 




%%%%%%%%%%%%%%%%%%%%%%%%%%%%%%%%%%%%%%%%%%%%%%%%%%%%%%%%%%%
\diapo

Voici le théorème de décomposition en éléments simples sur le corps des nombres complexes.


On part d'une fraction rationnelle $P/Q$,

et on décompose $Q$ en produit de facteurs irréductibles sur $\Cc$ :

$Q=(X-\alpha_1)^{k_1}\cdots(X-\alpha_r)^{k_r}$

\change

Alors il existe une et une seule écriture de $P/Q$ sous la forme :

d'un polynôme $E$,

de fractions 

une constante sur $(X-\alpha_1)^{k_1}$

une constante sur $(X-\alpha_1)^{k_1-1}$

etc jusqu'à une constante sur $(X-\alpha_1)$.

Noter que $(X-\alpha_1)$ est le premier facteur de $Q$ et qu'il y est présent avec un exposant de $k_1$.

On fait la même chose pour les second facteur $(X-\alpha_2)$ avec un exposant $k_2$ :

une constante sur $(X-\alpha_2)^{k_2}$

etc jusqu'à une constante sur $(X-\alpha_2)$.

Et on recommence ainsi pour tous les facteurs.



\change

Le polynôme $E$ s'appelle la \defi{partie polynomiale} (ou \defi{partie entière}).

\change

Les termes $\frac{a}{(X-\alpha)^i}$ sont les \defi{éléments simples} sur $\Cc$.

\change


Voici un premier exemple
 $\frac{1}{X^2+1}$
 
 Le dénominateur $X^2+1$ se décompose en $(X+\ii)$ facteur de $(X-\ii)$.
 
 Donc la décomposition est de la forme  $\frac{a}{X+\ii} + \frac{b}{X-\ii}$ 
 
 La partie entière est nulle,
 
 et je vous laisse vérifier que les constantes qui conviennent sont $a=\frac12 \ii$, $b=-\frac12\ii$.
 
 
 \change
 
Pour cette fraction plus compliquée ;

La partie entière est le polynôme $X+1$.

Le dénominateur est déjà décomposé $(X-2)^2(X+3)$

donc il y a trois éléments simples possibles 

$\frac{cst}{(X-2)^2}$ + $\frac{cst}{X-2}$ 

car le facteur $(X-2)$ est au carré dans $Q$


+ $\frac{cst}{X+3}$

Je vous laisse vérifier que les cst $-1$, $2$ et $-1$.


%%%%%%%%%%%%%%%%%%%%%%%%%%%%%%%%%%%%%%%%%%%%%%%%%%%%%%%%%%%
\diapo

Comment se calcule cette décomposition ?

Voyons cela sur l'exemple de cette fraction $\frac{P}{Q}$.



\change

La \textbf{première étape} est de déterminer la partie polynomiale : 
La partie polynomiale est le quotient de la division euclidienne de $P$ par $Q$.


\change

Il faut calculer cette division euclidienne et on obtient $P(X) = (X^2+1)Q(X)+ 2X^2-5X+9$.

\change

Donc la partie polynomiale est $E(X)=X^2+1$

\change

la fraction s'écrit
$\frac{P(X)}{Q(X)} =X^2+1 + \frac{2X^2-5X+9}{Q(X)}$

Notez que l'on s'est ramené par division euclidienne à une fraction $\frac{2X^2-5X+9}{Q(X)}$ où 
le degré du numérateur est strictement plus petit que le degré du dénominateur.

\change

\textbf{Deuxième étape : on factorise le dénominateur.}



$Q$ a pour racine évidente $+1$ (racine double) et $-2$ (racine simple) 

\change

et se factorise donc $Q(X)= (X-1)^2(X+2)$.

\change

\textbf{Troisième étape : décomposition théorique en éléments simples.}

\change

Le théorème de décomposition en éléments simples nous dit qu'il existe une unique décomposition :
$\frac{P(X)}{Q(X)}= E(X)+ \frac{a}{(X-1)^2} + \frac{b}{X-1} + \frac{c}{X+2}$.

Nous savons déjà que $E(X)=X^2+1$, il reste à trouver les nombres $a,b,c$.


%%%%%%%%%%%%%%%%%%%%%%%%%%%%%%%%%%%%%%%%%%%%%%%%%%%%%%%%%%%
\diapo

\textbf{Quatrième étape} : détermination des coefficients $a,b,c$.


Nous avons écrit notre fraction sous cette forme

et nous voulons l'écrire sous la forme  

$\frac{a}{(X-1)^2} + \frac{b}{X-1} + \frac{c}{X+2}$


Voici une première façon de déterminer $a,b,c$.

\change
On récrit cette somme de fraction sous la forme d'une seule fraction en réduisant au même dénominateur.

Ensuite on identifie cette fraction avec ce que l'on veut obtenir
au même dénominateur.

\change

Le coefficient de $X^2$, $b+c$ doit être égal à $2$,
le coefficient de $X$, $a+b-2c$ doit être égal à $-5$,
enfin le coefficient constant $2a-2b+c$ doit être $9$.

\change

Cela donne un système de trois équations à trois inconnues, 
que l'on sait et que l'on doit résoudre 

\change

et qui admet l'unique solution
$a=2$, $b=-1$, $c=3$.

\change

On a donc écrit $\frac{P}{Q}$ sous la forme 
$X^2+1 + \frac{2}{(X-1)^2} + \frac{-1}{X-1} + \frac{3}{X+2}.$

Les coefficients obtenus sont uniques mais la méthode que l'on a employée est souvent la plus longue.


%%%%%%%%%%%%%%%%%%%%%%%%%%%%%%%%%%%%%%%%%%%%%%%%%%%%%%%%%%%
\diapo

Voici une autre méthode plus efficace pour déterminer les coefficients $a,b,c$.


Notons $\frac{P_1(X)}{Q(X)}$ la fraction $\frac{2X^2-5X+9}{(X-1)^2(X+2)}$ dont la décomposition théorique est :
$\frac{a}{(X-1)^2} + \frac{b}{X-1} + \frac{c}{X+2}$

(on a pris la fraction sans sa partie polynomiale).


\change

Pour déterminer $a$ on multiplie la fraction $\frac{P_1}{Q}$ par $(X-1)^2$ 

$(X-1)^2$ est le dénominateur correspondant à la constante $a$).

et on évalue en $x=1$.

\change

Tout d'abord en partant de la décomposition théorique on a :
$$F_1(X)= (X-1)^2 \frac{P_1(X)}{Q(X)} = a + b(X-1) + c\frac{(X-1)^2}{X+2}$$

\change

Donc lorsque l'on fait $x=1$ on obtient $F_1(1)=a$

\change

Mais en repartant de l'écriture $\frac{P_1(X)}{Q(X)}$ on sait que $F_1(X) = \frac{2X^2-5X+9}{X+2}$

\change

donc on calcule $F_1(1)=2$.

\change

En conclusion $a$ est égal à $2$.

\change

On fait le même processus pour déterminer $c$ : on 
multiplie par $(X+2)$ et on évalue en $-2$.

On pose $F_2(X) = (X+2)\frac{P_1(X)}{Q(X)}$

\change

Cela vaut d'une part $\frac{2X^2-5X+9}{(X-1)^2}$ 

\change

et par la décomposition théorique c'est aussi
$a\frac{X+2}{(X-1)^2} + b\frac{X+2}{X-1} + c$ 

Maintenant on fait $x=-2$.

On a tout fait pour que $F_2(-2) = c$, mais d'autre part $F_2(-2) = 3$, 

On obtient $c=3$.

Il nous reste $b$ à trouver.

Comme les coefficients sont uniques tous les moyens sont bons pour les déterminer.

\change


Par exemple lorsque l'on évalue la décomposition théorique 
en $x=0$, on obtient :
$$\frac{P_1(0)}{Q(0)} = a - b + \frac c2$$

Mais ici en évaluant directement $\frac{P_1}{Q}$ en $0$ cela donne $\frac{9}{2}$

Comme on connaît déjà les valeurs de $a$ et $c$ on obtient $b=-1$.

\change

On retrouve bien sûr les mêmes coefficients et la même décomposition que précédemment.


%%%%%%%%%%%%%%%%%%%%%%%%%%%%%%%%%%%%%%%%%%%%%%%%%%%%%%%%%%%
\diapo

Voici la version de la décomposition en éléments simples pour les nombres réels.

On part une nouvelle fois d'une fraction rationnelle $P/Q$.

Notre fraction se décompose tout d'abord comme la somme
d'une partie entière (qui est encore une fois le quotient de la division euclidienne de $P$ par $Q$)
et d'éléments simples. 
Attention cette fois il y a deux type d'éléments simples :

le premier type est le même qu'avant : ce sont des fractions $\frac{a}{(X-\alpha)^i}$,

le second type sont des fractions $\frac{aX+b}{(X^2+\alpha X + \beta)^i}$,

notez bien que le numérateur est ici de degré $1$.

\change

Les éléments simples correspondent à la décomposition de $Q$ en facteurs irréductibles.
Il y a des facteurs de degré $1$ qui correspondent à ce premier type d'éléments simples
et des facteurs irréductibles de degré $2$, qui correspondent à ce deuxième type.


\change

Comme auparavant les exposants $i$ qui apparaissent dans les dénominateurs sont inférieurs ou égaux à la 
puissance correspondante dans cette factorisation.
% 
% Par exemple si $Q(X)$ a un facteur irréductible de degré $2$ avec un exposant $3$ alors
% on peut avoir des éléments simples avec des exposants $i=1$, $i=2$ ou $i=3$ (où bien 
% les trois à la fois).

%%%%%%%%%%%%%%%%%%%%%%%%%%%%%%%%%%%%%%%%%%%%%%%%%%%%%%%%%%%
\diapo

Voyons un exemple avec cette fraction rationnelle 
$\frac{P(X)}{Q(X)}$.

\change

Comme $\deg P < \deg Q=5$ alors la partie polynomiale $E(X)$ est nulle.

\change

Le dénominateur est déjà factorisé sur $\Rr$ car $X^2+X+1$ est irréductible.

\change

La décomposition théorique est donc :
$$\frac{P(X)}{Q(X)} = \frac{aX+b}{(X^2+X+1)^2}+\frac{cX+d}{X^2+X+1}+\frac{e}{X-1}.$$

Noter que comme ce facteur irréductible à un exposant $2$,
alors on a deux éléments simples pour ce facteurs. Un avec le facteur de degré $2$
et l'autre de degré $1$.

Pour ces éléments simples le numérateur est un polynôme de degré $1$.

Le dernier facteur $X-1$ a un exposant $1$ donc le dernier élément simple 
est une constante divisé par le facteur de degré $1$.


\change

Il faut ensuite mener au mieux les calculs pour déterminer les coefficients afin d'obtenir :
$$\frac{P(X)}{Q(X)} = \frac{2X+1}{(X^2+X+1)^2}+\frac{-1}{X^2+X+1}+\frac{3}{X-1}.$$

%%%%%%%%%%%%%%%%%%%%%%%%%%%%%%%%%%%%%%%%%%%%%%%%%%%%%%%%%%%
\diapo

La décomposition des fractions rationnelles vous permet de vérifier 
que vous savez mener à bien des calculs de façon rigoureuse et persévérante !

\end{document}