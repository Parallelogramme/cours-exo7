
%%%%%%%%%%%%%%%%%% PREAMBULE %%%%%%%%%%%%%%%%%%


\documentclass[12pt]{article}

\usepackage{amsfonts,amsmath,amssymb,amsthm}
\usepackage[utf8]{inputenc}
\usepackage[T1]{fontenc}
\usepackage[francais]{babel}


% packages
\usepackage{amsfonts,amsmath,amssymb,amsthm}
\usepackage[utf8]{inputenc}
\usepackage[T1]{fontenc}
%\usepackage{lmodern}

\usepackage[francais]{babel}
\usepackage{fancybox}
\usepackage{graphicx}

\usepackage{float}

%\usepackage[usenames, x11names]{xcolor}
\usepackage{tikz}
\usepackage{datetime}

\usepackage{mathptmx}
%\usepackage{fouriernc}
%\usepackage{newcent}
\usepackage[mathcal,mathbf]{euler}

%\usepackage{palatino}
%\usepackage{newcent}


% Commande spéciale prompteur

%\usepackage{mathptmx}
%\usepackage[mathcal,mathbf]{euler}
%\usepackage{mathpple,multido}

\usepackage[a4paper]{geometry}
\geometry{top=2cm, bottom=2cm, left=1cm, right=1cm, marginparsep=1cm}

\newcommand{\change}{{\color{red}\rule{\textwidth}{1mm}\\}}

\newcounter{mydiapo}

\newcommand{\diapo}{\newpage
\hfill {\normalsize  Diapo \themydiapo \quad \texttt{[\jobname]}} \\
\stepcounter{mydiapo}}


%%%%%%% COULEURS %%%%%%%%%%

% Pour blanc sur noir :
%\pagecolor[rgb]{0.5,0.5,0.5}
% \pagecolor[rgb]{0,0,0}
% \color[rgb]{1,1,1}



%\DeclareFixedFont{\myfont}{U}{cmss}{bx}{n}{18pt}
\newcommand{\debuttexte}{
%%%%%%%%%%%%% FONTES %%%%%%%%%%%%%
\renewcommand{\baselinestretch}{1.5}
\usefont{U}{cmss}{bx}{n}
\bfseries

% Taille normale : commenter le reste !
%Taille Arnaud
%\fontsize{19}{19}\selectfont

% Taille Barbara
%\fontsize{21}{22}\selectfont

%Taille François
\fontsize{25}{30}\selectfont

%Taille Pascal
%\fontsize{25}{30}\selectfont

%Taille Laura
%\fontsize{30}{35}\selectfont


%\myfont
%\usefont{U}{cmss}{bx}{n}

%\Huge
%\addtolength{\parskip}{\baselineskip}
}


% \usepackage{hyperref}
% \hypersetup{colorlinks=true, linkcolor=blue, urlcolor=blue,
% pdftitle={Exo7 - Exercices de mathématiques}, pdfauthor={Exo7}}


%section
% \usepackage{sectsty}
% \allsectionsfont{\bf}
%\sectionfont{\color{Tomato3}\upshape\selectfont}
%\subsectionfont{\color{Tomato4}\upshape\selectfont}

%----- Ensembles : entiers, reels, complexes -----
\newcommand{\Nn}{\mathbb{N}} \newcommand{\N}{\mathbb{N}}
\newcommand{\Zz}{\mathbb{Z}} \newcommand{\Z}{\mathbb{Z}}
\newcommand{\Qq}{\mathbb{Q}} \newcommand{\Q}{\mathbb{Q}}
\newcommand{\Rr}{\mathbb{R}} \newcommand{\R}{\mathbb{R}}
\newcommand{\Cc}{\mathbb{C}} 
\newcommand{\Kk}{\mathbb{K}} \newcommand{\K}{\mathbb{K}}

%----- Modifications de symboles -----
\renewcommand{\epsilon}{\varepsilon}
\renewcommand{\Re}{\mathop{\text{Re}}\nolimits}
\renewcommand{\Im}{\mathop{\text{Im}}\nolimits}
%\newcommand{\llbracket}{\left[\kern-0.15em\left[}
%\newcommand{\rrbracket}{\right]\kern-0.15em\right]}

\renewcommand{\ge}{\geqslant}
\renewcommand{\geq}{\geqslant}
\renewcommand{\le}{\leqslant}
\renewcommand{\leq}{\leqslant}

%----- Fonctions usuelles -----
\newcommand{\ch}{\mathop{\mathrm{ch}}\nolimits}
\newcommand{\sh}{\mathop{\mathrm{sh}}\nolimits}
\renewcommand{\tanh}{\mathop{\mathrm{th}}\nolimits}
\newcommand{\cotan}{\mathop{\mathrm{cotan}}\nolimits}
\newcommand{\Arcsin}{\mathop{\mathrm{Arcsin}}\nolimits}
\newcommand{\Arccos}{\mathop{\mathrm{Arccos}}\nolimits}
\newcommand{\Arctan}{\mathop{\mathrm{Arctan}}\nolimits}
\newcommand{\Argsh}{\mathop{\mathrm{Argsh}}\nolimits}
\newcommand{\Argch}{\mathop{\mathrm{Argch}}\nolimits}
\newcommand{\Argth}{\mathop{\mathrm{Argth}}\nolimits}
\newcommand{\pgcd}{\mathop{\mathrm{pgcd}}\nolimits} 

\newcommand{\Card}{\mathop{\text{Card}}\nolimits}
\newcommand{\Ker}{\mathop{\text{Ker}}\nolimits}
\newcommand{\id}{\mathop{\text{id}}\nolimits}
\newcommand{\ii}{\mathrm{i}}
\newcommand{\dd}{\mathrm{d}}
\newcommand{\Vect}{\mathop{\text{Vect}}\nolimits}
\newcommand{\Mat}{\mathop{\mathrm{Mat}}\nolimits}
\newcommand{\rg}{\mathop{\text{rg}}\nolimits}
\newcommand{\tr}{\mathop{\text{tr}}\nolimits}
\newcommand{\ppcm}{\mathop{\text{ppcm}}\nolimits}

%----- Structure des exercices ------

\newtheoremstyle{styleexo}% name
{2ex}% Space above
{3ex}% Space below
{}% Body font
{}% Indent amount 1
{\bfseries} % Theorem head font
{}% Punctuation after theorem head
{\newline}% Space after theorem head 2
{}% Theorem head spec (can be left empty, meaning ‘normal’)

%\theoremstyle{styleexo}
\newtheorem{exo}{Exercice}
\newtheorem{ind}{Indications}
\newtheorem{cor}{Correction}


\newcommand{\exercice}[1]{} \newcommand{\finexercice}{}
%\newcommand{\exercice}[1]{{\tiny\texttt{#1}}\vspace{-2ex}} % pour afficher le numero absolu, l'auteur...
\newcommand{\enonce}{\begin{exo}} \newcommand{\finenonce}{\end{exo}}
\newcommand{\indication}{\begin{ind}} \newcommand{\finindication}{\end{ind}}
\newcommand{\correction}{\begin{cor}} \newcommand{\fincorrection}{\end{cor}}

\newcommand{\noindication}{\stepcounter{ind}}
\newcommand{\nocorrection}{\stepcounter{cor}}

\newcommand{\fiche}[1]{} \newcommand{\finfiche}{}
\newcommand{\titre}[1]{\centerline{\large \bf #1}}
\newcommand{\addcommand}[1]{}
\newcommand{\video}[1]{}

% Marge
\newcommand{\mymargin}[1]{\marginpar{{\small #1}}}



%----- Presentation ------
\setlength{\parindent}{0cm}

%\newcommand{\ExoSept}{\href{http://exo7.emath.fr}{\textbf{\textsf{Exo7}}}}

\definecolor{myred}{rgb}{0.93,0.26,0}
\definecolor{myorange}{rgb}{0.97,0.58,0}
\definecolor{myyellow}{rgb}{1,0.86,0}

\newcommand{\LogoExoSept}[1]{  % input : echelle
{\usefont{U}{cmss}{bx}{n}
\begin{tikzpicture}[scale=0.1*#1,transform shape]
  \fill[color=myorange] (0,0)--(4,0)--(4,-4)--(0,-4)--cycle;
  \fill[color=myred] (0,0)--(0,3)--(-3,3)--(-3,0)--cycle;
  \fill[color=myyellow] (4,0)--(7,4)--(3,7)--(0,3)--cycle;
  \node[scale=5] at (3.5,3.5) {Exo7};
\end{tikzpicture}}
}



\theoremstyle{definition}
%\newtheorem{proposition}{Proposition}
%\newtheorem{exemple}{Exemple}
%\newtheorem{theoreme}{Théorème}
\newtheorem{lemme}{Lemme}
\newtheorem{corollaire}{Corollaire}
%\newtheorem*{remarque*}{Remarque}
%\newtheorem*{miniexercice}{Mini-exercices}
%\newtheorem{definition}{Définition}




%definition d'un terme
\newcommand{\defi}[1]{{\color{myorange}\textbf{\emph{#1}}}}
\newcommand{\evidence}[1]{{\color{blue}\textbf{\emph{#1}}}}



 %----- Commandes divers ------

\newcommand{\codeinline}[1]{\texttt{#1}}

%%%%%%%%%%%%%%%%%%%%%%%%%%%%%%%%%%%%%%%%%%%%%%%%%%%%%%%%%%%%%
%%%%%%%%%%%%%%%%%%%%%%%%%%%%%%%%%%%%%%%%%%%%%%%%%%%%%%%%%%%%%


\begin{document}

\debuttexte

%%%%%%%%%%%%%%%%%%%%%%%%%%%%%%%%%%%%%%%%%%%%%%%%%%%%%%%%%%%
\diapo

Nous abordons les propriétés arithmétiques des polynômes.

\change

Vous allez voir qu'il y a de grandes similarités entre l'arithmétique des entiers
et l'arithmétique des polynômes :

\change

On commence par la Division euclidienne de polynômes.

\change

Nous définirons le pgcd de deux polynômes.

\change

Nous étudierons le théorème de Bézout 

et ses conséquences.



%%%%%%%%%%%%%%%%%%%%%%%%%%%%%%%%%%%%%%%%%%%%%%%%%%%%%%%%%%%
\diapo

Prenons deux polynômes $A,B$ à coefficients dans $\Kk$.

Rappelons que $\Kk$ désigne l'un des corps $\Qq, \Rr$ ou $\Cc$.

On dit que $B$ \defi{divise} $A$ s'il existe  $Q\in\Kk[X]$ tel que $A=BQ$.

\change

On note alors $B|A$.

On dit aussi que $A$ est multiple de $B$ ou que $A$ est divisible par $B$.

\change

Commençons par les propriétés évidentes comme $A|A$, $1|A$ et $A|0$

\change

Puis nous avons que si à la fois
 $A|B$ et $B|A$, alors, on ne peut dire que $A=B$ mais presque,
 
il existe une constante non nulle $\lambda$ telle que   $A=\lambda B$.
 
\change

Si $A|B$ et $B|C$ alors $A|C$.

\change

Enfin si $C|A$ et si $C|B$, et quelque soit les polynômes $U,V$ alors  $C|(AU+BV)$.

%%%%%%%%%%%%%%%%%%%%%%%%%%%%%%%%%%%%%%%%%%%%%%%%%%%%%%%%%%%
\diapo

Vous connaissez la division euclidienne de deux entiers, voici la division 
euclidienne de deux polynômes.

Fixons deux polynômes $A,B$ --avec $B$ non nul.

Il existe un polynôme $Q$ et il existe un polynôme $R$

tels que $A=BQ+R$.

En plus on peut choisir $Q$ et $R$ de sorte que $\deg R < \deg B$.

Avec cette condition de degré les polynômes $Q$ et $R$ sont uniques.

Théorème : ``Il \evidence{existe} des polynômes $Q$ et $R$ \evidence{uniques} tels que :
$A=BQ+R \quad \text{ et } \quad \deg R < \deg B$''

\change

Bien sûr par analogie avec l'arithmétique des entiers :

$Q$ s'appelle le \defi{quotient} et $R$ le \defi{reste}.

\change

Retenez bien la condition sur le degré, qui permet l'unicité :
$\deg R < \deg B$

cela signifie que ou bien $R=0$
ou bien $0 \le \deg R < \deg B$.

\change

Bien sûr, si le reste est nul alors $A=BQ$,
donc $B|A$.

Et réciproquement si $B|A$ alors le reste est nul.


%%%%%%%%%%%%%%%%%%%%%%%%%%%%%%%%%%%%%%%%%%%%%%%%%%%%%%%%%%%
\diapo


Passons à la pratique de la division euclidienne,

si vous n'êtes pas à l'aise avec les divisions, commencez donc par revoir les divisions d'entiers

car pour les polynômes la démarche est similaire.

On souhaite calculer la division euclidienne de ce polynôme $A$ par ce polynôme $B$.

\change


On va trouver un quotient $Q$ et un reste $R$


\change

qui vérifient $A=BQ+R$ et $\deg R = 1$ qui est strictement plus petit que $\deg B=2$.


\change

On pose une division de polynômes comme on pose une division euclidienne de deux entiers.

Ici $A$, ici $B$, ici on va chercher le quotient,

et le reste sera là.

[split]

Combien de fois peut-on mettre le polynômes $X^2-X+1$,

dans $2X^4-X^3-2X^2+3X-1$ ?

\change

La réponse est $2X^2$.

En effet si on multiplie $X^2-X+1$ par $2X^2$, alors on va obtenir un polynôme
qui commence par $2X^4$ comme lui.

Le quotient commence donc par $2X^2$.

\change

Pour continuer on multiplie $2X^2$ par $X^2-X+1$ qui vaut $2X^4-2X^3+2X^2$.

\change

Que l'on soustrait à notre polynôme de départ.

On a tout fait pour les termes de degré $4$ s'annulent.

\change

Donc pas de terme de degré $4$,

$-X^3$ moins ($-2X^3$) donne $+X^3$,

$-2X^2$ moins $2X^2$ donne $-4X^2$.

Puis $3X$ moins $0$, $-1$ moins $0$.

On obtient donc un reste provisoire égal

à $X^3-4X^2+3X-1$,

mais ce reste est de degré $3$ n'est pas de degré inférieur au degré du diviseur [montrer].

Donc on recommence.

Combien de fois peut-on mettre  $X^2-X+1$,

dans $X^3-4X^2+3X-1$,

réponse $X$ fois,

\change

on calcule $X$ fois $X^2-X+1$,

\change

que l'on soustrait au reste provisoire.

\change


Nouveau reste est $-3X^2 +2X - 1$.

Bien sûr on est passé d'un reste de degré $3$ à un reste de degré $2$.

Mais on veut un reste de degré strictement inférieur à $2$ donc on recommence.

\change

Combien de fois peut-on mettre  $X^2-X+1$,

dans $-3X^2 +2X - 1$,

réponse $-3$ fois.

\change

On multiplie  $X^2-X+1$ par $-3$

\change

On soustrait

\change


Et on obtient un nouveau reste qui est $-X+2$.

Comme ce reste est de degré $1$ il est bien strictement inférieur au degré de notre diviseur.

On arête ici le processus.

On trouve donc un quotient qui est $Q = 2X^2+X-3$

et le reste $R=-X+2$.


On n'oublie pas de vérifier qu'effectivement $A=BQ+R$.

%%%%%%%%%%%%%%%%%%%%%%%%%%%%%%%%%%%%%%%%%%%%%%%%%%%%%%%%%%%
\diapo

Calculons un autre exemple avec
$X^4-3X^3+X+1$ divisé par $X^2+2$ 


\change

\change

On pose la division.

On prend soin de bien présenter les poynômes.

Par exemple ici on laisse de la place pour les monomes en $X^2$.

\change

Pour obtenir un terme $X^4$,

on doit multiplier $X^2+2$ par $X^2$.

\change

On calcule $X^2$ par $X^2+2$

et on soustrait pour obtenir ce polynôme de degré $3$.


On continue, pour obtenir un polynôme qui commence par $-3X^3$ on doit
multiplier $X^2+2$ par $-3X$.

\change 

Donc le quotient commence par $X^2-3X$

\change

On calcule $X^2+2$ fois $-3X$

et on soustrait.

\change

On doit multiplier par $-2$ pour obtenir un coeffcient $-2X^2+...$


\change

on soustrait
 
et on obtient un reste qui vaut $7X+5$.

Son degré est bien strictement inferieur au degré du quotient donc on s'arete là !

\change

On trouve un quotient égal à $X^2-3X-2$ et 
un reste égale à $7X+5$ et on fait une vérification rapide.



%%%%%%%%%%%%%%%%%%%%%%%%%%%%%%%%%%%%%%%%%%%%%%%%%%%%%%%%%%%
\diapo

Nous allons définir le pgcd de deux polynômes.

Soient  $A,B$ deux polynômes non nuls.

Il existe un unique polynôme unitaire de plus grand degré qui
divise à la fois $A$ et $B$.

Rappelons qu'unitaire signifie que le coefficient dominant est $1$.

\change

Cet unique polynôme est appelé le \defi{pgcd} 
(plus grand diviseur commun) de $A$ et $B$ 
que l'on note $\pgcd(A,B)$.

\change

Notez que l'on suppose que le $\pgcd$ est unitaire pour avoir l'unicité.


% 
% \change
% 
% Comme pour les entiers
% Si $A|B$
% 
% alors le pgcd de $A$ est de $B$ est $A$.
% 
% Mais comme on n'est pas sûr que soit unitaire,
% on doit diviser par son coefficient dominant.

\change

Enfin une propriété similaire à celle des entiers :
 Si $A=BQ+R$ alors $\pgcd(A,B) = \pgcd(B,R)$
 
 Ce sera important pour la suite.
 
 
%%%%%%%%%%%%%%%%%%%%%%%%%%%%%%%%%%%%%%%%%%%%%%%%%%%%%%%%%%%
\diapo

Le calcul des pgcd se fait comme pour les entiers par l'algorithme d'Euclide,

c'est à dire une successions de division euclidiennes.

\change

On souhaite calculer le pgcd de deux polynômes $A$ et $B$

on écrit d'abord la division euclidienne de $A$ par $B$.

On obtient un certain reste $R_1$.

\change

Si ce reste n'est pas nul alors 
on divise maintenant $B$ par $R_1$.


\change

Et on recommence,

\change

à chaque étape c'est le diviseur de la ligne au-dessus
que l'on divise par le reste de la ligne d'au dessus.

\change

On continue ainsi

\change

Jusqu'à ce que l'on obtienne un reste nul.

\change

Le pgcd est le dernier reste non nul : ici $R_k$.

Au besoin il faut le rendre unitaire en divisant par son coefficient dominant.

\change

Nous avons donc effectuer une série de divisions euclidiennes.

\change

Et à chaque étape le degré du reste diminue strictement,

\change

donc nous sommes sûr qu'au  bout d'un certain temps le reste sera nul.

\change

Enfin ce qui justifie que l'algorithme donne le bon résultat

c'est que nous avons vu que $\pgcd(A,B)=\gcd(B,R_1)$

puis en itérant $\pgcd(B,R_1)= \pgcd(R_1,R_2)$

etc... jusqu'à $\pgcd$ de $R_k$ avec le polynôme nul qui est $R_k$.

%%%%%%%%%%%%%%%%%%%%%%%%%%%%%%%%%%%%%%%%%%%%%%%%%%%%%%%%%%%
\diapo

Commençons par un exemple simple avec le calcul
du pgcd de  $X^4-1$ et $X^3-1$.

\change

On commence l'algorithme d'Euclide en divisant $A$ par $B$ :

$X^4-1  =  (X^3-1) \times X + X-1$

On itère le processus, avec la division euclidienne de ce polynôme (le diviseur) 
par celui là (le reste)

\change

Ici c'est simplement 

$X^3-1 =  (X-1)\times (X^2+X+1) + 0$

On obtient donc un reste nul.

\change

Comme pour les entiers le pgcd est le dernier reste non nul, 

donc $\pgcd(X^4-1, X^3-1)=X-1$.  



%%%%%%%%%%%%%%%%%%%%%%%%%%%%%%%%%%%%%%%%%%%%%%%%%%%%%%%%%%%
\diapo

Calculons le pgcd de  $A=X^5+X^4+2X^3+X^2+X+2$ et $B=X^4+2X^3+X^2-4$. 

\change

La division euclidienne de $A$ par $B$ s'écrit ainsi.

\change

On divise ensuite $B$ par le reste obtenu.

\change

Ce qui était le diviseur devient le dividende,
et ce qui était le reste devient le diviseur.

\change

On obtient ici un reste nul. C'est à dire que $X^2+X+2$
divise ce polynôme.

Le pgcd est le dernier reste non nul.

\change

Mais n'oublions pas que par définition le pgcd est un polynôme unitaire
(son coefficient dominant est $1$).


Donc le pgcd de $A$ et $B$ est $X^2+X+2$


%%%%%%%%%%%%%%%%%%%%%%%%%%%%%%%%%%%%%%%%%%%%%%%%%%%%%%%%%%%
\diapo

Soient $A, B$ deux polynômes et $D$ leur pgcd.


Le théorème de Bézout assure qu'il existe deux polynômes 
$U, V$ tels que $AU+BV=D$.


\change

Ce théorème découle de l'algorithme d'Euclide et plus spécialement de sa remontée
comme on le voit sur l'exemple suivant.

Nous avons calculé $\pgcd(X^4-1, X^3-1) = X-1$. 

\change

Nous remontons l'algorithme d'Euclide, ici il n'y avait qu'une ligne :
$X^4-1  =  (X^3-1) \times X + X-1$,

\change

que nous réécrivons 
$X-1 = -(X^4-1) + (X^3-1) \times X$.

\change
 
Donc $U=-1$ et $V=X$ conviennent.

%%%%%%%%%%%%%%%%%%%%%%%%%%%%%%%%%%%%%%%%%%%%%%%%%%%%%%%%%%%
\diapo

Voyons le théorème de Bézout sur un exemple plus compliqué.

Nous avions calculer par l'algorithme d'Euclide que 
pour ces deux polynômes leur pgcd était $D=X^2+X+2$

\change

Voici les étapes de l'algorithme d'Euclide :
division euclidienne de $A$ par $B$

puis division euclidienne de $B$ par le reste

et pour la dernière division euclidienne le reste est nul.



\change

On part donc du pgcd $D$ qui est le dernier reste non nul.

On récrit l'avant dernière ligne sous cette forme.

Le polynôme $3X^3+2X^2+5X-2$ qui apparaît est le reste de la ligne du dessus.

\change


Donc on remplace ce polynôme par une expression en fonction de $A$ et $B$ 

\change

et on regroupe les termes.


\change

Donc en posant $U=\frac{1}{14}(3X+4)$ et $V=-\frac{1}{14}(3X^2+X+5)$

\change

nous avons bien écrit $AU+BV=D$.

%%%%%%%%%%%%%%%%%%%%%%%%%%%%%%%%%%%%%%%%%%%%%%%%%%%%%%%%%%%
\diapo

Nous dirons que deux polynômes $A$ et $ B$ sont \defi{premiers entre eux} si  $\pgcd(A,B)=1$.

\change

Pour $A,B$ quelconques on peut se ramener à des polynômes premiers entre eux :
si  $\pgcd(A,B)=D$ alors $A$ et $B$ s'écrivent : $A=DA'$, $B=DB'$ avec $\pgcd(A',B')=1$.




%%%%%%%%%%%%%%%%%%%%%%%%%%%%%%%%%%%%%%%%%%%%%%%%%%%%%%%%%%%
\diapo

On termine par des corollaires très importants du théorème de Bézout.

Le premier corollaire s'appelle aussi le théorème de Bézout.


Deux polynômes $A$ et $B$ sont premiers entre eux si
et seulement s'il existe deux polynômes $U$ et $V$ tels que $AU+BV=1$.

\change

Deuxième conséquence :

Pour trois polynômes $A,B,C$

Si $C|A$ et $C|B$ alors $C|\pgcd(A,B)$.

\change

Enfin voici le lemme de Gauss

Pour trois polynômes $A,B,C$

Si $A$ divise le produit $B \times C$ et que $A$ et $B$ sont premiers entre eux alors $A|C$.

Attardez vous sur ces trois corollaires importants !

%%%%%%%%%%%%%%%%%%%%%%%%%%%%%%%%%%%%%%%%%%%%%%%%%%%%%%%%%%%
\diapo

A vous de faire ces exercices ; il y a pas mal de calculs, mais savoir
bien poser les divisions euclidiennes, calculer des pgcd et les coefficients de Bézout
demande rigueur et persévérance ce qui est très important.


\end{document}