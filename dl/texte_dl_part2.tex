
%%%%%%%%%%%%%%%%%% PREAMBULE %%%%%%%%%%%%%%%%%%


\documentclass[12pt]{article}

\usepackage{amsfonts,amsmath,amssymb,amsthm}
\usepackage[utf8]{inputenc}
\usepackage[T1]{fontenc}
\usepackage[francais]{babel}


% packages
\usepackage{amsfonts,amsmath,amssymb,amsthm}
\usepackage[utf8]{inputenc}
\usepackage[T1]{fontenc}
%\usepackage{lmodern}

\usepackage[francais]{babel}
\usepackage{fancybox}
\usepackage{graphicx}

\usepackage{float}

%\usepackage[usenames, x11names]{xcolor}
\usepackage{tikz}
\usepackage{datetime}

\usepackage{mathptmx}
%\usepackage{fouriernc}
%\usepackage{newcent}
\usepackage[mathcal,mathbf]{euler}

%\usepackage{palatino}
%\usepackage{newcent}


% Commande spéciale prompteur

%\usepackage{mathptmx}
%\usepackage[mathcal,mathbf]{euler}
%\usepackage{mathpple,multido}

\usepackage[a4paper]{geometry}
\geometry{top=2cm, bottom=2cm, left=1cm, right=1cm, marginparsep=1cm}

\newcommand{\change}{{\color{red}\rule{\textwidth}{1mm}\\}}

\newcounter{mydiapo}

\newcommand{\diapo}{\newpage
\hfill {\normalsize  Diapo \themydiapo \quad \texttt{[\jobname]}} \\
\stepcounter{mydiapo}}


%%%%%%% COULEURS %%%%%%%%%%

% Pour blanc sur noir :
%\pagecolor[rgb]{0.5,0.5,0.5}
% \pagecolor[rgb]{0,0,0}
% \color[rgb]{1,1,1}



%\DeclareFixedFont{\myfont}{U}{cmss}{bx}{n}{18pt}
\newcommand{\debuttexte}{
%%%%%%%%%%%%% FONTES %%%%%%%%%%%%%
\renewcommand{\baselinestretch}{1.5}
\usefont{U}{cmss}{bx}{n}
\bfseries

% Taille normale : commenter le reste !
%Taille Arnaud
%\fontsize{19}{19}\selectfont

% Taille Barbara
%\fontsize{21}{22}\selectfont

%Taille François
\fontsize{25}{30}\selectfont

%Taille Pascal
%\fontsize{25}{30}\selectfont

%Taille Laura
%\fontsize{30}{35}\selectfont


%\myfont
%\usefont{U}{cmss}{bx}{n}

%\Huge
%\addtolength{\parskip}{\baselineskip}
}


% \usepackage{hyperref}
% \hypersetup{colorlinks=true, linkcolor=blue, urlcolor=blue,
% pdftitle={Exo7 - Exercices de mathématiques}, pdfauthor={Exo7}}


%section
% \usepackage{sectsty}
% \allsectionsfont{\bf}
%\sectionfont{\color{Tomato3}\upshape\selectfont}
%\subsectionfont{\color{Tomato4}\upshape\selectfont}

%----- Ensembles : entiers, reels, complexes -----
\newcommand{\Nn}{\mathbb{N}} \newcommand{\N}{\mathbb{N}}
\newcommand{\Zz}{\mathbb{Z}} \newcommand{\Z}{\mathbb{Z}}
\newcommand{\Qq}{\mathbb{Q}} \newcommand{\Q}{\mathbb{Q}}
\newcommand{\Rr}{\mathbb{R}} \newcommand{\R}{\mathbb{R}}
\newcommand{\Cc}{\mathbb{C}} 
\newcommand{\Kk}{\mathbb{K}} \newcommand{\K}{\mathbb{K}}

%----- Modifications de symboles -----
\renewcommand{\epsilon}{\varepsilon}
\renewcommand{\Re}{\mathop{\text{Re}}\nolimits}
\renewcommand{\Im}{\mathop{\text{Im}}\nolimits}
%\newcommand{\llbracket}{\left[\kern-0.15em\left[}
%\newcommand{\rrbracket}{\right]\kern-0.15em\right]}

\renewcommand{\ge}{\geqslant}
\renewcommand{\geq}{\geqslant}
\renewcommand{\le}{\leqslant}
\renewcommand{\leq}{\leqslant}

%----- Fonctions usuelles -----
\newcommand{\ch}{\mathop{\mathrm{ch}}\nolimits}
\newcommand{\sh}{\mathop{\mathrm{sh}}\nolimits}
\renewcommand{\tanh}{\mathop{\mathrm{th}}\nolimits}
\newcommand{\cotan}{\mathop{\mathrm{cotan}}\nolimits}
\newcommand{\Arcsin}{\mathop{\mathrm{Arcsin}}\nolimits}
\newcommand{\Arccos}{\mathop{\mathrm{Arccos}}\nolimits}
\newcommand{\Arctan}{\mathop{\mathrm{Arctan}}\nolimits}
\newcommand{\Argsh}{\mathop{\mathrm{Argsh}}\nolimits}
\newcommand{\Argch}{\mathop{\mathrm{Argch}}\nolimits}
\newcommand{\Argth}{\mathop{\mathrm{Argth}}\nolimits}
\newcommand{\pgcd}{\mathop{\mathrm{pgcd}}\nolimits} 

\newcommand{\Card}{\mathop{\text{Card}}\nolimits}
\newcommand{\Ker}{\mathop{\text{Ker}}\nolimits}
\newcommand{\id}{\mathop{\text{id}}\nolimits}
\newcommand{\ii}{\mathrm{i}}
\newcommand{\dd}{\mathrm{d}}
\newcommand{\Vect}{\mathop{\text{Vect}}\nolimits}
\newcommand{\Mat}{\mathop{\mathrm{Mat}}\nolimits}
\newcommand{\rg}{\mathop{\text{rg}}\nolimits}
\newcommand{\tr}{\mathop{\text{tr}}\nolimits}
\newcommand{\ppcm}{\mathop{\text{ppcm}}\nolimits}

%----- Structure des exercices ------

\newtheoremstyle{styleexo}% name
{2ex}% Space above
{3ex}% Space below
{}% Body font
{}% Indent amount 1
{\bfseries} % Theorem head font
{}% Punctuation after theorem head
{\newline}% Space after theorem head 2
{}% Theorem head spec (can be left empty, meaning ‘normal’)

%\theoremstyle{styleexo}
\newtheorem{exo}{Exercice}
\newtheorem{ind}{Indications}
\newtheorem{cor}{Correction}


\newcommand{\exercice}[1]{} \newcommand{\finexercice}{}
%\newcommand{\exercice}[1]{{\tiny\texttt{#1}}\vspace{-2ex}} % pour afficher le numero absolu, l'auteur...
\newcommand{\enonce}{\begin{exo}} \newcommand{\finenonce}{\end{exo}}
\newcommand{\indication}{\begin{ind}} \newcommand{\finindication}{\end{ind}}
\newcommand{\correction}{\begin{cor}} \newcommand{\fincorrection}{\end{cor}}

\newcommand{\noindication}{\stepcounter{ind}}
\newcommand{\nocorrection}{\stepcounter{cor}}

\newcommand{\fiche}[1]{} \newcommand{\finfiche}{}
\newcommand{\titre}[1]{\centerline{\large \bf #1}}
\newcommand{\addcommand}[1]{}
\newcommand{\video}[1]{}

% Marge
\newcommand{\mymargin}[1]{\marginpar{{\small #1}}}



%----- Presentation ------
\setlength{\parindent}{0cm}

%\newcommand{\ExoSept}{\href{http://exo7.emath.fr}{\textbf{\textsf{Exo7}}}}

\definecolor{myred}{rgb}{0.93,0.26,0}
\definecolor{myorange}{rgb}{0.97,0.58,0}
\definecolor{myyellow}{rgb}{1,0.86,0}

\newcommand{\LogoExoSept}[1]{  % input : echelle
{\usefont{U}{cmss}{bx}{n}
\begin{tikzpicture}[scale=0.1*#1,transform shape]
  \fill[color=myorange] (0,0)--(4,0)--(4,-4)--(0,-4)--cycle;
  \fill[color=myred] (0,0)--(0,3)--(-3,3)--(-3,0)--cycle;
  \fill[color=myyellow] (4,0)--(7,4)--(3,7)--(0,3)--cycle;
  \node[scale=5] at (3.5,3.5) {Exo7};
\end{tikzpicture}}
}



\theoremstyle{definition}
%\newtheorem{proposition}{Proposition}
%\newtheorem{exemple}{Exemple}
%\newtheorem{theoreme}{Théorème}
\newtheorem{lemme}{Lemme}
\newtheorem{corollaire}{Corollaire}
%\newtheorem*{remarque*}{Remarque}
%\newtheorem*{miniexercice}{Mini-exercices}
%\newtheorem{definition}{Définition}




%definition d'un terme
\newcommand{\defi}[1]{{\color{myorange}\textbf{\emph{#1}}}}
\newcommand{\evidence}[1]{{\color{blue}\textbf{\emph{#1}}}}



 %----- Commandes divers ------

\newcommand{\codeinline}[1]{\texttt{#1}}

%%%%%%%%%%%%%%%%%%%%%%%%%%%%%%%%%%%%%%%%%%%%%%%%%%%%%%%%%%%%%
%%%%%%%%%%%%%%%%%%%%%%%%%%%%%%%%%%%%%%%%%%%%%%%%%%%%%%%%%%%%%



\begin{document}

\debuttexte

%%%%%%%%%%%%%%%%%%%%%%%%%%%%%%%%%%%%%%%%%%%%%%%%%%%%%%%%%%%
\diapo

\change

Après avoir vu les formules de Taylor nous allons passer aux développements limités proprement
dit. 

\change
On commence par la définition et l'existence des développements limités

\change

On monte ensuite l'unicité

\change 

Puis on attaque le coeur de ce chapitre : les formules des développements limités
des fonctions usuelles à l'origine $\exp$, $\cos$ $sin$ $\ln$ et puissance.

\change

Une fois que l'on connaît ces formules on explique comment trouver
les DL des fonctions en un point quelconque.



%%%%%%%%%%%%%%%%%%%%%%%%%%%%%%%%%%%%%%%%%%%%%%%%%%%%%%%%%%%
\diapo

Soit $I$ un intervalle ouvert et $f : I \to \Rr$ une fonction quelconque.

Pour un réel $a$ de l'intervalle $I$ et un entier $n$, 

on dit que $f$ admet un \defi{développement limité} en $a$  à l'ordre $n$, 

s'il existe des réels $c_0, c_1,\ldots,c_n$

et une fonction $\epsilon$ telle que $\epsilon(x)\to0$ quand $x\to a$


de sorte que pour tout $x\in I$ 
on ait 
$$f(x)=c_0+c_1 (x-a)+\cdots+c_n(x-a)^n+(x-a)^n\epsilon(x).$$

Attardons nous un peu sur cette définition :

tout d'abord on abrège ``développement limité'' en DL.

Et l'égalité précédente s'appelle un DL de $f$ en $a$ à l'ordre $n$

\change

 Le terme $c_0+c_1(x-a)+\cdots+c_n(x-a)^n$ est appelé la \defi{partie polynomiale} du DL.

\change

Le terme $(x-a)^n\epsilon(x)$ est appelé le \defi{reste} du DL.

Bien sûr vous aurez compris, la terminologie est la même que pour les formules de Taylor.

Il y a une bonne raison pour cela : la première façon d'obtenir des DL est d'utiliser les formules de Taylor.

\change

En effet la formule de Taylor permet de prouver l'existence d'un développement limité
pour une fonction suffisamment régulière : 

Si $f$ est de classe $\mathcal{C}^n$ 
alors $f$ admet un DL en $a$ à l'ordre $n$
$$f(x)= f(a)+\frac{f'(a)}{1!}(x-a)+ \frac{f''(a)}{2!}(x-a)^2+\cdots
+\frac{f^{(n)}(a)}{n!}(x-a)^n+(x-a)^n\epsilon(x)$$

où $\lim_{x\to a}\epsilon(x)=0$

\change

Vous aurez reconnu la formule de Taylor, plus précisément la formule de Taylor-Young.

\change

En posant $c_k = \frac{f^{(k)}(a)}{k!}$ on obtient le développement limité

\change

La situation la plus courante est lorsque $a=0$ et
la formule de Taylor-Young à l'origine fournit alors le DL suivant :

$$f(x)= f(0)+f'(0)x+f''(0)\frac{x^2}{2!}+\cdots
+f^{(n)}(0)\frac{x^n}{n!} + x^n\epsilon(x)$$

%%%%%%%%%%%%%%%%%%%%%%%%%%%%%%%%%%%%%%%%%%%%%%%%%%%%%%%%%%%
\diapo

Après l'existence voici l'unicité 

Proposition :

Si $f$ admet un développement limité alors ce développement limité est unique

\change

La preuve est simple

On suppose que l'on a un premier développement limité de $f$,




\change

et un second ...



\change

La différence donne l'égalité 
$
(d_0-c_0)+(d_1-c_1)(x-a)+\cdots+(d_n-c_n)(x-a)^n+(x-a)^n(\epsilon_2(x)-\epsilon_1(x))=0
$

\change

Lorsque l'on fait $x=a$ dans cette égalité 

\change

alors on trouve $d_0-c_0=0$.

\change

Ensuite on peut diviser cette égalité par $x-a$  

\change

Pour obtenir une nouvelle égalité 
$(d_1-c_1)+(d_2-c_2)(x-a)+\cdots+(d_n-c_n)(x-a)^{n-1}+(x-a)^{n-1}(\epsilon_2(x)-\epsilon_1(x))=0$.

\change 

En évaluant en $x=a$ on obtient $d_1-c_1=0$ et on continue ainsi de suite

\change

On trouve donc $c_0=d_0$, $c_1=d_1$, \ldots
Les coefficients sont égaux.

\change

Et comme les parties polynomiales sont égales alors par l'égalité initiale les restes aussi sont égaux.


%%%%%%%%%%%%%%%%%%%%%%%%%%%%%%%%%%%%%%%%%%%%%%%%%%%%%%%%%%%
\diapo

L'unicité permet de dire <<le développement limité de $f$>>
à la place de <<un développement limité>>

Mais l'unicité a d'autres conséquence, par exemple 

Si $f$ est une fonction paire alors la partie polynomiale de son DL en $0$
ne contient que des monômes de degrés pairs   


Si $f$ est une fonction impaire alors il n'y a que des monômes de degré impairs.

\change

Par exemple la fonction $\cos x$ est paire et son son DL en $0$ commence par 
$\cos x=1-\frac{x^2}{2!}+\frac{x^4}{4!}-\frac{x^6}{6!}+\cdots$

\change

La preuve est aisée 

Prenons une fonction $f$ paire.

Ecrivons le DL de $f(x)$

\change

Alors  $f(-x)$ s'écrit de la façon suivante,
notez les signes $-$ pour les coefficients impairs

\change

Comme $f$ est paire alors $f(x)=f(-x)$ 

\change


Par l'unicité du DL alors les coefficients sont égaux

donc $c_1=-c_1$, $c_3=-c_3$, \ldots \ 

\change

et ainsi $c_1=0$, $c_3=0$,\ldots

\change

Voici quelques remarques 

L'unicité du DL et la formule de Taylor-Young 
permettent de calculer les coefficients du DL
$c_k = \frac{f^{(k)}(a)}{k!}$

Mais cela peut fonctionner dans l'autre sens 
si l'on connaît le DL et que $f$ est de classe $\mathcal{C}^n$
alors on peut calculer les nombres dérivés à partir de la partie polynomiale.

\change

En particulier $c_0=f(a)$ et $c_1=f'(a)$

\change

Par conséquent $y=c_0+c_1(x-a)$ est l'équation de la tangente au graphe de $f$
au point d'abscisse $a$.


%%%%%%%%%%%%%%%%%%%%%%%%%%%%%%%%%%%%%%%%%%%%%%%%%%%%%%%%%%%
\diapo

Les DL suivants en $0$ proviennent de la formule de Taylor-Young.

Nous commençons par le dl de l'exponentiel 
que l'on a déjà rencontré dans la partie précédente.

$\exp x=1+\frac{x}{1!}+\frac{x^2}{2!}+\frac{x^3}{3!}+\cdots+\frac{x^n}{n!}
+x^n\epsilon(x)$

\change

Il est aussi très utile de savoir l'écrire à l'aide des $\Sigma$.

$\exp x=\sum_ {k=0}^n \frac{x^k}{k!} \ \ + o(x^n)$

le reste est ici noté avec la notation <<petit o>>

\change

Deuxième formule fondamentale celle du logarithme

$\ln(1+x)=x-\frac{x^2}{2}+\frac{x^3}{3}-\cdots
+(-1)^{n-1}\frac{x^{n}}{n} +x^{n}\epsilon(x)$

\change

Qui s'écrit aussi 
$\ln(1+x)=\sum_ {k=1}^n (-1)^{k-1}\frac{x^k}{k}\ \  + o(x^n)$

Remarquez bien qu'il n'y a pas de terme constant, pas de factorielle aux dénominateurs, 
et que les signes alternent  $-$ $+$ $-$ $+$...

\change

A partir du DL de l'exponentielle on déduit $4$ autres formules qui sont faciles à mémoriser

Tout d'abord le DL de $\ch$

$\ch x=1+\frac{x^2}{2!}+\frac{x^4}{4!}+\cdots+\frac{x^{2n}}{(2n)!}
+x^{2n+1}\epsilon(x)$ 

C'est la partie paire du DL de $\exp x$. 
C'est-à-dire que l'on ne retient que les
monômes de degré pair. 

\change

Et voici le DL de $\sh x$ qui est la partie impaire.

$\sh x=\frac{x}{1!}+\frac{x^3}{3!}+\frac{x^5}{5!}+\cdots
+\frac{x^{2n+1}}{(2n+1)!} 
+x^{2n+2}\epsilon(x)$

\change

Enfin $\cos$ qui est la partie paire du DL de $\exp x$ mais en alternant cette fois le signe $+/-$ du monôme

$\cos x=1-\frac{x^2}{2!}+\frac{x^4}{4!}-\cdots+(-1)^n\frac{x^{2n}}{(2n)!}
+x^{2n+1}\epsilon(x)$ 

\change

Et enfin $\sin$ est la partie impaire de $\exp x$ en alternant aussi les signes

$\sin x=\frac{x}{1!}-\frac{x^3}{3!}+\frac{x^5}{5!}-\cdots
+(-1)^n\frac{x^{2n+1}}{(2n+1)!} 
+x^{2n+2}\epsilon(x)$



Arrêtez-vous de longues minutes pour noter et apprendre par coeur toutes ces formules.


%%%%%%%%%%%%%%%%%%%%%%%%%%%%%%%%%%%%%%%%%%%%%%%%%%%%%%%%%%%
\diapo


On continue avec autre formule importante celle pour les puissances.

Pour $\alpha$ un réel le DL de la fonction $(1+x)^{\alpha}$ en $0$ est

$(1+x)^{\alpha}=1+\alpha x+\frac{\alpha(\alpha-1)}{2!}x^2+\cdots
+\frac{\alpha(\alpha-1)...(\alpha-n+1)}{n!}x^n+x^n\epsilon(x)$

\change

En prenant $\alpha=-1$ on obtient le $DL$ de $\frac{1}{1+x}$
${\displaystyle \frac{1}{1+x}}=1-x+x^2-x^3+\cdots+(-1)^nx^n+x^n\epsilon(x)$

\change

Mais il est plus facile de retenir le DL de $\frac{1}{1-x}$

${\displaystyle \frac{1}{1-x}} = 1+x+x^2+\cdots+x^n+x^n\epsilon(x)$


Il se retrouve aussi avec la formule de la somme d'une suite géométrique

\change

Enfin pour $\alpha = \frac12$ on a $(1+x)^{\frac12} = \sqrt{1+x}$

On trouve 
$\sqrt{1+x}  =\ \ 1 + \frac{x}{2} - \frac{1}{8}x^2+ \cdots +$

Dont il faut connaître les trois premiers termes.


%%%%%%%%%%%%%%%%%%%%%%%%%%%%%%%%%%%%%%%%%%%%%%%%%%%%%%%%%%%
\diapo

On souhaite maintenant calculer le DL d'une fonction en un point quelconque.

\change

Nous allons utiliser que 
$f(x)$ admet un DL en $a$ si et seulement si la 
fonction $f(x+a)$ admet un DL en $0$

Souvent on ramène donc le problème en $0$ en faisant le changement de variables $h=x-a$. 

\change

Voyons des exemples.

On vient de voir la formule du DL de $\exp x$ en $x=0$ mais ici on 
souhaite trouver le DL de $\exp x$ en $x=1$.

\change

On pose $h=x-1$. Si $x$ est proche de $1$ alors $h$ est proche de $0$.

Nous allons donc nous ramener à un DL de $\exp h$ en $h=0$. 

\change 

On écrit $\exp x$ 

\change 

sous la forme 
$= \exp( 1+ (x-1) )$

\change

et comme $e^{a+b}=e^a e^b$ c'est égale à 
$ = \exp(1) \exp (x-1)$

\change

Ou encore $e \exp h$

ou $e=\exp(1)$ et $h=x-1$.


\change

Maintenant on utilise le DL de $\exp h$ en $0$ :
pour obtenir :

$e \left(1+h+ \frac{h^2}{2!} + \cdots + \frac{h^n}{n!}+h^n\epsilon(h)\right)$

\change

On revient aux $x$ en remplaçant cette fois çi $h$ par $x-1$.

On obtient bien le DL de $\exp x$ en $1$

\change

car 
$\epsilon(x-1) \to 0 $ quand $x\to1$.


(pause)

\change

C'est parti pour le même travail avec la fonction $\ln(1+3x)$ en $1$ à l'ordre $3$. 

\change

On pose $h=x-1$.

Il faut se ramener à un DL du type $\ln(1+h)$ en $h=0$.

\change

Comme $h=x-1$ alors $x=h+1$.

\change

Ainsi $\ln(1+3x) = \ln\big(1+3(1+h)\big)$

\change 

c-a-d 
$\ln(4 + 3h)$

\change

factorisons par $4$ 

\change

pour obtenir

$\ln 4 + \ln\big(1+\frac{3h}{4}\big)$

grâce à la formule $\ln(a\times b)=\ln a + \ln b$.


On sait que $x$ est proche de $1$ donc $h$ est proche de $0$
donc $\frac{3h}{4}$ est aussi proche de $0$.
On peut donc utiliser la formule du DL de $\ln (1+u)$ avec $u=\frac{3h}{4}$.

On obtient ceci

\change 

et se souvenant que $h=x-1$ 

\change

on obtient le DL de $\ln(1+x)$ en $1$.

\change



%%%%%%%%%%%%%%%%%%%%%%%%%%%%%%%%%%%%%%%%%%%%%%%%%%%%%%%%%%%
\diapo


Apprenez les formules des développements limités vous en aurez besoin

pour ces questions

mais aussi continuellement dans la suite du cours.
 
\end{document}