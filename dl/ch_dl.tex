\documentclass[class=report,crop=false]{standalone}
\usepackage[screen]{../exo7book}

\begin{document}

%====================================================================
\chapitre{Développements limités}
%====================================================================

\insertvideo{vlFWMeBUTXo}{partie 1. Formules de Taylor}

\insertvideo{gFpLfhXjLSY}{partie 2. Développements limités au voisinage d'un point}

\insertvideo{_AS7bwOsxd4}{partie 3. Opérations sur les DL}

\insertvideo{bS8MxViqUUE}{partie 4. Applications}

\insertfiche{fic00163.pdf}{Développements limités}

%%%%%%%%%%%%%%%%%%%%%%%%%%%%%%%%%%%%%%%%%%%%%%%%%%%%%%%%%%%%%%%%
\section*{Motivation}

Prenons l'exemple de la fonction exponentielle.
Une idée du comportement de la fonction $f(x)=\exp x$ autour du point $x=0$ est donné par sa tangente,
dont l'équation est $y=1+x$. Nous avons approximé le graphe par une droite.
Si l'on souhaite faire mieux, quelle parabole d'équation $y = c_0 + c_1x + c_2 x^2$ approche le mieux
le graphe de $f$ autour de $x=0$ ? Il s'agit de la parabole d'équation $y=1+x+\frac12 x^2$.
Cette équation à la propriété remarquable que si on note $g(x)=\exp x - \big(1+x+\frac12 x^2\big)$ alors
$g(0)=0$, $g'(0)=0$ et $g''(0)=0$. Trouver l'équation de cette parabole c'est faire un développement limité à l'ordre $2$
de la fonction $f$.
Bien sûr si l'on veut être plus précis, on continuerait avec une courbe du troisième degré
qui serait en fait $y = 1+x+\frac12 x^2 + \frac16 x^3$.

\myfigure{1}{
\tikzinput{fig_dl01}
}

\bigskip

Dans ce chapitre, pour n'importe quelle fonction, nous allons trouver le polynôme de degré $n$
qui approche le mieux la fonction. Les résultats ne sont valables que pour $x$ autour d'une valeur
fixée (ce sera souvent autour de $0$). Ce polynôme sera calculé à partir des dérivées successives au point
considéré. Sans plus attendre, voici la formule, dite formule de Taylor-Young :
$$f(x)= f(0)+f'(0)x+f''(0)\frac{x^2}{2!}+\cdots
+f^{(n)}(0)\frac{x^n}{n!} + x^n\epsilon(x).$$
La partie polynomiale $f(0)+f'(0)x+\cdots+f^{(n)}(0)\frac{x^n}{n!}$
est le polynôme de degré $n$ qui approche le mieux $f(x)$ autour de $x=0$.
La partie $x^n\epsilon(x)$ est le \og reste \fg{}  dans lequel $\epsilon(x)$ est une fonction
qui tend vers $0$ (quand $x$ tend vers $0$) et qui est négligeable devant la partie polynomiale.




%%%%%%%%%%%%%%%%%%%%%%%%%%%%%%%%%%%%%%%%%%%%%%%%%%%%%%%%%%%%%%%%
\section{Formules de Taylor}


Nous allons voir trois formules de Taylor, elles auront toutes la même partie polynomiale
mais donnent plus ou moins d'informations sur le reste. Nous commencerons par la formule de
Taylor avec reste intégral qui donne une expression exacte du reste. Puis la formule de Taylor avec
reste $f^{(n+1)}(c)$ qui permet d'obtenir un encadrement du reste et nous terminons avec la formule de Taylor-Young
très pratique si l'on n'a pas besoin d'information sur le reste.

\bigskip

Soit $I \subset \Rr$ un intervalle ouvert. Pour $n \in \Nn^*$, on dit
que $f : I \to \Rr$ est une fonction de \defi{classe $\mathcal{C}^n$}\index{fonction!de classe Cn@de classe $\mathcal{C}^n$}
si $f$ est $n$ fois dérivable sur $I$ et $f^{(n)}$ est continue.
$f$ est de  \defi{classe $\mathcal{C}^0$} si $f$ est continue sur $I$.
$f$ est de  \defi{classe $\mathcal{C}^\infty$}\index{fonction!de classe Cy@de classe $\mathcal{C}^\infty$} si $f$ est de classe $\mathcal{C}^n$
pour tout $n\in\Nn$.





%---------------------------------------------------------------
\subsection{Formule de Taylor avec reste intégral}

\begin{theoreme}[Formule de Taylor avec reste intégral]
\index{formule!de Taylor avec reste integral@de Taylor avec reste intégral}
Soit $f : I\to\Rr$ une fonction de classe $\mathcal{C}^{n+1}$ ($n \in \Nn$)
et soit $a,x \in I$.
Alors
\mybox{
$\begin{array}{lr}
f(x)=f(a)+f'(a)(x-a)+\frac{f''(a)}{2!}(x-a)^2+\cdots & \\[2ex]
&\hspace*{-4cm} \cdots+\frac{f^{(n)}(a)}{n!}(x-a)^n+\int_a^x \frac{f^{(n+1)}(t)}{n!}(x-t)^ndt.  
\end{array}$
}
\end{theoreme}


Nous noterons $T_n(x)$ la partie polynomiale de la
formule de Taylor (elle dépend de $n$ mais aussi de $f$ et~$a$)  :
$$T_n(x) =f(a)+f'(a)(x-a)+\frac{f''(a)}{2!}(x-a)^2+\cdots
+\frac{f^{(n)}(a)}{n!}(x-a)^n.$$

\begin{remarque*}
En écrivant $x=a+h$ (et donc $h=x-a$)
la formule de Taylor précédente devient (pour tout $a$ et $a+h$ de $I$) :
$$f(a+h)=f(a)+f'(a)h+\frac{f''(a)}{2!}h^2+\cdots
+\frac{f^{(n)}(a)}{n!}h^n+\int_0^{h} \frac{f^{(n+1)}(a+t)}{n!}(h-t)^ndt$$
\end{remarque*}

\begin{exemple}
La fonction $f(x)=\exp x$ est de classe $\mathcal{C}^{n+1}$ sur $I=\Rr$ pour tout $n$. Fixons $a\in\Rr$.
Comme $f'(x)=\exp x$, $f''(x)=\exp x$,\ldots alors pour tout $x\in\Rr$ :
$$\exp x=\exp a+\exp a \cdot (x-a)+\cdots+\frac{\exp a}{n!}(x-a)^n+\int_a^x\frac{\exp t}{n!}(x-t)^ndt.$$
Bien sûr si l'on se place en $a=0$ alors on retrouve le début de notre approximation de la fonction
exponentielle en $x=0$ : $\exp x=1+x+\frac{x^2}{2!}+\frac{x^3}{3!}+\cdots$
\end{exemple}



\begin{proof}[Preuve du théorème]
Montrons cette formule de Taylor par récurrence sur $k \le n$:
{\small
\[
f(b)=f(a)+f'(a)(b-a)+\frac{f''(a)}{2!}(b-a)^2+\cdots
+\frac{f^{(k)}(a)}{k!}(b-a)^k+\int_a^b f^{(k+1)}(t)\frac{(b-t)^k}{k!}dt.
\]}
(Pour éviter les confusions entre ce qui varie et ce qui est fixe dans cette
preuve on remplace $x$ par $b$.)
\\

\textbf{Initialisation.}
Pour $n=0$, une primitive de $f'(t)$ est $f(t)$ donc
$\int_a^b f'(t) \, dt=f(b)-f(a)$, donc
$f(b)=f(a)+\int_a^b f'(t) \, dt$.
(On rappelle que par convention $(b-t)^0=1$ et $0!=1$.)

\textbf{Hérédité.}
Supposons la formule vraie au rang $k-1$. Elle s'écrit
$f(b)=f(a)+f'(a)(b-a)+\cdots
+\frac{f^{(k-1)}(a)}{(k-1)!}(b-a)^{k-1}
+\int_a^bf^{(k)}(t)\frac{(b-t)^{k-1}}{(k-1)!}dt$.

On effectue une intégration par parties dans l'intégrale
$\int_a^b f^{(k)}(t) \frac{(b-t)^{k-1}}{(k-1)!} \, dt$. En posant
$u(t)=f^{(k)}(t)$ et $v'(t) = \frac{(b-t)^{k-1}}{(k-1)!}$,
on a $u'(t)= f^{(k+1)}(t)$ et $v(t) = - \frac{(b-t)^{k}}{k!}$; alors
\begin{align*}
\int_a^b f^{(k)}(t) \frac{(b-t)^{k-1}}{(k-1)!} \, dt
  & = \left[-f^{(k)}(t)\frac{(b-t)^{k}}{k!}\right]_a^b + \int_a^b f^{(k+1)}(t) \frac{(b-t)^{k}}{k!} \, dt\\
  & = f^{(k)}(a)\frac{(b-a)^{k}}{k!} + \int_a^b f^{(k+1)}(t) \frac{(b-t)^{k}}{k!}\, dt . \\
\end{align*}

Ainsi lorsque l'on remplace cette expression dans la formule au rang $k-1$ on obtient la formule au rang~$k$.

\textbf{Conclusion.} Par le principe de récurrence la formule de Taylor est vraie
pour tous les entiers $n$ pour lesquels $f$ est classe $\mathcal{C}^{n+1}$.
\end{proof}


%---------------------------------------------------------------

\subsection{Formule de Taylor avec reste $f^{(n+1)}(c)$}

\begin{theoreme}[Formule de Taylor avec reste $f^{(n+1)}(c)$]
\index{formule!de Taylor avec reste f@de Taylor avec reste $f^{(n+1)}(c)$}
Soit $f : I\to\Rr$ une fonction de classe $\mathcal{C}^{n+1}$ ($n \in \Nn$)
et soit $a,x \in I$.
Il existe un réel $c$ entre $a$ et $x$ tel que :
\mybox{
$\begin{array}{lr}
f(x)=f(a)+f'(a)(x-a)+\frac{f''(a)}{2!}(x-a)^2+\cdots & \\[2ex]
& \hspace*{-4cm} \cdots+\frac{f^{(n)}(a)}{n!}(x-a)^n +\frac{f^{(n+1)}(c)}{(n+1)!}
(x-a)^{n+1}. \end{array}$
}

\end{theoreme}




\begin{exemple}
Soient $a,x \in\Rr$. Pour tout entier $n\ge0$ il existe $c$ entre $a$ et $x$ tel que
$\exp x=\exp a+\exp a \cdot (x-a)+\cdots+\frac{\exp a}{n!}(x-a)^n+\frac{\exp c}{(n+1)!}(x-a)^{n+1}.$
\end{exemple}

Dans la plupart des cas on ne connaîtra pas ce $c$. Mais ce théorème permet d'encadrer le reste.
Ceci s'exprime par le corollaire suivant :
\begin{corollaire}
Si en plus la fonction $|f^{(n+1)}|$ est majorée sur $I$ par un réel $M$, alors pour tout $a, x\in
I$, on~a :
\[
\big|f(x)-T_n(x)\big|\le M\frac{|x-a|^{n+1}}{(n+1)! \ } \cdotp
\]
\end{corollaire}

\begin{exemple}
Approximation de $\sin(0,01)$.

Soit $f(x)=\sin x$. Alors $f'(x)=\cos x$, $f''(x)=-\sin x$, $f^{(3)}(x)=-\cos x$, $f^{(4)}(x)=\sin x$.
On obtient donc $f(0)=0$, $f'(0)=1$, $f''(0)=0$, $f^{(3)}(0)=-1$.
La formule de Taylor ci-dessus en $a=0$ à l'ordre $3$ devient :
$f(x)=0+1\cdot x +0\cdot \frac{x^2}{2!}-1\frac{x^3}{3!} + f^{(4)}(c)\frac{x^4}{4!}$,
c'est-à-dire $f(x)= x -\frac{x^3}{6} + f^{(4)}(c)\frac{x^4}{24}$,
pour un certain $c$ entre $0$ et $x$.

Appliquons ceci pour $x=0,01$. Le reste étant petit on trouve
alors
\[
\sin(0,01) \approx 0,01 - \frac{(0,01)^3}{6}=0,00999983333\ldots
\]

On peut même savoir quelle est la précision de cette approximation :
comme $f^{(4)}(x)=\sin x$ alors  $|f^{(4)}(c)|\le 1$.
Donc
$\big|f(x) - \big(x -\frac{x^3}{6} \big) \big| \le \frac{x^4}{4!}$.
Pour $x=0,01$ cela donne :
$\big|\sin(0,01) -  \big(0,01 - \frac{(0,01)^3}{6}\big)\big| \le \frac{(0,01)^4}{24}$.
Comme $\frac{(0,01)^4}{24} \approx 4,16 \cdot 10^{-10}$ alors
notre approximation donne au moins $8$ chiffres exacts après la virgule.
\end{exemple}



\begin{remarque*}
\sauteligne
\begin{itemize}
  \item Dans ce théorème l'hypothèse $f$ de classe $\mathcal{C}^{n+1}$ peut-être affaiblie
en $f$ est \og $n+1$ fois dérivable sur $I$ \fg{}.

  \item \og le réel $c$ est entre $a$ et $x$ \fg{} signifie \og $c\in ]a,x[$ ou $c\in ]x,a[$ \fg{}.

  \item Pour $n=0$ c'est exactement l'énoncé du théorème des accroissements finis :
il existe $c\in]a,b[$ tel que $f(b)=f(a)+f'(c)(b-a)$.

  \item Si $I$ est un intervalle fermé borné et $f$ de classe $\mathcal{C}^{n+1}$,
alors $f^{(n+1)}$ est continue sur $I$ donc il existe un $M$ tel que
$|f^{(n+1)}(x)|\le M$ pour tout $x\in I$. Ce qui permet toujours d'appliquer le corollaire.
\end{itemize}
\end{remarque*}

Pour la preuve du théorème nous aurons besoin d'un résultat préliminaire.
\begin{lemme}[\'Egalité de la moyenne]
\index{egalite de la moyenne@égalité de la moyenne}
Supposons $a<b$ et soient $u,v : [a,b] \to \Rr$ deux fonctions continues avec $v$ positive ou nulle.
Alors il existe $c\in[a,b]$ tel que $\int_a^b u(t)v(t)\, dt = u(c) \int_a^b v(t)\, dt$.
\end{lemme}

\begin{proof}
Notons $m = \inf_{t\in[a,b]} u(t)$ et $M=\sup_{t\in[a,b]} u(t)$.
On a alors $m\int_a^b v(t)\, dt \le \int_a^b u(t)v(t)\, dt\le M\int_a^b v(t)\, dt$ (car $v\ge0$).
Ainsi $m \le \frac{\int_a^b u(t)v(t)\, dt}{\int_a^b v(t)\, dt }\le M$.
Puisque $u$ est continue sur $[a,b]$ elle prend toutes les valeurs comprises entre $m$ et $M$
(théorème des valeurs intermédiaires). Donc il existe $c \in[a,b]$ avec
$u(c)=\frac{\int_a^b u(t)v(t)\, dt}{\int_a^b v(t)\, dt}$.
\end{proof}


\begin{proof}[Preuve du théorème]
Pour la preuve nous montrerons la formule de Taylor pour $f(b)$ en supposant $a<b$.
Nous montrerons seulement $c\in [a,b]$ au lieu de $c\in ]a,b[$.

Posons $u(t)=f^{(n+1)}(t)$ et $v(t)= \frac{(b-t)^n}{n!}$ (qui est bien positive ou nulle).
La formule de Taylor avec reste intégral s'écrit
$f(b) = T_n(a)+ \int_a^b u(t)v(t) \, dt$.
Par le lemme, il existe $c\in[a,b]$ tel que
$\int_a^b u(t)v(t) \, dt = u(c) \int_a^b v(t)\, dt$.
Ainsi le reste est
$\int_a^b u(t)v(t) \, dt = f^{(n+1)}(c) \int_a^b \frac{(b-t)^n}{n!} \, dt
=f^{(n+1)}(c) \left[ -\frac{(b-t)^{n+1}}{(n+1)!} \right]_a^b
= f^{(n+1)}(c)  \frac{(b-a)^{n+1}}{(n+1)!}$.
Ce qui donne la formule recherchée.
\end{proof}




%---------------------------------------------------------------
\subsection{Formule de Taylor-Young}

\begin{theoreme}[Formule de Taylor-Young]
\index{formule!de Taylor-Young}
Soit $f: I \to \Rr$ une fonction de classe $\mathcal{C}^n$ et soit $a\in I$.
Alors pour tout $x \in I$ on a :
\mybox{
$\begin{array}{lr}
f(x)= f(a)+f'(a)(x-a)+\frac{f''(a)}{2!}(x-a)^2+\cdots &\\[2ex]
& \hspace*{-4cm}\cdots+\frac{f^{(n)}(a)}{n!}(x-a)^n +(x-a)^n\epsilon(x),   
 \end{array}
$}
où $\epsilon$ est une fonction définie sur $I$ telle que $\epsilon(x) \xrightarrow[x\to a]{} 0$.
\end{theoreme}

\begin{proof}
$f$ étant une fonction de classe $\mathcal{C}^n$ nous appliquons la formule de Taylor avec reste $f^{(n)}(c)$
au rang $n-1$. Pour tout $x$, il existe $c=c(x)$ compris entre $a$ et $x$ tel que
$f(x)=f(a)+f'(a)(x-a)+\frac{f''(a)}{2!}(x-a)^2+\cdots
+\frac{f^{(n-1)}(a)}{(n-1)!}(x-a)^{n-1} +\frac{f^{(n)}(c)}{n!}(x-a)^{n}.$
Que nous réécrivons :
$f(x)=f(a)+f'(a)(x-a)+\frac{f''(a)}{2!}(x-a)^2+\cdots
+\frac{f^{(n)}(a)}{n!}(x-a)^n + \frac{f^{(n)}(c)-f^{(n)}(a)}{n!}(x-a)^{n}.$
On pose $\epsilon(x)=\frac{f^{(n)}(c)-f^{(n)}(a)}{n!}$.
Puisque $f^{(n)}$ est continue et que $c(x) \to a$ alors $\lim_{x\to a} \epsilon(x) = 0$.
\end{proof}


%---------------------------------------------------------------
\subsection{Un exemple}

Soit $f : ]-1,+\infty[ \to \Rr$, $x \mapsto \ln(1+x)$; $f$ est infiniment dérivable.
Nous allons calculer les formules de Taylor en $0$ pour les premiers ordres.

Tous d'abord $f(0)=0$. Ensuite $f'(x)=\frac{1}{1+x}$ donc $f'(0)=1$.
Ensuite $f''(x) = -\frac{1}{(1+x)^2}$ donc $f''(0)=-1$.
Puis $f^{(3)}(x)= + 2 \frac{1}{(1+x)^3}$ donc $f^{(3)}(0)= + 2$.
Par récurrence on montre que $f^{(n)}(x) = (-1)^{n-1} (n-1)!\frac{1}{(1+x)^n}$
et donc $f^{(n)}(0)= (-1)^{n-1} (n-1)!$.
Ainsi pour $n>0$ : $\frac{f^{(n)}(0)}{n!}x^n =  (-1)^{n-1} \frac{(n-1)!}{n!} x^n= (-1)^{n-1}\frac{x^n}{n}$.

Voici donc les premiers polynômes de Taylor :
$$T_0(x)=0 \qquad T_1(x) = x \qquad T_2(x) = x -\frac{x^2}{2}
\qquad T_3(x) = x -\frac{x^2}{2} + \frac{x^3}{3}$$

Les formules de Taylor nous disent que les restes sont de plus en plus petits lorsque $n$ croît.
Sur le dessins les graphes des polynômes $T_0, T_1, T_2, T_3$ s'approchent de plus en plus du graphe de $f$.
Attention ceci n'est vrai qu'autour de $0$.

\myfigure{1}{
\tikzinput{fig_dl02}
}

Pour $n$ quelconque nous avons calculé que le polynôme de Taylor en $0$ est
$$T_n(x)=\sum_{k=1}^n (-1)^{k-1}\frac{x^k}{k} = x-\frac{x^2}{2} + \frac{x^3}{3}-\cdots + (-1)^{n-1}\frac{x^n}{n}.$$


%---------------------------------------------------------------
\subsection{Résumé}


Il y a donc trois formules de Taylor qui s'écrivent toutes sous la forme
$$f(x) = T_n(x) + R_n(x)$$
où $T_n(x)$ est toujours le même polynôme de Taylor :
$$T_n(x) =f(a)+f'(a)(x-a)+\frac{f''(a)}{2!}(x-a)^2+\cdots
+\frac{f^{(n)}(a)}{n!}(x-a)^n.$$

C'est l'expression du reste $R_n(x)$ qui change (attention le reste n'a aucune raison d'être un polynôme).

\begin{align*}
R_n(x) & = \int_a^x \frac{f^{(n+1)}(t)}{n!}(x-t)^ndt
 & \quad   \text{Taylor avec reste intégral} \\
R_n(x) & =\frac{f^{(n+1)}(c)}{(n+1)!}(x-a)^{n+1}
 & \quad   \text{Taylor avec reste } f^{(n+1)}(c),\   c \text{ entre } a \text{ et } x \\
R_n(x) & = (x-a)^n\epsilon(x)
 & \quad \text{Taylor-Young avec } \epsilon(x) \xrightarrow[x\to a]{} 0 \\
\end{align*}
Selon les situations l'une des formulations est plus adaptée que les autres.
Bien souvent nous n'avons pas besoin de beaucoup d'information sur le reste et
c'est donc la formule de Taylor-Young qui sera la plus utile.

Notons que les trois formules ne requièrent pas exactement les mêmes hypothèses:
Taylor avec reste intégral à l'ordre $n$ exige une fonction de classe $\mathcal{C}^{n+1}$,
Taylor avec reste une fonction $n+1$ fois dérivable, et Taylor-Young une fonction $\mathcal{C}^n$.
Une hypothèse plus restrictive donne logiquement une conclusion plus forte.
Cela dit, pour les fonctions de classe $\mathcal{C}^\infty$ que l'on manipule le plus souvent,
les trois hypothèses sont toujours vérifiées.

\bigskip

\textbf{Notation.}
Le terme $(x-a)^n\epsilon(x)$ où $\epsilon(x) \xrightarrow[x\to 0]{} 0$
est souvent abrégé en \og \defi{petit o} \fg{} de $(x-a)^n$ et est noté $o((x-a)^n)$.
Donc $o((x-a)^n)$ est une fonction telle que $\lim_{x\to a}\frac{o((x-a)^n)}{(x-a)^n}=0$.
Il faut s'habituer à cette notation qui simplifie les écritures, mais il faut toujours garder à l'esprit
ce qu'elle signifie.

\bigskip

\textbf{Cas particulier : Formule de Taylor-Young au voisinage de $0$.}
On se ramène souvent au cas particulier où $a=0$, la formule de Taylor-Young s'écrit alors

\mybox{$\displaystyle f(x)= f(0)+f'(0)x+f''(0)\frac{x^2}{2!}+\cdots
+f^{(n)}(0)\frac{x^n}{n!} + x^n\epsilon(x)$}
où \ $\lim_{x\to0}\epsilon (x)=0$.

Et avec la notation \og petit o \fg{} cela donne :
$$f(x)= f(0)+f'(0)x+f''(0)\frac{x^2}{2!}+\cdots
+f^{(n)}(0)\frac{x^n}{n!} + o(x^n)$$


%---------------------------------------------------------------
%\subsection{Mini-exercices}

\begin{miniexercices}
\sauteligne
\begin{enumerate}
  \item Écrire les trois formules de Taylor en $0$ pour $x \mapsto \cos x$, $x\mapsto \exp(-x)$ et $x\mapsto \sh x$.
  \item Écrire les formules de Taylor en $0$ à l'ordre $2$ pour $x \mapsto \frac{1}{\sqrt{1+x}}$, $x\mapsto \tan x$.
  \item Écrire les formules de Taylor en $1$ pour $x \mapsto x^3-9x^2+14x+3$.
  \item Avec une formule de Taylor à l'ordre $2$ de $\sqrt{1+x}$, trouver une approximation
de $\sqrt{1,01}$. Idem avec $\ln(0,99)$.
\end{enumerate}
\end{miniexercices}

%%%%%%%%%%%%%%%%%%%%%%%%%%%%%%%%%%%%%%%%%%%%%%%%%%%%%%%%%%%%%%%%
\section{Développements limités au voisinage d'un point}


%---------------------------------------------------------------
\subsection{Définition et existence}


Soit $I$ un intervalle ouvert et $f : I \to \Rr$ une fonction quelconque.

\begin{definition}
Pour $a\in I$ et $n\in \Nn$, on dit que $f$ admet un
\defi{développement limité}\index{developpement limite@développement limité} (\defi{DL}) au point $a$ et à l'ordre $n$, s'il existe
des réels $c_0, c_1,\ldots,c_n$
et une fonction $\epsilon : I \to \Rr$ telle que $\lim_{x\to a} \epsilon(x)=0$
de sorte que pour tout $x\in I$ :
$$f(x)=c_0+c_1 (x-a)+\cdots+c_n(x-a)^n+(x-a)^n\epsilon(x).$$
\begin{itemize}
\item L'égalité précédente s'appelle un DL de $f$ au voisinage de $a$ à l'ordre $n$ .

\item Le terme $c_0+c_1(x-a)+\cdots+c_n(x-a)^n$ est appelé la \defi{partie polynomiale}\index{partie polynomiale} du DL.

\item Le terme $(x-a)^n\epsilon(x)$ est appelé le \defi{reste} du DL.
\end{itemize}
\end{definition}


La formule de Taylor-Young permet d'obtenir immédiatement des développements limités
en posant $c_k = \frac{f^{(k)}(a)}{k!}$ :
\begin{proposition}
Si $f$ est de classe $\mathcal{C}^n$ au voisinage d'un point $a$
alors $f$ admet un DL au point $a$ à l'ordre $n$, qui provient de la
formule de Taylor-Young:
$$f(x)= f(a)+\frac{f'(a)}{1!}(x-a)+ \frac{f''(a)}{2!}(x-a)^2+\cdots
+\frac{f^{(n)}(a)}{n!}(x-a)^n+(x-a)^n\epsilon(x)$$
où $\lim_{x\to a}\epsilon(x)=0$.
\end{proposition}

\begin{remarque*}
\sauteligne
\begin{enumerate}
  \item Si $f$ est de classe $\mathcal{C}^n$
au voisinage d'un point $0$, un DL en $0$ à l'ordre $n$ est l'expression :
$$f(x)= f(0)+f'(0)x+f''(0)\frac{x^2}{2!}+\cdots
+f^{(n)}(0)\frac{x^n}{n!} + x^n\epsilon(x)$$

  \item Si $f$ admet un DL en un point $a$ à l'ordre $n$
alors elle en possède un pour tout $k \le n$.
En effet
\begin{eqnarray*}
f(x) & = & f(a)+\frac{f'(a)}{1!}(x-a)+\cdots + \frac{f^{(k)}(a)}{k!}(x-a)^k
\\
&& + \underbrace{\frac{f^{(k+1)}(a)}{(k+1)!}(x-a)^{k+1} + \cdots
+\frac{f^{(n)}(a)}{n!}(x-a)^n+(x-a)^n\epsilon(x)}_{=(x-a)^k\eta(x)}
\end{eqnarray*}
où $\lim_{x\to a}\eta(x)=0$.
\end{enumerate}
\end{remarque*}

%---------------------------------------------------------------
\subsection{Unicité}

\begin{proposition}
Si $f$ admet un DL alors ce DL est unique.
\end{proposition}

\begin{proof}
Écrivons deux DL de $f$ :
$f(x)=c_0+c_1(x-a)+\cdots+c_n(x-a)^n+(x-a)^n\epsilon_1(x)$
et $f(x)=d_0+d_1(x-a)+\cdots+d_n(x-a)^n+(x-a)^n\epsilon_2(x)$.
En effectuant la différence on obtient:
\[
(d_0-c_0)+(d_1-c_1)(x-a)+\cdots+(d_n-c_n)(x-a)^n+(x-a)^n(\epsilon_2(x)-\epsilon_1(x))=0.
\]
Lorsque l'on fait $x=a$ dans cette égalité alors on trouve $d_0-c_0=0$.
Ensuite on peut diviser cette égalité par $x-a$  :
$(d_1-c_1)+(d_2-c_2)(x-a)+\cdots+(d_n-c_n)(x-a)^{n-1}+(x-a)^{n-1}(\epsilon_2(x)-\epsilon_1(x))=0$.
En évaluant en $x=a$ on obtient $d_1-c_1=0$, etc.
On trouve $c_0=d_0$, $c_1=d_1$, \ldots, $c_n=d_n$.
Les parties polynomiales sont égales et donc les restes aussi.
\end{proof}




\begin{corollaire}
Si $f$ est paire (resp. impaire) alors la partie polynomiale de son DL en $0$
ne contient que des monômes de degrés pairs (resp. impairs).
\end{corollaire}

Par exemple $x \mapsto \cos x$ est paire et nous verrons que son DL en $0$ commence par :
$\cos x=1-\frac{x^2}{2!}+\frac{x^4}{4!}-\frac{x^6}{6!}+\cdots$.

\begin{proof}
 $f(x)=c_0+c_1 x +c_2x^2+c_3x^3+\cdots+c_nx^n+x^n\epsilon(x)$. Si $f$ est paire alors
$f(x)=f(-x)=c_0-c_1 x+c_2x^2-c_3x^3 +\cdots+(-1)^nc_nx^n+x^n\epsilon(x)$.
Par l'unicité du DL en $0$ on trouve $c_1=-c_1$, $c_3=-c_3$, \ldots \
et donc $c_1=0$, $c_3=0$,\ldots
\end{proof}


\begin{remarque*}
\sauteligne
\begin{enumerate}
  \item L'unicité du DL et la formule de Taylor-Young
prouve que si l'on connaît le DL et que $f$ est de classe $\mathcal{C}^n$ alors
on peut calculer les nombres dérivés à partir de la partie polynomiale par la formule
$c_k = \frac{f^{(k)}(a)}{k!}$. Cependant dans la majorité des cas on fera l'inverse :
on trouve le DL à partir des dérivées.

  \item  Si $f$ admet un DL en un point $a$ à l'ordre $n \ge
   0$ alors $f$ est continue en $a$ et $c_0=f(a)$.

\item Si $f$ admet un DL en un point $a$ à l'ordre $n \ge1$,
alors $f$ est dérivable en $a$ et on a  $c_0=f(a)$ et $c_1=f'(a)$.
Par conséquent $y=c_0+c_1(x-a)$ est l'équation de la tangente au graphe de $f$
au point d'abscisse $a$.

\item Plus subtil: $f$ peut admettre un DL à l'ordre $2$ en
   un point $a$ \emph{sans} admettre une dérivée seconde en $a$.
Soit par exemple $f(x) = x^3 \sin\frac1x$. Alors $f$ est dérivable mais
$f'$ ne l'est pas. Pourtant $f$ admet un DL en $0$ à l'ordre $2$ :
$f(x)=x^2 \epsilon(x)$ (la partie polynomiale est nulle).
\end{enumerate}
\end{remarque*}

%---------------------------------------------------------------
\subsection{DL des fonctions usuelles à l'origine}


Les DL suivants en $0$ proviennent de la formule de Taylor-Young.
\index{developpement limite@développement limité}
\begin{center}
\mybox{$\exp x=1+\frac{x}{1!}+\frac{x^2}{2!}+\frac{x^3}{3!}+\cdots+\frac{x^n}{n!}
+x^n\epsilon(x)$}

\smallskip


$\ch x=1+\frac{x^2}{2!}+\frac{x^4}{4!}+\cdots+\frac{x^{2n}}{(2n)!}
+x^{2n+1}\epsilon(x)$

\smallskip

$\sh x=\frac{x}{1!}+\frac{x^3}{3!}+\frac{x^5}{5!}+\cdots
+\frac{x^{2n+1}}{(2n+1)!}
+x^{2n+2}\epsilon(x)$

\smallskip

$\cos x=1-\frac{x^2}{2!}+\frac{x^4}{4!}-\cdots+(-1)^n\frac{x^{2n}}{(2n)!}
+x^{2n+1}\epsilon(x)$

\smallskip

$\sin x=\frac{x}{1!}-\frac{x^3}{3!}+\frac{x^5}{5!}-\cdots
+(-1)^n\frac{x^{2n+1}}{(2n+1)!}
+x^{2n+2}\epsilon(x)$

\smallskip

\mybox{$\ln(1+x)=x-\frac{x^2}{2}+\frac{x^3}{3}-\cdots
+(-1)^{n-1}\frac{x^{n}}{n} +x^{n}\epsilon(x)$}

\smallskip

\mybox{$(1+x)^{\alpha}=1+\alpha x+\frac{\alpha(\alpha-1)}{2!}x^2+\cdots
+\frac{\alpha(\alpha-1)...(\alpha-n+1)}{n!}x^n+x^n\epsilon(x)$}

\smallskip

${\displaystyle \frac{1}{1+x}}=1-x+x^2-x^3+\cdots+(-1)^nx^n+x^n\epsilon(x)$

\smallskip

${\displaystyle \frac{1}{1-x}} = 1+x+x^2+\cdots+x^n+x^n\epsilon(x)$

\smallskip

$\sqrt{1+x}  =\ \ 1 + \frac{x}{2} - \frac{1}{8}x^2+ \cdots +
(-1)^{n-1} \frac{1\cdot1\cdot3\cdot5\cdots(2n-3)}{2^n n!}x^n\ \  + x^n\epsilon(x)$
\end{center}


Ils sont tous à apprendre par cœur. C'est facile avec les remarques suivantes :
\begin{itemize}
  \item Le DL de $\ch x$ est la partie paire du DL de $\exp x$. C'est-à-dire que l'on ne retient que les
monômes de degré pair. Alors que le DL de $\sh x$ est la partie impaire.

  \item Le DL de $\cos x$ est la partie paire du DL de $\exp x$ en alternant le signe $+/-$ du monôme.
Pour $\sin x$ c'est la partie impaire de $\exp x$ en alternant aussi les signes.

  \item On notera que la précision du DL de $\sin x$ est meilleure que l'application naïve de la formule de Taylor
  le prévoit ($x^{2n+2} \epsilon(x)$ au lieu de $x^{2n+1} \epsilon(x)$) ;
  c'est parce que le DL est en fait à l'ordre $2n+2$,
  avec un terme polynomial en $x^{2n+2}$ nul (donc absent). Le même phénomène est vrai pour tous les DL pairs ou impairs (dont $\sh x, \cos x, \ch x$).

  \item Pour $\ln(1+x)$ n'oubliez pas qu'il n'y a pas de terme constant, pas de factorielle aux dénominateurs,
et que les signes alternent.

  \item Il faut aussi savoir écrire le DL à l'aide des sommes formelles (et ici des \og petits o \fg{}):
$$\exp x=\sum_ {k=1}^n \frac{x^k}{k!} \ \ + o(x^n) \qquad \text{ et } \qquad \ln(1+x)=\sum_ {k=1}^n (-1)^{k-1}\frac{x^k}{k}\ \  + o(x^n)$$

  \item La DL de $(1+x)^\alpha$ est valide pour tout $\alpha \in \Rr$. Pour $\alpha = -1$ on retombe sur le DL
de $(1+x)^{-1} = \frac{1}{1+x}$. Mais on retient souvent le DL de $\frac{1}{1-x}$ qui est très facile.
Il se retrouve aussi avec la somme d'une suite géométrique :
$1+x+x^2+\cdots+x^n = \frac{1-x^{n+1}}{1-x}= \frac{1}{1-x} - \frac{x^{n+1}}{1-x} = \frac{1}{1-x} + x^n\epsilon(x)$.

  \item Pour $\alpha = \frac12$ on retrouve $(1+x)^{\frac12} = \sqrt{1+x} = 1 + \frac{x}{2} - \frac{1}{8}x^2+\cdots$.
Dont il faut connaître les trois premiers termes.
\end{itemize}


%---------------------------------------------------------------
\subsection{DL des fonctions en un point quelconque}


La fonction $f$ admet un DL au voisinage d'un point $a$ si et seulement si la
fonction $x \mapsto f(x+a)$ admet un DL au voisinage de $0$.
Souvent on ramène donc le problème en $0$ en faisant le changement de variables $h=x-a$.

\begin{exemple}
\sauteligne
\begin{enumerate}
  \item DL de $f(x)=\exp x$ en $1$.

On pose $h=x-1$. Si $x$ est proche de $1$ alors $h$ est proche de $0$.
Nous allons nous ramener à un DL de $\exp h$ en $h=0$. On note $e=\exp 1$.
\begin{eqnarray*}
\exp x 
  &=& \exp( 1+ (x-1) ) = \exp(1) \exp (x-1) = e \exp h \\
  &=& e \left(1+h+ \frac{h^2}{2!} + \cdots + \frac{h^n}{n!}+h^n\epsilon(h)\right)
\\
&=&  e \left(1+(x-1)+\frac{(x-1)^2}{2!}+\cdots
+\frac{(x-1)^n}{n!}+(x-1)^n\epsilon(x-1)\right)
\end{eqnarray*}
où $\lim_{x\to1}\epsilon(x-1)=0$.

  \item DL de $g(x)=\sin x$ en $\pi/2$.

Sachant $\sin x=\sin(\frac{\pi}{2}+x-\frac{\pi}{2})
=\cos (x-\frac{\pi}{2})$ on se ramène au DL de
$\cos h$ quand $h=x-\frac{\pi}{2} \to 0$.
On a donc
$\sin x =1-\frac{(x-\frac{\pi}{2})^2}{2!}+\cdots
+(-1)^n\frac{(x-\frac{\pi}{2})^{2n}}{(2n)!}
+(x-\frac{\pi}{2})^{2n+1}\epsilon(x-\frac{\pi}{2})$,  où
$\lim_{x\to\pi/2}\epsilon(x-\frac{\pi}{2})=0$.

  \item DL de $\ell(x)=\ln(1+3x)$ en $1$ à l'ordre $3$.

Il faut se ramener à un DL du type $\ln(1+h)$ en $h=0$.
On pose $h=x-1$ (et donc $x=1+h$).

On a $\ell(x)=\ln(1+3x) = \ln\big(1+3(1+h)\big)
=  \ln(4 + 3h) = \ln\big(4 \cdot (1+\frac{3h}{4})\big)
= \ln 4 + \ln\big(1+\frac{3h}{4}\big)
= \ln 4 + \frac{3h}{4} - \frac12 \big(\frac{3h}{4} \big)^2
+ \frac13 \big(\frac{3h}{4} \big)^3 + h^3 \epsilon(h)
= \ln 4 + \frac{3(x-1)}{4} - \frac{9}{32}(x-1)^2
+ \frac{9}{64}(x-1)^3 + (x-1)^3 \epsilon(x-1)$
 où $\lim_{x\to1}\epsilon(x-1)=0$.
\end{enumerate}
\end{exemple}

%---------------------------------------------------------------
%\subsection{Mini-exercices}

\begin{miniexercices}
\sauteligne
\begin{enumerate}
  \item Calculer le DL en $0$ de $x\mapsto \ch x$ par la formule de Taylor-Young.
Retrouver ce DL en utilisant que $\ch x = \frac{e^x+e^{-x}}{2}$.
  \item Écrire le DL en $0$ à l'ordre $3$ de $\sqrt[3]{1+x}$. Idem avec $\frac{1}{\sqrt{1+x}}$.
  \item Écrire le DL en $2$ à l'ordre $2$ de $\sqrt{x}$.
  \item Justifier l'expression du DL de $\frac{1}{1-x}$ à l'aide de l'unicité du DL
et de la somme d'une suite géométrique.
\end{enumerate}
\end{miniexercices}



%%%%%%%%%%%%%%%%%%%%%%%%%%%%%%%%%%%%%%%%%%%%%%%%%%%%%%%%%%%%%%%%
\section{Opérations sur les développements limités}

%---------------------------------------------------------------
\subsection{Somme et produit}

On suppose que $f$ et $g$ sont deux fonctions qui admettent des DL en $0$ à l'ordre $n$ :
$$f(x)=c_0+c_1x + \cdots +c_nx^n + x^n\epsilon_1(x)\qquad
g(x)=d_0+d_1x + \cdots +d_nx^n + x^n\epsilon_2(x)$$

\begin{proposition}
\sauteligne
\begin{itemize}
  \item $f+g$  admet un DL en $0$ l'ordre $n$ qui est :
$$(f+g)(x)=f(x)+g(x)=(c_0+d_0)+(c_1+d_1)x+\cdots+(c_n+d_n)x^n +x^n\epsilon(x).$$

  \item $f\times g$ admet un DL en $0$ l'ordre $n$ qui est :
$(f \times g)(x)=f(x) \times g(x)= T_n(x)+x^n\epsilon (x)$
où $T_n(x)$ est le  polynôme
$(c_0+c_1x + \cdots +c_nx^n)\times(d_0+d_1x + \cdots +d_nx^n)$
tronqué à l'ordre $n$.
\end{itemize}
\end{proposition}

\defi{Tronquer} un polynôme à l'ordre $n$ signifie que l'on conserve seulement
les monômes de degré $\le n$.

\begin{exemple}
Calculer le DL de $\cos x \times \sqrt{1+x}$ en $0$  à l'ordre $2$.
On sait que 
$\cos x=1-\frac{1}{2}x^2+x^2\epsilon_1(x)$
et $\sqrt{1+x}=1+\frac{1}{2}x-\frac{1}{8}x^2+x^2\epsilon_2(x)$.
Donc :
\begin{align*}
\cos x \times \sqrt{1+x}
  & = \left( 1-\frac{1}{2}x^2+x^2\epsilon_1(x)\right)\times \left(1+\frac{1}{2}x-\frac{1}{8}x^2+x^2\epsilon_2(x)\right)  \\
  & = 1+\frac{1}{2}x-\frac{1}{8}x^2+x^2\epsilon_2(x)\quad \text{on développe}\\
  &  \qquad -\frac{1}{2}x^2\left(1+\frac{1}{2}x-\frac{1}{8}x^2+x^2\epsilon_2(x)\right) \\
  &  \qquad +x^2\epsilon_1(x)\left(1+\frac{1}{2}x-\frac{1}{8}x^2+x^2\epsilon_2(x)\right) \\
  & = 1+\frac{1}{2}x-\frac{1}{8}x^2+x^2\epsilon_2(x) \quad \quad \text{on développe encore} \\
  &  \qquad -\frac{1}{2}x^2-\frac{1}{4}x^3+\frac{1}{16}x^4-\frac12x^4\epsilon_2(x)\\
  &  \qquad +x^2\epsilon_1(x) +\frac{1}{2}x^3\epsilon_1(x)-\frac{1}{8}x^4\epsilon_1(x)+x^4\epsilon_1(x)\epsilon_2(x) \\
  & = \underbrace{1+\frac{1}{2}x + \left(-\frac{1}{8}x^2-\frac{1}{2}x^2\right)}_{\text{partie tronquée à l'ordre $2$}}
\quad \text{termes de degré $0$ et $1$, $2$}\\
  & \qquad  + \underbrace{
  \begin{array}{cc}
  x^2\epsilon_2(x) -\frac{1}{4}x^3+\frac{1}{16}x^4-\frac12x^4\epsilon_2(x) +x^2\epsilon_1(x) \\
+\frac{1}{2}x^3\epsilon_1(x)-\frac{1}{8}x^4\epsilon_1(x)+x^4\epsilon_1(x)\epsilon_2(x)
  \end{array}
}_{\text{reste de la forme }
x^2\epsilon(x)} \ \text{et les autres}\\
  & = 1+\frac{1}{2}x-\frac{5}{8}x^2+x^2\epsilon(x)
\end{align*}
On a en fait écrit beaucoup de choses superflues, qui à la fin sont dans le reste et n'avaient pas besoin d'être explicitées !
Avec l'habitude les calculs se font très vite car on n'écrit plus les termes inutiles. Voici le même calcul
avec la notation \og petit o \fg{} : dès qu'apparaît un terme $x^2\epsilon_1(x)$ ou un terme $x^3$,... on écrit juste $o(x^2)$
(ou si l'on préfère $x^2\epsilon(x)$).
\begin{align*}
\cos x \times \sqrt{1+x}
  & = \left( 1-\frac{1}{2}x^2+ o(x^2)\right)\times \left(1+\frac{1}{2}x-\frac{1}{8}x^2+o(x^2)\right)  \quad \text{on développe}\\
  & = 1+\frac{1}{2}x-\frac{1}{8}x^2+o(x^2)\\
  &  \qquad -\frac{1}{2}x^2 + o(x^2) \\
  &  \qquad + o(x^2) \\
  & = 1+\frac{1}{2}x-\frac{5}{8}x^2+o(x^2) \\
\end{align*}
La notation «petit o» évite de devoir donner un nom à chaque fonction, en ne gardant que
sa propriété principale, qui est de décroître vers $0$ au moins à une certaine vitesse.
Comme on le voit dans cet exemple, $o(x^2)$ \emph{absorbe} les éléments de même ordre de grandeur
ou plus petits que lui: $o(x^2)-\frac{1}{4}x^3 + \frac{1}{2} x^2 o(x^2) = o(x^2)$.
Mais il faut bien comprendre que les différents $o(x^2)$ écrits ne correspondent pas à la même fonction,
ce qui justifie que cette égalité ne soit pas fausse!
\end{exemple}



%---------------------------------------------------------------
\subsection{Composition}

On écrit encore :
$$f(x) = C(x) + x^n\epsilon_1(x)=c_0+c_1x + \cdots +c_nx^n + x^n\epsilon_1(x)$$
$$g(x) = D(x) + x^n\epsilon_2(x)=d_0+d_1x + \cdots +d_nx^n + x^n\epsilon_2(x)$$
\begin{proposition}
Si $g(0)=0$ (c'est-à-dire $d_0=0$) alors la fonction $f\circ g$ admet un DL en $0$ à l'ordre $n$
dont la partie polynomiale est
le polynôme tronqué à l'ordre $n$ de la composition $C(D(x))$.
\end{proposition}

\begin{exemple}
Calcul du DL de $h(x)=\sin\big(\ln(1+x)\big)$ en $0$ à l'ordre $3$.

\begin{itemize}
  \item On pose ici $f(u)=\sin u$ et $g(x)=\ln(1+x)$ (pour plus de clarté il est préférable de donner 
  des noms différents aux variables des deux fonctions, ici $x$ et $u$).
On a bien $f\circ g(x) = \sin\big(\ln(1+x)\big)$ et $g(0)=0$.

  \item On écrit le DL à l'ordre 3 de $f(u)=\sin u = u-\frac{u^3}{3!}+u^3\epsilon_1(u)$ pour $u$ proche de $0$.

  \item Et on pose $u=g(x)=\ln(1+x)=x-\frac{x^2}{2}+\frac{x^3}{3}+x^3\epsilon_2(x)$ pour $x$ proche de $0$.

  \item On aura besoin de calculer un DL à l'ordre 3 de $u^2$ (qui est bien sûr le produit $u\times u$):
$u^2 = \big(x-\frac{x^2}{2}+\frac{x^3}{3}+x^3\epsilon_2(x)\big)^2 = x^2-x^3+x^3\epsilon_3(x)$
et aussi $u^3$ qui est $u \times u^2$, $u^3=x^3+x^3\epsilon_4(x)$.
  \item Donc
$h(x)=f\circ g(x)= f(u) = u-\frac{u^3}{3!}+u^3\epsilon_1(u)=
\big(x-\frac{1}{2}x^{2}+\frac{1}{3}x^3\big) -\frac16 x^3  +x^3\epsilon(x)
= x-\frac{1}{2}x^{2}+\frac16 x^3  +x^3\epsilon(x)$.
\end{itemize}
\end{exemple}


\begin{exemple}
Soit $h(x)=\sqrt{\cos x}$. On cherche le DL de $h$ en $0$ à l'ordre $4$.

On utilise cette fois la notation \og petit o \fg{}.
On connaît le DL de $f(u)=\sqrt{1+u}$ en $u=0$ à l'ordre $2$ :
$f(u)=\sqrt{1+u}=1+\frac{1}{2}u-\frac{1}{8}u^2 + o(u^2)$.

Et si on pose $u(x)=\cos x-1$ alors on a $h(x)= f\big(u(x)\big)$ et $u(0)=0$.
D'autre part le DL de $u(x)$ en $x=0$ à l'ordre $4$ est :
$u=-\frac{1}{2}x^2+\frac{1}{24}x^4+o(x^4)$.
On trouve alors
$u^2 = \frac{1}{4}x^4 + o(x^4)$.


Et ainsi
\begin{align*}
h(x) & = f\big(u\big) = 1+\frac{1}{2}u-\frac{1}{8}u^2 + o(u^2) \\
     & = 1 + \frac{1}{2}\big(-\frac{1}{2}x^2+\frac{1}{24}x^4\big)-\frac{1}{8}\big(\frac{1}{4}x^4\big) + o(x^4) \\
     & =1-\frac{1}{4}x^2+\frac{1}{48}x^4 -\frac{1}{32}x^4+o(x^4) \\
     & = 1-\frac{1}{4}x^2-\frac{1}{96}x^4+o(x^4) \\
\end{align*}
\end{exemple}


%---------------------------------------------------------------
\subsection{Division}

Voici comment calculer le DL d'un quotient $f/g$. Soient
$$f(x)=c_0+c_1x + \cdots +c_nx^n + x^n\epsilon_1(x) \qquad
g(x)=d_0+d_1x + \cdots +d_nx^n + x^n\epsilon_2(x)$$
Nous allons utiliser le DL de $\frac{1}{1+u} = 1-u+u^2-u^3+\cdots$.

\begin{enumerate}
  \item Si $d_0= 1$ on pose $u = d_1x + \cdots +d_nx^n + x^n\epsilon_2(x) $
et le quotient s'écrit $f/g = f \times \frac{1}{1+u}$.
  \item Si $d_0$ est quelconque avec $d_0 \neq 0$ alors
on se ramène au cas précédent en écrivant
\[
\frac{1}{g(x)} =\frac{1}{d_0} \frac{1}{1+\frac{d_1}{d_0}x+\cdots+\frac{d_n}{d_0}x^n
+\frac{x^n\epsilon_2(x)}{d_0}}.
\]

  \item Si $d_0=0$ alors on factorise par $x^k$ (pour un certain $k$) afin de se ramener aux cas précédents.
\end{enumerate}

\begin{exemple}
\sauteligne
\begin{enumerate}
  \item DL de $\tan x$ en $0$ à l'ordre $5$.

Tout d'abord
$\sin x=x-\frac{x^3}{6}+\frac{x^5}{120}+x^5\epsilon(x)$.
D'autre part
 $\cos  x=1-\frac{x^2}{2}+\frac{x^4}{24} +x^5\epsilon(x)=1+u$
en posant $u= -\frac{x^2}{2}+\frac{x^4}{24} +x^5\epsilon(x)$.

Nous aurons besoin de $u^2$ et $u^3$ :
$u^2 = \big(-\frac{x^2}{2}+\frac{x^4}{24} +x^5\epsilon(x)\big)^2
= \frac{x^4}{4}+x^5\epsilon(x)$ et en fait $u^3 = x^5\epsilon(x)$.
(On note abusivement $\epsilon(x)$ pour différents restes.)

Ainsi
\begin{eqnarray*}
\frac{1}{\cos x} 
  &=& \frac{1}{1+u} =1-u+u^2-u^3+u^3\epsilon(u) \\
  &=& 1+\frac{x^2}{2}-\frac{x^4}{24}+\frac{x^4}{4}+x^5\epsilon(x) \\
  &=& 1+\frac{x^2}{2}+\frac{5}{24}x^4+x^5\epsilon(x).
\end{eqnarray*}
Finalement
\begin{eqnarray*}
\tan x 
  &=& \sin x \times \frac1{\cos x} \\
  &=& \big(x-\frac{x^3}{6}+\frac{x^5}{120}+x^5\epsilon(x)\big)\times\big(1+\frac{x^2}{2}+\frac{5}{24}x^4+x^5\epsilon(x)\big) \\
  &=& x +\frac{x^3}{3}+\frac{2}{15}x^5+x^5\epsilon(x).
\end{eqnarray*}

  \item DL de $\frac{1+x}{2+x}$ en $0$ à l'ordre $4$.
\begin{eqnarray*}
\frac{1+x}{2+x} 
  &=& (1+x)\frac12\frac{1}{1+\frac{x}{2}} \\
  &=& \frac12(1+x) \left( 1-\frac{x}{2}+\left(\frac{x}{2}\right)^2-\left(\frac{x}{2}\right)^3
+\left( \frac{x}{2} \right)^4 + o(x^4) \right) \\
  &=& \frac12+\frac{x}{4}-\frac{x^2}{8}+\frac{x^3}{16}-\frac{x^4}{32} + o(x^4)
\end{eqnarray*}

  \item Si l'on souhaite calculer le DL de $\frac{\sin x}{\sh x}$ en $0$ à l'ordre $4$
alors on écrit
\begin{eqnarray*}
\frac{\sin x}{\sh x} &=&
\frac{x-\frac{x^3}{3!} + \frac{x^5}{5!}+ o(x^5)}{x+\frac{x^3}{3!} + \frac{x^5}{5!} + o(x^5)}
= \frac{x\big(1-\frac{x^2}{3!} + \frac{x^4}{5!}+ o(x^4)\big)}{x\big(1+\frac{x^2}{3!}+ \frac{x^4}{5!} + o(x^4)\big)}
\\
&=& \big(1-\frac{x^2}{3!}  + \frac{x^4}{5!}+ o(x^4)\big) \times \frac{1}{1+\frac{x^2}{3!}+ \frac{x^4}{5!} + o(x^4)}\\
&=& \cdots \  = 1-\frac{x^2}{3}+\frac{x^4}{18} + o(x^4)
\end{eqnarray*}
\end{enumerate}
\end{exemple}


\textbf{Autre méthode.}
Soit
$f(x)=C(x) + x^n\epsilon_1(x)$ et $g(x)=D(x) + x^n\epsilon_2(x)$.
Alors on écrit la division suivant les puissances croissantes de
$C$ par $D$ à l'ordre $n$ : $C=DQ+ x^{n+1}R$ avec $\deg Q \le n$.
Alors $Q$ est la partie polynomiale du DL en $0$ à l'ordre $n$ de $f/g$.

\begin{exemple}
DL de $\frac{2+x+2x^3}{1+x^2}$ à l'ordre $2$.
On pose $C(x)=2+x+2x^3$ et $g(x)=D(x)=1+x^2$ alors
$C(x)=D(x)\times (2+x-2x^2) + x^3(1+2x)$. On a donc $Q(x)=2+x-2x^2$,
$R(x) = 1+2x$. Et donc lorsque l'on divise cette égalité par $D(x)$ on obtient
$\frac{f(x)}{g(x)}=2+x-2x^2 + x^2\epsilon(x)$.
\end{exemple}


%---------------------------------------------------------------
\subsection{Intégration}

Soit $f : I\to \Rr$ une fonction de classe $\mathcal{C}^n$ dont le DL
en $a\in I$ à l'ordre $n$ est $f(x)=c_0+c_1(x-a)+c_2(x-a)^2+\cdots+c_n(x-a)^n+(x-a)^n\epsilon(x)$.
\begin{theoreme}
Notons $F$ une primitive de $f$.
Alors $F$ admet un DL en $a$ à l'ordre $n+1$ qui s'écrit :
\begin{multline*}
F(x)=F(a)+c_0(x-a)+c_1\frac{(x-a)^2}{2}+ c_2\frac{(x-a)^3}{3}+\cdots \\
\cdots +c_n\frac{(x-a)^{n+1}}{n+1}+(x-a)^{n+1}\eta(x)  
\end{multline*}
où $\displaystyle\lim_{x\to a}\eta(x)=0$.
\end{theoreme}

Cela signifie que l'on intègre la partie polynomiale terme à terme pour obtenir le DL de $F(x)$ à la constante $F(a)$ près.

\begin{proof}
On a $F(x)-F(a)=\int_a^xf(t)dt
=a_0(x-a)+\cdots+\frac{a_n}{n+1}(x-a)^{n+1}+\int_a^x(t-a)^{n}
\epsilon(t)dt$.
Notons $\eta(x)=\frac{1}{(x-a)^{n+1}}\int_a^x(t-a)^{n}\epsilon(t)dt$.
(Remarque : la fonction $\epsilon$ est continue : elle est continue en $a$ par définition,  
et elle est continue en dehors de $a$ en écrivant 
$\epsilon(x) = \frac{1}{(x-a)^n}\big(f(x)-(c_0+c_1(x-a)+c_2(x-a)^2+\cdots+c_n(x-a)^n)\big)$.)
Alors :

$|\eta(x)|
  \le \left|\frac{1}{(x-a)^{n+1}} \int_a^x|(t-a)^{n}| \cdot \sup_{t\in[a,x]}|\epsilon(t)|dt\right|
=\left|\frac{1}{(x-a)^{n+1}}\right| \cdot \sup_{t\in[a,x]}|\epsilon(t)| \cdot \int_a^x|(t-a)^{n}| dt
=\frac{1}{n+1}\sup_{t\in[a,x]}|\epsilon(t)|$.

Mais $\sup_{t\in[a,x]}|\epsilon(t)| \to 0$ lorsque $x\to a$. Donc $\eta(x)\to 0$ quand $x\to a$.

\end{proof}

\begin{exemple}
Calcul du DL de $\arctan x$.


On sait que $\arctan' x=\frac{1}{1+x^2}$. En posant $f(x)=\frac{1}{1+x^2}$
et $F(x)=\arctan x$,  on écrit
$$\arctan' x=\frac{1}{1+x^2}=\sum_{k=0}^{n}(-1)^kx^{2k}+x^{2n}\epsilon(x).$$

Et comme $\arctan(0)=0$ alors
$\arctan x=\sum_{k=0}^{n}\frac{(-1)^k}{2k+1}x^{2k+1}+x^{2n+1}\epsilon(x)
=x-\frac{x^3}{3}+\frac{x^5}{5}-\frac{x^7}{7} +\cdots$
\end{exemple}

\begin{exemple}
La méthode est la même pour obtenir un DL de $\arcsin x$ en $0$ à l'ordre $5$.

$\arcsin' x=(1-x^2)^{-\frac{1}{2}}
=1-\frac{1}{2}(-x^2)
+\frac{-\frac{1}{2}(-\frac{1}{2}-1)}{2}(-x^2)^2+x^4\epsilon(x)
=1+\frac{1}{2}x^2 + \frac{3}{8}x^4+x^4\epsilon(x)$.

Donc $\arcsin x =x+\frac{1}{6}x^3+\frac{3}{40}x^5+x^5\epsilon(x)$.
\end{exemple}


%---------------------------------------------------------------
%\subsection{Mini-exercices}

\begin{miniexercices}
\sauteligne
\begin{enumerate}
  \item Calculer le DL en $0$ à l'ordre $3$ de $\exp(x) -\frac{1}{1+x}$, puis de $x\cos(2x)$ et $\cos(x)\times \sin(2x)$.
  \item Calculer le DL en $0$ à l'ordre $2$ de $\sqrt{1+2\cos x}$, puis de $\exp\big(\sqrt{1+2\cos x}\big)$.
  \item Calculer le DL en $0$ à l'ordre $3$ de $\ln(1+\sin x)$. Idem à l'ordre $6$ pour $\big(\ln(1+x^2)\big)^2$.
  \item Calculer le DL en $0$ à l'ordre $n$ de $\frac{\ln(1+x^3)}{x^3}$. Idem à l'ordre $3$ avec $\frac{e^x}{1+x}$.
  \item Par intégration retrouver la formule du DL de $\ln(1+x)$. Idem à l'ordre $3$ pour $\arccos x$.
\end{enumerate}
\end{miniexercices}


%%%%%%%%%%%%%%%%%%%%%%%%%%%%%%%%%%%%%%%%%%%%%%%%%%%%%%%%%%%%%%%%
\section{Applications des développements limités}

Voici les applications les plus remarquables des développements limités.
On utilisera aussi les DL lors de l'étude locale des courbes paramétrées
lorsqu'il y a des points singuliers.


%---------------------------------------------------------------
\subsection{Calculs de limites}

Les DL sont très efficaces pour calculer des limites ayant des formes indéterminées !
Il suffit juste de remarquer que
si $f(x) = c_0+c_1(x-a)+\cdots$ alors $\lim_{x\to a} f(x) = c_0$.

\begin{exemple}
Limite en $0$ de $\displaystyle \frac{\ln(1+x)-\tan x+\frac{1}{2}\sin^2x}{3x^2\sin^2x}$.

Notons $\frac{f(x)}{g(x)}$ cette fraction.
En $0$ on a $f(x)=\ln(1+x)-\tan x+\frac{1}{2}\sin^2x
=\big(x-\frac{x^2}{2}+\frac{x^3}{3}-\frac{x^4}{4}+o(x^4)\big)
-\big(x+\frac{x^3}{3}+o(x^4)\big)
 +\frac{1}{2}\big(x-\frac{x^3}{6}+o(x^3)\big)^2
=-\frac{x^2}{2}-\frac{x^4}{4}+\frac{1}{2}(x^2-\frac{1}{3}x^4)+o(x^4)
= -\frac{5}{12}x^4 + o(x^4)$
et $g(x)=3x^2\sin^2x=3x^2\big(x+o(x)\big)^2 =3x^4+o(x^4)$.

Ainsi $\frac{f(x)}{g(x)}= \frac{-\frac{5}{12}x^4 + o(x^4)}{3x^4+o(x^4)} = \frac{-\frac{5}{12} + o(1)}{3+o(1)}$
en notant $o(1)$ une fonction (inconnue) tendant vers $0$ quand $x \to 0$.
Donc $\lim_{x\to0}\frac{f(x)}{g(x)}=-\frac{5}{36}$.

Note : en calculant le DL à un ordre inférieur (2 par exemple), on n'aurait pas pu conclure,
car on aurait obtenu $\frac{f(x)}{g(x)} = \frac{o(x^2)}{o(x^2)}$, ce qui ne lève pas l'indétermination.
De façon générale, on calcule les DL à l'ordre le plus bas possible, et si cela ne suffit pas, on augmente progressivement l'ordre (donc la précision de l'approximation).
\end{exemple}



%---------------------------------------------------------------
\subsection{Position d'une courbe par rapport à sa tangente}

\begin{proposition}
Soit $f : I \to \Rr$ une fonction admettant un DL en $a$ :
$f(x)=c_0+c_1(x-a)+c_k(x-a)^k+(x-a)^k\epsilon(x)$,
où $k$ est le plus petit entier $\ge2$ tel que le coefficient $c_k$ soit non nul.
Alors l'équation de la tangente à la courbe de $f$ en $a$ est : $y=c_0+c_1(x-a)$ et
la position de la courbe par rapport à la tangente pour $x$ proche de $a$ est
donnée par le signe $f(x)-y$, c'est-à-dire le signe de $c_k(x-a)^k$.
\end{proposition}


Il y a $3$ cas possibles.
\begin{itemize}
  \item Si ce signe est positif alors la courbe est au-dessus de la tangente.
  \myfigure{1}{
\tikzinput{fig_dl03}
}
  \item Si ce signe est négatif alors la courbe est en dessous de la tangente.
  \myfigure{1}{
\tikzinput{fig_dl04}
}
  \item Si ce signe change (lorsque l'on passe de $x<a$ à $x>a$) alors la courbe traverse
la tangente au point d'abscisse $a$. C'est un \defi{point d'inflexion}\index{point d'inflexion}.
\myfigure{1}{
\tikzinput{fig_dl05}
}
\end{itemize}

Comme le DL de $f$ en $a$ à l'ordre $2$ s’écrit aussi
$f(x)=f(a)+f'(a)(x-a) + \frac{f''(a)}{2}(x-a)^2 + (x-a)^2\epsilon(x)$,
alors l'équation de la tangente est aussi $y=f(a)+f'(a)(x-a)$.
Si en plus $f''(a)\neq 0$ alors $f(x)-y$ garde un signe constant autour de $a$.
En conséquence si $a$ est un point d'inflexion alors $f''(a)=0$. (La réciproque est fausse.)


\begin{exemple}
Soit $f(x)=x^4-2x^3+1$.

\begin{enumerate}
  \item Déterminons la tangente en $\frac{1}{2}$ du graphe de $f$
et précisons la position du graphe par rapport à la tangente.

On a $f'(x)=4x^3-6x^2$, $f''(x)=12x^2-12x$, donc $f''(\frac{1}{2}) = -3 \neq 0$ et $k=2$.
% $f^{(3)}(x)=24x-12$, $f^{(4)}(x)=24$ et pour $k\ge 5$, $f^{(5)}(x)=0$

On en déduit le DL de $f$ en $\frac{1}{2}$ par la formule de Taylor-Young :
$f(x)=f(\frac12)+f'(\frac12)(x-\frac12)+\frac{f''(\frac12)}{2!}(x-\frac12)^2 +(x-\frac12)^2 \epsilon(x)
=\frac{13}{16} -(x- \frac12) -\frac{3}{2}(x-\frac12)^2 + (x-\frac12)^2 \epsilon(x)$.

%$f(x)=f(\frac12)+f'(\frac12)(x-\frac12)+\frac{f''(\frac12)}{2!}(x-\frac12)^2
%+\frac{f^{(3)}(\frac12)}{3!}(x-\frac12)^3+\frac{f^{(4)}(\frac12)}{4!}(x-\frac12)^4
%=\frac{13}{16} -(x- \frac12) -\frac{3}{2}(x-\frac12)^2
%+(x-\frac12)^2\epsilon(x).$

Donc la tangente en $\frac12$ est
$y= \frac{13}{16} -(x- \frac12)$ et le graphe de $f$ est en dessous
de la tangente car $f(x)-y =  \big(-\frac32 + \epsilon(x)\big)(x-\frac12)^2$ est négatif
autour de $x=\frac12$.


  \item Déterminons les points d'inflexion.

Les points d'inflexion sont à chercher parmi les solutions de $f''(x)=0$. Donc parmi $x=0$ et $x=1$.
\begin{itemize}
\item Le DL en $0$ est $f(x)= 1-2x^3+x^4$ (il s'agit juste d'écrire les monômes par degrés croissants !).
L'équation de la tangente au point d'abscisse $0$ est donc $y=1$ (une tangente horizontale).
Comme $-2x^3$ change de signe en $0$ alors $0$ est un point d'inflexion de $f$.

\item Le DL en $1$ : on calcule $f(1)$, $f'(1)$, \ldots pour trouver le DL en $1$
$f(x)= -2(x-1) + 2(x-1)^3+(x-1)^4$.
L'équation de la tangente au point d'abscisse $1$ est donc $y=-2(x-1)$.
Comme $2(x-1)^3$ change de signe en $1$, $1$ est aussi un point d'inflexion de $f$.
\end{itemize}
\end{enumerate}

\myfigure{0.7}{
\tikzinput{fig_dl06a}
\tikzinput{fig_dl06b}
}

\end{exemple}

%---------------------------------------------------------------
\subsection{Développement limité en $+\infty$}

Soit $f$ une fonction définie sur un intervalle $I = ]x_0,+\infty[$.
On dit que $f$ admet un \defi{DL en $+\infty$}\index{developpement limite@développement limité!en l'infini@en $+\infty$} à l'ordre $n$
s'il existe des réels $c_0,c_1,\ldots,c_n$ tels que
$$f(x)=c_0+\frac{c_1}{x}+\cdots+\frac{c_n}{x^n}
+\frac{1}{x^n}\epsilon\big(\frac{1}{x}\big)$$
où $\epsilon\big(\frac{1}{x}\big)$ tend vers $0$ quand $x\to+\infty$.

\begin{exemple}
$f(x)=\ln\big(2+\frac{1}{x}\big)=\ln2+\ln\big(1+\frac{1}{2x}\big)
=\ln2+\frac{1}{2x}-\frac{1}{8x^2}+\frac{1}{24x^3}+\cdots
+(-1)^{n-1}\frac{1}{n2^nx^n} +\frac{1}{x^{n}}\epsilon(\frac{1}{x}),$
où $\lim_{x\to\infty}\epsilon(\frac{1}{x})=0$

\myfigure{1}{
\tikzinput{fig_dl07}
}

Cela nous permet d'avoir une idée assez précise du comportement de $f$ au voisinage de $+\infty$.
Lorsque $x\to +\infty$ alors $f(x)\to \ln2$.
Et le second terme est $+\frac 12x$, donc est positif, cela signifie
que la fonction $f(x)$ tend vers $\ln 2$ tout en restant au-dessus de $\ln 2$.
\end{exemple}


\begin{remarque*}
\sauteligne
\begin{enumerate}
  \item Un DL en $+\infty$ s'appelle aussi un développement asymptotique\index{developpement limite@développement limité!asymptotique}.
  \item Dire que la fonction $x \mapsto f(x)$ admet un DL en $+\infty$ à l'ordre $n$ est équivalent à
dire que la fonction $x \to f(\frac{1}{x})$ admet un DL en $0^+$ à l'ordre $n$.
  \item On peut définir de même ce qu'est un DL en $-\infty$.
\end{enumerate}
\end{remarque*}


\begin{proposition}
On suppose que la fonction $x\mapsto\frac{f(x)}{x}$
admet un DL en $+\infty$ (ou en $-\infty$) :
$\frac{f(x)}{x}= a_0 +\frac{a_1}{x}+\frac{a_k}{x^k}+\frac{1}{x^k}\epsilon(\frac{1}{x}),$
où $k$ est
le plus petit entier $\ge 2$ tel que le coefficient de  $\frac{1}{x^k}$ soit
non nul.
Alors $\lim_{x\to+\infty} f(x)-(a_0x+a_1) =0$ (resp. $x\to -\infty$) : la droite $y= a_0x+a_1$ est une 
\defi{asymptote}\index{asymptote}
à la courbe de $f$ en $+\infty$ (ou $-\infty$) et la position de la courbe par rapport à
l'asymptote est donnée par le signe de $f(x)-y$, c'est-à-dire le signe de $\frac{a_k}{x^{k-1}}$.
\end{proposition}

\myfigure{1}{
\tikzinput{fig_dl08}
}

\begin{proof}
On a $\lim_{x\to+\infty} \big(f(x)-a_0x-a_1\big)
=\lim_{x\to+\infty}\frac{a_k}{x^{k-1}}+\frac{1}{x^{k-1}}\epsilon(\frac{1}{x})=0$.
Donc $y=a_0x+a_1$ est une asymptote à la courbe de $f$.
Ensuite on calcule la différence $f(x)-a_0x-a_1=\frac{a_k}{x^{k-1}}+\frac{1}{x^{k-1}}\epsilon(\frac{1}{x})
=\frac{a_k}{x^{k-1}}\big(1+\frac{1}{a_k}\epsilon(\frac{1}{x})\big)$.
\end{proof}

\begin{exemple}
Asymptotes de  $f(x)=\exp{\frac1x} \cdot \sqrt{x^2-1}$.

\myfigure{1}{
\tikzinput{fig_dl09}
}

\begin{enumerate}
\item En $+\infty$,
\begin{align*}
\frac{f(x)}{x}
  & = \exp{\frac1x} \cdot\frac{\sqrt{x^2-1}}{x} =\exp{\frac1x} \cdot\sqrt{1-\frac{1}{x^2}} \\
  & = \Big(1+\frac{1}{x}+\frac{1}{2x^2}+\frac{1}{6x^3} +\frac{1}{x^3}\epsilon(\frac1x)\Big)
\cdot\Big(1-\frac{1}{2x^2}+\frac{1}{x^3}\epsilon(\frac{1}{x})\Big) \\
  & = \cdots = 1+\frac{1}{x}-\frac{1}{3x^3} +\frac{1}{x^3}\epsilon(\frac{1}{x}) \\
\end{align*}


Donc l'asymptote de $f$ en $+\infty$ est $y=x+1$. Comme
$f(x)-x-1=-\frac{1}{3x^2}+\frac{1}{x^2}\epsilon(\frac{1}{x})$ quand
$x\to+\infty$, le graphe de $f$ reste en dessous de l'asymptote.

\item En $-\infty$.
$\frac{f(x)}{x}=\exp{\frac1x}\cdot\frac{\sqrt{x^2-1}}{x}
=-\exp{\frac1x}\cdot\sqrt{1-\frac{1}{x^2}}
=-1-\frac{1}{x}+\frac{1}{3x^3}+\frac{1}{x^3}\epsilon(\frac{1}{x}).$
Donc $y=-x-1$ est une asymptote de $f$ en $-\infty$. On a
$f(x)+x+1=\frac{1}{3x^2}+\frac{1}{x^2}\epsilon(\frac{1}{x})$ quand
$x\to-\infty$ ; le graphe de $f$ reste au-dessus de l'asymptote.
\end{enumerate}
\end{exemple}


%---------------------------------------------------------------
%\subsection{Mini-exercices}

\begin{miniexercices}
\sauteligne
\begin{enumerate}
  \item Calculer la limite de $\displaystyle \frac{\sin x - x}{x^3}$ lorsque $x$ tend vers $0$.
Idem avec $\displaystyle \frac{\sqrt{1+x}-\sh \frac x2}{x^k}$ (pour $k=1,2,3,\ldots$).
  \item Calculer la limite de $\displaystyle \frac{\sqrt x - 1}{\ln x}$ lorsque $x$ tend vers $1$.
Idem pour $\displaystyle \left(\frac{1-x}{1+x}\right)^{\frac{1}{x}}$,
puis $\displaystyle \frac{1}{\tan^2 x}-\frac{1}{x^2}$ lorsque $x$ tend vers $0$.
  \item Soit $f(x)=\exp x + \sin x$. Calculer l'équation de la tangente en $x=0$ et la position du graphe.
Idem avec $g(x)= \sh x$.
  \item Calculer le DL en $+\infty$ à l'ordre $5$ de $\frac{x}{x^2-1}$. Idem à l'ordre $2$ pour
$\big(1+\frac 1x\big)^{x}$.
  \item Soit $f(x)=\sqrt{\frac{x^3+1}{x+1}}$. Déterminer l'asymptote en $+\infty$ et la position du graphe
par rapport à cette asymptote.
\end{enumerate}
\end{miniexercices}


\auteurs{

Rédaction : Arnaud Bodin

Basé sur des cours de Guoting Chen et Marc Bourdon

Relecture : Pascal Romon

Dessins : Benjamin Boutin
}

\finchapitre
\end{document}


