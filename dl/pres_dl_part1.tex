
%%%%%%%%%%%%%%%%%% PREAMBULE %%%%%%%%%%%%%%%%%%

\documentclass[aspectratio=169,utf8]{beamer}
%\documentclass[aspectratio=169,handout]{beamer}

\usetheme{Boadilla}
%\usecolortheme{seahorse}
\usecolortheme[RGB={245,66,24}]{structure}
\useoutertheme{infolines}

% packages
\usepackage{amsfonts,amsmath,amssymb,amsthm}
\usepackage[utf8]{inputenc}
\usepackage[T1]{fontenc}
\usepackage{lmodern}

\usepackage[francais]{babel}
\usepackage{fancybox}
\usepackage{graphicx}

\usepackage{float}
\usepackage{xfrac}

%\usepackage[usenames, x11names]{xcolor}
\usepackage{tikz}
\usepackage{pgfplots}
\usepackage{datetime}



%-----  Package unités -----
\usepackage{siunitx}
\sisetup{locale = FR,detect-all,per-mode = symbol}

%\usepackage{mathptmx}
%\usepackage{fouriernc}
%\usepackage{newcent}
%\usepackage[mathcal,mathbf]{euler}

%\usepackage{palatino}
%\usepackage{newcent}
% \usepackage[mathcal,mathbf]{euler}



% \usepackage{hyperref}
% \hypersetup{colorlinks=true, linkcolor=blue, urlcolor=blue,
% pdftitle={Exo7 - Exercices de mathématiques}, pdfauthor={Exo7}}


%section
% \usepackage{sectsty}
% \allsectionsfont{\bf}
%\sectionfont{\color{Tomato3}\upshape\selectfont}
%\subsectionfont{\color{Tomato4}\upshape\selectfont}

%----- Ensembles : entiers, reels, complexes -----
\newcommand{\Nn}{\mathbb{N}} \newcommand{\N}{\mathbb{N}}
\newcommand{\Zz}{\mathbb{Z}} \newcommand{\Z}{\mathbb{Z}}
\newcommand{\Qq}{\mathbb{Q}} \newcommand{\Q}{\mathbb{Q}}
\newcommand{\Rr}{\mathbb{R}} \newcommand{\R}{\mathbb{R}}
\newcommand{\Cc}{\mathbb{C}} 
\newcommand{\Kk}{\mathbb{K}} \newcommand{\K}{\mathbb{K}}

%----- Modifications de symboles -----
\renewcommand{\epsilon}{\varepsilon}
\renewcommand{\Re}{\mathop{\text{Re}}\nolimits}
\renewcommand{\Im}{\mathop{\text{Im}}\nolimits}
%\newcommand{\llbracket}{\left[\kern-0.15em\left[}
%\newcommand{\rrbracket}{\right]\kern-0.15em\right]}

\renewcommand{\ge}{\geqslant}
\renewcommand{\geq}{\geqslant}
\renewcommand{\le}{\leqslant}
\renewcommand{\leq}{\leqslant}
\renewcommand{\epsilon}{\varepsilon}

%----- Fonctions usuelles -----
\newcommand{\ch}{\mathop{\text{ch}}\nolimits}
\newcommand{\sh}{\mathop{\text{sh}}\nolimits}
\renewcommand{\tanh}{\mathop{\text{th}}\nolimits}
\newcommand{\cotan}{\mathop{\text{cotan}}\nolimits}
\newcommand{\Arcsin}{\mathop{\text{arcsin}}\nolimits}
\newcommand{\Arccos}{\mathop{\text{arccos}}\nolimits}
\newcommand{\Arctan}{\mathop{\text{arctan}}\nolimits}
\newcommand{\Argsh}{\mathop{\text{argsh}}\nolimits}
\newcommand{\Argch}{\mathop{\text{argch}}\nolimits}
\newcommand{\Argth}{\mathop{\text{argth}}\nolimits}
\newcommand{\pgcd}{\mathop{\text{pgcd}}\nolimits} 


%----- Commandes divers ------
\newcommand{\ii}{\mathrm{i}}
\newcommand{\dd}{\text{d}}
\newcommand{\id}{\mathop{\text{id}}\nolimits}
\newcommand{\Ker}{\mathop{\text{Ker}}\nolimits}
\newcommand{\Card}{\mathop{\text{Card}}\nolimits}
\newcommand{\Vect}{\mathop{\text{Vect}}\nolimits}
\newcommand{\Mat}{\mathop{\text{Mat}}\nolimits}
\newcommand{\rg}{\mathop{\text{rg}}\nolimits}
\newcommand{\tr}{\mathop{\text{tr}}\nolimits}


%----- Structure des exercices ------

\newtheoremstyle{styleexo}% name
{2ex}% Space above
{3ex}% Space below
{}% Body font
{}% Indent amount 1
{\bfseries} % Theorem head font
{}% Punctuation after theorem head
{\newline}% Space after theorem head 2
{}% Theorem head spec (can be left empty, meaning ‘normal’)

%\theoremstyle{styleexo}
\newtheorem{exo}{Exercice}
\newtheorem{ind}{Indications}
\newtheorem{cor}{Correction}


\newcommand{\exercice}[1]{} \newcommand{\finexercice}{}
%\newcommand{\exercice}[1]{{\tiny\texttt{#1}}\vspace{-2ex}} % pour afficher le numero absolu, l'auteur...
\newcommand{\enonce}{\begin{exo}} \newcommand{\finenonce}{\end{exo}}
\newcommand{\indication}{\begin{ind}} \newcommand{\finindication}{\end{ind}}
\newcommand{\correction}{\begin{cor}} \newcommand{\fincorrection}{\end{cor}}

\newcommand{\noindication}{\stepcounter{ind}}
\newcommand{\nocorrection}{\stepcounter{cor}}

\newcommand{\fiche}[1]{} \newcommand{\finfiche}{}
\newcommand{\titre}[1]{\centerline{\large \bf #1}}
\newcommand{\addcommand}[1]{}
\newcommand{\video}[1]{}

% Marge
\newcommand{\mymargin}[1]{\marginpar{{\small #1}}}

\def\noqed{\renewcommand{\qedsymbol}{}}


%----- Presentation ------
\setlength{\parindent}{0cm}

%\newcommand{\ExoSept}{\href{http://exo7.emath.fr}{\textbf{\textsf{Exo7}}}}

\definecolor{myred}{rgb}{0.93,0.26,0}
\definecolor{myorange}{rgb}{0.97,0.58,0}
\definecolor{myyellow}{rgb}{1,0.86,0}

\newcommand{\LogoExoSept}[1]{  % input : echelle
{\usefont{U}{cmss}{bx}{n}
\begin{tikzpicture}[scale=0.1*#1,transform shape]
  \fill[color=myorange] (0,0)--(4,0)--(4,-4)--(0,-4)--cycle;
  \fill[color=myred] (0,0)--(0,3)--(-3,3)--(-3,0)--cycle;
  \fill[color=myyellow] (4,0)--(7,4)--(3,7)--(0,3)--cycle;
  \node[scale=5] at (3.5,3.5) {Exo7};
\end{tikzpicture}}
}


\newcommand{\debutmontitre}{
  \author{} \date{} 
  \thispagestyle{empty}
  \hspace*{-10ex}
  \begin{minipage}{\textwidth}
    \titlepage  
  \vspace*{-2.5cm}
  \begin{center}
    \LogoExoSept{2.5}
  \end{center}
  \end{minipage}

  \vspace*{-0cm}
  
  % Astuce pour que le background ne soit pas discrétisé lors de la conversion pdf -> png
\begin{tikzpicture}
        \fill[opacity=0,green!60!black] (0,0)--++(0,0)--++(0,0)--++(0,0)--cycle; 
\end{tikzpicture}

% toc S'affiche trop tot :
% \tableofcontents[hideallsubsections, pausesections]
}

\newcommand{\finmontitre}{
  \end{frame}
  \setcounter{framenumber}{0}
} % ne marche pas pour une raison obscure

%----- Commandes supplementaires ------

% \usepackage[landscape]{geometry}
% \geometry{top=1cm, bottom=3cm, left=2cm, right=10cm, marginparsep=1cm
% }
% \usepackage[a4paper]{geometry}
% \geometry{top=2cm, bottom=2cm, left=2cm, right=2cm, marginparsep=1cm
% }

%\usepackage{standalone}


% New command Arnaud -- november 2011
\setbeamersize{text margin left=24ex}
% si vous modifier cette valeur il faut aussi
% modifier le decalage du titre pour compenser
% (ex : ici =+10ex, titre =-5ex

\theoremstyle{definition}
%\newtheorem{proposition}{Proposition}
%\newtheorem{exemple}{Exemple}
%\newtheorem{theoreme}{Théorème}
%\newtheorem{lemme}{Lemme}
%\newtheorem{corollaire}{Corollaire}
%\newtheorem*{remarque*}{Remarque}
%\newtheorem*{miniexercice}{Mini-exercices}
%\newtheorem{definition}{Définition}

% Commande tikz
\usetikzlibrary{calc}
\usetikzlibrary{patterns,arrows}
\usetikzlibrary{matrix}
\usetikzlibrary{fadings} 

%definition d'un terme
\newcommand{\defi}[1]{{\color{myorange}\textbf{\emph{#1}}}}
\newcommand{\evidence}[1]{{\color{blue}\textbf{\emph{#1}}}}
\newcommand{\assertion}[1]{\emph{\og#1\fg}}  % pour chapitre logique
%\renewcommand{\contentsname}{Sommaire}
\renewcommand{\contentsname}{}
\setcounter{tocdepth}{2}



%------ Figures ------

\def\myscale{1} % par défaut 
\newcommand{\myfigure}[2]{  % entrée : echelle, fichier figure
\def\myscale{#1}
\begin{center}
\footnotesize
{#2}
\end{center}}


%------ Encadrement ------

\usepackage{fancybox}


\newcommand{\mybox}[1]{
\setlength{\fboxsep}{7pt}
\begin{center}
\shadowbox{#1}
\end{center}}

\newcommand{\myboxinline}[1]{
\setlength{\fboxsep}{5pt}
\raisebox{-10pt}{
\shadowbox{#1}
}
}

%--------------- Commande beamer---------------
\newcommand{\beameronly}[1]{#1} % permet de mettre des pause dans beamer pas dans poly


\setbeamertemplate{navigation symbols}{}
\setbeamertemplate{footline}  % tiré du fichier beamerouterinfolines.sty
{
  \leavevmode%
  \hbox{%
  \begin{beamercolorbox}[wd=.333333\paperwidth,ht=2.25ex,dp=1ex,center]{author in head/foot}%
    % \usebeamerfont{author in head/foot}\insertshortauthor%~~(\insertshortinstitute)
    \usebeamerfont{section in head/foot}{\bf\insertshorttitle}
  \end{beamercolorbox}%
  \begin{beamercolorbox}[wd=.333333\paperwidth,ht=2.25ex,dp=1ex,center]{title in head/foot}%
    \usebeamerfont{section in head/foot}{\bf\insertsectionhead}
  \end{beamercolorbox}%
  \begin{beamercolorbox}[wd=.333333\paperwidth,ht=2.25ex,dp=1ex,right]{date in head/foot}%
    % \usebeamerfont{date in head/foot}\insertshortdate{}\hspace*{2em}
    \insertframenumber{} / \inserttotalframenumber\hspace*{2ex} 
  \end{beamercolorbox}}%
  \vskip0pt%
}


\definecolor{mygrey}{rgb}{0.5,0.5,0.5}
\setlength{\parindent}{0cm}
%\DeclareTextFontCommand{\helvetica}{\fontfamily{phv}\selectfont}

% background beamer
\definecolor{couleurhaut}{rgb}{0.85,0.9,1}  % creme
\definecolor{couleurmilieu}{rgb}{1,1,1}  % vert pale
\definecolor{couleurbas}{rgb}{0.85,0.9,1}  % blanc
\setbeamertemplate{background canvas}[vertical shading]%
[top=couleurhaut,middle=couleurmilieu,midpoint=0.4,bottom=couleurbas] 
%[top=fondtitre!05,bottom=fondtitre!60]



\makeatletter
\setbeamertemplate{theorem begin}
{%
  \begin{\inserttheoremblockenv}
  {%
    \inserttheoremheadfont
    \inserttheoremname
    \inserttheoremnumber
    \ifx\inserttheoremaddition\@empty\else\ (\inserttheoremaddition)\fi%
    \inserttheorempunctuation
  }%
}
\setbeamertemplate{theorem end}{\end{\inserttheoremblockenv}}

\newenvironment{theoreme}[1][]{%
   \setbeamercolor{block title}{fg=structure,bg=structure!40}
   \setbeamercolor{block body}{fg=black,bg=structure!10}
   \begin{block}{{\bf Th\'eor\`eme }#1}
}{%
   \end{block}%
}


\newenvironment{proposition}[1][]{%
   \setbeamercolor{block title}{fg=structure,bg=structure!40}
   \setbeamercolor{block body}{fg=black,bg=structure!10}
   \begin{block}{{\bf Proposition }#1}
}{%
   \end{block}%
}

\newenvironment{corollaire}[1][]{%
   \setbeamercolor{block title}{fg=structure,bg=structure!40}
   \setbeamercolor{block body}{fg=black,bg=structure!10}
   \begin{block}{{\bf Corollaire }#1}
}{%
   \end{block}%
}

\newenvironment{mydefinition}[1][]{%
   \setbeamercolor{block title}{fg=structure,bg=structure!40}
   \setbeamercolor{block body}{fg=black,bg=structure!10}
   \begin{block}{{\bf Définition} #1}
}{%
   \end{block}%
}

\newenvironment{lemme}[0]{%
   \setbeamercolor{block title}{fg=structure,bg=structure!40}
   \setbeamercolor{block body}{fg=black,bg=structure!10}
   \begin{block}{\bf Lemme}
}{%
   \end{block}%
}

\newenvironment{remarque}[1][]{%
   \setbeamercolor{block title}{fg=black,bg=structure!20}
   \setbeamercolor{block body}{fg=black,bg=structure!5}
   \begin{block}{Remarque #1}
}{%
   \end{block}%
}


\newenvironment{exemple}[1][]{%
   \setbeamercolor{block title}{fg=black,bg=structure!20}
   \setbeamercolor{block body}{fg=black,bg=structure!5}
   \begin{block}{{\bf Exemple }#1}
}{%
   \end{block}%
}


\newenvironment{miniexercice}[0]{%
   \setbeamercolor{block title}{fg=structure,bg=structure!20}
   \setbeamercolor{block body}{fg=black,bg=structure!5}
   \begin{block}{Mini-exercices}
}{%
   \end{block}%
}


\newenvironment{tp}[0]{%
   \setbeamercolor{block title}{fg=structure,bg=structure!40}
   \setbeamercolor{block body}{fg=black,bg=structure!10}
   \begin{block}{\bf Travaux pratiques}
}{%
   \end{block}%
}
\newenvironment{exercicecours}[1][]{%
   \setbeamercolor{block title}{fg=structure,bg=structure!40}
   \setbeamercolor{block body}{fg=black,bg=structure!10}
   \begin{block}{{\bf Exercice }#1}
}{%
   \end{block}%
}
\newenvironment{algo}[1][]{%
   \setbeamercolor{block title}{fg=structure,bg=structure!40}
   \setbeamercolor{block body}{fg=black,bg=structure!10}
   \begin{block}{{\bf Algorithme}\hfill{\color{gray}\texttt{#1}}}
}{%
   \end{block}%
}


\setbeamertemplate{proof begin}{
   \setbeamercolor{block title}{fg=black,bg=structure!20}
   \setbeamercolor{block body}{fg=black,bg=structure!5}
   \begin{block}{{\footnotesize Démonstration}}
   \footnotesize
   \smallskip}
\setbeamertemplate{proof end}{%
   \end{block}}
\setbeamertemplate{qed symbol}{\openbox}


\makeatother
\usecolortheme[RGB={179,179,12}]{structure}

%%%%%%%%%%%%%%%%%%%%%%%%%%%%%%%%%%%%%%%%%%%%%%%%%%%%%%%%%%%%%
%%%%%%%%%%%%%%%%%%%%%%%%%%%%%%%%%%%%%%%%%%%%%%%%%%%%%%%%%%%%%


\begin{document}


\title{{\bf Développements limités}}
\subtitle{Formules de Taylor}

\begin{frame}
  
  \debutmontitre

  \pause

{\footnotesize
\hfill
\setbeamercovered{transparent=50}
\begin{minipage}{0.6\textwidth}
  \begin{itemize}
    \item<3-> Formule de Taylor avec reste intégral
    \item<4-> Formule de Taylor avec reste $f^{(n+1)}(c)$
    \item<5-> Formule de Taylor-Young
  \end{itemize}
\end{minipage}
}

\end{frame}

\setcounter{framenumber}{0}


%%%%%%%%%%%%%%%%%%%%%%%%%%%%%%%%%%%%%%%%%%%%%%%%%%%%%%%%%%%%%%%%


\section*{Motivation}


\begin{frame}

\hfil Comportement de $f(x)=\exp x$ autour de $x=0$

\bigskip

\hspace*{-25mm}
\begin{minipage}{0.62\textwidth}
\myfigure{0.95}{
\tikzinput{fig_dl01} 
}  
\end{minipage}
\hspace*{5mm}
\begin{minipage}{0.51\textwidth}
\begin{itemize}  
\uncover<3->{  \item tangente : $y=1+x$ }

\uncover<4->{  \item parabole : \only<4,5>{$y = c_0 + c_1x + c_2 x^2$ \vphantom{$\frac12x^2$}}\only<6->{$y=1+x+\frac12 x^2$} }


  \begin{itemize}

\uncover<7->{   \item $g(x)=\exp x - \big(1+x+\frac12 x^2\big)$ alors $g(0)=0$, $g'(0)=0$, $g''(0)=0$}

\uncover<8->{   \item développement limité à l'ordre $2$ }

  \end{itemize}

\uncover<9->{ \item $y = 1+x+\frac12 x^2 + \frac16 x^3$}
\end{itemize}  



\end{minipage}
\hspace*{-5mm}

\end{frame}




%---------------------------------------------------------------
\section{Trois formules}

\begin{frame}
\mybox{$\displaystyle f(x)= f(0)+f'(0)x+f''(0)\frac{x^2}{2!}+\cdots
+f^{(n)}(0)\frac{x^n}{n!} + x^n\epsilon(x)$}

\pause

\begin{itemize}
  \item le polynôme de degré $n$ qui approche le mieux la fonction
\pause

  \item autour d'une valeur fixée (ici autour de $x=0$)
\pause

  \item calculé à partir des dérivées successives (en $0$)
\pause

  \item un reste
\end{itemize}

\medskip
\pause


Trois formules de Taylor
\pause
\begin{itemize}


  \item Taylor avec reste intégral
\pause

 \item Taylor avec reste $f^{(n+1)}(c)$
\pause

  \item Taylor-Young
\end{itemize}

\medskip
\pause

$f : I \to \Rr$ est de \defi{classe $\mathcal{C}^n$}
si $f$ est $n$ fois dérivable et $f^{(n)}$ est continue


\end{frame}

%---------------------------------------------------------------
\section{Formule de Taylor avec reste intégral}

\begin{frame}

\begin{theoreme}[Formule de Taylor avec reste intégral]
Soit $f : I\to\Rr$ de classe $\mathcal{C}^{n+1}$ et soit $a,x \in I$
\mybox{\small $\begin{array}{c}\displaystyle
f(x)=f(a)+f'(a)(x-a)+\frac{f''(a)}{2!}(x-a)^2+\cdots + \qquad \qquad \qquad \\
\displaystyle\hfill +\frac{f^{(n)}(a)}{n!}(x-a)^n \pause +\int_a^x \frac{f^{(n+1)}(t)}{n!}(x-t)^ndt
\end{array}
$\pause}
\end{theoreme}
%  
% $$T_n(x) =f(a)+f'(a)(x-a)+\frac{f''(a)}{2!}(x-a)^2+\cdots
% +\frac{f^{(n)}(a)}{n!}(x-a)^n$$

\pause 

\begin{exemple}
$f(x)=\exp x$, 
\pause 
$f'(x)=\exp x$, $f''(x)=\exp x$,\ldots 

\pause

$\exp x=\exp a+\exp a \cdot (x-a)+\cdots+\frac{\exp a}{n!}(x-a)^n+\int_a^x\frac{\exp t}{n!}(x-t)^ndt$

\pause

Avec $a=0$ : $\exp x=1+x+\frac{x^2}{2!}+\frac{x^3}{3!}+\cdots $
\end{exemple}



\end{frame}



%---------------------------------------------------------------

\section{Formule de Taylor avec reste $f^{(n+1)}(c)$}

\begin{frame}
\begin{theoreme}[Formule de Taylor avec reste $f^{(n+1)}(c)$]
$f : I\to\Rr$ de classe $\mathcal{C}^{n+1}$ et soit $a,x \in I$

Il existe un réel $c$ entre $a$ et $x$ tel que 
\mybox{\small $f(x)=f(a)+f'(a)(x-a)+\cdots
+\frac{f^{(n)}(a)}{n!}(x-a)^n \pause +\frac{f^{(n+1)}(c)}{(n+1)!}(x-a)^{n+1}$\pause}
\end{theoreme}

\pause


\begin{exemple}
Soient $a,x \in\Rr$. Il existe $c$ entre $a$ et $x$ tel que 
$\exp x=\exp a+\exp a \cdot (x-a)+\cdots+\frac{\exp a}{n!}(x-a)^n+\frac{\exp c}{(n+1)!}(x-a)^{n+1}$  
\end{exemple}

\pause

\begin{remarque}
Pour $n=0$ c'est exactement l'énoncé du théorème des accroissements finis :
\pause
il existe $c\in]a,b[$ tel que $f(b)=f(a)+f'(c)(b-a)$
\end{remarque}

\end{frame}


\begin{frame}
\begin{corollaire}
Si en plus $|f^{(n+1)}|$ est majorée sur $I$ par un réel $M$, alors 
\[
\big|f(x)-T_n(x)\big|\le M\frac{|x-a|^{n+1}}{(n+1)! \ } 
\]
\end{corollaire}

\pause

\begin{exemple}[Approximation de $\sin(0,01)$]

$f(x)=\sin x$, 
\pause 
{\small 
$f'(x)=\cos x$,\! $f''(x)=-\sin x$,\! $f^{(3)}(x)=-\cos x$,\! $f^{(4)}(x)=\sin x$}

\pause

$f(0)=0$, $f'(0)=1$, $f''(0)=0$, $f^{(3)}(0)=-1$

\pause

Formule de Taylor en $a=0$ à l'ordre $3$ 
$f(x)=0+1\cdot x +0\cdot \frac{x^2}{2!}-1\frac{x^3}{3!} + f^{(4)}(c)\frac{x^4}{4!}
\pause = x -\frac{x^3}{6} + f^{(4)}(c)\frac{x^4}{24}$

\pause

Pour $x=0,01$ : $\sin(0,01) \approx 0,01 - \frac{(0,01)^3}{6}=0,00999983333\ldots$

\pause 

Précision : comme   $|f^{(4)}(c)|\le 1$ alors $\big|f(x) - \big(x -\frac{x^3}{6} \big) \big| \le \frac{x^4}{24}$

\pause

Pour $x=0,01$ : $\big|\sin(0,01) -  \big(0,01 - \frac{(0,01)^3}{6}\big)\big| \le \frac{(0,01)^4}{24} \approx 4 \cdot 10^{-10}$

\pause

\hfil\hfil $\sin(0,01) =0,\alert{00999983}\ldots$
\end{exemple}

\end{frame}



%---------------------------------------------------------------
\section{Formule de Taylor-Young}

\begin{frame}

\begin{theoreme}[Formule de Taylor-Young]
Soit $f: I \to \Rr$ de classe $\mathcal{C}^n$ et soit $a\in I$

Alors pour tout $x \in I$ on a :
\mybox{$f(x)= f(a)+f'(a)(x-a)+\cdots
+\frac{f^{(n)}(a)}{n!}(x-a)^n + (x-a)^n\epsilon(x)$}

\hfil \qquad \qquad avec $\epsilon(x) \xrightarrow[x\to a]{} 0$
\end{theoreme}

\end{frame}

%---------------------------------------------------------------
\section{Un exemple}

\begin{frame}
\ 
{\small
$f(x) = \ln(1+x)$  \qquad  \uncover<3->{$f : ]-1,+\infty[ \to \Rr$ infiniment dérivable}

\begin{itemize}
 \uncover<4->{ \item $f(0)=0$}  
\hfill \uncover<5->{$T_0(x)=0$}
 \uncover<7->{ \item $f'(x)=\frac{1}{1+x}$ donc $f'(0)=1$} 
\hfill \uncover<8->{$T_1(x) = x$}
 \uncover<10->{ \item $f''(x) = -\frac{1}{(1+x)^2}$ donc $f''(0)=-1$} 
\hfill \uncover<11->{$T_2(x) = x -\frac{x^2}{2}$}
 \uncover<13->{ \item $f^{(3)}(x)= + 2 \frac{1}{(1+x)^3}$ donc $f^{(3)}(0)= + 2$} 
\hfill \uncover<14->{$T_3(x) = x -\frac{x^2}{2} + \frac{x^3}{3}$}
\end{itemize}
}

\uncover<2->{
\myfigure{0.9}{\tikzinput{fig_dl02}}
}

\end{frame}


\begin{frame}

$f(x) = \ln(1+x)$ \quad $n > 0$

\pause
\medskip

\begin{itemize}
  \item $f'(x)=\frac{1}{1+x}$

\pause
\medskip

  \item $f^{(n)}(x) = (-1)^{n-1} (n-1)!\frac{1}{(1+x)^n}$ donc $f^{(n)}(0)= (-1)^{n-1} (n-1)!$

\pause
\medskip

  \item Ainsi pour $n>0$ : $\frac{f^{(n)}(0)}{n!}x^n = (-1)^{n-1}\frac{x^n}{n}$

\pause
\medskip

  \item $\displaystyle T_n(x)=\sum_{k=1}^n (-1)^{k-1}\frac{x^k}{k} = x-\frac{x^2}{2} + \frac{x^3}{3}-\cdots + (-1)^{n-1}\frac{x^n}{n}$

\end{itemize}


\end{frame}



%---------------------------------------------------------------

\section{Résumé}

\begin{frame}

\textbf{Trois formules de Taylor} \qquad $f(x) = T_n(x) + R_n(x)$
\pause
$$T_n(x) =f(a)+f'(a)(x-a)+\frac{f''(a)}{2!}(x-a)^2+\cdots
+\frac{f^{(n)}(a)}{n!}(x-a)^n$$
\pause
$$
\begin{array}{rcl@{\vrule depth 3ex height 4ex width 0mm \ }c}
\hline
R_n(x) & = & \displaystyle \int_a^x \tfrac{f^{(n+1)}(t)}{n!}(x-t)^ndt 
 & {\footnotesize \text{Taylor avec reste intégral}} 
 \\ \hline \pause
R_n(x) & = &\frac{f^{(n+1)}(c)}{(n+1)!}(x-a)^{n+1}   
 & {\footnotesize \begin{minipage}{0.4\textwidth}\center Taylor avec reste  $f^{(n+1)}(c)$ \\
 $c$ entre $a$ et $x$ \end{minipage}}
 \\ \hline\pause
R_n(x) & = & (x-a)^n\epsilon(x) 
 & {\footnotesize \text{Taylor-Young avec } \epsilon(x) \xrightarrow[x\to a]{} 0} \\  \hline
\end{array}
$$

\pause 

\begin{remarque}
En posant $x=a+h$ (et donc $h=x-a$) et avec $\epsilon(h) \xrightarrow[h\to 0]{} 0$

\hfil $f(a+h)=f(a)+f'(a)h+\frac{f''(a)}{2!}h^2+\cdots
+\frac{f^{(n)}(a)}{n!}h^n+h^n \epsilon(h)$
\end{remarque}

\end{frame}




\begin{frame}


\evidence{Formule de Taylor-Young au voisinage de $0$}


\mybox{$\displaystyle f(x)= f(0)+f'(0)x+f''(0)\frac{x^2}{2!}+\cdots
+f^{(n)}(0)\frac{x^n}{n!} + x^n\epsilon(x)$}
où \ $\lim_{x\to0}\epsilon (x)=0$

\pause

\begin{remarque}[: notation \og petit o \fg]
\hfil $x^n\epsilon(x)$ où $\epsilon(x) \xrightarrow[x\to 0]{} 0$  \quad 
se note \quad $o\big( x^n\big)$

\pause

$$f(x)= f(0)+f'(0)x+f''(0)\frac{x^2}{2!}+\cdots
+f^{(n)}(0)\frac{x^n}{n!} + o(x^n)$$
\end{remarque}

\end{frame}



%---------------------------------------------------------------
\section{Preuve de la formule de Taylor avec reste intégral}


\begin{frame}

\vspace*{-1ex}

\textbf{Preuve de la formule de Taylor \uncover<3->{par récurrence sur $n$}}

\pause

{\small 
\[
f(b)=f(a)+f'(a)(b-a)+\cdots
+\frac{f^{(n)}(a)}{n!}(b-a)^k+\int_a^b f^{(n+1)}(t)\frac{(b-t)^n}{n!}dt
\]

\pause

\pause


\textbf{Initialisation.} $n=0$ 
$\int_a^b f'(t) \, dt=f(b)-f(a)$ donc 
$f(b)=f(a)+\int_a^b f'(t) \, dt$

\pause
\medskip


\textbf{Hérédité.}
Supposons la formule vraie au rang $n-1$
 
$f(b)=f(a)+f'(a)(b-a)+\cdots
+\frac{f^{(n-1)}(a)}{(n-1)!}(b-a)^{n-1}
+\int_a^bf^{(n)}(t)\frac{(b-t)^{n-1}}{(n-1)!}dt$

\pause

Intégration par parties $u(t)=f^{(n)}(t)$ et $v'(t) = \frac{(b-t)^{n-1}}{(n-1)!}$

\pause

\hfill et donc  $u'(t)= f^{(n+1)}(t)$ et $v(t) = - \frac{(b-t)^{n}}{n!}$

\pause
\medskip

$
\begin{array}{rcl}
\displaystyle \int_a^b f^{(n)}(t) \tfrac{(b-t)^{n-1}}{(n-1)!} \, dt 
\pause
  & = & \displaystyle \left[-f^{(n)}(t)\tfrac{(b-t)^{n}}{n!}\right]_a^b + \int_a^b f^{(n+1)}(t) \tfrac{(b-t)^{n}}{n!} \, dt \\
\pause
  & = & \displaystyle f^{(n)}(a)\tfrac{(b-a)^{n}}{n!} + \int_a^b f^{(n+1)}(t) \tfrac{(b-t)^{n}}{n!}\, dt \\
\end{array}
$  



\pause
\medskip

\textbf{Conclusion.}\! La formule est vraie pour tous les $n$ pour lesquels $f$ est classe\! $\mathcal{C}^{n+1}$ 

}

\end{frame}

%%%%%%%%%%%%%%%%%%%%%%%%%%%%%%%%%%%%%%%%%%%%%%%%%%%%%%%%%%%%%%%%
\section{Mini-exercices}

\begin{frame}

\begin{miniexercice}
\begin{enumerate}
  \item Écrire les trois formules de Taylor en $0$ pour $x \mapsto \cos x$, $x\mapsto \exp(-x)$ et $x\mapsto \sh x$.
  \item Écrire les formules de Taylor en $0$ à l'ordre $2$ pour $x \mapsto \frac{1}{\sqrt{1+x}}$, $x\mapsto \tan x$.
  \item Écrire les formules de Taylor en $1$ pour $x \mapsto x^3-9x^2+14x+3$.
  \item Avec une formule de Taylor à l'ordre $2$ de $\sqrt{1+x}$, trouver une approximation 
de $\sqrt{1,01}$. Idem avec $\ln(0,99)$.
\end{enumerate}
\end{miniexercice}

\end{frame}

\end{document}