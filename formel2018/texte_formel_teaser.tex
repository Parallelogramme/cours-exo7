
%%%%%%%%%%%%%%%%%% PREAMBULE %%%%%%%%%%%%%%%%%%


\documentclass[12pt]{article}

\usepackage{amsfonts,amsmath,amssymb,amsthm}
\usepackage[utf8]{inputenc}
\usepackage[T1]{fontenc}
\usepackage[francais]{babel}


% packages
\usepackage{amsfonts,amsmath,amssymb,amsthm}
\usepackage[utf8]{inputenc}
\usepackage[T1]{fontenc}
%\usepackage{lmodern}

\usepackage[francais]{babel}
\usepackage{fancybox}
\usepackage{graphicx}

\usepackage{float}

%\usepackage[usenames, x11names]{xcolor}
\usepackage{tikz}
\usepackage{datetime}

\usepackage{mathptmx}
%\usepackage{fouriernc}
%\usepackage{newcent}
\usepackage[mathcal,mathbf]{euler}

%\usepackage{palatino}
%\usepackage{newcent}


% Commande spéciale prompteur

%\usepackage{mathptmx}
%\usepackage[mathcal,mathbf]{euler}
%\usepackage{mathpple,multido}

\usepackage[a4paper]{geometry}
\geometry{top=2cm, bottom=2cm, left=1cm, right=1cm, marginparsep=1cm}

\newcommand{\change}{{\color{red}\rule{\textwidth}{1mm}\\}}

\newcounter{mydiapo}

\newcommand{\diapo}{\newpage
\hfill {\normalsize  Diapo \themydiapo \quad \texttt{[\jobname]}} \\
\stepcounter{mydiapo}}


%%%%%%% COULEURS %%%%%%%%%%

% Pour blanc sur noir :
%\pagecolor[rgb]{0.5,0.5,0.5}
% \pagecolor[rgb]{0,0,0}
% \color[rgb]{1,1,1}



%\DeclareFixedFont{\myfont}{U}{cmss}{bx}{n}{18pt}
\newcommand{\debuttexte}{
%%%%%%%%%%%%% FONTES %%%%%%%%%%%%%
\renewcommand{\baselinestretch}{1.5}
\usefont{U}{cmss}{bx}{n}
\bfseries

% Taille normale : commenter le reste !
%Taille Arnaud
%\fontsize{19}{19}\selectfont

% Taille Barbara
%\fontsize{21}{22}\selectfont

%Taille François
\fontsize{25}{30}\selectfont

%Taille Pascal
%\fontsize{25}{30}\selectfont

%Taille Laura
%\fontsize{30}{35}\selectfont


%\myfont
%\usefont{U}{cmss}{bx}{n}

%\Huge
%\addtolength{\parskip}{\baselineskip}
}


% \usepackage{hyperref}
% \hypersetup{colorlinks=true, linkcolor=blue, urlcolor=blue,
% pdftitle={Exo7 - Exercices de mathématiques}, pdfauthor={Exo7}}


%section
% \usepackage{sectsty}
% \allsectionsfont{\bf}
%\sectionfont{\color{Tomato3}\upshape\selectfont}
%\subsectionfont{\color{Tomato4}\upshape\selectfont}

%----- Ensembles : entiers, reels, complexes -----
\newcommand{\Nn}{\mathbb{N}} \newcommand{\N}{\mathbb{N}}
\newcommand{\Zz}{\mathbb{Z}} \newcommand{\Z}{\mathbb{Z}}
\newcommand{\Qq}{\mathbb{Q}} \newcommand{\Q}{\mathbb{Q}}
\newcommand{\Rr}{\mathbb{R}} \newcommand{\R}{\mathbb{R}}
\newcommand{\Cc}{\mathbb{C}} 
\newcommand{\Kk}{\mathbb{K}} \newcommand{\K}{\mathbb{K}}

%----- Modifications de symboles -----
\renewcommand{\epsilon}{\varepsilon}
\renewcommand{\Re}{\mathop{\text{Re}}\nolimits}
\renewcommand{\Im}{\mathop{\text{Im}}\nolimits}
%\newcommand{\llbracket}{\left[\kern-0.15em\left[}
%\newcommand{\rrbracket}{\right]\kern-0.15em\right]}

\renewcommand{\ge}{\geqslant}
\renewcommand{\geq}{\geqslant}
\renewcommand{\le}{\leqslant}
\renewcommand{\leq}{\leqslant}

%----- Fonctions usuelles -----
\newcommand{\ch}{\mathop{\mathrm{ch}}\nolimits}
\newcommand{\sh}{\mathop{\mathrm{sh}}\nolimits}
\renewcommand{\tanh}{\mathop{\mathrm{th}}\nolimits}
\newcommand{\cotan}{\mathop{\mathrm{cotan}}\nolimits}
\newcommand{\Arcsin}{\mathop{\mathrm{Arcsin}}\nolimits}
\newcommand{\Arccos}{\mathop{\mathrm{Arccos}}\nolimits}
\newcommand{\Arctan}{\mathop{\mathrm{Arctan}}\nolimits}
\newcommand{\Argsh}{\mathop{\mathrm{Argsh}}\nolimits}
\newcommand{\Argch}{\mathop{\mathrm{Argch}}\nolimits}
\newcommand{\Argth}{\mathop{\mathrm{Argth}}\nolimits}
\newcommand{\pgcd}{\mathop{\mathrm{pgcd}}\nolimits} 

\newcommand{\Card}{\mathop{\text{Card}}\nolimits}
\newcommand{\Ker}{\mathop{\text{Ker}}\nolimits}
\newcommand{\id}{\mathop{\text{id}}\nolimits}
\newcommand{\ii}{\mathrm{i}}
\newcommand{\dd}{\mathrm{d}}
\newcommand{\Vect}{\mathop{\text{Vect}}\nolimits}
\newcommand{\Mat}{\mathop{\mathrm{Mat}}\nolimits}
\newcommand{\rg}{\mathop{\text{rg}}\nolimits}
\newcommand{\tr}{\mathop{\text{tr}}\nolimits}
\newcommand{\ppcm}{\mathop{\text{ppcm}}\nolimits}

%----- Structure des exercices ------

\newtheoremstyle{styleexo}% name
{2ex}% Space above
{3ex}% Space below
{}% Body font
{}% Indent amount 1
{\bfseries} % Theorem head font
{}% Punctuation after theorem head
{\newline}% Space after theorem head 2
{}% Theorem head spec (can be left empty, meaning ‘normal’)

%\theoremstyle{styleexo}
\newtheorem{exo}{Exercice}
\newtheorem{ind}{Indications}
\newtheorem{cor}{Correction}


\newcommand{\exercice}[1]{} \newcommand{\finexercice}{}
%\newcommand{\exercice}[1]{{\tiny\texttt{#1}}\vspace{-2ex}} % pour afficher le numero absolu, l'auteur...
\newcommand{\enonce}{\begin{exo}} \newcommand{\finenonce}{\end{exo}}
\newcommand{\indication}{\begin{ind}} \newcommand{\finindication}{\end{ind}}
\newcommand{\correction}{\begin{cor}} \newcommand{\fincorrection}{\end{cor}}

\newcommand{\noindication}{\stepcounter{ind}}
\newcommand{\nocorrection}{\stepcounter{cor}}

\newcommand{\fiche}[1]{} \newcommand{\finfiche}{}
\newcommand{\titre}[1]{\centerline{\large \bf #1}}
\newcommand{\addcommand}[1]{}
\newcommand{\video}[1]{}

% Marge
\newcommand{\mymargin}[1]{\marginpar{{\small #1}}}



%----- Presentation ------
\setlength{\parindent}{0cm}

%\newcommand{\ExoSept}{\href{http://exo7.emath.fr}{\textbf{\textsf{Exo7}}}}

\definecolor{myred}{rgb}{0.93,0.26,0}
\definecolor{myorange}{rgb}{0.97,0.58,0}
\definecolor{myyellow}{rgb}{1,0.86,0}

\newcommand{\LogoExoSept}[1]{  % input : echelle
{\usefont{U}{cmss}{bx}{n}
\begin{tikzpicture}[scale=0.1*#1,transform shape]
  \fill[color=myorange] (0,0)--(4,0)--(4,-4)--(0,-4)--cycle;
  \fill[color=myred] (0,0)--(0,3)--(-3,3)--(-3,0)--cycle;
  \fill[color=myyellow] (4,0)--(7,4)--(3,7)--(0,3)--cycle;
  \node[scale=5] at (3.5,3.5) {Exo7};
\end{tikzpicture}}
}



\theoremstyle{definition}
%\newtheorem{proposition}{Proposition}
%\newtheorem{exemple}{Exemple}
%\newtheorem{theoreme}{Théorème}
\newtheorem{lemme}{Lemme}
\newtheorem{corollaire}{Corollaire}
%\newtheorem*{remarque*}{Remarque}
%\newtheorem*{miniexercice}{Mini-exercices}
%\newtheorem{definition}{Définition}




%definition d'un terme
\newcommand{\defi}[1]{{\color{myorange}\textbf{\emph{#1}}}}
\newcommand{\evidence}[1]{{\color{blue}\textbf{\emph{#1}}}}



 %----- Commandes divers ------

\newcommand{\codeinline}[1]{\texttt{#1}}

%%%%%%%%%%%%%%%%%%%%%%%%%%%%%%%%%%%%%%%%%%%%%%%%%%%%%%%%%%%%%
%%%%%%%%%%%%%%%%%%%%%%%%%%%%%%%%%%%%%%%%%%%%%%%%%%%%%%%%%%%%%

\newcommand{\un}[1]{[Arnaud] {\color{blue}{#1}}}
\newcommand{\deux}[1]{[François] {\color{red}{#1}}}
\newcommand{\trois}[1]{[Niels] {\color{orange}{#1}}}


\begin{document}

\debuttexte




%%%%%%%%%%%%%%%%%%%%%%%%%%%%%%%%%%%%%%%%%%%%%%%%%%%%%%%%%%%
\diapo


L'ordinateur peut-il nous aider à visualiser, à réfléchir, voir à résoudre des problèmes mathématiques ?

Oui, notamment par l'intermédiaire du calcul numérique.

\change
Mais le calcul formel apporte beaucoup plus, il permet : 

de manipuler des objets mathématiques abstraits, 

de faire du calcul algébrique, 

de rechercher des formules 

de compléter ou même de faire des démonstrations.

Un exemple assez récent est la preuve du théorème des 4 couleurs !

\change
Le logiciel choisit pour ce cours est Sage.

Il est gratuit et de prise en main facile.

Avec le calcul formel, nous pratiquerons les mathématiques de façon à la fois théorique et concrète.

En travaillant les mathématiques autrement, vous comprendrez mieux les notions fondamentales abordées.

A la fin du cours, vous serez capables de demander à la machine de résoudre bon nombre d'exercices de première année de cycle universitaire.

Chaque semaine sera consacrée à un thème comme par exemple :

\change
- les suites,

\change
- les espaces vectoriels et les matrices,

\change
- les courbes et les surfaces,

\change
- les polynômes,

\change
- les intégrales,

bien d'autres choses encore...

et plusieurs exercices vous seront proposés à la fin de chaque thème ! 

\change
 
Il ne s'agit pas d'un cours de mathématiques classique : le but est de vous faire manipuler des objets mathématiques, 
de mener des expériences, en adoptant un point de vue formel.

\change
Nul besoin pour cela d'avoir déjà des connaissances en calcul formel, nous introduirons pas à pas les commandes de l'outil Sage. 

\change
Il est tout de fois conseillé d'avoir déjà quelques notions d'algorithmique :

\change
par exemple savoir programmer une boucle ou définir une fonction.

\change

\change
Ce cours de 7 semaines est proposé par l'université de Lille.

Il s'adresse aux étudiants de première année de licence.

\change
Le travail se fera avec des vidéos,

\change
un polycopié clair et précis,

\change
des travaux pratiques sous forme d'énigmes à résoudre,

\change
et bien sûr des professeurs pour répondre à vos questions !

A très bientôt !

\end{document}


















%%%%%%%%%%%%%%%%%%%%%%%%%%%%%%%%%%%%%%%%%%%%%%%%%%%%%%%%%%%
\diapo


L'ordinateur peut-il nous aider à résoudre des problèmes mathématiques ?

Oui comme c'est la cas pour la preuve du théorème des 4 couleurs !

Le calcul formel manipule des objets mathématiques abstraits, 

il permet de faire du calcul algébrique, 

de deviner des formules 

%et même d'en démontrer quelques unes.
et même de faire des démonstrations.

\change


%Oui grâce au calcul formel ! 

%Le théorème des 4 couleurs affirme que l'on peut colorier
%n'importe quelle carte avec seulement 4 couleurs, sans que deux pays voisins ait la même couleur ?
%La preuve nécessite l'utilisation de l'ordinateur pour venir à bout de la démonstration.

%Oui l'ordinateur peut nous aider à résoudre des problèmes de maths grâce au calcul formel !


\change

%Vous utiliserez le logiciel gratuit Sage 

Le logiciel choisit pour ce cours est Sage.

Il est gratuit et de prise en main facile.

%Vous apprendrez à utiliser le logiciel gratuit Sage.

Avec le calcul formel, nous pratiquerons les maths de façon concrète et ludique.

En travaillant les mathématiques autrement vous comprendrez mieux les notions abordées.

A la fin du cours la machine vous aidera à résoudre bon nombre d'exercices de première année.

%
%
%Faire des maths autrement, améliora votre comprendre 
%
%qui vous permettra d'adopter une démarche expérimentale et vous aidera dans les démonstrations. 
%
%Une bonne compréhension des mathématiques et une utilisation juste et efficace de la machine 
%

\change

Chaque semaine sera consacrée à un thème comme par exemple :

- les suites,

\change

- les matrices et les espaces vectoriels,

\change

- les courbes et les surfaces,

\change

- les polynômes,

\change


- les intégrales,

et bien d'autres choses encore...

\change


Chaque semaine vous aurez aussi plusieurs exercices à rendre ! 

\change
 
Il ne s'agit pas d'un cours de maths classique : le but est de vous faire manipuler des objets mathématiques, 
d'expérimenter, en adoptant le point de vue du calcul formel.

\change

Nous 
%vous
introduirons pas à pas le logiciel Sage. 

\change

Nul besoin de connaître le calcul formel, mais
il est quand même préférable d'avoir déjà des notions d'algorithmique :
par exemple savoir programmer une boucle ou définir une fonction.



\change


\change

Ce cours de six semaines est proposé par l'université de Lille.

Il s'adresse aux étudiants de première et aussi de deuxième année de licence.

\change

Vous travaillerez avec des vidéos,

\change

un polycopié clair et précis,


\change

des travaux pratiques sous forme d'énigmes à résoudre.


\change

Et bien sûr des professeurs pour répondre à vos questions !

A très bientôt !


\end{document}