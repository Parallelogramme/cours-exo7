
%%%%%%%%%%%%%%%%%% PREAMBULE %%%%%%%%%%%%%%%%%%


\documentclass[12pt]{article}

\usepackage{amsfonts,amsmath,amssymb,amsthm}
\usepackage[utf8]{inputenc}
\usepackage[T1]{fontenc}
\usepackage[francais]{babel}


% packages
\usepackage{amsfonts,amsmath,amssymb,amsthm}
\usepackage[utf8]{inputenc}
\usepackage[T1]{fontenc}
%\usepackage{lmodern}

\usepackage[francais]{babel}
\usepackage{fancybox}
\usepackage{graphicx}

\usepackage{float}

%\usepackage[usenames, x11names]{xcolor}
\usepackage{tikz}
\usepackage{datetime}

\usepackage{mathptmx}
%\usepackage{fouriernc}
%\usepackage{newcent}
\usepackage[mathcal,mathbf]{euler}

%\usepackage{palatino}
%\usepackage{newcent}


% Commande spéciale prompteur

%\usepackage{mathptmx}
%\usepackage[mathcal,mathbf]{euler}
%\usepackage{mathpple,multido}

\usepackage[a4paper]{geometry}
\geometry{top=2cm, bottom=2cm, left=1cm, right=1cm, marginparsep=1cm}

\newcommand{\change}{{\color{red}\rule{\textwidth}{1mm}\\}}

\newcounter{mydiapo}

\newcommand{\diapo}{\newpage
\hfill {\normalsize  Diapo \themydiapo \quad \texttt{[\jobname]}} \\
\stepcounter{mydiapo}}


%%%%%%% COULEURS %%%%%%%%%%

% Pour blanc sur noir :
%\pagecolor[rgb]{0.5,0.5,0.5}
% \pagecolor[rgb]{0,0,0}
% \color[rgb]{1,1,1}



%\DeclareFixedFont{\myfont}{U}{cmss}{bx}{n}{18pt}
\newcommand{\debuttexte}{
%%%%%%%%%%%%% FONTES %%%%%%%%%%%%%
\renewcommand{\baselinestretch}{1.5}
\usefont{U}{cmss}{bx}{n}
\bfseries

% Taille normale : commenter le reste !
%Taille Arnaud
%\fontsize{19}{19}\selectfont

% Taille Barbara
%\fontsize{21}{22}\selectfont

%Taille François
\fontsize{25}{30}\selectfont

%Taille Pascal
%\fontsize{25}{30}\selectfont

%Taille Laura
%\fontsize{30}{35}\selectfont


%\myfont
%\usefont{U}{cmss}{bx}{n}

%\Huge
%\addtolength{\parskip}{\baselineskip}
}


% \usepackage{hyperref}
% \hypersetup{colorlinks=true, linkcolor=blue, urlcolor=blue,
% pdftitle={Exo7 - Exercices de mathématiques}, pdfauthor={Exo7}}


%section
% \usepackage{sectsty}
% \allsectionsfont{\bf}
%\sectionfont{\color{Tomato3}\upshape\selectfont}
%\subsectionfont{\color{Tomato4}\upshape\selectfont}

%----- Ensembles : entiers, reels, complexes -----
\newcommand{\Nn}{\mathbb{N}} \newcommand{\N}{\mathbb{N}}
\newcommand{\Zz}{\mathbb{Z}} \newcommand{\Z}{\mathbb{Z}}
\newcommand{\Qq}{\mathbb{Q}} \newcommand{\Q}{\mathbb{Q}}
\newcommand{\Rr}{\mathbb{R}} \newcommand{\R}{\mathbb{R}}
\newcommand{\Cc}{\mathbb{C}} 
\newcommand{\Kk}{\mathbb{K}} \newcommand{\K}{\mathbb{K}}

%----- Modifications de symboles -----
\renewcommand{\epsilon}{\varepsilon}
\renewcommand{\Re}{\mathop{\text{Re}}\nolimits}
\renewcommand{\Im}{\mathop{\text{Im}}\nolimits}
%\newcommand{\llbracket}{\left[\kern-0.15em\left[}
%\newcommand{\rrbracket}{\right]\kern-0.15em\right]}

\renewcommand{\ge}{\geqslant}
\renewcommand{\geq}{\geqslant}
\renewcommand{\le}{\leqslant}
\renewcommand{\leq}{\leqslant}

%----- Fonctions usuelles -----
\newcommand{\ch}{\mathop{\mathrm{ch}}\nolimits}
\newcommand{\sh}{\mathop{\mathrm{sh}}\nolimits}
\renewcommand{\tanh}{\mathop{\mathrm{th}}\nolimits}
\newcommand{\cotan}{\mathop{\mathrm{cotan}}\nolimits}
\newcommand{\Arcsin}{\mathop{\mathrm{Arcsin}}\nolimits}
\newcommand{\Arccos}{\mathop{\mathrm{Arccos}}\nolimits}
\newcommand{\Arctan}{\mathop{\mathrm{Arctan}}\nolimits}
\newcommand{\Argsh}{\mathop{\mathrm{Argsh}}\nolimits}
\newcommand{\Argch}{\mathop{\mathrm{Argch}}\nolimits}
\newcommand{\Argth}{\mathop{\mathrm{Argth}}\nolimits}
\newcommand{\pgcd}{\mathop{\mathrm{pgcd}}\nolimits} 

\newcommand{\Card}{\mathop{\text{Card}}\nolimits}
\newcommand{\Ker}{\mathop{\text{Ker}}\nolimits}
\newcommand{\id}{\mathop{\text{id}}\nolimits}
\newcommand{\ii}{\mathrm{i}}
\newcommand{\dd}{\mathrm{d}}
\newcommand{\Vect}{\mathop{\text{Vect}}\nolimits}
\newcommand{\Mat}{\mathop{\mathrm{Mat}}\nolimits}
\newcommand{\rg}{\mathop{\text{rg}}\nolimits}
\newcommand{\tr}{\mathop{\text{tr}}\nolimits}
\newcommand{\ppcm}{\mathop{\text{ppcm}}\nolimits}

%----- Structure des exercices ------

\newtheoremstyle{styleexo}% name
{2ex}% Space above
{3ex}% Space below
{}% Body font
{}% Indent amount 1
{\bfseries} % Theorem head font
{}% Punctuation after theorem head
{\newline}% Space after theorem head 2
{}% Theorem head spec (can be left empty, meaning ‘normal’)

%\theoremstyle{styleexo}
\newtheorem{exo}{Exercice}
\newtheorem{ind}{Indications}
\newtheorem{cor}{Correction}


\newcommand{\exercice}[1]{} \newcommand{\finexercice}{}
%\newcommand{\exercice}[1]{{\tiny\texttt{#1}}\vspace{-2ex}} % pour afficher le numero absolu, l'auteur...
\newcommand{\enonce}{\begin{exo}} \newcommand{\finenonce}{\end{exo}}
\newcommand{\indication}{\begin{ind}} \newcommand{\finindication}{\end{ind}}
\newcommand{\correction}{\begin{cor}} \newcommand{\fincorrection}{\end{cor}}

\newcommand{\noindication}{\stepcounter{ind}}
\newcommand{\nocorrection}{\stepcounter{cor}}

\newcommand{\fiche}[1]{} \newcommand{\finfiche}{}
\newcommand{\titre}[1]{\centerline{\large \bf #1}}
\newcommand{\addcommand}[1]{}
\newcommand{\video}[1]{}

% Marge
\newcommand{\mymargin}[1]{\marginpar{{\small #1}}}



%----- Presentation ------
\setlength{\parindent}{0cm}

%\newcommand{\ExoSept}{\href{http://exo7.emath.fr}{\textbf{\textsf{Exo7}}}}

\definecolor{myred}{rgb}{0.93,0.26,0}
\definecolor{myorange}{rgb}{0.97,0.58,0}
\definecolor{myyellow}{rgb}{1,0.86,0}

\newcommand{\LogoExoSept}[1]{  % input : echelle
{\usefont{U}{cmss}{bx}{n}
\begin{tikzpicture}[scale=0.1*#1,transform shape]
  \fill[color=myorange] (0,0)--(4,0)--(4,-4)--(0,-4)--cycle;
  \fill[color=myred] (0,0)--(0,3)--(-3,3)--(-3,0)--cycle;
  \fill[color=myyellow] (4,0)--(7,4)--(3,7)--(0,3)--cycle;
  \node[scale=5] at (3.5,3.5) {Exo7};
\end{tikzpicture}}
}



\theoremstyle{definition}
%\newtheorem{proposition}{Proposition}
%\newtheorem{exemple}{Exemple}
%\newtheorem{theoreme}{Théorème}
\newtheorem{lemme}{Lemme}
\newtheorem{corollaire}{Corollaire}
%\newtheorem*{remarque*}{Remarque}
%\newtheorem*{miniexercice}{Mini-exercices}
%\newtheorem{definition}{Définition}




%definition d'un terme
\newcommand{\defi}[1]{{\color{myorange}\textbf{\emph{#1}}}}
\newcommand{\evidence}[1]{{\color{blue}\textbf{\emph{#1}}}}



 %----- Commandes divers ------

\newcommand{\codeinline}[1]{\texttt{#1}}
\newcommand{\Sage}{\texttt{Sage}}
%%%%%%%%%%%%%%%%%%%%%%%%%%%%%%%%%%%%%%%%%%%%%%%%%%%%%%%%%%%%%
%%%%%%%%%%%%%%%%%%%%%%%%%%%%%%%%%%%%%%%%%%%%%%%%%%%%%%%%%%%%%


\begin{document}

\debuttexte


%%%%%%%%%%%%%%%%%%%%%%%%%%%%%%%%%%%%%%%%%%%%%%%%%%%%%%%%%%%
\diapo

Dans cette partie, nous allons voir comment calculer des intégrales avec Sage.

\change

Voici le plan que nous allons suivre :

\change

nous commencerons par donner les commandes de base

\change

nous calculerons des intégrales par intégration par parties

\change

puis nous verrons comment le calcul formel permet aussi le changement de variable.

%%%%%%%%%%%%%%%%%%%%%%%%%%%%%%%%%%%%%%%%%%%%%%%%%%%%%%%%%%%
\diapo

Dans ce premier travail, calculons des primitives, 
à la main puis à l'aide de commandes de Sage.


%%%%%%%%%%%%%%%%%%%%%%%%%%%%%%%%%%%%%%%%%%%%%%%%%%%%%%%%%%%
\diapo


\change

Voici comment faire avec Sage. La commande
\codeinline{integral(f(x),x)} renvoie une primitive de $f$.

\change
Comme ici :
$\displaystyle\int x^4 + \frac{1}{x^2} + \exp(x)+ \cos(x) \; \dd x 
  = \frac15 x^5 - \frac1x + \exp(x) + \sin(x)$

\change

\change

\change

Même chose ici : $\displaystyle\int  x \sin(x^2) \; \dd x 
  = - \frac12\cos(x^2)$.
  
La machine renvoie une seule primitive. Pour avoir l'ensemble des primitives, il faut
bien entendu ajouter une constante.

\change

\change

\change

On peut également intégrer des fonctions avec des paramètres.

\change
Souvent, pour intégrer des fractions rationnelles il est préférable d'écrire la décomposition en éléments simples.



\change  
Cela se fait avec la commande \codeinline{partialfraction}. Par exemple pour
$f = x^4/(x^2-1)$ la commande \codeinline{f.partialfraction(x)} 
renvoie la décomposition en éléments simples: %sur $\Qq$.
$x^2 + 1 - \frac12\frac{1}{x + 1} + \frac12\frac{1}{x - 1}$.

\change

\change
Sous cette forme on sait calculer la primitive, ce que Sage confirme.

%%%%%%%%%%%%%%%%%%%%%%%%%%%%%%%%%%%%%%%%%%%%%%%%%%%%%%%%%%%
\diapo

Dans ce deuxième travail, nous passons au calcul d'intégrales. 

\change
%Voici la réponse, qui est très semblable à la précédente :

La commande
\codeinline{integral(f(x),x,a,b)} renvoie l'intégrale de la fonction $f$ de variable $x$ entre les bornes $a$ et $b$. 

Il suffit donc d'ajouter par rapport à ce qui précède les bornes en fin de commande. Ici on intègre pour $x$ allant de $1$ à $3$.

\change
L'intégrale est bien sûr un nombre et non plus une fonction comme auparavant pour les primitives.

\change
Aucun problème pour les autres intégrales, comme celle ci

\change
ou celle là.


%%%%%%%%%%%%%%%%%%%%%%%%%%%%%%%%%%%%%%%%%%%%%%%%%%%%%%%%%%%
\diapo

Rappelons la formule d'intégration par parties.

$$\int_a^b u(x) \, v'(x)\;\dd x= \big[uv\big]_a^b - \int_a^b u'(x) \, v(x)\;\dd x$$

Cette formule est aussi valable pour les primitives en "oubliant" les bornes $a$ et $b$.


%%%%%%%%%%%%%%%%%%%%%%%%%%%%%%%%%%%%%%%%%%%%%%%%%%%%%%%%%%%
\diapo

Dans ces travaux pratiques, nous allons nous intéresser aux primitives de la fonction $x^n\exp(x)$, pour un entier $n$ quelconque.

Le but est d'obtenir une formule close.
 Nous trouverons plusieurs approches. 
% en essayant d'abord d'obtenir  directement une formule.

Nous tenterons ensuite de conjecturer puis de prouver une relation de récurrence pour $F_n$ en suivant la méthode vue dans les séances précédentes.

Dans cette dernière question, nous obtiendrons une formule directe pour $F_n$, en suivant les mêmes étapes qu'à la question précédente.



%%%%%%%%%%%%%%%%%%%%%%%%%%%%%%%%%%%%%%%%%%%%%%%%%%%%%%%%%%%
\diapo

[granf $F$/petit $f$]


\change
La commande directe \codeinline{integral(f,x)}
  renvoie un résultat mystérieux contenant une \emph{fonction spéciale} $\Gamma$.  

Ceci ne nous aide guère.% pour calculer cette primitive.

\change

Par contre, il n'y a aucun problème pour calculer les primitives $F_n$ 
  pour les premières valeurs de $n$. 
  
Voici les quelques lignes de code qui :

- définissent les variables,

- la fonction à intégrer,

- la fonction [grand] $F(n)$ qui calcule une primitive pour un entier $n$ donné.

- et enfin une boucle qui appelle cette fonction et affiche chaque primitive
pour des valeurs de $n$ variant de $0$ à $9$.

\change

Voici les premières primitives.

%%%%%%%%%%%%%%%%%%%%%%%%%%%%%%%%%%%%%%%%%%%%%%%%%%%%%%%%%%%
\diapo

\change

Au vu des premiers termes, on conjecture la relation 
$F_n(x) = x^n \exp(x) - nF_{n-1}(x)$.

On vérifie facilement à l'aide d'une boucle que cette conjecture 
est vraie pour les premières valeurs de $n$.

\change    
    Par contre, Sage ne sait pas vérifier formellement cette identité 
    (c'est-à-dire lorsque $n$ est une variable formelle).

\change
Nous prouvons donc à la main la formule en faisant une intégration par parties.


On pose par exemple $u=x^n$ et $v'=\exp(x)$ 

ce qui donne $u'=nx^{n-1}$ et $v = \exp(x)$.

\change
Ainsi     
    $$   F_n =  \int x^n \exp(x)\; \dd x$$
    \begin{center}
      $F_n$ : la primitive de $x^n \exp(x)$
    \end{center}
\change 
    \begin{center}
      s'exprime comme
    \end{center}
   $$ = \big[x^n\exp(x)\big] - \int nx^{n-1}\exp(x)\; \dd x$$
   
\change
    \begin{center}
      ce qui se réécrit : 
    \end{center}
$$= x^n\exp(x) - n F_{n-1}(x).$$
  
  
\change
Cela nous donne une nouvelle façon de calculer nos primitives.
Voici le code d'une fonction récursive FF basée sur la relation de récurrence que nous venons d'établir.

~

Cette formulation récursive est très élégante mais peut-être pas si simple à comprendre.

~

Tout d'abord si $n=0$, on sait que la primitive est l'exponentielle, on renvoie donc cette fonction, c'est la condition d'arrêt de notre récurrence.

~

Si $n$ est plus grand que $1$ alors on sait que 

$$F_n(x) = x^n \exp(x) - nF_{n-1}(x)$$

la fonction renvoie donc $x^n \exp(x) - nF_{n-1}(x)$.

Mais le travail n'est pas fini pour la machine, puisque le résultat fait lui-même
appel à la fonction au rang $n-1$, qui lui même aura besoin du résultat au rang $n-2$, etc

jusqu'à ce que l'on arrive au rang $0$ pour lequel la condition initiale stoppe la récurrence. 

La machine est alors en mesure de répondre à la question pour le rang $1$, puis pour le rang $2$... et enfin pour le rang $n$ comme demandé.


%%%%%%%%%%%%%%%%%%%%%%%%%%%%%%%%%%%%%%%%%%%%%%%%%%%%%%%%%%%
\diapo
Concernant la formule directe, on conjecture avec un peu d'observation et de réflexion que : 
  $$F_n(x) = \exp x \sum_{k=0}^n (-1)^{n-k} \frac{n!}{k!} x^k$$

Mais comment vérifier que cette formulation est exacte ?

\change
Appelons $G_n$ le second membre de cette expression.

\change
Alors, en isolant le dernier terme de la somme, $G_n$ se réécrit :

\change
$$n G_{n-1} + \exp(x)x^n$$

\change
Autrement dit, $G_n$ vérifie une relation de récurrence.

Cette relation est la même que celle vérifiée par $F_n$.

%En effet :
%$$\begin{array}{rcl} 
%    G_n(x) 
%    &=& \exp(x)\sum_{k=0}^{n-1} (-1)^{n-k} \frac{n!}{k!} x^k  + \exp(x)x^n \\
%    \pause
%    &=& n G_{n-1}(x)+ \exp(x)x^n       
%    \end{array}$$
    
%C'est la même relation de récurrence que pour $F_n$.
De plus, les termes initiaux $G_0$ et $F_0$ coïncident.

En conclusion, $G_n=F_n$  pour tout $n$.

\change

Voici une nouvelle fonction qui calcule l'expression de notre primitive.
C'est tout simplement l'implémentation de notre formule directe.


%%%%%%%%%%%%%%%%%%%%%%%%%%%%%%%%%%%%%%%%%%%%%%%%%%%%%%%%%%%
\diapo


Avant d'aborder notre prochain travail, rappelons la formule de changement de variable.

On considère une fonction $f$  définie sur un intervalle $I$ et $\varphi : J \to I$ 
une bijection de classe $\mathcal{C}^1$. 

La formule de changement de variable pour les intégrales est l'égalité suivante :
$$\int_a^b f(x) \; \dd x 
= \int_{\varphi^{-1}(a)}^{\varphi^{-1}(b)} f\big(\varphi(u)\big)\cdot\varphi'(u) \; \dd u$$


Cette formule est aussi valable pour les primitives, en omettant les bornes des intégrales !

%%%%%%%%%%%%%%%%%%%%%%%%%%%%%%%%%%%%%%%%%%%%%%%%%%%%%%%%%%%
\diapo

Dans cette première question du tp, nous allons procéder à un changement de variable simple.

On souhaite calculer $\displaystyle\int \sqrt{1-x^2} \; \dd x$ en simplifiant l'intégrale à l'aide de Sage, grâce au changement de variable $x = \sin u$.

Nous introduisons pour ce faire la fonction $\varphi(u) = \sin u$,

et calculons $f\big( \varphi(u) \big)$ et  $\varphi'(u)$.

L'objectif est d'avoir progressé dans la résolution de notre problème, c'est-à-dire que la fonction $g$ définie par $g(u) = f\big(\varphi(u)\big)\cdot\varphi'(u)$ soit plus facile à intégrer.

Nous en calculerons alors une primitive et répondrons ainsi à la question initialement posée.

%et vous en déduisez une primitive de $f(x) = \sqrt{1-x^2}$.


%%%%%%%%%%%%%%%%%%%%%%%%%%%%%%%%%%%%%%%%%%%%%%%%%%%%%%%%%%%
\diapo

%Voici le code de notre solution.

On commence par définir la fonction $f(x) = \sqrt{1-x^2}$.

Ce qu'on appelle changement de variable $x = \sin u$ signifie que l'on choisit pour $\varphi$ la fonction $\sin u$.

Voici l'implémentation du code :

la déclaration des variables,

la fonction $f$

et la changement de variable.


\change
    
On obtient $f\big( \varphi(u) \big)$ en substituant à la variable $x$ l'expression $\sin u$. 
    
    
cela donne $f\big( \varphi(u) \big) = \sqrt{1-\sin^2 u} = \sqrt{\cos^2 u} = |\cos u|$.

Ce qui s'obtient par la commande \codeinline{frondphi = $f(x=\phi)$}, puis en simplifiant.    

    % (Notez que Sage\ \og oublie \fg{} les valeurs absolues, car il calcule en fait la 
    % racine carrée complexe et pas réelle.
    % Ce sera heureusement sans conséquence pour la suite.
    % En effet pour que $f(x)$ soit définie, il faut $x \in [-1,1]$, comme $x = \sin u$
    % alors $u \in [-\frac\pi2,\frac\pi2]$ et donc $\cos u \ge 0$.)    
 
Enfin $\varphi'(u) = \cos u$ s'obtient par la commande \codeinline{dphi = diff(phi,u)}.
    
\change


On définit la fonction $g$ comme demandé dans l'énoncé.
%$g(u) = f\big(\varphi(u)\big)\cdot\varphi'(u)$ 
	
Vous noterez qu'après le changement de variable notre nouvelle fonction $g(u)$
est très facile à intégrer. C'est tout l'intérêt d'un changement de variable !

[grand $G$/petit $g$] On note $G$ une primitive de $g$.

\change

[grand $F$/petit $f$]

Pour obtenir une primitive de $f$, il ne faut pas oublier de revenir à la variable $x$.

Comme $x = \sin u$ alors $u = \arcsin x$.

La commande de substitution \codeinline{F = G(u = arcsin(x))} donne alors la primitive recherchée, 

ici c'est $\frac{\arcsin x}{2} + \frac x2 \sqrt{1-x^2}$.

On retrouve bien la solution que propose \Sage\ lorsqu'on lui demande directement une primitive de $f$. 


%%%%%%%%%%%%%%%%%%%%%%%%%%%%%%%%%%%%%%%%%%%%%%%%%%%%%%%%%%%
\diapo

Dans la deuxième question de ce tp, on effectue un changement de variable avec une définition un peu différente.%donné dans le sens inverse.

On souhaite calculer la primitive suivante : 
$$\displaystyle\int \frac{\dd x}{1+\left(\frac{x+1}{x}\right)^{1/3}}$$

Pour cela nous allons utiliser un changement de variable défini par une équation : $u^3 = \frac{x+1}{x}$.
    
Il faut commencer par expliciter le changement de variable $\varphi(u)$ et $\varphi^{-1}(x)$.

Nous ne corrigerons pas ici les questions suivantes mais il s'agit ni plus ni moins de reprendre 
les questions précédentes et d'automatiser les procédures afin 
 de calculer des primitives mais aussi des intégrales à l'aide d'un changement de variable donné.

%Dans cette dernière question du tp, on demande comme à la question précédente d'écrire une fonction qui effectue le changement de variable, mais cette fois-ci pas pour calculer la primitive de la fonction $f$, mais son intégrale entre deux bornes $a$ et $b$ données.
%
%Bien sur, l'équation qui définit le lien entre les deux variables est encore un paramètre.
%
%Il faut ensuite appliquer la fonction à l'exemple suivant.



%%%%%%%%%%%%%%%%%%%%%%%%%%%%%%%%%%%%%%%%%%%%%%%%%%%%%%%%%%%
\diapo

Pour calculer la primitive $\int \frac{\dd x}{1+\left(\frac{x+1}{x}\right)^{1/3}}$, l'énoncé  nous recommande le changement de variable $u^3 = \frac{x+1}{x}$.
  
La seule difficulté supplémentaire est qu'il faut ici exprimer $x$ en fonction de $u$ 
mais aussi $u$ en fonction de $x$.

Pour cela introduisons formellement l'équation reliant $u$ et $x$ par la commande : 
  \codeinline{$eqn = u^3 == (x+1)/x$}. 
  
  
  On peut maintenant résoudre l'équation afin d'obtenir
  $\phi(u)$ (c'est-à-dire $x$ en fonction de $u$) par :  
  \codeinline{phi = solve(eqn,x)[0].rhs()} 
  
  et $\phi^{-1}(x)$ (c'est-à-dire $u$ en fonction de $x$) : 
  \codeinline{phiinv = solve(eqn,u)[0].rhs()}.
  
  Le reste se déroule alors comme précédemment.




%%%%%%%%%%%%%%%%%%%%%%%%%%%%%%%%%%%%%%%%%%%%%%%%%%%%%%%%%%%%
%\diapo
%
%Voici le code de notre solution.
%
%
%\begin{enumerate}
%\setcounter{enumi}{2}
%		
%	\item La fonction \codeinline{integralechgtvar} dépend 
%	de la fonction $f$, et de l'équation $eqn$ définissant le changement de variable.
%
%	Notre fonction reprend les étapes de la question précédente :
%
%	On commence par extraire les fonctions $\phi$ et son inverse qui incarnent le changement de variable à partir de l'équation reliant les variables, etc.
%
%        Il est alors aisé d'appliquer notre fonction à un exemple donné en spécialisant les variables en $f(x)=(\cos x + 1)^n\sin x $ pour la fonction et $u= \cos x + 1$ pour l'équation.
%
%\end{enumerate}
%
%
%%%%%%%%%%%%%%%%%%%%%%%%%%%%%%%%%%%%%%%%%%%%%%%%%%%%%%%%%%%%
%\diapo
%% \change
%Voici notre fonction.
%% \change
%\begin{enumerate}
%
%
%		\setcounter{enumi}{3}
%
%	\item		
%Le seule différence avec la fonction précédente est 
%
%d'une part qu'il faut calculer les nouvelles bornes $\phi^{-1}a$ et $\phi^{-1}b$,
%
%d'autre part utiliser la commande \codeinline{integral} en utilisant ces nouvelles bornes.
%
%
%\end{enumerate}




\end{document}
