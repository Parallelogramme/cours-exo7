\documentclass[class=report,crop=false]{standalone}
\usepackage[screen]{../exo7book}

% Commande ponctuelle
\newcommand{\alenvers}[1]{\rotatebox[origin=c]{180}{#1}}
\newcommand{\vect}{\overrightarrow}

\begin{document}

%====================================================================
\chapitre{Calcul formel}
%====================================================================

%\insertvideo{nT94JZKl3cg}{partie 2. Structures de contrôle avec \Sage}

%%%%%%%%%%%%%%%%%%%%%%%%%%%%%%%%%%%%%%%%%%%%%%%%%%%%%%%%%%%%%%%%
\setcounter{section}{1}
\section{Structures de contrôle avec \Sage}

%--------------------------------------------------------
\subsection{Boucles}

%--------------------
\subsubsection*{Boucle \codeinline{for} (pour)} 

Pour faire varier un élément dans un ensemble, on utilise
l'instruction "\codeinline{for x in ensemble:}". Le bloc d'instructions qui suit sera 
alors successivement exécuté pour toutes les valeurs de $x$.

\insertcode{algos/structures-tex1.sage}{structures.sage (1)}

Notez encore une fois que le bloc d'instructions exécuté 
est délimité par les espaces en début de ligne.
Un exemple fréquent d'utilisation est "\codeinline{for k in range(n):}" qui fait varier $k$ de $0$ à $n-1$.


%--------------------
\subsubsection*{Liste \codeinline{range} (intervalle)}


En fait \codeinline{range(n)} renvoie la liste des $n$ premiers entiers : \codeinline{[0,1,2,...,n-1]}

Plus généralement \codeinline{range(a,b)} renvoie la liste \codeinline{[a,a+1,...,b-1]} d'entiers consécutifs 
alors que  \codeinline{range(a,b,c)} effectue une saut de $c$ termes,
par exemple \codeinline{range(0,101,5)} renvoie la liste \codeinline{[0,5,10,15,...,100]}.


Une autre façon légèrement différente pour définir des listes d'entiers 
est l'instruction \codeinline{[a..b]}
qui renvoie la liste des entiers $k$ tels que $a\le k \le b$.
Par exemple après l'instruction \codeinline{for k in [-7..12]:}
l'entier $k$ va prendre successivement les valeurs $-7$, $-6$, $-5$, ..., 
$-1$, $0$, $+1$,... jusqu'à $+12$.


La boucle \codeinline{for} permet plus généralement de parcourir n'importe quelle liste,
par exemple \codeinline{[0.13,0.31,0.53,0.98]}. 
L'instruction \codeinline{for x in [0.13,0.31,0.53,0.98]:} fera prendre à $x$ les $4$ valeurs
de $\{0.13 , 0.31, 0.53, 0.98\}$.

%--------------------
\subsubsection*{Boucle \codeinline{while} (tant que)}

\insertcode{algos/structures-tex2.sage}{structures.sage (2)}


La boucle \codeinline{while} exécute un bloc d'instructions 
tant que l'expression \codeinline{condition} est vérifiée. Lorsqu'elle ne l'est plus, on passe
aux instructions suivantes.


Voici, pour illustrer, le calcul de la racine carrée entière d'un entier $n$ : 
\insertcode{algos/structures-tex3.sage}{structures.sage (3)}

Lorsque la recherche est terminée, $k$ est le plus petit entier dont le carré dépasse $n$, 
par conséquent la racine carrée entière de $n$ est $k-1$.

Notez qu'utiliser une boucle << tant que >> comporte des risques, en effet
il faut toujours s'assurer que la boucle se termine.

%Voici quelques instructions utiles pour les boucles :
%\codeinline{break} termine immédiatement la boucle alors que 
%\codeinline{continue} passe directement à l'itération suivante (sans exécuter le reste du bloc). 


%--------------------
\subsubsection*{Test \codeinline{if... else} (si... sinon)} 
 
\insertcode{algos/structures-tex4.sage}{structures.sage (4)}

Si la condition est vérifiée, c'est le premier bloc d'instructions qui est exécuté,
si la condition n'est pas vérifiée, c'est le second. 
On passe ensuite aux instructions suivantes.


%--------------------------------------------------------
\subsection{Booléens et conditions}


%--------------------
\subsubsection*{Booléens}

Une \defi{expression booléenne} est une expression
qui peut seulement prendre les valeurs  \og Vrai \fg\
ou \og Faux \fg\ et qui sont codées par \codeinline{True}
ou \codeinline{False}.
Les expressions booléennes s'obtiennent 
principalement par des conditions.

%--------------------
\subsubsection*{Quelques conditions}

Voici une liste de conditions :
\begin{itemize}
  \item \codeinline{a < b} : teste l'inégalité stricte $a<b$,
  \item \codeinline{a > b} : teste l'inégalité stricte $a>b$,
  \item \codeinline{a <= b} : teste l'inégalité large $a\le b$,
  \item \codeinline{a >= b} : teste l'inégalité large $a\ge b$,
  \item \codeinline{a == b} : teste l'égalité $a=b$,
  \item \codeinline{a <> b} (ou \codeinline{a != b}) : teste la non égalité $a\neq b$.
  \item \codeinline{a in B} : teste si l'élément $a$ appartient à $B$.
\end{itemize}



\begin{remarque*}

\begin{enumerate}
  \item Une condition prend la valeur \codeinline{True} si elle est vraie et \codeinline{False} sinon.
Par exemple \codeinline{x == 2} renvoie \codeinline{True} si $x$ vaut $2$ et \codeinline{False} sinon.
La valeur de $x$ n'est pas modifiée pas le test.
  
  \item Il ne faut surtout pas confondre le test d'égalité
\codeinline{x == 2} avec l'affectation \codeinline{x = 2} (après cette dernière instruction 
$x$ vaut $2$).
  
  \item On peut combiner des conditions avec les opérateurs \codeinline{and},
  \codeinline{or}, \codeinline{not}. Par exemple :
  \codeinline{ (n>0) and (not (is_prime(n))) } est vraie
  si et seulement si $n$ est strictement positif et non premier.
  
\end{enumerate}  
\end{remarque*}

\begin{tp}
\sauteligne
\begin{enumerate}
  \item Pour deux assertions logiques $P$ et $Q$, écrire l'assertion $P\implies Q$.
  
  \item Une \defi{tautologie} est une assertion vraie quelles que soient les valeurs des paramètres,
  par exemple $(P \text{ ou } (\text{non\;} P))$ est vraie que l'assertion $P$ soit vraie ou fausse.
  Montrer que l'assertion suivante est une tautologie :
  $$\text{non\;} \bigg( \big[\text{non\;}(P \text{ et } Q)  \text{ et } (Q \text{ ou } R) \big] 
  \text{ et } \big[P \text{ et } (\text{non\;} R)\big] \bigg)$$
\end{enumerate}
  
\end{tp}

Il faut se souvenir du cours de logique qui définit l'assertion
\og$P\implies Q$\fg\ comme étant \og$\text{non\;}(P) \text{ ou } Q$\fg.
Il suffit donc de renvoyer : \codeinline{not(P) or Q}.


Pour l'examen de la tautologie, on teste toutes les possibilités et on 
vérifie que le résultat est vrai dans chaque cas !

\insertcode{algos/structures-tex10.sage}{structures.sage (10)}

%--------------------------------------------------------
\subsection{Fonctions informatiques}

Une fonction informatique prend en entrée un ou plusieurs paramètres
et renvoie un ou plusieurs résultats.

\insertcode{algos/structures-tex5.sage}{structures.sage (5)}

Par exemple, voici comment définir une fonction valeur absolue.
\insertcode{algos/structures-tex6.sage}{structures.sage (6)}

Dans ce cas, \codeinline{valeur_absolue(-3)} renvoie $+3$.

Il est possible d'utiliser cette fonction dans le reste du programme ou dans une autre fonction.
Notez aussi qu'une fonction peut prendre plusieurs paramètres.

Par exemple, que fait la fonction suivante ? Quel nom aurait été plus approprié ?

\insertcode{algos/structures-tex7.sage}{structures.sage (7)}

\begin{tp}
On considère trois réels distincts $a$, $b$ et $c$, ainsi que le polynôme défini par :
$$P(X) = \frac{(X-a)(X-b)}{(c-a)(c-b)}+\frac{(X-b)(X-c)}{(a-b)(a-c)}
+  \frac{(X-a)(X-c)}{(b-a)(b-c)}-1$$
\begin{enumerate}
  \item Définir trois variables par \codeinline{var('a,b,c')}.
  Définir une fonction \codeinline{polynome(x)} qui renvoie $P(x)$.
  
  \item Calculer $P(a)$, $P(b)$, $P(c)$.
  
  \item Comment se fait-il qu'un polynôme de degré $2$ ait $3$ racines distinctes ?
  Expliquez ! (Vous pourrez appliquer la méthode \codeinline{full_simplify()} à votre fonction.)
\end{enumerate}
  
\end{tp}


%--------------------------------------------------------
\subsection{Variable locale/variable globale}

Il faut bien faire la distinction entre les variables locales et les variables globales.

Par exemple, qu'affiche l'instruction \codeinline{print(x)} dans ce petit programme ?

\insertcode{algos/structures-tex9.sage}{structures.sage (9)}

Il faut bien comprendre que tous les \codeinline{x} de la fonction
%\codeinline{def incremente(x):},    \codeinline{x = x+1},     \codeinline{return x}
\codeinline{incremente}
représentent une variable locale qui n'existe que dans cette fonction et disparaît 
en dehors. On aurait tout aussi bien pu définir
\codeinline{def incremente(y):     y = y+1     return y}
ou bien utiliser la variable \codeinline{truc} ou \codeinline{zut}.
Par contre, les \codeinline{x} des lignes
\codeinline{x=2}, \codeinline{incremente(x)}, \codeinline{print(x)}
correspondent à une variable globale.
Une variable globale existe partout (sauf dans les fonctions où une variable
locale porte le même nom !).


En mathématique, on parle plutôt de variable muette au lieu de variable locale, par exemple
dans 
$$\sum_{k=0}^n k^2$$
$k$ est une variable muette, on aurait pu tout aussi bien la nommer $i$ ou $x$.
Par contre, il faut au préalable avoir défini $n$ (comme une variable globale), par exemple
\og $n=7$ \fg\ ou \og fixons un entier $n$ \fg...





    

%--------------------------------------------------------
\subsection{Conjecture de Syracuse}

Mettez immédiatement en pratique les structures de contrôle avec le travail suivant.

\begin{tp}
La \defi{suite de Syracuse} de terme initial $u_0 \in \Nn^*$ est définie par récurrence
pour $n\ge0$ par :
$$u_{n+1} = 
\begin{cases}
  \frac{u_n}{2} & \text{ si $u_n$ est pair,} \\
  3u_n + 1 & \text{ si $u_n$ est impair.} \\
  \end{cases}$$
\begin{enumerate}
  \item \'Ecrire une fonction qui, à partir de $u_0$ et de $n$, calcule le terme $u_n$ de la suite de Syracuse.
  \item Vérifier pour différentes valeurs du terme initial $u_0$, que la suite de Syracuse
  atteint la valeur $1$ au bout d'un certain rang (puis devient périodique : 
  $...,1,4,2,1,4,2,...$). \'Ecrire une fonction qui pour un certain choix de $u_0$ 
  renvoie le plus petit entier $n$ tel que $u_n=1$.
\end{enumerate}

\end{tp}
Personne ne sait à ce jour prouver que pour toutes les valeurs du terme initial $u_0$,
la suite de Syracuse atteint toujours la valeur $1$.

Voici le code pour calculer le $n$-ème terme de la suite.
\insertcode{algos/suite-syracuse-tex1.sage}{suite-syracuse.sage (1)}

Quelques remarques : on a utilisé une boucle \codeinline{for} pour calculer les $n$ premiers
termes de la suite ainsi qu'un test \codeinline{if...else}. 
%Expliquons la condition \codeinline{u\%2 == 0} ainsi que l'opération \codeinline{u = u//2}.

L'instruction \codeinline{u\%2} renvoie le reste de la division de $u$ par $2$. 
Le booléen \codeinline{u\%2 == 0} teste donc la parité de $u$.
L'instruction \codeinline{u//2} renvoie le quotient de la division euclidienne de $u$ par $2$.

\begin{remarque*}
Si $a \in \Zz$ et $n\in\Nn^*$ alors
\begin{itemize}
  \item \codeinline{a\%n} est le \defi{reste} de la division euclidienne de $a$ par $n$.
  C'est donc l'entier $r$ tel que $r \equiv a \pmod{n}$ et $0 \le r < n$. 
  
  \item \codeinline{a//n} est le \defi{quotient} de la division euclidienne de $a$ par $n$.
  C'est donc l'entier $q$ tel que $a=nq+r$ et $0\leq r <n$.
\end{itemize}

Il ne faut pas confondre la division de nombres réels : \codeinline{a/b}
et le quotient de la division euclidienne de deux entiers : \codeinline{a//b}.
Par exemple \codeinline{7/2} est la fraction $\frac72$ dont la valeur numérique est $3,5$
alors que \codeinline{7//2} renvoie $3$. On a \codeinline{7\%2} qui vaut $1$.
On a bien $7 = 2\times 3 + 1$.

Dans le cadre de l'exercice, \codeinline{u\%2} renvoie le reste de la division euclidienne 
de $u$ par $2$. Celui-ci vaut $0$ si $u$ est pair et $1$ si $u$ est impair.
Ainsi \codeinline{u\%2 == 0} teste si $u$ est pair ou impair.
Ensuite \codeinline{u = u//2} divise $u$ par $2$
(en fait cette opération n'est effectuée que pour $u$ pair).
%, donc \codeinline{u = u/2} donne le même résultat).  
\end{remarque*}


Pour calculer à partir de quel rang on obtient la valeur $1$, on utilise une boucle 
\codeinline{while} dont les instructions sont exécutées tant que $u \neq 1$.
Par construction, lorsque la boucle se termine c'est que la valeur de $u$ est $1$.

\insertcode{algos/suite-syracuse-tex2.sage}{suite-syracuse.sage (2)}

\finchapitre

\end{document}

