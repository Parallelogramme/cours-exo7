
%%%%%%%%%%%%%%%%%% PREAMBULE %%%%%%%%%%%%%%%%%%


\documentclass[12pt]{article}

\usepackage{amsfonts,amsmath,amssymb,amsthm}
\usepackage[utf8]{inputenc}
\usepackage[T1]{fontenc}
\usepackage[francais]{babel}


% packages
\usepackage{amsfonts,amsmath,amssymb,amsthm}
\usepackage[utf8]{inputenc}
\usepackage[T1]{fontenc}
%\usepackage{lmodern}

\usepackage[francais]{babel}
\usepackage{fancybox}
\usepackage{graphicx}

\usepackage{float}

%\usepackage[usenames, x11names]{xcolor}
\usepackage{tikz}
\usepackage{datetime}

\usepackage{mathptmx}
%\usepackage{fouriernc}
%\usepackage{newcent}
\usepackage[mathcal,mathbf]{euler}

%\usepackage{palatino}
%\usepackage{newcent}


% Commande spéciale prompteur

%\usepackage{mathptmx}
%\usepackage[mathcal,mathbf]{euler}
%\usepackage{mathpple,multido}

\usepackage[a4paper]{geometry}
\geometry{top=2cm, bottom=2cm, left=1cm, right=1cm, marginparsep=1cm}

\newcommand{\change}{{\color{red}\rule{\textwidth}{1mm}\\}}

\newcounter{mydiapo}

\newcommand{\diapo}{\newpage
\hfill {\normalsize  Diapo \themydiapo \quad \texttt{[\jobname]}} \\
\stepcounter{mydiapo}}


%%%%%%% COULEURS %%%%%%%%%%

% Pour blanc sur noir :
%\pagecolor[rgb]{0.5,0.5,0.5}
% \pagecolor[rgb]{0,0,0}
% \color[rgb]{1,1,1}



%\DeclareFixedFont{\myfont}{U}{cmss}{bx}{n}{18pt}
\newcommand{\debuttexte}{
%%%%%%%%%%%%% FONTES %%%%%%%%%%%%%
\renewcommand{\baselinestretch}{1.5}
\usefont{U}{cmss}{bx}{n}
\bfseries

% Taille normale : commenter le reste !
%Taille Arnaud
%\fontsize{19}{19}\selectfont

% Taille Barbara
%\fontsize{21}{22}\selectfont

%Taille François
\fontsize{25}{30}\selectfont

%Taille Pascal
%\fontsize{25}{30}\selectfont

%Taille Laura
%\fontsize{30}{35}\selectfont


%\myfont
%\usefont{U}{cmss}{bx}{n}

%\Huge
%\addtolength{\parskip}{\baselineskip}
}


% \usepackage{hyperref}
% \hypersetup{colorlinks=true, linkcolor=blue, urlcolor=blue,
% pdftitle={Exo7 - Exercices de mathématiques}, pdfauthor={Exo7}}


%section
% \usepackage{sectsty}
% \allsectionsfont{\bf}
%\sectionfont{\color{Tomato3}\upshape\selectfont}
%\subsectionfont{\color{Tomato4}\upshape\selectfont}

%----- Ensembles : entiers, reels, complexes -----
\newcommand{\Nn}{\mathbb{N}} \newcommand{\N}{\mathbb{N}}
\newcommand{\Zz}{\mathbb{Z}} \newcommand{\Z}{\mathbb{Z}}
\newcommand{\Qq}{\mathbb{Q}} \newcommand{\Q}{\mathbb{Q}}
\newcommand{\Rr}{\mathbb{R}} \newcommand{\R}{\mathbb{R}}
\newcommand{\Cc}{\mathbb{C}} 
\newcommand{\Kk}{\mathbb{K}} \newcommand{\K}{\mathbb{K}}

%----- Modifications de symboles -----
\renewcommand{\epsilon}{\varepsilon}
\renewcommand{\Re}{\mathop{\text{Re}}\nolimits}
\renewcommand{\Im}{\mathop{\text{Im}}\nolimits}
%\newcommand{\llbracket}{\left[\kern-0.15em\left[}
%\newcommand{\rrbracket}{\right]\kern-0.15em\right]}

\renewcommand{\ge}{\geqslant}
\renewcommand{\geq}{\geqslant}
\renewcommand{\le}{\leqslant}
\renewcommand{\leq}{\leqslant}

%----- Fonctions usuelles -----
\newcommand{\ch}{\mathop{\mathrm{ch}}\nolimits}
\newcommand{\sh}{\mathop{\mathrm{sh}}\nolimits}
\renewcommand{\tanh}{\mathop{\mathrm{th}}\nolimits}
\newcommand{\cotan}{\mathop{\mathrm{cotan}}\nolimits}
\newcommand{\Arcsin}{\mathop{\mathrm{Arcsin}}\nolimits}
\newcommand{\Arccos}{\mathop{\mathrm{Arccos}}\nolimits}
\newcommand{\Arctan}{\mathop{\mathrm{Arctan}}\nolimits}
\newcommand{\Argsh}{\mathop{\mathrm{Argsh}}\nolimits}
\newcommand{\Argch}{\mathop{\mathrm{Argch}}\nolimits}
\newcommand{\Argth}{\mathop{\mathrm{Argth}}\nolimits}
\newcommand{\pgcd}{\mathop{\mathrm{pgcd}}\nolimits} 

\newcommand{\Card}{\mathop{\text{Card}}\nolimits}
\newcommand{\Ker}{\mathop{\text{Ker}}\nolimits}
\newcommand{\id}{\mathop{\text{id}}\nolimits}
\newcommand{\ii}{\mathrm{i}}
\newcommand{\dd}{\mathrm{d}}
\newcommand{\Vect}{\mathop{\text{Vect}}\nolimits}
\newcommand{\Mat}{\mathop{\mathrm{Mat}}\nolimits}
\newcommand{\rg}{\mathop{\text{rg}}\nolimits}
\newcommand{\tr}{\mathop{\text{tr}}\nolimits}
\newcommand{\ppcm}{\mathop{\text{ppcm}}\nolimits}

%----- Structure des exercices ------

\newtheoremstyle{styleexo}% name
{2ex}% Space above
{3ex}% Space below
{}% Body font
{}% Indent amount 1
{\bfseries} % Theorem head font
{}% Punctuation after theorem head
{\newline}% Space after theorem head 2
{}% Theorem head spec (can be left empty, meaning ‘normal’)

%\theoremstyle{styleexo}
\newtheorem{exo}{Exercice}
\newtheorem{ind}{Indications}
\newtheorem{cor}{Correction}


\newcommand{\exercice}[1]{} \newcommand{\finexercice}{}
%\newcommand{\exercice}[1]{{\tiny\texttt{#1}}\vspace{-2ex}} % pour afficher le numero absolu, l'auteur...
\newcommand{\enonce}{\begin{exo}} \newcommand{\finenonce}{\end{exo}}
\newcommand{\indication}{\begin{ind}} \newcommand{\finindication}{\end{ind}}
\newcommand{\correction}{\begin{cor}} \newcommand{\fincorrection}{\end{cor}}

\newcommand{\noindication}{\stepcounter{ind}}
\newcommand{\nocorrection}{\stepcounter{cor}}

\newcommand{\fiche}[1]{} \newcommand{\finfiche}{}
\newcommand{\titre}[1]{\centerline{\large \bf #1}}
\newcommand{\addcommand}[1]{}
\newcommand{\video}[1]{}

% Marge
\newcommand{\mymargin}[1]{\marginpar{{\small #1}}}



%----- Presentation ------
\setlength{\parindent}{0cm}

%\newcommand{\ExoSept}{\href{http://exo7.emath.fr}{\textbf{\textsf{Exo7}}}}

\definecolor{myred}{rgb}{0.93,0.26,0}
\definecolor{myorange}{rgb}{0.97,0.58,0}
\definecolor{myyellow}{rgb}{1,0.86,0}

\newcommand{\LogoExoSept}[1]{  % input : echelle
{\usefont{U}{cmss}{bx}{n}
\begin{tikzpicture}[scale=0.1*#1,transform shape]
  \fill[color=myorange] (0,0)--(4,0)--(4,-4)--(0,-4)--cycle;
  \fill[color=myred] (0,0)--(0,3)--(-3,3)--(-3,0)--cycle;
  \fill[color=myyellow] (4,0)--(7,4)--(3,7)--(0,3)--cycle;
  \node[scale=5] at (3.5,3.5) {Exo7};
\end{tikzpicture}}
}



\theoremstyle{definition}
%\newtheorem{proposition}{Proposition}
%\newtheorem{exemple}{Exemple}
%\newtheorem{theoreme}{Théorème}
\newtheorem{lemme}{Lemme}
\newtheorem{corollaire}{Corollaire}
%\newtheorem*{remarque*}{Remarque}
%\newtheorem*{miniexercice}{Mini-exercices}
%\newtheorem{definition}{Définition}




%definition d'un terme
\newcommand{\defi}[1]{{\color{myorange}\textbf{\emph{#1}}}}
\newcommand{\evidence}[1]{{\color{blue}\textbf{\emph{#1}}}}



 %----- Commandes divers ------

\newcommand{\codeinline}[1]{\texttt{#1}}
\newcommand{\vect}{\overrightarrow}
\newcommand{\Sage}{\texttt{Sage}}
%%%%%%%%%%%%%%%%%%%%%%%%%%%%%%%%%%%%%%%%%%%%%%%%%%%%%%%%%%%%%
%%%%%%%%%%%%%%%%%%%%%%%%%%%%%%%%%%%%%%%%%%%%%%%%%%%%%%%%%%%%%


\begin{document}

\debuttexte


%%%%%%%%%%%%%%%%%%%%%%%%%%%%%%%%%%%%%%%%%%%%%%%%%%%%%%%%%%%
\diapo

Nous allons maintenant utiliser le calcul formel non seulement pour visualiser
des courbes et des surfaces mais aussi pour les étudier.

\change


\change
Nous allons commencer par nous intéresser à une courbe paramétrée particulière

\change
puis nous aborderons l'étude de la projection stéréographique.


%%%%%%%%%%%%%%%%%%%%%%%%%%%%%%%%%%%%%%%%%%%%%%%%%%%%%%%%%%%
\diapo

Nous nous donnons une courbe paramétrée définie par 
$  x(t) =  t^3-2t$
et $y(t) =  t^2-t$

Nous commencerons bien sûr par tracer la courbe

~

avant de calculer les points d'intersection de la courbe avec l'axe des ordonnées.

~

Puis nous calculerons les points en lesquels la tangente est verticale, puis horizontale.

~
  
Enfin nous chercherons les éventuels points doubles.
%cette courbe se recoupe elle-même en un point que vous déterminerez.


%%%%%%%%%%%%%%%%%%%%%%%%%%%%%%%%%%%%%%%%%%%%%%%%%%%%%%%%%%%
\diapo

Voici la courbe qui a l'allure d'une boucle. 

~

Ce graphique nous permet déjà de proposer, avant de les démontrer, des réponses aux questions posées. 

~


La courbe recoupe l'axe des ordonnées en trois points.

~


Il y a deux points où la tangente est verticale et un point où la tangente est horizontale.

~

Enfin, voici le point double.


%%%%%%%%%%%%%%%%%%%%%%%%%%%%%%%%%%%%%%%%%%%%%%%%%%%%%%%%%%%
\diapo

Passons maintenant à l'étude de cette courbe paramétrée définie par $x = t^3-2*t$ et $y = t^2-t$.


\change
L'équation $x(t)=0$ caractérise les valeurs du paramètre $t$ des points d'intersection de la courbe
avec l'axe des ordonnées.

\change
On résout cette équation par 
\codeinline{solve(x==0,t)}.

\change
On obtient trois solutions $t \in \{-\sqrt2,0,+\sqrt2\}$.

\change
On reporte ces valurs dans les coordonnées $x$ et $y$ et on trouve les trois points
$\{(0,2+\sqrt2)),(0,0),(0,2-\sqrt2)\}$.

\change
Pour les tangentes, on définit les fonctions dérivées $x'(t)$ et $y'(t)$.

\change
Les valeurs du paramètre $t$ des points en lesquels la tangente à la courbe est verticale 

\change
s'obtiennent en résolvant l'équation $x'(t)=0$, 


%\change
On trouve ici deux valeurs de $t$ (qui ne sont pas détaillées ici), ce qui correspond à deux points de la courbe.

\change

On fait de même pour le cas des tangentes horizontales.

~

Au passage, on s'assurera bien qu'il n'y a pas de points singuliers (c'est-à-dire que $x'$ et $y'$ ne s'annulent pas en même temps).
  
  
%%%%%%%%%%%%%%%%%%%%%%%%%%%%%%%%%%%%%%%%%%%%%%%%%%%%%%%%%%%
\diapo

Trouver les points doubles est souvent un calcul délicat.


Il s'agit en effet ici de résoudre le système d'équations -- ici polynomiales :
   $$\left\{
  \begin{array}{l}
  x(s) =  x(t)\\[1mm]
  y(s) =  y(t)
  \end{array}
  \right.\qquad  s,t \in \Rr.$$ 
  
  
\change

Il faut tout d'abord exclure la solution évidente $t=s$. 

\change

Algébriquement, cela signifie
  que $(s-t)$ divise les polynômes en deux variables $x(s)-x(t)$ et
  $y(s)-y(t)$. 

\change

Autrement dit, il s'agit maintenant de résoudre :
   $$\left\{
  \begin{array}{l}
  \dfrac{x(s)-x(t)}{s-t} = 0  \\[1mm]
  \dfrac{y(s)-y(t)}{s-t} = 0
  \end{array}
  \right.\qquad  s,t \in \Rr.$$   

%%%%%%%%%%%%%%%%%%%%%%%%%%%%%%%%%%%%%%%%%%%%%%%%%%%%%%%%%%%
\diapo

  Le code suivant calcule les éventuels points doubles :
  
  Voici la définition de la courbe paramétrée $x$ et $y$ en fonction de $t$.

\change

  
  Voici l'équation $x(s)-x(t)$,

\change
  
  que l'on divise par $s-t$ (et au passage on simplifie).

\change
  
  On fait de même pour $y(s)-y(t)$
  
\change

  Et on cherche les solutions $(s,t)$ qui vérifient simultanément les deux équations  !
 
\change   
  On trouve une unique solution pour les paramètres 
  $(s,t) = \big(\frac{1-\sqrt5}{2}, \frac{1+\sqrt5}{2}\big)$, 
  

~

Il existe donc un unique point double dont on calcule les coordonnées : $(1,1)$.


%%%%%%%%%%%%%%%%%%%%%%%%%%%%%%%%%%%%%%%%%%%%%%%%%%%%%%%%%%%
\diapo


%Voici un tp un peu long mais intéressant.
Voici un tp avec un contenu assez riche, qui, je pense, va vous intéresser.

~


On considère la sphère $\mathcal{S}$ centrée à l'origine et de rayon $1$.

~


Le pôle Nord est le point $N$ de coordonnées $(0,0,1)$. 

~

On appelle $\mathcal{P}$ le plan équatorial d'équation $(z=0)$.


~

Pour un point $S$ de la sphère, on trace la droite passant par le pôle Nord
et ce point $S$. 

On appelle alors $P$ le point d'intersection de cette droite avec le plan équatorial.

~

La \defi{projection stéréographique}
est l'application qui à un point $S$ associe le point $P$.


%%%%%%%%%%%%%%%%%%%%%%%%%%%%%%%%%%%%%%%%%%%%%%%%%%%%%%%%%%%
\diapo

Le premier travail consiste à donner une définition algébrique de la projection stéréographique.

~

Il s'agit de montrer que la formule est donnée par : 
  $$  \Phi(x,y,z) = \left( \frac{x}{1-z}, \frac{y}{1-z} \right)$$


Pour cela nous écrirons une relation de colinéarité entre les vecteurs
$NP$ et $NS$ et nous calculerons à la main ou avec Sage le coefficient $k$ de colinéarité.

La projection stéréographique est bijective, pour le prouver,


l'inverse $\Psi$ étant proposée par l'énoncé, il nous suffira de vérifier que $\Phi \circ \Psi$ est l'identité.

%%%%%%%%%%%%%%%%%%%%%%%%%%%%%%%%%%%%%%%%%%%%%%%%%%%%%%%%%%%
\diapo


Imaginons maintenant qu'une courbe $\mathcal{C}'$ soit tracée dans le plan de projection.

~

Quel est le tracé de son image inverse $\mathcal{C}$ qui est une courbe de la sphère ?

 
~

La projection stéréographique est une transformation surprenante :

Nous vérifierons graphiquement que
\begin{itemize}
\item  \og La projection stéréographique envoie les cercles de la sphère sur des cercles ou des droites du plan.\fg
 
 
\item et que \og La projection stéréographique préserve les angles.\fg\ 


En particulier, deux courbes qui se coupent  à angle droit, s'envoient sur deux courbes qui se coupent à angle droit.
 
\end{itemize} 


Nous étudierons le cas particulier intéressant où la courbe projetée est la spirale logarithmique.

~

La courbe inverse (sur la sphère) porte le nom de loxodromie.

~

Enfin, un autre cas particulier que nous ne corrigerons pas ici.
%[On ne corrigera pas ici la dernière question.]


%%%%%%%%%%%%%%%%%%%%%%%%%%%%%%%%%%%%%%%%%%%%%%%%%%%%%%%%%%%
\diapo

Le vecteur $\vect{NP}$ est colinéaire au vecteur $\vect{NS}$, donc il existe
$k\in \Rr$ tel que  $\vect{NP} = k \vect{NS}$.


\change
Mais nous voulons de plus que le point $P$ appartienne au plan $\mathcal{P}$ de l'équateur, qui a pour équation $z_P = 0$.


\change
Ce qui nous permettra de trouver $k = \frac{1}{1-z}$,

\change
et donc si $S=(x,y,z)$, alors le point $P$ défini par $N + k\vect{NS}$ aura pour coordonnées 
$$\left( \frac{x}{1-z}, \frac{y}{1-z}, 0 \right).$$
   
\change

On laisse Sage\ faire les calculs :

On déclare toutes les variables en jeu, y compris le coefficient $k$ à déterminer.

On définit le pôle Nord $N$ ou plus exactement la variable $N$ comme le vecteur $\vect{ON}$,

Un point $S$ de la sphère,

Et le projeté $P$ sur le plan de l'équateur.

On définit enfin le vecteur $V$ qui correspond en fait à la relation de colinéarité souhaitée.

~

Ce vecteur $V$ étant censé être nul, examinons sa troisième coordonnée appelée ici $eq$ et demandons à \Sage\ de résoudre l'équation en $k$.

La machine nous fournit une seule solution que l'on extrait et qui est bien celle annoncée plus haut : $1/(1-z)$.

On en déduit alors l'expression du point $P$ en fonction de [petits] $x,y,z$.   
 

%%%%%%%%%%%%%%%%%%%%%%%%%%%%%%%%%%%%%%%%%%%%%%%%%%%%%%%%%%%
\diapo
[[grand X,Y, petit x,y,z]]

Aucun problème pour implémenter la projection stéréographique : 

on note petit $x$, petit $y$, petit $z$ les coordonnées du point $S$ de la sphère,

on calcule $X = \frac{x}{1-z}$, $Y = \frac{y}{1-z}$

et on renvoie les coordonnées $(X,Y)$ de ce point. La troisième coordonnée n'est pas nécessaire puisque $P$ se situe dans le plan de l'équateur.

%(on est dans le plan équatorial, on ne considère pas la troisième coordonnée qui est toujours nulle).


%%%%%%%%%%%%%%%%%%%%%%%%%%%%%%%%%%%%%%%%%%%%%%%%%%%%%%%%%%%
\diapo

[[grand X,Y, petit x,y,z]]


L'énoncé propose une expression de l'inverse $\Psi$ :

On définit donc $r=1+X^2+Y^2$, puis $\displaystyle x=\frac{2X}{r}$, $\displaystyle y=\frac{2Y}{r}$ et $\displaystyle z=1-\frac{2}{r}$.


\change
Nous allons vérifier formellement que $\Psi$ est la bijection réciproque de la projection stéréographique, c'est-à-dire que 
$\Phi\big( \Psi(X,Y) \big) = (X,Y)$ pour tout $(X,Y) \in \Rr^2$.
  
  
\change
Pour cela, considérons des variables $X,Y$ et calculons l'image de $X,Y$ par $\Psi$.
  
Puis calculons l'image de cette image par la projection stéréographique,

\change  
Comme nous obtenons les variables $X$, $Y$ de départ, cela prouve 
que $\Psi$ est bien la bijection réciproque de $\Phi$. 

%%%%%%%%%%%%%%%%%%%%%%%%%%%%%%%%%%%%%%%%%%%%%%%%%%%%%%%%%%%
\diapo


[grand X/petit x]

 Voici le code qui permet de tracer la courbes inverse d'une courbe donnée.
 
~
 
La courbe du plan paramétrée par $(X(t),Y(t))$ est considérée ici comme une ligne polygonale avec un pas assez petit.


~

C'est-à-dire que l'on calcule le point de paramètre $t=a$,
puis pour le point de paramètre $t+\epsilon$, puis $t+2\epsilon$, jusqu'à $t=b$. Ici le pas $\epsilon$ vaut $0,1$. 
  
  
~

 On calcule, point par point, l'image inverse par la projection stéréographique des points précédents.
 
  
 On trace enfin la sphère et les lignes polygonales du plan et de la sphère.

\change
 Par exemple \codeinline{G = courbes(t\^{}3,t\^{}2,-2,2)}, trace la courbe
 du plan d'équation paramétrique $\big(t^3,t^2\big)$, $t\in[-2,2]$, ainsi
 que son image par la projection stéréographique inverse.
%%%%%%%%%%%%%%%%%%%%%%%%%%%%%%%%%%%%%%%%%%%%%%%%%%%%%%%%%%%
\diapo

Voici deux cercles, en rouge, tracés sur la sphère, et leurs images par projection stéréographique : 
un cercle et une droite.
 
En fait un cercle de la sphère s'envoie toujours sur un cercle du plan,
sauf pour les cercles passant par le pôle Nord qui eux s'envoient sur une droite.
 
 
Remarquez que les cercles rouge s'intersectent ici à angles droits,
et que c'est aussi le cas pour le cercle et la droite en vert.
C'est une propriété générale : la projection stéréographique préserve les angles.


\change
On termine par une loxodromie de la sphère ici en rouge, 
qui est l'image inverse de la spirale logarithmique du plan ici en vert.

La loxodromie est bien connue des navigateurs, car elle
correspond à une navigation à cap constant.


\end{document}
