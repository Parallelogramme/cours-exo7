
%%%%%%%%%%%%%%%%%% PREAMBULE %%%%%%%%%%%%%%%%%%

\documentclass[aspectratio=169,utf8]{beamer}
%\documentclass[aspectratio=169,handout]{beamer}

\usetheme{Boadilla}
%\usecolortheme{seahorse}
\usecolortheme[RGB={245,66,24}]{structure}
\useoutertheme{infolines}

% packages
\usepackage{amsfonts,amsmath,amssymb,amsthm}
\usepackage[utf8]{inputenc}
\usepackage[T1]{fontenc}
\usepackage{lmodern}

\usepackage[francais]{babel}
\usepackage{fancybox}
\usepackage{graphicx}

\usepackage{float}
\usepackage{xfrac}

%\usepackage[usenames, x11names]{xcolor}
\usepackage{tikz}
\usepackage{pgfplots}
\usepackage{datetime}



%-----  Package unités -----
\usepackage{siunitx}
\sisetup{locale = FR,detect-all,per-mode = symbol}

%\usepackage{mathptmx}
%\usepackage{fouriernc}
%\usepackage{newcent}
%\usepackage[mathcal,mathbf]{euler}

%\usepackage{palatino}
%\usepackage{newcent}
% \usepackage[mathcal,mathbf]{euler}



% \usepackage{hyperref}
% \hypersetup{colorlinks=true, linkcolor=blue, urlcolor=blue,
% pdftitle={Exo7 - Exercices de mathématiques}, pdfauthor={Exo7}}


%section
% \usepackage{sectsty}
% \allsectionsfont{\bf}
%\sectionfont{\color{Tomato3}\upshape\selectfont}
%\subsectionfont{\color{Tomato4}\upshape\selectfont}

%----- Ensembles : entiers, reels, complexes -----
\newcommand{\Nn}{\mathbb{N}} \newcommand{\N}{\mathbb{N}}
\newcommand{\Zz}{\mathbb{Z}} \newcommand{\Z}{\mathbb{Z}}
\newcommand{\Qq}{\mathbb{Q}} \newcommand{\Q}{\mathbb{Q}}
\newcommand{\Rr}{\mathbb{R}} \newcommand{\R}{\mathbb{R}}
\newcommand{\Cc}{\mathbb{C}} 
\newcommand{\Kk}{\mathbb{K}} \newcommand{\K}{\mathbb{K}}

%----- Modifications de symboles -----
\renewcommand{\epsilon}{\varepsilon}
\renewcommand{\Re}{\mathop{\text{Re}}\nolimits}
\renewcommand{\Im}{\mathop{\text{Im}}\nolimits}
%\newcommand{\llbracket}{\left[\kern-0.15em\left[}
%\newcommand{\rrbracket}{\right]\kern-0.15em\right]}

\renewcommand{\ge}{\geqslant}
\renewcommand{\geq}{\geqslant}
\renewcommand{\le}{\leqslant}
\renewcommand{\leq}{\leqslant}
\renewcommand{\epsilon}{\varepsilon}

%----- Fonctions usuelles -----
\newcommand{\ch}{\mathop{\text{ch}}\nolimits}
\newcommand{\sh}{\mathop{\text{sh}}\nolimits}
\renewcommand{\tanh}{\mathop{\text{th}}\nolimits}
\newcommand{\cotan}{\mathop{\text{cotan}}\nolimits}
\newcommand{\Arcsin}{\mathop{\text{arcsin}}\nolimits}
\newcommand{\Arccos}{\mathop{\text{arccos}}\nolimits}
\newcommand{\Arctan}{\mathop{\text{arctan}}\nolimits}
\newcommand{\Argsh}{\mathop{\text{argsh}}\nolimits}
\newcommand{\Argch}{\mathop{\text{argch}}\nolimits}
\newcommand{\Argth}{\mathop{\text{argth}}\nolimits}
\newcommand{\pgcd}{\mathop{\text{pgcd}}\nolimits} 


%----- Commandes divers ------
\newcommand{\ii}{\mathrm{i}}
\newcommand{\dd}{\text{d}}
\newcommand{\id}{\mathop{\text{id}}\nolimits}
\newcommand{\Ker}{\mathop{\text{Ker}}\nolimits}
\newcommand{\Card}{\mathop{\text{Card}}\nolimits}
\newcommand{\Vect}{\mathop{\text{Vect}}\nolimits}
\newcommand{\Mat}{\mathop{\text{Mat}}\nolimits}
\newcommand{\rg}{\mathop{\text{rg}}\nolimits}
\newcommand{\tr}{\mathop{\text{tr}}\nolimits}


%----- Structure des exercices ------

\newtheoremstyle{styleexo}% name
{2ex}% Space above
{3ex}% Space below
{}% Body font
{}% Indent amount 1
{\bfseries} % Theorem head font
{}% Punctuation after theorem head
{\newline}% Space after theorem head 2
{}% Theorem head spec (can be left empty, meaning ‘normal’)

%\theoremstyle{styleexo}
\newtheorem{exo}{Exercice}
\newtheorem{ind}{Indications}
\newtheorem{cor}{Correction}


\newcommand{\exercice}[1]{} \newcommand{\finexercice}{}
%\newcommand{\exercice}[1]{{\tiny\texttt{#1}}\vspace{-2ex}} % pour afficher le numero absolu, l'auteur...
\newcommand{\enonce}{\begin{exo}} \newcommand{\finenonce}{\end{exo}}
\newcommand{\indication}{\begin{ind}} \newcommand{\finindication}{\end{ind}}
\newcommand{\correction}{\begin{cor}} \newcommand{\fincorrection}{\end{cor}}

\newcommand{\noindication}{\stepcounter{ind}}
\newcommand{\nocorrection}{\stepcounter{cor}}

\newcommand{\fiche}[1]{} \newcommand{\finfiche}{}
\newcommand{\titre}[1]{\centerline{\large \bf #1}}
\newcommand{\addcommand}[1]{}
\newcommand{\video}[1]{}

% Marge
\newcommand{\mymargin}[1]{\marginpar{{\small #1}}}

\def\noqed{\renewcommand{\qedsymbol}{}}


%----- Presentation ------
\setlength{\parindent}{0cm}

%\newcommand{\ExoSept}{\href{http://exo7.emath.fr}{\textbf{\textsf{Exo7}}}}

\definecolor{myred}{rgb}{0.93,0.26,0}
\definecolor{myorange}{rgb}{0.97,0.58,0}
\definecolor{myyellow}{rgb}{1,0.86,0}

\newcommand{\LogoExoSept}[1]{  % input : echelle
{\usefont{U}{cmss}{bx}{n}
\begin{tikzpicture}[scale=0.1*#1,transform shape]
  \fill[color=myorange] (0,0)--(4,0)--(4,-4)--(0,-4)--cycle;
  \fill[color=myred] (0,0)--(0,3)--(-3,3)--(-3,0)--cycle;
  \fill[color=myyellow] (4,0)--(7,4)--(3,7)--(0,3)--cycle;
  \node[scale=5] at (3.5,3.5) {Exo7};
\end{tikzpicture}}
}


\newcommand{\debutmontitre}{
  \author{} \date{} 
  \thispagestyle{empty}
  \hspace*{-10ex}
  \begin{minipage}{\textwidth}
    \titlepage  
  \vspace*{-2.5cm}
  \begin{center}
    \LogoExoSept{2.5}
  \end{center}
  \end{minipage}

  \vspace*{-0cm}
  
  % Astuce pour que le background ne soit pas discrétisé lors de la conversion pdf -> png
\begin{tikzpicture}
        \fill[opacity=0,green!60!black] (0,0)--++(0,0)--++(0,0)--++(0,0)--cycle; 
\end{tikzpicture}

% toc S'affiche trop tot :
% \tableofcontents[hideallsubsections, pausesections]
}

\newcommand{\finmontitre}{
  \end{frame}
  \setcounter{framenumber}{0}
} % ne marche pas pour une raison obscure

%----- Commandes supplementaires ------

% \usepackage[landscape]{geometry}
% \geometry{top=1cm, bottom=3cm, left=2cm, right=10cm, marginparsep=1cm
% }
% \usepackage[a4paper]{geometry}
% \geometry{top=2cm, bottom=2cm, left=2cm, right=2cm, marginparsep=1cm
% }

%\usepackage{standalone}


% New command Arnaud -- november 2011
\setbeamersize{text margin left=24ex}
% si vous modifier cette valeur il faut aussi
% modifier le decalage du titre pour compenser
% (ex : ici =+10ex, titre =-5ex

\theoremstyle{definition}
%\newtheorem{proposition}{Proposition}
%\newtheorem{exemple}{Exemple}
%\newtheorem{theoreme}{Théorème}
%\newtheorem{lemme}{Lemme}
%\newtheorem{corollaire}{Corollaire}
%\newtheorem*{remarque*}{Remarque}
%\newtheorem*{miniexercice}{Mini-exercices}
%\newtheorem{definition}{Définition}

% Commande tikz
\usetikzlibrary{calc}
\usetikzlibrary{patterns,arrows}
\usetikzlibrary{matrix}
\usetikzlibrary{fadings} 

%definition d'un terme
\newcommand{\defi}[1]{{\color{myorange}\textbf{\emph{#1}}}}
\newcommand{\evidence}[1]{{\color{blue}\textbf{\emph{#1}}}}
\newcommand{\assertion}[1]{\emph{\og#1\fg}}  % pour chapitre logique
%\renewcommand{\contentsname}{Sommaire}
\renewcommand{\contentsname}{}
\setcounter{tocdepth}{2}



%------ Figures ------

\def\myscale{1} % par défaut 
\newcommand{\myfigure}[2]{  % entrée : echelle, fichier figure
\def\myscale{#1}
\begin{center}
\footnotesize
{#2}
\end{center}}


%------ Encadrement ------

\usepackage{fancybox}


\newcommand{\mybox}[1]{
\setlength{\fboxsep}{7pt}
\begin{center}
\shadowbox{#1}
\end{center}}

\newcommand{\myboxinline}[1]{
\setlength{\fboxsep}{5pt}
\raisebox{-10pt}{
\shadowbox{#1}
}
}

%--------------- Commande beamer---------------
\newcommand{\beameronly}[1]{#1} % permet de mettre des pause dans beamer pas dans poly


\setbeamertemplate{navigation symbols}{}
\setbeamertemplate{footline}  % tiré du fichier beamerouterinfolines.sty
{
  \leavevmode%
  \hbox{%
  \begin{beamercolorbox}[wd=.333333\paperwidth,ht=2.25ex,dp=1ex,center]{author in head/foot}%
    % \usebeamerfont{author in head/foot}\insertshortauthor%~~(\insertshortinstitute)
    \usebeamerfont{section in head/foot}{\bf\insertshorttitle}
  \end{beamercolorbox}%
  \begin{beamercolorbox}[wd=.333333\paperwidth,ht=2.25ex,dp=1ex,center]{title in head/foot}%
    \usebeamerfont{section in head/foot}{\bf\insertsectionhead}
  \end{beamercolorbox}%
  \begin{beamercolorbox}[wd=.333333\paperwidth,ht=2.25ex,dp=1ex,right]{date in head/foot}%
    % \usebeamerfont{date in head/foot}\insertshortdate{}\hspace*{2em}
    \insertframenumber{} / \inserttotalframenumber\hspace*{2ex} 
  \end{beamercolorbox}}%
  \vskip0pt%
}


\definecolor{mygrey}{rgb}{0.5,0.5,0.5}
\setlength{\parindent}{0cm}
%\DeclareTextFontCommand{\helvetica}{\fontfamily{phv}\selectfont}

% background beamer
\definecolor{couleurhaut}{rgb}{0.85,0.9,1}  % creme
\definecolor{couleurmilieu}{rgb}{1,1,1}  % vert pale
\definecolor{couleurbas}{rgb}{0.85,0.9,1}  % blanc
\setbeamertemplate{background canvas}[vertical shading]%
[top=couleurhaut,middle=couleurmilieu,midpoint=0.4,bottom=couleurbas] 
%[top=fondtitre!05,bottom=fondtitre!60]



\makeatletter
\setbeamertemplate{theorem begin}
{%
  \begin{\inserttheoremblockenv}
  {%
    \inserttheoremheadfont
    \inserttheoremname
    \inserttheoremnumber
    \ifx\inserttheoremaddition\@empty\else\ (\inserttheoremaddition)\fi%
    \inserttheorempunctuation
  }%
}
\setbeamertemplate{theorem end}{\end{\inserttheoremblockenv}}

\newenvironment{theoreme}[1][]{%
   \setbeamercolor{block title}{fg=structure,bg=structure!40}
   \setbeamercolor{block body}{fg=black,bg=structure!10}
   \begin{block}{{\bf Th\'eor\`eme }#1}
}{%
   \end{block}%
}


\newenvironment{proposition}[1][]{%
   \setbeamercolor{block title}{fg=structure,bg=structure!40}
   \setbeamercolor{block body}{fg=black,bg=structure!10}
   \begin{block}{{\bf Proposition }#1}
}{%
   \end{block}%
}

\newenvironment{corollaire}[1][]{%
   \setbeamercolor{block title}{fg=structure,bg=structure!40}
   \setbeamercolor{block body}{fg=black,bg=structure!10}
   \begin{block}{{\bf Corollaire }#1}
}{%
   \end{block}%
}

\newenvironment{mydefinition}[1][]{%
   \setbeamercolor{block title}{fg=structure,bg=structure!40}
   \setbeamercolor{block body}{fg=black,bg=structure!10}
   \begin{block}{{\bf Définition} #1}
}{%
   \end{block}%
}

\newenvironment{lemme}[0]{%
   \setbeamercolor{block title}{fg=structure,bg=structure!40}
   \setbeamercolor{block body}{fg=black,bg=structure!10}
   \begin{block}{\bf Lemme}
}{%
   \end{block}%
}

\newenvironment{remarque}[1][]{%
   \setbeamercolor{block title}{fg=black,bg=structure!20}
   \setbeamercolor{block body}{fg=black,bg=structure!5}
   \begin{block}{Remarque #1}
}{%
   \end{block}%
}


\newenvironment{exemple}[1][]{%
   \setbeamercolor{block title}{fg=black,bg=structure!20}
   \setbeamercolor{block body}{fg=black,bg=structure!5}
   \begin{block}{{\bf Exemple }#1}
}{%
   \end{block}%
}


\newenvironment{miniexercice}[0]{%
   \setbeamercolor{block title}{fg=structure,bg=structure!20}
   \setbeamercolor{block body}{fg=black,bg=structure!5}
   \begin{block}{Mini-exercices}
}{%
   \end{block}%
}


\newenvironment{tp}[0]{%
   \setbeamercolor{block title}{fg=structure,bg=structure!40}
   \setbeamercolor{block body}{fg=black,bg=structure!10}
   \begin{block}{\bf Travaux pratiques}
}{%
   \end{block}%
}
\newenvironment{exercicecours}[1][]{%
   \setbeamercolor{block title}{fg=structure,bg=structure!40}
   \setbeamercolor{block body}{fg=black,bg=structure!10}
   \begin{block}{{\bf Exercice }#1}
}{%
   \end{block}%
}
\newenvironment{algo}[1][]{%
   \setbeamercolor{block title}{fg=structure,bg=structure!40}
   \setbeamercolor{block body}{fg=black,bg=structure!10}
   \begin{block}{{\bf Algorithme}\hfill{\color{gray}\texttt{#1}}}
}{%
   \end{block}%
}


\setbeamertemplate{proof begin}{
   \setbeamercolor{block title}{fg=black,bg=structure!20}
   \setbeamercolor{block body}{fg=black,bg=structure!5}
   \begin{block}{{\footnotesize Démonstration}}
   \footnotesize
   \smallskip}
\setbeamertemplate{proof end}{%
   \end{block}}
\setbeamertemplate{qed symbol}{\openbox}


\makeatother
\usecolortheme[RGB={150,93,42}]{structure}

   
%%%%%%%%%%%%%%%%%%%%%%%%%%%%%%%%%%%%%%%%%%%%%%%%%%%%%%%%%%%%%
%%%%%%%%%%%%%%%%%%%%%%%%%%%%%%%%%%%%%%%%%%%%%%%%%%%%%%%%%%%%%


\begin{document}


\title{{\bf Dimension finie}}
\subtitle{Famille génératrice}

\begin{frame}
  
  \debutmontitre

  \pause

{\footnotesize
\hfill
\setbeamercovered{transparent=50}
\begin{minipage}{0.6\textwidth}
  \begin{itemize}
    \item<3-> Définition
    \item<4-> Exemples
    \item<5-> Liens entre familles génératrices
  \end{itemize}
\end{minipage}
}

\end{frame}

\setcounter{framenumber}{0}


%%%%%%%%%%%%%%%%%%%%%%%%%%%%%%%%%%%%%%%%%%%%%%%%%%%%%%%%%%%%%%%%
\section{Définition}

\begin{frame}
Soit $E$ un $\Kk$-espace vectoriel

Soient $v_1,\dots ,v_p$ des vecteurs de $E$

\begin{mydefinition}

La famille $\{v_1,\dots ,v_p\}$ est une \defi{famille génératrice} de $E$ 
si tout vecteur de $E$ est une combinaison linéaire des vecteurs $v_1,\dots ,v_p$, c'est-à-dire :

$${\color<2>{red}\forall v \in E } \qquad {\color<3>{red}\exists \lambda_1, \ldots,\lambda_p \in \Kk }\qquad
{\color<4>{red}v=\lambda_1 v_1+\cdots + \lambda_p v_p}$$
\end{mydefinition}

\end{frame}


%%%%%%%%%%%%%%%%%%%%%%%%%%%%%%%%%%%%%%%%%%%%%%%%%%%%%%%%%%%%%%%%
\section{Exemples}

\begin{frame}
\begin{exemple}
\begin{itemize}
\item
$v_1={\color<4>{red}\left(\begin{smallmatrix}1\\0\\0\end{smallmatrix}\right)}\qquad
v_2={\color<5>{red}\left(\begin{smallmatrix}0\\1\\0\end{smallmatrix}\right)}\qquad
v_3={\color<6>{red}\left(\begin{smallmatrix}0\\0\\1\end{smallmatrix}\right)}$ 
\pause
\item
$v=\left(\begin{smallmatrix}x\\y\\z\end{smallmatrix}\right)$
\pause
$
={\color<7>{red}x}{\color<4>{red}\left(\begin{smallmatrix}1\\0\\0\end{smallmatrix}\right)}
+{\color<8>{red}y}{\color<5>{red}\left(\begin{smallmatrix}0\\1\\0\end{smallmatrix}\right)}
+{\color<9>{red}z}{\color<6>{red}\left(\begin{smallmatrix}0\\0\\1\end{smallmatrix}\right)}
$
\pause\pause\pause\pause\pause\pause\pause
\item
La famille $\{v_1,v_2,v_3\}$ est génératrice de $\Rr^3$
\end{itemize}
\end{exemple}
\end{frame}



\begin{frame}
\begin{exemple}
\begin{itemize}
\item
$v_1=\left(\begin{smallmatrix}1\\1\\1\end{smallmatrix}\right)\qquad
v_2=\left(\begin{smallmatrix}1\\2\\3\end{smallmatrix}\right)$

\pause
\item 
La famille $\{v_1, v_2\}$ est-elle g\'en\'eratrice de $\Rr^3$?

\pause
\item
$v=\left(\begin{smallmatrix}0\\1\\0\end{smallmatrix}\right)\in\Rr^3$%\notin\Vect (v_1,v_2)$

\pause
\item

$v=\lambda_1 v_1 + \lambda_2 v_2$

\pause
$\iff\left(\begin{smallmatrix}0\\1\\0\end{smallmatrix}\right)= 
\lambda_1 \left(\begin{smallmatrix}1\\1\\1\end{smallmatrix}\right) + 
\lambda_2 \left(\begin{smallmatrix}1\\2\\3\end{smallmatrix}\right)
\iff
\left\{\begin{array}{rcl} 
\lambda_1 + \lambda_2 & = & 0\\
\lambda_1 + 2\lambda_2 & = & 1\\
\lambda_1 + 3\lambda_2 & = & 0
\end{array}\right.
$

\pause
Ce système n'a {\color{red}pas} de solution

\pause
\item
$\{v_1, v_2\}$ \emph{n'est pas} une famille g\'en\'eratrice de $\Rr^3$ 
\end{itemize}
\end{exemple}
\end{frame}



\begin{frame}
\begin{exemple}
Soit $E=\Rr^2$
\begin{itemize}

\item 

\begin{itemize}
\item
$v_1=\left(\begin{smallmatrix}1\\0\end{smallmatrix}\right)\qquad v_2=\left(\begin{smallmatrix}0\\1\end{smallmatrix}\right)$
 
 \pause
  \item $\{v_1, v_2\}$ est g\'en\'eratrice de $\Rr^2$ car 
  $\left(\begin{smallmatrix}x\\y\end{smallmatrix}\right)=
  x\left(\begin{smallmatrix}1\\0\end{smallmatrix}\right)
  +y\left(\begin{smallmatrix}0\\1\end{smallmatrix}\right)$

  \end{itemize}
  



  \pause
  \item 
  
  \begin{itemize}
  \item
  $v'_1=\left(\begin{smallmatrix}2\\1\end{smallmatrix}\right) \qquad 
  v'_2 = \left(\begin{smallmatrix}1\\1\end{smallmatrix}\right)$
  
  \pause
  \item La famille $\{v_1', v_2'\}$ est-elle g\'en\'eratrice de $\Rr^2$?
   
  \pause 
  \item 
$$v=\begin{pmatrix}x\\y\end{pmatrix}={\color<6>{red}\lambda} v_1' +{\color<6>{red}\mu} v_2'
\iff \left \{ \begin{array}{rcl}
2\lambda + \mu &=& x\\
\lambda+\mu &=& y
\end{array}\right. $$


\pause\pause
\item
$\lambda =x -y\qquad \mu =-x+2y$

\pause
\item $\{v_1', v_2'\}$ est aussi une famille g\'en\'eratrice de $\Rr^2$
\end{itemize}

\pause
\item Il existe plusieurs familles g\'en\'eratrices de $\Rr^2$
\end{itemize}
\end{exemple}
\end{frame}

%%%%%%%%%%%%%%%%%%%%%%%%%%%%%%%%%%%%%%%%%%%%%%%%%%%%%%%%%%%%%%%%
\section{Liens entre familles génératrices}

\begin{frame}
\begin{proposition}
Soit $\mathcal{F} = \left\{ v_1, v_2, \dots , v_p\right\}$ une famille génératrice de $E$.
Alors $\mathcal{F}' = \left\{ v_1', v_2', \dots , v_q'\right\}$ est aussi une famille
génératrice de $E$ si et seulement si tout vecteur de $\mathcal{F}$ 
est une combinaison linéaire de vecteurs de $\mathcal{F}'$
\end{proposition}


\end{frame}


\begin{frame}
\begin{proposition}
Si la famille de vecteurs  $\mathcal{F} =\{v_1,\ldots,v_p\}$ engendre $E$ et si l'un des vecteurs, 
par exemple $v_p$, est combinaison linéaire des autres, alors la famille 
$\mathcal{F}  \setminus \{v_p\}= \{v_1,\dots ,v_{p-1}\}$
est encore  génératrice de $E$
\end{proposition}


\pause
\begin{proof}
\begin{itemize}
\item
$v_1,\dots ,v_p$ engendrent $E$ : $\forall v \in E\quad\exists \lambda_1, \ldots, \lambda_p$
$$v=\lambda_1 v_1+ \cdots + \lambda_p v_p$$

\pause
\item $v_p$ combinaison lin\'eaire de $\{v_1,\dots ,v_{p-1}\}$ : 
$$v_p = \alpha_1 v_1 + \cdots + \alpha_{p-1}v_{p-1}$$


\pause
\item
$v = \lambda_1 v_1+ \cdots + \lambda_{p-1}v_{p-1} + \lambda_{p}
\left( \alpha_1 v_1+\cdots+\alpha_{p-1}v_{p-1} \right)$
$
\iff v=\left ( \lambda_1+\lambda_{p}\alpha_1\right )v_1+\dots + 
 \left ( \lambda_{p-1}+\lambda_{p}\alpha_{p-1}\right )v_{p-1}$
 \end{itemize}
\end{proof}


\end{frame}



%%%%%%%%%%%%%%%%%%%%%%%%%%%%%%%%%%%%%%%%%%%%%%%%%%%%%%%%%%%%%%%%
\section{Mini-exercices}

\begin{frame}
\begin{miniexercice}
\begin{enumerate}
  \item \`A quelle condition sur $t\in\Rr$, la famille
$\left\{ 
  \left(\begin{smallmatrix} 0 \\ t-1 \end{smallmatrix}\right),
  \left(\begin{smallmatrix} t \\ -t \end{smallmatrix}\right)
  \left(\begin{smallmatrix} t^2-t \\ t-1 \end{smallmatrix}\right)
  \right\}$ 
  est une famille génératrice de $\Rr^2$ ?
  
  \item   Même question avec la famille  
  $\left\{ 
  \left(\begin{smallmatrix} 1 \\ 0 \\ t \end{smallmatrix}\right) 
  \left(\begin{smallmatrix} 1 \\ t \\ t^2 \end{smallmatrix}\right) 
  \left(\begin{smallmatrix} 1 \\ t^2 \\ 1 \end{smallmatrix}\right) 
  \right\}$ de $\Rr^3$.  
   
  \item Montrer qu'une famille de vecteurs contenant une famille 
  génératrice est encore une famille génératrice de $E$.
    
  \item Montrer que si $f : E \to F$ est une application linéaire \emph{surjective}
  et que $\{ v_1, \ldots, v_p \}$  est une famille génératrice de $E$ ,
  alors $\big\{ f(v_1), \ldots, f(v_p) \big\}$ est une famille génératrice de $F$.
\end{enumerate}
\end{miniexercice}
\end{frame}

\end{document}