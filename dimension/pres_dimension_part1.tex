
%%%%%%%%%%%%%%%%%% PREAMBULE %%%%%%%%%%%%%%%%%%

\documentclass[aspectratio=169,utf8]{beamer}
%\documentclass[aspectratio=169,handout]{beamer}

\usetheme{Boadilla}
%\usecolortheme{seahorse}
\usecolortheme[RGB={245,66,24}]{structure}
\useoutertheme{infolines}

% packages
\usepackage{amsfonts,amsmath,amssymb,amsthm}
\usepackage[utf8]{inputenc}
\usepackage[T1]{fontenc}
\usepackage{lmodern}

\usepackage[francais]{babel}
\usepackage{fancybox}
\usepackage{graphicx}

\usepackage{float}
\usepackage{xfrac}

%\usepackage[usenames, x11names]{xcolor}
\usepackage{tikz}
\usepackage{pgfplots}
\usepackage{datetime}



%-----  Package unités -----
\usepackage{siunitx}
\sisetup{locale = FR,detect-all,per-mode = symbol}

%\usepackage{mathptmx}
%\usepackage{fouriernc}
%\usepackage{newcent}
%\usepackage[mathcal,mathbf]{euler}

%\usepackage{palatino}
%\usepackage{newcent}
% \usepackage[mathcal,mathbf]{euler}



% \usepackage{hyperref}
% \hypersetup{colorlinks=true, linkcolor=blue, urlcolor=blue,
% pdftitle={Exo7 - Exercices de mathématiques}, pdfauthor={Exo7}}


%section
% \usepackage{sectsty}
% \allsectionsfont{\bf}
%\sectionfont{\color{Tomato3}\upshape\selectfont}
%\subsectionfont{\color{Tomato4}\upshape\selectfont}

%----- Ensembles : entiers, reels, complexes -----
\newcommand{\Nn}{\mathbb{N}} \newcommand{\N}{\mathbb{N}}
\newcommand{\Zz}{\mathbb{Z}} \newcommand{\Z}{\mathbb{Z}}
\newcommand{\Qq}{\mathbb{Q}} \newcommand{\Q}{\mathbb{Q}}
\newcommand{\Rr}{\mathbb{R}} \newcommand{\R}{\mathbb{R}}
\newcommand{\Cc}{\mathbb{C}} 
\newcommand{\Kk}{\mathbb{K}} \newcommand{\K}{\mathbb{K}}

%----- Modifications de symboles -----
\renewcommand{\epsilon}{\varepsilon}
\renewcommand{\Re}{\mathop{\text{Re}}\nolimits}
\renewcommand{\Im}{\mathop{\text{Im}}\nolimits}
%\newcommand{\llbracket}{\left[\kern-0.15em\left[}
%\newcommand{\rrbracket}{\right]\kern-0.15em\right]}

\renewcommand{\ge}{\geqslant}
\renewcommand{\geq}{\geqslant}
\renewcommand{\le}{\leqslant}
\renewcommand{\leq}{\leqslant}
\renewcommand{\epsilon}{\varepsilon}

%----- Fonctions usuelles -----
\newcommand{\ch}{\mathop{\text{ch}}\nolimits}
\newcommand{\sh}{\mathop{\text{sh}}\nolimits}
\renewcommand{\tanh}{\mathop{\text{th}}\nolimits}
\newcommand{\cotan}{\mathop{\text{cotan}}\nolimits}
\newcommand{\Arcsin}{\mathop{\text{arcsin}}\nolimits}
\newcommand{\Arccos}{\mathop{\text{arccos}}\nolimits}
\newcommand{\Arctan}{\mathop{\text{arctan}}\nolimits}
\newcommand{\Argsh}{\mathop{\text{argsh}}\nolimits}
\newcommand{\Argch}{\mathop{\text{argch}}\nolimits}
\newcommand{\Argth}{\mathop{\text{argth}}\nolimits}
\newcommand{\pgcd}{\mathop{\text{pgcd}}\nolimits} 


%----- Commandes divers ------
\newcommand{\ii}{\mathrm{i}}
\newcommand{\dd}{\text{d}}
\newcommand{\id}{\mathop{\text{id}}\nolimits}
\newcommand{\Ker}{\mathop{\text{Ker}}\nolimits}
\newcommand{\Card}{\mathop{\text{Card}}\nolimits}
\newcommand{\Vect}{\mathop{\text{Vect}}\nolimits}
\newcommand{\Mat}{\mathop{\text{Mat}}\nolimits}
\newcommand{\rg}{\mathop{\text{rg}}\nolimits}
\newcommand{\tr}{\mathop{\text{tr}}\nolimits}


%----- Structure des exercices ------

\newtheoremstyle{styleexo}% name
{2ex}% Space above
{3ex}% Space below
{}% Body font
{}% Indent amount 1
{\bfseries} % Theorem head font
{}% Punctuation after theorem head
{\newline}% Space after theorem head 2
{}% Theorem head spec (can be left empty, meaning ‘normal’)

%\theoremstyle{styleexo}
\newtheorem{exo}{Exercice}
\newtheorem{ind}{Indications}
\newtheorem{cor}{Correction}


\newcommand{\exercice}[1]{} \newcommand{\finexercice}{}
%\newcommand{\exercice}[1]{{\tiny\texttt{#1}}\vspace{-2ex}} % pour afficher le numero absolu, l'auteur...
\newcommand{\enonce}{\begin{exo}} \newcommand{\finenonce}{\end{exo}}
\newcommand{\indication}{\begin{ind}} \newcommand{\finindication}{\end{ind}}
\newcommand{\correction}{\begin{cor}} \newcommand{\fincorrection}{\end{cor}}

\newcommand{\noindication}{\stepcounter{ind}}
\newcommand{\nocorrection}{\stepcounter{cor}}

\newcommand{\fiche}[1]{} \newcommand{\finfiche}{}
\newcommand{\titre}[1]{\centerline{\large \bf #1}}
\newcommand{\addcommand}[1]{}
\newcommand{\video}[1]{}

% Marge
\newcommand{\mymargin}[1]{\marginpar{{\small #1}}}

\def\noqed{\renewcommand{\qedsymbol}{}}


%----- Presentation ------
\setlength{\parindent}{0cm}

%\newcommand{\ExoSept}{\href{http://exo7.emath.fr}{\textbf{\textsf{Exo7}}}}

\definecolor{myred}{rgb}{0.93,0.26,0}
\definecolor{myorange}{rgb}{0.97,0.58,0}
\definecolor{myyellow}{rgb}{1,0.86,0}

\newcommand{\LogoExoSept}[1]{  % input : echelle
{\usefont{U}{cmss}{bx}{n}
\begin{tikzpicture}[scale=0.1*#1,transform shape]
  \fill[color=myorange] (0,0)--(4,0)--(4,-4)--(0,-4)--cycle;
  \fill[color=myred] (0,0)--(0,3)--(-3,3)--(-3,0)--cycle;
  \fill[color=myyellow] (4,0)--(7,4)--(3,7)--(0,3)--cycle;
  \node[scale=5] at (3.5,3.5) {Exo7};
\end{tikzpicture}}
}


\newcommand{\debutmontitre}{
  \author{} \date{} 
  \thispagestyle{empty}
  \hspace*{-10ex}
  \begin{minipage}{\textwidth}
    \titlepage  
  \vspace*{-2.5cm}
  \begin{center}
    \LogoExoSept{2.5}
  \end{center}
  \end{minipage}

  \vspace*{-0cm}
  
  % Astuce pour que le background ne soit pas discrétisé lors de la conversion pdf -> png
\begin{tikzpicture}
        \fill[opacity=0,green!60!black] (0,0)--++(0,0)--++(0,0)--++(0,0)--cycle; 
\end{tikzpicture}

% toc S'affiche trop tot :
% \tableofcontents[hideallsubsections, pausesections]
}

\newcommand{\finmontitre}{
  \end{frame}
  \setcounter{framenumber}{0}
} % ne marche pas pour une raison obscure

%----- Commandes supplementaires ------

% \usepackage[landscape]{geometry}
% \geometry{top=1cm, bottom=3cm, left=2cm, right=10cm, marginparsep=1cm
% }
% \usepackage[a4paper]{geometry}
% \geometry{top=2cm, bottom=2cm, left=2cm, right=2cm, marginparsep=1cm
% }

%\usepackage{standalone}


% New command Arnaud -- november 2011
\setbeamersize{text margin left=24ex}
% si vous modifier cette valeur il faut aussi
% modifier le decalage du titre pour compenser
% (ex : ici =+10ex, titre =-5ex

\theoremstyle{definition}
%\newtheorem{proposition}{Proposition}
%\newtheorem{exemple}{Exemple}
%\newtheorem{theoreme}{Théorème}
%\newtheorem{lemme}{Lemme}
%\newtheorem{corollaire}{Corollaire}
%\newtheorem*{remarque*}{Remarque}
%\newtheorem*{miniexercice}{Mini-exercices}
%\newtheorem{definition}{Définition}

% Commande tikz
\usetikzlibrary{calc}
\usetikzlibrary{patterns,arrows}
\usetikzlibrary{matrix}
\usetikzlibrary{fadings} 

%definition d'un terme
\newcommand{\defi}[1]{{\color{myorange}\textbf{\emph{#1}}}}
\newcommand{\evidence}[1]{{\color{blue}\textbf{\emph{#1}}}}
\newcommand{\assertion}[1]{\emph{\og#1\fg}}  % pour chapitre logique
%\renewcommand{\contentsname}{Sommaire}
\renewcommand{\contentsname}{}
\setcounter{tocdepth}{2}



%------ Figures ------

\def\myscale{1} % par défaut 
\newcommand{\myfigure}[2]{  % entrée : echelle, fichier figure
\def\myscale{#1}
\begin{center}
\footnotesize
{#2}
\end{center}}


%------ Encadrement ------

\usepackage{fancybox}


\newcommand{\mybox}[1]{
\setlength{\fboxsep}{7pt}
\begin{center}
\shadowbox{#1}
\end{center}}

\newcommand{\myboxinline}[1]{
\setlength{\fboxsep}{5pt}
\raisebox{-10pt}{
\shadowbox{#1}
}
}

%--------------- Commande beamer---------------
\newcommand{\beameronly}[1]{#1} % permet de mettre des pause dans beamer pas dans poly


\setbeamertemplate{navigation symbols}{}
\setbeamertemplate{footline}  % tiré du fichier beamerouterinfolines.sty
{
  \leavevmode%
  \hbox{%
  \begin{beamercolorbox}[wd=.333333\paperwidth,ht=2.25ex,dp=1ex,center]{author in head/foot}%
    % \usebeamerfont{author in head/foot}\insertshortauthor%~~(\insertshortinstitute)
    \usebeamerfont{section in head/foot}{\bf\insertshorttitle}
  \end{beamercolorbox}%
  \begin{beamercolorbox}[wd=.333333\paperwidth,ht=2.25ex,dp=1ex,center]{title in head/foot}%
    \usebeamerfont{section in head/foot}{\bf\insertsectionhead}
  \end{beamercolorbox}%
  \begin{beamercolorbox}[wd=.333333\paperwidth,ht=2.25ex,dp=1ex,right]{date in head/foot}%
    % \usebeamerfont{date in head/foot}\insertshortdate{}\hspace*{2em}
    \insertframenumber{} / \inserttotalframenumber\hspace*{2ex} 
  \end{beamercolorbox}}%
  \vskip0pt%
}


\definecolor{mygrey}{rgb}{0.5,0.5,0.5}
\setlength{\parindent}{0cm}
%\DeclareTextFontCommand{\helvetica}{\fontfamily{phv}\selectfont}

% background beamer
\definecolor{couleurhaut}{rgb}{0.85,0.9,1}  % creme
\definecolor{couleurmilieu}{rgb}{1,1,1}  % vert pale
\definecolor{couleurbas}{rgb}{0.85,0.9,1}  % blanc
\setbeamertemplate{background canvas}[vertical shading]%
[top=couleurhaut,middle=couleurmilieu,midpoint=0.4,bottom=couleurbas] 
%[top=fondtitre!05,bottom=fondtitre!60]



\makeatletter
\setbeamertemplate{theorem begin}
{%
  \begin{\inserttheoremblockenv}
  {%
    \inserttheoremheadfont
    \inserttheoremname
    \inserttheoremnumber
    \ifx\inserttheoremaddition\@empty\else\ (\inserttheoremaddition)\fi%
    \inserttheorempunctuation
  }%
}
\setbeamertemplate{theorem end}{\end{\inserttheoremblockenv}}

\newenvironment{theoreme}[1][]{%
   \setbeamercolor{block title}{fg=structure,bg=structure!40}
   \setbeamercolor{block body}{fg=black,bg=structure!10}
   \begin{block}{{\bf Th\'eor\`eme }#1}
}{%
   \end{block}%
}


\newenvironment{proposition}[1][]{%
   \setbeamercolor{block title}{fg=structure,bg=structure!40}
   \setbeamercolor{block body}{fg=black,bg=structure!10}
   \begin{block}{{\bf Proposition }#1}
}{%
   \end{block}%
}

\newenvironment{corollaire}[1][]{%
   \setbeamercolor{block title}{fg=structure,bg=structure!40}
   \setbeamercolor{block body}{fg=black,bg=structure!10}
   \begin{block}{{\bf Corollaire }#1}
}{%
   \end{block}%
}

\newenvironment{mydefinition}[1][]{%
   \setbeamercolor{block title}{fg=structure,bg=structure!40}
   \setbeamercolor{block body}{fg=black,bg=structure!10}
   \begin{block}{{\bf Définition} #1}
}{%
   \end{block}%
}

\newenvironment{lemme}[0]{%
   \setbeamercolor{block title}{fg=structure,bg=structure!40}
   \setbeamercolor{block body}{fg=black,bg=structure!10}
   \begin{block}{\bf Lemme}
}{%
   \end{block}%
}

\newenvironment{remarque}[1][]{%
   \setbeamercolor{block title}{fg=black,bg=structure!20}
   \setbeamercolor{block body}{fg=black,bg=structure!5}
   \begin{block}{Remarque #1}
}{%
   \end{block}%
}


\newenvironment{exemple}[1][]{%
   \setbeamercolor{block title}{fg=black,bg=structure!20}
   \setbeamercolor{block body}{fg=black,bg=structure!5}
   \begin{block}{{\bf Exemple }#1}
}{%
   \end{block}%
}


\newenvironment{miniexercice}[0]{%
   \setbeamercolor{block title}{fg=structure,bg=structure!20}
   \setbeamercolor{block body}{fg=black,bg=structure!5}
   \begin{block}{Mini-exercices}
}{%
   \end{block}%
}


\newenvironment{tp}[0]{%
   \setbeamercolor{block title}{fg=structure,bg=structure!40}
   \setbeamercolor{block body}{fg=black,bg=structure!10}
   \begin{block}{\bf Travaux pratiques}
}{%
   \end{block}%
}
\newenvironment{exercicecours}[1][]{%
   \setbeamercolor{block title}{fg=structure,bg=structure!40}
   \setbeamercolor{block body}{fg=black,bg=structure!10}
   \begin{block}{{\bf Exercice }#1}
}{%
   \end{block}%
}
\newenvironment{algo}[1][]{%
   \setbeamercolor{block title}{fg=structure,bg=structure!40}
   \setbeamercolor{block body}{fg=black,bg=structure!10}
   \begin{block}{{\bf Algorithme}\hfill{\color{gray}\texttt{#1}}}
}{%
   \end{block}%
}


\setbeamertemplate{proof begin}{
   \setbeamercolor{block title}{fg=black,bg=structure!20}
   \setbeamercolor{block body}{fg=black,bg=structure!5}
   \begin{block}{{\footnotesize Démonstration}}
   \footnotesize
   \smallskip}
\setbeamertemplate{proof end}{%
   \end{block}}
\setbeamertemplate{qed symbol}{\openbox}


\makeatother
\usecolortheme[RGB={150,93,42}]{structure}

   
%%%%%%%%%%%%%%%%%%%%%%%%%%%%%%%%%%%%%%%%%%%%%%%%%%%%%%%%%%%%%
%%%%%%%%%%%%%%%%%%%%%%%%%%%%%%%%%%%%%%%%%%%%%%%%%%%%%%%%%%%%%


\begin{document}


\title{{\bf Dimension finie}}
\subtitle{Famille libre}

\begin{frame}
  
  \debutmontitre

  \pause

{\footnotesize
\hfill
\setbeamercovered{transparent=50}
\begin{minipage}{0.6\textwidth}
  \begin{itemize}
    \item<3-> Combinaison linéaire
    \item<4-> Définition
    \item<5-> Exemples
    \item<6-> Famille liée
    \item<7-> Interprétation géométrique
  \end{itemize}
\end{minipage}
}

\end{frame}

\setcounter{framenumber}{0}


%%%%%%%%%%%%%%%%%%%%%%%%%%%%%%%%%%%%%%%%%%%%%%%%%%%%%%%%%%%%%%%%
\section{Combinaison linéaire (rappel)}

\begin{frame}
%Soit $E$ un $\Kk$-espace vectoriel
\begin{itemize}
  \item Soient   {\color<2>{red}$v_1, v_2, \ldots, v_p$} 
 des  vecteurs d'un $\Kk$-espace vectoriel $E$ ({\color<3>{red}$p \geq 1$})
  \item Soient {\color<4>{red}$\lambda_1, \lambda_2, \ldots,  \lambda_p$} des éléments de $\Kk$
\end{itemize}
\pause\pause\pause\pause
\begin{mydefinition}
\begin{itemize}
  \item Le vecteur
 {\color<5>{red}$$u={\color<6>{red}\lambda_1} v_1+{\color<6>{red}\lambda_2}v_2+ \cdots + {\color<6>{red}\lambda_p}v_p$$}
est une \defi{combinaison linéaire} des vecteurs $v_1, v_2, \ldots, v_p$
 \pause
  \item Les scalaires $\lambda_1, \lambda_2, \ldots , \lambda_p$ 
 sont les \defi{coefficients} de la combinaison linéaire
\end{itemize}

\end{mydefinition}

\end{frame}




%%%%%%%%%%%%%%%%%%%%%%%%%%%%%%%%%%%%%%%%%%%%%%%%%%%%%%%%%%%%%%%%
\section{Définitions}


\begin{frame}
\begin{mydefinition}

Une famille {\color<2>{red}$\{ v_1, v_2,\ldots, v_p \}$} de $E$ est une \defi{famille libre} (ou 
\defi{linéairement indépendante}) si toute combinaison linéaire nulle
{\color<3>{red}$$\lambda_1 v_1+\lambda_2 v_2+\cdots+\lambda_p v_p=0$$}est telle que tous ses coefficients sont nuls, c'est-à-dire 
{\color<4>{red}$$\lambda_1=0 \quad \lambda_2=0 \quad \ldots \quad \lambda_p=0$$}
\vspace*{-3ex}
\end{mydefinition}

\pause\pause\pause\pause

\begin{remarque}
\begin{itemize}
\item
Si la famille $\{ v_1, v_2,\ldots, v_p \}$ de $E$ n'est pas libre, on dit que la famille est \defi{liée} ou \defi{linéairement dépendante}
\pause
\item 
Dans ce cas, il existe une combinaison lin\'eaire nulle de $\{ v_1, v_2,\ldots, v_p \}$ avec {\color<6>{red}au moins un coefficient non nul}
\end{itemize}
\end{remarque}


\end{frame}


%%%%%%%%%%%%%%%%%%%%%%%%%%%%%%%%%%%%%%%%%%%%%%%%%%%%%%%%%%%%%%%%
\section{Premiers exemples}


\begin{frame}
\begin{exemple}
\begin{itemize}
\item Dans $\Rr^3$, soit
$$
\left\{
{\color<4>{red}\begin{pmatrix}
1\\2\\3
\end{pmatrix}
}, 
{\color<5>{red}
\begin{pmatrix}
4\\5\\6
\end{pmatrix}
}, 
{\color<6>{red}
\begin{pmatrix}
2\\1\\0
\end{pmatrix}
}
\right\}$$
\pause
\item
Est-ce une famille libre ou liée?
\pause
\item
Posons
$${\color<8,9,10>{red}\lambda_1}{\color<4>{red}\left(\begin{smallmatrix}
{\color<8>{red}1}\\{\color<9>{red}2}\\{\color<10>{red}3}
\end{smallmatrix}\right)}
+{\color<8,9,10>{red}\lambda_2} {\color<5>{red}\left(\begin{smallmatrix}
{\color<8>{red}4}\\{\color<9>{red}5}\\{\color<10>{red}6}
\end{smallmatrix}\right)}
+{\color<8,9,10>{red}\lambda_3} {\color<6>{red}\left(\begin{smallmatrix}
{\color<8>{red}2}\\{\color<9>{red}1}\\{\color<10>{red}0}
\end{smallmatrix}\right)} = 
\left(\begin{smallmatrix}
{\color<8>{red}0}\\{\color<9>{red}0}\\{\color<10>{red}0}
\end{smallmatrix}\right)$$
\pause\pause\pause\pause
$$\iff \left \{ \begin{matrix}
{\color<8>{red}\lambda_1}&{\color<8>{red}+}&{\color<8>{red}4\lambda_2}&{\color<8>{red}+}&{\color<8>{red}2 \lambda_{3}}&{\color<8>{red}=}&{\color<8>{red}0 }\cr
{\color<9>{red}2\lambda_1}&{\color<9>{red}+}&{\color<9>{red}5\lambda_2}&{\color<9>{red}+}&{\color<9>{red}\lambda_{3}}&{\color<9>{red}=}&{\color<9>{red}0}\cr
{\color<10>{red}3\lambda_1}&{\color<10>{red}+}&{\color<10>{red}6\lambda_2}&&&{\color<10>{red}=}&{\color<10>{red}0}\cr
\end{matrix}\right .$$
\end{itemize}
\end{exemple}
\end{frame}


\begin{frame}
\begin{exemple}
\begin{itemize}
\setlength{\itemsep}{9pt}
\item
Apr\`es r\'eduction de Gauss
$$\hspace*{-2em}\lambda_1\left(\begin{smallmatrix}
1\\2\\3
\end{smallmatrix}\right)
+\lambda_2\left(\begin{smallmatrix}
4\\5\\6
\end{smallmatrix}\right) 
+\lambda_3\left(\begin{smallmatrix}
2\\1\\0
\end{smallmatrix}\right) = 
\left(\begin{smallmatrix}
0\\0\\0
\end{smallmatrix}\right)
\iff \left \{ \begin{matrix}
\lambda_1&& &-&2 \lambda_{3}&=&0 \cr
&&\lambda_2&+&\lambda_{3}&=&0\cr
\end{matrix}\right .$$
le syst\`eme a une infinité de solutions
\pause
\item
Par exemple $\lambda_3=1 \Rightarrow \lambda_1=2$, $ \lambda_2=-1$

\pause
\item $2\left(\begin{smallmatrix}
1\\2\\3
\end{smallmatrix}\right)
-\left(\begin{smallmatrix}
4\\5\\6
\end{smallmatrix}\right) 
+\left(\begin{smallmatrix}
2\\1\\0
\end{smallmatrix}\right) = 
\left(\begin{smallmatrix}
0\\0\\0
\end{smallmatrix}\right)$
\pause
\item La famille 
$\left\{\left(\begin{smallmatrix}
1\\2\\3
\end{smallmatrix}\right),
\left(\begin{smallmatrix}
4\\5\\6
\end{smallmatrix}\right),
\left(\begin{smallmatrix}
2\\1\\0
\end{smallmatrix}\right)\right\}$ est donc une famille liée  
\end{itemize}
\end{exemple}
\end{frame}







\begin{frame}
\begin{exemple}
\begin{itemize}
\item
$v_1=\left(\begin{smallmatrix}  1\\1\\1 \end{smallmatrix}\right)\quad
v_2 = \left(\begin{smallmatrix} 2\\-1\\0 \end{smallmatrix}\right)\quad
v_3 = \left(\begin{smallmatrix} 2\\1\\1  \end{smallmatrix}\right)$
\pause
\item
La famille $\{v_1, v_2, v_3\}$ est-elle libre ou liée ?
\pause
\item $\lambda_1 v_1+ \lambda_2 v_2 + \lambda_3 v_3 = 0
\pause
\iff \left \{ \begin{matrix}
\lambda_1&+&2\lambda_2&+& 2\lambda_{3}&=&0 \cr
\lambda_1&-&\lambda_2&+&\lambda_{3}&=&0\cr
\lambda_1&&&+&\lambda_{3}&=&0\cr
\end{matrix}\right.$
\pause
\item
On résout ce système et on trouve comme unique solution
$\lambda_1=0, \quad\lambda_2=0,\quad\lambda_{3}=0$
\pause
\item
La famille $\{v_1, v_2, v_3\}$ est donc une famille libre
\end{itemize}
\end{exemple}
\end{frame}


% 
% \begin{frame}
% \begin{exemple}
% \begin{itemize}
% \item
%  $v_1=\left(\begin{matrix}2\\ -1\\ 0\\3\end{matrix}\right)\quad
% v_2=\left(\begin{matrix}1\\2\\5\\ -1\end{matrix}\right)\quad
% v_3 = \left(\begin{matrix}7\\ -1\\     5\\8\end{matrix}\right)$
% \pause
% \item
% $3v_1 + v_2 - v_3 = 0$
% \pause
% \item
% Donc $\{v_1, v_2, v_3\}$ forme une famille liée
% \end{itemize}
% 
% \end{exemple}
% \end{frame}




%%%%%%%%%%%%%%%%%%%%%%%%%%%%%%%%%%%%%%%%%%%%%%%%%%%%%%%%%%%%%%%%
\section{Autres exemples}

\begin{frame}
\begin{exemple}
\begin{itemize}
\item
Dans le $\Rr$-espace vectoriel $\Rr[X]$
$$
\begin{matrix}
P_1(X)&=&1&-&X & &\\
P_2(X)&=&5&+&3X&-&2X^2\\
P_3(X)&=&1&+&3X&-&X^2
\end{matrix}
$$
\pause
\item $3P_1(X) - P_2(X) + 2P_3(X) = 0$
\pause
\item
$\{P_1, P_2, P_3\}$ forme une famille liée 
\end{itemize}
\end{exemple}
\end{frame}


%\begin{frame}
%\begin{exemple}
%\begin{itemize}
%\item
%Dans le $\Rr$-espace vectoriel des fonctions de $\Rr$
%dans $\Rr$, soit
% $$\{\cos, \sin\}$$
% \pause
% \item
%Est-ce une famille libre?
%\pause
%\item
%Posons  $\qquad\lambda \cos+\mu \sin =0$\\
%$\Leftrightarrow \forall x \in \Rr, ~~\lambda \cos(x) + \mu \sin(x)=0.  $\\
%\pause
%\item
%Pour $x=0$, cette égalité donne $\lambda =0$. \\
%Pour 
%$x=\frac{\pi}2$, elle donne $\mu=0$. 
%\pause
%\item
%Donc la famille 
%$\{\cos , \sin\}$ est libre. 
%\pause
%\item
%En revanche $\{\cos^2, \sin^2,1\}$ est liée car $$\cos ^2 + \sin^2 -1 =0.$$
%%Les coefficients de dépendance linéaire sont $\lambda_1=1, \lambda_2=1, \lambda_3=-1$.
%\end{itemize}
%\end{exemple}
%\end{frame}


%%%%%%%%%%%%%%%%%%%%%%%%%%%%%%%%%%%%%%%%%%%%%%%%%%%%%%%%%%%%%%%%
\section{Famille liée}

\begin{frame}
%Soit $E$ un $\Kk$-espace vectoriel.
%Si $v\neq  0$, la famille à un seul vecteur $\{v\}$ est libre (et liée si $v= 0$).
%Considérons le cas particulier d'une famille de deux vecteurs.
\begin{proposition}
La famille $\{ v_1, v_2\}$ est liée si
et seulement si $v_1$ est un multiple de $v_2$ ou bien
$v_2$ est un multiple de $v_1$
\end{proposition}
\pause
\begin{proof}
\begin{itemize}
  \item Si $\{ v_1, v_2\}$ est liée\begin{itemize}
  \item
    il existe une combinaison lin\'eaire nulle 
$\lambda_1 v_1+\lambda_2 v_2= 0$ avec au moins un coefficient non nul 
\pause
\item
Si $\lambda_1\neq0$,
%on peut diviser par $\lambda_1$, ce qui donne 
\ $ v_1=-\frac{\lambda_2}{\lambda_1} v_2$ \ et $ v_1$ est un multiple de $v_2$
\item
\pause
Si \ $\lambda_2\neq 0$, $ v_2=-\frac{\lambda_1}{\lambda_2} v_1$ \ et $v_2$ est un multiple de $ v_1$  
\end{itemize}
  \pause
  \item Réciproquement 
  \begin{itemize}
  \item si $ v_1$ est un multiple de $ v_2$, il
existe   $\mu$ tel que $ v_1=\mu  v_2$
\pause
 \item donc $1 v_1+(-\mu) v_2= 0$
\pause 
\item la famille $\{ v_1, v_2\}$ est liée
\pause
\item
Même conclusion si c'est $v_2$ qui est un multiple de $ v_1$
\end{itemize}
\end{itemize}
\end{proof}
\end{frame}


\begin{frame}
Soit $E$ un $\Kk$-espace vectoriel 
\begin{theoreme}
\label{carac liee}
Une famille $\mathcal{F}=\{v_1, v_2,\ldots, v_p\}$ de $p\ge 2$ 
vecteurs de $E$ est une famille liée si et seulement si 
au moins un des vecteurs de $\mathcal{F}$ est combinaison linéaire 
des autres vecteurs de $\mathcal{F}$
\end{theoreme}
\end{frame}


%%%%%%%%%%%%%%%%%%%%%%%%%%%%%%%%%%%%%%%%%%%%%%%%%%%%%%%%%%%%%%%%
\section{Interprétation géométrique}
%
%\begin{frame}
% %Dans $\Rr^n$, deux vecteurs sont linéairement dépendants si et seulement s'ils sont colinéaires.
%%Ils sont donc sur une même droite vectorielle.
%\myfigure{.5}{
%\begin{tikzpicture}

%       \draw[->,>=latex,thick, gray] (-2,0)--(5.5,0);% node[below,black] {$x$};
%        \draw[->,>=latex,thick, gray] (0,-2.5)--(0,4.5); % node[right,black] {$y$};

      % \draw[dashed,green!60!black] (1.1,2.2)--(5,3.5)--(3.9,1.3);


       \draw[gray] (-5,-1.66)--(5,1.66) ;



       \draw[->,>=latex,thick, myred] (0,0)--(3,1) node[midway, below right] {$v_1$};
       \draw[->,>=latex,thick, myred] (0,0)--(-3/2,-1/2) node[midway, below right] {$v_2$};
      % \draw[->,>=latex,thick, green!60!black] (0,0)--(5,3.5) node[above] {$v=\lambda_1 v_1 + \lambda_2 v_2$};

       \fill (0,0) circle (2pt) node[above] {$0$};
\end{tikzpicture}
 
%}
%
%\pause
%\myfigure{.9}{
%\begin{tikzpicture}
%\fill[green] (0,0)--(0,2)--(3.5,1.5)--(3.5, -0.5);
      \draw[->,>=latex,thick] (0,0)--(2,0) node[below,black] {$e_2$};
       \draw[->,>=latex,thick] (0,0)--(0,2) node[right,black] {$e_3$};  
       \draw[->,>=latex,thick] (0,0)--(-1,-1.25) node[left,black] {$e_1$};   
           
  \draw[->,>=latex,thick, myred] (0,0)--(3,0.5) node[below right] {$v_1$}; 
  \draw[->,>=latex,thick, myred] (0,0)--(.5,1) node[above] {$v_2$}; 
 \draw[->,>=latex,thick, myred] (0,0)--(3.5,1.5) node[above] {$v_3$}; 
   \draw[dashed,myred] (0.5,1)--(3.5,1.5)--(3,0.5); 

\end{tikzpicture}
 
%}
%\end{frame}

\begin{frame}
\myfigure{.4}{
\tikzinput{fig_dimension04} 
}
\pause

\only<2>{
\myfigure{.8}{
\tikzinput{fig_dimension05} 
}
}


\only<3,4>{
\myfigure{.8}{
\tikzinput{fig_dimension05bis} 
}
}

\pause\pause



\begin{proposition}
  Soit $\mathcal{F}=\{ v_1, v_2,\ldots ,v_p\}$ une famille de vecteurs de $\Rr^n$. Si
  $\mathcal{F}$ contient plus de $n$ éléments (c'est-à-dire $p > n$), alors 
  $\mathcal{F}$ est une famille liée
\end{proposition}

\end{frame}



%%%%%%%%%%%%%%%%%%%%%%%%%%%%%%%%%%%%%%%%%%%%%%%%%%%%%%%%%%%%%%%%
\section{Mini-exercices}

\begin{frame}
\begin{miniexercice}
\begin{enumerate}
  \item Pour quelles valeurs de $t\in\Rr$, 
  $\left\{ 
  \left(\begin{smallmatrix} -1 \\ t \end{smallmatrix}\right),
  \left(\begin{smallmatrix} t^2 \\ -t \end{smallmatrix}\right)
  \right\}$ 
  est une famille libre de $\Rr^2$ ?
  Même question avec la famille 
  $\left\{ 
  \left(\begin{smallmatrix} 1 \\ t \\ t^2 \end{smallmatrix}\right) 
  \left(\begin{smallmatrix} t^2 \\ 1 \\ 1 \end{smallmatrix}\right) 
  \left(\begin{smallmatrix} 1 \\ t \\ 1 \end{smallmatrix}\right) 
  \right\}$ de $\Rr^3$.
  
  
  \item Montrer que toute famille contenant une famille liée est liée.

  \item Montrer que toute famille inclue dans une famille libre est libre.
  
  \item Montrer que si $f : E \to F$ est une application linéaire et que 
  $\{ v_1, \ldots, v_p \}$ est une famille liée de $E$, alors 
  $\{ f(v_1), \ldots, f(v_p) \}$ est une famille liée de $F$.
  
  \item Montrer que si $f : E \to F$ est une application linéaire \emph{injective}
  et que $\{ v_1, \ldots, v_p \}$ est une famille libre de $E$, alors 
  $\{ f(v_1), \ldots, f(v_p) \}$ est une famille libre de $F$.  

\end{enumerate}
\end{miniexercice}
\end{frame}

\end{document}