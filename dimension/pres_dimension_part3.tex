
%%%%%%%%%%%%%%%%%% PREAMBULE %%%%%%%%%%%%%%%%%%

\documentclass[aspectratio=169,utf8]{beamer}
%\documentclass[aspectratio=169,handout]{beamer}

\usetheme{Boadilla}
%\usecolortheme{seahorse}
\usecolortheme[RGB={245,66,24}]{structure}
\useoutertheme{infolines}

% packages
\usepackage{amsfonts,amsmath,amssymb,amsthm}
\usepackage[utf8]{inputenc}
\usepackage[T1]{fontenc}
\usepackage{lmodern}

\usepackage[francais]{babel}
\usepackage{fancybox}
\usepackage{graphicx}

\usepackage{float}
\usepackage{xfrac}

%\usepackage[usenames, x11names]{xcolor}
\usepackage{tikz}
\usepackage{pgfplots}
\usepackage{datetime}



%-----  Package unités -----
\usepackage{siunitx}
\sisetup{locale = FR,detect-all,per-mode = symbol}

%\usepackage{mathptmx}
%\usepackage{fouriernc}
%\usepackage{newcent}
%\usepackage[mathcal,mathbf]{euler}

%\usepackage{palatino}
%\usepackage{newcent}
% \usepackage[mathcal,mathbf]{euler}



% \usepackage{hyperref}
% \hypersetup{colorlinks=true, linkcolor=blue, urlcolor=blue,
% pdftitle={Exo7 - Exercices de mathématiques}, pdfauthor={Exo7}}


%section
% \usepackage{sectsty}
% \allsectionsfont{\bf}
%\sectionfont{\color{Tomato3}\upshape\selectfont}
%\subsectionfont{\color{Tomato4}\upshape\selectfont}

%----- Ensembles : entiers, reels, complexes -----
\newcommand{\Nn}{\mathbb{N}} \newcommand{\N}{\mathbb{N}}
\newcommand{\Zz}{\mathbb{Z}} \newcommand{\Z}{\mathbb{Z}}
\newcommand{\Qq}{\mathbb{Q}} \newcommand{\Q}{\mathbb{Q}}
\newcommand{\Rr}{\mathbb{R}} \newcommand{\R}{\mathbb{R}}
\newcommand{\Cc}{\mathbb{C}} 
\newcommand{\Kk}{\mathbb{K}} \newcommand{\K}{\mathbb{K}}

%----- Modifications de symboles -----
\renewcommand{\epsilon}{\varepsilon}
\renewcommand{\Re}{\mathop{\text{Re}}\nolimits}
\renewcommand{\Im}{\mathop{\text{Im}}\nolimits}
%\newcommand{\llbracket}{\left[\kern-0.15em\left[}
%\newcommand{\rrbracket}{\right]\kern-0.15em\right]}

\renewcommand{\ge}{\geqslant}
\renewcommand{\geq}{\geqslant}
\renewcommand{\le}{\leqslant}
\renewcommand{\leq}{\leqslant}
\renewcommand{\epsilon}{\varepsilon}

%----- Fonctions usuelles -----
\newcommand{\ch}{\mathop{\text{ch}}\nolimits}
\newcommand{\sh}{\mathop{\text{sh}}\nolimits}
\renewcommand{\tanh}{\mathop{\text{th}}\nolimits}
\newcommand{\cotan}{\mathop{\text{cotan}}\nolimits}
\newcommand{\Arcsin}{\mathop{\text{arcsin}}\nolimits}
\newcommand{\Arccos}{\mathop{\text{arccos}}\nolimits}
\newcommand{\Arctan}{\mathop{\text{arctan}}\nolimits}
\newcommand{\Argsh}{\mathop{\text{argsh}}\nolimits}
\newcommand{\Argch}{\mathop{\text{argch}}\nolimits}
\newcommand{\Argth}{\mathop{\text{argth}}\nolimits}
\newcommand{\pgcd}{\mathop{\text{pgcd}}\nolimits} 


%----- Commandes divers ------
\newcommand{\ii}{\mathrm{i}}
\newcommand{\dd}{\text{d}}
\newcommand{\id}{\mathop{\text{id}}\nolimits}
\newcommand{\Ker}{\mathop{\text{Ker}}\nolimits}
\newcommand{\Card}{\mathop{\text{Card}}\nolimits}
\newcommand{\Vect}{\mathop{\text{Vect}}\nolimits}
\newcommand{\Mat}{\mathop{\text{Mat}}\nolimits}
\newcommand{\rg}{\mathop{\text{rg}}\nolimits}
\newcommand{\tr}{\mathop{\text{tr}}\nolimits}


%----- Structure des exercices ------

\newtheoremstyle{styleexo}% name
{2ex}% Space above
{3ex}% Space below
{}% Body font
{}% Indent amount 1
{\bfseries} % Theorem head font
{}% Punctuation after theorem head
{\newline}% Space after theorem head 2
{}% Theorem head spec (can be left empty, meaning ‘normal’)

%\theoremstyle{styleexo}
\newtheorem{exo}{Exercice}
\newtheorem{ind}{Indications}
\newtheorem{cor}{Correction}


\newcommand{\exercice}[1]{} \newcommand{\finexercice}{}
%\newcommand{\exercice}[1]{{\tiny\texttt{#1}}\vspace{-2ex}} % pour afficher le numero absolu, l'auteur...
\newcommand{\enonce}{\begin{exo}} \newcommand{\finenonce}{\end{exo}}
\newcommand{\indication}{\begin{ind}} \newcommand{\finindication}{\end{ind}}
\newcommand{\correction}{\begin{cor}} \newcommand{\fincorrection}{\end{cor}}

\newcommand{\noindication}{\stepcounter{ind}}
\newcommand{\nocorrection}{\stepcounter{cor}}

\newcommand{\fiche}[1]{} \newcommand{\finfiche}{}
\newcommand{\titre}[1]{\centerline{\large \bf #1}}
\newcommand{\addcommand}[1]{}
\newcommand{\video}[1]{}

% Marge
\newcommand{\mymargin}[1]{\marginpar{{\small #1}}}

\def\noqed{\renewcommand{\qedsymbol}{}}


%----- Presentation ------
\setlength{\parindent}{0cm}

%\newcommand{\ExoSept}{\href{http://exo7.emath.fr}{\textbf{\textsf{Exo7}}}}

\definecolor{myred}{rgb}{0.93,0.26,0}
\definecolor{myorange}{rgb}{0.97,0.58,0}
\definecolor{myyellow}{rgb}{1,0.86,0}

\newcommand{\LogoExoSept}[1]{  % input : echelle
{\usefont{U}{cmss}{bx}{n}
\begin{tikzpicture}[scale=0.1*#1,transform shape]
  \fill[color=myorange] (0,0)--(4,0)--(4,-4)--(0,-4)--cycle;
  \fill[color=myred] (0,0)--(0,3)--(-3,3)--(-3,0)--cycle;
  \fill[color=myyellow] (4,0)--(7,4)--(3,7)--(0,3)--cycle;
  \node[scale=5] at (3.5,3.5) {Exo7};
\end{tikzpicture}}
}


\newcommand{\debutmontitre}{
  \author{} \date{} 
  \thispagestyle{empty}
  \hspace*{-10ex}
  \begin{minipage}{\textwidth}
    \titlepage  
  \vspace*{-2.5cm}
  \begin{center}
    \LogoExoSept{2.5}
  \end{center}
  \end{minipage}

  \vspace*{-0cm}
  
  % Astuce pour que le background ne soit pas discrétisé lors de la conversion pdf -> png
\begin{tikzpicture}
        \fill[opacity=0,green!60!black] (0,0)--++(0,0)--++(0,0)--++(0,0)--cycle; 
\end{tikzpicture}

% toc S'affiche trop tot :
% \tableofcontents[hideallsubsections, pausesections]
}

\newcommand{\finmontitre}{
  \end{frame}
  \setcounter{framenumber}{0}
} % ne marche pas pour une raison obscure

%----- Commandes supplementaires ------

% \usepackage[landscape]{geometry}
% \geometry{top=1cm, bottom=3cm, left=2cm, right=10cm, marginparsep=1cm
% }
% \usepackage[a4paper]{geometry}
% \geometry{top=2cm, bottom=2cm, left=2cm, right=2cm, marginparsep=1cm
% }

%\usepackage{standalone}


% New command Arnaud -- november 2011
\setbeamersize{text margin left=24ex}
% si vous modifier cette valeur il faut aussi
% modifier le decalage du titre pour compenser
% (ex : ici =+10ex, titre =-5ex

\theoremstyle{definition}
%\newtheorem{proposition}{Proposition}
%\newtheorem{exemple}{Exemple}
%\newtheorem{theoreme}{Théorème}
%\newtheorem{lemme}{Lemme}
%\newtheorem{corollaire}{Corollaire}
%\newtheorem*{remarque*}{Remarque}
%\newtheorem*{miniexercice}{Mini-exercices}
%\newtheorem{definition}{Définition}

% Commande tikz
\usetikzlibrary{calc}
\usetikzlibrary{patterns,arrows}
\usetikzlibrary{matrix}
\usetikzlibrary{fadings} 

%definition d'un terme
\newcommand{\defi}[1]{{\color{myorange}\textbf{\emph{#1}}}}
\newcommand{\evidence}[1]{{\color{blue}\textbf{\emph{#1}}}}
\newcommand{\assertion}[1]{\emph{\og#1\fg}}  % pour chapitre logique
%\renewcommand{\contentsname}{Sommaire}
\renewcommand{\contentsname}{}
\setcounter{tocdepth}{2}



%------ Figures ------

\def\myscale{1} % par défaut 
\newcommand{\myfigure}[2]{  % entrée : echelle, fichier figure
\def\myscale{#1}
\begin{center}
\footnotesize
{#2}
\end{center}}


%------ Encadrement ------

\usepackage{fancybox}


\newcommand{\mybox}[1]{
\setlength{\fboxsep}{7pt}
\begin{center}
\shadowbox{#1}
\end{center}}

\newcommand{\myboxinline}[1]{
\setlength{\fboxsep}{5pt}
\raisebox{-10pt}{
\shadowbox{#1}
}
}

%--------------- Commande beamer---------------
\newcommand{\beameronly}[1]{#1} % permet de mettre des pause dans beamer pas dans poly


\setbeamertemplate{navigation symbols}{}
\setbeamertemplate{footline}  % tiré du fichier beamerouterinfolines.sty
{
  \leavevmode%
  \hbox{%
  \begin{beamercolorbox}[wd=.333333\paperwidth,ht=2.25ex,dp=1ex,center]{author in head/foot}%
    % \usebeamerfont{author in head/foot}\insertshortauthor%~~(\insertshortinstitute)
    \usebeamerfont{section in head/foot}{\bf\insertshorttitle}
  \end{beamercolorbox}%
  \begin{beamercolorbox}[wd=.333333\paperwidth,ht=2.25ex,dp=1ex,center]{title in head/foot}%
    \usebeamerfont{section in head/foot}{\bf\insertsectionhead}
  \end{beamercolorbox}%
  \begin{beamercolorbox}[wd=.333333\paperwidth,ht=2.25ex,dp=1ex,right]{date in head/foot}%
    % \usebeamerfont{date in head/foot}\insertshortdate{}\hspace*{2em}
    \insertframenumber{} / \inserttotalframenumber\hspace*{2ex} 
  \end{beamercolorbox}}%
  \vskip0pt%
}


\definecolor{mygrey}{rgb}{0.5,0.5,0.5}
\setlength{\parindent}{0cm}
%\DeclareTextFontCommand{\helvetica}{\fontfamily{phv}\selectfont}

% background beamer
\definecolor{couleurhaut}{rgb}{0.85,0.9,1}  % creme
\definecolor{couleurmilieu}{rgb}{1,1,1}  % vert pale
\definecolor{couleurbas}{rgb}{0.85,0.9,1}  % blanc
\setbeamertemplate{background canvas}[vertical shading]%
[top=couleurhaut,middle=couleurmilieu,midpoint=0.4,bottom=couleurbas] 
%[top=fondtitre!05,bottom=fondtitre!60]



\makeatletter
\setbeamertemplate{theorem begin}
{%
  \begin{\inserttheoremblockenv}
  {%
    \inserttheoremheadfont
    \inserttheoremname
    \inserttheoremnumber
    \ifx\inserttheoremaddition\@empty\else\ (\inserttheoremaddition)\fi%
    \inserttheorempunctuation
  }%
}
\setbeamertemplate{theorem end}{\end{\inserttheoremblockenv}}

\newenvironment{theoreme}[1][]{%
   \setbeamercolor{block title}{fg=structure,bg=structure!40}
   \setbeamercolor{block body}{fg=black,bg=structure!10}
   \begin{block}{{\bf Th\'eor\`eme }#1}
}{%
   \end{block}%
}


\newenvironment{proposition}[1][]{%
   \setbeamercolor{block title}{fg=structure,bg=structure!40}
   \setbeamercolor{block body}{fg=black,bg=structure!10}
   \begin{block}{{\bf Proposition }#1}
}{%
   \end{block}%
}

\newenvironment{corollaire}[1][]{%
   \setbeamercolor{block title}{fg=structure,bg=structure!40}
   \setbeamercolor{block body}{fg=black,bg=structure!10}
   \begin{block}{{\bf Corollaire }#1}
}{%
   \end{block}%
}

\newenvironment{mydefinition}[1][]{%
   \setbeamercolor{block title}{fg=structure,bg=structure!40}
   \setbeamercolor{block body}{fg=black,bg=structure!10}
   \begin{block}{{\bf Définition} #1}
}{%
   \end{block}%
}

\newenvironment{lemme}[0]{%
   \setbeamercolor{block title}{fg=structure,bg=structure!40}
   \setbeamercolor{block body}{fg=black,bg=structure!10}
   \begin{block}{\bf Lemme}
}{%
   \end{block}%
}

\newenvironment{remarque}[1][]{%
   \setbeamercolor{block title}{fg=black,bg=structure!20}
   \setbeamercolor{block body}{fg=black,bg=structure!5}
   \begin{block}{Remarque #1}
}{%
   \end{block}%
}


\newenvironment{exemple}[1][]{%
   \setbeamercolor{block title}{fg=black,bg=structure!20}
   \setbeamercolor{block body}{fg=black,bg=structure!5}
   \begin{block}{{\bf Exemple }#1}
}{%
   \end{block}%
}


\newenvironment{miniexercice}[0]{%
   \setbeamercolor{block title}{fg=structure,bg=structure!20}
   \setbeamercolor{block body}{fg=black,bg=structure!5}
   \begin{block}{Mini-exercices}
}{%
   \end{block}%
}


\newenvironment{tp}[0]{%
   \setbeamercolor{block title}{fg=structure,bg=structure!40}
   \setbeamercolor{block body}{fg=black,bg=structure!10}
   \begin{block}{\bf Travaux pratiques}
}{%
   \end{block}%
}
\newenvironment{exercicecours}[1][]{%
   \setbeamercolor{block title}{fg=structure,bg=structure!40}
   \setbeamercolor{block body}{fg=black,bg=structure!10}
   \begin{block}{{\bf Exercice }#1}
}{%
   \end{block}%
}
\newenvironment{algo}[1][]{%
   \setbeamercolor{block title}{fg=structure,bg=structure!40}
   \setbeamercolor{block body}{fg=black,bg=structure!10}
   \begin{block}{{\bf Algorithme}\hfill{\color{gray}\texttt{#1}}}
}{%
   \end{block}%
}


\setbeamertemplate{proof begin}{
   \setbeamercolor{block title}{fg=black,bg=structure!20}
   \setbeamercolor{block body}{fg=black,bg=structure!5}
   \begin{block}{{\footnotesize Démonstration}}
   \footnotesize
   \smallskip}
\setbeamertemplate{proof end}{%
   \end{block}}
\setbeamertemplate{qed symbol}{\openbox}


\makeatother
\usecolortheme[RGB={150,93,42}]{structure}
   
%%%%%%%%%%%%%%%%%%%%%%%%%%%%%%%%%%%%%%%%%%%%%%%%%%%%%%%%%%%%%
%%%%%%%%%%%%%%%%%%%%%%%%%%%%%%%%%%%%%%%%%%%%%%%%%%%%%%%%%%%%%


\begin{document}


\title{{\bf Dimension finie}}
\subtitle{Base}

\begin{frame}
  
  \debutmontitre

  \pause

{\footnotesize
\hfill
\setbeamercovered{transparent=50}
\begin{minipage}{0.6\textwidth}
  \begin{itemize}
    \item<3-> Définition
    \item<4-> Exemples
    \item<5-> Existence d'une base
    \item<6-> Théorème de la base incomplète
  \end{itemize}
\end{minipage}
}

\end{frame}

\setcounter{framenumber}{0}


%%%%%%%%%%%%%%%%%%%%%%%%%%%%%%%%%%%%%%%%%%%%%%%%%%%%%%%%%%%%%%%%
\section{Définition}

\begin{frame}
Soit $E$ un $\Kk$-espace vectoriel
\begin{mydefinition} 
Une famille $\mathcal{B}= (v_1, v_2, \dots , v_n)$ de vecteurs de $E$ 
est une \defi{base} de $E$
si $\mathcal{B}$ est une famille libre \evidence{et} génératrice
\end{mydefinition}


\pause

\begin{theoreme}
\label{th:coordonnees}
Soit $\mathcal{B} = (v_1, v_2, \dots , v_n)$ une base de l'espace vectoriel $E$.
Tout vecteur $v \in E$ s'exprime de façon unique comme combinaison 
linéaire d'éléments de $\mathcal{B}$.
Autrement dit, il \evidence{existe} des scalaires $\lambda_1,\ldots,\lambda_n \in \Kk$ 
\evidence{uniques} tels que :
$$v = {\color<4>{red}{\color<3>{red}\lambda_1}} v_1 + {\color<5>{red}{\color<3>{red}\lambda_2}} v_2 + \dots + {\color<3>{red}\lambda_n} v_n$$
\end{theoreme}



\end{frame}



%%%%%%%%%%%%%%%%%%%%%%%%%%%%%%%%%%%%%%%%%%%%%%%%%%%%%%%%%%%%%%%%
\section{Exemples}

\begin{frame}
\begin{exemple}
\begin{enumerate}
  \item 
  $e_1=\left(\begin{smallmatrix}1\\0\end{smallmatrix}\right)\qquad
  e_2 = \left(\begin{smallmatrix}0\\1\end{smallmatrix}\right)\qquad$ forment la
 \defi{base canonique} de $\Rr^2$
 
\only<1>{
 \myfigure{.8}{
\tikzinput{fig_dimension06} }
}
\only<2->{
\myfigure{.8}{
\tikzinput{fig_dimension06bis} }
}


  \pause
  \item $v_1=\left(\begin{smallmatrix}3\\1\end{smallmatrix}\right)\qquad 
  v_2=\left(\begin{smallmatrix}1\\2\end{smallmatrix}\right)\qquad$
forment aussi une base de $\Rr^2$


  \pause
  \item 
$e_1 = \left(\begin{smallmatrix}1\\0\\0\end{smallmatrix}\right)~
  e_2 = \left(\begin{smallmatrix}0\\1\\0\end{smallmatrix}\right)~
  e_3 = \left(\begin{smallmatrix}0\\0\\1\end{smallmatrix}\right)~$
   forment la \defi{base canonique} de $\Rr^3$  
   \pause
$\qquad\quad v = \left(\begin{smallmatrix}a_1\\ a_2\\ a_3\end{smallmatrix}\right)
= a_1 \left(\begin{smallmatrix}1\\0\\0\end{smallmatrix}\right) 
+ a_2 \left(\begin{smallmatrix}0\\1\\0\end{smallmatrix}\right)
+ a_3 \left(\begin{smallmatrix}0\\0\\1\end{smallmatrix}\right)$
\end{enumerate}
\end{exemple}

\end{frame}





\begin{frame}
\begin{exemple}

$v_1 = \left(\begin{smallmatrix}1\\2\\1\end{smallmatrix}\right)~ 
  v_2 = \left(\begin{smallmatrix}2\\9\\0\end{smallmatrix}\right)~
  v_3 = \left(\begin{smallmatrix}3\\3\\4\end{smallmatrix}\right)$
  forment-ils une base de $\Rr^3$ ?

   
\pause
\begin{enumerate}
  \item $\mathcal{B} = (v_1, v_2, v_3)$ est-elle une famille génératrice de $\Rr^3$ ?
\pause  
  \item $\mathcal{B} = (v_1, v_2, v_3)$ est-elle une famille libre de $\Rr^3$ ? 
\end{enumerate}

\pause
\begin{enumerate}
  \item $\mathcal{B} = (v_1, v_2, v_3)$ est-elle une famille génératrice de $\Rr^3$ ?

\pause  
   $v = \left(\begin{smallmatrix}a_1\\ a_2\\ a_3\end{smallmatrix}\right)\in\Rr^3$\pause,
   on cherche $\lambda_1, \lambda_2, \lambda_3 \in \Rr$ tels que 
   
$ v = \left( \begin{smallmatrix}a_1\\ a_2\\ a_3\end{smallmatrix}\right) = \lambda_1 \left(\begin{smallmatrix}1\\2\\1\end{smallmatrix} \right) + \lambda_2
      \left(\begin{smallmatrix}2\\9\\0\end{smallmatrix} \right)+ \lambda_3 \left(\begin{smallmatrix}3\\3\\4\end{smallmatrix}\right)
     \pause
      = \left(\begin{smallmatrix}\lambda_1 + 2\lambda_2 + 3\lambda_3\\ 2\lambda_1 + 9
      \lambda_2 + 3\lambda_3\\ \lambda_1 + 4\lambda_3\end{smallmatrix}\right)
$\pause
$$ \iff
   \left\{
\begin{array}{ccccccc}
\lambda_1 &+ &2\lambda_2 &+ &3\lambda_3 & = & a_1\\
2\lambda_1 &+ &9\lambda_2 &+ &3\lambda_3 & = & a_2\\
\lambda_1 &&& + &4\lambda_3 & = & a_3\\
\end{array} \right.
$$ 
\end{enumerate}
\end{exemple}
\end{frame}



\begin{frame}
\begin{exemple}

$v_1 = \left(\begin{smallmatrix}1\\2\\1\end{smallmatrix}\right)~ 
  v_2 = \left(\begin{smallmatrix}2\\9\\0\end{smallmatrix}\right)~
  v_3 = \left(\begin{smallmatrix}3\\3\\4\end{smallmatrix}\right)$
  forment-ils une base de $\Rr^3$ ?

\begin{enumerate}
  \setcounter{enumi}{1}
  \item $\mathcal{B} = (v_1, v_2, v_3)$ est-elle une famille libre de $\Rr^3$ ? 
\vspace*{-1ex}\pause  
      $$ \lambda_1 v_1+ \lambda_2 v_2 + \lambda_3 v_3= 0 \pause \iff 
\left\{
\begin{array}{ccccccc}
\lambda_1 &+ &2\lambda_2 &+ &3\lambda_3 & = & 0\\
2\lambda_1 &+ &9\lambda_2 &+ &3\lambda_3 & = & 0\\
\lambda_1 &&& + &4\lambda_3 & = & 0\\
\end{array} \right.      
$$
\vspace*{-2ex}
\end{enumerate}
\pause
\begin{itemize}
  \item Les deux systèmes ont la même matrice $A = \left(\begin{smallmatrix}
1 & 2 & 3\\
2 & 9 & 3\\
1 &0 & 4\end{smallmatrix}\right)$
\pause
  \item $A$ est inversible
\pause  
  \item Le premier système admet une solution $(\lambda_1,\lambda_2,\lambda_3)$ 
  quel que soit $(a_1,a_2,a_3)$
\pause  
  \item Le second système admet pour seule solution $(0,0,0)$
\pause  
  \item Conclusion : $\mathcal{B}$ est une famille génératrice et libre : c'est une base
  
\end{itemize}

\end{exemple}
\end{frame}
% \uncover<4,5,6,7,8,9,10,11>{
% {\color<4,5,6,7,8,9,10,11>{red}
%    }\\
% }
%    
%    
%    \only<6,7,8>{\small
% 
% }
%     
% \uncover<9,10,11>{ 
% $$
%    \left\{
% \begin{array}{ccccccc}
% \lambda_1 &+ &2\lambda_2 &+ &3\lambda_3 & = & a_1\\
% 2\lambda_1 &+ &9\lambda_2 &+ &3\lambda_3 & = & a_2\\
% \lambda_1 &&& + &4\lambda_3 & = & a_3\\
% \end{array} \right.
% $$ 
%  }   
%  
% \uncover<4,9,10,11>{{\color<4,9,10,11>{red}
%  }
%  }
%    
%    \only<10>{
%      }
% %     \only<>{
% %  $$ \lambda_1 v_1+ \lambda_2 v_2 + \lambda_3 v_3= 0 \pause \Leftrightarrow \left(\begin{smallmatrix}\lambda_1 + 2\lambda_2 + 3\lambda_3\\ 2\lambda_1 + 9
% %      \lambda_2 + 3\lambda_3\\ \lambda_1 + 4\lambda_3\end{smallmatrix}\right) = \left(\begin{smallmatrix} 0\\0\\0\end{smallmatrix}\right)$$
% %  }
% 
% \uncover<11>{
% $$\left\{
% \begin{array}{ccccccc}
% \lambda_1 &+ &2\lambda_2 &+ &3\lambda_3 & = & 0\\
% 2\lambda_1 &+ &9\lambda_2 &+ &3\lambda_3 & = & 0\\
% \lambda_1 && &+ &4\lambda_3 & = & 0
% \end{array}
% \right.
% $$
% }
% 
% \end{exemple}
% \end{frame}

\begin{frame}
\begin{exemple}
\begin{enumerate}
\item
$
e_1 = \left(\begin{smallmatrix}1\\0\\\vdots\\0\end{smallmatrix} \right)\, 
e_2 = \left(\begin{smallmatrix}0\\1\\\vdots\\0\end{smallmatrix}\right) \, \ldots  \,
e_n = \left(\begin{smallmatrix}0\\\vdots\\0\\1\end{smallmatrix}\right)$
 \defi{base canonique} de $\Kk^n$

 \medskip
\pause
\item $v_1 = \left(\begin{smallmatrix}1\\0\\0\\\vdots\\0\end{smallmatrix}\right) \,
v_2 = \left(\begin{smallmatrix}1\\2\\0\\\vdots\\0\end{smallmatrix}\right) \, \ldots \, 
v_n = \left(\begin{smallmatrix}1\\2\\3\\\vdots\\n\end{smallmatrix}\right)$ \quad base de $\Kk^n$

\end{enumerate}
\end{exemple}
\end{frame}


\begin{frame}
\begin{exemple}{\small
\begin{enumerate}
\item  $(1,X,X^2, \ldots , X^n)$ \defi{base canonique} de $\Rr_n[X]$ 

\pause
\medskip
\item
  $ (1,1+X,1+X+X^2,\ldots,1+X+X^2+\cdots+X^n)$ autre base de $\Rr_n[X]$
  

\end{enumerate}}
\end{exemple}
\end{frame}
%%%%%%%%%%%%%%%%%%%%%%%%%%%%%%%%%%%%%%%%%%%%%%%%%%%%%%%%%%%%%%%%
\section{Existence d'une base}

\begin{frame}

\begin{theoreme}[d'existence d'une base]
Tout espace vectoriel admettant une famille finie génératrice
admet une base
\end{theoreme}


\end{frame}



%%%%%%%%%%%%%%%%%%%%%%%%%%%%%%%%%%%%%%%%%%%%%%%%%%%%%%%%%%%%%%%%
\section{Théorème de la base incomplète}

\begin{frame}
\begin{theoreme}[de la base incomplète]
Soit $E$ un $\Kk$-espace vectoriel admettant une famille génératrice finie

\pause
\begin{enumerate}
  \item \emph{Toute famille libre $\mathcal{L}$ peut être complétée en une base.}
  C'est-à-dire qu'il existe une famille $\mathcal{F}$ telle que 
  $\mathcal{L} \cup \mathcal{F}$ soit une famille libre et génératrice de $E$
  
  \pause
  \item \emph{De toute famille génératrice $\mathcal{G}$ on peut extraire une base de $E$.}
  C'est-à-dire qu'il existe une famille $\mathcal{B} \subset \mathcal{G}$ telle que 
  $\mathcal{B}$ soit une famille libre et génératrice de $E$
\end{enumerate}
\end{theoreme}



\end{frame}





%%%%%%%%%%%%%%%%%%%%%%%%%%%%%%%%%%%%%%%%%%%%%%%%%%%%%%%%%%%%%%%%
\section{Preuves}

\begin{frame}
\begin{theoreme}
\label{th:superbaseincomplete}
Soit $\mathcal{G}$ une famille génératrice finie de $E$ 
et $\mathcal{L}$ une famille libre de $E$. 
Alors il existe un sous-ensemble $\mathcal{F}$ de $\mathcal{G}$ telle que
$\mathcal{L} \cup \mathcal{F}$ soit une base de $E$
\end{theoreme}
\end{frame}





\begin{frame}
\begin{exemple}

$
\begin{array}{ll}
P_1(X)=1  &\uncover<2->{\qquad E = \text{Vect}\big(P_1, P_2, P_3, P_4, P_5\big)}
\\ P_2(X)=X &\uncover<3->{\qquad \mathcal{G} = \{P_1, P_2, P_3, P_4, P_5\}}
\\ P_3(X)=X+1 &\uncover<4->{\qquad \mathcal{L} = \varnothing}
\\P_4(X)=1+X^3&\\ P_5(X)=X-X^3&
\end{array}
$
\\
\pause\pause\pause\pause
\medskip
{\color<5>{red}Cherchons
$\mathcal{F} \subset \mathcal{G}$ telle que $\mathcal{F}$ soit une base de $E$}

\pause
\begin{itemize} 
  \item \'Etape 0.  $\mathcal{L}$ n'est pas génératrice, 
  on passe à l'étape suivante

  \pause
  \item \'Etape 1. $ \mathcal{L} \cup \{ P_1 \} = \{ P_1 \}$ famille libre mais pas génératrice  
  \pause
  \item \'Etape 2. 
 $\{ P_1, P_2 \}$ est une famille libre mais pas génératrice 
  
  \pause
  \item \'Etape 3. \begin{itemize}
  \item
  $\{P_1, P_2, P_3 \}$ est une famille liée car $P_3 =  P_1 + P_2$. $P_3$ est exclu
  
  \pause
  \item
   $\{P_1, P_2, P_4\}$ 
  est une famille libre \pause
  et
  aussi une famille génératrice car   $P_5 = P_1+P_2-P_4$
  \pause
  \item l'algorithme s'arr\^ete et $\mathcal{F} = \{P_1, P_2, P_4\}$ est une base de $E$
  \end{itemize}
\end{itemize}  
\end{exemple}

\end{frame}


%%%%%%%%%%%%%%%%%%%%%%%%%%%%%%%%%%%%%%%%%%%%%%%%%%%%%%%%%%%%%%%%
\section{Mini-exercices}

\begin{frame}
\begin{miniexercice}
\begin{enumerate}
  \item Trouver toutes les façons d'obtenir une base de $\Rr^2$ 
  avec les vecteurs suivants :
  $v_1=\left(\begin{smallmatrix}-1\\-3\end{smallmatrix}\right)$, 
  $v_2=\left(\begin{smallmatrix}3\\3\end{smallmatrix}\right)$, 
  $v_3=\left(\begin{smallmatrix}0\\0\end{smallmatrix}\right)$, 
  $v_4=\left(\begin{smallmatrix}2\\0\end{smallmatrix}\right)$, 
  $v_5=\left(\begin{smallmatrix}2\\6\end{smallmatrix}\right)$.
          
  
  \item Montrer que la famille $\{ v_1, v_2, v_3, v_4\}$ des vecteurs
  $v_1=\left(\begin{smallmatrix}2\\1\\-3\end{smallmatrix}\right)$,
  $v_2=\left(\begin{smallmatrix}2\\3\\-1\end{smallmatrix}\right)$,
  $v_3=\left(\begin{smallmatrix}-1\\2\\4\end{smallmatrix}\right)$,
  $v_4=\left(\begin{smallmatrix}1\\1\\-1\end{smallmatrix}\right)$ 
  est une famille génératrice du sous-espace vectoriel d'équation $2x-y+z=0$ de $\Rr^3$.
  En extraire une base.

  \item Déterminer une base du sous-espace vectoriel $E_1$ de $\Rr^3$ d'équation
  $x+3y-2z=0$. Compléter cette base en une base de $\Rr^3$. 
  Idem avec $E_2$ vérifiant les deux équations $x+3y-2z=0$ et $y=z$.
  
  \item Donner une base de l'espace vectoriel des matrices $3\times 3$ ayant une diagonale
  nulle. Idem avec l'espace vectoriel des polynômes $P \in \Rr_n[X]$ vérifiant 
  $P(0)=0$, $P'(0)=0$. 
  
\end{enumerate}
\end{miniexercice}
\end{frame}

\end{document}