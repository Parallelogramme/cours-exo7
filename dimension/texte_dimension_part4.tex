
%%%%%%%%%%%%%%%%%% PREAMBULE %%%%%%%%%%%%%%%%%%


\documentclass[12pt]{article}

\usepackage{amsfonts,amsmath,amssymb,amsthm}
\usepackage[utf8]{inputenc}
\usepackage[T1]{fontenc}
\usepackage[francais]{babel}


% packages
\usepackage{amsfonts,amsmath,amssymb,amsthm}
\usepackage[utf8]{inputenc}
\usepackage[T1]{fontenc}
%\usepackage{lmodern}

\usepackage[francais]{babel}
\usepackage{fancybox}
\usepackage{graphicx}

\usepackage{float}

%\usepackage[usenames, x11names]{xcolor}
\usepackage{tikz}
\usepackage{datetime}

\usepackage{mathptmx}
%\usepackage{fouriernc}
%\usepackage{newcent}
\usepackage[mathcal,mathbf]{euler}

%\usepackage{palatino}
%\usepackage{newcent}


% Commande spéciale prompteur

%\usepackage{mathptmx}
%\usepackage[mathcal,mathbf]{euler}
%\usepackage{mathpple,multido}

\usepackage[a4paper]{geometry}
\geometry{top=2cm, bottom=2cm, left=1cm, right=1cm, marginparsep=1cm}

\newcommand{\change}{{\color{red}\rule{\textwidth}{1mm}\\}}

\newcounter{mydiapo}

\newcommand{\diapo}{\newpage
\hfill {\normalsize  Diapo \themydiapo \quad \texttt{[\jobname]}} \\
\stepcounter{mydiapo}}


%%%%%%% COULEURS %%%%%%%%%%

% Pour blanc sur noir :
%\pagecolor[rgb]{0.5,0.5,0.5}
% \pagecolor[rgb]{0,0,0}
% \color[rgb]{1,1,1}



%\DeclareFixedFont{\myfont}{U}{cmss}{bx}{n}{18pt}
\newcommand{\debuttexte}{
%%%%%%%%%%%%% FONTES %%%%%%%%%%%%%
\renewcommand{\baselinestretch}{1.5}
\usefont{U}{cmss}{bx}{n}
\bfseries

% Taille normale : commenter le reste !
%Taille Arnaud
%\fontsize{19}{19}\selectfont

% Taille Barbara
%\fontsize{21}{22}\selectfont

%Taille François
\fontsize{25}{30}\selectfont

%Taille Pascal
%\fontsize{25}{30}\selectfont

%Taille Laura
%\fontsize{30}{35}\selectfont


%\myfont
%\usefont{U}{cmss}{bx}{n}

%\Huge
%\addtolength{\parskip}{\baselineskip}
}


% \usepackage{hyperref}
% \hypersetup{colorlinks=true, linkcolor=blue, urlcolor=blue,
% pdftitle={Exo7 - Exercices de mathématiques}, pdfauthor={Exo7}}


%section
% \usepackage{sectsty}
% \allsectionsfont{\bf}
%\sectionfont{\color{Tomato3}\upshape\selectfont}
%\subsectionfont{\color{Tomato4}\upshape\selectfont}

%----- Ensembles : entiers, reels, complexes -----
\newcommand{\Nn}{\mathbb{N}} \newcommand{\N}{\mathbb{N}}
\newcommand{\Zz}{\mathbb{Z}} \newcommand{\Z}{\mathbb{Z}}
\newcommand{\Qq}{\mathbb{Q}} \newcommand{\Q}{\mathbb{Q}}
\newcommand{\Rr}{\mathbb{R}} \newcommand{\R}{\mathbb{R}}
\newcommand{\Cc}{\mathbb{C}} 
\newcommand{\Kk}{\mathbb{K}} \newcommand{\K}{\mathbb{K}}

%----- Modifications de symboles -----
\renewcommand{\epsilon}{\varepsilon}
\renewcommand{\Re}{\mathop{\text{Re}}\nolimits}
\renewcommand{\Im}{\mathop{\text{Im}}\nolimits}
%\newcommand{\llbracket}{\left[\kern-0.15em\left[}
%\newcommand{\rrbracket}{\right]\kern-0.15em\right]}

\renewcommand{\ge}{\geqslant}
\renewcommand{\geq}{\geqslant}
\renewcommand{\le}{\leqslant}
\renewcommand{\leq}{\leqslant}

%----- Fonctions usuelles -----
\newcommand{\ch}{\mathop{\mathrm{ch}}\nolimits}
\newcommand{\sh}{\mathop{\mathrm{sh}}\nolimits}
\renewcommand{\tanh}{\mathop{\mathrm{th}}\nolimits}
\newcommand{\cotan}{\mathop{\mathrm{cotan}}\nolimits}
\newcommand{\Arcsin}{\mathop{\mathrm{Arcsin}}\nolimits}
\newcommand{\Arccos}{\mathop{\mathrm{Arccos}}\nolimits}
\newcommand{\Arctan}{\mathop{\mathrm{Arctan}}\nolimits}
\newcommand{\Argsh}{\mathop{\mathrm{Argsh}}\nolimits}
\newcommand{\Argch}{\mathop{\mathrm{Argch}}\nolimits}
\newcommand{\Argth}{\mathop{\mathrm{Argth}}\nolimits}
\newcommand{\pgcd}{\mathop{\mathrm{pgcd}}\nolimits} 

\newcommand{\Card}{\mathop{\text{Card}}\nolimits}
\newcommand{\Ker}{\mathop{\text{Ker}}\nolimits}
\newcommand{\id}{\mathop{\text{id}}\nolimits}
\newcommand{\ii}{\mathrm{i}}
\newcommand{\dd}{\mathrm{d}}
\newcommand{\Vect}{\mathop{\text{Vect}}\nolimits}
\newcommand{\Mat}{\mathop{\mathrm{Mat}}\nolimits}
\newcommand{\rg}{\mathop{\text{rg}}\nolimits}
\newcommand{\tr}{\mathop{\text{tr}}\nolimits}
\newcommand{\ppcm}{\mathop{\text{ppcm}}\nolimits}

%----- Structure des exercices ------

\newtheoremstyle{styleexo}% name
{2ex}% Space above
{3ex}% Space below
{}% Body font
{}% Indent amount 1
{\bfseries} % Theorem head font
{}% Punctuation after theorem head
{\newline}% Space after theorem head 2
{}% Theorem head spec (can be left empty, meaning ‘normal’)

%\theoremstyle{styleexo}
\newtheorem{exo}{Exercice}
\newtheorem{ind}{Indications}
\newtheorem{cor}{Correction}


\newcommand{\exercice}[1]{} \newcommand{\finexercice}{}
%\newcommand{\exercice}[1]{{\tiny\texttt{#1}}\vspace{-2ex}} % pour afficher le numero absolu, l'auteur...
\newcommand{\enonce}{\begin{exo}} \newcommand{\finenonce}{\end{exo}}
\newcommand{\indication}{\begin{ind}} \newcommand{\finindication}{\end{ind}}
\newcommand{\correction}{\begin{cor}} \newcommand{\fincorrection}{\end{cor}}

\newcommand{\noindication}{\stepcounter{ind}}
\newcommand{\nocorrection}{\stepcounter{cor}}

\newcommand{\fiche}[1]{} \newcommand{\finfiche}{}
\newcommand{\titre}[1]{\centerline{\large \bf #1}}
\newcommand{\addcommand}[1]{}
\newcommand{\video}[1]{}

% Marge
\newcommand{\mymargin}[1]{\marginpar{{\small #1}}}



%----- Presentation ------
\setlength{\parindent}{0cm}

%\newcommand{\ExoSept}{\href{http://exo7.emath.fr}{\textbf{\textsf{Exo7}}}}

\definecolor{myred}{rgb}{0.93,0.26,0}
\definecolor{myorange}{rgb}{0.97,0.58,0}
\definecolor{myyellow}{rgb}{1,0.86,0}

\newcommand{\LogoExoSept}[1]{  % input : echelle
{\usefont{U}{cmss}{bx}{n}
\begin{tikzpicture}[scale=0.1*#1,transform shape]
  \fill[color=myorange] (0,0)--(4,0)--(4,-4)--(0,-4)--cycle;
  \fill[color=myred] (0,0)--(0,3)--(-3,3)--(-3,0)--cycle;
  \fill[color=myyellow] (4,0)--(7,4)--(3,7)--(0,3)--cycle;
  \node[scale=5] at (3.5,3.5) {Exo7};
\end{tikzpicture}}
}



\theoremstyle{definition}
%\newtheorem{proposition}{Proposition}
%\newtheorem{exemple}{Exemple}
%\newtheorem{theoreme}{Théorème}
\newtheorem{lemme}{Lemme}
\newtheorem{corollaire}{Corollaire}
%\newtheorem*{remarque*}{Remarque}
%\newtheorem*{miniexercice}{Mini-exercices}
%\newtheorem{definition}{Définition}




%definition d'un terme
\newcommand{\defi}[1]{{\color{myorange}\textbf{\emph{#1}}}}
\newcommand{\evidence}[1]{{\color{blue}\textbf{\emph{#1}}}}



 %----- Commandes divers ------

\newcommand{\codeinline}[1]{\texttt{#1}}

%%%%%%%%%%%%%%%%%%%%%%%%%%%%%%%%%%%%%%%%%%%%%%%%%%%%%%%%%%%%%
%%%%%%%%%%%%%%%%%%%%%%%%%%%%%%%%%%%%%%%%%%%%%%%%%%%%%%%%%%%%%


\begin{document}

\debuttexte


%%%%%%%%%%%%%%%%%%%%%%%%%%%%%%%%%%%%%%%%%%%%%%%%%%%%%%%%%%%
\diapo  

\change
Dans cette le\c{c}on nous allons définir ce qu'est la dimension d'un espace vectoriel. La le\c{c}on s'articule comme suit :

\change
tout d'abord nous donnerons la définition de la dimension d'un espace vectoriel

\change
puis nous examinerons quelques exemples

\change
et enfin nous donnerons quelques compléments sur les notions introduites.

%%%%%%%%%%%%%%%%%%%%%%%%%%%%%%%%%%%%%%%%%%%%%%%%%%%%%%%%%%%
\diapo


Dans toute cette le\c{c}on nous allons nous restreindre au cas des 
espaces vectoriels appelés de dimension finie :
Un espace vectoriel est dit de \defi{dimension finie} s'il 
possède une base ayant un nombre fini d'éléments.\\

Remarquons que par le théorème d'existence d'une base, cela est équivalent à l'existence d'une 
famille finie génératrice.


\change


On va pouvoir parler de \defi{la} dimension d'un espace vectoriel grâce au théorème suivant :

Toutes les bases d'un espace vectoriel $E$ de dimension 
finie ont le même nombre d'éléments. 



\change

Ce nombre d'éléments est ce qu'on appelle la \defi{dimension} de $E$. On la note $\dim E$.\\



Ainsi pour déterminer la dimension d'un espace vectoriel, 
il suffit de trouver une base de $E$.
Le théorème  de la dimension prouve que toute autre base a le même nombre d'éléments.\\

%Ramarquons aussi que par convention, on attribue à l'espace vectoriel réduit au vecteur nul la dimension $0$.


%%%%%%%%%%%%%%%%%%%%%%%%%%%%%%%%%%%%%%%%%%%%%%%%%%%%%%%%%%%
\diapo
Passons aux exemples.

 La base canonique de $\Rr^2$ est constituée des vecteurs de coordonnées 
$\left( 
\left(\begin{smallmatrix} 1\\0 \end{smallmatrix}\right),
\left(\begin{smallmatrix} 0\\1 \end{smallmatrix}\right)
\right)$. La dimension de $\Rr^2$ est donc $2$.

\change
  
 Les vecteurs de coordonnées $\left(
  \left(\begin{smallmatrix}2\\1\end{smallmatrix}\right),
  \left(\begin{smallmatrix}1\\1\end{smallmatrix}\right) \right)$
  forment aussi une base de $\Rr^2$, et illustrent qu'une autre base contient 
  le même nombre d'éléments.
  
  \change
  Plus généralement, $\Kk^n$ est de dimension $n$, car  la base canonique de $\Kk^n$
   contient $n$ éléments.
  
  \change
  La dimension de l'espace vectoriel des polynômes de degré 
  inférieur ou égal à $n$ est *$n+1$* car la base formée des vecteurs 
    $(1,X,X^2,\ldots,X^n)$ contient $n+1$ éléments.

Bien s\^ur, il existe aussi des espaces vectoriels qui ne sont pas de dimension finie, comme par exemple, l'espace vectoriel de tous les polynômes,
 l'espace vectoriel des fonctions de $\Rr$ dans $\Rr$, 
 o\`u l'espace vectoriel des suites réelles. 
%  Ces espaces possèdent également des bases, 
%  mais l'existence de base est plus délicate à montrer et sort du cadre de ce cours.


%%%%%%%%%%%%%%%%%%%%%%%%%%%%%%%%%%%%%%%%%%%%%%%%%%%%%%%%%%%
\diapo
Nous avons vu que l'ensemble des solutions d'un système 
d'équations linéaires \evidence{homogène} est un espace vectoriel. 
On considère par exemple le système suivant

\change
On vérifie que la solution générale de ce système est donnée par $$ x_1 = -s-t\qquad x_2 = s\qquad x_3 = -t\qquad x_4 = 0\qquad x_5 = t\, $$
avec $s$ et $t$ des paramètres réels.


\change
Ainsi les vecteurs solutions s'écrivent sous la forme suivante : 

\change
on peut isoler les paramètres $s$ et $t$

\change

\change

Un vecteur solution est donc combinaison linéaire des vecteurs  $v_1$ et $v_2$ suivants. 
Ceci montre que les vecteurs $v_1$ et $v_2$
 engendrent l'espace des solutions du système.  
D'autre part, on vérifie que $v_1$ et $v_2$ ne sont par colinéaires. 
Donc $(v_1, v_2)$ est une base de l'espace des solutions du système. 
Ceci montre que l'espace vectoriel des solutions du système est de dimension $2$.  


%%%%%%%%%%%%%%%%%%%%%%%%%%%%%%%%%%%%%%%%%%%%%%%%%%%%%%%%%%%
\diapo
Lorsqu'un espace vectoriel est de dimension finie, 
le fait de connaître sa dimension est une information très importante. Nous allons voir quelques propriétés 
qui permettent d'exploiter cette information.\\

Tout d'abord, dans un espace vectoriel de dimension finie, 
le nombre d'éléments d'une famille libre est toujours 
inférieur ou égal au nombre d'éléments d'une famille génératrice.

Lemme : "Le cardinal d'une famille libre $\mathcal{L}$ 
est inférieur ou égal au cardinal d'une famille génératrice $\mathcal{G}$"


\change
Cela a pour conséquence très importante 
que dans un espace vectoriel de dimension $n$, 

\change
(1) n'importe quelle famille libre a au plus $n$ éléments, 

\change
(2) n'importe quelle famille génératrice à au moins $n$ éléments. \\

% En particulier, une base étant une famille libre, elle possède au plus $n$ éléments, \\
% 
% et étant une famille génératrice, elle possède au moins $n$ éléments. \\
% 
% En conséquence, toute base de $E$ possède exactement $n$ éléments. C'est le contenu du théorème de la dimension 
% que nous avons vu précédemment.

%%%%%%%%%%%%%%%%%%%%%%%%%%%%%%%%%%%%%%%%%%%%%%%%%%%%%%%%%%%
\diapo
Il nous reste à énoncer un résultat important et très utile. Lorsqu'on connait la dimension $n$ d'un espace vectoriel il est plus facile de montrer qu'une famille de $n$ vecteurs est une base. \\

En effet, si la dimension de $E$ est $n$ et si la famille possède exactement $n$ éléments, il y a équivalence entre :\\

$\mathcal{F}$ est une base \\

$\mathcal{F}$ est une famille libre \\
 
 $\mathcal{F}$ est une famille génératrice \\
 
 Autrement dit, lorsque le nombre de vecteurs considéré est 
exactement égal à la dimension de l'espace vectoriel, 
l'une des deux conditions --~être une famille libre ou bien \^etre une famille génératrice~--
suffit pour que ces vecteurs déterminent une base de $E$.

\change
La démonstration de ce théorème est la suivante. 
Pour montrer qu'une famille libre de $n$ éléments est une base de $E$, nous utilisons le théorème de la base incomplète. En effet d'après le théorème de la base incomplète, il est possible d'ajouter à la famille $\mathcal{F}$ des vecteurs pour en faire une base. Mais une base compte $n$ vecteurs, et $\mathcal{F}$ aussi. Ainsi le nombre de vecteurs à ajouter à $\mathcal{F}$ est nul. Ce qui veut dire que $\mathcal{F}$ est déjà une base.

\change
De m\^eme pour montrer qu'une famille génératrice ayant $n$ éléments est une base de $E$, nous utilisons également le théorème de la base incomplète pour extraire de $\mathcal{F}$ une base. D'après le théorème de la dimension, une base de $E$ possède toujours $n$ éléments, on a donc besoin des $n$ éléments de $\mathcal{F}$, c'est-à-dire de la famille complète,  pour en faire une base. Ainsi les $n$ éléments de $\mathcal{F}$ formaient déjà une base de $E$.

%%%%%%%%%%%%%%%%%%%%%%%%%%%%%%%%%%%%%%%%%%%%%%%%%%%%%%%%%%%
\diapo
Voyons un exemple d'application du théorème précédent.
Considérons les vecteurs $v_1$, $v_2$ et $v_3$ suivants.


\change
Pour quelles valeurs du paramètre  $t\in\Rr$ les vecteurs $(v_1,v_2,v_3)$ suivants forment-ils  une base de $\Rr^3$ ?


\change
Puisque la dimension de $\Rr^3$ est 3, 
  il suffit  de montrer soit que la famille est libre soit que la famille est génératrice.
  Dans la pratique, il est souvent plus facile de vérifier qu'une famille est libre.

\change
On cherche donc pour quelles valeurs de $t$ la famille est libre

\change
Considérons une combinaison linéaire nulle de $v_1$, $v_2$ , $v_3$ avec pour coefficients $\lambda_1$, $\lambda_2$ et $\lambda_3$. 

\change
Ces coefficients sont assujettis au système suivant.

\change
En retranchant à la deuxième ligne la première, nous obtenons que $2\lambda_2 = 0$. Puis en retranchant à la dernière ligne 4 x la première, nous obtenons 
$(t-4)\lambda_2 + (t-4) \lambda_3 = 0$.

\change
Ainsi $\lambda_2$ est nul et la troisième équation se simplifie. \\

Si $t=4$, le système a des solutions non nulles, comme par exemple, $\lambda_1 = 1$, $\lambda_2 = 0$ et $\lambda_3 = -1$. Dans ce cas, la famille $v_1$, $v_2$ , $v_3$ n'est pas libre, et n'est donc pas non plus une base.\\


Si $t$ est différent de $4$ alors $\lambda_3$ est nécessairement nul, ce qui implique, par la première équation que $\lambda_1$ l'est également. En conséquence, dans la cas o\`u $t\neq4$, la famille $v_1$, $v_2$ , $v_3$  est libre.

\change
Par le théorème précédent, c'est alors aussi une famille génératrice de $\Rr^3$, donc une base.
    

%%%%%%%%%%%%%%%%%%%%%%%%%%%%%%%%%%%%%%%%%%%%%%%%%%%%%%%%%%%
\diapo
Voici une liste d'exercices pour vous entrainer et vérifier votre compréhension du cours.

\end{document}
