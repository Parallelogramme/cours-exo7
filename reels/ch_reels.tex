\documentclass[class=report,crop=false]{standalone}
\usepackage[screen]{../exo7book}

\begin{document}

%====================================================================
\chapitre{Les nombres réels}
%====================================================================

\insertvideo{NCWWVven9Cs}{partie 1. L'ensemble des nombres rationnels $\Qq$}

\insertvideo{83z7Bpz7Fzo}{partie 2. Propriétés de $\Rr$}

\insertvideo{rYOyqI9YLgA}{partie 3. Densité de $\Qq$ dans $\Rr$}

\insertvideo{sBnmcj3jTFY}{partie 4. Borne supérieure}

\insertfiche{fic00009.pdf}{Propriétés de $\Rr$}



%%%%%%%%%%%%%%%%%%%%%%%%%%%%%%%%%%%%%%%%%%%%%%%%%%%%%%%%%%%%%%%%
\section*{Motivation}

Voici une introduction, non seulement à ce chapitre sur les nombres réels,
mais aussi aux premiers chapitres de ce cours d'analyse.

Aux temps des Babyloniens (en Mésopotamie de 3000 à 600 avant J.C.)
le système de numération était en base $60$, c'est-à-dire que tous
les nombres étaient exprimés sous la forme $a+\frac{b}{60} + \frac{c}{60^2}+ \cdots$.
On peut imaginer que pour les applications pratiques c'était largement suffisant (par exemple
estimer la surface d'un champ, le diviser en deux parties égales, calculer le rendement par unité de surface,...).
En langage moderne cela correspond à compter uniquement avec des nombres rationnels $\Qq$.

\medskip

Les pythagoriciens (vers 500 avant J.C. en Grèce) montrent que $\sqrt 2$ n'entre pas ce cadre là.
C'est-à-dire que $\sqrt2$ ne peut s'écrire sous la forme $\frac pq$ avec $p$ et $q$ deux entiers.
C'est un double saut conceptuel : d'une part concevoir que $\sqrt 2$ est de nature différente
mais surtout d'en donner une démonstration.

\medskip

Le fil rouge de ce cours va être deux exemples très simples : les nombres $\sqrt{10}$ et $1,10^{1/12}$.
Le premier représente par exemple la diagonale d'un rectangle de base $3$ et de hauteur $1$ ; le second correspond par exemple
au taux d'intérêt mensuel d'un taux annuel de $10\,\%$.
Dans ce premier chapitre vous allez apprendre à montrer que $\sqrt{10}$ n'est pas un nombre rationnel
mais aussi à encadrer $\sqrt{10}$ et $1,10^{1/12}$ entre deux entiers consécutifs.

\medskip

Pour pouvoir calculer des décimales après la virgule, voire des centaines de décimales, nous aurons besoin
d'outils beaucoup plus sophistiqués :
\begin{itemize}
  \item une construction solide des nombres réels,

  \item l'étude des suites et de leur limites,

  \item l'étude des fonctions continues et des fonctions dérivables.
\end{itemize}

Ces trois points sont liés et permettent de répondre à notre problème, car par exemple nous verrons
en étudiant la fonction $f(x)=x^2-10$ que la suite des rationnels $(u_n)$ définie par
$u_0=3$ et $u_{n+1}=\frac12 \left( u_n+\frac{10}{u_n} \right)$ tend très vite vers $\sqrt{10}$.
Cela nous permettra de calculer des centaines de décimales de $\sqrt{10}$ et de certifier qu'elles sont exactes~:
$$\sqrt{10} = 3,1622776601683793319988935444327185337195551393252168\ldots$$


%%%%%%%%%%%%%%%%%%%%%%%%%%%%%%%%%%%%%%%%%%%%%%%%%%%%%%%%%%%%%%%%
\section{L'ensemble des nombres rationnels $\Qq$}

%---------------------------------------------------------------
\subsection{\'Ecriture décimale}


Par définition, l'ensemble des \defi{nombres rationnels}\index{nombre!rationnel} est
\[
\Qq = \left\{ \frac{p}{q}  \mid p\in \Zz, q\in \Nn^*\right\}.
\]

On a noté $\Nn^*=\Nn\setminus\left\{ 0 \right\}$.

Par exemple : $\frac25$ ; $\frac{-7}{10}$ ; $\frac36=\frac12$.

Les nombres décimaux\index{nombre!décimal}, c'est-à-dire les nombres de la forme $\frac{a}{10^n}$,
avec $a\in \Zz$ et $n\in \Nn$, fournissent d'autres exemples :
$$1,234=1234\times 10^{-3}=\frac{1234}{1000} \qquad
0,00345=345\times 10^{-5}=\frac{345}{100\,000}.$$

\begin{proposition}
Un nombre est rationnel si et seulement s'il admet une écriture décimale périodique ou finie.
\end{proposition}

Par exemple :
$$\frac{3}{5}=0,6\qquad \frac{1}{3}=0,3333\ldots \qquad
1,179\,\underleftrightarrow{325}\,\underleftrightarrow{325}\,\underleftrightarrow{325}\ldots $$


Nous n'allons pas donner la démonstration mais le sens direct ($\implies$) repose
sur la division euclidienne.
Pour la réciproque ($\Longleftarrow$) voyons comment cela marche sur un exemple :
Montrons que $x = 12,34\,\underleftrightarrow{2021}\,\underleftrightarrow{2021}\ldots$ est un rationnel.

L'idée est d'abord de faire apparaître la partie périodique juste après la virgule.
Ici la période commence deux chiffres après la virgule, donc on multiplie par $100$ :
\begin{equation}
\label{eq:per1}
100x = 1234,\underleftrightarrow{2021}\,\underleftrightarrow{2021}\ldots
\end{equation}

Maintenant on va décaler tout vers la gauche de la longueur d'une période, donc ici on multiplie
encore par $10\,000$ pour décaler de $4$ chiffres :
\begin{equation}
\label{eq:per2}
10\,000 \times 100  x = 1234 \, 2021,\underleftrightarrow{2021}\ldots
\end{equation}


Les parties après la virgule des deux lignes \eqref{eq:per1} et \eqref{eq:per2}
sont les mêmes, donc si on les soustrait en faisant \eqref{eq:per2}-\eqref{eq:per1} alors les parties décimales s'annulent :
$$10\,000 \times 100 x-100x=12\,342\,021-1234$$
donc $999\,900x=12\,340\,787$ donc
$$x=\frac{12\,340\,787}{999\,900}.$$
Et donc bien sûr $x\in \Qq$.


%---------------------------------------------------------------
\subsection{$\sqrt2$ n'est pas un nombre rationnel}

Il existe des nombres qui ne sont pas rationnels, les \defi{irrationnels}\index{nombre!irrationnel}.
Les nombres irrationnels apparaissent naturellement dans les figures géométriques :
par exemple la diagonale d'un carré de côté $1$ est le nombre irrationnel $\sqrt{2}$ ;
la circonférence d'un cercle de rayon $\frac12$ est $\pi$ qui est également un nombre irrationnel.
Enfin $e=\exp(1)$ est aussi irrationnel.

\myfigure{2}{
\tikzinput{fig_reels01}
\qquad
\tikzinput{fig_reels02}
}


Nous allons prouver que $\sqrt{2}$ n'est pas un nombre rationnel.
\begin{proposition}
\sauteligne
\label{prop:rac2irr}
\mybox{$\sqrt{2}\notin \Qq$}
\end{proposition}

\begin{proof}
Par l'absurde supposons que $\sqrt{2}$ soit un nombre rationnel.
Alors il existe des entiers $p\in \Zz$ et $q\in \Nn^*$ tels que $\sqrt{2}=\frac pq$,
de plus --ce sera important pour la suite-- on suppose que $p$ et $q$ sont premiers entre eux
(c'est-à-dire que la fraction $\frac pq$ est sous une écriture irréductible).

En élevant au carré, l'égalité  $\sqrt{2}=\frac pq$ devient $2q^2=p^2$.
Cette dernière égalité est une égalité d'entiers. L'entier de gauche est pair, donc on en déduit que
$p^2$ est pair ; en terme de divisibilité  $2$ divise $p^2$.

Mais si $2$ divise $p^2$ alors $2$ divise $p$ (cela se prouve par facilement l'absurde).
Donc il existe un entier $p'\in \Zz$ tel que $p=2p'$.

Repartons de l'égalité $2q^2=p^2$ et remplaçons $p$ par $2p'$. Cela donne
$2q^2=4p'^2$. Donc $q^2=2p'^2$. Maintenant cela entraîne que $2$ divise $q^2$
et comme avant alors $2$ divise $q$.

Nous avons prouvé que $2$ divise à la fois $p$ et $q$. Cela rentre en contradiction avec le fait que $p$ et $q$ sont premiers
entre eux. Notre hypothèse de départ est donc fausse : $\sqrt2$ n'est pas un nombre rationnel.
\end{proof}



Comme ce résultat est important en voici une deuxième démonstration, assez différente, mais
toujours par l'absurde.
\begin{proof}[Autre démonstration]
Par l'absurde, supposons $\sqrt{2}=\frac pq$, donc $q\sqrt{2}=p\in \Nn$.
Considérons l'ensemble
$$\mathcal N=\big\{ n\in \Nn^* \mid n\sqrt{2}\in \Nn\big\}.$$

Cet ensemble n'est pas vide car on vient de voir que $q\sqrt{2}=p \in \Nn$ donc $q \in \mathcal{N}$.
Ainsi $\mathcal N$ est une partie non vide de $\Nn$, elle admet donc un plus petit élément $n_0=\min \mathcal N$.

Posons
$$n_1=n_0\sqrt{2}-n_0=n_0(\sqrt{2}-1),$$
 il découle de cette dernière égalité
et de $1<\sqrt{2}<2$ que $0<n_1<n_0$.

De plus $n_1\sqrt{2}=(n_0\sqrt{2}-n_0)\sqrt{2}=2n_0-n_0\sqrt{2}\in \Nn$.
Donc $n_1 \in \mathcal N$ et $n_1<n_0$ : on vient de trouver un élément $n_1$ de $\mathcal{N}$ strictement plus petit que
$n_0$ qui était le minimum. C'est une contradiction.

Notre hypothèse de départ est fausse, donc $\sqrt2 \notin \Qq$.
\end{proof}

\begin{exercicecours}
Montrer que $\sqrt{10} \notin \Qq$.
\end{exercicecours}

\bigskip
\bigskip

On représente souvent les nombres réels sur une \og droite numérique\fg{} :
\myfigure{1}{
\tikzinput{fig_reels03}
}

Il est bon de connaître les premières décimales de certains réels
$\sqrt{2}\simeq 1,4142\ldots$ \quad
$\pi\simeq 3,14159265\ldots$ \quad $e\simeq 2,718\ldots$

\bigskip

Il est souvent pratique de rajouter les deux extrémités à la droite numérique.
\begin{definition}
\[ \overline{\Rr}=\Rr\cup\{-\infty,\infty\}  \]
\end{definition}




%---------------------------------------------------------------
%\subsection{Mini-exercices}

\begin{miniexercices}
\sauteligne
\begin{enumerate}
  \item Montrer que la somme de deux rationnels est un rationnel. Montrer que le produit de
deux rationnels est un rationnel. Montrer que l'inverse d'un rationnel non nul est un rationnel.
Qu'en est-il pour les irrationnels ?

  \item \'Ecrire les nombres suivants sous forme d'une fraction :
$0,1212$ ;\quad  $0,12\, \underleftrightarrow{12}\ldots$ ;\quad  $78,33\,456\,\underleftrightarrow{456}\ldots$

  \item Sachant $\sqrt2\notin \Qq$, montrer $2-3\sqrt2 \notin \Qq$, $1-\frac{1}{\sqrt2} \notin \Qq$.

  \item Notons $D$ l'ensemble des nombres de la forme $\frac{a}{2^n}$ avec $a\in \Zz$ et $n\in \Nn$.
Montrer que $\frac13 \notin D$. Trouver $x\in D$ tel que $1234<x<1234,001$.

  \item Montrer que $\frac{\sqrt2}{\sqrt3} \notin \Qq$.

  \item Montrer que $\log 2 \notin \Qq$
($\log 2$ est le logarithme décimal de $2$ : c'est le nombre réel tel que $10^{\log 2} = 2$).
\end{enumerate}
\end{miniexercices}


%%%%%%%%%%%%%%%%%%%%%%%%%%%%%%%%%%%%%%%%%%%%%%%%%%%%%%%%%%%%%%%%
\section{Propriétés de $\Rr$}
\index{nombre!reel@réel}
%---------------------------------------------------------------
\subsection{Addition et multiplication}

Ce sont les propriétés que vous avez toujours pratiquées. Pour $a,b,c \in \Rr$ on a :

\begin{center}
\begin{tabular}{ll}
$a+b=b+a$ & $a\times b=b\times a$ \\
$0+a=a$ & $1\times a =a\text{ si }a\neq 0$ \\
$a+b=0 \iff a=-b$ & $ab=1 \iff a=\frac{1}{b}$\\
$(a+b)+c=a+(b+c)$ & $(a\times b)\times c=a\times (b\times c)$\\
&\\
$a\times(b+c)=a\times b+a\times c$ & \\
$a\times b=0 \iff (a=0 \text{ ou } b=0)$ &
\end{tabular}
\end{center}

On résume toutes ces propriétés en disant que :
\begin{propriete*}[$\Rr1$]
$(\Rr,+,\times)$ est un \evidence{corps commutatif}.
\end{propriete*}
%---------------------------------------------------------------
\subsection{Ordre sur $\Rr$}

Nous allons voir que les réels sont ordonnés. La notion d'ordre est générale et nous allons définir cette notion sur un ensemble
quelconque. Cependant gardez à l'esprit que pour nous $E=\Rr$ et $\mathcal{R}=\le$.

\begin{definition}
Soit $E$ un ensemble.
\begin{enumerate}
\item Une \defi{relation} $\mathcal R$ sur $E$ est un sous-ensemble
de l'ensemble produit $E\times E$. Pour $(x,y)\in E\times E$,
on dit que $x$ est en relation avec $y$ et on note $x\mathcal R y$ pour dire que $(x,y)\in\mathcal R$.

\item Une relation $\mathcal R$ est une \defi{relation d'ordre}\index{relation d'ordre} si
\begin{itemize}
  \item $\mathcal R$ est \defi{réflexive} : pour tout $x\in E$, $x\mathcal R x$,
  \item $\mathcal R$ est \defi{antisymétrique} : pour tout $x,y\in E$, $\left( x\mathcal R y \text{ et } y\mathcal R x\right) \implies x=y$,
  \item $\mathcal R$ est \defi{transitive} : pour tout $x,y,z\in E$, $\left( x\mathcal R y \text{ et } y\mathcal R z\right) \implies x\mathcal R z$.
\end{itemize}
\end{enumerate}
\end{definition}
\begin{definition}
Une relation d'ordre $\mathcal R$ sur un ensemble $E$ est \defi{totale} si pour tout $x,y\in E$ on a $ x\mathcal R y \text{ ou } y\mathcal R x$. On dit aussi que $(E,\mathcal R)$ est un \defi{ensemble totalement ordonné}.
\end{definition}

\begin{propriete*}[$\Rr2$]
La relation $\leq$ sur $\Rr$ est une relation d'ordre, et de plus, elle est totale.
\end{propriete*}
Nous avons donc :
\begin{itemize}
  \item pour tout $x \in \Rr$, $x \le x$,
  \item pour tout $x,y \in \Rr$, si $x \le y$ et $y\le x$ alors $x=y$,
  \item pour tout $x,y,z \in \Rr$ si $x\le y$ et $y\le z$ alors $x\le z$.
\end{itemize}

\begin{remarque*}
Pour $(x,y)\in \Rr^2$ on a par définition :
\begin{align*}
x \leq y &\iff  y-x \in \Rr_+ \\
x < y &\iff \left(x\leq y \text{ et } x\neq y\right).
\end{align*}
\end{remarque*}

Les opérations de $\Rr$ sont compatibles avec la relation d'ordre $\leq$ au sens suivant, pour des réels $a,b,c,d$ :
\begin{align*}
 \left (a\leq b \text{ et } c\leq d \right) &\implies a+c\leq b+d \\
         \left (a\leq b \text{ et } c \geq 0\right)& \implies a\times c\leq b\times c \\
 \left (a\leq b \text{ et } c \leq 0\right)& \implies a\times c\geq b\times c .
\end{align*}

On définit le maximum de deux réels $a$ et $b$ par :
\[ \max(a,b)=
\begin{cases}
a & \text{ si } a\geq b \\
b & \text{ si } b>a.
\end{cases}
\]

\begin{exercicecours}
Comment définir $\max(a,b,c)$, $\max(a_1,a_2,\ldots,a_n)$ ? Et $\min(a,b)$ ?
\end{exercicecours}


%---------------------------------------------------------------
\subsection{Propriété d'Archimède}
\begin{propriete*}[$\Rr3$, Propriété d'Archimède]
\label{propr:archi}
$\Rr$ est \defi{archimédien}\index{archimedien@archimédien}, c'est-à-dire :
\[ \forall x \in \Rr\;\;\; \exists n \in \Nn \;\;\; n>x\]
\centerline{\og{}Pour tout réel $x$, il existe un entier naturel $n$ strictement plus grand que $x$.\fg{}}
\end{propriete*}

Cette propriété peut sembler évidente, elle est pourtant essentielle puisque elle
permet de définir la partie entière d'un nombre réel :
\begin{proposition}
\label{prop:part_ent}
Soit $x\in \Rr$, il \evidence{existe} un \evidence{unique} entier relatif, la 
\defi{partie entière}\index{partie entiere@partie entière} notée $E(x)$, tel que :
\mybox{$E(x)\leq x <E(x)+1$}
\end{proposition}


\begin{exemple}
\sauteligne
\begin{itemize}
\item $E(2,853)=2$, $E(\pi)=3$, $E(-3,5)=-4$.
\item $E(x)=3 \iff 3\leq x <4$.
\end{itemize}
\end{exemple}
\begin{remarque*}
\sauteligne
\begin{itemize}
\item On note aussi $E(x)=[x]$.
\item Voici le graphe de la fonction partie entière $x\mapsto E(x)$ :

\myfigure{0.8}{
\tikzinput{fig_reels04}
}
\end{itemize}
\end{remarque*}

Pour la démonstration de la proposition \ref{prop:part_ent} il y a deux choses à établir : d'abord qu'un tel entier $E(x)$ existe
et ensuite qu'il est unique.

\begin{proof}
~\\
\textbf{Existence.} Supposons $x\geq 0$, par la propriété d'Archimède
(Propriété $\Rr3$) il existe $n\in \Nn$ tel que $n>x$.
L'ensemble $K=\big\{k\in \Nn \mid k\leq x \big\}$ est donc fini (car pour tout $k$ dans $K$, on a $0 \le k < n$).
Il admet donc un plus grand élément $k_{max}=\max K$. On a alors $k_{max}\leq x$
car $k_{max}\in K$, et $k_{max}+1 > x$ car $k_{max}+1\notin K$.
Donc $k_{max}\leq x< k_{max}+1$ et on prend donc $E(x)=k_{max}$.

\medskip

\textbf{Unicité.} Si $k$ et $\ell$ sont deux entiers relatifs vérifiant
$k\leq x< k+1$ et $\ell\leq x< \ell+1$, on a donc $k \leq x < \ell+1$, donc par
transitivité $k<\ell+1$. En échangeant les rôles de $\ell$ et $k$,
on a aussi $\ell<k+1$. On en conclut que $\ell-1<k<\ell+1$, mais il n'y a qu'un seul entier compris strictement entre $\ell-1$
et $\ell+1$, c'est $\ell$. Ainsi $k=\ell$.

Le cas $x<0$ est similaire.
\end{proof}


\begin{exemple}
Encadrons $\sqrt{10}$ et $1,1^{1/12}$ par deux entiers consécutifs.

\begin{itemize}
  \item Nous savons $3^2 = 9 < 10$ donc $3=\sqrt{3^2} < \sqrt{10}$ (la fonction racine carrée est croissante).
De même $4^2=16 > 10$ donc $4=\sqrt{4^2} > \sqrt{10}$.
Conclusion : $3 < \sqrt{10} < 4$ ce qui implique $E\big(\sqrt{10}\big)=3$.

  \item On procède sur le même principe. $1^{12} < 1,10 < 2^{12}$ donc en passant à
la racine $12$-ième (c'est-à-dire à la puissance $\frac1{12}$) on obtient :
$1 < 1,1^{1/12} < 2$ et donc $E\big(1,1^{1/12}\big)=1$.
\end{itemize}
\end{exemple}


%---------------------------------------------------------------
\subsection{Valeur absolue}

Pour un nombre réel $x$, on définit la \defi{valeur absolue}\index{valeur absolue} de $x$ par :
\mybox{$\displaystyle |x|=
\begin{cases}
x & \text{ si } x\geq 0 \\
-x & \text{ si } x<0
\end{cases}
$}

Voici le graphe de la fonction $x\mapsto |x|$ :

\myfigure{1}{
\tikzinput{fig_reels05}
}

\begin{proposition}
\sauteligne
\begin{enumerate}
  \item $|x|\geq 0$ \quad ; \quad  $|-x|=|x|$ \quad  ;  \quad $|x|>0 \iff x\neq 0$
  \item $\sqrt{x^2}=|x|$
  \item $|xy|=|x||y|$
  \item \defi{Inégalité triangulaire}\index{inegalite@inégalité!triangulaire} \myboxinline{$|x+y|\leq |x|+|y|$}
  \item \defi{Seconde inégalité triangulaire} $\big||x|-|y|\big|\leq |x-y|$
\end{enumerate}
\end{proposition}

\begin{proof}[Démonstration des inégalités triangulaires]
~
\begin{itemize}
\item $-|x|\leq x \leq |x|$ et $-|y|\leq y \leq |y|$.
En additionnant $-\left(|x|+|y|\right)\leq x+y\leq |x|+|y|$, donc $|x+y|\leq |x|+|y|$.

\item Puisque $x=(x-y)+y$, on a d'après la première inégalité :
$|x| = \big| (x-y)+y \big| \leq |x-y|+ |y|$. Donc $|x|-|y|\leq |x-y|$, et en intervertissant
les rôles de $x$ et $y$, on a aussi $|y|-|x|\leq |y-x|$.
Comme $|y-x|= |x-y|$ on a donc $\big||x|-|y|\big|\leq |x-y|$.
\end{itemize}
\end{proof}

Sur la droite numérique, $|x-y|$ représente la distance entre les
réels $x$ et $y$ ; en particulier $|x|$ représente la distance entre les réels $x$ et $0$.

\myfigure{1}{
\tikzinput{fig_reels06}
}

\begin{center}
\begin{tikzpicture}[>=latex]

\end{tikzpicture}
\end{center}

De plus on a :
\begin{itemize}
  \item $|x-a|<r \iff a-r< x <a+r$.
  \item Ou encore, comme on le verra bientôt, $|x-a|<r \iff x \in ]a-r,a+r[$.
\end{itemize}
\myfigure{1}{
\tikzinput{fig_reels07}
}

\begin{exercicecours}
Soit $a\in \Rr\backslash \{0\}$ et $x\in \Rr$ tel que $|x-a|<|a|$. Montrer que
$x\neq 0$ et ensuite que $x$ est du même signe que $a$.
\end{exercicecours}




%---------------------------------------------------------------
%\subsection{Mini-exercices}

\begin{miniexercices}
\sauteligne
\begin{enumerate}
\item On munit l'ensemble $\mathcal P(\Rr)$ des parties de $\Rr$ de la relation $\mathcal R$ définie par
$A\mathcal R B$ si $A\subset B$. Montrer qu'il s'agit d'une relation d'ordre. Est-elle totale ?

\item Soient $x,y$ deux réels. Montrer que $|x|\geq \big||x+y|-|y|\big|$.

\item Soient $x_1,\ldots,x_n$ des réels. Montrer que $|x_1+\cdots+x_n|\leq |x_1|+\cdots+|x_n|$. Dans quel cas a-t-on égalité ?

\item Soient $x,y>0$ des réels. Comparer $E(x+y)$ avec $E(x)+E(y)$.
Comparer $E(x\times y)$ et $E(x)\times E(y)$.

\item Soit $x>0$ un réel. Encadrer $\frac{E(x)}{x}$. Quelle est la limite de $\frac{E(x)}{x}$ lorsque $x \to +\infty$ ?

\item On note $\{x\}=x-E(x)$ la \emph{partie fractionnaire} de $x$, de sorte que $x=E(x) + \{x\}$.
Représenter les graphes des fonctions $x \mapsto E(x)$, $x \mapsto \{x\}$, $x \mapsto E(x)-\{x\}$.
\end{enumerate}
\end{miniexercices}


%%%%%%%%%%%%%%%%%%%%%%%%%%%%%%%%%%%%%%%%%%%%%%%%%%%%%%%%%%%%%%%%
\section{Densité de $\Qq$ dans $\Rr$}

%---------------------------------------------------------------
\subsection{Intervalle}

\begin{definition}
Un \defi{intervalle de $\Rr$}\index{intervalle} est un sous-ensemble $I$ de $\Rr$ vérifiant la propriété :
\[ \forall a,b \in I \;\;  \forall x\in \Rr\;\; \left( a\leq x \leq b \implies x \in I \right )\]
\end{definition}

\begin{remarque*}
\sauteligne
\begin{itemize}
\item Par définition $I=\varnothing$ est un intervalle.
\item $I=\Rr$ est aussi un intervalle.
\end{itemize}
\end{remarque*}



\begin{definition}
Un \defi{intervalle ouvert} est un sous-ensemble de $\Rr$ de la forme
\( ]a,b[=\big\{x \in \Rr\;  |\; a<x<b\big\}\), où $a$ et $b$ sont des éléments de $\overline{\Rr}$.
\end{definition}

Même si cela semble évident il faut justifier qu'un intervalle ouvert est un intervalle~(!).
En effet soient $a',b'$ des éléments de $]a,b[$ et $x\in \Rr$ tel que $a'\leq x \leq b'$.
Alors on a $a<a'\leq x \leq b'<b$, donc $x\in ]a,b[$.

% \medskip
%
% Les intervalles de $\Rr$ sont d'une des formes suivantes :
% $]a,b[$, $[a,b]$, $[a,b[$, $]a,b]$, $]-\infty,a[$, $]-\infty,a]$, $]a,+\infty[$, $[a,+\infty[$.

\bigskip


La notion de voisinage sera utile pour les limites.
\begin{definition}
Soit $a$ un réel, $V\subset \Rr$ un sous-ensemble.
On dit que $V$ est un \defi{voisinage}\index{voisinage} de $a$ s'il existe un intervalle
ouvert $I$ tel que $a\in I$ et $I\subset V$.
\end{definition}

\myfigure{1}{
\tikzinput{fig_reels08}
}


%---------------------------------------------------------------
\subsection{Densité}
\begin{theoreme}
\sauteligne
\begin{enumerate}
  \item $\Qq$ est \defi{dense}\index{densite@densité} dans $\Rr$ :
tout intervalle ouvert (non vide) de $\Rr$ contient une infinité de rationnels.
  \item $\Rr\backslash\Qq$ est dense dans $\Rr$ :
tout intervalle ouvert (non vide) de $\Rr$ contient une infinité d'irrationnels.
\end{enumerate}
\end{theoreme}


\begin{proof}
On commence par remarquer que tout intervalle ouvert non vide de $\Rr$
contient un intervalle du type $]a,b[$, $a,b \in \Rr$. On peut donc supposer que $I=]a,b[$ par la suite.
\begin{enumerate}
\item \emph{Tout intervalle contient un rationnel.}

On commence par montrer l'affirmation :
\begin{equation}
\label{eq:ratiodense}
\forall a,b \in \Rr \ \  \left(a<b \implies \exists r\in \Qq \quad a<r<b \right)
\end{equation}
Donnons d'abord l'idée de la preuve. Trouver un tel rationnel $r=\frac pq$,
avec $p\in \Zz$ et $q\in \Nn^*$, revient à trouver de tels entiers $p$ et $q$
vérifiant $qa<p<qb$. Cela revient à trouver un $q\in \Nn^*$ tel que l'intervalle ouvert
$]qa,qb[$ contienne un entier $p$. Il suffit pour cela que la longueur $qb-qa=q(b-a)$
de l'intervalle dépasse strictement $1$, ce qui équivaut à $q>\frac{1}{b-a}$.

Passons à la rédaction définitive. D'après la propriété d'Archimède (propriété $\Rr3$),
il existe un entier $q$ tel que $q>\frac{1}{b-a}$. Comme $b-a>0$, on a $q\in \Nn^*$.
Posons $p=E(aq)+1$. Alors $p-1\leq aq<p$. On en déduit d'une part $a<\frac pq$, et d'autre part
$\frac pq- \frac 1q\leq a$, donc $\frac pq \leq a+\frac 1q < a+b-a=b$. Donc $\frac pq\in ]a,b[$.
On a montré l'affirmation \eqref{eq:ratiodense}.

\item  \emph{Tout intervalle contient un irrationnel.}

Partant de $a$, $b$ réels tels que $a<b$, on peut
appliquer l'implication de l'affirmation \eqref{eq:ratiodense} au
 couple $(a-\sqrt{2},b-\sqrt{2})$. On en déduit qu'il existe un
rationnel $r$ dans l'intervalle $]a-\sqrt{2},b-\sqrt{2}[$ et
par translation $r+\sqrt{2}\in ]a,b[$. Or $r+\sqrt{2}$ est irrationnel,
car sinon comme les rationnels sont stables par somme, $\sqrt{2}=-r+r+\sqrt{2}$
serait rationnel, ce qui est faux d'après la proposition \ref{prop:rac2irr}.
On a donc montré que si $a<b$, l'intervalle $]a,b[$ contient aussi un irrationnel.

\item \emph{Tout intervalle contient une infinité de rationnels et d'irrationnels.}

On va déduire de l'existence d'un rationnel et d'un irrationnel
dans tout intervalle $]a,b[$ le fait qu'il existe une infinité
de chaque dans un tel intervalle ouvert. En effet pour un entier $N\geq 1$, on
 considère l'ensemble de $N$ sous-intervalles ouverts disjoints deux à deux :
\[ \Big]a,a+\frac{b-a}{N}\Big[\ , \quad
\Big]a+\frac{b-a}{N},a+\frac{2(b-a)}{N}\Big[\ ,\quad \ldots \quad
\Big]a+\frac{(N-1)(b-a)}{N},b\Big[ \ .\]
Chaque sous-intervalle contient un rationnel et un irrationnel,
donc $]a,b[$ contient (au moins) $N$ rationnels et $N$ irrationnels.
Comme ceci est vrai pour tout entier $N\geq 1$, l'intervalle ouvert
$]a,b[$ contient alors une infinité de rationnels et une infinité d'irrationnels.
\end{enumerate}

\end{proof}

%---------------------------------------------------------------
%\subsection{Mini-exercices}

\begin{miniexercices}
\sauteligne
\begin{enumerate}
  \item Montrer qu'une intersection d'intervalles est un intervalle. Qu'en est-il pour une réunion ?
Trouver une condition nécessaire et suffisante afin que la réunion de deux intervalles soit un intervalle.
%\item Soit $a\in \Rr$. Montrer qu'une intersection de voisinages de $a$ est un voisinage de $a$. Qu'en est il pour une réunion ?
  \item Montrer que l'ensemble des nombres décimaux (c'est-à-dire ceux de la forme
$\frac{a}{10^n}$, avec $n\in \Nn$ et $a\in \Zz$) est dense dans $\Rr$.
  \item Construire un rationnel compris strictement entre $123$ et $123,001$. Ensuite construire un irrationnel.
Sauriez-vous en construire une infinité ? Et entre $\pi$ et $\pi+0,001$ ?
  \item Montrer que si $z=e^{i\alpha}$ et $z'=e^{i\beta}$ sont deux nombres complexes de module $1$,
avec $\alpha<\beta$, il existe un entier $n\in \Nn^*$ et une racine $n$-ième de
l'unité $z=e^{i\gamma}$ avec $\alpha<\gamma<\beta$.
%    \item Montrer qu'il existe une suite $(u_n)_{n\in \Nn}$ d'éléments de $\Rr$ telle que l'ensemble $\{ u_n, n\in \Nn \}$ est dense dans $\Rr$.
\end{enumerate}
\end{miniexercices}


%%%%%%%%%%%%%%%%%%%%%%%%%%%%%%%%%%%%%%%%%%%%%%%%%%%%%%%%%%%%%%%%
\section{Borne supérieure}

%---------------------------------------------------------------
\subsection{Maximum, minimum}


\begin{definition}
\label{def:max}
Soit $A$ une partie non vide de $\Rr$. Un réel $\alpha$ est un
\defi{plus grand élément}\index{plus grand element@plus grand élément} de $A$ si :
 \centerline{$\alpha \in A$ \qquad et  \qquad  $\forall x \in A \;\; x\leq \alpha$.}
S'il existe, le plus grand élément est unique, on le note alors $\max A$.

Le \defi{plus petit élément}\index{plus petit element@plus petit élément} de $A$, noté $\min A$, 
s'il existe est le réel
$\alpha$ tel que $\alpha \in A$ et $\forall x \in A \;\; x \ge \alpha$.
\end{definition}



Le plus grand élément s'appelle aussi le \defi{maximum}\index{maximum} et le plus petit élément, 
le \defi{minimum}\index{minimum}.
Il faut garder à l'esprit que le plus grand élément ou le plus petit élément n'existent pas toujours.

\begin{exemple}
\sauteligne
\begin{itemize}
  \item $\max[a,b]=b$ , $\min [a,b]=a$.
  \item L'intervalle $]a,b[$ n'a pas de plus grand élément, ni de plus petit élément.
  \item L'intervalle $[0,1[$ a pour plus petit élément $0$ et n'a pas de plus grand élément.
\end{itemize}
\end{exemple}


\begin{exemple}
Soit $A=\big\{ 1-\frac{1}{n} \mid n\in \Nn^* \big\}$.


Notons $u_n=1-\frac{1}{n}$ pour $n\in \Nn^* $. Alors $A=\big\{ u_n \mid n\in \Nn^* \big\}$.
Voici une représentation graphique de $A$ sur la droite numérique :
\myfigure{1.5}{
\tikzinput{fig_reels09}
}

\begin{enumerate}
  \item $A$ n'a pas de plus grand élément : Supposons qu'il existe un plus grand élément $\alpha=\max A$.
On aurait alors $u_n \le \alpha$, pour tout $u_n$. Ainsi $1-\frac{1}{n} \le \alpha$ donc $\alpha \ge 1-\frac{1}{n}$.
\`A la limite lorsque $n \to +\infty$ cela implique $\alpha \ge 1$.
Comme $\alpha$ est le plus grand élément de $A$ alors $\alpha \in A$.
Donc il existe $n_0$ tel que $\alpha = u_{n_0}$. Mais alors $\alpha = 1-\frac{1}{n_0} <1$.
Ce qui est en contradiction avec $\alpha \ge 1$. Donc $A$ n'a pas de maximum.

  \item $\min A=0$ : Il y a deux choses à vérifier tout d'abord pour $n=1$, $u_1=0$ donc $0 \in A$.
Ensuite pour tout $n\ge 1$, $u_n \ge 0$. Ainsi $\min A = 0$.
\end{enumerate}
\end{exemple}



%---------------------------------------------------------------
\subsection{Majorants, minorants}


\begin{definition}
\label{def:majorant-minorant}
Soit $A$ une partie non vide de $\Rr$. Un réel $M$ est un
\defi{majorant}\index{majorant} de $A$ si $\forall x \in A \;\; x\leq M$.

Un réel $m$ est un \defi{minorant}\index{minorant} de $A$ si $\forall x \in A \;\; x\geq m$.
\end{definition}

\begin{exemple}
\sauteligne
\begin{itemize}
\item $3$ est un majorant de $]0,2[$ ;
\item $-7,-\pi,0$ sont des minorants de $]0,+\infty[$ mais il n'y a pas de majorant.
\end{itemize}
\end{exemple}


Si un majorant (resp. un minorant) de $A$ existe on dit que $A$ est \defi{majorée} (resp. \defi{minorée}).

Comme pour le minimum et le maximum il n'existe pas toujours de majorant ni de minorant,
en plus on n'a pas l'unicité.

\begin{exemple}
Soit $A=[0,1[$.
\myfigure{1.5}{
\tikzinput{fig_reels10}
}
\begin{enumerate}
  \item les majorants de $A$ sont exactement les éléments de $[1,+\infty[$,
  \item les minorants de $A$ sont exactement les éléments de $]-\infty,0]$.
\end{enumerate}
\end{exemple}


%---------------------------------------------------------------
\subsection{Borne supérieure, borne inférieure}


\begin{definition}
\label{def:sup-inf}
Soit $A$ une partie non vide de $\Rr$ et $\alpha$ un réel.
\begin{enumerate}
  \item $\alpha$ est la \defi{borne supérieure}\index{borne superieure@borne supérieure} de $A$ si $\alpha$ est un majorant de $A$ et
si c'est le plus petit des majorants. S'il existe on le note $\sup A$.
  \item $\alpha$ est la \defi{borne inférieure}\index{borne inferieure@borne inférieure} de $A$ si $\alpha$ est un minorant de
$A$ et si c'est le plus grand des minorants. S'il existe on le note $\inf A$.
\end{enumerate}
\end{definition}

\begin{exemple}
Soit $A=]0,1]$.
\begin{enumerate}
  \item $\sup A=1$ : en effet les majorants de $A$ sont les éléments de $[1,+\infty[$. Donc le plus petit des majorants est $1$.
  \item $\inf A=0$ : les minorants sont les éléments de $]-\infty,0]$ donc le plus grand des minorants est $0$.
\end{enumerate}
\end{exemple}

\begin{exemple}
\sauteligne
\begin{itemize}
  \item $\sup[a,b]=b$,
  \item $\inf[a,b]=a$,
  \item $\sup]a,b[=b$,
  \item $]0,+\infty[$ n'admet pas de borne supérieure,
  \item $\inf ]0,+\infty[ =0$.
\end{itemize}
\end{exemple}



\begin{theoreme}[$\Rr4$]
Toute partie de $\Rr$ non vide et majorée admet une borne supérieure.
\end{theoreme}

De la même façon : Toute partie de $\Rr$ non vide et minorée admet une borne inférieure.

\begin{remarque*}
C'est tout l'intérêt de la borne supérieure par rapport à la notion de plus grand élément,
dès qu'une partie est bornée elle admet toujours une borne supérieure et une borne inférieure.
Ce qui n'est pas le cas pour le plus grand ou plus petit élément. Gardez à l'esprit l'exemple $A=[0,1[$.
\end{remarque*}

\begin{proposition}[Caractérisation de la borne supérieure]
Soit $A$ une partie non vide et majorée de $\Rr$. La borne supérieure de $A$ est l'unique réel $\sup A$ tel que
\begin{enumerate}
\item[(i)] si $x\in A$, alors $x\leq \sup A$,
\item[(ii)] pour tout $y<\sup A$, il existe $x\in A$ tel que $y<x$.
\end{enumerate}
\end{proposition}

\begin{exemple}
Reprenons l'exemple de la partie $A=\big\{ 1-\frac{1}{n} \mid n\in \Nn^* \big\}$.

\myfigure{1.4}{
\tikzinput{fig_reels09}
}
\begin{enumerate}
  \item Nous avions vu que $\min A = 0$. Lorsque le plus petit élément d'une partie existe alors
la borne inférieure vaut ce plus petit élément : donc $\inf A= \min A=0$.

  \item \emph{Première méthode pour $\sup A$.} Montrons que $\sup A=1$ en utilisant la définition
de la borne supérieure. Soit $M$ un majorant de $A$ alors $M \ge 1-\frac 1n$, pour tout $n\ge 1$.
Donc à la limite $M \ge 1$. Réciproquement si $M\ge 1$ alors $M$ est un majorant de $A$.
Donc les majorants sont les éléments de $[1,+\infty[$. Ainsi le plus petit des majorants est $1$ et donc $\sup A=1$.

  \item \emph{Deuxième méthode pour $\sup A$.} Montrons que $\sup A=1$ en utilisant la caractérisation
de la borne supérieure.
  \begin{enumerate}
     \item[(i)] Si $x\in A$, alors $x\leq 1$ ($1$ est bien un majorant de $A$) ;
     \item[(ii)] pour tout $y< 1$, il existe $x\in A$ tel que $y<x$  : en effet prenons $n$ suffisamment grand tel que
$0<\frac 1n < 1-y$. Alors on a $y < 1-\frac 1n < 1$. Donc $x=1-\frac 1n \in A$ convient.
   \end{enumerate}
Par la caractérisation de la borne supérieure, $\sup A=1$.
\end{enumerate}
\end{exemple}


\begin{proof}~
\begin{enumerate}
\item Montrons que $\sup A$ vérifie ces deux propriétés.
La borne supérieure est en particulier un majorant, donc vérifie la première
propriété. Pour la seconde, fixons $y<\sup A$. Comme $\sup A$ est le plus
petit des majorants de $A$ alors $y$ n'est pas un majorant de $A$.
Donc il existe $x\in A$ tel que $y<x$. Autrement dit $\sup A$ vérifie
également la seconde propriété.

\item Montrons que réciproquement si un nombre $\alpha$
vérifie ces deux propriétés, il s'agit de $\sup A$. La première propriété
montre que $\alpha$ est un majorant de $A$. Supposons par l'absurde que
$\alpha$ n'est pas le plus petit des majorants. Il existe donc un autre
majorant $y$ de $A$ vérifiant $y<\alpha$. La deuxième propriété montre
l'existence d'un élément $x$ de $A$ tel que $y<x$, ce qui contredit le
fait que $y$ est un majorant de $A$. Cette contradiction montre donc que
$\alpha$ est bien le plus petit des majorants de $A$, à savoir $\sup A$.
\end{enumerate}
\end{proof}

\medskip
Nous anticipons sur la suite pour donner une autre caractérisation, très utile, de la borne supérieure.
\begin{proposition}
Soit $A$ une partie non vide et majorée de $\Rr$. La borne supérieure de $A$ est l'unique réel $\sup A$ tel que
\begin{enumerate}
\item[(i)] $\sup A$ est un majorant de $A$,
\item[(ii)] il existe une suite $(x_n)_{n\in\Nn}$ d'éléments de $A$ qui converge vers $\sup A$.
\end{enumerate}
\end{proposition}


%---------------------------------------------------------------
\subsection*{Remarques historiques}

\begin{itemize}
\item Les propriétés $\Rr1$, $\Rr2$, $\Rr3$ et le théorème $\Rr4$
sont intrinsèques à la construction de $\Rr$ (que nous admettons).

  \item Il y a un grand saut entre $\Qq$ et $\Rr$ :
on peut donner un sens précis à l'assertion
 \og il y a beaucoup plus de nombres irrationnels que de nombres rationnels \fg{},
bien que ces deux ensembles soient infinis, et m\^eme denses dans $\Rr$.

D'autre part, la construction du corps des réels $\Rr$ est beaucoup plus récente
que celle de $\Qq$ dans l'histoire des mathématiques.


\item La construction de $\Rr$ devient une nécessité après l'introduction
du calcul infinitésimal (Newton et Leibniz vers 1670). Jusqu'alors l'existence d'une borne supérieure était considérée
comme évidente et souvent confondue avec le plus grand élément.

\item Ce n'est pourtant que beaucoup plus tard, dans les années $1860$-$1870$
(donc assez récemment dans l'histoire des mathématiques) que deux constructions complètes de $\Rr$ sont données :
\begin{itemize}
\item Les coupures de Dedekind : $\mathcal C$ est une coupure si $\mathcal C \subset \Qq$
et si $\forall r\in \mathcal C$ on a $r'<r\implies r'\in \mathcal C$.
\item Le suites de Cauchy : ce sont les suites $(u_n)_{n\in \Nn}$ vérifiant la propriété
\[ \forall \epsilon >0\;\; \exists N \in \Nn \qquad  \left( m\geq N\;\;,n\geq  N\right)\implies |u_m-u_n|\leq \epsilon \;\; .\]
Les réels sont l'ensemble des suites de Cauchy (où l'on identifie deux suites de Cauchy
dont la différence tend vers $0$).
\end{itemize}

\end{itemize}


%---------------------------------------------------------------
%\subsection{Mini-exercices}

\begin{miniexercices}
\sauteligne
\begin{enumerate}
  \item Soit $A$ une partie de $\Rr$. On note $-A=\{-x| x\in A\}$. Montrer que $\min A=-\max(-A)$,
c'est-à-dire que si l'une des deux quantités a un sens, l'autre aussi, et on a égalité.
  \item Soit $A$ une partie de $\Rr$. Montrer que $A$ admet un plus petit
élément si et seulement si $A$ admet une borne inférieure qui appartient à $A$.
\item Même exercice, mais en remplaçant $\min$ par $\inf$ et $\max$ par $\sup$.
 % \item Soit $f:\Rr\to \Rr$ une fonction croissante. Comparer $f(\sup A)$ et $\sup f(A)$. Mêmes questions avec $\max$, $\min$, $\inf$, puis reprendre l'exercice avec $f$ décroissante.
  \item Soit $A=\big\{(-1)^n \frac{n}{n+1}\mid n\in \Nn\big\}$. Déterminer, s'ils existent, le plus grand élément, le plus petit élément,
les majorants, les minorants, la borne supérieure et la borne inférieure.
  \item Même question avec $A= \big\{ \frac{1}{1+x} \mid x \in [0,+\infty[ \big\}$.
%    \item Soit $f:[0,1]\to[0,1]$ une fonction croissante. Montrer que $A=\big\{x\in[0,1]| x\geq f(x) \big\}$
% admet une borne inférieure $a$, puis que ce $a$ est en fait le minimum de $A$, et finalement que $f(a)=a$.
\end{enumerate}
\end{miniexercices}

\vspace*{-1ex}

\auteurs{
Arnaud Bodin, 
Niels Borne, 
Laura Desideri
}

\finchapitre
\end{document}


