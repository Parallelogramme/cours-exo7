
%%%%%%%%%%%%%%%%%% PREAMBULE %%%%%%%%%%%%%%%%%%

\documentclass[aspectratio=169,utf8]{beamer}
%\documentclass[aspectratio=169,handout]{beamer}

\usetheme{Boadilla}
%\usecolortheme{seahorse}
\usecolortheme[RGB={245,66,24}]{structure}
\useoutertheme{infolines}

% packages
\usepackage{amsfonts,amsmath,amssymb,amsthm}
\usepackage[utf8]{inputenc}
\usepackage[T1]{fontenc}
\usepackage{lmodern}

\usepackage[francais]{babel}
\usepackage{fancybox}
\usepackage{graphicx}

\usepackage{float}
\usepackage{xfrac}

%\usepackage[usenames, x11names]{xcolor}
\usepackage{tikz}
\usepackage{pgfplots}
\usepackage{datetime}



%-----  Package unités -----
\usepackage{siunitx}
\sisetup{locale = FR,detect-all,per-mode = symbol}

%\usepackage{mathptmx}
%\usepackage{fouriernc}
%\usepackage{newcent}
%\usepackage[mathcal,mathbf]{euler}

%\usepackage{palatino}
%\usepackage{newcent}
% \usepackage[mathcal,mathbf]{euler}



% \usepackage{hyperref}
% \hypersetup{colorlinks=true, linkcolor=blue, urlcolor=blue,
% pdftitle={Exo7 - Exercices de mathématiques}, pdfauthor={Exo7}}


%section
% \usepackage{sectsty}
% \allsectionsfont{\bf}
%\sectionfont{\color{Tomato3}\upshape\selectfont}
%\subsectionfont{\color{Tomato4}\upshape\selectfont}

%----- Ensembles : entiers, reels, complexes -----
\newcommand{\Nn}{\mathbb{N}} \newcommand{\N}{\mathbb{N}}
\newcommand{\Zz}{\mathbb{Z}} \newcommand{\Z}{\mathbb{Z}}
\newcommand{\Qq}{\mathbb{Q}} \newcommand{\Q}{\mathbb{Q}}
\newcommand{\Rr}{\mathbb{R}} \newcommand{\R}{\mathbb{R}}
\newcommand{\Cc}{\mathbb{C}} 
\newcommand{\Kk}{\mathbb{K}} \newcommand{\K}{\mathbb{K}}

%----- Modifications de symboles -----
\renewcommand{\epsilon}{\varepsilon}
\renewcommand{\Re}{\mathop{\text{Re}}\nolimits}
\renewcommand{\Im}{\mathop{\text{Im}}\nolimits}
%\newcommand{\llbracket}{\left[\kern-0.15em\left[}
%\newcommand{\rrbracket}{\right]\kern-0.15em\right]}

\renewcommand{\ge}{\geqslant}
\renewcommand{\geq}{\geqslant}
\renewcommand{\le}{\leqslant}
\renewcommand{\leq}{\leqslant}
\renewcommand{\epsilon}{\varepsilon}

%----- Fonctions usuelles -----
\newcommand{\ch}{\mathop{\text{ch}}\nolimits}
\newcommand{\sh}{\mathop{\text{sh}}\nolimits}
\renewcommand{\tanh}{\mathop{\text{th}}\nolimits}
\newcommand{\cotan}{\mathop{\text{cotan}}\nolimits}
\newcommand{\Arcsin}{\mathop{\text{arcsin}}\nolimits}
\newcommand{\Arccos}{\mathop{\text{arccos}}\nolimits}
\newcommand{\Arctan}{\mathop{\text{arctan}}\nolimits}
\newcommand{\Argsh}{\mathop{\text{argsh}}\nolimits}
\newcommand{\Argch}{\mathop{\text{argch}}\nolimits}
\newcommand{\Argth}{\mathop{\text{argth}}\nolimits}
\newcommand{\pgcd}{\mathop{\text{pgcd}}\nolimits} 


%----- Commandes divers ------
\newcommand{\ii}{\mathrm{i}}
\newcommand{\dd}{\text{d}}
\newcommand{\id}{\mathop{\text{id}}\nolimits}
\newcommand{\Ker}{\mathop{\text{Ker}}\nolimits}
\newcommand{\Card}{\mathop{\text{Card}}\nolimits}
\newcommand{\Vect}{\mathop{\text{Vect}}\nolimits}
\newcommand{\Mat}{\mathop{\text{Mat}}\nolimits}
\newcommand{\rg}{\mathop{\text{rg}}\nolimits}
\newcommand{\tr}{\mathop{\text{tr}}\nolimits}


%----- Structure des exercices ------

\newtheoremstyle{styleexo}% name
{2ex}% Space above
{3ex}% Space below
{}% Body font
{}% Indent amount 1
{\bfseries} % Theorem head font
{}% Punctuation after theorem head
{\newline}% Space after theorem head 2
{}% Theorem head spec (can be left empty, meaning ‘normal’)

%\theoremstyle{styleexo}
\newtheorem{exo}{Exercice}
\newtheorem{ind}{Indications}
\newtheorem{cor}{Correction}


\newcommand{\exercice}[1]{} \newcommand{\finexercice}{}
%\newcommand{\exercice}[1]{{\tiny\texttt{#1}}\vspace{-2ex}} % pour afficher le numero absolu, l'auteur...
\newcommand{\enonce}{\begin{exo}} \newcommand{\finenonce}{\end{exo}}
\newcommand{\indication}{\begin{ind}} \newcommand{\finindication}{\end{ind}}
\newcommand{\correction}{\begin{cor}} \newcommand{\fincorrection}{\end{cor}}

\newcommand{\noindication}{\stepcounter{ind}}
\newcommand{\nocorrection}{\stepcounter{cor}}

\newcommand{\fiche}[1]{} \newcommand{\finfiche}{}
\newcommand{\titre}[1]{\centerline{\large \bf #1}}
\newcommand{\addcommand}[1]{}
\newcommand{\video}[1]{}

% Marge
\newcommand{\mymargin}[1]{\marginpar{{\small #1}}}

\def\noqed{\renewcommand{\qedsymbol}{}}


%----- Presentation ------
\setlength{\parindent}{0cm}

%\newcommand{\ExoSept}{\href{http://exo7.emath.fr}{\textbf{\textsf{Exo7}}}}

\definecolor{myred}{rgb}{0.93,0.26,0}
\definecolor{myorange}{rgb}{0.97,0.58,0}
\definecolor{myyellow}{rgb}{1,0.86,0}

\newcommand{\LogoExoSept}[1]{  % input : echelle
{\usefont{U}{cmss}{bx}{n}
\begin{tikzpicture}[scale=0.1*#1,transform shape]
  \fill[color=myorange] (0,0)--(4,0)--(4,-4)--(0,-4)--cycle;
  \fill[color=myred] (0,0)--(0,3)--(-3,3)--(-3,0)--cycle;
  \fill[color=myyellow] (4,0)--(7,4)--(3,7)--(0,3)--cycle;
  \node[scale=5] at (3.5,3.5) {Exo7};
\end{tikzpicture}}
}


\newcommand{\debutmontitre}{
  \author{} \date{} 
  \thispagestyle{empty}
  \hspace*{-10ex}
  \begin{minipage}{\textwidth}
    \titlepage  
  \vspace*{-2.5cm}
  \begin{center}
    \LogoExoSept{2.5}
  \end{center}
  \end{minipage}

  \vspace*{-0cm}
  
  % Astuce pour que le background ne soit pas discrétisé lors de la conversion pdf -> png
\begin{tikzpicture}
        \fill[opacity=0,green!60!black] (0,0)--++(0,0)--++(0,0)--++(0,0)--cycle; 
\end{tikzpicture}

% toc S'affiche trop tot :
% \tableofcontents[hideallsubsections, pausesections]
}

\newcommand{\finmontitre}{
  \end{frame}
  \setcounter{framenumber}{0}
} % ne marche pas pour une raison obscure

%----- Commandes supplementaires ------

% \usepackage[landscape]{geometry}
% \geometry{top=1cm, bottom=3cm, left=2cm, right=10cm, marginparsep=1cm
% }
% \usepackage[a4paper]{geometry}
% \geometry{top=2cm, bottom=2cm, left=2cm, right=2cm, marginparsep=1cm
% }

%\usepackage{standalone}


% New command Arnaud -- november 2011
\setbeamersize{text margin left=24ex}
% si vous modifier cette valeur il faut aussi
% modifier le decalage du titre pour compenser
% (ex : ici =+10ex, titre =-5ex

\theoremstyle{definition}
%\newtheorem{proposition}{Proposition}
%\newtheorem{exemple}{Exemple}
%\newtheorem{theoreme}{Théorème}
%\newtheorem{lemme}{Lemme}
%\newtheorem{corollaire}{Corollaire}
%\newtheorem*{remarque*}{Remarque}
%\newtheorem*{miniexercice}{Mini-exercices}
%\newtheorem{definition}{Définition}

% Commande tikz
\usetikzlibrary{calc}
\usetikzlibrary{patterns,arrows}
\usetikzlibrary{matrix}
\usetikzlibrary{fadings} 

%definition d'un terme
\newcommand{\defi}[1]{{\color{myorange}\textbf{\emph{#1}}}}
\newcommand{\evidence}[1]{{\color{blue}\textbf{\emph{#1}}}}
\newcommand{\assertion}[1]{\emph{\og#1\fg}}  % pour chapitre logique
%\renewcommand{\contentsname}{Sommaire}
\renewcommand{\contentsname}{}
\setcounter{tocdepth}{2}



%------ Figures ------

\def\myscale{1} % par défaut 
\newcommand{\myfigure}[2]{  % entrée : echelle, fichier figure
\def\myscale{#1}
\begin{center}
\footnotesize
{#2}
\end{center}}


%------ Encadrement ------

\usepackage{fancybox}


\newcommand{\mybox}[1]{
\setlength{\fboxsep}{7pt}
\begin{center}
\shadowbox{#1}
\end{center}}

\newcommand{\myboxinline}[1]{
\setlength{\fboxsep}{5pt}
\raisebox{-10pt}{
\shadowbox{#1}
}
}

%--------------- Commande beamer---------------
\newcommand{\beameronly}[1]{#1} % permet de mettre des pause dans beamer pas dans poly


\setbeamertemplate{navigation symbols}{}
\setbeamertemplate{footline}  % tiré du fichier beamerouterinfolines.sty
{
  \leavevmode%
  \hbox{%
  \begin{beamercolorbox}[wd=.333333\paperwidth,ht=2.25ex,dp=1ex,center]{author in head/foot}%
    % \usebeamerfont{author in head/foot}\insertshortauthor%~~(\insertshortinstitute)
    \usebeamerfont{section in head/foot}{\bf\insertshorttitle}
  \end{beamercolorbox}%
  \begin{beamercolorbox}[wd=.333333\paperwidth,ht=2.25ex,dp=1ex,center]{title in head/foot}%
    \usebeamerfont{section in head/foot}{\bf\insertsectionhead}
  \end{beamercolorbox}%
  \begin{beamercolorbox}[wd=.333333\paperwidth,ht=2.25ex,dp=1ex,right]{date in head/foot}%
    % \usebeamerfont{date in head/foot}\insertshortdate{}\hspace*{2em}
    \insertframenumber{} / \inserttotalframenumber\hspace*{2ex} 
  \end{beamercolorbox}}%
  \vskip0pt%
}


\definecolor{mygrey}{rgb}{0.5,0.5,0.5}
\setlength{\parindent}{0cm}
%\DeclareTextFontCommand{\helvetica}{\fontfamily{phv}\selectfont}

% background beamer
\definecolor{couleurhaut}{rgb}{0.85,0.9,1}  % creme
\definecolor{couleurmilieu}{rgb}{1,1,1}  % vert pale
\definecolor{couleurbas}{rgb}{0.85,0.9,1}  % blanc
\setbeamertemplate{background canvas}[vertical shading]%
[top=couleurhaut,middle=couleurmilieu,midpoint=0.4,bottom=couleurbas] 
%[top=fondtitre!05,bottom=fondtitre!60]



\makeatletter
\setbeamertemplate{theorem begin}
{%
  \begin{\inserttheoremblockenv}
  {%
    \inserttheoremheadfont
    \inserttheoremname
    \inserttheoremnumber
    \ifx\inserttheoremaddition\@empty\else\ (\inserttheoremaddition)\fi%
    \inserttheorempunctuation
  }%
}
\setbeamertemplate{theorem end}{\end{\inserttheoremblockenv}}

\newenvironment{theoreme}[1][]{%
   \setbeamercolor{block title}{fg=structure,bg=structure!40}
   \setbeamercolor{block body}{fg=black,bg=structure!10}
   \begin{block}{{\bf Th\'eor\`eme }#1}
}{%
   \end{block}%
}


\newenvironment{proposition}[1][]{%
   \setbeamercolor{block title}{fg=structure,bg=structure!40}
   \setbeamercolor{block body}{fg=black,bg=structure!10}
   \begin{block}{{\bf Proposition }#1}
}{%
   \end{block}%
}

\newenvironment{corollaire}[1][]{%
   \setbeamercolor{block title}{fg=structure,bg=structure!40}
   \setbeamercolor{block body}{fg=black,bg=structure!10}
   \begin{block}{{\bf Corollaire }#1}
}{%
   \end{block}%
}

\newenvironment{mydefinition}[1][]{%
   \setbeamercolor{block title}{fg=structure,bg=structure!40}
   \setbeamercolor{block body}{fg=black,bg=structure!10}
   \begin{block}{{\bf Définition} #1}
}{%
   \end{block}%
}

\newenvironment{lemme}[0]{%
   \setbeamercolor{block title}{fg=structure,bg=structure!40}
   \setbeamercolor{block body}{fg=black,bg=structure!10}
   \begin{block}{\bf Lemme}
}{%
   \end{block}%
}

\newenvironment{remarque}[1][]{%
   \setbeamercolor{block title}{fg=black,bg=structure!20}
   \setbeamercolor{block body}{fg=black,bg=structure!5}
   \begin{block}{Remarque #1}
}{%
   \end{block}%
}


\newenvironment{exemple}[1][]{%
   \setbeamercolor{block title}{fg=black,bg=structure!20}
   \setbeamercolor{block body}{fg=black,bg=structure!5}
   \begin{block}{{\bf Exemple }#1}
}{%
   \end{block}%
}


\newenvironment{miniexercice}[0]{%
   \setbeamercolor{block title}{fg=structure,bg=structure!20}
   \setbeamercolor{block body}{fg=black,bg=structure!5}
   \begin{block}{Mini-exercices}
}{%
   \end{block}%
}


\newenvironment{tp}[0]{%
   \setbeamercolor{block title}{fg=structure,bg=structure!40}
   \setbeamercolor{block body}{fg=black,bg=structure!10}
   \begin{block}{\bf Travaux pratiques}
}{%
   \end{block}%
}
\newenvironment{exercicecours}[1][]{%
   \setbeamercolor{block title}{fg=structure,bg=structure!40}
   \setbeamercolor{block body}{fg=black,bg=structure!10}
   \begin{block}{{\bf Exercice }#1}
}{%
   \end{block}%
}
\newenvironment{algo}[1][]{%
   \setbeamercolor{block title}{fg=structure,bg=structure!40}
   \setbeamercolor{block body}{fg=black,bg=structure!10}
   \begin{block}{{\bf Algorithme}\hfill{\color{gray}\texttt{#1}}}
}{%
   \end{block}%
}


\setbeamertemplate{proof begin}{
   \setbeamercolor{block title}{fg=black,bg=structure!20}
   \setbeamercolor{block body}{fg=black,bg=structure!5}
   \begin{block}{{\footnotesize Démonstration}}
   \footnotesize
   \smallskip}
\setbeamertemplate{proof end}{%
   \end{block}}
\setbeamertemplate{qed symbol}{\openbox}


\makeatother
\usecolortheme[RGB={205,134,0}]{structure}

%%%%%%%%%%%%%%%%%%%%%%%%%%%%%%%%%%%%%%%%%%%%%%%%%%%%%%%%%%%%%
%%%%%%%%%%%%%%%%%%%%%%%%%%%%%%%%%%%%%%%%%%%%%%%%%%%%%%%%%%%%%



\begin{document}



\title{{\bf Les nombres réels}}
\subtitle{Propriétés de $\Rr$}

\begin{frame}
  
  \debutmontitre

  \pause

{\footnotesize
\hfill
\setbeamercovered{transparent=50}
\begin{minipage}{0.6\textwidth}
  \begin{itemize}
   \item<3-> Addition et multiplication
   \item<4-> Ordre sur $\Rr$
   \item<5-> Propriété d'Archimède
   \item<6-> Valeur absolue
  \end{itemize}
\end{minipage}
}

\end{frame}

\setcounter{framenumber}{0}

\section*{Addition et multiplication}

\begin{frame}


\begin{proposition}[$\Rr1$]
$(\Rr,+,\times)$ est un \evidence{corps commutatif}
\end{proposition}

\pause
\bigskip
\bigskip

\begin{center}
\begin{tabular}{ll}
$a+b=b+a$ & $a\times b=b\times a$ \\
$0+a=a$ & $1\times a =a\text{ si }a\neq 0$ \\
$a+b=0 \iff a=-b$ & $ab=1 \iff a=\frac{1}{b}$\\
$(a+b)+c=a+(b+c)\qquad$  & $(a\times b)\times c=a\times (b\times c)$\\
&\\
\end{tabular}
\end{center}
\pause
\begin{center}
$a\times(b+c)=a\times b+a\times c$ \\
\ \\
$a\times b=0 \iff (a=0 \text{ ou } b=0)$
\end{center}
	
\end{frame}

\section*{Ordre sur $\Rr$}

\begin{frame}
Soit $E$ un ensemble


\begin{mydefinition}

\begin{enumerate}
\item 
\begin{itemize}
 \item Une \defi{relation} $\mathcal R$ sur $E$ est un sous-ensemble 
de l'ensemble produit $E\times E$
\pause
 \item Pour $(x,y)\in E\times E$, 
on dit que $x$ est en relation avec $y$ et on note $x\mathcal R y$ pour dire que $(x,y)\in\mathcal R$
\end{itemize}
\pause
\item Une relation $\mathcal R$ est une \defi{relation d'ordre} si
\begin{itemize}
  \item $\mathcal R$ est \defi{réflexive} : pour tout $x\in E$, $x\mathcal R x$
  \item $\mathcal R$ est \defi{antisymétrique}\!:\! pour tout $x,y \in E$, $ x\mathcal R y \text{ et } y\mathcal R x \Rightarrow  x=y$
  \item $\mathcal R$ est \defi{transitive} : pour tout $x,y,z\in E, x\mathcal R y \text{ et } y\mathcal R z \implies x\mathcal R z$
\end{itemize}
\end{enumerate}
\end{mydefinition}\pause


\begin{mydefinition}
La relation d'ordre $\mathcal R$ est \defi{totale} si pour tout $x,y\in E$ on a $ x\mathcal R y \text{ ou } y\mathcal R x$
\end{mydefinition}
\end{frame}

\begin{frame}

\begin{proposition}[$\Rr2$]
La relation $\leq$ sur $\Rr$ est une relation d'ordre totale
\end{proposition}

\pause
\bigskip

\begin{remarque}
\begin{itemize}
\item 
\begin{itemize}
  \item réflexive : pour tout $x \in \Rr$, $x \le x$
  \item antisymétrique : pour tout $x,y \in \Rr$, si $x \le y$ et $y\le x$ alors $x=y$
  \item transitive : pour tout $x,y,z \in \Rr$ si $x\le y$ et $y\le z$ alors $x\le z$
\end{itemize}
\pause

\item	Les opérations de $\Rr$ sont compatibles avec la relation d'ordre $\leq$ 
\vspace*{-2ex}
\begin{align*}
 \left (a\leq b \text{ et } c\leq d \right) &\implies a+c\leq b+d \\
         \left (a\leq b \text{ et } c \geq 0\right)& \implies a\times c\leq b\times c \\
 \left (a\leq b \text{ et } c \leq 0\right)& \implies a\times c\geq b\times c
\end{align*}
\end{itemize}
\end{remarque}

\end{frame}

\section*{Propriété d'Archimède}

\begin{frame}
\begin{proposition}[$\Rr3$ \  Propriété d'Archimède]
\label{propr:archi}
\quad $\Rr$ est \defi{archimédien}
\qquad \myboxinline{$\forall x \in \Rr\;\;\; \exists n \in \Nn \;\;\; n>x$}
\end{proposition}

\bigskip
\pause


Soit $x\in \Rr$, définissons la \defi{partie entière} $E(x)$
\pause
\begin{proposition}
\label{prop:part_ent}
Il \evidence{existe} un \evidence{unique} entier $E(x) \in \Zz$ tel que 
\mybox{$E(x)\leq x <E(x)+1$}
\end{proposition}

\pause


\begin{exemple} 
\begin{itemize}
\item $E(2,853)=2$, \pause $E(\pi)=3$, \pause $E(-3,5)=-4$
\pause
\item $E(x)=3 \iff 3\leq x <4$
\end{itemize}
\end{exemple}


\end{frame}

\begin{frame}
	\myfigure{1}{
\tikzinput{fig_reels04} 
}
\end{frame}

\begin{frame}

\begin{exemple}[Partie entière de $\sqrt{10}$]
\pause
\begin{itemize}
 \item $3^2 = 9 < 10$ \pause donc $3=\sqrt{3^2} < \sqrt{10}$
\pause
 \item $4^2=16 > 10$ \pause donc $4=\sqrt{4^2} > \sqrt{10}$
\pause
 \item Conclusion : $3 < \sqrt{10} < 4$ donc $E\big(\sqrt{10}\big)=3$
\end{itemize}
\end{exemple}
\pause

\begin{proof}
\evidence{Existence.} Soit $x\geq 0$
\pause
\begin{itemize}
 \item par la propriété d'Archimède  il existe $n\in \Nn$ tel que $n>x$
 \pause
 \item $K=\big\{k\in \Nn \mid k\leq x \big\}$ est donc fini. \pause Notons $k_{max}=\max K$
 \pause
 \item $k_{max}\leq x$ car $k_{max}\in K$ \pause et $k_{max}+1 > x$ car $k_{max}+1\notin K$
 \pause
 \item $k_{max}\leq x< k_{max}+1$, on prend $E(x)=k_{max}$
\end{itemize}

\pause

\evidence{Unicité.}
\begin{itemize}
 \item Si $k$ et $\ell$ sont deux entiers tels que $k\leq x< k+1$ et $\ell\leq x< \ell+1$
\pause  
 \item donc $k \leq x < \ell+1$ ; \pause on a aussi $\ell<k+1$
\pause 
 \item $\ell-1<k<\ell+1$. \pause Ainsi $k=\ell$
\end{itemize}
\vspace*{-3ex}
\end{proof}
	
\end{frame}

\section*{Valeur absolue}

\begin{frame}
\begin{mydefinition}
La \defi{valeur absolue} de $x$ est
\mybox{$\displaystyle |x|=
\begin{cases}
x & \text{ si } x\geq 0 \\
-x & \text{ si } x<0
\end{cases}
$}
\end{mydefinition}

\pause


\myfigure{1.5}{
\tikzinput{fig_reels05} 
}
	
\end{frame}

\begin{frame}
\begin{proposition}
\begin{enumerate}
  \item $|x|\geq 0$ \quad \quad \quad  $|-x|=|x|$ \quad  \quad  \quad $|x|>0 \iff x\neq 0$
\pause
  \item $\sqrt{x^2}=|x|$
\pause
  \item $|xy|=|x||y|$
\pause
\medskip
  \item \defi{Inégalité triangulaire} \myboxinline{$|x+y|\leq |x|+|y|$} 
\pause
\medskip
  \item \defi{Seconde inégalité triangulaire} $\big||x|-|y|\big|\leq |x-y|$
\end{enumerate}
\end{proposition}

\pause
\bigskip

\myfigure{1}{
\tikzinput{fig_reels06} 
}

\end{frame}

\begin{frame}
\mybox{$|x+y|\leq |x|+|y|$} 

\begin{proof}
\pause
\begin{itemize}
 \item $-|x|\leq x \leq |x|$ \quad et \quad $-|y|\leq y \leq |y|$
 \pause
 \item $-\left(|x|+|y|\right)\leq x+y\leq |x|+|y|$
 \pause
 \item $|x+y|\leq |x|+|y|$
\end{itemize}
\vspace*{-2ex}
\end{proof}

 \pause
 
\begin{remarque}
\begin{itemize}
  \item $|x-a|<r \iff a-r< x <a+r$
  \pause
  \item  $|x-a|<r \iff x \in ]a-r,a+r[$
\end{itemize}
\pause
\myfigure{1}{
\tikzinput{fig_reels07} 
}
\end{remarque}
\end{frame}

\section*{Mini-exercices}
\begin{frame}
\begin{miniexercice}
\begin{enumerate}
\item On munit l'ensemble $\mathcal P(\Rr)$ des parties de $\Rr$ de la relation $\mathcal R$ définie par 
$A\mathcal R B$ si $A\subset B$. Montrer qu'il s'agit d'une relation d'ordre. Est-elle totale ?

\item Soient $x,y$ deux réels. Montrer que $|x|\geq \big||x+y|-|y|\big|$.

\item Soient $x_1,\ldots,x_n$ des réels. Montrer que $|x_1+\cdots+x_n|\leq |x_1|+\cdots+|x_n|$. Dans quel cas a-t-on égalité ?     
        
\item Soient $x,y>0$ des réels. Comparer $E(x+y)$ avec $E(x)+E(y)$. 
Comparer $E(x\times y)$ et $E(x)\times E(y)$.

\item Soit $x>0$ un réel. Encadrer $\frac{E(x)}{x}$. Quelle est la limite de $\frac{E(x)}{x}$ lorsque $x \to +\infty$ ?

\item On note $\{x\}=x-E(x)$ la \emph{partie fractionnaire} de $x$, de sorte que $x=E(x) + \{x\}$.
Représenter les graphes des fonctions $x \mapsto E(x)$, $x \mapsto \{x\}$, $x \mapsto E(x)-\{x\}$.
\end{enumerate}
\end{miniexercice}
\end{frame}




\end{document}
