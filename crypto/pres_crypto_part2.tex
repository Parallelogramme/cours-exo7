
%%%%%%%%%%%%%%%%%% PREAMBULE %%%%%%%%%%%%%%%%%%

\documentclass[aspectratio=169,utf8]{beamer}
%\documentclass[aspectratio=169,handout]{beamer}

\usetheme{Boadilla}
%\usecolortheme{seahorse}
\usecolortheme[RGB={245,66,24}]{structure}
\useoutertheme{infolines}

% packages
\usepackage{amsfonts,amsmath,amssymb,amsthm}
\usepackage[utf8]{inputenc}
\usepackage[T1]{fontenc}
\usepackage{lmodern}

\usepackage[francais]{babel}
\usepackage{fancybox}
\usepackage{graphicx}

\usepackage{float}
\usepackage{xfrac}

%\usepackage[usenames, x11names]{xcolor}
\usepackage{tikz}
\usepackage{pgfplots}
\usepackage{datetime}



%-----  Package unités -----
\usepackage{siunitx}
\sisetup{locale = FR,detect-all,per-mode = symbol}

%\usepackage{mathptmx}
%\usepackage{fouriernc}
%\usepackage{newcent}
%\usepackage[mathcal,mathbf]{euler}

%\usepackage{palatino}
%\usepackage{newcent}
% \usepackage[mathcal,mathbf]{euler}



% \usepackage{hyperref}
% \hypersetup{colorlinks=true, linkcolor=blue, urlcolor=blue,
% pdftitle={Exo7 - Exercices de mathématiques}, pdfauthor={Exo7}}


%section
% \usepackage{sectsty}
% \allsectionsfont{\bf}
%\sectionfont{\color{Tomato3}\upshape\selectfont}
%\subsectionfont{\color{Tomato4}\upshape\selectfont}

%----- Ensembles : entiers, reels, complexes -----
\newcommand{\Nn}{\mathbb{N}} \newcommand{\N}{\mathbb{N}}
\newcommand{\Zz}{\mathbb{Z}} \newcommand{\Z}{\mathbb{Z}}
\newcommand{\Qq}{\mathbb{Q}} \newcommand{\Q}{\mathbb{Q}}
\newcommand{\Rr}{\mathbb{R}} \newcommand{\R}{\mathbb{R}}
\newcommand{\Cc}{\mathbb{C}} 
\newcommand{\Kk}{\mathbb{K}} \newcommand{\K}{\mathbb{K}}

%----- Modifications de symboles -----
\renewcommand{\epsilon}{\varepsilon}
\renewcommand{\Re}{\mathop{\text{Re}}\nolimits}
\renewcommand{\Im}{\mathop{\text{Im}}\nolimits}
%\newcommand{\llbracket}{\left[\kern-0.15em\left[}
%\newcommand{\rrbracket}{\right]\kern-0.15em\right]}

\renewcommand{\ge}{\geqslant}
\renewcommand{\geq}{\geqslant}
\renewcommand{\le}{\leqslant}
\renewcommand{\leq}{\leqslant}
\renewcommand{\epsilon}{\varepsilon}

%----- Fonctions usuelles -----
\newcommand{\ch}{\mathop{\text{ch}}\nolimits}
\newcommand{\sh}{\mathop{\text{sh}}\nolimits}
\renewcommand{\tanh}{\mathop{\text{th}}\nolimits}
\newcommand{\cotan}{\mathop{\text{cotan}}\nolimits}
\newcommand{\Arcsin}{\mathop{\text{arcsin}}\nolimits}
\newcommand{\Arccos}{\mathop{\text{arccos}}\nolimits}
\newcommand{\Arctan}{\mathop{\text{arctan}}\nolimits}
\newcommand{\Argsh}{\mathop{\text{argsh}}\nolimits}
\newcommand{\Argch}{\mathop{\text{argch}}\nolimits}
\newcommand{\Argth}{\mathop{\text{argth}}\nolimits}
\newcommand{\pgcd}{\mathop{\text{pgcd}}\nolimits} 


%----- Commandes divers ------
\newcommand{\ii}{\mathrm{i}}
\newcommand{\dd}{\text{d}}
\newcommand{\id}{\mathop{\text{id}}\nolimits}
\newcommand{\Ker}{\mathop{\text{Ker}}\nolimits}
\newcommand{\Card}{\mathop{\text{Card}}\nolimits}
\newcommand{\Vect}{\mathop{\text{Vect}}\nolimits}
\newcommand{\Mat}{\mathop{\text{Mat}}\nolimits}
\newcommand{\rg}{\mathop{\text{rg}}\nolimits}
\newcommand{\tr}{\mathop{\text{tr}}\nolimits}


%----- Structure des exercices ------

\newtheoremstyle{styleexo}% name
{2ex}% Space above
{3ex}% Space below
{}% Body font
{}% Indent amount 1
{\bfseries} % Theorem head font
{}% Punctuation after theorem head
{\newline}% Space after theorem head 2
{}% Theorem head spec (can be left empty, meaning ‘normal’)

%\theoremstyle{styleexo}
\newtheorem{exo}{Exercice}
\newtheorem{ind}{Indications}
\newtheorem{cor}{Correction}


\newcommand{\exercice}[1]{} \newcommand{\finexercice}{}
%\newcommand{\exercice}[1]{{\tiny\texttt{#1}}\vspace{-2ex}} % pour afficher le numero absolu, l'auteur...
\newcommand{\enonce}{\begin{exo}} \newcommand{\finenonce}{\end{exo}}
\newcommand{\indication}{\begin{ind}} \newcommand{\finindication}{\end{ind}}
\newcommand{\correction}{\begin{cor}} \newcommand{\fincorrection}{\end{cor}}

\newcommand{\noindication}{\stepcounter{ind}}
\newcommand{\nocorrection}{\stepcounter{cor}}

\newcommand{\fiche}[1]{} \newcommand{\finfiche}{}
\newcommand{\titre}[1]{\centerline{\large \bf #1}}
\newcommand{\addcommand}[1]{}
\newcommand{\video}[1]{}

% Marge
\newcommand{\mymargin}[1]{\marginpar{{\small #1}}}

\def\noqed{\renewcommand{\qedsymbol}{}}


%----- Presentation ------
\setlength{\parindent}{0cm}

%\newcommand{\ExoSept}{\href{http://exo7.emath.fr}{\textbf{\textsf{Exo7}}}}

\definecolor{myred}{rgb}{0.93,0.26,0}
\definecolor{myorange}{rgb}{0.97,0.58,0}
\definecolor{myyellow}{rgb}{1,0.86,0}

\newcommand{\LogoExoSept}[1]{  % input : echelle
{\usefont{U}{cmss}{bx}{n}
\begin{tikzpicture}[scale=0.1*#1,transform shape]
  \fill[color=myorange] (0,0)--(4,0)--(4,-4)--(0,-4)--cycle;
  \fill[color=myred] (0,0)--(0,3)--(-3,3)--(-3,0)--cycle;
  \fill[color=myyellow] (4,0)--(7,4)--(3,7)--(0,3)--cycle;
  \node[scale=5] at (3.5,3.5) {Exo7};
\end{tikzpicture}}
}


\newcommand{\debutmontitre}{
  \author{} \date{} 
  \thispagestyle{empty}
  \hspace*{-10ex}
  \begin{minipage}{\textwidth}
    \titlepage  
  \vspace*{-2.5cm}
  \begin{center}
    \LogoExoSept{2.5}
  \end{center}
  \end{minipage}

  \vspace*{-0cm}
  
  % Astuce pour que le background ne soit pas discrétisé lors de la conversion pdf -> png
\begin{tikzpicture}
        \fill[opacity=0,green!60!black] (0,0)--++(0,0)--++(0,0)--++(0,0)--cycle; 
\end{tikzpicture}

% toc S'affiche trop tot :
% \tableofcontents[hideallsubsections, pausesections]
}

\newcommand{\finmontitre}{
  \end{frame}
  \setcounter{framenumber}{0}
} % ne marche pas pour une raison obscure

%----- Commandes supplementaires ------

% \usepackage[landscape]{geometry}
% \geometry{top=1cm, bottom=3cm, left=2cm, right=10cm, marginparsep=1cm
% }
% \usepackage[a4paper]{geometry}
% \geometry{top=2cm, bottom=2cm, left=2cm, right=2cm, marginparsep=1cm
% }

%\usepackage{standalone}


% New command Arnaud -- november 2011
\setbeamersize{text margin left=24ex}
% si vous modifier cette valeur il faut aussi
% modifier le decalage du titre pour compenser
% (ex : ici =+10ex, titre =-5ex

\theoremstyle{definition}
%\newtheorem{proposition}{Proposition}
%\newtheorem{exemple}{Exemple}
%\newtheorem{theoreme}{Théorème}
%\newtheorem{lemme}{Lemme}
%\newtheorem{corollaire}{Corollaire}
%\newtheorem*{remarque*}{Remarque}
%\newtheorem*{miniexercice}{Mini-exercices}
%\newtheorem{definition}{Définition}

% Commande tikz
\usetikzlibrary{calc}
\usetikzlibrary{patterns,arrows}
\usetikzlibrary{matrix}
\usetikzlibrary{fadings} 

%definition d'un terme
\newcommand{\defi}[1]{{\color{myorange}\textbf{\emph{#1}}}}
\newcommand{\evidence}[1]{{\color{blue}\textbf{\emph{#1}}}}
\newcommand{\assertion}[1]{\emph{\og#1\fg}}  % pour chapitre logique
%\renewcommand{\contentsname}{Sommaire}
\renewcommand{\contentsname}{}
\setcounter{tocdepth}{2}



%------ Figures ------

\def\myscale{1} % par défaut 
\newcommand{\myfigure}[2]{  % entrée : echelle, fichier figure
\def\myscale{#1}
\begin{center}
\footnotesize
{#2}
\end{center}}


%------ Encadrement ------

\usepackage{fancybox}


\newcommand{\mybox}[1]{
\setlength{\fboxsep}{7pt}
\begin{center}
\shadowbox{#1}
\end{center}}

\newcommand{\myboxinline}[1]{
\setlength{\fboxsep}{5pt}
\raisebox{-10pt}{
\shadowbox{#1}
}
}

%--------------- Commande beamer---------------
\newcommand{\beameronly}[1]{#1} % permet de mettre des pause dans beamer pas dans poly


\setbeamertemplate{navigation symbols}{}
\setbeamertemplate{footline}  % tiré du fichier beamerouterinfolines.sty
{
  \leavevmode%
  \hbox{%
  \begin{beamercolorbox}[wd=.333333\paperwidth,ht=2.25ex,dp=1ex,center]{author in head/foot}%
    % \usebeamerfont{author in head/foot}\insertshortauthor%~~(\insertshortinstitute)
    \usebeamerfont{section in head/foot}{\bf\insertshorttitle}
  \end{beamercolorbox}%
  \begin{beamercolorbox}[wd=.333333\paperwidth,ht=2.25ex,dp=1ex,center]{title in head/foot}%
    \usebeamerfont{section in head/foot}{\bf\insertsectionhead}
  \end{beamercolorbox}%
  \begin{beamercolorbox}[wd=.333333\paperwidth,ht=2.25ex,dp=1ex,right]{date in head/foot}%
    % \usebeamerfont{date in head/foot}\insertshortdate{}\hspace*{2em}
    \insertframenumber{} / \inserttotalframenumber\hspace*{2ex} 
  \end{beamercolorbox}}%
  \vskip0pt%
}


\definecolor{mygrey}{rgb}{0.5,0.5,0.5}
\setlength{\parindent}{0cm}
%\DeclareTextFontCommand{\helvetica}{\fontfamily{phv}\selectfont}

% background beamer
\definecolor{couleurhaut}{rgb}{0.85,0.9,1}  % creme
\definecolor{couleurmilieu}{rgb}{1,1,1}  % vert pale
\definecolor{couleurbas}{rgb}{0.85,0.9,1}  % blanc
\setbeamertemplate{background canvas}[vertical shading]%
[top=couleurhaut,middle=couleurmilieu,midpoint=0.4,bottom=couleurbas] 
%[top=fondtitre!05,bottom=fondtitre!60]



\makeatletter
\setbeamertemplate{theorem begin}
{%
  \begin{\inserttheoremblockenv}
  {%
    \inserttheoremheadfont
    \inserttheoremname
    \inserttheoremnumber
    \ifx\inserttheoremaddition\@empty\else\ (\inserttheoremaddition)\fi%
    \inserttheorempunctuation
  }%
}
\setbeamertemplate{theorem end}{\end{\inserttheoremblockenv}}

\newenvironment{theoreme}[1][]{%
   \setbeamercolor{block title}{fg=structure,bg=structure!40}
   \setbeamercolor{block body}{fg=black,bg=structure!10}
   \begin{block}{{\bf Th\'eor\`eme }#1}
}{%
   \end{block}%
}


\newenvironment{proposition}[1][]{%
   \setbeamercolor{block title}{fg=structure,bg=structure!40}
   \setbeamercolor{block body}{fg=black,bg=structure!10}
   \begin{block}{{\bf Proposition }#1}
}{%
   \end{block}%
}

\newenvironment{corollaire}[1][]{%
   \setbeamercolor{block title}{fg=structure,bg=structure!40}
   \setbeamercolor{block body}{fg=black,bg=structure!10}
   \begin{block}{{\bf Corollaire }#1}
}{%
   \end{block}%
}

\newenvironment{mydefinition}[1][]{%
   \setbeamercolor{block title}{fg=structure,bg=structure!40}
   \setbeamercolor{block body}{fg=black,bg=structure!10}
   \begin{block}{{\bf Définition} #1}
}{%
   \end{block}%
}

\newenvironment{lemme}[0]{%
   \setbeamercolor{block title}{fg=structure,bg=structure!40}
   \setbeamercolor{block body}{fg=black,bg=structure!10}
   \begin{block}{\bf Lemme}
}{%
   \end{block}%
}

\newenvironment{remarque}[1][]{%
   \setbeamercolor{block title}{fg=black,bg=structure!20}
   \setbeamercolor{block body}{fg=black,bg=structure!5}
   \begin{block}{Remarque #1}
}{%
   \end{block}%
}


\newenvironment{exemple}[1][]{%
   \setbeamercolor{block title}{fg=black,bg=structure!20}
   \setbeamercolor{block body}{fg=black,bg=structure!5}
   \begin{block}{{\bf Exemple }#1}
}{%
   \end{block}%
}


\newenvironment{miniexercice}[0]{%
   \setbeamercolor{block title}{fg=structure,bg=structure!20}
   \setbeamercolor{block body}{fg=black,bg=structure!5}
   \begin{block}{Mini-exercices}
}{%
   \end{block}%
}


\newenvironment{tp}[0]{%
   \setbeamercolor{block title}{fg=structure,bg=structure!40}
   \setbeamercolor{block body}{fg=black,bg=structure!10}
   \begin{block}{\bf Travaux pratiques}
}{%
   \end{block}%
}
\newenvironment{exercicecours}[1][]{%
   \setbeamercolor{block title}{fg=structure,bg=structure!40}
   \setbeamercolor{block body}{fg=black,bg=structure!10}
   \begin{block}{{\bf Exercice }#1}
}{%
   \end{block}%
}
\newenvironment{algo}[1][]{%
   \setbeamercolor{block title}{fg=structure,bg=structure!40}
   \setbeamercolor{block body}{fg=black,bg=structure!10}
   \begin{block}{{\bf Algorithme}\hfill{\color{gray}\texttt{#1}}}
}{%
   \end{block}%
}


\setbeamertemplate{proof begin}{
   \setbeamercolor{block title}{fg=black,bg=structure!20}
   \setbeamercolor{block body}{fg=black,bg=structure!5}
   \begin{block}{{\footnotesize Démonstration}}
   \footnotesize
   \smallskip}
\setbeamertemplate{proof end}{%
   \end{block}}
\setbeamertemplate{qed symbol}{\openbox}


\makeatother
\usecolortheme[RGB={127,0,0}]{structure}

% Commande spécifique à ce chapitre

\newcommand{\Python}{\texttt{Python}}
\renewcommand{\evidence}[1]{{\color{blue}\textbf{#1}}}

\usepackage{textcomp}

\usepackage{listings}
\lstset{
  upquote=true,
  columns=flexible,
  keepspaces=true,
  basicstyle=\ttfamily,
  commentstyle=\color{gray},
  language=Python,
  showstringspaces=false,
  aboveskip=0em,  
  belowskip=0em,
  escapeinside=||
}

\lstset{
  literate={é}{{\'e}}1
           {è}{{\`e}}1
           {à}{{\`a}}1
}


\newcommand{\codeinline}[1]{\lstinline!#1!}

\definecolor{coul_prive}{rgb}{0.93,0.26,0}
\definecolor{coul_public}{rgb}{0.06,0.63,0}

\newcommand{\prive}[1]{{\bf\color{coul_prive} #1}}
\newcommand{\public}[1]{{\bf\color{coul_public} #1}}

\usepackage{array}
\usepackage{colortbl}

%%%%%%%%%%%%%%%%%%%%%%%%%%%%%%%%%%%%%%%%%%%%%%%%%%%%%%%%%%%%%
%%%%%%%%%%%%%%%%%%%%%%%%%%%%%%%%%%%%%%%%%%%%%%%%%%%%%%%%%%%%%


\begin{document}


\title{{\bf Cryptographie}}
\subtitle{Le chiffre de Vigenère}

\begin{frame}
  
  \debutmontitre

  \pause

{\footnotesize
\hfill
\setbeamercovered{transparent=50}
\begin{minipage}{0.6\textwidth}
  \begin{itemize}
    \item<3-> Chiffrement par substitution
    \item<4-> Chiffrement de Vigenère 
%    \item<5-> Algorithmes
  \end{itemize}
\end{minipage}
}

\end{frame}

\setcounter{framenumber}{0}


%%%%%%%%%%%%%%%%%%%%%%%%%%%%%%%%%%%%%%%%%%%%%%%%%%%%%%%%%%%%%%%%
\section{Chiffrement par substitution}

\begin{frame}

\hfill\hfill\evidence{Chiffrement par substitution}

\begin{itemize}
  \item On associe maintenant à chaque lettre une autre lettre %(sans ordre fixe ou règle générale)

  \pause

  \medskip
  
\setlength{\extrarowheight}{3pt}  
% \renewcommand{\arraystretch}{1.1}   
\hspace*{-6em}
{\small \bf
\begin{tabular}{*{26}{>{\centering\arraybackslash}m{0.15em}}}
%\begin{tabular}{*{26}{c}}
%\hline
\prive{A}&\prive{B}&\prive{C}&\prive{D}&\prive{E}&\prive{F}&\prive{G}&\prive{H}&\prive{I}&\prive{J}&\prive{K}&\prive{L}&\prive{M}&\prive{N}&\prive{O}&\prive{P}&\prive{Q}&\prive{R}&\prive{S}&\prive{T}&\prive{U}&\prive{V}&\prive{W}&\prive{X}&\prive{Y}&\prive{Z}\\
%\hline
\public{F}&\public{Q}&\public{B}&\public{M}&\public{X}&\public{I}&\public{T}&\public{E}&\public{P}&\public{A}&\public{L}&\public{W}&\public{H}&\public{S}&\public{D}&\public{O}&\public{Z}&\public{K}&\public{V}&\public{G}&\public{R}&\public{C}&\public{N}&\public{Y}&\public{J}&\public{U}\\
%\hline
\end{tabular}  
}
  \pause
  \medskip
  
  \item \prive{ETRE \ OU \ NE \ PAS \ ETRE \ TELLE \ EST \ LA \ QUESTION}
    \pause
  \item \prive{E} $\mapsto$ \public{X} \quad \prive{T} $\mapsto$ \public{G} \quad \prive{R} $\mapsto$ \public{K} \quad ...
  \pause
  \item \public{XGKX  DR  SX  OFV  XGKX  GXWWX  XVG  WF  ZRXVGPDS}
  \pause
  \item Pour le décrypter on fait les substitutions inverses
  \pause
  \item Avantage :  espace des clés gigantesque
  \pause
  \item Inconvénients
  \pause
  \begin{itemize}
    \item la clé est longue : $26$ lettres "FQBMX..."
  \pause  
    \item sécurité faible
  \end{itemize}

\end{itemize}

\end{frame}


\begin{frame}

\hfill\hfill\evidence{Espace des clés}

\begin{itemize}\setlength{\itemsep}{8pt}
  \item Une clé correspond à une bijection 
$$C : \big\{ A,B,\ldots,Z \big\}\longrightarrow \big\{ A,B,\ldots,Z \big\}$$
  \pause
  \item Il y a \  $26!$  \  choix possibles
  \begin{itemize}
    \pause
    \item Pour crypter la lettre A, il y a $26$ choix 
    \pause
    \item pour B, il reste $25$ choix
    \pause
    \item pour C, il reste $24$ choix...
    \pause
    \item pour Z, une seule possibilité
    \pause
    \item $26\times 25 \times 24 \times \cdots \times 2 \times 1 \pause = 26!$ \ choix de clés
  \end{itemize}
   
  \pause
  \item $26! =403\,291\,461\,126\,605\,635\,584\,000\,000\simeq 4 \times 10^{26}$ clés
  \pause
  \item Si un ordinateur pouvait tester $1 \; 000\; 000$ de clés par seconde, il lui faudrait alors 
$12$ millions d'années pour tout énumérer
\end{itemize}


\end{frame}


\begin{frame}

\hfill\hfill\evidence{Attaque statistique}
\pause

\begin{itemize}\setlength{\itemsep}{8pt}
  \item Faiblesse : une même lettre est toujours cryptée de la même façon
  \begin{center}\prive{E} $\longmapsto$ \public{X} \end{center}
  \pause
  \item Les lettres n'apparaissent pas avec la même fréquence
  \pause
  \item Lettres les plus fréquentes 
  
  \medskip
  \prive{E S A I N T R U L O D C P M V Q G F H B X J Y Z K W}
  
  
  \bigskip

\pause

\setlength{\extrarowheight}{3pt}  
\hspace*{-1em}{
\small
\begin{tabular}{*{9}{c}}
%\hline
\prive E&\prive S&\prive A&\prive I&\prive N&\prive T&\prive R&\prive U&\prive L\\
\hline
%14.699\%&8.012\%&7.540\%&7.184\%&6.895\%&6.888\%&6.496\%&6.129\%&5.636\%&5.298\%&3.661\%\\
%14.69\%&8.01\%&7.54\%&7.18\%&6.89\%&6.88\%&6.49\%&6.12\%&5.63\%&5.29\%&3.66\%\\
14.6\%&8.0\%&7.5\%&7.1\%&6.8\%&6.8\%&6.4\%&6.1\%&5.6\%\\
%\hline
\end{tabular}
}
  \pause
  \item Attaque
  \begin{itemize}
  \pause
    \item dans le texte crypté, on cherche la lettre qui apparaît le plus
\pause         
             \item cela devrait être $C(\prive{E})$
\pause    
    \item la lettre qui apparaît ensuite devrait être $C(\prive{S})$
\pause    
    \item puis $C(\prive{A})$...
\pause    
    \item message initial sous forme de texte à trous
\pause    
    \item deviner les lettres manquantes
  \end{itemize}
  
  
\end{itemize}
\end{frame}


\begin{frame}

\begin{itemize}
        \item Par exemple, tentons de déchiffrer la phrase suivante : \\
\centerline{\public{LHLZ \  HFQ  \ BC HFFPZ \  WH \  YOUPFH  \ MUPZH}}
\pause  
  \item On compte les apparitions des lettres : \\
\centerline{\public{H} : 6 \qquad \public{F} : 4 \qquad \public{P} : 3 \qquad \public{Z} : 3}
\pause  
  \item On suppose donc que le \public{H} crypte la lettre \prive{E}, le \public{F} la lettre \prive{S} \\
  \pause
\centerline{*\prive{E}** \ \prive{ES}* \ ** \ \prive{ESS}** \ *\prive{E} \ ***\prive{SE} \ ****\prive{E}}
\pause   
  \item
  \begin{itemize}
    \item  \public{P} et \public{Z} devraient se décrypter en \prive{A} et \prive{I} (ou \prive{I} et \prive{A})
\pause  
    \item le quatrième mot \public{HFFPZ} décrypté en \prive{ESS}**
\pause  
    \item se complèterait en \prive{ESSAI} ou \prive{ESSIA} 
\pause    
    \item Ainsi \public{P} crypte \prive{A} et \public{Z} crypte \prive{I}
\pause    
    \item La phrase est maintenant : \\
\centerline{*\prive{E}*\prive{I} \  \prive{ES}* \  ** \  \prive{ESSAI} \  *\prive{E} \ ***\prive{ASE} \ **\prive{AIE}}
  \end{itemize}
\pause  
  \item On décrypte le message : \\
\centerline{\prive{CECI \ EST \ UN \ ESSAI \ DE \ PHRASE \ VRAIE}}

\end{itemize}


\end{frame}


%%%%%%%%%%%%%%%%%%%%%%%%%%%%%%%%%%%%%%%%%%%%%%%%%%%%%%%%%%%%%%%%
\section{Le chiffre de Vigenère}




\begin{frame}

\hfill\hfill\evidence{Chiffrement de Vigenère}

\begin{itemize}
  \item On regroupe les lettres par blocs de longueur $k$
  \pause
  \item Exemple avec des blocs de longueur $k=4$ \\
  
\centerline{\prive{CETTE \ PHRASE \  NE \ VEUT \ RIEN \ DIRE}}
\pause
devient \\
\centerline{\prive{CETT \ EPHR \ ASEN \ EVEU \ TRIE \ NDIR \ E}}
  \pause
  \item Une clé est constituée de $k$ nombres de $0$ à $25$ : $(n_1,n_2,\ldots,n_k)$
  \pause
  \item Retour au chiffrement de César : 
\begin{itemize}
\pause
  \item décalage de $n_1$ pour la première lettre de chaque bloc
  \pause
  \item décalage de $n_2$ pour la deuxième lettre de chaque bloc
  \pause
  \item ...
  \item décalage de $n_k$ pour la $k$-ème lettre de chaque bloc 
\end{itemize}

  \pause
  \item Exemple : chiffrons le bloc \prive{CETT} avec la clé $(3,1,5,2)$
  \pause
  \begin{itemize}[<+->]
  \item décalage de $3$ pour \prive{C} donne \public{F}
  \item décalage de $1$ pour \prive{E} donne \public{F}
  \item décalage de $5$ pour le premier \prive{T} donne \public{Y}
  \item décalage de $2$ pour le deuxième \prive{T} donne \public{V}
  \item \prive{CETT} devient \public{FFYV}
\end{itemize}

\end{itemize}

\end{frame}

\begin{frame}

\hfill\hfill\evidence{Mathématiques}

\begin{itemize}\setlength{\itemsep}{8pt}
  \item L'élément de base est un \defi{bloc}
%  \pause
%  \item La fonction de chiffrement associe à un bloc de longueur $k$, un autre bloc de longueur $k$
  \pause
  \bigskip
  
\hspace*{-2em}\myboxinline{\small $C_{n_1,n_2,\ldots,n_k} \!:\! \left\{\begin{array}{rcl}
\Zz/26\Zz \times  \cdots \times \Zz/26\Zz
& \longrightarrow & \Zz/26\Zz \times  \cdots \times \Zz/26\Zz \\
 (x_1,x_2,\ldots,x_k) & \longmapsto & (x_1+n_1,x_2+n_2,\ldots,x_k+n_k) \\
\end{array}\right.\hspace*{-1em}$}
  \pause
  \item Chacune des composantes est un chiffrement de César
  \pause
  \item La fonction de déchiffrement est $C_{-n_1,-n_2,\ldots,-n_k}$
\end{itemize}

\end{frame}


\begin{frame}

\hfill\hfill\evidence{Espace des clés et attaque}

\pause

\begin{itemize}\setlength{\itemsep}{8pt}
  \item Pour une clé de longueur $k$, il y a $26^k$ choix possibles.
  \pause
  \item Exemple : $k=4$, $26^4 = 456\;976$ clés
  \pause
  \item Même faiblesse que le chiffrement par substitution
  \pause
  \item \prive{ALPH \ ABET} \pause les deux \prive{A} sont cryptés par la même lettre
\begin{center}\prive{ALPH \ ABET} $\longrightarrow$ \public{DMUJ \ DCJV} \end{center}
  \pause
  \item Attaque 
  \pause
  \begin{itemize}
    \item on découpe notre message en plusieurs listes
    \pause
    \item les premières lettres de chaque bloc, les deuxièmes lettres de chaque bloc...
    \pause
    \item attaque statistique sur chacun de ces regroupements
    \pause
    \item la taille des blocs doit être petite devant la longueur du texte
  \end{itemize}
  
\end{itemize}

\end{frame}



\end{document}
