
%%%%%%%%%%%%%%%%%% PREAMBULE %%%%%%%%%%%%%%%%%%

\documentclass[aspectratio=169,utf8]{beamer}
%\documentclass[aspectratio=169,handout]{beamer}

\usetheme{Boadilla}
%\usecolortheme{seahorse}
\usecolortheme[RGB={245,66,24}]{structure}
\useoutertheme{infolines}

% packages
\usepackage{amsfonts,amsmath,amssymb,amsthm}
\usepackage[utf8]{inputenc}
\usepackage[T1]{fontenc}
\usepackage{lmodern}

\usepackage[francais]{babel}
\usepackage{fancybox}
\usepackage{graphicx}

\usepackage{float}
\usepackage{xfrac}

%\usepackage[usenames, x11names]{xcolor}
\usepackage{tikz}
\usepackage{pgfplots}
\usepackage{datetime}



%-----  Package unités -----
\usepackage{siunitx}
\sisetup{locale = FR,detect-all,per-mode = symbol}

%\usepackage{mathptmx}
%\usepackage{fouriernc}
%\usepackage{newcent}
%\usepackage[mathcal,mathbf]{euler}

%\usepackage{palatino}
%\usepackage{newcent}
% \usepackage[mathcal,mathbf]{euler}



% \usepackage{hyperref}
% \hypersetup{colorlinks=true, linkcolor=blue, urlcolor=blue,
% pdftitle={Exo7 - Exercices de mathématiques}, pdfauthor={Exo7}}


%section
% \usepackage{sectsty}
% \allsectionsfont{\bf}
%\sectionfont{\color{Tomato3}\upshape\selectfont}
%\subsectionfont{\color{Tomato4}\upshape\selectfont}

%----- Ensembles : entiers, reels, complexes -----
\newcommand{\Nn}{\mathbb{N}} \newcommand{\N}{\mathbb{N}}
\newcommand{\Zz}{\mathbb{Z}} \newcommand{\Z}{\mathbb{Z}}
\newcommand{\Qq}{\mathbb{Q}} \newcommand{\Q}{\mathbb{Q}}
\newcommand{\Rr}{\mathbb{R}} \newcommand{\R}{\mathbb{R}}
\newcommand{\Cc}{\mathbb{C}} 
\newcommand{\Kk}{\mathbb{K}} \newcommand{\K}{\mathbb{K}}

%----- Modifications de symboles -----
\renewcommand{\epsilon}{\varepsilon}
\renewcommand{\Re}{\mathop{\text{Re}}\nolimits}
\renewcommand{\Im}{\mathop{\text{Im}}\nolimits}
%\newcommand{\llbracket}{\left[\kern-0.15em\left[}
%\newcommand{\rrbracket}{\right]\kern-0.15em\right]}

\renewcommand{\ge}{\geqslant}
\renewcommand{\geq}{\geqslant}
\renewcommand{\le}{\leqslant}
\renewcommand{\leq}{\leqslant}
\renewcommand{\epsilon}{\varepsilon}

%----- Fonctions usuelles -----
\newcommand{\ch}{\mathop{\text{ch}}\nolimits}
\newcommand{\sh}{\mathop{\text{sh}}\nolimits}
\renewcommand{\tanh}{\mathop{\text{th}}\nolimits}
\newcommand{\cotan}{\mathop{\text{cotan}}\nolimits}
\newcommand{\Arcsin}{\mathop{\text{arcsin}}\nolimits}
\newcommand{\Arccos}{\mathop{\text{arccos}}\nolimits}
\newcommand{\Arctan}{\mathop{\text{arctan}}\nolimits}
\newcommand{\Argsh}{\mathop{\text{argsh}}\nolimits}
\newcommand{\Argch}{\mathop{\text{argch}}\nolimits}
\newcommand{\Argth}{\mathop{\text{argth}}\nolimits}
\newcommand{\pgcd}{\mathop{\text{pgcd}}\nolimits} 


%----- Commandes divers ------
\newcommand{\ii}{\mathrm{i}}
\newcommand{\dd}{\text{d}}
\newcommand{\id}{\mathop{\text{id}}\nolimits}
\newcommand{\Ker}{\mathop{\text{Ker}}\nolimits}
\newcommand{\Card}{\mathop{\text{Card}}\nolimits}
\newcommand{\Vect}{\mathop{\text{Vect}}\nolimits}
\newcommand{\Mat}{\mathop{\text{Mat}}\nolimits}
\newcommand{\rg}{\mathop{\text{rg}}\nolimits}
\newcommand{\tr}{\mathop{\text{tr}}\nolimits}


%----- Structure des exercices ------

\newtheoremstyle{styleexo}% name
{2ex}% Space above
{3ex}% Space below
{}% Body font
{}% Indent amount 1
{\bfseries} % Theorem head font
{}% Punctuation after theorem head
{\newline}% Space after theorem head 2
{}% Theorem head spec (can be left empty, meaning ‘normal’)

%\theoremstyle{styleexo}
\newtheorem{exo}{Exercice}
\newtheorem{ind}{Indications}
\newtheorem{cor}{Correction}


\newcommand{\exercice}[1]{} \newcommand{\finexercice}{}
%\newcommand{\exercice}[1]{{\tiny\texttt{#1}}\vspace{-2ex}} % pour afficher le numero absolu, l'auteur...
\newcommand{\enonce}{\begin{exo}} \newcommand{\finenonce}{\end{exo}}
\newcommand{\indication}{\begin{ind}} \newcommand{\finindication}{\end{ind}}
\newcommand{\correction}{\begin{cor}} \newcommand{\fincorrection}{\end{cor}}

\newcommand{\noindication}{\stepcounter{ind}}
\newcommand{\nocorrection}{\stepcounter{cor}}

\newcommand{\fiche}[1]{} \newcommand{\finfiche}{}
\newcommand{\titre}[1]{\centerline{\large \bf #1}}
\newcommand{\addcommand}[1]{}
\newcommand{\video}[1]{}

% Marge
\newcommand{\mymargin}[1]{\marginpar{{\small #1}}}

\def\noqed{\renewcommand{\qedsymbol}{}}


%----- Presentation ------
\setlength{\parindent}{0cm}

%\newcommand{\ExoSept}{\href{http://exo7.emath.fr}{\textbf{\textsf{Exo7}}}}

\definecolor{myred}{rgb}{0.93,0.26,0}
\definecolor{myorange}{rgb}{0.97,0.58,0}
\definecolor{myyellow}{rgb}{1,0.86,0}

\newcommand{\LogoExoSept}[1]{  % input : echelle
{\usefont{U}{cmss}{bx}{n}
\begin{tikzpicture}[scale=0.1*#1,transform shape]
  \fill[color=myorange] (0,0)--(4,0)--(4,-4)--(0,-4)--cycle;
  \fill[color=myred] (0,0)--(0,3)--(-3,3)--(-3,0)--cycle;
  \fill[color=myyellow] (4,0)--(7,4)--(3,7)--(0,3)--cycle;
  \node[scale=5] at (3.5,3.5) {Exo7};
\end{tikzpicture}}
}


\newcommand{\debutmontitre}{
  \author{} \date{} 
  \thispagestyle{empty}
  \hspace*{-10ex}
  \begin{minipage}{\textwidth}
    \titlepage  
  \vspace*{-2.5cm}
  \begin{center}
    \LogoExoSept{2.5}
  \end{center}
  \end{minipage}

  \vspace*{-0cm}
  
  % Astuce pour que le background ne soit pas discrétisé lors de la conversion pdf -> png
\begin{tikzpicture}
        \fill[opacity=0,green!60!black] (0,0)--++(0,0)--++(0,0)--++(0,0)--cycle; 
\end{tikzpicture}

% toc S'affiche trop tot :
% \tableofcontents[hideallsubsections, pausesections]
}

\newcommand{\finmontitre}{
  \end{frame}
  \setcounter{framenumber}{0}
} % ne marche pas pour une raison obscure

%----- Commandes supplementaires ------

% \usepackage[landscape]{geometry}
% \geometry{top=1cm, bottom=3cm, left=2cm, right=10cm, marginparsep=1cm
% }
% \usepackage[a4paper]{geometry}
% \geometry{top=2cm, bottom=2cm, left=2cm, right=2cm, marginparsep=1cm
% }

%\usepackage{standalone}


% New command Arnaud -- november 2011
\setbeamersize{text margin left=24ex}
% si vous modifier cette valeur il faut aussi
% modifier le decalage du titre pour compenser
% (ex : ici =+10ex, titre =-5ex

\theoremstyle{definition}
%\newtheorem{proposition}{Proposition}
%\newtheorem{exemple}{Exemple}
%\newtheorem{theoreme}{Théorème}
%\newtheorem{lemme}{Lemme}
%\newtheorem{corollaire}{Corollaire}
%\newtheorem*{remarque*}{Remarque}
%\newtheorem*{miniexercice}{Mini-exercices}
%\newtheorem{definition}{Définition}

% Commande tikz
\usetikzlibrary{calc}
\usetikzlibrary{patterns,arrows}
\usetikzlibrary{matrix}
\usetikzlibrary{fadings} 

%definition d'un terme
\newcommand{\defi}[1]{{\color{myorange}\textbf{\emph{#1}}}}
\newcommand{\evidence}[1]{{\color{blue}\textbf{\emph{#1}}}}
\newcommand{\assertion}[1]{\emph{\og#1\fg}}  % pour chapitre logique
%\renewcommand{\contentsname}{Sommaire}
\renewcommand{\contentsname}{}
\setcounter{tocdepth}{2}



%------ Figures ------

\def\myscale{1} % par défaut 
\newcommand{\myfigure}[2]{  % entrée : echelle, fichier figure
\def\myscale{#1}
\begin{center}
\footnotesize
{#2}
\end{center}}


%------ Encadrement ------

\usepackage{fancybox}


\newcommand{\mybox}[1]{
\setlength{\fboxsep}{7pt}
\begin{center}
\shadowbox{#1}
\end{center}}

\newcommand{\myboxinline}[1]{
\setlength{\fboxsep}{5pt}
\raisebox{-10pt}{
\shadowbox{#1}
}
}

%--------------- Commande beamer---------------
\newcommand{\beameronly}[1]{#1} % permet de mettre des pause dans beamer pas dans poly


\setbeamertemplate{navigation symbols}{}
\setbeamertemplate{footline}  % tiré du fichier beamerouterinfolines.sty
{
  \leavevmode%
  \hbox{%
  \begin{beamercolorbox}[wd=.333333\paperwidth,ht=2.25ex,dp=1ex,center]{author in head/foot}%
    % \usebeamerfont{author in head/foot}\insertshortauthor%~~(\insertshortinstitute)
    \usebeamerfont{section in head/foot}{\bf\insertshorttitle}
  \end{beamercolorbox}%
  \begin{beamercolorbox}[wd=.333333\paperwidth,ht=2.25ex,dp=1ex,center]{title in head/foot}%
    \usebeamerfont{section in head/foot}{\bf\insertsectionhead}
  \end{beamercolorbox}%
  \begin{beamercolorbox}[wd=.333333\paperwidth,ht=2.25ex,dp=1ex,right]{date in head/foot}%
    % \usebeamerfont{date in head/foot}\insertshortdate{}\hspace*{2em}
    \insertframenumber{} / \inserttotalframenumber\hspace*{2ex} 
  \end{beamercolorbox}}%
  \vskip0pt%
}


\definecolor{mygrey}{rgb}{0.5,0.5,0.5}
\setlength{\parindent}{0cm}
%\DeclareTextFontCommand{\helvetica}{\fontfamily{phv}\selectfont}

% background beamer
\definecolor{couleurhaut}{rgb}{0.85,0.9,1}  % creme
\definecolor{couleurmilieu}{rgb}{1,1,1}  % vert pale
\definecolor{couleurbas}{rgb}{0.85,0.9,1}  % blanc
\setbeamertemplate{background canvas}[vertical shading]%
[top=couleurhaut,middle=couleurmilieu,midpoint=0.4,bottom=couleurbas] 
%[top=fondtitre!05,bottom=fondtitre!60]



\makeatletter
\setbeamertemplate{theorem begin}
{%
  \begin{\inserttheoremblockenv}
  {%
    \inserttheoremheadfont
    \inserttheoremname
    \inserttheoremnumber
    \ifx\inserttheoremaddition\@empty\else\ (\inserttheoremaddition)\fi%
    \inserttheorempunctuation
  }%
}
\setbeamertemplate{theorem end}{\end{\inserttheoremblockenv}}

\newenvironment{theoreme}[1][]{%
   \setbeamercolor{block title}{fg=structure,bg=structure!40}
   \setbeamercolor{block body}{fg=black,bg=structure!10}
   \begin{block}{{\bf Th\'eor\`eme }#1}
}{%
   \end{block}%
}


\newenvironment{proposition}[1][]{%
   \setbeamercolor{block title}{fg=structure,bg=structure!40}
   \setbeamercolor{block body}{fg=black,bg=structure!10}
   \begin{block}{{\bf Proposition }#1}
}{%
   \end{block}%
}

\newenvironment{corollaire}[1][]{%
   \setbeamercolor{block title}{fg=structure,bg=structure!40}
   \setbeamercolor{block body}{fg=black,bg=structure!10}
   \begin{block}{{\bf Corollaire }#1}
}{%
   \end{block}%
}

\newenvironment{mydefinition}[1][]{%
   \setbeamercolor{block title}{fg=structure,bg=structure!40}
   \setbeamercolor{block body}{fg=black,bg=structure!10}
   \begin{block}{{\bf Définition} #1}
}{%
   \end{block}%
}

\newenvironment{lemme}[0]{%
   \setbeamercolor{block title}{fg=structure,bg=structure!40}
   \setbeamercolor{block body}{fg=black,bg=structure!10}
   \begin{block}{\bf Lemme}
}{%
   \end{block}%
}

\newenvironment{remarque}[1][]{%
   \setbeamercolor{block title}{fg=black,bg=structure!20}
   \setbeamercolor{block body}{fg=black,bg=structure!5}
   \begin{block}{Remarque #1}
}{%
   \end{block}%
}


\newenvironment{exemple}[1][]{%
   \setbeamercolor{block title}{fg=black,bg=structure!20}
   \setbeamercolor{block body}{fg=black,bg=structure!5}
   \begin{block}{{\bf Exemple }#1}
}{%
   \end{block}%
}


\newenvironment{miniexercice}[0]{%
   \setbeamercolor{block title}{fg=structure,bg=structure!20}
   \setbeamercolor{block body}{fg=black,bg=structure!5}
   \begin{block}{Mini-exercices}
}{%
   \end{block}%
}


\newenvironment{tp}[0]{%
   \setbeamercolor{block title}{fg=structure,bg=structure!40}
   \setbeamercolor{block body}{fg=black,bg=structure!10}
   \begin{block}{\bf Travaux pratiques}
}{%
   \end{block}%
}
\newenvironment{exercicecours}[1][]{%
   \setbeamercolor{block title}{fg=structure,bg=structure!40}
   \setbeamercolor{block body}{fg=black,bg=structure!10}
   \begin{block}{{\bf Exercice }#1}
}{%
   \end{block}%
}
\newenvironment{algo}[1][]{%
   \setbeamercolor{block title}{fg=structure,bg=structure!40}
   \setbeamercolor{block body}{fg=black,bg=structure!10}
   \begin{block}{{\bf Algorithme}\hfill{\color{gray}\texttt{#1}}}
}{%
   \end{block}%
}


\setbeamertemplate{proof begin}{
   \setbeamercolor{block title}{fg=black,bg=structure!20}
   \setbeamercolor{block body}{fg=black,bg=structure!5}
   \begin{block}{{\footnotesize Démonstration}}
   \footnotesize
   \smallskip}
\setbeamertemplate{proof end}{%
   \end{block}}
\setbeamertemplate{qed symbol}{\openbox}


\makeatother
\usecolortheme[RGB={192,41,0}]{structure}

% Commande spécifique à ce chapitre
\newcommand{\Sage}{\texttt{Sage}}

\usepackage{textcomp}

\usepackage{listings}
\lstset{
  upquote=true,
  columns=flexible,
  keepspaces=true,
  basicstyle=\ttfamily,
  commentstyle=\color{gray},
  language=Python,
  showstringspaces=false,
  aboveskip=0em,  
  belowskip=0em,
  escapeinside=||,
  breaklines=true,
  postbreak=\raisebox{0ex}[0ex][0ex]{\qquad\ensuremath{\color{red}\hookrightarrow\space}},
}

\lstset{
  literate={é}{{\'e}}1
           {è}{{\`e}}1
           {à}{{\`a}}1
}

\newcommand{\codeinline}[1]{\lstinline!#1!}


%%%%%%%%%%%%%%%%%%%%%%%%%%%%%%%%%%%%%%%%%%%%%%%%%%%%%%%%%%%%%
%%%%%%%%%%%%%%%%%%%%%%%%%%%%%%%%%%%%%%%%%%%%%%%%%%%%%%%%%%%%%


\begin{document}


\title{{\bf Calcul formel}}
\subtitle{Structures de contrôle avec \Sage}

\begin{frame}
  
  \debutmontitre

  \pause

{\footnotesize
\hfill
\setbeamercovered{transparent=50}
\begin{minipage}{0.6\textwidth}
  \begin{itemize}
    \item<3-> Boucles
    \item<4-> Booléens et conditions
    \item<5-> Fonctions informatiques
    \item<6-> Variable locale/variable globale 
    \item<7-> Conjecture de Syracuse     
  \end{itemize}
\end{minipage}
}

\end{frame}

\setcounter{framenumber}{0}





%%%%%%%%%%%%%%%%%%%%%%%%%%%%%%%%%%%%%%%%%%%%%%%%%%%%%%%%%%%%%%%%
\section{Boucles}

\begin{frame}[fragile]
%--------------------
\evidence{Boucle \codeinline{for} (pour)} 

\medskip
\pause

%\insertcode{formel/Algos/structures-tex1.sage}{structures.sage (1)}
\begin{algo}[structures.sage (1)]
\begin{lstlisting}
for x in ensemble:
    première ligne de la boucle
    deuxième ligne de la boucle
    ...
    dernière ligne de la boucle
instructions suivantes  
\end{lstlisting}
\end{algo}

\end{frame}


\begin{frame}
%--------------------
\evidence{Liste \codeinline{range} (intervalle)}

\pause
\medskip
\begin{itemize}
  \item \codeinline{for k in range(n):} 

    qui fait varier $k$ de $0$ à $n-1$
  \smallskip\pause
  \item \codeinline{range(n)} renvoie \codeinline{[0,1,2,...,n-1]}
   \smallskip\pause
  \item \codeinline{range(a,b)} renvoie \codeinline{[a,a+1,...,b-1]}
   \smallskip\pause
  \item \codeinline{range(a,b,c)} effectue une saut de $c$ termes
  
   \smallskip\pause
  Exemple : \codeinline{range(0,101,5)} renvoie \codeinline{[0,5,10,15,...,100]}
   \smallskip\pause
  \item \codeinline{[a..b]} : liste des entiers $k$ tels que $a\le k \le b$
  
   \smallskip\pause
  \item \codeinline{for x in [0.13,0.31,0.53,0.98]:} 
  
  fait prendre à $x$ les $4$ valeurs
de $\{0.13 , 0.31, 0.53, 0.98\}$
\end{itemize}

\end{frame}


\begin{frame}[fragile]
%--------------------
\evidence{Boucle \codeinline{while} (tant que)}

\medskip
\pause

%\insertcode{formel/Algos/structures-tex2.sage}{structures.sage (2)}
\begin{algo}[structures.sage (2)]
\begin{lstlisting}
while condition:
    première ligne de la boucle
    deuxième ligne de la boucle
    ...
    dernière ligne de la boucle
instructions suivantes
\end{lstlisting}
\end{algo}

\end{frame}


\begin{frame}[fragile]

Voici le calcul de la racine carrée entière d'un entier $n$ 
\medskip
\pause

%\insertcode{formel/Algos/structures-tex3.sage}{structures.sage (3)}
\begin{algo}[structures.sage (3)]
\begin{lstlisting}
n = 123456789     # l'entier, on veut la racine carrée|\pause|
k = 1             # le premier candidat|\pause|
while k*k <= n:   # tant que k^2 ne dépasse pas n|\pause|
    k = k+1       # on passe au candidat suivant|\pause|
print(k-1)        # la racine cherchée    
\end{lstlisting}
\end{algo}

\medskip
\pause
\`A la fin, la racine carrée entière de $n$ est $k-1$

\end{frame}


\begin{frame}[fragile]

%--------------------
\evidence{Test \codeinline{if... else} (si... sinon)} 
 
 \medskip
 \pause
 
%\insertcode{formel/Algos/structures-tex4.sage}{structures.sage (4)}
\begin{algo}[structures.sage (4)]
\begin{lstlisting}
if condition:
    première ligne d'instruction
    ...
    dernière ligne d'instruction
else:
    première ligne d'instruction
    ...
    dernière ligne d'instruction
autres instructions
\end{lstlisting}
\end{algo}

\end{frame}




%%%%%%%%%%%%%%%%%%%%%%%%%%%%%%%%%%%%%%%%%%%%%%%%%%%%%%%%%%%%%%%%
\section{Booléens et conditions}

\begin{frame}

%--------------------
\evidence{Booléens}

\medskip
\pause

Une \defi{expression booléenne} prend 
les valeurs  \og Vrai \fg\ (\codeinline{True})
ou \og Faux \fg\ (\codeinline{False})


\bigskip
\pause

%--------------------
\evidence{Quelques conditions}

\pause

\begin{itemize}[<+->]
  \item \codeinline{a < b} : teste l'inégalité stricte $a<b$
  \item \codeinline{a > b} : teste l'inégalité stricte $a>b$
  \item \codeinline{a <= b} : teste l'inégalité large $a\le b$
  \item \codeinline{a >= b} : teste l'inégalité large $a\ge b$
  \item \codeinline{a == b} : teste l'égalité $a=b$
  \item \codeinline{a <> b} (ou \codeinline{a != b}) : teste la non égalité $a\neq b$
  \item \codeinline{a in B} : teste si l'élément $a$ appartient à $B$
\end{itemize}

\end{frame}


\begin{frame}

\begin{remarque}  
\pause 
\begin{itemize}
  \item Une condition prend la valeur \codeinline{True} 
  si elle est vraie et \codeinline{False} sinon

  \pause  
\smallskip

Exemple : \codeinline{x == 2} renvoie \codeinline{True} si $x$ vaut $2$ et \codeinline{False} sinon

\medskip
\pause

  \item Ne pas confondre le test d'égalité
\codeinline{x == 2} avec l'affectation \codeinline{x = 2} 

\pause
\medskip


  \item Combiner des conditions avec les opérateurs \codeinline{and},
  \codeinline{or}, \codeinline{not} 
  
\smallskip
\pause

  Exemple :
  $$\mbox{\codeinline{ (n>0) and (not (is_prime(n))) }}$$
 est vraie
  si et seulement si $n$ est strictement positif et non premier
  
\end{itemize}  
\end{remarque}
\end{frame}


\begin{frame}[fragile]
\begin{tp}~
\begin{enumerate}
  \item Pour deux assertions logiques $P$, $Q$ écrire l'assertion $P\implies Q$.
  
  \item Une \defi{tautologie} est une assertion vraie quelles que soient les valeurs des paramètres,
  par exemple $(P \text{ ou } (\text{non\;} P))$ est vraie que l'assertion $P$ soit vraie ou fausse.
  Montrer que l'assertion suivante est une tautologie :
  
  \vspace*{-2ex}
  $$\text{non\;} \bigg( \big[\text{non\;}(P \text{ et } Q)  \text{ et } (Q \text{ ou } R) \big] 
  \text{ et } \big[P \text{ et } (\text{non\;} R)\big] \bigg)$$
\end{enumerate}
  
\end{tp}

\pause
\smallskip

\og$P\implies Q$\fg\ : \codeinline{not(P) or Q}

\pause

%\insertcode{formel/Algos/structures-tex10.sage}{structures.sage (10)}
{\footnotesize
\begin{algo}[structures.sage (5)]
\begin{lstlisting}
for P in {True, False}:
  for Q in {True, False}:
    for R in {True, False}:
      print(not( (not(P and Q) and (Q or R)) and ( P and (not R) ) ))
\end{lstlisting}
\end{algo}
}

\end{frame}




%%%%%%%%%%%%%%%%%%%%%%%%%%%%%%%%%%%%%%%%%%%%%%%%%%%%%%%%%%%%%%%%
\section{Fonctions}


\begin{frame}[fragile]


\evidence{Fonctions informatiques}

\medskip
\pause


%\insertcode{formel/Algos/structures-tex5.sage}{structures.sage (5)}
\begin{algo}[structures.sage (6)]
\begin{lstlisting}
def mafonction (mesvariables):
    première ligne d'instruction
    ...
    dernière ligne d'instruction    
    return monresultat
\end{lstlisting}
\end{algo}

\end{frame}


\begin{frame}[fragile]
%\insertcode{formel/Algos/structures-tex6.sage}{structures.sage (6)}
\begin{algo}[structures.sage (7)]
\begin{lstlisting}
def valeur_absolue(x):
    if x >= 0:
        return x
    else: 
        return -x
\end{lstlisting}
\end{algo}

\pause
\codeinline{valeur_absolue(-3)} renvoie $+3$

\bigskip
\pause

Que fait la fonction suivante ? Quel nom aurait été plus approprié ?

%\insertcode{formel/Algos/structures-tex7.sage}{structures.sage (7)}
\begin{algo}[structures.sage (8)]
\begin{lstlisting}
def ma_fonction(x,y):
    resultat = (x+y+valeur_absolue(x-y))/2
    return resultat
\end{lstlisting}
\end{algo}
\end{frame}


% \begin{frame}[fragile]
% \begin{tp}~
% On considère trois réels distincts $a,b,c$ et on définit le polynôme
% $$P(X) = \frac{(X-a)(X-b)}{(c-a)(c-b)}+\frac{(X-b)(X-c)}{(a-b)(a-c)}
% +  \frac{(X-a)(X-c)}{(b-a)(b-c)}-1$$
% \begin{enumerate}
%   \item Définir trois variables  \codeinline{var('a,b,c')}.
%   Définir une fonction \codeinline{polynome(x)} qui renvoie $P(x)$.
%   
%   \item Calculer $P(a)$, $P(b)$, $P(c)$.
%   
%   \item Comment un polynôme de degré $2$ pourrait avoir $3$ racines distinctes ?
%   Expliquez ! Vous pourrez utiliser la méthode \codeinline{full_simplify()} à votre fonction.
% \end{enumerate}
%   
% \end{tp}
% \end{frame}


\begin{frame}[fragile]

Qu'affiche l'instruction \codeinline{print(x)} dans ce petit programme ?

\medskip

%\insertcode{formel/Algos/structures-tex9.sage}{structures.sage (9)}
\begin{algo}[structures.sage (9)]
\begin{lstlisting}
x = 2
def incremente(x):
    x = x+1
    return x
incremente(x)
print(x)
\end{lstlisting}
\end{algo}

\pause
\bigskip

$$\sum_{k=0}^n k^2$$

\end{frame}


%%%%%%%%%%%%%%%%%%%%%%%%%%%%%%%%%%%%%%%%%%%%%%%%%%%%%%%%%%%%%%%%
\section{Conjecture de Syracuse}

\begin{frame}
\begin{tp}
La \defi{suite de Syracuse} de terme initial $u_0 \in \Nn^*$ est définie par récurrence
pour $n\ge0$ par :
$$u_{n+1} = 
\begin{cases}
  \frac{u_n}{2} & \text{ si $u_n$ est pair} \\
  3u_n + 1 & \text{ si $u_n$ est impair} \\
  \end{cases}$$
\begin{enumerate}
  \item \'Ecrire une fonction qui, à partir de $u_0$ et de $n$, calcule le terme $u_n$ de la suite de Syracuse.
  \item Vérifier pour différentes valeurs du terme initial $u_0$, que la suite de Syracuse
  atteint la valeur $1$ au bout d'un certain rang (puis devient périodique : 
  $...,1,4,2,1,4,2,...$). \'Ecrire une fonction qui pour un certain choix de $u_0$ 
  renvoie le premier rang $n$ tel que $u_n=1$.
\end{enumerate}

\end{tp}

\end{frame}


\begin{frame}[fragile]

%\insertcode{formel/Algos/suite-syracuse-tex1.sage}{suite-syracuse.sage (1)}
\begin{algo}[suite-syracuse.sage (1)]
\begin{lstlisting}
def syracuse(u0,n):
    u = u0
    for k in range(n):
        if u%2 == 0:
            u = u//2
        else:
            u = 3*u+1
    return u
\end{lstlisting}
\end{algo}
\end{frame}


\begin{frame}


Si $a \in \Zz$ et $n\in\Nn^*$ alors
\begin{itemize}
  \item \codeinline{a\%n} est le \defi{reste} de la division euclidienne de $a$ par $n$.
 
\pause
C'est donc l'entier $r$ tel que $r \equiv a \pmod{n}$ et $0 \le r < n$

 
\pause  

  \item \codeinline{a//n} est le \defi{quotient} de la division euclidienne de $a$ par $n$.
 
\pause  
  C'est donc l'entier $q$ tel que $a=nq+r$ et $0\leq r <n$
  
  
\end{itemize}

 
\pause
\bigskip
\begin{itemize}
  \item Ne pas confondre \codeinline{a/b} (division de nombres réels)
  et \\
  \codeinline{a//b} (quotient de la division euclidienne de deux entiers)  
\pause
  \begin{itemize}
    \item \codeinline{7/2} vaut $\frac72 = 3,5$
    \smallskip
    \item \codeinline{7//2} vaut $3$
  \end{itemize}
 
\pause  
  \item \codeinline{u\%2 == 0} teste si $u$ est pair
  
\end{itemize}


\end{frame}


\begin{frame}[fragile]

%\insertcode{formel/Algos/suite-syracuse-tex2.sage}{suite-syracuse.sage (2)}
\begin{algo}[suite-syracuse.sage (2)]
\begin{lstlisting}
def vol_syracuse(u0):
    u = u0
    n = 0
    while u <> 1:
        n = n+1
        if u%2 == 0:
            u = u//2
        else:
            u = 3*u+1
    return n
\end{lstlisting}
\end{algo}

\end{frame}



\end{document}
