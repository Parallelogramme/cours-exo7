
%%%%%%%%%%%%%%%%%% PREAMBULE %%%%%%%%%%%%%%%%%%


\documentclass[12pt]{article}

\usepackage{amsfonts,amsmath,amssymb,amsthm}
\usepackage[utf8]{inputenc}
\usepackage[T1]{fontenc}
\usepackage[francais]{babel}


% packages
\usepackage{amsfonts,amsmath,amssymb,amsthm}
\usepackage[utf8]{inputenc}
\usepackage[T1]{fontenc}
%\usepackage{lmodern}

\usepackage[francais]{babel}
\usepackage{fancybox}
\usepackage{graphicx}

\usepackage{float}

%\usepackage[usenames, x11names]{xcolor}
\usepackage{tikz}
\usepackage{datetime}

\usepackage{mathptmx}
%\usepackage{fouriernc}
%\usepackage{newcent}
\usepackage[mathcal,mathbf]{euler}

%\usepackage{palatino}
%\usepackage{newcent}


% Commande spéciale prompteur

%\usepackage{mathptmx}
%\usepackage[mathcal,mathbf]{euler}
%\usepackage{mathpple,multido}

\usepackage[a4paper]{geometry}
\geometry{top=2cm, bottom=2cm, left=1cm, right=1cm, marginparsep=1cm}

\newcommand{\change}{{\color{red}\rule{\textwidth}{1mm}\\}}

\newcounter{mydiapo}

\newcommand{\diapo}{\newpage
\hfill {\normalsize  Diapo \themydiapo \quad \texttt{[\jobname]}} \\
\stepcounter{mydiapo}}


%%%%%%% COULEURS %%%%%%%%%%

% Pour blanc sur noir :
%\pagecolor[rgb]{0.5,0.5,0.5}
% \pagecolor[rgb]{0,0,0}
% \color[rgb]{1,1,1}



%\DeclareFixedFont{\myfont}{U}{cmss}{bx}{n}{18pt}
\newcommand{\debuttexte}{
%%%%%%%%%%%%% FONTES %%%%%%%%%%%%%
\renewcommand{\baselinestretch}{1.5}
\usefont{U}{cmss}{bx}{n}
\bfseries

% Taille normale : commenter le reste !
%Taille Arnaud
%\fontsize{19}{19}\selectfont

% Taille Barbara
%\fontsize{21}{22}\selectfont

%Taille François
\fontsize{25}{30}\selectfont

%Taille Pascal
%\fontsize{25}{30}\selectfont

%Taille Laura
%\fontsize{30}{35}\selectfont


%\myfont
%\usefont{U}{cmss}{bx}{n}

%\Huge
%\addtolength{\parskip}{\baselineskip}
}


% \usepackage{hyperref}
% \hypersetup{colorlinks=true, linkcolor=blue, urlcolor=blue,
% pdftitle={Exo7 - Exercices de mathématiques}, pdfauthor={Exo7}}


%section
% \usepackage{sectsty}
% \allsectionsfont{\bf}
%\sectionfont{\color{Tomato3}\upshape\selectfont}
%\subsectionfont{\color{Tomato4}\upshape\selectfont}

%----- Ensembles : entiers, reels, complexes -----
\newcommand{\Nn}{\mathbb{N}} \newcommand{\N}{\mathbb{N}}
\newcommand{\Zz}{\mathbb{Z}} \newcommand{\Z}{\mathbb{Z}}
\newcommand{\Qq}{\mathbb{Q}} \newcommand{\Q}{\mathbb{Q}}
\newcommand{\Rr}{\mathbb{R}} \newcommand{\R}{\mathbb{R}}
\newcommand{\Cc}{\mathbb{C}} 
\newcommand{\Kk}{\mathbb{K}} \newcommand{\K}{\mathbb{K}}

%----- Modifications de symboles -----
\renewcommand{\epsilon}{\varepsilon}
\renewcommand{\Re}{\mathop{\text{Re}}\nolimits}
\renewcommand{\Im}{\mathop{\text{Im}}\nolimits}
%\newcommand{\llbracket}{\left[\kern-0.15em\left[}
%\newcommand{\rrbracket}{\right]\kern-0.15em\right]}

\renewcommand{\ge}{\geqslant}
\renewcommand{\geq}{\geqslant}
\renewcommand{\le}{\leqslant}
\renewcommand{\leq}{\leqslant}

%----- Fonctions usuelles -----
\newcommand{\ch}{\mathop{\mathrm{ch}}\nolimits}
\newcommand{\sh}{\mathop{\mathrm{sh}}\nolimits}
\renewcommand{\tanh}{\mathop{\mathrm{th}}\nolimits}
\newcommand{\cotan}{\mathop{\mathrm{cotan}}\nolimits}
\newcommand{\Arcsin}{\mathop{\mathrm{Arcsin}}\nolimits}
\newcommand{\Arccos}{\mathop{\mathrm{Arccos}}\nolimits}
\newcommand{\Arctan}{\mathop{\mathrm{Arctan}}\nolimits}
\newcommand{\Argsh}{\mathop{\mathrm{Argsh}}\nolimits}
\newcommand{\Argch}{\mathop{\mathrm{Argch}}\nolimits}
\newcommand{\Argth}{\mathop{\mathrm{Argth}}\nolimits}
\newcommand{\pgcd}{\mathop{\mathrm{pgcd}}\nolimits} 

\newcommand{\Card}{\mathop{\text{Card}}\nolimits}
\newcommand{\Ker}{\mathop{\text{Ker}}\nolimits}
\newcommand{\id}{\mathop{\text{id}}\nolimits}
\newcommand{\ii}{\mathrm{i}}
\newcommand{\dd}{\mathrm{d}}
\newcommand{\Vect}{\mathop{\text{Vect}}\nolimits}
\newcommand{\Mat}{\mathop{\mathrm{Mat}}\nolimits}
\newcommand{\rg}{\mathop{\text{rg}}\nolimits}
\newcommand{\tr}{\mathop{\text{tr}}\nolimits}
\newcommand{\ppcm}{\mathop{\text{ppcm}}\nolimits}

%----- Structure des exercices ------

\newtheoremstyle{styleexo}% name
{2ex}% Space above
{3ex}% Space below
{}% Body font
{}% Indent amount 1
{\bfseries} % Theorem head font
{}% Punctuation after theorem head
{\newline}% Space after theorem head 2
{}% Theorem head spec (can be left empty, meaning ‘normal’)

%\theoremstyle{styleexo}
\newtheorem{exo}{Exercice}
\newtheorem{ind}{Indications}
\newtheorem{cor}{Correction}


\newcommand{\exercice}[1]{} \newcommand{\finexercice}{}
%\newcommand{\exercice}[1]{{\tiny\texttt{#1}}\vspace{-2ex}} % pour afficher le numero absolu, l'auteur...
\newcommand{\enonce}{\begin{exo}} \newcommand{\finenonce}{\end{exo}}
\newcommand{\indication}{\begin{ind}} \newcommand{\finindication}{\end{ind}}
\newcommand{\correction}{\begin{cor}} \newcommand{\fincorrection}{\end{cor}}

\newcommand{\noindication}{\stepcounter{ind}}
\newcommand{\nocorrection}{\stepcounter{cor}}

\newcommand{\fiche}[1]{} \newcommand{\finfiche}{}
\newcommand{\titre}[1]{\centerline{\large \bf #1}}
\newcommand{\addcommand}[1]{}
\newcommand{\video}[1]{}

% Marge
\newcommand{\mymargin}[1]{\marginpar{{\small #1}}}



%----- Presentation ------
\setlength{\parindent}{0cm}

%\newcommand{\ExoSept}{\href{http://exo7.emath.fr}{\textbf{\textsf{Exo7}}}}

\definecolor{myred}{rgb}{0.93,0.26,0}
\definecolor{myorange}{rgb}{0.97,0.58,0}
\definecolor{myyellow}{rgb}{1,0.86,0}

\newcommand{\LogoExoSept}[1]{  % input : echelle
{\usefont{U}{cmss}{bx}{n}
\begin{tikzpicture}[scale=0.1*#1,transform shape]
  \fill[color=myorange] (0,0)--(4,0)--(4,-4)--(0,-4)--cycle;
  \fill[color=myred] (0,0)--(0,3)--(-3,3)--(-3,0)--cycle;
  \fill[color=myyellow] (4,0)--(7,4)--(3,7)--(0,3)--cycle;
  \node[scale=5] at (3.5,3.5) {Exo7};
\end{tikzpicture}}
}



\theoremstyle{definition}
%\newtheorem{proposition}{Proposition}
%\newtheorem{exemple}{Exemple}
%\newtheorem{theoreme}{Théorème}
\newtheorem{lemme}{Lemme}
\newtheorem{corollaire}{Corollaire}
%\newtheorem*{remarque*}{Remarque}
%\newtheorem*{miniexercice}{Mini-exercices}
%\newtheorem{definition}{Définition}




%definition d'un terme
\newcommand{\defi}[1]{{\color{myorange}\textbf{\emph{#1}}}}
\newcommand{\evidence}[1]{{\color{blue}\textbf{\emph{#1}}}}



 %----- Commandes divers ------

\newcommand{\codeinline}[1]{\texttt{#1}}

\newcommand{\Sage}{\texttt{Sage}}

%%%%%%%%%%%%%%%%%%%%%%%%%%%%%%%%%%%%%%%%%%%%%%%%%%%%%%%%%%%%%
%%%%%%%%%%%%%%%%%%%%%%%%%%%%%%%%%%%%%%%%%%%%%%%%%%%%%%%%%%%%%


\begin{document}

\debuttexte


%%%%%%%%%%%%%%%%%%%%%%%%%%%%%%%%%%%%%%%%%%%%%%%%%%%%%%%%%%%
\diapo

Bonjour,

Nous allons utiliser la puissance du calcul formel, non seulement pour faire des expérimentations sophistiquées, mais aussi pour effectuer l'essentiel des preuves (presque) à notre place.

%Non seulement nous allons pouvoir faire des expérimentations
%sophistiquées mais en plus l'ordinateur va faire l'essentiel
%des preuves à notre place.

\change

\change
Nous allons étudier en détail la suite de Fibonacci

\change
et nous prouverons ensuite l'identité de Cassini.

%%%%%%%%%%%%%%%%%%%%%%%%%%%%%%%%%%%%%%%%%%%%%%%%%%%%%%%%%%%
\diapo

Comment la machine peut-elle nous aider à faire des preuves ?

\change
Déclarons deux variables $a$ et $b$
par l'instruction \codeinline{var('a,b')}.

\change
Alors la commande

$expand( (a+b)^2-a^2-2*a*b-b^2 )$

renvoie $0$.

\change
Ce simple calcul doit être considéré comme une preuve par ordinateur
de l'identité 
$$(a+b)^2 = a^2+2ab+b^2$$
sans qu'il y ait obligation de vérification humaine.

\change
Cette "preuve" est fondée sur la confiance faite à \Sage\ 
pour manipuler les expressions algébriques.

\change
Nous allons maintenant mettre en oeuvre la démarche 
que l'on a adoptée, à savoir :

\change
\begin{enumerate}
  \item Je me pose des questions.

  \change
  \item J'écris un algorithme pour tenter d'y répondre.

  \change
  \item Je trouve une conjecture sur la base des résultats expérimentaux.

  \change
  \item Je prouve la conjecture.
\end{enumerate}


%%%%%%%%%%%%%%%%%%%%%%%%%%%%%%%%%%%%%%%%%%%%%%%%%%%%%%%%%%%
\diapo

La suite de Fibonacci est une suite récurrente double.

On initialise la suite par 
$F_0 = 0 \qquad F_1 = 1$.

Puis, pour $n\ge 2$, le terme général est défini par 
$F_{n} = F_{n-1} + F_{n-2}$.

\change
Voici un énoncé qui va nous conduire à la preuve d'un résultat que vous connaissez certainement.

Les questions permettent de suivre pas à pas notre démarche.

Dans un premier temps, nous construisons une fonction qui
calcule le terme $F_n$ pour un $n$ donné.

Ensuite, nous allons déterminer quelles pourraient être les valeurs des constantes $c$, $\phi$ et $\psi$ qui 
conviendraient pour avoir l'égalité
$$F_n = c \big(\phi^n-\psi^n\big)$$

Enfin, à l'aide de la machine, il sera établi formellement 
que la suite de Fibonacci s'écrit effectivement
sous cette forme.  $\frac{1}{\sqrt{5}}  \big(\phi^n-\psi^n\big)$.


Arrêtez ici la vidéo, lancez le logiciel Sage et tentez de répondre aux questions posées.

Ensuite continuer la vidéo.

PAUSE 3s


%%%%%%%%%%%%%%%%%%%%%%%%%%%%%%%%%%%%%%%%%%%%%%%%%%%%%%%%%%%
\diapo

Voici ce que vous devez avoir obtenu :

$F_0 = 0 \qquad F_1 = 1$ 

puis  
$1 \qquad 2 \qquad 3 \qquad 5 \qquad 8 \qquad 13 \qquad 21\qquad 34 $.


\change

Voici une fonction qui calcule le terme de rang $n$ de 
la suite de Fibonacci.

Tout d'abord si $n=0$ et $n=1$ alors on renvoie directement le résultat et on s'arrête là. En effet ces deux valeurs ne sont pas calculables, elles font partie de la définition de la suite.

Pour calculer un terme de rang $n\ge2$, nous allons appliquer 
la formule de récurrence.

Tout d'abord on initialise deux termes, 

celui-ci joue le rôle de
$F$ indice $n-2$ 

et celui-là le rôle de $F$ indice $n-1$.

On applique la formule de récurrence :
$F_{n} = F_{n-1} + F_{n-2}$

Ensuite, on décale tous les termes d'un rang.

On répète (à l'aide de la boucle "pour") cette opération jusqu'à obtenir le terme souhaité.

\change
Une autre petite boucle permet ici d'afficher les 10 premiers termes [montrer le haut].


%%%%%%%%%%%%%%%%%%%%%%%%%%%%%%%%%%%%%%%%%%%%%%%%%%%%%%%%%%%
\diapo

Nous souhaitons maintenant étudier la forme que pourrait avoir une formule 
de calcul direct des nombres de Fibonacci, c'est-à-dire obtenir $F_n$ en fonction de $n$.

On pose $G_n = \phi^n$ 

et on suppose que $G_n$ vérifie la relation de récurrence
$G_{n} = G_{n-1}+G_{n-2}$.

\change
C'est-à-dire que l'on doit résoudre $\phi^n = \phi^{n-1} + \phi^{n-2}$.

\change
Autrement dit $\phi$ est solution de l'équation
 $X^n = X^{n-1} + X^{n-2}$,
 
\change
On factorise cette équation,

\change
On écarte la solution triviale $X=0$, pour obtenir l'équation 
  $X^2-X-1 = 0$.  
  
\change
On profite de \Sage\ pour calculer les deux racines de 
cette dernière équation avec la commande "solve".

Noter qu'il s'agit bien de résoudre une égalité,
donc il y a deux symboles "égal".

\change
\Sage\ nous renvoie les deux solutions.



%%%%%%%%%%%%%%%%%%%%%%%%%%%%%%%%%%%%%%%%%%%%%%%%%%%%%%%%%%%
\diapo

Voyons comment automatiser la gestion des solutions.

\change
Donnons un nom à la liste des solutions.

\change
La liste contient deux solutions, par exemple la seconde est
\codeinline{x == 1/2*sqrt(5) + 1/2}.

\change
Pour en extraire juste la partie numérique, on utilise la méthode
\codeinline{rhs()} qui renvoie la partie droite d'une équation.

\change
Ce qui nous permet d'obtenir les deux solutions

$$\phi = \frac{1+\sqrt5}{2} \qquad \mbox{et} \qquad \psi  = \frac{1-\sqrt5}{2}.$$
  
  $\phi^n$ et $\psi^n$ vérifient la même relation de récurrence 
  que la suite de Fibonacci 
  mais bien sûr les valeurs prises 
  par $\phi^n$ et $\psi^n$ 
  ne sont pas entières.
  
\change
Voyons comment obtenir une formule du type 
$F_n = c  \big(\phi^n-\psi^n\big)$
  

\change
Il nous reste à déterminer $c$. 

Comme $F_1=1$ alors 
$c(\phi-\psi)=1$, ce qui donne $c = \frac{1}{\sqrt 5}$.


\change
La conséquence de notre analyse est que \emph{si} une formule 
  $F_n = c(\phi^n-\psi^n)$ existe alors les constantes sont
  nécessairement $c= \frac{1}{\sqrt 5}$, $\phi = \frac{1+\sqrt5}{2}$, 
  $\psi  = \frac{1-\sqrt5}{2}$.
  
Notez bien que n'avons pour l'instant que la partie analyse du problème.
Il faut maintenant faire la synthèse, c'est-à-dire  
prouver que cette forme convient.



%%%%%%%%%%%%%%%%%%%%%%%%%%%%%%%%%%%%%%%%%%%%%%%%%%%%%%%%%%%
\diapo
  Nous allons maintenant montrer que la suite $H_n$ définie avec 
  les constantes calculées précédemment est effectivement la suite de Fibonacci.
%  on a effectivement  $F_n$ qui est définit par cette formule.
  
\change
On définit pour cela une nouvelle fonction \codeinline{conjec} 
qui correspond à $H_n$, le terme de droite de la formule que l'on souhaite montrer.

\change
On vérifie avec \Sage\ que l'on a l'égalité souhaitée $F_n=H_n$ 
pour les premières valeurs de $n$ par exemple jusqu'à 9.
    
 

%%%%%%%%%%%%%%%%%%%%%%%%%%%%%%%%%%%%%%%%%%%%%%%%%%%%%%%%%%%
\diapo   
 
 
Voici la partie la plus importante : nous allons prouver que
$F_n=H_n$ quelque soit $n$.

L'idée de la preuve est la suivante, nous allons montrer que $H_n$ 
a les mêmes termes initiaux et vérifie la même relation de récurrence
que la suite de Fibonacci.

\change

Nous avons déjà vérifié que $H_n$ et $F_n$ avaient les mêmes premiers termes,

$F_0=H_0=0$ et $F_1=H_1=1$.

\change 
Vérifions à présent que $H_n$ satisfait la même relation de récurrence que $H_n$.


\change

Rappelons que l'on a défini $H_n$ par cette fonction.

\change
Nous allons montrer que $H_n$ vérifie exactement 
la même relation de récurrence que les nombres de Fibonacci.

Encore une fois, on peut soit le faire à la main, 
soit le demander à l'ordinateur de le faire pour nous.

\change
 
Le résultat, après simplification (\codeinline{simplify\_radical}), est $0$. 

\change

Comme le calcul est valable pour la variable formelle $n$, 
%en effet nous avons pris soin de formaliser la variable avec \codeinline{var('n')},

\change

alors il est vrai quel que soit $n\ge0$. 

Ainsi $H_n=H_{n-1}+H_{n-2}$.   


\change

Tirons le bilan de nos travaux :

Les suites $(F_n)$ et $(H_n)$ ont les mêmes termes initiaux et 
vérifient la même relation de récurrence. 
Elles ont donc les mêmes termes pour chaque rang c'est-à-dire : 
      pour tout $n\ge0$, $F_n=H_n$.

      
  La commande \codeinline{var('n')} est essentielle. Elle permet de réinitialiser 
  la variable $n$ en une variable formelle. 
  Après l'instruction \codeinline{var('n')} la variable $n$ redevient 
  une indéterminée \codeinline{'n'} sans valeur spécifique.

%%%%%%%%%%%%%%%%%%%%%%%%%%%%%%%%%%%%%%%%%%%%%%%%%%%%%%%%%%%
\diapo

A votre tour maintenant d'expérimenter, conjecturer et prouver 
\emph{l'identité de Cassini}.

Il s'agit de trouver une expression simple pour la quantité
$$F_{n+1}F_{n-1}-F_n^2.$$



%%%%%%%%%%%%%%%%%%%%%%%%%%%%%%%%%%%%%%%%%%%%%%%%%%%%%%%%%%%
\diapo

Vérifions ce que vous avez obtenu.


Notons $$C_n = F_{n+1}F_{n-1}-F_n^2.$$

\change

Après quelques tests, on conjecture que les valeurs sont plus ou moins un et plus précisément $(-1)^n$.
 

\change

Voyons une preuve de ce résultat à l'aide du calcul formel.

On rappelle la formule établie précédemment pour les nombres de Fibonacci :

\change

que l'on implémente en une fonction de Sage.

\change

Pour une variable formelle $n$, c'est-à-dire pour $n$ quelconque,

\change

définissons une quantité qui correspond à $C_n$.

\change

Ensuite demandons à la machine de simplifier l'expression $C_n$.

\change
L'ordinateur effectue (presque) tous les calculs pour nous.
Le seul problème est que \Sage\ échoue de peu à simplifier 
l'expression contenant des racines carrées. 

En effet, une fois simplifiée, on obtient :
$$C_n = (-1) ^n \frac{(\sqrt 5 -1)^n (\sqrt 5 + 1)^n}{2^{2n}}.$$

\change
Il ne reste qu'à conclure à la main. 
En utilisant l'identité remarquable $(a-b)(a+b)=a^2-b^2$,
 $(\sqrt 5 -1)(\sqrt 5 + 1) = (\sqrt 5)^2 - 1 = 4$
et donc $C_n = (-1)^n \frac{4^n}{2^{2n}} = (-1)^n$.



%%%%%%%%%%%%%%%%%%%%%%%%%%%%%%%%%%%%%%%%%%%%%%%%%%%%%%%%%%%
\diapo

Je vous laisse terminer cette séquence avec l'étude d'une autre 
suite récurrente double. 

Reprenez le schéma d'étude adopté pour la suite
de Fibonacci et trouvez une expression simple de cette suite.


\end{document}
