
%%%%%%%%%%%%%%%%%% PREAMBULE %%%%%%%%%%%%%%%%%%


\documentclass[12pt]{article}

\usepackage{amsfonts,amsmath,amssymb,amsthm}
\usepackage[utf8]{inputenc}
\usepackage[T1]{fontenc}
\usepackage[francais]{babel}


% packages
\usepackage{amsfonts,amsmath,amssymb,amsthm}
\usepackage[utf8]{inputenc}
\usepackage[T1]{fontenc}
%\usepackage{lmodern}

\usepackage[francais]{babel}
\usepackage{fancybox}
\usepackage{graphicx}

\usepackage{float}

%\usepackage[usenames, x11names]{xcolor}
\usepackage{tikz}
\usepackage{datetime}

\usepackage{mathptmx}
%\usepackage{fouriernc}
%\usepackage{newcent}
\usepackage[mathcal,mathbf]{euler}

%\usepackage{palatino}
%\usepackage{newcent}


% Commande spéciale prompteur

%\usepackage{mathptmx}
%\usepackage[mathcal,mathbf]{euler}
%\usepackage{mathpple,multido}

\usepackage[a4paper]{geometry}
\geometry{top=2cm, bottom=2cm, left=1cm, right=1cm, marginparsep=1cm}

\newcommand{\change}{{\color{red}\rule{\textwidth}{1mm}\\}}

\newcounter{mydiapo}

\newcommand{\diapo}{\newpage
\hfill {\normalsize  Diapo \themydiapo \quad \texttt{[\jobname]}} \\
\stepcounter{mydiapo}}


%%%%%%% COULEURS %%%%%%%%%%

% Pour blanc sur noir :
%\pagecolor[rgb]{0.5,0.5,0.5}
% \pagecolor[rgb]{0,0,0}
% \color[rgb]{1,1,1}



%\DeclareFixedFont{\myfont}{U}{cmss}{bx}{n}{18pt}
\newcommand{\debuttexte}{
%%%%%%%%%%%%% FONTES %%%%%%%%%%%%%
\renewcommand{\baselinestretch}{1.5}
\usefont{U}{cmss}{bx}{n}
\bfseries

% Taille normale : commenter le reste !
%Taille Arnaud
%\fontsize{19}{19}\selectfont

% Taille Barbara
%\fontsize{21}{22}\selectfont

%Taille François
\fontsize{25}{30}\selectfont

%Taille Pascal
%\fontsize{25}{30}\selectfont

%Taille Laura
%\fontsize{30}{35}\selectfont


%\myfont
%\usefont{U}{cmss}{bx}{n}

%\Huge
%\addtolength{\parskip}{\baselineskip}
}


% \usepackage{hyperref}
% \hypersetup{colorlinks=true, linkcolor=blue, urlcolor=blue,
% pdftitle={Exo7 - Exercices de mathématiques}, pdfauthor={Exo7}}


%section
% \usepackage{sectsty}
% \allsectionsfont{\bf}
%\sectionfont{\color{Tomato3}\upshape\selectfont}
%\subsectionfont{\color{Tomato4}\upshape\selectfont}

%----- Ensembles : entiers, reels, complexes -----
\newcommand{\Nn}{\mathbb{N}} \newcommand{\N}{\mathbb{N}}
\newcommand{\Zz}{\mathbb{Z}} \newcommand{\Z}{\mathbb{Z}}
\newcommand{\Qq}{\mathbb{Q}} \newcommand{\Q}{\mathbb{Q}}
\newcommand{\Rr}{\mathbb{R}} \newcommand{\R}{\mathbb{R}}
\newcommand{\Cc}{\mathbb{C}} 
\newcommand{\Kk}{\mathbb{K}} \newcommand{\K}{\mathbb{K}}

%----- Modifications de symboles -----
\renewcommand{\epsilon}{\varepsilon}
\renewcommand{\Re}{\mathop{\text{Re}}\nolimits}
\renewcommand{\Im}{\mathop{\text{Im}}\nolimits}
%\newcommand{\llbracket}{\left[\kern-0.15em\left[}
%\newcommand{\rrbracket}{\right]\kern-0.15em\right]}

\renewcommand{\ge}{\geqslant}
\renewcommand{\geq}{\geqslant}
\renewcommand{\le}{\leqslant}
\renewcommand{\leq}{\leqslant}

%----- Fonctions usuelles -----
\newcommand{\ch}{\mathop{\mathrm{ch}}\nolimits}
\newcommand{\sh}{\mathop{\mathrm{sh}}\nolimits}
\renewcommand{\tanh}{\mathop{\mathrm{th}}\nolimits}
\newcommand{\cotan}{\mathop{\mathrm{cotan}}\nolimits}
\newcommand{\Arcsin}{\mathop{\mathrm{Arcsin}}\nolimits}
\newcommand{\Arccos}{\mathop{\mathrm{Arccos}}\nolimits}
\newcommand{\Arctan}{\mathop{\mathrm{Arctan}}\nolimits}
\newcommand{\Argsh}{\mathop{\mathrm{Argsh}}\nolimits}
\newcommand{\Argch}{\mathop{\mathrm{Argch}}\nolimits}
\newcommand{\Argth}{\mathop{\mathrm{Argth}}\nolimits}
\newcommand{\pgcd}{\mathop{\mathrm{pgcd}}\nolimits} 

\newcommand{\Card}{\mathop{\text{Card}}\nolimits}
\newcommand{\Ker}{\mathop{\text{Ker}}\nolimits}
\newcommand{\id}{\mathop{\text{id}}\nolimits}
\newcommand{\ii}{\mathrm{i}}
\newcommand{\dd}{\mathrm{d}}
\newcommand{\Vect}{\mathop{\text{Vect}}\nolimits}
\newcommand{\Mat}{\mathop{\mathrm{Mat}}\nolimits}
\newcommand{\rg}{\mathop{\text{rg}}\nolimits}
\newcommand{\tr}{\mathop{\text{tr}}\nolimits}
\newcommand{\ppcm}{\mathop{\text{ppcm}}\nolimits}

%----- Structure des exercices ------

\newtheoremstyle{styleexo}% name
{2ex}% Space above
{3ex}% Space below
{}% Body font
{}% Indent amount 1
{\bfseries} % Theorem head font
{}% Punctuation after theorem head
{\newline}% Space after theorem head 2
{}% Theorem head spec (can be left empty, meaning ‘normal’)

%\theoremstyle{styleexo}
\newtheorem{exo}{Exercice}
\newtheorem{ind}{Indications}
\newtheorem{cor}{Correction}


\newcommand{\exercice}[1]{} \newcommand{\finexercice}{}
%\newcommand{\exercice}[1]{{\tiny\texttt{#1}}\vspace{-2ex}} % pour afficher le numero absolu, l'auteur...
\newcommand{\enonce}{\begin{exo}} \newcommand{\finenonce}{\end{exo}}
\newcommand{\indication}{\begin{ind}} \newcommand{\finindication}{\end{ind}}
\newcommand{\correction}{\begin{cor}} \newcommand{\fincorrection}{\end{cor}}

\newcommand{\noindication}{\stepcounter{ind}}
\newcommand{\nocorrection}{\stepcounter{cor}}

\newcommand{\fiche}[1]{} \newcommand{\finfiche}{}
\newcommand{\titre}[1]{\centerline{\large \bf #1}}
\newcommand{\addcommand}[1]{}
\newcommand{\video}[1]{}

% Marge
\newcommand{\mymargin}[1]{\marginpar{{\small #1}}}



%----- Presentation ------
\setlength{\parindent}{0cm}

%\newcommand{\ExoSept}{\href{http://exo7.emath.fr}{\textbf{\textsf{Exo7}}}}

\definecolor{myred}{rgb}{0.93,0.26,0}
\definecolor{myorange}{rgb}{0.97,0.58,0}
\definecolor{myyellow}{rgb}{1,0.86,0}

\newcommand{\LogoExoSept}[1]{  % input : echelle
{\usefont{U}{cmss}{bx}{n}
\begin{tikzpicture}[scale=0.1*#1,transform shape]
  \fill[color=myorange] (0,0)--(4,0)--(4,-4)--(0,-4)--cycle;
  \fill[color=myred] (0,0)--(0,3)--(-3,3)--(-3,0)--cycle;
  \fill[color=myyellow] (4,0)--(7,4)--(3,7)--(0,3)--cycle;
  \node[scale=5] at (3.5,3.5) {Exo7};
\end{tikzpicture}}
}



\theoremstyle{definition}
%\newtheorem{proposition}{Proposition}
%\newtheorem{exemple}{Exemple}
%\newtheorem{theoreme}{Théorème}
\newtheorem{lemme}{Lemme}
\newtheorem{corollaire}{Corollaire}
%\newtheorem*{remarque*}{Remarque}
%\newtheorem*{miniexercice}{Mini-exercices}
%\newtheorem{definition}{Définition}




%definition d'un terme
\newcommand{\defi}[1]{{\color{myorange}\textbf{\emph{#1}}}}
\newcommand{\evidence}[1]{{\color{blue}\textbf{\emph{#1}}}}



 %----- Commandes divers ------

\newcommand{\codeinline}[1]{\texttt{#1}}

\newcommand{\Sage}{\texttt{Sage}}
%%%%%%%%%%%%%%%%%%%%%%%%%%%%%%%%%%%%%%%%%%%%%%%%%%%%%%%%%%%%%
%%%%%%%%%%%%%%%%%%%%%%%%%%%%%%%%%%%%%%%%%%%%%%%%%%%%%%%%%%%%%


\begin{document}

\debuttexte


%%%%%%%%%%%%%%%%%%%%%%%%%%%%%%%%%%%%%%%%%%%%%%%%%%%%%%%%%%%
\diapo

Bonjour. 

Dans cette partie nous allons voir les principales façons de structurer
les instructions. 

\change

Si vous connaissez déjà un langage informatique et en 
particulier le langage Python ce sera très facile !

\change
Sinon nous verrons d'abord différents types de boucles,

\change
comment manipuler les expressions logiques et mettre en place des tests,

\change
comment regrouper du code en une fonction informatique.

\change
Il faudra bien comprendre la différence entre variable locale
et variable globale.

\change
Vous terminerez par un petit tp sur la conjecture de Syracuse.


%%%%%%%%%%%%%%%%%%%%%%%%%%%%%%%%%%%%%%%%%%%%%%%%%%%%%%%%%%%
\diapo


Pour faire varier un élément dans un ensemble on utilise
une boucle "pour".

L'instruction est "\codeinline{for x in ensemble:}", elle est terminée pas deux points. 

Le bloc d'instructions qui suit sera exécuté successivement 
pour toutes les valeurs de $x$ prises une à une dans ensemble.

Notez encore une fois que le bloc d'instructions exécuté 
est délimité par les espaces en début de ligne.


%%%%%%%%%%%%%%%%%%%%%%%%%%%%%%%%%%%%%%%%%%%%%%%%%%%%%%%%%%%
\diapo

\change

Un exemple fréquent de boucle "pour" 
est "\codeinline{for k in range(n):}" qui fait varier $k$ de $0$ à $n-1$.

\change
En fait \codeinline{range(n)} renvoie 
la liste des $n$ premiers entiers, c'est-à-dire \codeinline{[0,1,2,...,n-1]}.

Attention cela s’arrête bien à $n-1$ !

\change
Plus généralement \codeinline{range(a,b)} renvoie la liste des entiers de 
$a$ à $b-1$.

\change
Alors que \codeinline{range(a,b,c)} renvoie la même liste mas en effectuant des sauts de longueur $c$.

\change
Par exemple \codeinline{range(0,101,5)} renvoie 

la liste \codeinline{[0,5,10,15,...,100]}.

\change
Une autre façon légèrement différente de définir des listes d'entiers 
est l'instruction \codeinline{[a..b]} qui renvoie la liste des entiers $k$ tels que $a\le k \le b$.

Notez qu'ici la valeur $b$ figure bien dans la liste.

\change
La boucle "pour" permet plus généralement de parcourir n'importe quelle liste,
par exemple cette instruction fait prendre à $x$ les $4$ valeurs
de la liste.


%%%%%%%%%%%%%%%%%%%%%%%%%%%%%%%%%%%%%%%%%%%%%%%%%%%%%%%%%%%
\diapo

\change

La boucle "tant que" (\codeinline{while}) exécute le bloc d'instructions situé immédiatement après
tant que la condition est vérifiée. 

Lorsque la condition n'est plus vérifiée, la boucle s'interrompt et le programme passe
aux instructions suivantes.

Attention, vérifiez toujours avant le lancer le programme que la condition passe de vrai à faux au cours des calculs, 
cela afin que la boucle ne se répète pas indéfiniment. 

%%%%%%%%%%%%%%%%%%%%%%%%%%%%%%%%%%%%%%%%%%%%%%%%%%%%%%%%%%%
\diapo

Voyons un exemple simple d'utilisation d'une boucle "tant que".

Comment calculer la racine carrée entière d'un entier $n$ ?

\change

On part d'un entier $n$, on cherche la partie entière de sa racine carrée
en effectuant uniquement des calculs avec les entiers.

\change 

On part de $k=1$.

\change

Voici l'entête de la boucle, la condition est ici $k^2 \le n$,
qui est bien sûr vraie pour $k=1$ ce qui va provoquer la première exécution du bloc qui suit.

\change

Tant que la condition est vérifiée,
on incrémente $k$ de $1$.

Une fois $k$ augmenté on reprend la boucle : la condition est-elle encore 
vraie ? Si oui, on incrémente $k$. Etc.

Bien sûr à un certain moment $k^2$ sera strictement supérieur à $n$ et 
la condition deviendra fausse. La boucle sera alors terminée !

\change
On passera aux instructions suivantes : ici l'affichage de $k-1$.

\change
En effet, la racine carrée entière de $n$ est $k-1$ car $k-1$ est bien le dernier entier qui vérifiait la condition.

Notez qu'utiliser une boucle "tant que" comporte des risques, 
en effet il faut toujours s'assurer que la boucle se termine.


%%%%%%%%%%%%%%%%%%%%%%%%%%%%%%%%%%%%%%%%%%%%%%%%%%%%%%%%%%%
\diapo


\change
Passons aux tests.

Si la condition est vérifiée, c'est le premier bloc d'instructions 
qui est exécuté,

si la condition n'est pas vérifiée, alors c'est seulement le second bloc. 

On passe ensuite aux instructions suivantes.

%%%%%%%%%%%%%%%%%%%%%%%%%%%%%%%%%%%%%%%%%%%%%%%%%%%%%%%%%%%
\diapo

Nous venons d'étudier des strucutres de programmation utilisant des expressions logiques ou booléens.

Formalisons un peu cette notion.

\change
Une \defi{expression booléenne} est une expression
qui peut seulement prendre les valeurs  \og Vrai  \fg\
ou \og Faux  \fg.

Avec \Sage\ un booléen vaut donc soit \codeinline{True}, soit 
\codeinline{False}.

\change
Les expressions booléennes s'obtiennent 
principalement comme suit :


\change
\codeinline{a < b} : teste l'inégalité stricte $a<b$,


\change
\codeinline{a > b} : teste l'inégalité stricte $a>b$,

\change
\change  
voici les tests des inégalités larges

\change
Pour savoir si deux quantité sont égales on écrit
\codeinline{a == b}.
Ce test renvoie Vrai si $a$ et $b$ sont égaux, et Faux sinon.

Notez bien le double symbole égal !
 
\change
On peut à contrario tester la non égalité,

comme ceci ou comme cela.

\change
On peut également tester l'appartenance d'un élément à une structure, une liste par exemple, avec le mot "in" déjà évoqué pour la boucle "pour".

%%%%%%%%%%%%%%%%%%%%%%%%%%%%%%%%%%%%%%%%%%%%%%%%%%%%%%%%%%%
\diapo

\change

On vient de voir qu'une condition prend la valeur \codeinline{True} 
  si elle est vraie et \codeinline{False} sinon.

\change  
Par exemple : \codeinline{x == 2} renvoie Vraie 
si $x$ vaut $2$ et Faux sinon.

La valeur de $x$ n'est pas modifiée pas le test.

\change
Il ne faut surtout pas confondre le test d'égalité
\codeinline{x == 2} avec l'affectation \codeinline{x = 2} 
car après cette dernière instruction $x$ vaut $2$.
  
  
\change
On peut combiner des conditions avec les opérateurs logiques classiques
\codeinline{et}, \codeinline{ou}, \codeinline{non}. 

\change
Voici un exemple :
l'expression   \codeinline{ (n>0) and (not (is-prime(n)) } est vraie
si et seulement si $n$ est strictement positif et n'est pas un nombre 
premier.
  



%%%%%%%%%%%%%%%%%%%%%%%%%%%%%%%%%%%%%%%%%%%%%%%%%%%%%%%%%%%
\diapo


Voici votre premier travail de cette séquence :

tout d'abord pour deux assertions logiques, c'est-à-dire des booléens, 
$P$, $Q$, écrire avec \Sage\ l'assertion $P\implies Q$.
 
 
Seconde question. 
Une assertion est appelée une \defi{tautologie} si elle est
toujours vraie quelques soient les valeurs prises par les paramètres.

Un exemple simple est l'assertion 
$(P \text{ ou } (\text{non\;} P))$ 

si $P$ est vraie l'assertion est vraie;

si $P$ est fausse, alors non $P$ est vraie donc l'assertion
$(P \text{ ou } (\text{non\;} P))$ est encore vraie.

Ainsi cette assertion est toujours vraie.

Maintenant à vous de montrer que cette assertion un peu compliquée est toujours
vraie et ce quels que les valeurs Vrai ou Faux des trois paramètres $P,Q$ et $R$.

\change

Voyons comment faire.

Pour la première question, il faut se souvenir du cours de 
logique qui définit l'assertion
\og$P\implies Q$ \fg\ comme étant \og$\text{non\;}(P) \text{ ou } Q$ \fg.
Il suffit donc d'écrire : \codeinline{not(P) or Q}

\change
Pour savoir si la seconde assertion est toujours vraie, 
on teste toutes les combinaisons possibles $P,Q$ et $R$ (il y en $8$).

Pour chaque triplet possible, on demande à l'ordinateur d'afficher le statut de cette expression booléenne.

Dans notre cas, c'est à chaque fois Vrai.

Nous sommes donc en présence d'une tautologie.

%%%%%%%%%%%%%%%%%%%%%%%%%%%%%%%%%%%%%%%%%%%%%%%%%%%%%%%%%%%
\diapo


\change

Une fonction informatique ressemble à une fonction mathématique.

Elle prend en entrée des paramètres
et renvoie un résultat.


Voici comment définir une fonction.

On utilise d'abord le mot-clé "def", 
suivi du nom donné à la fonction. On précise enfin quelles sont la ou les variables passées en paramètre.

A chaque appel de la fonction on exécute toutes les instructions, 
qui sont bien sûr délimitées, nous y sommes maintenant habitués, par l'indentation.


La fonction renvoie une ou plusieurs valeurs que l'on précise par le mot-clé 
"return".

%%%%%%%%%%%%%%%%%%%%%%%%%%%%%%%%%%%%%%%%%%%%%%%%%%%%%%%%%%%
\diapo

Voici un petit exemple de fonction :
nous définissons notre propre fonction valeur absolue nommée "valeur\_absolue".

En entrée : une seule variable notée $x$.

La fonction est constituée d'un test : 

Si $x$ est positif alors on renvoie $x$.

Mais si $x$ est négatif alors 
on renvoie $-x$ qui lui est positif.

\change

Notre fonction s'utilise maintenant comme n'importe quelle autre fonction.

Par exemple \codeinline{valeur\_absolue(-3)} renvoie $3$.

\change

Il est maintenant possible d'utiliser cette fonction 
dans le reste du programme ou dans une autre fonction.

Il est tuojours important de produire du code intelligible et de nommer les objets au plus près de leur définition, avec des noms évocateurs.

C'est maintenant à vous de réfléchir !

Que renvoie la fonction suivante qui prend en entrée deux paramètres ? 

Et quel nom aurait-il été plus approprié de lui donner ?

%Noter aussi que, comme ici, une fonction peut prendre plusieurs paramètres.

%%%%%%%%%%%%%%%%%%%%%%%%%%%%%%%%%%%%%%%%%%%%%%%%%%%%%%%%%%%
\diapo

Avec l'introduction des fonctions, il faut dès maintenant être très prudent et faire la distinction entre des variables locales 
et des variables globales.

Une variable locale, c'est une variable qui n'existe qu'à 
l'intérieur d'une fonction. 
Alors qu'une variable globale existe et est accessible dans tout le corps de votre programme.

Voici une petit énigme : qu'affiche ces quelques lignes de code.
Proposez une réponse avant de vérifier si l'ordinateur est d'accord avec vous !

Il faut bien comprendre que tous les \codeinline{x} de la fonction
%\codeinline{def incremente(x):},    \codeinline{x = x+1},     \codeinline{return x}
\codeinline{incremente}
représentent une variable locale qui n'existe que dans 
cette fonction et disparaît 
en dehors. On aurait tout aussi bien pu remplacer les $x$ de ces trois lignes
par une variable y ou truc ou zut !


Par contre les \codeinline{x} des lignes
\codeinline{x=2}, \codeinline{incremente(x)}, \codeinline{print(x)}
correspondent à une variable globale.

Une variable globale existe partout 
(sauf dans les fonctions où une variable
locale porte le même nom !).

\change 

En mathématique, on parle plutôt de variable muette au lieu 
de variable locale, par exemple dans 
$$\sum_{k=0}^n k^2$$
$k$ est une variable muette, on aurait pu tout aussi bien la nommer $i$ ou $x$.
Par contre il faut au préalable avoir défini $n$ 
(comme une variable globale), par exemple \og $n=7$  \fg\ ou \og 
fixons un entier $n$  \fg...


%%%%%%%%%%%%%%%%%%%%%%%%%%%%%%%%%%%%%%%%%%%%%%%%%%%%%%%%%%%
\diapo

Mettez immédiatement en pratique les structures de contrôle étudiées précédemment 
avec le travail suivant.

La \defi{suite de Syracuse}  est  définie par récurrence
par un terme initial qui est un entier $u_0$

et par une relation de récurrence :
$$u_{n+1} = 
\begin{cases}
  \frac{u_n}{2} & \text{ si $u_n$ est pair} \\
  3u_n + 1 & \text{ si $u_n$ est impair} \\
  \end{cases}$$
  
  Vous devez écrire une fonction qui renvoie le terme $u_n$ 
  en fonction de $u_0$ et du rang $n$.
  
  Ensuite vérifiez expérimentalement
  que la suite atteint toujours la valeur $1$, 
  
  Modifiez ensuite votre fonction pour maintenant 
  renvoyer le premier rang $n$ pour lequel $u_n = 1$.
  

Cette suite est très célèbre. La simplicité de sa définition permet à tous de caluler ses premiers termes.
Mais, à ce jour, personne ne sait prouver que, pour toutes les valeurs 
du terme initial $u_0$, la suite de Syracuse atteint toujours la valeur $1$.


%%%%%%%%%%%%%%%%%%%%%%%%%%%%%%%%%%%%%%%%%%%%%%%%%%%%%%%%%%%
\diapo

Voici le code pour calculer le $n$-ème terme de la suite.

La fonction a deux paramètres : le terme initial $u_0$ et le rang $n$
du terme que l'on souhaite calculer.


On part du terme initial. 
$u$ joue le rôle du terme général $u_k$.
On l'initialise à $u_0$.


Le boucle "pour" (\codeinline{for}) est répétée $n$ fois.

On teste si le terme actel est pair (nous expliquerons plus tard cette instruction).

S'il est pair alors le nouveau terme est la moitié de l'actuel.
%$u_{k+1} = u_k/2$

S'il est impair, alors il vaut trois fois l'actuel plus un.%$u_{k+1} = 3u_k+1$.

On calcule ainsi $u_0, u_1, u_2,...$

et à la fin on a calculé le terme $u_n$ qui est renvoyé par la fonction. 

Vous pouvez maintenant faire vos expérimentations.

%%%%%%%%%%%%%%%%%%%%%%%%%%%%%%%%%%%%%%%%%%%%%%%%%%%%%%%%%%%
\diapo


Revenons sur quelques points de la fonction précédente.

Pour deux entiers $a$ et $n$ alors

\codeinline{a\%n} est 
le \defi{reste} de la division euclidienne de $a$ par $n$.

\change

Autrement dit $r \equiv a \pmod{n}$ et $0 \le r < n$.


\change

Le quotient $q$ de la division euclidienne de $a$ par $n$, s'obtient par
 \codeinline{a//n}.

\change 
 La division euclidienne de $a$ par $n$ s'écrit maintenant
 $a=nq+r$.

\change
Attention !
Il ne faut pas confondre la division de nombres réels : \codeinline{a/b}
et le quotient de la division euclidienne de deux entiers : \codeinline{a//b}.

\change
Par exemple \codeinline{7/2} est la fraction $\frac72$ dont la valeur numérique est $3,5$
alors que \codeinline{7//2} renvoie $3$. 


\change
Enfin, \codeinline{u\%2 == 0} teste la parité de $u$.

En effet, \codeinline{u\%2} renvoie le reste de la division euclidienne 
de $u$ par $2$. Donc $\codeinline{u\%2}$ vaut $0$ si $u$ est pair 
et $1$ si $u$ est impair.


%%%%%%%%%%%%%%%%%%%%%%%%%%%%%%%%%%%%%%%%%%%%%%%%%%%%%%%%%%%
\diapo

Voici comment calculer à partir de quel rang on obtient la valeur $1$.

On a modifié le code précédent en transformant la boucle "pour" en une boucle "tant que"

Ces instructions sont exécutées tant que $u \neq 1$.

Donc par construction et puisque tous les termes de la suite sont strictement positifs, quand la boucle se termine c'est que $u=1$ !


\end{document}
