
%%%%%%%%%%%%%%%%%% PREAMBULE %%%%%%%%%%%%%%%%%%


\documentclass[12pt]{article}

\usepackage{amsfonts,amsmath,amssymb,amsthm}
\usepackage[utf8]{inputenc}
\usepackage[T1]{fontenc}
\usepackage[francais]{babel}


% packages
\usepackage{amsfonts,amsmath,amssymb,amsthm}
\usepackage[utf8]{inputenc}
\usepackage[T1]{fontenc}
%\usepackage{lmodern}

\usepackage[francais]{babel}
\usepackage{fancybox}
\usepackage{graphicx}

\usepackage{float}

%\usepackage[usenames, x11names]{xcolor}
\usepackage{tikz}
\usepackage{datetime}

\usepackage{mathptmx}
%\usepackage{fouriernc}
%\usepackage{newcent}
\usepackage[mathcal,mathbf]{euler}

%\usepackage{palatino}
%\usepackage{newcent}


% Commande spéciale prompteur

%\usepackage{mathptmx}
%\usepackage[mathcal,mathbf]{euler}
%\usepackage{mathpple,multido}

\usepackage[a4paper]{geometry}
\geometry{top=2cm, bottom=2cm, left=1cm, right=1cm, marginparsep=1cm}

\newcommand{\change}{{\color{red}\rule{\textwidth}{1mm}\\}}

\newcounter{mydiapo}

\newcommand{\diapo}{\newpage
\hfill {\normalsize  Diapo \themydiapo \quad \texttt{[\jobname]}} \\
\stepcounter{mydiapo}}


%%%%%%% COULEURS %%%%%%%%%%

% Pour blanc sur noir :
%\pagecolor[rgb]{0.5,0.5,0.5}
% \pagecolor[rgb]{0,0,0}
% \color[rgb]{1,1,1}



%\DeclareFixedFont{\myfont}{U}{cmss}{bx}{n}{18pt}
\newcommand{\debuttexte}{
%%%%%%%%%%%%% FONTES %%%%%%%%%%%%%
\renewcommand{\baselinestretch}{1.5}
\usefont{U}{cmss}{bx}{n}
\bfseries

% Taille normale : commenter le reste !
%Taille Arnaud
%\fontsize{19}{19}\selectfont

% Taille Barbara
%\fontsize{21}{22}\selectfont

%Taille François
\fontsize{25}{30}\selectfont

%Taille Pascal
%\fontsize{25}{30}\selectfont

%Taille Laura
%\fontsize{30}{35}\selectfont


%\myfont
%\usefont{U}{cmss}{bx}{n}

%\Huge
%\addtolength{\parskip}{\baselineskip}
}


% \usepackage{hyperref}
% \hypersetup{colorlinks=true, linkcolor=blue, urlcolor=blue,
% pdftitle={Exo7 - Exercices de mathématiques}, pdfauthor={Exo7}}


%section
% \usepackage{sectsty}
% \allsectionsfont{\bf}
%\sectionfont{\color{Tomato3}\upshape\selectfont}
%\subsectionfont{\color{Tomato4}\upshape\selectfont}

%----- Ensembles : entiers, reels, complexes -----
\newcommand{\Nn}{\mathbb{N}} \newcommand{\N}{\mathbb{N}}
\newcommand{\Zz}{\mathbb{Z}} \newcommand{\Z}{\mathbb{Z}}
\newcommand{\Qq}{\mathbb{Q}} \newcommand{\Q}{\mathbb{Q}}
\newcommand{\Rr}{\mathbb{R}} \newcommand{\R}{\mathbb{R}}
\newcommand{\Cc}{\mathbb{C}} 
\newcommand{\Kk}{\mathbb{K}} \newcommand{\K}{\mathbb{K}}

%----- Modifications de symboles -----
\renewcommand{\epsilon}{\varepsilon}
\renewcommand{\Re}{\mathop{\text{Re}}\nolimits}
\renewcommand{\Im}{\mathop{\text{Im}}\nolimits}
%\newcommand{\llbracket}{\left[\kern-0.15em\left[}
%\newcommand{\rrbracket}{\right]\kern-0.15em\right]}

\renewcommand{\ge}{\geqslant}
\renewcommand{\geq}{\geqslant}
\renewcommand{\le}{\leqslant}
\renewcommand{\leq}{\leqslant}

%----- Fonctions usuelles -----
\newcommand{\ch}{\mathop{\mathrm{ch}}\nolimits}
\newcommand{\sh}{\mathop{\mathrm{sh}}\nolimits}
\renewcommand{\tanh}{\mathop{\mathrm{th}}\nolimits}
\newcommand{\cotan}{\mathop{\mathrm{cotan}}\nolimits}
\newcommand{\Arcsin}{\mathop{\mathrm{Arcsin}}\nolimits}
\newcommand{\Arccos}{\mathop{\mathrm{Arccos}}\nolimits}
\newcommand{\Arctan}{\mathop{\mathrm{Arctan}}\nolimits}
\newcommand{\Argsh}{\mathop{\mathrm{Argsh}}\nolimits}
\newcommand{\Argch}{\mathop{\mathrm{Argch}}\nolimits}
\newcommand{\Argth}{\mathop{\mathrm{Argth}}\nolimits}
\newcommand{\pgcd}{\mathop{\mathrm{pgcd}}\nolimits} 

\newcommand{\Card}{\mathop{\text{Card}}\nolimits}
\newcommand{\Ker}{\mathop{\text{Ker}}\nolimits}
\newcommand{\id}{\mathop{\text{id}}\nolimits}
\newcommand{\ii}{\mathrm{i}}
\newcommand{\dd}{\mathrm{d}}
\newcommand{\Vect}{\mathop{\text{Vect}}\nolimits}
\newcommand{\Mat}{\mathop{\mathrm{Mat}}\nolimits}
\newcommand{\rg}{\mathop{\text{rg}}\nolimits}
\newcommand{\tr}{\mathop{\text{tr}}\nolimits}
\newcommand{\ppcm}{\mathop{\text{ppcm}}\nolimits}

%----- Structure des exercices ------

\newtheoremstyle{styleexo}% name
{2ex}% Space above
{3ex}% Space below
{}% Body font
{}% Indent amount 1
{\bfseries} % Theorem head font
{}% Punctuation after theorem head
{\newline}% Space after theorem head 2
{}% Theorem head spec (can be left empty, meaning ‘normal’)

%\theoremstyle{styleexo}
\newtheorem{exo}{Exercice}
\newtheorem{ind}{Indications}
\newtheorem{cor}{Correction}


\newcommand{\exercice}[1]{} \newcommand{\finexercice}{}
%\newcommand{\exercice}[1]{{\tiny\texttt{#1}}\vspace{-2ex}} % pour afficher le numero absolu, l'auteur...
\newcommand{\enonce}{\begin{exo}} \newcommand{\finenonce}{\end{exo}}
\newcommand{\indication}{\begin{ind}} \newcommand{\finindication}{\end{ind}}
\newcommand{\correction}{\begin{cor}} \newcommand{\fincorrection}{\end{cor}}

\newcommand{\noindication}{\stepcounter{ind}}
\newcommand{\nocorrection}{\stepcounter{cor}}

\newcommand{\fiche}[1]{} \newcommand{\finfiche}{}
\newcommand{\titre}[1]{\centerline{\large \bf #1}}
\newcommand{\addcommand}[1]{}
\newcommand{\video}[1]{}

% Marge
\newcommand{\mymargin}[1]{\marginpar{{\small #1}}}



%----- Presentation ------
\setlength{\parindent}{0cm}

%\newcommand{\ExoSept}{\href{http://exo7.emath.fr}{\textbf{\textsf{Exo7}}}}

\definecolor{myred}{rgb}{0.93,0.26,0}
\definecolor{myorange}{rgb}{0.97,0.58,0}
\definecolor{myyellow}{rgb}{1,0.86,0}

\newcommand{\LogoExoSept}[1]{  % input : echelle
{\usefont{U}{cmss}{bx}{n}
\begin{tikzpicture}[scale=0.1*#1,transform shape]
  \fill[color=myorange] (0,0)--(4,0)--(4,-4)--(0,-4)--cycle;
  \fill[color=myred] (0,0)--(0,3)--(-3,3)--(-3,0)--cycle;
  \fill[color=myyellow] (4,0)--(7,4)--(3,7)--(0,3)--cycle;
  \node[scale=5] at (3.5,3.5) {Exo7};
\end{tikzpicture}}
}



\theoremstyle{definition}
%\newtheorem{proposition}{Proposition}
%\newtheorem{exemple}{Exemple}
%\newtheorem{theoreme}{Théorème}
\newtheorem{lemme}{Lemme}
\newtheorem{corollaire}{Corollaire}
%\newtheorem*{remarque*}{Remarque}
%\newtheorem*{miniexercice}{Mini-exercices}
%\newtheorem{definition}{Définition}




%definition d'un terme
\newcommand{\defi}[1]{{\color{myorange}\textbf{\emph{#1}}}}
\newcommand{\evidence}[1]{{\color{blue}\textbf{\emph{#1}}}}



 %----- Commandes divers ------

\newcommand{\codeinline}[1]{\texttt{#1}}

%%%%%%%%%%%%%%%%%%%%%%%%%%%%%%%%%%%%%%%%%%%%%%%%%%%%%%%%%%%%%
%%%%%%%%%%%%%%%%%%%%%%%%%%%%%%%%%%%%%%%%%%%%%%%%%%%%%%%%%%%%%



\begin{document}

\debuttexte


%%%%%%%%%%%%%%%%%%%%%%%%%%%%%%%%%%%%%%%%%%%%%%%%%%%%%%%%%%%
\diapo

\change
Nous allons voir qu'il existe un lien étroit entre 
les matrices et les applications linéaires.

\change
\`A une matrice on associe naturellement une application linéaire. 


\change
Nous verrons le lien entre les opérations sur les matrices et les opérations
sur les applications linéaires.

\change
Deux cas particuliers sont importants :
la matrice d'un endomorphisme

\change
et la matrice d'un isomorphisme.


%%%%%%%%%%%%%%%%%%%%%%%%%%%%%%%%%%%%%%%%%%%%%%%%%%%%%%%%%%%
\diapo

Soient $E$ et $F$ deux $\Kk$-espaces vectoriels de dimension finie.

D'ailleurs dans cette partie tous les espaces vectoriels sont de dimension finie.

\change
Soit $p$ la dimension de $E$ et on fixe $\mathcal{B}=(e_1, \dots ,e_p)$
une base de $E$.

\change
Soit $n$ la dimension de $F$ et on fixe 
$\mathcal{B}'=(f_1, \dots ,f_n)$ une base de $F$.

\change
Enfin soit $f : E \to F$ une application linéaire.


\change
Les propriétés des applications linéaires entre deux espaces
de dimension finie permettent d'affirmer plusieurs choses :


(1) l'application linéaire $f$ est déterminée de façon unique par l'image
d'une base de $E$, donc $f$ est déterminée seulement par les vecteurs 

$f(e_1), f(e_2), \ldots, f(e_p)$.

\change
(2) Pour $j$ compris entre $1$ et $p$, 
$f(e_j)$ est un vecteur de $F$ et donc s'écrit de manière unique comme combinaison
linéaire des vecteurs de la base $\mathcal{B}'$ de $F$.

\change
Il existe donc $n$ scalaires uniques $a_{1,j},a_{2,j}, \ldots , a_{n,j}$
tels que 
$f (e_j)=a_{1,j}f_1+a_{2,j}f_2+\dots +a_{n,j}f_n$

\change
On note $f(e_j)$ sous la forme d'un vecteur colonne, en rappelant que cette décomposition
dépend de la base $\mathcal{B}'$.

%%%%%%%%%%%%%%%%%%%%%%%%%%%%%%%%%%%%%%%%%%%%%%%%%%%%%%%%%%%
\diapo

On vient de voir qu'une application linéaire $f$ est entièrement déterminée par 
les $f(e_j)$ donc par 
les coefficients $(a_{i,j})$ où $i$ varie de $1$ à $n$, et $j$ de $1$ à $p$.

\change
Il est donc
naturel d'introduire la définition suivante : 

la matrice de l'application linéaire $f$ par
rapport à la base  ${\color{blue}\mathcal{B}}$ de l'espace de départ 
et la base $\mathcal{B}'$ de l'espace d'arrivée est la matrice des coefficients $(a_{i,j})$.

C'est donc une matrice ayant $n$ lignes et $p$ colonnes.

\change
Voici comment retrouver cette matrice

On va compléter la matrice colonne par colonne.

\change
La première colonne c'est l'écriture du vecteur $f(e_1)$,
l'image par $f$ du premier vecteur de la base $\mathcal{B}$

\change
que l'on décompose dans la base $\mathcal{B}'$ constituée des vecteurs $f_1,f_2,...$.

\change
La première colonne est donc bien formée des coefficients :
$a_{11}, a_{21},..., a_{n1}$.

On décompose ensuite $f(e_2)$ dans cette même base $f_1,f_2,...$.

\change
Pour le vecteur $f(e_j)$, 

\change
les coefficients sont :  
$a_{1j}, a_{2j}, ..., a_{nj}$.

On est bien en train d'écrire la matrice des coefficients $(a_{ij})$.

\change
Et voici la dernière colonne.

Notez notre code de couleur : bleu pour la base de départ et vert pour la base d'arrivée.

\change
Reprenons : la première colonne est constituée par les coordonnées du vecteur
$f({\color{blue}e_1})$ dans la base $\mathcal{B}'$,

la deuxième colonne est constituée par les coordonnées du vecteur
$f({\color{blue}e_2})$ dans la base $\mathcal{B}'$,...

\change
On résume cela en disant que la matrice de l'application linéaire $f$ par
rapport aux bases ${\color{blue}\mathcal{B}}$ et $\mathcal{B}'$ est la matrice dont les vecteurs colonnes 
sont l'image par $f$ des vecteurs de la base de départ $\mathcal{B}$, 
exprimée dans la base d'arrivée $\mathcal{B}'$.


Quelques remarques :
  - tout d'abord retenez la notation : matrice de $f$ par
rapport aux bases ${\color{blue}\mathcal{B}}$ et $\mathcal{B}'$
[Montrer $\Mat_{\mathcal{B},\mathcal{B}'}(f)$.]

  - La taille de cette matrice dépend
  uniquement de la dimension de l'espace de départ et de l'espace d'arrivée.
  
  -  Par contre, les coefficients de la matrice dépendent 
  du choix de la base $\mathcal{B}$ de $E$ et aussi de la base $\mathcal{B}'$ de $F$.

%%%%%%%%%%%%%%%%%%%%%%%%%%%%%%%%%%%%%%%%%%%%%%%%%%%%%%%%%%%
\diapo

Calculons un exemple de matrice.


Soit $f$ l'application linéaire de $\Rr^3$ dans $\Rr^2$ définie
par 
$f(x_1,x_2,x_3)=(x_1+x_2-x_3, x_1-2x_2+3x_3)$.

\change
Il est utile d'identifier vecteurs lignes et vecteurs colonnes ; 
ainsi on peut écrire en colonne  
$f \left(\begin{smallmatrix} x_1\\x_2\\x_3 \end{smallmatrix}\right) 
= \left(\begin{smallmatrix} x_1+x_2-x_3 \\ x_1-2x_2+3x_3 \end{smallmatrix}\right)$.

\change
Pour la base de départ on choisit $\mathcal{B}$ la base canonique de $\Rr^3$.

\change
Et pour la base d'arrivée on choisit $\mathcal{B}'$ 
la base canonique de $\Rr^2$. 

\change
Voici les trois vecteurs de la base départ

\change
et les deux de la base d'arrivée.

\change
Comment se calcule la matrice de $f$ dans les bases $\mathcal{B}$ et $\mathcal{B}'$ ?

\change
Il faut calculer les images de la base de départ et les exprimer dans la base d'arrivée.

\change
$f(e_1)$ c'est $f(1,0,0)$

\change
par la définition de $f$ cela vaut $(1,1)$

\change
ce qui s'exprime dans la base d'arrivée $f_1+f_2$. 

\change
De même $f(e_2) = f(0,1,0) =(1,-2)=f_1-2f_2$.

\change
Enfin $f(e_3) = f(0,0,1) =(-1,3)=-f_1+3f_2$.

\change
On peut maintenant compléter la matrice de $f$ :

\change
La première colonne c'est $f(e_1)$

\change
exprimer dans la base $(f_1,f_2)$.

\change
Par notre calcul la première colonne de la matrice 
    est donc $\left(\begin{smallmatrix}1\cr1\cr\end{smallmatrix}\right)$
    pour $1 f_1 + 1 f_2$.
    
\change
Pour $f(e_2)$ 

\change
c'est $f_1-2f_2$
donc la deuxième colonne de la matrice 
est $\left(\begin{smallmatrix}1\cr-2\cr\end{smallmatrix}\right)$.
    
\change
Comme $f(e_3) = -f_1+3f_2$


\change
La troisième colonne 
    est  $\left(\begin{smallmatrix}-1\cr3\cr\end{smallmatrix}\right)$.

%%%%%%%%%%%%%%%%%%%%%%%%%%%%%%%%%%%%%%%%%%%%%%%%%%%%%%%%%%%
\diapo

On reprend l'exemple précédent avec la même application linéaire $f$ mais on va maintenant changer les bases de l'espace de départ et de l'espace d'arrivée.

\change
Soient les vecteurs de $\Rr ^3$
$$\epsilon_1 = \begin{pmatrix}1\\1\\0\end{pmatrix} \quad
\epsilon_2 = \begin{pmatrix}1 \\ 0 \\ 1 \end{pmatrix} \quad
\epsilon_3 = \begin{pmatrix}0 \\ 1 \\ 1 \end{pmatrix}$$


\change
et $\phi_1$ et $\phi_2$  ces deux vecteurs de $\Rr^2$.

\change
On montre facilement que $\mathcal{B}_0 =(\epsilon_1,\epsilon_2,\epsilon_3)$ 
est une base de $\Rr^3$.
Ce sera notre nouvelle base de départ.

\change
Pour la base d'arrivée on choisit la base formée des deux vecteurs $(\phi_1,\phi_2)$.


\change
Comment calculer la matrice de $f$ dans les bases $\mathcal{B}_0$
et $\mathcal{B}_0'$ ?

\change
C'est la même méthode qu'auparavant mais cette fois il faut
exprimer les $f(\epsilon_i)$ dans la base $(\phi_1,\phi_2)$.


On commence par $f(\epsilon_1)$ qui est donc $f(1,1,0)$

\change
on calcule que cela fait $(2,-1)$ 

\change
qu'il faut exprimer en fonction de $\phi_1$ et $\phi_2$.
On trouve que c'est $3\phi_1-\phi_2$.

\change
Même chose pour $f(\epsilon_2)$ 

\change
et $f(\epsilon_3)$.

\change
On remplit la matrice colonne par colonne :
$(3,-1)$, $(-4,4)$, $(-1,1)$.

Cet exemple illustre bien le fait que la matrice de $f$ dépend du choix des
bases.

%%%%%%%%%%%%%%%%%%%%%%%%%%%%%%%%%%%%%%%%%%%%%%%%%%%%%%%%%%%
\diapo

Soient $f,g : E \to F$ deux applications linéaires 

\change
et soient $\mathcal{B}$ une base de $E$ 

\change
et $\mathcal{B}'$ une base de $F$.


\change
Alors :
$\Mat_{\mathcal{B},\mathcal{B}'} (f+g) = \Mat_{\mathcal{B},\mathcal{B}'} (f)
  + \Mat_{\mathcal{B},\mathcal{B}'} (g)$ :
  
la matrice associée à la somme de deux applications linéaires est la
somme des matrices, à condition de considérer la même base sur l'espace de départ 
pour les deux applications et la même base sur l'espace d'arrivée. 

\change
Et $\Mat_{\mathcal{B},\mathcal{B}'} ( \lambda f) = \lambda \Mat_{\mathcal{B},\mathcal{B}'} (f)$ :
la matrice de $\lambda f$ est $\lambda$ fois la matrice de $f$.

\change
Voyons une autre façon d'écrire cette proposition.

On note : $A = \Mat_{\mathcal{B},\mathcal{B}'} (f)$
et  $B = \Mat_{\mathcal{B},\mathcal{B}'} (g)$

\change
Si on pose $C = \Mat_{\mathcal{B},\mathcal{B}'} (f+g)$

et $D = \Mat_{\mathcal{B},\mathcal{B}'} (\lambda f)$

\change
Alors le premier  point de la proposition 
signifie 

$C = A+B$

\change
et le second point signifie $D = \lambda A$.



%%%%%%%%%%%%%%%%%%%%%%%%%%%%%%%%%%%%%%%%%%%%%%%%%%%%%%%%%%%
\diapo

Ce qui est le plus important va être la composition des applications linéaires.

\change
Soient $f : E \to F$  et $g : F \to G$ deux applications linéaires 

\change
et soient
$\mathcal{B}$ une base de $E$, 

$\mathcal{B}'$ une base de $F$

et $\mathcal{B}''$ une base de $G$.

\change
Alors :
$\Mat_{\mathcal{B},\mathcal{B}''} (g \circ f)
= \Mat_{\mathcal{B}',\mathcal{B}''} (g) \times \Mat_{\mathcal{B},\mathcal{B}'} (f)$.

Autrement dit, à condition de bien choisir les bases, la matrice
associée à la composition de deux applications linéaires est le
produit des matrices associées à chacune d'elles, dans le même ordre.

\change
Une autre façon de le dire, si on note :
$A = \Mat_{\mathcal{B},\mathcal{B}'} (f)$

$B = \Mat_{\mathcal{B}',\mathcal{B}''} (g)$

$C = \Mat_{\mathcal{B},\mathcal{B}''} (g \circ f)$

Alors $C = B\times A$.

En fait, le produit de matrices, qui semble compliqué au premier abord,
est défini afin de correspondre à la composition des applications linéaires.


%%%%%%%%%%%%%%%%%%%%%%%%%%%%%%%%%%%%%%%%%%%%%%%%%%%%%%%%%%%
\diapo

Voyons un exemple de composition d'application linéaire.

On considère d'abord 
$f : \Rr^2 \to \Rr^3$ 
dont la matrice dans des bases $\mathcal{B},\mathcal{B}'$
est cette matrice $A$.

\change
Puis $g : \Rr^3 \to \Rr^2$
dont voici la matrice $B$ entre les bases $\mathcal{B}',\mathcal{B}''$.

\change
Calculons la matrice associée à $g \circ f : \Rr^2 \to \Rr^2$.

\change
Cette matrice $\Mat_{\mathcal{B},\mathcal{B}''} (g \circ f)$

\change
que l'on note $C$ 

\change
est le produit des matrices $B \times A$. 

\change
Il ne reste plus qu'à calculer le produit de ces deux matrices.

\change
Pour obtenir la matrice $g\circ f$.



%%%%%%%%%%%%%%%%%%%%%%%%%%%%%%%%%%%%%%%%%%%%%%%%%%%%%%%%%%%
\diapo

Dans cette section, on étudie la cas où l'espace de départ et l'espace d'arrivée 
sont identiques :
$f : E \to E$ est un endomorphisme. 

\change
On pourrait choisir une base de départ différente de la base d'arrivée,

\change
mais on va se placer dans le cas spécifique où
on choisit la même base $\mathcal{B}$ au départ et à l'arrivée.

Alors on note  simplement $\Mat_{\mathcal{B}} (f)$ la matrice associée à $f$.
  
  
 \change
 Voyons quelques exemples. Tout d'abord l'identité, qui bien sûr à $x$ associe $x$ :
 
 \change
 alors quelle que soit la base $\mathcal{B}$ de $E$, 
 la matrice associée est la matrice identité.
 
 Attention ! Ce n'est plus vrai 
  si la base d'arrivée est différente de la base de départ.
  
 \change  
  Une homothétie de rapport $\lambda$ 
  envoie $x$ sur $\lambda x$.
  
  \change
  Sa matrice dans toute base est $\lambda I_n$.
  
  \change
  La symétrie centrale envoie $x$ sur $-x$
  
  \change
  sa matrice est moins l'identité.
  
  \change
  On termine par une rotation du plan.
  
  \change
  Soit $r_\theta$ la rotation d'angle $\theta$, centrée à l'origine.
  
  \change
  L'espace vectoriel est $\Rr^2$ muni de la base canonique.
  
  \change
  La rotation associe à $(x,y)$ le 
  couple 
  
  $(x \cos  \theta - y \sin\theta,x\sin \, \theta + y \cos\theta)$.
  
  \change
  Cela permet de calculer l'image du vecteur $(1,0)$ et du vecteur $(0,1)$
  et donne la matrice de la rotation :
  $\begin{pmatrix}
\cos\theta & -\sin\theta\\
\sin\theta & \cos\theta
\end{pmatrix}.$


%%%%%%%%%%%%%%%%%%%%%%%%%%%%%%%%%%%%%%%%%%%%%%%%%%%%%%%%%%%
\diapo

Voici le cas particulier de la puissance d'un endomorphisme de $E$

Soient $E$ un espace vectoriel de dimension finie et $\mathcal{B}$ une base de $E$.
Soit $f :  E \to E$ une application linéaire avec donc la même base
$\mathcal{B}$ au départ et à l'arrivée.


Alors
$\Mat_{\mathcal{B}} (f^p) = \big( \Mat_{\mathcal{B}} (f) \big)^p$

\change
Rappelons que $f^p$ désigne $f \circ f \circ\cdots \circ f$,
avec $p$ facteurs.

Si $A$ est la matrice associée à $f$,
alors $A^p$ est la matrice associée à $f^p$.

Par contre $A^p$ est bien le produit de $A\times A \cdots$,
avec $p$ facteurs.

\change
Voici un exemple avec  $r_\theta$ la matrice de la rotation d'angle $\theta$ dans le plan.


\change
La matrice de l'itération $r_\theta^p$  est donc
la matrice de $r_\theta$ à la puissance $p$.

\change
C'est donc $\begin{pmatrix}
\cos\theta & -\sin\theta\\
\sin\theta & \cos\theta
\end{pmatrix}^p$


\change
Un calcul par récurrence montre ensuite 
que cela vaut $= \begin{pmatrix}
\cos(p\theta) & -\sin(p\theta)\\
\sin(p\theta) & \cos(p\theta)
\end{pmatrix},$

ce qui est bien la matrice de la rotation d'angle $p \theta$ :  on vient de justifier par le calcul que 
composer $p$ fois la rotation d'angle $\theta$ revient à 
effectuer une rotation d'angle $p\theta$.


%%%%%%%%%%%%%%%%%%%%%%%%%%%%%%%%%%%%%%%%%%%%%%%%%%%%%%%%%%%
\diapo

Passons maintenant aux isomorphismes. 
Rappelons que $f : E \to F$ est appelé un isomorphisme 
si c'est une application linéaire bijective. Nous avons vu que cela entraîne 
$\dim E = \dim F$.

\change
Voici une caractérisation  des isomorphismes par leur matrice.


Les hypothèses sont les suivantes :

$E$ et $F$ sont deux espaces vectoriels de *même* dimension finie.

$f$ est une application linéaire de $E \to F$. 

$\mathcal{B}$ est une base de $E$,


$\mathcal{B}'$ est une base de $F$ 

et enfin $A$ désigne la matrice de $f$ entre les bases $\mathcal{B}$ et $\mathcal{B}'$.


\change
Premier résultat :

$f$ est bijective si et seulement si la matrice $A$ est inversible.

  Autrement dit, $f$ est un isomorphisme si et seulement si sa matrice associée 
   est inversible.

\change
Deuxième résultat :

  Dans le cas où $f$ est bijective, 
  
  alors la matrice de l'application linéaire
  $f^{-1}$ est la matrice $A^{-1}$. 
  
  \change
  Autrement dit, la matrice de l'inverse est l'inverse de la matrice.
  
 

%%%%%%%%%%%%%%%%%%%%%%%%%%%%%%%%%%%%%%%%%%%%%%%%%%%%%%%%%%%
\diapo

Voici le cas particulier très important d'un endomorphisme $f : E \to E$ où
$E$ est muni de la même base $\mathcal{B}$ au départ et à l'arrivée. On note $A = \Mat_{\mathcal{B}}(f)$.

\change
(1) $f$ est bijective si et seulement si $A$ est inversible.
  
(2) Si $f$ est bijective, alors la matrice associée à $f^{-1}$ 
dans la base $\mathcal{B}$ est $A^{-1}$.

\change
Ce qui se reformule 
$\Mat_{\mathcal{B}}(f^{-1}) = \big(\Mat_{\mathcal{B}}(f) \big)^{-1}$.




%%%%%%%%%%%%%%%%%%%%%%%%%%%%%%%%%%%%%%%%%%%%%%%%%%%%%%%%%%%
\diapo
Soit $r : \Rr^2 \to \Rr^2$ la rotation d'angle $\frac\pi6$ (centrée à l'origine).

\change
Sa matrice dans la base canonique est donnée par la matrice de rotation
$
  \begin{pmatrix}
\cos\theta & -\sin\theta\\
\sin\theta & \cos\theta
\end{pmatrix}$

\change
avec $\theta = \frac\pi6$,
voici la matrice de $r$.

\change
Maintenant on note $s$ la réflexion par rapport à la droite d'équation $(y=x)$. 

\change
Sa matrice est $B = \Mat_{\mathcal{B}} (s) = \begin{pmatrix} 0&1\\1&0 \end{pmatrix}$

Quelle est la matrice associée à $(s \circ r)^{-1}$ ?


\change
On commence par la matrice $s \circ r$ 

\change
qui est le produit $B \times A$

\change
on trouve 
$= \begin{pmatrix}
\frac12&\frac{\sqrt{3}}{2} & \\
\frac{\sqrt{3}}{2}&-\frac12& 
\end{pmatrix}$.

\change
Maintenant la matrice de $(s \circ r)^{-1}$ est 

\change
l'inverse de la matrice de $s \circ r$.

\change
On calcule l'inverse de la matrice

\change
pour obtenir la matrice de $(s \circ r)^{-1}$.

On note que la matrice $BA$ est égale à son inverse.

En termes d'applications linéaires, cela signifie que $s\circ r$ est son propre inverse.
  
%%%%%%%%%%%%%%%%%%%%%%%%%%%%%%%%%%%%%%%%%%%%%%%%%%%%%%%%%%%
\diapo

Voici quelques exercices où il s'agit d'écrire les matrices associées à des transformations géométriques.



\end{document}
