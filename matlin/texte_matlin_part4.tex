
%%%%%%%%%%%%%%%%%% PREAMBULE %%%%%%%%%%%%%%%%%%


\documentclass[12pt]{article}

\usepackage{amsfonts,amsmath,amssymb,amsthm}
\usepackage[utf8]{inputenc}
\usepackage[T1]{fontenc}
\usepackage[francais]{babel}


% packages
\usepackage{amsfonts,amsmath,amssymb,amsthm}
\usepackage[utf8]{inputenc}
\usepackage[T1]{fontenc}
%\usepackage{lmodern}

\usepackage[francais]{babel}
\usepackage{fancybox}
\usepackage{graphicx}

\usepackage{float}

%\usepackage[usenames, x11names]{xcolor}
\usepackage{tikz}
\usepackage{datetime}

\usepackage{mathptmx}
%\usepackage{fouriernc}
%\usepackage{newcent}
\usepackage[mathcal,mathbf]{euler}

%\usepackage{palatino}
%\usepackage{newcent}


% Commande spéciale prompteur

%\usepackage{mathptmx}
%\usepackage[mathcal,mathbf]{euler}
%\usepackage{mathpple,multido}

\usepackage[a4paper]{geometry}
\geometry{top=2cm, bottom=2cm, left=1cm, right=1cm, marginparsep=1cm}

\newcommand{\change}{{\color{red}\rule{\textwidth}{1mm}\\}}

\newcounter{mydiapo}

\newcommand{\diapo}{\newpage
\hfill {\normalsize  Diapo \themydiapo \quad \texttt{[\jobname]}} \\
\stepcounter{mydiapo}}


%%%%%%% COULEURS %%%%%%%%%%

% Pour blanc sur noir :
%\pagecolor[rgb]{0.5,0.5,0.5}
% \pagecolor[rgb]{0,0,0}
% \color[rgb]{1,1,1}



%\DeclareFixedFont{\myfont}{U}{cmss}{bx}{n}{18pt}
\newcommand{\debuttexte}{
%%%%%%%%%%%%% FONTES %%%%%%%%%%%%%
\renewcommand{\baselinestretch}{1.5}
\usefont{U}{cmss}{bx}{n}
\bfseries

% Taille normale : commenter le reste !
%Taille Arnaud
%\fontsize{19}{19}\selectfont

% Taille Barbara
%\fontsize{21}{22}\selectfont

%Taille François
\fontsize{25}{30}\selectfont

%Taille Pascal
%\fontsize{25}{30}\selectfont

%Taille Laura
%\fontsize{30}{35}\selectfont


%\myfont
%\usefont{U}{cmss}{bx}{n}

%\Huge
%\addtolength{\parskip}{\baselineskip}
}


% \usepackage{hyperref}
% \hypersetup{colorlinks=true, linkcolor=blue, urlcolor=blue,
% pdftitle={Exo7 - Exercices de mathématiques}, pdfauthor={Exo7}}


%section
% \usepackage{sectsty}
% \allsectionsfont{\bf}
%\sectionfont{\color{Tomato3}\upshape\selectfont}
%\subsectionfont{\color{Tomato4}\upshape\selectfont}

%----- Ensembles : entiers, reels, complexes -----
\newcommand{\Nn}{\mathbb{N}} \newcommand{\N}{\mathbb{N}}
\newcommand{\Zz}{\mathbb{Z}} \newcommand{\Z}{\mathbb{Z}}
\newcommand{\Qq}{\mathbb{Q}} \newcommand{\Q}{\mathbb{Q}}
\newcommand{\Rr}{\mathbb{R}} \newcommand{\R}{\mathbb{R}}
\newcommand{\Cc}{\mathbb{C}} 
\newcommand{\Kk}{\mathbb{K}} \newcommand{\K}{\mathbb{K}}

%----- Modifications de symboles -----
\renewcommand{\epsilon}{\varepsilon}
\renewcommand{\Re}{\mathop{\text{Re}}\nolimits}
\renewcommand{\Im}{\mathop{\text{Im}}\nolimits}
%\newcommand{\llbracket}{\left[\kern-0.15em\left[}
%\newcommand{\rrbracket}{\right]\kern-0.15em\right]}

\renewcommand{\ge}{\geqslant}
\renewcommand{\geq}{\geqslant}
\renewcommand{\le}{\leqslant}
\renewcommand{\leq}{\leqslant}

%----- Fonctions usuelles -----
\newcommand{\ch}{\mathop{\mathrm{ch}}\nolimits}
\newcommand{\sh}{\mathop{\mathrm{sh}}\nolimits}
\renewcommand{\tanh}{\mathop{\mathrm{th}}\nolimits}
\newcommand{\cotan}{\mathop{\mathrm{cotan}}\nolimits}
\newcommand{\Arcsin}{\mathop{\mathrm{Arcsin}}\nolimits}
\newcommand{\Arccos}{\mathop{\mathrm{Arccos}}\nolimits}
\newcommand{\Arctan}{\mathop{\mathrm{Arctan}}\nolimits}
\newcommand{\Argsh}{\mathop{\mathrm{Argsh}}\nolimits}
\newcommand{\Argch}{\mathop{\mathrm{Argch}}\nolimits}
\newcommand{\Argth}{\mathop{\mathrm{Argth}}\nolimits}
\newcommand{\pgcd}{\mathop{\mathrm{pgcd}}\nolimits} 

\newcommand{\Card}{\mathop{\text{Card}}\nolimits}
\newcommand{\Ker}{\mathop{\text{Ker}}\nolimits}
\newcommand{\id}{\mathop{\text{id}}\nolimits}
\newcommand{\ii}{\mathrm{i}}
\newcommand{\dd}{\mathrm{d}}
\newcommand{\Vect}{\mathop{\text{Vect}}\nolimits}
\newcommand{\Mat}{\mathop{\mathrm{Mat}}\nolimits}
\newcommand{\rg}{\mathop{\text{rg}}\nolimits}
\newcommand{\tr}{\mathop{\text{tr}}\nolimits}
\newcommand{\ppcm}{\mathop{\text{ppcm}}\nolimits}

%----- Structure des exercices ------

\newtheoremstyle{styleexo}% name
{2ex}% Space above
{3ex}% Space below
{}% Body font
{}% Indent amount 1
{\bfseries} % Theorem head font
{}% Punctuation after theorem head
{\newline}% Space after theorem head 2
{}% Theorem head spec (can be left empty, meaning ‘normal’)

%\theoremstyle{styleexo}
\newtheorem{exo}{Exercice}
\newtheorem{ind}{Indications}
\newtheorem{cor}{Correction}


\newcommand{\exercice}[1]{} \newcommand{\finexercice}{}
%\newcommand{\exercice}[1]{{\tiny\texttt{#1}}\vspace{-2ex}} % pour afficher le numero absolu, l'auteur...
\newcommand{\enonce}{\begin{exo}} \newcommand{\finenonce}{\end{exo}}
\newcommand{\indication}{\begin{ind}} \newcommand{\finindication}{\end{ind}}
\newcommand{\correction}{\begin{cor}} \newcommand{\fincorrection}{\end{cor}}

\newcommand{\noindication}{\stepcounter{ind}}
\newcommand{\nocorrection}{\stepcounter{cor}}

\newcommand{\fiche}[1]{} \newcommand{\finfiche}{}
\newcommand{\titre}[1]{\centerline{\large \bf #1}}
\newcommand{\addcommand}[1]{}
\newcommand{\video}[1]{}

% Marge
\newcommand{\mymargin}[1]{\marginpar{{\small #1}}}



%----- Presentation ------
\setlength{\parindent}{0cm}

%\newcommand{\ExoSept}{\href{http://exo7.emath.fr}{\textbf{\textsf{Exo7}}}}

\definecolor{myred}{rgb}{0.93,0.26,0}
\definecolor{myorange}{rgb}{0.97,0.58,0}
\definecolor{myyellow}{rgb}{1,0.86,0}

\newcommand{\LogoExoSept}[1]{  % input : echelle
{\usefont{U}{cmss}{bx}{n}
\begin{tikzpicture}[scale=0.1*#1,transform shape]
  \fill[color=myorange] (0,0)--(4,0)--(4,-4)--(0,-4)--cycle;
  \fill[color=myred] (0,0)--(0,3)--(-3,3)--(-3,0)--cycle;
  \fill[color=myyellow] (4,0)--(7,4)--(3,7)--(0,3)--cycle;
  \node[scale=5] at (3.5,3.5) {Exo7};
\end{tikzpicture}}
}



\theoremstyle{definition}
%\newtheorem{proposition}{Proposition}
%\newtheorem{exemple}{Exemple}
%\newtheorem{theoreme}{Théorème}
\newtheorem{lemme}{Lemme}
\newtheorem{corollaire}{Corollaire}
%\newtheorem*{remarque*}{Remarque}
%\newtheorem*{miniexercice}{Mini-exercices}
%\newtheorem{definition}{Définition}




%definition d'un terme
\newcommand{\defi}[1]{{\color{myorange}\textbf{\emph{#1}}}}
\newcommand{\evidence}[1]{{\color{blue}\textbf{\emph{#1}}}}



 %----- Commandes divers ------

\newcommand{\codeinline}[1]{\texttt{#1}}

%%%%%%%%%%%%%%%%%%%%%%%%%%%%%%%%%%%%%%%%%%%%%%%%%%%%%%%%%%%%%
%%%%%%%%%%%%%%%%%%%%%%%%%%%%%%%%%%%%%%%%%%%%%%%%%%%%%%%%%%%%%
\newcommand{\Pass}{\mathop{\text{P}}\nolimits}

\begin{document}

\debuttexte

%%%%%%%%%%%%%%%%%%%%%%%%%%%%%%%%%%%%%%%%%%%%%%%%%%%%%%%%%%%
\diapo
\change
Voici le point culminant de ce chapitre, nous allons approfondir les relations entre matrices
et applications linéaires.

\change
On commence par une relation simple : comment décrire l'action d'une application linéaire en terme de matrices.

\change
Il y a cependant une différence entre matrice et applications linéaires.
Pour expliquer cette différence nous aurons besoin de la notion de matrice de passage d'une base à une autre.

\change
Ceci nous permettra d'énoncer la formule de changement de base.

\change
On terminera par la notion de matrices semblables.

%%%%%%%%%%%%%%%%%%%%%%%%%%%%%%%%%%%%%%%%%%%%%%%%%%%%%%%%%%%
\diapo


Soit $E$ un espace vectoriel de dimension finie et
soit $\mathcal{B} = (e_1,e_2, \dots ,e_p )$ une base de $E$. 

\change
Pour chaque $x \in E$, il existe des coordonnées $x_i$ uniques
telles que 
$x=x_1e_1+x_2e_2+\dots +x_p e_p.$

\change
La matrice des coordonnées de $x$ est  
noté $\Mat_\mathcal{B} (x)$ ou encore comme un
vecteur colonne, 
$\left(\begin{smallmatrix}x_1\cr x_2\cr \vdots \cr x_p\end{smallmatrix}\right)_\mathcal{B}$
avec l'indice $\mathcal{B}$ pour rappeler que ce sont les coordonnées dans la base $\mathcal{B}$.


\change
Soient $E$ et $F$ deux $\Kk$-espaces vectoriels de dimension finie
et  $f : E \to F$ une application linéaire. 

\change
Soit $x$ un élément de $E$ et posons $y=f(x)$.

Le but de cette proposition est de traduire l'égalité vectorielle
$y=f (x)$  par une égalité matricielle.

\change
Soit $\mathcal{B}$ une base de $E$ 

\change
et soit $\mathcal{B}'$ une base de $F$.


\change
On note encore $A = \Mat_{\mathcal{B},\mathcal{B}'} (f)$
  
\change
pour un vecteur $x$ de $E$ on note $X = \Mat_{\mathcal{B}} (x) = 
  \left(\begin{smallmatrix}x_1\cr x_2\cr \vdots \cr x_p\end{smallmatrix}\right)_\mathcal{B}$.

[petit $x$ grand $X$]

\change
Pour un vecteur $y \in F$, on note $Y = \Mat_{\mathcal{B}'} (y) = 
  \left(\begin{smallmatrix}y_1\cr y_2\cr \vdots \cr y_n\end{smallmatrix}\right)_{\mathcal{B}'}$.

[petit $y$, grand $Y$]
  
\change
Alors, si $y = f(x)$, on a $Y = AX$.

\change
Une autre façon d'écrire cette égalité matricielle $Y = AX$ est
$\Mat_{\mathcal{B}'} \big( f(x) \big) 
= \Mat_{\mathcal{B},\mathcal{B}'} (f) \times \Mat_{\mathcal{B}} (x)$

%%%%%%%%%%%%%%%%%%%%%%%%%%%%%%%%%%%%%%%%%%%%%%%%%%%%%%%%%%%
\diapo

Soient $E$ un $\Kk$-espace vectoriel de dimension $3$  
et fixons un base $\mathcal{B}$ une base de
$E$ constituée de vecteurs $(e_1,e_2,e_3)$. 

\change
Soit $f$ l'endomorphisme de $E$ dont la matrice dans la base
$\mathcal{B}$ est égale à cette matrice $A$.

\change
On se propose de déterminer le noyau de $f$.

\change
Par définition 
$x  \in \Ker f$

\change
ssi $f(x)= 0_E$

\change
En termes de coordonnées cela signifie que le vecteur $f(x)$ est 
le vecteur nul.

\change
Et en terme de matrice cela signifie $AX$ est égal au vecteur nul. [grand $X$]

\change
Le vecteur inconnu $X$ est constitué des trois coordonnées $x_1,x_2,x_3$.

\change
Cette égalité matricielle se transforme en 
ce système linéaire où les trois inconnues sont
$x_1,x_2,x_3$.

\change
Il s'agit maintenant de résoudre ce système linéaire,

\change
en fait la deuxième équation est la somme des deux autres,
on peut s'en passer.

\change
Si on choisit $t$ comme paramètre pour $x_3$, alors $x_2=-t$ et par la première équation $x_1=t$.

Donc le noyau est l'ensemble des triplets $(t,-t,t)$.

\change
Ces vecteurs sont tous colinéaire au vecteur $(1,-1,1)$.

Donc le noyau est de dimension $1$, et il est engendré par le vecteur $(1,-1,1)$.

%%%%%%%%%%%%%%%%%%%%%%%%%%%%%%%%%%%%%%%%%%%%%%%%%%%%%%%%%%%
\diapo

On continue notre exemple, mais maintenant on va calculer l'image de $f$.

\change
On vient donc de montrer que le noyau est de dimension $1$. 

\change
Par le théorème du rang, la dimension de l'image $\Im f$ 
est la dimension de l'espace $E$ moins la dimension du noyau, donc 
la dimension de l'image est $3-1=2$.

\change
Par définition de la matrice de $f$, ces trois vecteurs colonnes
sont dans l'image de $f$.

Mais comme l'image est de dimension $2$,
on prend deux vecteurs, en vérifiant qu'ils sont linéairement indépendants.

\change
Les deux premiers vecteurs de la matrice $A$ conviennent, 

et ainsi ils engendrent $\Im f$ : 

Une base de l'image est donc donnée par 
$\left(\begin{smallmatrix}1\\2\\1\end{smallmatrix}\right)_{\mathcal{B}}$
et $\left(\begin{smallmatrix}2\\3\\1\end{smallmatrix}\right)_{\mathcal{B}}$.

%%%%%%%%%%%%%%%%%%%%%%%%%%%%%%%%%%%%%%%%%%%%%%%%%%%%%%%%%%%
\diapo


Soit $E$ un espace vectoriel de dimension finie $n$. 
On sait que toutes les bases de $E$ ont $n$ éléments.

\change
Soit $\mathcal{B}$ une base de $E$. Soit $\mathcal{B}'$ une autre base de $E$.

\change
On appelle \defi{matrice de passage} de la base $\mathcal{B}$ vers la base 
$\mathcal{B}'$,  la matrice carrée de taille $n$ dont la $j$-ème colonne 
est formée des coordonnées du $j$-ème vecteur de la base $\mathcal{B}'$, 
par rapport à la base $\mathcal{B}$.

On notera cette matrice $\Pass_{\mathcal{B},\mathcal{B'}}$.

\change
On résume ceci par l'incantation :

La matrice de passage $\Pass_{\mathcal{B},\mathcal{B'}}$ est constituée - en colonnes - des
coordonnées des vecteurs de la nouvelle base $\mathcal{B}'$
exprimés dans l'ancienne base $\mathcal{B}$. 


%%%%%%%%%%%%%%%%%%%%%%%%%%%%%%%%%%%%%%%%%%%%%%%%%%%%%%%%%%%
\diapo


Soit l'espace vectoriel réel $\Rr^2$.
On considère deux vecteurs 
$
e_1 = \begin{pmatrix}1\\0\end{pmatrix} \qquad
e_2 = \begin{pmatrix}1\\1\end{pmatrix}$

\change
et deux autres vecteurs 
$
\epsilon_1 = \begin{pmatrix}1\\2\end{pmatrix} \qquad
\epsilon_2 = \begin{pmatrix}5\\4\end{pmatrix}.$


\change
Les deux premiers vecteurs forment une base que l'on appelle $\mathcal{B}$.

\change
Mais les deux autres vecteurs forment aussi une base,
que l'on appelle $\mathcal{B}'$.

\change
On souhaite savoir quelle est la matrice de passage de la base $\mathcal{B}$ vers 
la base $\mathcal{B}'$.

\change
Il faut exprimer $\epsilon_1$ en fonction de $(e_1,e_2)$.
Et aussi $\epsilon_2$ en fonction de $(e_1,e_2)$.

\change
On calcule que :
$\epsilon_1 = -e_1+2e_2$

\change
ce qui donne les coordonnées $\begin{pmatrix}-1\\2\end{pmatrix}$ dans la base ${\mathcal{B}}$.

\change
Pour $\epsilon_2$ on trouve $\epsilon_2 = e_1+4e_2$

\change
donc $\begin{pmatrix}1\\4\end{pmatrix}$ dans la base ${\mathcal{B}}$.

\change
La matrice de passage est donc composée des vecteurs de la nouvelle base $(\epsilon_1,\epsilon_2)$
écrits en colonnes et exprimés dans l'ancienne base $(e_1,e_2)$.

On trouve donc 
$$\begin{pmatrix}-1&1\\2&4\end{pmatrix}$$

%%%%%%%%%%%%%%%%%%%%%%%%%%%%%%%%%%%%%%%%%%%%%%%%%%%%%%%%%%%
\diapo


On peut également interpréter une matrice de passage comme la matrice associée à
l'application identité de $E$ par rapport à des bases bien choisies.

Proposition : 
La matrice de passage $\Pass_{\mathcal{B},\mathcal{B}'}$
de la base $\mathcal{B}$ vers la base $\mathcal{B}'$
est la matrice associée à l'identité 
où $E$ est l'espace de départ muni de la base $\mathcal{B}'$,
et $E$ est aussi l'espace d'arrivée, mais muni de la base $\mathcal{B}$.

Avec nos notations habituelles on a donc :

$\Pass_{\mathcal{B},\mathcal{B'}} = \Mat_{\mathcal{B}',\mathcal{B}} (\id_E)$

Faites bien attention à l'inversion de l'ordre des bases !


Cette interprétation permet de démontrer plusieurs résultats de façon très élégante et avec un
minimum de calculs. 

\change
Voyons quelques uns de ces résultats :


Tout d'abord la matrice de passage d'une base $\mathcal{B}$ vers une base $\mathcal{B}'$ 
  est inversible et son inverse est égale à la matrice de passage de la base $\mathcal{B}'$ 
  vers la base $\mathcal{B}$ : 
  
Ce qui s'écrit : $\Pass_{\mathcal{B}',\mathcal{B}} = \big( \Pass_{\mathcal{B},\mathcal{B}'} \big)^{-1}$
  

\change
Enfin si $\mathcal{B}$, $\mathcal{B}'$ et $\mathcal{B}''$ sont trois bases, alors
on peut passer facilement de la première à la troisième base par la formule :

$\Pass_{\mathcal{B},\mathcal{B}''} = \Pass_{\mathcal{B},\mathcal{B}'}
  \times \Pass_{\mathcal{B}',\mathcal{B}''}$


%%%%%%%%%%%%%%%%%%%%%%%%%%%%%%%%%%%%%%%%%%%%%%%%%%%%%%%%%%%
\diapo


Soit $E = \Rr^3$ muni de sa base canonique $\mathcal{B}$. 

\change
Définissons deux nouvelles bases :

une première base $\mathcal{B}_1$
que voici

\change
Et un une seconde base $\mathcal{B}_2$
que voilà.

Pour ces deux bases les vecteurs sont exprimés dans la base canonique
de $\Rr^3$.

\change
Vous vous demandez quelle est la matrice de passage de $\mathcal{B}_1$ vers $\mathcal{B}_2$.

\change
Tout d'abord il est très facile de calculer la matrice de passage de la base canonique vers la base 
$\mathcal{B}_1$. 

Comme elle est constituée des coordonnées de la nouvelle base $\mathcal{B}_1$ dans l'ancienne base $\mathcal{B}$ (qui est la base canonique), il s'agit simplement de copier les vecteurs colonnes de la base $\mathcal{B}_1$.

\change
De même pour calculer la matrice de passage de la base canonique vers la base 
$\mathcal{B}_2$, on copie et on juxtapose les vecteurs colonnes pour former la matrice.

Revenons à notre question : calculer la matrice de passage de $\mathcal{B}_1$ vers $\mathcal{B}_2$ ?

\change
La proposition vue juste avant implique la relation

$\Pass_{\mathcal{B}, \mathcal{B}_2} = \Pass_{\mathcal{B}, \mathcal{B}_1} \times \Pass_{\mathcal{B}_1, \mathcal{B}_2}$.

\change
On y est presque ! Cette relation implique 
que 
$\Pass_{\mathcal{B}_1, \mathcal{B}_2} =  \Pass_{\mathcal{B}, \mathcal{B}_1}^{-1} \times \Pass_{\mathcal{B}, \mathcal{B}_2}$.


\change
En appliquant la méthode de Gauss pour calculer cet inverse 

\change
on trouve alors :
$$\Pass_{\mathcal{B}_1, \mathcal{B}_2} 
= \begin{pmatrix}
1 & 0 & -3\\
2 & -1 & -1\\
0 & 0 & 1
\end{pmatrix}.$$


%%%%%%%%%%%%%%%%%%%%%%%%%%%%%%%%%%%%%%%%%%%%%%%%%%%%%%%%%%%
\diapo

Nous allons maintenant étudier l'effet d'un changement de bases sur
les coordonnées d'un vecteur.

Soient $\mathcal{B}$ et 
$\mathcal{B}'$ deux bases d'un même espace vectoriel $E$. 


\change
Soit $\Pass_{\mathcal{B},\mathcal{B}'}$ la matrice de passage de 
la base $\mathcal{B}$ vers la base $\mathcal{B}'$. 

\change
Pour $x \in E$, il se décompose dans la base $\mathcal{B}$ en $x=\sum_{i=1}^n x_ie_i$

\change
on note $X$ le vecteur $\left(\begin{smallmatrix}x_1\cr x_2\cr \vdots \cr x_n
\end{smallmatrix}\right)$ des coordonnées dans la base $\mathcal{B}$.

\change
Ce *même* $x$ se décompose dans la base $\mathcal{B}'$ en $x=\sum_{i=1}^n x'_ie'_i$.

\change
On note cette fois $X'$ le vecteur 
$\left(\begin{smallmatrix}x'_1\cr x'_2\cr \vdots \cr x'_n
\end{smallmatrix}\right)$ 
des coordonnées dans la base $\mathcal{B}'$.

\change
Comment retrouver grand $X'$ à partir de $X$ ?
C'est simple il suffit d'utiliser la matrice de passage.

La formule donne la relation 

$X = \Pass_{\mathcal{B},\mathcal{B}'} \times X'$


Notez bien l'ordre ! On a en fait exprimer $X$ en fonction de $X'$.
Pour $X'$ en fonction de $X$, c'est bien sûr $X' = P^{-1} X$.

\change
Voyons la démonstration :

Nous avons que la matrice de passage $P$ de la base $\mathcal{B}$ vers la base $\mathcal{B}'$
est la matrice de l'identité de $E$ dans $E$ avec $\mathcal{B}'$ 
comme base de départ et $\mathcal{B}$ comme base d'arrivée.

\change
$X$ 

\change
par définition c'est $\Mat_\mathcal{B} (x)$

\change
On utilise ensuite que $x$ c'est l'identité appliqué à $x$ !

\change
Maintenant on applique la première proposition de cette séquence à l'application linéaire  $f$ égale l'identité, qui nous dit que le vecteur $f(x)$ c'est la matrice de $f$ fois le vecteur $x$. 

Donc ici, le vecteur $x$ dans la base d'arrivée $\mathcal{B}$

c'est la matrice de identité de la base $\mathcal{B}'$ à la base $\mathcal{B}$

fois le vecteur $x$ dans la base de départ $\mathcal{B}'$.


\change
On a rappelé que ceci était la matrice de passage de la base $\mathcal{B}$ vers la base $\mathcal{B}'$.

Ainsi on obtient la formule souhaitée !


%%%%%%%%%%%%%%%%%%%%%%%%%%%%%%%%%%%%%%%%%%%%%%%%%%%%%%%%%%%
\diapo

Voici comment passer simplement de la matrice d'une application linéaire dans une base à la matrice de la même application linéaire dans une autre base.

\change
Nous allons voir cette formule de changement de base dans 
le cas particulier d'un endomorphisme :

Voici les données $f : E \to E$ est une application linéaire.  

\change
Et on a deux bases $\mathcal{B}$, $\mathcal{B}'$ de $E$.

\change 
$A$ est la matrice de l'application linéaire $f$ dans la base 
  $\mathcal{B}$. On choisit donc ici la même base $\mathcal{B}$ au départ et à l'arrivée.
  
\change
Par contre soit $B$ la matrice de l'application linéaire $f$ dans
  la base $\mathcal{B}'$. Ici c'est $\mathcal{B}'$ qui est la base 
  de départ et d'arrivée.
  
  
\change
Enfin soit $P$ la matrice de passage de la base $\mathcal{B}$ à la base $\mathcal{B}'$.
  
\change
La formule de changement de base est la suivante :

$B = P^{-1} A P$.

Si on connaît la matrice de $f$ dans une base [montrer $A$]
alors on peut calculer la matrice de $f$ dans n'importe quelle autre base [montrer $B$].

C'est une formule essentielle de l'algèbre linéaire.



%%%%%%%%%%%%%%%%%%%%%%%%%%%%%%%%%%%%%%%%%%%%%%%%%%%%%%%%%%%
\diapo

\change
Reprenons les deux bases de $\Rr^3$ d'un exemple étudié précédemment :

Voici $\mathcal{B}_1$

et voilà $\mathcal{B}_2$.

On considère  $f : \Rr^3 \to \Rr^3$ l'application linéaire dont
la matrice  est cette matrice $A$,
avec $\mathcal{B}_1$ comme base au départ et à l'arrivée.

\change
Le problème est le suivant : 
Que vaut la matrice de $f$ cette fois dans la base $\mathcal{B}_2$ ?

\change
Nous avons déjà calculé que la matrice de passage de $\mathcal{B}_1$ vers $\mathcal{B}_2$ 
était cette matrice.

\change
On applique la formule du changement de base :

\change
La matrice $B$ de $f$ dans la base $\mathcal{B}_2$ 
est donnée par la formule 
$B = P^{-1} A P$.

\change
On calcule l'inverse de $P$, et on effectue les produits

\change
pour obtenir la matrice $B$
qui est ici une matrice diagonale avec $1,2,3$ sur la diagonale.


C'est souvent l'intérêt des changements de base : se ramener à une matrice plus simple.


Par exemple ici, il est maintenant facile de calculer les puissances $B^k$, 
pour en déduire les $A^k$.


%%%%%%%%%%%%%%%%%%%%%%%%%%%%%%%%%%%%%%%%%%%%%%%%%%%%%%%%%%%
\diapo


Je considère ici des matrices carrées de taille $n$.

La matrice $B$ est dite \defi{semblable} à la matrice $A$ s'il existe une
matrice inversible $P$ telle que
$B=P^{-1}AP$.
   
   
\change
C'est un bon exercice de montrer que la relation << être semblable >> est une relation 
d'équivalence :

\change
cela signifie que :
La relation est \evidence{réflexive} : une matrice $A$ est semblable à elle-même.

\change
La relation est \evidence{symétrique} : si $A$ est semblable à $B$, 
  alors $B$ est semblable à $A$.

\change
Enfin La relation est \evidence{transitive} : si $A$ est semblable à $B$, 
  et $B$ est semblable à $C$, alors $A$ est semblable à $C$.

  

Compte tenu de ces propriétés, on dit simplement que $A$ et $B$ sont 
semblables.

\change
On termine par une remarque fondamentale, qui est une conséquence de la formule de changement de base et qui conclue le lien entre matrice et application linéaire :

Corollaire : Deux matrices semblables représentent le même endomorphisme, mais exprimé dans
des bases différentes.



%%%%%%%%%%%%%%%%%%%%%%%%%%%%%%%%%%%%%%%%%%%%%%%%%%%%%%%%%%%
\diapo

Vous êtes arrivé à un seuil important de l'algèbre,
il est temps pour vous de vous entraîner 

\end{document}
