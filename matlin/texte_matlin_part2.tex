
%%%%%%%%%%%%%%%%%% PREAMBULE %%%%%%%%%%%%%%%%%%


\documentclass[12pt]{article}

\usepackage{amsfonts,amsmath,amssymb,amsthm}
\usepackage[utf8]{inputenc}
\usepackage[T1]{fontenc}
\usepackage[francais]{babel}


% packages
\usepackage{amsfonts,amsmath,amssymb,amsthm}
\usepackage[utf8]{inputenc}
\usepackage[T1]{fontenc}
%\usepackage{lmodern}

\usepackage[francais]{babel}
\usepackage{fancybox}
\usepackage{graphicx}

\usepackage{float}

%\usepackage[usenames, x11names]{xcolor}
\usepackage{tikz}
\usepackage{datetime}

\usepackage{mathptmx}
%\usepackage{fouriernc}
%\usepackage{newcent}
\usepackage[mathcal,mathbf]{euler}

%\usepackage{palatino}
%\usepackage{newcent}


% Commande spéciale prompteur

%\usepackage{mathptmx}
%\usepackage[mathcal,mathbf]{euler}
%\usepackage{mathpple,multido}

\usepackage[a4paper]{geometry}
\geometry{top=2cm, bottom=2cm, left=1cm, right=1cm, marginparsep=1cm}

\newcommand{\change}{{\color{red}\rule{\textwidth}{1mm}\\}}

\newcounter{mydiapo}

\newcommand{\diapo}{\newpage
\hfill {\normalsize  Diapo \themydiapo \quad \texttt{[\jobname]}} \\
\stepcounter{mydiapo}}


%%%%%%% COULEURS %%%%%%%%%%

% Pour blanc sur noir :
%\pagecolor[rgb]{0.5,0.5,0.5}
% \pagecolor[rgb]{0,0,0}
% \color[rgb]{1,1,1}



%\DeclareFixedFont{\myfont}{U}{cmss}{bx}{n}{18pt}
\newcommand{\debuttexte}{
%%%%%%%%%%%%% FONTES %%%%%%%%%%%%%
\renewcommand{\baselinestretch}{1.5}
\usefont{U}{cmss}{bx}{n}
\bfseries

% Taille normale : commenter le reste !
%Taille Arnaud
%\fontsize{19}{19}\selectfont

% Taille Barbara
%\fontsize{21}{22}\selectfont

%Taille François
\fontsize{25}{30}\selectfont

%Taille Pascal
%\fontsize{25}{30}\selectfont

%Taille Laura
%\fontsize{30}{35}\selectfont


%\myfont
%\usefont{U}{cmss}{bx}{n}

%\Huge
%\addtolength{\parskip}{\baselineskip}
}


% \usepackage{hyperref}
% \hypersetup{colorlinks=true, linkcolor=blue, urlcolor=blue,
% pdftitle={Exo7 - Exercices de mathématiques}, pdfauthor={Exo7}}


%section
% \usepackage{sectsty}
% \allsectionsfont{\bf}
%\sectionfont{\color{Tomato3}\upshape\selectfont}
%\subsectionfont{\color{Tomato4}\upshape\selectfont}

%----- Ensembles : entiers, reels, complexes -----
\newcommand{\Nn}{\mathbb{N}} \newcommand{\N}{\mathbb{N}}
\newcommand{\Zz}{\mathbb{Z}} \newcommand{\Z}{\mathbb{Z}}
\newcommand{\Qq}{\mathbb{Q}} \newcommand{\Q}{\mathbb{Q}}
\newcommand{\Rr}{\mathbb{R}} \newcommand{\R}{\mathbb{R}}
\newcommand{\Cc}{\mathbb{C}} 
\newcommand{\Kk}{\mathbb{K}} \newcommand{\K}{\mathbb{K}}

%----- Modifications de symboles -----
\renewcommand{\epsilon}{\varepsilon}
\renewcommand{\Re}{\mathop{\text{Re}}\nolimits}
\renewcommand{\Im}{\mathop{\text{Im}}\nolimits}
%\newcommand{\llbracket}{\left[\kern-0.15em\left[}
%\newcommand{\rrbracket}{\right]\kern-0.15em\right]}

\renewcommand{\ge}{\geqslant}
\renewcommand{\geq}{\geqslant}
\renewcommand{\le}{\leqslant}
\renewcommand{\leq}{\leqslant}

%----- Fonctions usuelles -----
\newcommand{\ch}{\mathop{\mathrm{ch}}\nolimits}
\newcommand{\sh}{\mathop{\mathrm{sh}}\nolimits}
\renewcommand{\tanh}{\mathop{\mathrm{th}}\nolimits}
\newcommand{\cotan}{\mathop{\mathrm{cotan}}\nolimits}
\newcommand{\Arcsin}{\mathop{\mathrm{Arcsin}}\nolimits}
\newcommand{\Arccos}{\mathop{\mathrm{Arccos}}\nolimits}
\newcommand{\Arctan}{\mathop{\mathrm{Arctan}}\nolimits}
\newcommand{\Argsh}{\mathop{\mathrm{Argsh}}\nolimits}
\newcommand{\Argch}{\mathop{\mathrm{Argch}}\nolimits}
\newcommand{\Argth}{\mathop{\mathrm{Argth}}\nolimits}
\newcommand{\pgcd}{\mathop{\mathrm{pgcd}}\nolimits} 

\newcommand{\Card}{\mathop{\text{Card}}\nolimits}
\newcommand{\Ker}{\mathop{\text{Ker}}\nolimits}
\newcommand{\id}{\mathop{\text{id}}\nolimits}
\newcommand{\ii}{\mathrm{i}}
\newcommand{\dd}{\mathrm{d}}
\newcommand{\Vect}{\mathop{\text{Vect}}\nolimits}
\newcommand{\Mat}{\mathop{\mathrm{Mat}}\nolimits}
\newcommand{\rg}{\mathop{\text{rg}}\nolimits}
\newcommand{\tr}{\mathop{\text{tr}}\nolimits}
\newcommand{\ppcm}{\mathop{\text{ppcm}}\nolimits}

%----- Structure des exercices ------

\newtheoremstyle{styleexo}% name
{2ex}% Space above
{3ex}% Space below
{}% Body font
{}% Indent amount 1
{\bfseries} % Theorem head font
{}% Punctuation after theorem head
{\newline}% Space after theorem head 2
{}% Theorem head spec (can be left empty, meaning ‘normal’)

%\theoremstyle{styleexo}
\newtheorem{exo}{Exercice}
\newtheorem{ind}{Indications}
\newtheorem{cor}{Correction}


\newcommand{\exercice}[1]{} \newcommand{\finexercice}{}
%\newcommand{\exercice}[1]{{\tiny\texttt{#1}}\vspace{-2ex}} % pour afficher le numero absolu, l'auteur...
\newcommand{\enonce}{\begin{exo}} \newcommand{\finenonce}{\end{exo}}
\newcommand{\indication}{\begin{ind}} \newcommand{\finindication}{\end{ind}}
\newcommand{\correction}{\begin{cor}} \newcommand{\fincorrection}{\end{cor}}

\newcommand{\noindication}{\stepcounter{ind}}
\newcommand{\nocorrection}{\stepcounter{cor}}

\newcommand{\fiche}[1]{} \newcommand{\finfiche}{}
\newcommand{\titre}[1]{\centerline{\large \bf #1}}
\newcommand{\addcommand}[1]{}
\newcommand{\video}[1]{}

% Marge
\newcommand{\mymargin}[1]{\marginpar{{\small #1}}}



%----- Presentation ------
\setlength{\parindent}{0cm}

%\newcommand{\ExoSept}{\href{http://exo7.emath.fr}{\textbf{\textsf{Exo7}}}}

\definecolor{myred}{rgb}{0.93,0.26,0}
\definecolor{myorange}{rgb}{0.97,0.58,0}
\definecolor{myyellow}{rgb}{1,0.86,0}

\newcommand{\LogoExoSept}[1]{  % input : echelle
{\usefont{U}{cmss}{bx}{n}
\begin{tikzpicture}[scale=0.1*#1,transform shape]
  \fill[color=myorange] (0,0)--(4,0)--(4,-4)--(0,-4)--cycle;
  \fill[color=myred] (0,0)--(0,3)--(-3,3)--(-3,0)--cycle;
  \fill[color=myyellow] (4,0)--(7,4)--(3,7)--(0,3)--cycle;
  \node[scale=5] at (3.5,3.5) {Exo7};
\end{tikzpicture}}
}



\theoremstyle{definition}
%\newtheorem{proposition}{Proposition}
%\newtheorem{exemple}{Exemple}
%\newtheorem{theoreme}{Théorème}
\newtheorem{lemme}{Lemme}
\newtheorem{corollaire}{Corollaire}
%\newtheorem*{remarque*}{Remarque}
%\newtheorem*{miniexercice}{Mini-exercices}
%\newtheorem{definition}{Définition}




%definition d'un terme
\newcommand{\defi}[1]{{\color{myorange}\textbf{\emph{#1}}}}
\newcommand{\evidence}[1]{{\color{blue}\textbf{\emph{#1}}}}



 %----- Commandes divers ------

\newcommand{\codeinline}[1]{\texttt{#1}}

%%%%%%%%%%%%%%%%%%%%%%%%%%%%%%%%%%%%%%%%%%%%%%%%%%%%%%%%%%%%%
%%%%%%%%%%%%%%%%%%%%%%%%%%%%%%%%%%%%%%%%%%%%%%%%%%%%%%%%%%%%%



\begin{document}

\debuttexte


%%%%%%%%%%%%%%%%%%%%%%%%%%%%%%%%%%%%%%%%%%%%%%%%%%%%%%%%%%%
\diapo

\change
Lorsque l'espace de départ d'une application linéaire est de dimension finie,
la théorie de la dimension fournit de nouvelles propriétés très
riches pour cette application linéaire.

\change
Nous verrons comment construire une application linéaire
à partir d'une base.

\change
Nous définirons le rang d'une application linéaire comme étant la dimension de l'image.

\change
Le théorème du rang sera le résultat crucial de ce chapitre, nous en verrons
l'énoncé et des applications,

\change
et en particulier le cas des applications linéaires 
entre deux espaces de même dimension

%%%%%%%%%%%%%%%%%%%%%%%%%%%%%%%%%%%%%%%%%%%%%%%%%%%%%%%%%%%
\diapo

Une application linéaire $f : E \to F$, d'un espace vectoriel de dimension finie dans un 
espace vectoriel quelconque, est entièrement déterminée par les 
images des vecteurs d'une base de l'espace vectoriel $E$ de départ.

\change
C'est ce qu'affirme le théorème suivant :

Soient $E$ et $F$ deux espaces vectoriels sur un même corps $\Kk$.
On suppose que l'espace vectoriel $E$ est de dimension finie $n$
et que $(e_1,\dots,e_n)$ est une base de $E$.
Alors pour tout choix $(v_1, \ldots ,v_n)$ 
de $n$ vecteurs de l'espace d'arrivée $F$, il existe une et une seule application linéaire $f : E \to F$
 telle que, pour tout $i=1,\ldots,n$ :
$$f(e_i)=v_i.$$

\change
Voyons comment définir explicitement cette application sur un exemple.

On souhaite construire une application linéaire $f$ qui va de $\Rr^n$ dans 
$\Rr[X]$ et qui vérifie $f(e_i) = (X+1)^i$ où les $e_i$ sont ici les vecteurs 
de la base canonique de $\Rr^n$.

\change
Nous allons voir que l'on n'a pas le choix en calculant $f(x_1,\ldots,x_n)$

\change
Un vecteur $x$ se décompose en coordonnées 

$(x_1,\ldots,x_n)$ dans la base canonique,

donc $f(x_1,\ldots,x_n)= f(x_1e_1+\cdots +x_n e_n)$ 

\change
par linéarité c'est égal à $ x_1f(e_1)+\cdots +x_nf(e_n)$.

\change
Comme les $f(e_i)$ sont imposés cela donne 

$\sum_{i=1}^n x_i(X+1)^i.$



%%%%%%%%%%%%%%%%%%%%%%%%%%%%%%%%%%%%%%%%%%%%%%%%%%%%%%%%%%%
\diapo


Soit $f : E \to F$ une application linéaire. Je rappelle que 
$f(E)$ est l'image de $f$, c'est-à-dire $\Im f = \big\{ f(x) | x \in E \big\}$. 
$\Im f$ est un sous-espace vectoriel de $F$.

\change
Si $E$ est de dimension finie, alors, lorsque l'on applique le théorème précédent on obtient que :


(1) $\Im f = f(E)$ est un espace vectoriel de dimension *finie*.

\change
(2) Si $(e_1,\ldots,e_n)$ est une base de $E$, alors
  $\Im f = \Vect \big( f(e_1),\ldots,f(e_n) \big)$.
  

La dimension de cet espace vectoriel $\Im f$ s'appelle par définition
le \defi{rang de $f$}.

\change
On le note ainsi :

Le rang est donc par définition égal à $\dim \Im f$ 

c-à-d à  $ \dim \Vect \big( f(e_1),\ldots,f(e_n) \big)$

\change
Il n'est pas toujours facile de calculer le rang d'une application linéaire, mais on a 
des inégalités immédiates :


Le rang est plus petit que la dimension de $E$. Et, si $F$ est de dimension finie 
le rang est plus que la dimension de $F$ :

On résume cela en $\rg(f) \le \min \left ( \dim E, \dim F \right).$

%%%%%%%%%%%%%%%%%%%%%%%%%%%%%%%%%%%%%%%%%%%%%%%%%%%%%%%%%%%
\diapo

Soit $f : \Rr^3 \to \Rr^2$ l'application linéaire définie
par $f(x,y,z) = (3x-4y+2z,2x-3y-z)$. Quel est le rang de $f$ ?

\change
On note $e_1,e_2,e_3$

\change
les vecteurs de la base canonique de $\Rr^3$. 

\change
Il s'agit de trouver le rang de la famille $\{v_1,v_2,v_3\}$ :
où $v_i = f(e_i)$.

On commence par $v_1 = f(e_1) = f\left(\begin{smallmatrix} 1 \\ 0 \\ 0\end{smallmatrix}\right) 
= \left(\begin{smallmatrix} 3 \\ 2 \end{smallmatrix}\right)$

\change
On fait de même pour $v_2 = f(e_2) = f\left(\begin{smallmatrix} 0 \\ 1 \\ 0\end{smallmatrix}\right) 
= \left(\begin{smallmatrix} -4 \\ -3 \end{smallmatrix}\right)$

\change
et $v_3 = f(e_3) = f\left(\begin{smallmatrix} 0 \\ 0 \\ 1 \end{smallmatrix}\right) 
= \left(\begin{smallmatrix} 2 \\ -1 \end{smallmatrix}\right)$

\change
On construit une matrice formée de ces $3$ vecteurs colonnes.
Il s'agit donc de trouver le rang de la matrice 
$$A = \begin{pmatrix}
3 & -4 & 2 \\
2 & -3 & -1 \\
\end{pmatrix}.$$
[lire en colonnes]

\change
Commençons par encadrer le rang sans faire de calculs.

Nous avons une famille de $3$ vecteurs donc $\rg f \le 3$.

\change
 Mais en fait les vecteurs $v_1,v_2,v_3$ vivent dans un espace de dimension $2$ donc 
  $\rg f \le 2$.
  
\change
$f$ n'est pas l'application linéaire nulle, par exemple $v_1$ n'est pas nul
  donc $\rg f \ge 1$.
  
\change
Ainsi le rang de $f$ vaut $1$ ou $2$.

\change
Il est facile de voir que $v_1$ et $v_2$ sont linéairement indépendants, donc
le rang est $2$.

Résumons : 
$$\rg f = \rg \big(f(e_1), f(e_2), f(e_3) \big) = \dim \Vect (v_1, v_2, v_3) = 2$$


Une remarque : il est encore plus facile de voir que le rang de la matrice $A$ est $2$ 
en notant que ses deux seules lignes ne sont pas des vecteurs colinéaires.



%%%%%%%%%%%%%%%%%%%%%%%%%%%%%%%%%%%%%%%%%%%%%%%%%%%%%%%%%%%
\diapo

Le  théorème du rang que nous allons voir est un résultat fondamental dans la théorie des 
applications linéaires en dimension finie.


On se place toujours dans la même situation : 

$f : E \to F$ est une application linéaire entre deux $\Kk$-espaces 
  vectoriels,
  
\change
l'espace vectoriel $E$ de départ est un espace vectoriel de dimension finie.
 
\change
Rappelons que le \defi{noyau} de $f$ est définie par $\Ker f=\big\{x \in E \mid f(x)=0_{F}\big\}$ ;

rappelons aussi que c'est un sous-espace vectoriel de $E$, 
et comme $E$ de dimension finie alors $\Ker f$ est aussi de dimension finie.
  
\change
L'\defi{image} de $f$ est par définition $f(E) = \big\{ f(x) \mid x \in E \big\}$ ;
 c'est un sous-espace vectoriel de l'espace d'arrivée $F$ et comme $E$ est de dimension
 finie alors $\Im f$ est de dimension finie.
 
\change
Le théorème du rang donne une relation entre la
dimension du noyau et la dimension de l'image de $f$.

Théorème : 
On reprend nos hypothèses :
$f : E \to F$ est une application linéaire avec
$E$ de dimension finie. 

Alors $\dim E =\dim \Ker f + \dim \Im f$


\change
Autrement dit : $\dim E = \dim \Ker f + \rg f$

Dans la pratique, cette formule sert à déterminer la dimension du noyau
connaissant le rang, ou bien le rang connaissant la dimension du noyau.



%%%%%%%%%%%%%%%%%%%%%%%%%%%%%%%%%%%%%%%%%%%%%%%%%%%%%%%%%%%
\diapo

Soit l'application linéaire $f$ de $\Rr^4$ dans $\Rr^3$
qui à $ (x_1,x_2,x_3,x_4) $ associe 

$ (x_1-x_2+x_3,2x_1+2x_2+6x_3+4x_4,-x_1-2x_3-x_4) $.

Calculons le rang de $f$ et la dimension du noyau de $f$.  

\change
Nous allons voir deux méthodes. Dans la première méthode
on calcule d'abord le noyau.

\change
  Dire que $ (x_1,x_2,x_3,x_4) $  appartient au noyau de $f$ signifie $f(x_1,x_2,x_3,x_4) = (0,0,0)$
  
  \change
  Par l'expression de $f$,
  les trois coordonnées de l'image doivent être nulle, ce qui est écrire
  le système linéaire suivant.
  
  \change
  On résout ce système et après calculs on trouve qu'il est équivalent 
  à celui là.
  
  \change
  On choisit $x_3$ et $x_4$ comme paramètres ce qui permet d'avoir
   
  $\Ker f = \bigg\{ (-2x_3-x_4,-x_3-x_4,x_3,x_4) \mid x_3,x_4 \in \Rr\bigg\}$
  
  \change
  On récrit cet espace comme combinaison linéaire de deux vecteurs :
  
  $  = \left\{ x_3 \left(\begin{smallmatrix}-2 \\-1 \\1 \\0 \end{smallmatrix}\right)
  + x_4 \left(\begin{smallmatrix}-1\\-1\\0\\1 \end{smallmatrix}\right)\mid x_3,x_4 \in \Rr \right\}$
  
  \change
  C'est-à-dire $\Ker f  = \Vect \left( \left(\begin{smallmatrix}-2 \\-1 \\1 \\0 \end{smallmatrix}\right),
  \left(\begin{smallmatrix}-1\\-1\\0\\1 \end{smallmatrix}\right)
  \right) $
  
  \change
  Les deux vecteurs définissant le noyau étant linéairement indépendants, 
  $\dim \Ker f = 2$. 
  
  \change
  On applique maintenant le théorème du rang pour en déduire sans calculs la dimension de l'image :
  $\dim \Im f$
  
  \change
  vaut par le théorème du rang $ = \dim \Rr^4 - \dim \Ker f$
  
  \change
  c-à-d $4-2 = 2$.
  
  Donc le rang de $f$ est $2$.


%%%%%%%%%%%%%%%%%%%%%%%%%%%%%%%%%%%%%%%%%%%%%%%%%%%%%%%%%%%
\diapo

On reprend la même fonction de $\Rr^4$ dans $\Rr^3$,

\change
et dans cette seconde méthode on calcule d'abord l'image. 

\change
On note $(e_1,e_2,e_3,e_4)$ la base canonique de $\Rr^4$.

\change
Calculons $v_i$ l'image de $e_i$ par $f$ :

On commence par $v_1 = f(e_1) = f\left(\begin{smallmatrix} 1 \\ 0 \\ 0 \\0 \end{smallmatrix}\right) 
= \left(\begin{smallmatrix} 1 \\ 2 \\ -1 \end{smallmatrix}\right)$

\change
$v_2 = f(e_2) = f\left(\begin{smallmatrix} 0 \\ 1 \\ 0 \\ 0\end{smallmatrix}\right) 
= \left(\begin{smallmatrix} -1 \\ 2 \\ 0 \end{smallmatrix}\right)$

\change
$v_3 = f(e_3) = f\left(\begin{smallmatrix} 0 \\ 0 \\ 1 \\  0\end{smallmatrix}\right)
= \left(\begin{smallmatrix} 1 \\ 6 \\ -2 \end{smallmatrix}\right)$

\change
$v_4 = f(e_4) = f\left(\begin{smallmatrix} 0 \\ 0 \\ 0 \\1 \end{smallmatrix}\right) 
= \left(\begin{smallmatrix} 0 \\ 4 \\ -1 \end{smallmatrix}\right)$

\change
Le rang de l'image de $f$, c'est le rang de ces $4$ vecteurs, c'est-à-dire 
le rang de cette matrice, où chaque colonne est l'un des vecteurs $v_i$.

\change
Pour calculer le rang de la matrice $A$, on la réduit sous une forme échelonnée 
par colonne, par la méthode du pivot de Gauss.


\change
Donc le rang de $A$ est $2$, 

\change
ainsi $\rg f  = 2$.

\change
Maintenant, par le théorème du rang, 
$\dim \Ker f = \dim \Rr^4 - \rg f = 4-2 =2$.

On trouve bien sûr le même résultat que par la première méthode.


%%%%%%%%%%%%%%%%%%%%%%%%%%%%%%%%%%%%%%%%%%%%%%%%%%%%%%%%%%%
\diapo

Soit l'application linéaire 
$$\begin{array}{rcl}
f : \Rr_n[X] &\longrightarrow&  \Rr_n[X] \\
        P(X) &\longmapsto& P''(X) 
  \end{array}$$
où $P''(X)$ est la dérivée seconde de $P(X)$.

\change
On souhaite calculer le rang et la dimension du noyau de $f$.

\change
Comme dans l'exemple précédent on va voir deux méthodes.

  On calcule d'abord le noyau :
  
  \change
  Etre dans le noyau pour un polynome $P(X)$
  signifie $ \iff f\big(P(X)\big) = 0$ ce qui équivaut par définition de la fonction $f$ à $P''(X) = 0$
  
  on intègre une fois, on obtient que $P'(X)=a$, et en intègrant une seconde fois, on a $P(X) = aX+b$
  
  où $a,b$ sont des constantes.
  
  \change
  Cela prouve que $\Ker f$ est engendré par deux polynômes : $1$ (le polynôme constant) 
  et $X$.  
  
  Donc $\dim \Ker f = 2$.
  
  \change
  Par le théorème du rang, $\rg f = \dim \Im f =$
  
  $= \dim \Rr_n[X] - \dim \Ker f = (n+1) - 2 = n-1$.
  
  \change
  Deuxième méthode. On calcule d'abord l'image :
  
  \change
  Tout d'abord  une base de l'espace de départ est $(1,X,X^2,\ldots,X^n)$.
  
  \change
  donc $\rg f$ est par définition la dimension de 
  
  $\Vect \big(f(1), f(X), \ldots, f(X^n) \big)$.
  
  \change
  Tout d'abord par dérivation, $f(1)=0$ et $f(X)=0$. 
  
  \change
  Par contre pour $k \ge 2$, $f(X^k) = k(k-1)X^{k-2}$.
  
  \change
  Voici les vecteurs dans l'image
  
  $\big\{f(X^2), f(X^3), \ldots, f(X^n) \big\} $
  
  $= \big\{2, 6X, 12X^2, \ldots, n(n-1)X^{n-2}\big\}$
  
  \change
  Comme les degrés sont échelonnés, il est clair qu'ils engendrent 
  un espace de dimension $n-1$, donc $\rg f = n-1$.
  
  \change
  Par le théorème du rang, 
  
  $\dim \Ker f = \dim \Rr_n[X] - \rg f = (n+1) - (n-1)$ 
  
  on retrouve bien $2$.


%%%%%%%%%%%%%%%%%%%%%%%%%%%%%%%%%%%%%%%%%%%%%%%%%%%%%%%%%%%
\diapo

Rappelons qu'un \defi{isomorphisme} est une application linéaire bijective.

Il n'est pas très difficile de montrer que lorsque l'on a un
isomorphisme de $E$ dans $F$ alors
les espaces vectoriels de départ et d'arrivée ont la même
dimension, c-à-d $\dim E =\dim F$.


\change
Nous allons démontrer une sorte de réciproque, qui est extrêmement utile.


Voici l'énoncé : 
Soit $f : E \to F$ une application linéaire avec $E$ et $F$ 
de dimension finie.

L'hypothèse est maintenant que $\dim E = \dim F$. Alors
les $3$ assertions suivantes sont équivalentes :

\change
(i) $f$ est bijective

(ii) $f$ est injective

(iii) $f$ est surjective

En d'autres termes, si on sait prouver que $f$ est injective alors $f$
est automatiquement bijective, donc est un isomorphisme.
De même si on sait prouver que $f$ est surjective alors $f$
est automatiquement bijective.

Mais attention à l'hypothèse, il faut bien sûr que $f$ soit une application linéaire,
mais surtout que l'espace de départ et d'arrivée aient la même dimension.


\change
La démonstration est immédiate à partir du théorème du rang. 

\change
En effet, la propriété
$f$ injective équivaut à $\Ker f = \{0\}$, 

\change
donc d'après le théorème du rang, 
$f$ est injective si et seulement si $\dim \Im f =\dim
E$ (car la dimension du noyau est nulle). 

\change
Comme $\dim E = \dim F$

alors notre assertion équivaut à  $\dim \Im f=\dim F$.

\change
Cela équivaut ainsi à $\Im f =F$, 

\change
c'est-à-dire $f$ est surjective.

%%%%%%%%%%%%%%%%%%%%%%%%%%%%%%%%%%%%%%%%%%%%%%%%%%%%%%%%%%%
\diapo

Considérons $f : \Rr^2 \to \Rr^2$ définie par 

$f(x,y) = (x-y,x+y)$.

\change
Voici une façon simple de montrer que cette application linéaire est bijective :

\change
remarquons d'abord que l'espace de départ et l'espace d'arrivée ont même dimension.

\change
Ensuite on calcule le noyau :
$(x,y) \in \Ker f$ 

\change
$\iff f(x,y)=0$

\change
c-à-d ssi le couple $(x-y,x+y)$ est égal au couple $(0,0)$.

\change
En passant en système on obtient

$\left\{
\begin{array}{rcl}
x+y & = & 0 \\
x-y & = & 0 \\
\end{array}
\right. $
% \iff 
% \left\{
% \begin{array}{rcl}
% x & = & 0 \\
% y & = & 0 \\
% \end{array}
% \right.

\change
$\iff (x,y) = (0,0)$

\change
Ainsi $\Ker f $ est réduit au vecteur nul, ce qui prouve que $f$ est injective

\change
et donc, par le théorème du rang ou plus précisément le théorème de la diapo précédente
$f$ est une application linéaire bijective, c-à-d un isomorphisme.


%%%%%%%%%%%%%%%%%%%%%%%%%%%%%%%%%%%%%%%%%%%%%%%%%%%%%%%%%%%
\diapo

Le théorème du rang est sûrement le résultat
le plus important concernant les applications
linéaires en dimension finie. Passez donc du temps
sur cette partie et ses exercices.


\end{document}
