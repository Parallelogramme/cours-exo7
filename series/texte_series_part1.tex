
%%%%%%%%%%%%%%%%%% PREAMBULE %%%%%%%%%%%%%%%%%%


\documentclass[12pt]{article}

\usepackage{amsfonts,amsmath,amssymb,amsthm}
\usepackage[utf8]{inputenc}
\usepackage[T1]{fontenc}
\usepackage[francais]{babel}


% packages
\usepackage{amsfonts,amsmath,amssymb,amsthm}
\usepackage[utf8]{inputenc}
\usepackage[T1]{fontenc}
%\usepackage{lmodern}

\usepackage[francais]{babel}
\usepackage{fancybox}
\usepackage{graphicx}

\usepackage{float}

%\usepackage[usenames, x11names]{xcolor}
\usepackage{tikz}
\usepackage{datetime}

\usepackage{mathptmx}
%\usepackage{fouriernc}
%\usepackage{newcent}
\usepackage[mathcal,mathbf]{euler}

%\usepackage{palatino}
%\usepackage{newcent}


% Commande spéciale prompteur

%\usepackage{mathptmx}
%\usepackage[mathcal,mathbf]{euler}
%\usepackage{mathpple,multido}

\usepackage[a4paper]{geometry}
\geometry{top=2cm, bottom=2cm, left=1cm, right=1cm, marginparsep=1cm}

\newcommand{\change}{{\color{red}\rule{\textwidth}{1mm}\\}}

\newcounter{mydiapo}

\newcommand{\diapo}{\newpage
\hfill {\normalsize  Diapo \themydiapo \quad \texttt{[\jobname]}} \\
\stepcounter{mydiapo}}


%%%%%%% COULEURS %%%%%%%%%%

% Pour blanc sur noir :
%\pagecolor[rgb]{0.5,0.5,0.5}
% \pagecolor[rgb]{0,0,0}
% \color[rgb]{1,1,1}



%\DeclareFixedFont{\myfont}{U}{cmss}{bx}{n}{18pt}
\newcommand{\debuttexte}{
%%%%%%%%%%%%% FONTES %%%%%%%%%%%%%
\renewcommand{\baselinestretch}{1.5}
\usefont{U}{cmss}{bx}{n}
\bfseries

% Taille normale : commenter le reste !
%Taille Arnaud
%\fontsize{19}{19}\selectfont

% Taille Barbara
%\fontsize{21}{22}\selectfont

%Taille François
\fontsize{25}{30}\selectfont

%Taille Pascal
%\fontsize{25}{30}\selectfont

%Taille Laura
%\fontsize{30}{35}\selectfont


%\myfont
%\usefont{U}{cmss}{bx}{n}

%\Huge
%\addtolength{\parskip}{\baselineskip}
}


% \usepackage{hyperref}
% \hypersetup{colorlinks=true, linkcolor=blue, urlcolor=blue,
% pdftitle={Exo7 - Exercices de mathématiques}, pdfauthor={Exo7}}


%section
% \usepackage{sectsty}
% \allsectionsfont{\bf}
%\sectionfont{\color{Tomato3}\upshape\selectfont}
%\subsectionfont{\color{Tomato4}\upshape\selectfont}

%----- Ensembles : entiers, reels, complexes -----
\newcommand{\Nn}{\mathbb{N}} \newcommand{\N}{\mathbb{N}}
\newcommand{\Zz}{\mathbb{Z}} \newcommand{\Z}{\mathbb{Z}}
\newcommand{\Qq}{\mathbb{Q}} \newcommand{\Q}{\mathbb{Q}}
\newcommand{\Rr}{\mathbb{R}} \newcommand{\R}{\mathbb{R}}
\newcommand{\Cc}{\mathbb{C}} 
\newcommand{\Kk}{\mathbb{K}} \newcommand{\K}{\mathbb{K}}

%----- Modifications de symboles -----
\renewcommand{\epsilon}{\varepsilon}
\renewcommand{\Re}{\mathop{\text{Re}}\nolimits}
\renewcommand{\Im}{\mathop{\text{Im}}\nolimits}
%\newcommand{\llbracket}{\left[\kern-0.15em\left[}
%\newcommand{\rrbracket}{\right]\kern-0.15em\right]}

\renewcommand{\ge}{\geqslant}
\renewcommand{\geq}{\geqslant}
\renewcommand{\le}{\leqslant}
\renewcommand{\leq}{\leqslant}

%----- Fonctions usuelles -----
\newcommand{\ch}{\mathop{\mathrm{ch}}\nolimits}
\newcommand{\sh}{\mathop{\mathrm{sh}}\nolimits}
\renewcommand{\tanh}{\mathop{\mathrm{th}}\nolimits}
\newcommand{\cotan}{\mathop{\mathrm{cotan}}\nolimits}
\newcommand{\Arcsin}{\mathop{\mathrm{Arcsin}}\nolimits}
\newcommand{\Arccos}{\mathop{\mathrm{Arccos}}\nolimits}
\newcommand{\Arctan}{\mathop{\mathrm{Arctan}}\nolimits}
\newcommand{\Argsh}{\mathop{\mathrm{Argsh}}\nolimits}
\newcommand{\Argch}{\mathop{\mathrm{Argch}}\nolimits}
\newcommand{\Argth}{\mathop{\mathrm{Argth}}\nolimits}
\newcommand{\pgcd}{\mathop{\mathrm{pgcd}}\nolimits} 

\newcommand{\Card}{\mathop{\text{Card}}\nolimits}
\newcommand{\Ker}{\mathop{\text{Ker}}\nolimits}
\newcommand{\id}{\mathop{\text{id}}\nolimits}
\newcommand{\ii}{\mathrm{i}}
\newcommand{\dd}{\mathrm{d}}
\newcommand{\Vect}{\mathop{\text{Vect}}\nolimits}
\newcommand{\Mat}{\mathop{\mathrm{Mat}}\nolimits}
\newcommand{\rg}{\mathop{\text{rg}}\nolimits}
\newcommand{\tr}{\mathop{\text{tr}}\nolimits}
\newcommand{\ppcm}{\mathop{\text{ppcm}}\nolimits}

%----- Structure des exercices ------

\newtheoremstyle{styleexo}% name
{2ex}% Space above
{3ex}% Space below
{}% Body font
{}% Indent amount 1
{\bfseries} % Theorem head font
{}% Punctuation after theorem head
{\newline}% Space after theorem head 2
{}% Theorem head spec (can be left empty, meaning ‘normal’)

%\theoremstyle{styleexo}
\newtheorem{exo}{Exercice}
\newtheorem{ind}{Indications}
\newtheorem{cor}{Correction}


\newcommand{\exercice}[1]{} \newcommand{\finexercice}{}
%\newcommand{\exercice}[1]{{\tiny\texttt{#1}}\vspace{-2ex}} % pour afficher le numero absolu, l'auteur...
\newcommand{\enonce}{\begin{exo}} \newcommand{\finenonce}{\end{exo}}
\newcommand{\indication}{\begin{ind}} \newcommand{\finindication}{\end{ind}}
\newcommand{\correction}{\begin{cor}} \newcommand{\fincorrection}{\end{cor}}

\newcommand{\noindication}{\stepcounter{ind}}
\newcommand{\nocorrection}{\stepcounter{cor}}

\newcommand{\fiche}[1]{} \newcommand{\finfiche}{}
\newcommand{\titre}[1]{\centerline{\large \bf #1}}
\newcommand{\addcommand}[1]{}
\newcommand{\video}[1]{}

% Marge
\newcommand{\mymargin}[1]{\marginpar{{\small #1}}}



%----- Presentation ------
\setlength{\parindent}{0cm}

%\newcommand{\ExoSept}{\href{http://exo7.emath.fr}{\textbf{\textsf{Exo7}}}}

\definecolor{myred}{rgb}{0.93,0.26,0}
\definecolor{myorange}{rgb}{0.97,0.58,0}
\definecolor{myyellow}{rgb}{1,0.86,0}

\newcommand{\LogoExoSept}[1]{  % input : echelle
{\usefont{U}{cmss}{bx}{n}
\begin{tikzpicture}[scale=0.1*#1,transform shape]
  \fill[color=myorange] (0,0)--(4,0)--(4,-4)--(0,-4)--cycle;
  \fill[color=myred] (0,0)--(0,3)--(-3,3)--(-3,0)--cycle;
  \fill[color=myyellow] (4,0)--(7,4)--(3,7)--(0,3)--cycle;
  \node[scale=5] at (3.5,3.5) {Exo7};
\end{tikzpicture}}
}



\theoremstyle{definition}
%\newtheorem{proposition}{Proposition}
%\newtheorem{exemple}{Exemple}
%\newtheorem{theoreme}{Théorème}
\newtheorem{lemme}{Lemme}
\newtheorem{corollaire}{Corollaire}
%\newtheorem*{remarque*}{Remarque}
%\newtheorem*{miniexercice}{Mini-exercices}
%\newtheorem{definition}{Définition}




%definition d'un terme
\newcommand{\defi}[1]{{\color{myorange}\textbf{\emph{#1}}}}
\newcommand{\evidence}[1]{{\color{blue}\textbf{\emph{#1}}}}



 %----- Commandes divers ------

\newcommand{\codeinline}[1]{\texttt{#1}}

%%%%%%%%%%%%%%%%%%%%%%%%%%%%%%%%%%%%%%%%%%%%%%%%%%%%%%%%%%%%%
%%%%%%%%%%%%%%%%%%%%%%%%%%%%%%%%%%%%%%%%%%%%%%%%%%%%%%%%%%%%%


\begin{document}

\debuttexte


%%%%%%%%%%%%%%%%%%%%%%%%%%%%%%%%%%%%%%%%%%%%%%%%%%%%%%%%%%%
\diapo

Voici la première leçon du chapitre consacré aux séries. Les séries sont un cas particulier de suite.

\change
Nous allons y voir beaucoup de choses :

\change
après quelques définitions générales,

\change
nous verrons l'exemple des séries géométriques,

\change
et le cas des séries convergentes.

\change
Puis nous verrons les liens entre suites et séries ;

\change
une première propriété fondamentale : le terme d'une série convergente tend vers 0 ;

on terminera par d'autres propriétés essentielles.

 %%%%%%%%%%%%%%%%%%%%%%%%%%%%%%%%%%%%%%%%%%%%%%%%%%%%%%%%%%%
\diapo

Dans ce chapitre nous allons nous intéresser à des sommes ayant une infinité de termes.

\change
Par exemple que peut bien valoir la somme infinie suivante :
\[1+\frac{1}{2} +\frac{1}{4} +\frac{1}{8}+\cdots \ \ = \ \ ? \]

\change
Cette question a été popularisée sous le nom du \evidence{paradoxe de Zénon}. 

\change
On tire une flèche à $2$~mètres d'une cible. 
Elle met un certain laps de temps pour parcourir
la moitié de la distance, à savoir un mètre. Puis il lui faut encore du 
temps pour parcourir la moitié de la distance restante, et de nouveau un 
certain temps pour la moitié de la distance encore restante. 

\change
On ajoute ainsi une infinité de durées non nulles, et Zénon en conclut 
que la flèche n'atteint jamais sa cible ! Zénon ne concevait pas qu'une infinité 
de distances finies puisse être parcourue en un temps fini.

\change
Et pourtant nous allons voir dans ce chapitre que la somme d'une infinité de termes
peut être une valeur finie.


%%%%%%%%%%%%%%%%%%%%%%%%%%%%%%%%%%%%%%%%%%%%%%%%%%%%%%%%%%%
\diapo

Commençons tout de suite par définir ce qu'est une série. 
Soit $(u_k)_{k \ge 0}$ une suite de nombres réels (ou de nombres complexes).

\change
On pose
$$S_n=u_0+u_1+u_2+\cdots+ u_n,$$

\change
ce que l'on note également $\sum_{k=0}^n u_k.$

\change
La suite $(S_n)_{n \ge 0}$ s'appelle la série de terme général 
$u_k$.

\change
Cette série est notée par la somme infinie $\sum_{k \ge 0} u_k $.

\change
La suite $(S_n)$ s'appelle aussi la suite des sommes partielles.

\change
Voici un exemple absolument fondamental. 
Fixons $q$ un nombre réel ou complexe et définissons la suite $(u_k)_{k \ge 0}$ par
$u_k = q^k$ ; c'est une suite géométrique.

\change
La série géométrique $\sum_{k \ge 0} q^k$ est la suite des sommes partielles :

\change
$S_0 = 1$

\change
$S_1 = 1 + q$

\change
$S_2 = 1+q+q^2$

\change
etc

$S_n = 1+q+q^2+\cdots+q^n$ 

et ainsi de suite.

%%%%%%%%%%%%%%%%%%%%%%%%%%%%%%%%%%%%%%%%%%%%%%%%%%%%%%%%%%%
\diapo


Si la suite $(S_n)_{n \ge 0}$ admet une limite finie dans $\Rr$ (ou dans $\Cc$),
on note
$S = \sum_{k=0}^{+\infty} u_k$ pour cette $\lim_{n\to+\infty} S_n.$

\change
On dit alors que $\displaystyle S= \sum_{k=0}^{+\infty} u_k$ la somme de la série $\sum_{k \ge 0} u_k$,

\change
et on dira que la série est convergente.

\change
Sinon, on dit qu'elle est divergente. 

\change
On peut noter une série de différentes façons, et bien sûr avec différents symboles pour l'indice : 

\change
Pour notre part, on fera la distinction entre une série quelconque $\displaystyle \sum_{k \ge 0} u_k$,
et on réservera la notation $\displaystyle \sum_{k=0}^{+\infty} u_k$, 
à une série *convergente* ou à sa somme.

%%%%%%%%%%%%%%%%%%%%%%%%%%%%%%%%%%%%%%%%%%%%%%%%%%%%%%%%%%%
\diapo

Passons au cas des séries géométriques. On a la proposition suivante. 

Soit $q$ un nombre complexe.

\change
La série géométrique $\sum_{k \ge 0} q^k$ est convergente si et seulement 
si le module de $q$ est strictement inférieur à $1$. 

\change
De plus, lorsque la série converge, on connaît la valeur de sa somme : on a alors 

$\displaystyle \sum_{k=0}^{+\infty} q^k =1+q+q^2+q^3+\cdots= \frac{1}{1-q}$.

Il est fondamental de bien comprendre connaître les séries géométriques, qui sont des séries de références, 
ainsi que que la valeur de la somme.


%%%%%%%%%%%%%%%%%%%%%%%%%%%%%%%%%%%%%%%%%%%%%%%%%%%%%%%%%%%
\diapo

Pour démontrer ce résultat, considérons les sommes partielles :

$S_n=1+q+q^2+q^3+\cdots+q^n.$

\change
\'Ecartons tout de suite le cas $q=1$, pour lequel cette 
somme est $1+1+1...$ donc $S_n = n+1$. 

\change
Et donc dans ce cas $S_n \to +\infty$, 

\change
et la série diverge.

\change  
Supposons à présent $q \neq 1$ et multiplions $S_n$ par $1-q$ :

\change
$(1-q)S_n$ est égal à $S_n$, c-à-d $(1+q+q^2+\cdots+q^n)$ 
moins $qS_n$ qui vaut $(q+q^2+q^3+\cdots+q^{n+1})$.

\change
On obtient après simplification $1-q^{n+1}$.

\change
Donc
$\displaystyle S_n  = \frac{1-q^{n+1}}{1-q}$

\change
Si $|q|<1$, 

\change
alors $q^n \to 0$, donc $q^{n+1}$ également 

\change
et ainsi $S_n \to \frac{1}{1-q}$.

\change
Dans ce cas la série $\sum_{k \ge 0} q^k$ converge.

\change
Si $|q| \ge 1$, 

\change
alors la suite $(q^n)$ n'a pas de limite finie : elle peut ou bien tendre vers $+\infty$, par exemple si $q=2$ ; ou bien être divergente, par exemple si $q=-1$.

\change
Ainsi si $|q| \ge 1$, $(S_n)$ n'a pas de limite finie.

\change
Donc la série $\sum_{k \ge 0} q^k$ diverge.


%%%%%%%%%%%%%%%%%%%%%%%%%%%%%%%%%%%%%%%%%%%%%%%%%%%%%%%%%%%
\diapo

Voyons quelques exemples d'utilisation de la proposition précédente.

\change
Tout d'abord, la série géométrique de raison $q=\frac 12$

\change
c-à-d $\displaystyle \sum_{k=0}^{+\infty} \frac{1}{2^k}$ 

\change
vaut d'après la formule précédente
$\frac{1}{1-\frac12} $

\change
qui fait $ 2$.

\change
Cela résout le paradoxe de Zénon : la flèche arrive bien jusqu'au mur !

\change
Calculons la série géométrique de raison $q=\frac 13$, avec premier terme $\frac{1}{3^3}$,

\change
c-à-d $\displaystyle \sum_{k=3}^{+\infty} \frac{1}{3^k}$.

On se ramène à la série géométrique commençant à $k=0$ en ajoutant et retranchant les premiers termes.

\change
La somme que l'on souhaite calculée vaut donc 

$ \displaystyle\sum_{k=0}^{+\infty} \frac{1}{3^k} \  -  1 - \frac 13-\frac 1{3^2} $

\change
c-à-d, par la même formule $\frac{1}{1-\frac13} - \frac{13}{9}$

\change
ce qui fait $ \frac32-\frac{13}{9} = \frac{1}{18}$.

Le fait de calculer la somme d'une série à partir de $k=0$ est
purement conventionnel. 

\change
On peut toujours effectuer un changement
d'indice pour se ramener à une somme à partir de $0$. 

\change
Ainsi, une autre façon pour calculer la même série que précédemment
est de faire le changement d'indice $n=k-3$ (et donc $k=n+3$):

\change
$ \displaystyle\sum_{k=3}^{+\infty} \frac{1}{3^k} $ est donc égale à $  \displaystyle\sum_{n=0}^{+\infty} \frac{1}{3^{n+3}}$

\change
$
=  \displaystyle\sum_{n=0}^{+\infty} \frac{1}{3^3}\frac{1}{3^n}$

\change
En factorisant par $\frac{1}{3^3}$ on obtient

$ \frac{1}{3^3} \displaystyle\sum_{n=0}^{+\infty} \frac{1}{3^n}$

\change
et donc
$ \dfrac{1}{27}\dfrac{1}{1-\frac13}= \dfrac{1}{18}$.

\change
Utilisons enfin la proposition précédente pour calculer cette série.

\change
En effet, on remarque que 

$\displaystyle \sum_{k=0} ^{+\infty} (-1)^k \left(\frac{1}{2}\right)^{2k} =\sum_{k=0} ^{+\infty} \left(-\frac{1}{4}\right)^k$

\change
et on obtient donc $\dfrac{1}{1-\dfrac{-1}{4}}$

\change
$=\dfrac{4}{5}.$
 


%%%%%%%%%%%%%%%%%%%%%%%%%%%%%%%%%%%%%%%%%%%%%%%%%%%%%%%%%%%
\diapo

La convergence d'une série ne dépend pas de ses premiers termes : changer un nombre fini de termes d'une série ne change pas sa nature, convergente ou divergente. Par contre, si elle est convergente, sa somme est évidemment modifiée.

Une façon pratique d'étudier la convergence d'une série est d'étudier son reste :

\change
le reste d'ordre~$n$ d'une série convergente $\sum_{k = 0}^{+\infty} u_k$ est :

$R_n = u_{n+1}+u_{n+2}+\cdots = \sum_{k=n+1}^{+\infty} u_k$

\change
On a alors la proposition : si une série est convergente, alors 
$S=S_n+R_n$ (pour tout $n\ge0$) 

\change
De plus $\lim_{n\to+\infty} R_n=0$.

%\change
%La démonstration est immédiate : on scinde la somme $S$
%
%\change
%en distinguant les entiers $k\leq n$ des entiers $k\geq n+1$ et on obtient que $S=S_n+R_n$.
%
%\change
%Donc $R_n = S-S_n$ et tend vers $0$.


%%%%%%%%%%%%%%%%%%%%%%%%%%%%%%%%%%%%%%%%%%%%%%%%%%%%%%%%%%%
\diapo

Il n'y a pas de différence entre l'étude des suites et des séries. 
On passe de l'une à l'autre très facilement.

Tout d'abord rappelons qu'à une série $\sum_{k \ge 0} u_k$, on associe 
la somme partielle $S_n=\sum_{k=0}^n u_k$ et que par définition
la série converge si la suite $(S_n)_{n\ge0}$ converge.

\change
Réciproquement si on veut étudier une suite $(a_k)_{k\ge0}$ on peut utiliser le résultat suivant :

\change
Une somme télescopique est une série de la forme

$\sum_{k\ge0} (a_{k+1}-a_k).$

\change
Cette série est convergente si et seulement si la suite $(a_k)_{k\ge0}$ est convergente.

\change
Dans ce cas on a :

$\displaystyle\sum_{k=0}^{+\infty} (a_{k+1}-a_k) = \lim a_k - a_0.$

\change
La démonstration est immédiate, et repose sur le fait qu'en développant les sommes partielles, quasiment tous les termes se simplifient deux à deux, et on obtient

$S_n=a_{n+1}-a_0 $.


%%%%%%%%%%%%%%%%%%%%%%%%%%%%%%%%%%%%%%%%%%%%%%%%%%%%%%%%%%%
\diapo

Voici un exemple très important pour la suite.

La série $$\sum_{k=0}^{+\infty} \frac{1}{(k+1)(k+2)}=\frac{1}{1\cdot 2}+\frac{1}{2\cdot 3}+
 \frac{1}{3\cdot 4}+\cdots$$
est convergente et sa somme vaut $1$. 

\change
En effet, elle peut être écrite comme somme télescopique, et plus précisément la somme partielle vérifie :
$$S_n=\sum_{k=0}^n \frac{1}{(k+1)(k+2)}= \sum_{k=0}^n \left(\frac{1}{k+1}-\frac{1}{k+2}\right)$$

Vérifiez bien que $\frac{1}{k+1}-\frac{1}{k+2} = \frac{1}{(k+1)(k+2)}$.

\change
C'est une somme téléescopique comme dans le résultat précédent, 
presque tous les termes se simplifient et $S_n$ vaut donc

$1- \frac{1}{n+2}$
qui tend vers $1$ lorsque $n\to+\infty$.

\change
Par changement d'indice, on a aussi que les séries 
$\sum_{k=1}^{+\infty} \frac{1}{k(k+1)}$ et 
$\sum_{k=2}^{+\infty} \frac{1}{k(k-1)}$ sont convergentes et de même somme $1$.


%%%%%%%%%%%%%%%%%%%%%%%%%%%%%%%%%%%%%%%%%%%%%%%%%%%%%%%%%%%
\diapo

On a le théorème très important suivant :
Si la série $\sum_{k\ge0} u_k$ converge, 
alors la suite des termes généraux $(u_k)_{k \ge 0}$ tend vers $0$.

\change
La démonstration de ce résultat est immédiate.  Le point clé est que l'on retrouve le terme général $u_n$ à partir des sommes partielles $S_n$ par la formule
$$u_n = S_n - S_{n-1}.$$

\change
La contraposée de ce résultat est souvent utilisée : 

Une série dont le terme général ne tend pas vers $0$ ne peut pas
converger : elle est donc divergente.

\change
Par exemple les séries $\sum_{k \ge 1} (1+\frac{1}{k})$ et $\sum_{k \ge 1} k^2$ sont 
divergentes, car le termes général ne tend pas vers $0$.

\change
C'est une erreur classique de penser que si le terme
général tend vers $0$ alors la série converge. C'est faux, comme on le verra
bientôt. Comprenez bien l'énoncé qui dit que ``si la série converge alors
le terme général tend vers $0$'' mais la réciproque est fausse en général.

% Plus intéressant, la série $\sum u_k$ de terme général 
% $$
% u_k = \left\{
% \begin{array}{ll}
% 1&\text{ si } k=2^\ell \quad \text{ pour un certain } \ell \ge 0 \\
% 0&\text{ sinon }
% \end{array}
% \right.
% $$
% diverge. En effet, même si les termes valant $1$ sont très rares, il y en a quand
% même une infinité !

%%%%%%%%%%%%%%%%%%%%%%%%%%%%%%%%%%%%%%%%%%%%%%%%%%%%%%%%%%%
\diapo

On a le résultat suivant sur la linéarité des séries.

Soient $\sum a_k$ et $\sum b_k$ deux séries convergentes de sommes respectives $A$ et $B$, 

et soient $\lambda, \mu \in \Rr$ (ou $\Cc$).

\change
Alors la série $\sum (\lambda a_k+\mu b_k)$ est convergente et de somme $\lambda A+\mu B$. 
 
\change
On a donc
$$\sum_{k=0}^{+\infty} (\lambda a_k+\mu b_k) = 
 \lambda \sum_{k=0}^{+\infty} a_k+ \mu \sum_{k=0}^{+\infty} b_k.$$

\change
Par exemple la série

$
\displaystyle\sum_{k=0}^{+\infty} \left(\frac{1}{2^k}+\frac{5}{3^k}\right)
$

\change
peut être décomposée en la somme de deux série géométriques

$ 
\displaystyle\sum_{k=0}^{+\infty} \frac{1}{2^k}+
5\sum_{k=0}^{+\infty} \frac{1}{3^k}$

\change
ce qui donne

$
\dfrac{1}{1-\frac{1}{2}}+5\dfrac{1}{1-\frac{1}{3}} 
$

\change
c-à-d
$
\dfrac{19}{2}\;.
$

% \change
% Comme application, la convergence d'une série à termes complexes équivaut à la convergence des parties réelle et imaginaire :
%  
% Soit $(u_k)_{k\ge0}$ une suite de nombres complexes. Pour tout $k$, notons
% $u_k = a_k + \ii b_k$, avec $a_k$ la partie réelle de $u_k$ et $b_k$ la partie imaginaire. 
% 
% \change
% La série $\sum u_k$ converge si et seulement si les deux séries 
% $\sum a_k$ et $\sum b_k$ convergent. 
% 
% \change
% Si c'est le cas, on a :
% $$
% \sum_{k=0}^{+\infty} u_k = 
% \sum_{k=0}^{+\infty} a_k + \ii \sum_{k=0}^{+\infty} b_k\;. 
% $$

%%%%%%%%%%%%%%%%%%%%%%%%%%%%%%%%%%%%%%%%%%%%%%%%%%%%%%%%%%%
%\diapo
%
%Considérons par exemple la série géométrique $\sum_{k\ge0} r^k$, où
%$r = \rho e^{\ii\theta}$ est un complexe de module $\rho<1$ et d'argument $\theta$.
%
%\change
%Comme le module de $r$ est strictement inférieur à $1$, alors la série converge et 
%$$\sum_{k=0}^{+\infty} r^k = \frac{1}{1-r}.$$
%
%\change
%D'autre part, $r^k = \rho^k e^{\ii k\theta}$ par la formule de Moivre. 
%
%\change
%Les parties réelle et imaginaire de $r^k$ sont
%$$
%a_k=\rho^k\cos(k\theta)
%\quad\text{ et }\quad 
%b_k=\rho^k\sin(k\theta)\;.
%$$
%
%\change
%On déduit de la proposition précédente que :
%$$
%\sum_{k=0}^{+\infty} a_k = \Re \left(\sum_{k=0}^{+\infty} r^k\right) = \Re \left(\frac{1}{1-r}\right) $$
%
%\change
%et
%$$
%\sum_{k=0}^{+\infty} b_k = \Im \left(\frac{1}{1-r}\right)\;. 
%$$
%
%\change
%Le calcul donne :
%$$
%\sum_{k=0}^{+\infty} \rho^k\cos(k\theta) = 
%\frac{1-\rho\cos\theta}{1+\rho^2-2\rho\cos\theta} 
%$$
%
%\change
%et 
%$$
%\sum_{k=0}^{+\infty} \rho^k\sin(k\theta) = 
%\frac{\rho\sin\theta}{1+\rho^2-2\rho\cos\theta}
%\;. 
%$$  

%%%%%%%%%%%%%%%%%%%%%%%%%%%%%%%%%%%%%%%%%%%%%%%%%%%%%%%%%%%
\diapo

Commençons par rappeler  le critère de Cauchy pour les suites de nombres réels (ou complexes).

Une suite $(s_n)$ converge si et seulement si elle est une suite de Cauchy, c'est-à-dire :

\change
$$\forall \epsilon>0\quad \exists n_0\in\Nn \quad \forall m,n \ge n_0 \qquad |s_n-s_m|<\epsilon$$


\change
Lorsqu'on applique ce critère à la suite des sommes partielles,
cela nous donne le théorème suivant pour les séries  : 

Une série  $\displaystyle\sum_{k\geq0} u_k$ converge si et seulement si  
$$\forall \epsilon>0 \quad \exists n_0\in\Nn \quad \forall m,n\ge n_0 \qquad \big|u_{n}+\cdots +u_m\big| < \epsilon \; .$$

\change
On le formule aussi de la façon suivante :
$$ \forall \epsilon>0 \quad \exists n_0\in\Nn \quad
\forall m,n \ge n_0 \qquad  \left| \sum_{k=n}^m u_k\right| <\epsilon$$

\change
ou encore
$$ \forall \epsilon>0 \quad \exists n_0\in\Nn \quad
\forall n\ge n_0 \quad \forall p\in\Nn 
\qquad  \big|u_{n}+\cdots +u_{n+p}\big|<\epsilon$$


%%%%%%%%%%%%%%%%%%%%%%%%%%%%%%%%%%%%%%%%%%%%%%%%%%%%%%%%%%%
\diapo

Nous allons utiliser le critère de Cauchy pour montrer que la série harmonique 

$\displaystyle \sum_{k \ge 1} \frac{1}{k} = 1 + \frac12+\frac13+\frac14+\cdots$ est une série divergente.

\change
Retenez bien cet exemple !
La série  $\sum u_k$ diverge, bien que son terme général 
$u_k = \frac{1}{k} $ tende vers $0$.

Ce n'est donc pas parce que le terme général tend vers $0$, qu'un série converge.

\change
Montrons que cette série est divergente. Sa somme partielle est $S_n = \sum_{k=1}^{n} \frac{1}{k}$.

Calculons la différence de deux sommes partielles, afin de conserver 
les termes entre $n+1$ (qui joue le rôle de $n$)
et $2n$ (qui joue le rôle de $m$) :

\change
$$
S_{2n}-S_{n} = \frac{1}{n+1}+\cdots+\frac{1}{2n}$$

il y a $n$ terme et chacun est plus grand que $1/(2n)$
donc 

$$S_{2n}-S_{n} =\ge
\frac{n}{2n}=\frac{1}{2} 
$$

\change
La suite des sommes partielles n'est pas de Cauchy (car $\frac12$ 
n'est pas inférieur à $\epsilon = \frac14$ par exemple), 
donc la série ne converge pas.

%%%%%%%%%%%%%%%%%%%%%%%%%%%%%%%%%%%%%%%%%%%%%%%%%%%%%%%%%%%
\diapo

On termine par une étude plus poussée de la série harmonique.

Pour la série harmonique et sa somme partielle
$\displaystyle S_n = \sum_{k = 1}^n \frac{1}{k}$, 

on a $\lim S_n=+\infty.$

\change
Soit $M>0$. On choisit $m\in \Nn$ tel que $m\ge 2M$. 

\change
Alors pour $n\ge 2^m$ on a:

$S_n  =  1+\frac{1}{2}+\frac{1}{3}+\cdots+\frac{1}{2^m}+\cdots+ \frac{1}{n}$

\change
$\ge 1+\frac{1}{2}+\frac{1}{3}+\cdots+ \frac{1}{2^m}$

On "oublie" les derniers termes, qui sont positifs.


\change
L'astuce consiste à regrouper les termes. On trouve donc que $S_n$ est supérieur à 

$ 1 + \frac{1}{2}+ \left(\frac{1}{3}+\frac{1}{4}\right) 
  +\left(\frac{1}{5}+\frac{1}{6}+\frac{1}{7}+ \frac{1}{8}\right) +\left(\frac{1}{9}+\cdots +\frac{1}{16}\right)+  \cdots
+\left(\frac{1}{2^{m-1}+1}+\cdots + \frac{1}{2^m}\right)$

\change
Entre chaque parenthèses il y a successivement $2$ termes,

\change 
$4$ termes, ...

\change
jusqu'à $2^m-(2^{m-1}+1)+1= 2^m-2^{m-1}=2^{m-1} $ termes.

\change
Alors $S_n$ est supérieur à 

$1+\frac{1}{2}+2 \frac{1}{4}+ 4\;\frac{1}{8}+ 8 \frac{1}{16}+\cdots +
2^{m-1} \frac{1}{2^m} $

\change
$= 1+m \frac{1}{2}\ge M  $

\change
Ainsi pour tout $M>0$ il existe $n_0 \ge 0$ tel que, pour tout $n \ge n_0$, on ait $S_n \ge M$.

Ainsi
$(S_n)$ tend vers $+\infty$. Cela redémontre bien sûr que la série harmonique diverge.

%%%%%%%%%%%%%%%%%%%%%%%%%%%%%%%%%%%%%%%%%%%%%%%%%%%%%%%%%%%
\diapo

Avant de vous lancez dans la suite du cours, prenez bien le temps de comprendre
et d'apprendre toutes les notions de cette séquence.


\end{document}
