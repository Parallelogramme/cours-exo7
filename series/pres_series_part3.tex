
%%%%%%%%%%%%%%%%%% PREAMBULE %%%%%%%%%%%%%%%%%%

\documentclass[aspectratio=169,utf8]{beamer}
%\documentclass[aspectratio=169,handout]{beamer}

\usetheme{Boadilla}
%\usecolortheme{seahorse}
\usecolortheme[RGB={245,66,24}]{structure}
\useoutertheme{infolines}

% packages
\usepackage{amsfonts,amsmath,amssymb,amsthm}
\usepackage[utf8]{inputenc}
\usepackage[T1]{fontenc}
\usepackage{lmodern}

\usepackage[francais]{babel}
\usepackage{fancybox}
\usepackage{graphicx}

\usepackage{float}
\usepackage{xfrac}

%\usepackage[usenames, x11names]{xcolor}
\usepackage{tikz}
\usepackage{pgfplots}
\usepackage{datetime}



%-----  Package unités -----
\usepackage{siunitx}
\sisetup{locale = FR,detect-all,per-mode = symbol}

%\usepackage{mathptmx}
%\usepackage{fouriernc}
%\usepackage{newcent}
%\usepackage[mathcal,mathbf]{euler}

%\usepackage{palatino}
%\usepackage{newcent}
% \usepackage[mathcal,mathbf]{euler}



% \usepackage{hyperref}
% \hypersetup{colorlinks=true, linkcolor=blue, urlcolor=blue,
% pdftitle={Exo7 - Exercices de mathématiques}, pdfauthor={Exo7}}


%section
% \usepackage{sectsty}
% \allsectionsfont{\bf}
%\sectionfont{\color{Tomato3}\upshape\selectfont}
%\subsectionfont{\color{Tomato4}\upshape\selectfont}

%----- Ensembles : entiers, reels, complexes -----
\newcommand{\Nn}{\mathbb{N}} \newcommand{\N}{\mathbb{N}}
\newcommand{\Zz}{\mathbb{Z}} \newcommand{\Z}{\mathbb{Z}}
\newcommand{\Qq}{\mathbb{Q}} \newcommand{\Q}{\mathbb{Q}}
\newcommand{\Rr}{\mathbb{R}} \newcommand{\R}{\mathbb{R}}
\newcommand{\Cc}{\mathbb{C}} 
\newcommand{\Kk}{\mathbb{K}} \newcommand{\K}{\mathbb{K}}

%----- Modifications de symboles -----
\renewcommand{\epsilon}{\varepsilon}
\renewcommand{\Re}{\mathop{\text{Re}}\nolimits}
\renewcommand{\Im}{\mathop{\text{Im}}\nolimits}
%\newcommand{\llbracket}{\left[\kern-0.15em\left[}
%\newcommand{\rrbracket}{\right]\kern-0.15em\right]}

\renewcommand{\ge}{\geqslant}
\renewcommand{\geq}{\geqslant}
\renewcommand{\le}{\leqslant}
\renewcommand{\leq}{\leqslant}
\renewcommand{\epsilon}{\varepsilon}

%----- Fonctions usuelles -----
\newcommand{\ch}{\mathop{\text{ch}}\nolimits}
\newcommand{\sh}{\mathop{\text{sh}}\nolimits}
\renewcommand{\tanh}{\mathop{\text{th}}\nolimits}
\newcommand{\cotan}{\mathop{\text{cotan}}\nolimits}
\newcommand{\Arcsin}{\mathop{\text{arcsin}}\nolimits}
\newcommand{\Arccos}{\mathop{\text{arccos}}\nolimits}
\newcommand{\Arctan}{\mathop{\text{arctan}}\nolimits}
\newcommand{\Argsh}{\mathop{\text{argsh}}\nolimits}
\newcommand{\Argch}{\mathop{\text{argch}}\nolimits}
\newcommand{\Argth}{\mathop{\text{argth}}\nolimits}
\newcommand{\pgcd}{\mathop{\text{pgcd}}\nolimits} 


%----- Commandes divers ------
\newcommand{\ii}{\mathrm{i}}
\newcommand{\dd}{\text{d}}
\newcommand{\id}{\mathop{\text{id}}\nolimits}
\newcommand{\Ker}{\mathop{\text{Ker}}\nolimits}
\newcommand{\Card}{\mathop{\text{Card}}\nolimits}
\newcommand{\Vect}{\mathop{\text{Vect}}\nolimits}
\newcommand{\Mat}{\mathop{\text{Mat}}\nolimits}
\newcommand{\rg}{\mathop{\text{rg}}\nolimits}
\newcommand{\tr}{\mathop{\text{tr}}\nolimits}


%----- Structure des exercices ------

\newtheoremstyle{styleexo}% name
{2ex}% Space above
{3ex}% Space below
{}% Body font
{}% Indent amount 1
{\bfseries} % Theorem head font
{}% Punctuation after theorem head
{\newline}% Space after theorem head 2
{}% Theorem head spec (can be left empty, meaning ‘normal’)

%\theoremstyle{styleexo}
\newtheorem{exo}{Exercice}
\newtheorem{ind}{Indications}
\newtheorem{cor}{Correction}


\newcommand{\exercice}[1]{} \newcommand{\finexercice}{}
%\newcommand{\exercice}[1]{{\tiny\texttt{#1}}\vspace{-2ex}} % pour afficher le numero absolu, l'auteur...
\newcommand{\enonce}{\begin{exo}} \newcommand{\finenonce}{\end{exo}}
\newcommand{\indication}{\begin{ind}} \newcommand{\finindication}{\end{ind}}
\newcommand{\correction}{\begin{cor}} \newcommand{\fincorrection}{\end{cor}}

\newcommand{\noindication}{\stepcounter{ind}}
\newcommand{\nocorrection}{\stepcounter{cor}}

\newcommand{\fiche}[1]{} \newcommand{\finfiche}{}
\newcommand{\titre}[1]{\centerline{\large \bf #1}}
\newcommand{\addcommand}[1]{}
\newcommand{\video}[1]{}

% Marge
\newcommand{\mymargin}[1]{\marginpar{{\small #1}}}

\def\noqed{\renewcommand{\qedsymbol}{}}


%----- Presentation ------
\setlength{\parindent}{0cm}

%\newcommand{\ExoSept}{\href{http://exo7.emath.fr}{\textbf{\textsf{Exo7}}}}

\definecolor{myred}{rgb}{0.93,0.26,0}
\definecolor{myorange}{rgb}{0.97,0.58,0}
\definecolor{myyellow}{rgb}{1,0.86,0}

\newcommand{\LogoExoSept}[1]{  % input : echelle
{\usefont{U}{cmss}{bx}{n}
\begin{tikzpicture}[scale=0.1*#1,transform shape]
  \fill[color=myorange] (0,0)--(4,0)--(4,-4)--(0,-4)--cycle;
  \fill[color=myred] (0,0)--(0,3)--(-3,3)--(-3,0)--cycle;
  \fill[color=myyellow] (4,0)--(7,4)--(3,7)--(0,3)--cycle;
  \node[scale=5] at (3.5,3.5) {Exo7};
\end{tikzpicture}}
}


\newcommand{\debutmontitre}{
  \author{} \date{} 
  \thispagestyle{empty}
  \hspace*{-10ex}
  \begin{minipage}{\textwidth}
    \titlepage  
  \vspace*{-2.5cm}
  \begin{center}
    \LogoExoSept{2.5}
  \end{center}
  \end{minipage}

  \vspace*{-0cm}
  
  % Astuce pour que le background ne soit pas discrétisé lors de la conversion pdf -> png
\begin{tikzpicture}
        \fill[opacity=0,green!60!black] (0,0)--++(0,0)--++(0,0)--++(0,0)--cycle; 
\end{tikzpicture}

% toc S'affiche trop tot :
% \tableofcontents[hideallsubsections, pausesections]
}

\newcommand{\finmontitre}{
  \end{frame}
  \setcounter{framenumber}{0}
} % ne marche pas pour une raison obscure

%----- Commandes supplementaires ------

% \usepackage[landscape]{geometry}
% \geometry{top=1cm, bottom=3cm, left=2cm, right=10cm, marginparsep=1cm
% }
% \usepackage[a4paper]{geometry}
% \geometry{top=2cm, bottom=2cm, left=2cm, right=2cm, marginparsep=1cm
% }

%\usepackage{standalone}


% New command Arnaud -- november 2011
\setbeamersize{text margin left=24ex}
% si vous modifier cette valeur il faut aussi
% modifier le decalage du titre pour compenser
% (ex : ici =+10ex, titre =-5ex

\theoremstyle{definition}
%\newtheorem{proposition}{Proposition}
%\newtheorem{exemple}{Exemple}
%\newtheorem{theoreme}{Théorème}
%\newtheorem{lemme}{Lemme}
%\newtheorem{corollaire}{Corollaire}
%\newtheorem*{remarque*}{Remarque}
%\newtheorem*{miniexercice}{Mini-exercices}
%\newtheorem{definition}{Définition}

% Commande tikz
\usetikzlibrary{calc}
\usetikzlibrary{patterns,arrows}
\usetikzlibrary{matrix}
\usetikzlibrary{fadings} 

%definition d'un terme
\newcommand{\defi}[1]{{\color{myorange}\textbf{\emph{#1}}}}
\newcommand{\evidence}[1]{{\color{blue}\textbf{\emph{#1}}}}
\newcommand{\assertion}[1]{\emph{\og#1\fg}}  % pour chapitre logique
%\renewcommand{\contentsname}{Sommaire}
\renewcommand{\contentsname}{}
\setcounter{tocdepth}{2}



%------ Figures ------

\def\myscale{1} % par défaut 
\newcommand{\myfigure}[2]{  % entrée : echelle, fichier figure
\def\myscale{#1}
\begin{center}
\footnotesize
{#2}
\end{center}}


%------ Encadrement ------

\usepackage{fancybox}


\newcommand{\mybox}[1]{
\setlength{\fboxsep}{7pt}
\begin{center}
\shadowbox{#1}
\end{center}}

\newcommand{\myboxinline}[1]{
\setlength{\fboxsep}{5pt}
\raisebox{-10pt}{
\shadowbox{#1}
}
}

%--------------- Commande beamer---------------
\newcommand{\beameronly}[1]{#1} % permet de mettre des pause dans beamer pas dans poly


\setbeamertemplate{navigation symbols}{}
\setbeamertemplate{footline}  % tiré du fichier beamerouterinfolines.sty
{
  \leavevmode%
  \hbox{%
  \begin{beamercolorbox}[wd=.333333\paperwidth,ht=2.25ex,dp=1ex,center]{author in head/foot}%
    % \usebeamerfont{author in head/foot}\insertshortauthor%~~(\insertshortinstitute)
    \usebeamerfont{section in head/foot}{\bf\insertshorttitle}
  \end{beamercolorbox}%
  \begin{beamercolorbox}[wd=.333333\paperwidth,ht=2.25ex,dp=1ex,center]{title in head/foot}%
    \usebeamerfont{section in head/foot}{\bf\insertsectionhead}
  \end{beamercolorbox}%
  \begin{beamercolorbox}[wd=.333333\paperwidth,ht=2.25ex,dp=1ex,right]{date in head/foot}%
    % \usebeamerfont{date in head/foot}\insertshortdate{}\hspace*{2em}
    \insertframenumber{} / \inserttotalframenumber\hspace*{2ex} 
  \end{beamercolorbox}}%
  \vskip0pt%
}


\definecolor{mygrey}{rgb}{0.5,0.5,0.5}
\setlength{\parindent}{0cm}
%\DeclareTextFontCommand{\helvetica}{\fontfamily{phv}\selectfont}

% background beamer
\definecolor{couleurhaut}{rgb}{0.85,0.9,1}  % creme
\definecolor{couleurmilieu}{rgb}{1,1,1}  % vert pale
\definecolor{couleurbas}{rgb}{0.85,0.9,1}  % blanc
\setbeamertemplate{background canvas}[vertical shading]%
[top=couleurhaut,middle=couleurmilieu,midpoint=0.4,bottom=couleurbas] 
%[top=fondtitre!05,bottom=fondtitre!60]



\makeatletter
\setbeamertemplate{theorem begin}
{%
  \begin{\inserttheoremblockenv}
  {%
    \inserttheoremheadfont
    \inserttheoremname
    \inserttheoremnumber
    \ifx\inserttheoremaddition\@empty\else\ (\inserttheoremaddition)\fi%
    \inserttheorempunctuation
  }%
}
\setbeamertemplate{theorem end}{\end{\inserttheoremblockenv}}

\newenvironment{theoreme}[1][]{%
   \setbeamercolor{block title}{fg=structure,bg=structure!40}
   \setbeamercolor{block body}{fg=black,bg=structure!10}
   \begin{block}{{\bf Th\'eor\`eme }#1}
}{%
   \end{block}%
}


\newenvironment{proposition}[1][]{%
   \setbeamercolor{block title}{fg=structure,bg=structure!40}
   \setbeamercolor{block body}{fg=black,bg=structure!10}
   \begin{block}{{\bf Proposition }#1}
}{%
   \end{block}%
}

\newenvironment{corollaire}[1][]{%
   \setbeamercolor{block title}{fg=structure,bg=structure!40}
   \setbeamercolor{block body}{fg=black,bg=structure!10}
   \begin{block}{{\bf Corollaire }#1}
}{%
   \end{block}%
}

\newenvironment{mydefinition}[1][]{%
   \setbeamercolor{block title}{fg=structure,bg=structure!40}
   \setbeamercolor{block body}{fg=black,bg=structure!10}
   \begin{block}{{\bf Définition} #1}
}{%
   \end{block}%
}

\newenvironment{lemme}[0]{%
   \setbeamercolor{block title}{fg=structure,bg=structure!40}
   \setbeamercolor{block body}{fg=black,bg=structure!10}
   \begin{block}{\bf Lemme}
}{%
   \end{block}%
}

\newenvironment{remarque}[1][]{%
   \setbeamercolor{block title}{fg=black,bg=structure!20}
   \setbeamercolor{block body}{fg=black,bg=structure!5}
   \begin{block}{Remarque #1}
}{%
   \end{block}%
}


\newenvironment{exemple}[1][]{%
   \setbeamercolor{block title}{fg=black,bg=structure!20}
   \setbeamercolor{block body}{fg=black,bg=structure!5}
   \begin{block}{{\bf Exemple }#1}
}{%
   \end{block}%
}


\newenvironment{miniexercice}[0]{%
   \setbeamercolor{block title}{fg=structure,bg=structure!20}
   \setbeamercolor{block body}{fg=black,bg=structure!5}
   \begin{block}{Mini-exercices}
}{%
   \end{block}%
}


\newenvironment{tp}[0]{%
   \setbeamercolor{block title}{fg=structure,bg=structure!40}
   \setbeamercolor{block body}{fg=black,bg=structure!10}
   \begin{block}{\bf Travaux pratiques}
}{%
   \end{block}%
}
\newenvironment{exercicecours}[1][]{%
   \setbeamercolor{block title}{fg=structure,bg=structure!40}
   \setbeamercolor{block body}{fg=black,bg=structure!10}
   \begin{block}{{\bf Exercice }#1}
}{%
   \end{block}%
}
\newenvironment{algo}[1][]{%
   \setbeamercolor{block title}{fg=structure,bg=structure!40}
   \setbeamercolor{block body}{fg=black,bg=structure!10}
   \begin{block}{{\bf Algorithme}\hfill{\color{gray}\texttt{#1}}}
}{%
   \end{block}%
}


\setbeamertemplate{proof begin}{
   \setbeamercolor{block title}{fg=black,bg=structure!20}
   \setbeamercolor{block body}{fg=black,bg=structure!5}
   \begin{block}{{\footnotesize Démonstration}}
   \footnotesize
   \smallskip}
\setbeamertemplate{proof end}{%
   \end{block}}
\setbeamertemplate{qed symbol}{\openbox}


\makeatother
% Couleur à définir
   
%%%%%%%%%%%%%%%%%%%%%%%%%%%%%%%%%%%%%%%%%%%%%%%%%%%%%%%%%%%%%
%%%%%%%%%%%%%%%%%%%%%%%%%%%%%%%%%%%%%%%%%%%%%%%%%%%%%%%%%%%%%


\begin{document}


\title{{\bf Séries}}
\subtitle{Séries alternées}

\begin{frame}
  
  \debutmontitre

  \pause

{\footnotesize
\hfill
\setbeamercovered{transparent=50}
\begin{minipage}{0.6\textwidth}
  \begin{itemize}
    \item<3-> Critère de Leibniz
    \item<4-> Reste
    \item<5-> Contre-exemple
  \end{itemize}
\end{minipage}
}

\end{frame}

\setcounter{framenumber}{0}



%%%%%%%%%%%%%%%%%%%%%%%%%%%%%%%%%%%%%%%%%%%%%%%%%%%%%%%%%%%%%%%%
\section{Critère de Leibniz}

\begin{frame}

Soit $(u_k)_{k\ge0}$ une suite telle que $u_k \ge 0$

\pause
La série $\displaystyle\sum_{k \ge 0} (-1)^k u_k$ s'appelle une \defi{série alternée}

\pause
\begin{theoreme}[Critère de Leibniz]
\pause
Supposons que la suite $(u_k)_{k\ge0}$ vérifie :
\begin{enumerate}
  \item\pause $u_k  \ge 0$ pour tout $k \ge 0$
  \item\pause la suite $(u_k)$ est une suite décroissante
  \item\pause $\displaystyle\lim_{k\to+\infty} u_k=0$
\end{enumerate}
\pause
Alors la série alternée $\displaystyle \sum_{k=0}^{+\infty} (-1)^k u_k$ converge
\end{theoreme}

\end{frame}


\begin{frame}
\begin{proof}
\begin{itemize}
\item Les suites $(S_{2n+1})$ et $(S_{2n})$ sont adjacentes :
\begin{itemize}
  \item\pause La suite $(S_{2n+1})$ est croissante \pause car 
  $S_{2n+1}-S_{2n-1}=u_{2n}-u_{2n+1}\ge 0$
  
  \item\pause La suite $(S_{2n})$ est décroissante \pause car
  $S_{2n}-S_{2n-2}= u_{2n}-u_{2n-1}\le 0$
  
  \item\pause $S_{2n} \ge S_{2n+1}$ \pause car 
  $S_{2n+1} - S_{2n} = -u_{2n+1} \le 0$ 
  
  \item\pause $S_{2n+1} - S_{2n}$ tend vers $0$ \pause car $S_{2n+1} - S_{2n} = -u_{2n+1} \to 0$
\end{itemize}
\item\pause Donc $(S_{2n+1})$ et $(S_{2n})$ sont convergentes et ont la même limite $S$
\item\pause On conclut que $(S_n)$ converge vers $S$
\item\pause De plus on a montré :
\begin{itemize}
  \item $S_{2n+1} \le S \le S_{2n}$ pour tout $n$
\item\pause et donc aussi
 \begin{eqnarray*}
 &\onslide<14->{0\ge} R_{2n}= S-S_{2n} \pause\onslide<15->{\ge S_{2n+1}-S_{2n}} \pause\pause =-u_{2n+1}\\\pause
 \text{et } &0 \le R_{2n+1}= S-S_{2n+1} \le  S_{2n+2} -S_{2n+1} = u_{2n+2}
 \end{eqnarray*}
 \item\pause ainsi pour tout $n$ on a 
 $|R_n|=|S-S_n|\le  u_{n+1}$\qedhere
\end{itemize}
\end{itemize}
\end{proof}
\end{frame}

\begin{frame}
\begin{exemple}
La \defi{série harmonique alternée} 
$$\sum_{k=0}^{+\infty} (-1)^{k} \frac{1}{k+1} = 1-\frac{1}{2}+\frac{1}{3}-\frac{1}{4} + \cdots $$
converge

\vspace{.3cm}\pause 
En effet, en posant $u_k = \frac{1}{k+1}$, alors
\begin{enumerate}
  \item\pause  $u_k\ge0$
  \item\pause  $(u_k)$ est une suite décroissante
  \item\pause  la suite $(u_k)$ tend vers $0$
\end{enumerate}
\pause 
Par le critère de Leibniz, la série alternée
$\displaystyle\sum_{k=0}^{+\infty} (-1)^{k} \frac{1}{k+1}$ converge
\end{exemple}
\end{frame}

%%%%%%%%%%%%%%%%%%%%%%%%%%%%%%%%%%%%%%%%%%%%%%%%%%%%%%%%%%%%%%%%
\section{Reste}

\begin{frame}

\begin{corollaire}
Soit une série alternée $\displaystyle \sum_{k=0}^{+\infty} (-1)^k u_k$ vérifiant 
les hypothèses du critère de Leibniz. \pause
Soit $S$ sa somme et $(S_n)$ la suite des sommes partielles
\begin{enumerate}
  \item\pause La somme $S$ vérifie les encadrements :
  $$ S_1\le S_3\le \cdots \le S_{2n+1} \le \cdots \le S 
  \le  \cdots\le S_{2n} \le \cdots\le S_2\le S_0$$
  \item\pause De plus, pour le reste $\displaystyle R_n=S-S_n =\sum_{k=n+1}^{+\infty} (-1)^k u_k$, on  a
$$\big|R_n\big|\le u_{n+1}$$  
\end{enumerate}
\end{corollaire}

\pause
La vitesse de convergence de la série alternée est donc
dictée par la décroissance vers $0$ de la suite $(u_k)$
\end{frame}

\begin{frame}
\begin{exemple}
Notons $S$ la somme de la série harmonique alternée 
$\displaystyle\sum_{k=0}^{+\infty} \frac{(-1)^{k}}{k+1}$
\begin{itemize}
  \item\pause
$S_0 = 1$, \ \pause
$S_1 = 1-\frac{1}{2}$, \ \pause
$S_2 = 1-\frac{1}{2}+\frac{1}{3}$, \pause

$S_3 = 1-\frac{1}{2}+\frac{1}{3}-\frac{1}{4}$, \
$S_4 = 1-\frac{1}{2}+\frac{1}{3}-\frac{1}{4}+\frac{1}{5}$ \ \ldots

\item\pause D'après le corollaire
\begin{multline*}
1-\frac{1}{2} \le 1-\frac{1}{2}+\frac{1}{3}-\frac{1}{4} \le \cdots \le S_{2n+1} \le \cdots \le S \\
  \le  \cdots\le S_{2n} \le \cdots 
  \le 1-\frac{1}{2}+\frac{1}{3}-\frac{1}{4}+\frac{1}{5}
  \le 1-\frac{1}{2}+\frac{1}{3} \le 1
\end{multline*}
\item\pause On en déduit 
$$S_3 = \frac{35}{60} \simeq 0,58333\ldots \le S \le S_4 = \frac{47}{60} \simeq 0,78333\ldots$$
\end{itemize}
\end{exemple}
\end{frame}

\begin{frame}
\begin{exemple}
Notons toujours $S$ la somme de la série harmonique alternée 
$\displaystyle\sum_{k=0}^{+\infty} \frac{(-1)^{k}}{k+1}$
\begin{itemize}
  \item\pause Pour $n=200$ on obtient 
$$S_{201} \simeq 0,69067\ldots \le S \le S_{200} \simeq 0,69562\ldots$$
\item\pause Ce qui donne les deux premières décimales de $S \simeq 0,69\ldots$
\item\pause Majoration de l'erreur commise en approchant $S$ par $S_{200}$ :
\pause
$$|S-S_{200}| = |R_{200}| \le u_{201} = \frac{1}{202} < 5 \cdot 10^{-3}$$
\end{itemize}
\end{exemple}
\end{frame}

%%%%%%%%%%%%%%%%%%%%%%%%%%%%%%%%%%%%%%%%%%%%%%%%%%%%%%%%%%%%%%%%
\section{Contre-exemple}

\begin{frame}

Attention ! Dans le critère de Leibniz :
\begin{enumerate}
  \item\pause on ne peut pas se passer de la condition 
  de décroissance de la suite $(u_k)$ 
  
  \item\pause il n'est pas possible de remplacer
$u_k$ par un équivalent à l'infini, car la décroissance n'est pas conservée par
équivalence
\end{enumerate}

\end{frame}

\begin{frame}
\begin{exemple}
Voici deux séries alternées :
$$
\sum_{k \ge 2}  \frac{(-1)^k}{\sqrt{k}}
\quad \text{ converge,}
\qquad\qquad
\sum_{k \ge 2}  \frac{(-1)^k}{\sqrt{k}+(-1)^k}
\quad \text{ diverge}
$$ 
\begin{itemize}
  \item\pause  Le critère de Leibniz s'applique à la première : 
  \pause $u_k = \frac{1}{\sqrt k}$ est une suite 
  \pause positive, \pause décroissante, \pause qui tend vers $0$

\pause 
donc la série alternée  $\sum_{k \ge 2}  \frac{(-1)^k}{\sqrt{k}}$ converge


\item\pause  Le critère de Leibniz ne s'applique pas à la seconde : \pause la suite $v_k = \frac{1}{\sqrt{k}+(-1)^k}$ est bien positive et tend vers $0$, \pause mais elle n'est pas
décroissante

\item\pause Cependant :
$$
v_k = \frac{1}{\sqrt{k}+(-1)^k} \ \sim \ \frac{1}{\sqrt{k}} = u_k
$$
\end{itemize}
\end{exemple}
\end{frame}

\begin{frame}
\begin{exemple}

Montrons que $\displaystyle\sum_{k \ge 2} \frac{(-1)^k}{\sqrt{k}+(-1)^k}$ diverge

\begin{itemize}
  \item\pause  Calculons la différence $w_k = (-1)^ku_k-(-1)^kv_k$
  \pause 
$$
w_k 
=  \tfrac{(-1)^k}{\sqrt{k}}-\tfrac{(-1)^k}{\sqrt{k}+(-1)^k} 
\pause = (-1)^k\tfrac{\sqrt{k}+(-1)^k-\sqrt{k}}{k+(-1)^k\sqrt{k}} 
\pause =  \tfrac{1}{k+(-1)^k\sqrt{k}} 
\pause \sim  \frac{1}{k} 
$$

\item\pause 
Ainsi la série de terme général $w_k$
diverge

\item\pause  On sait que la série $\sum_{k \ge 2} (-1)^k u_k$
est convergente

\pause 
Si par l'absurde la série
$\sum_{k \ge 2}(-1)^k v_k$ était convergente, \pause alors par linéarité la série
$\sum_{k \ge 2} w_k = \sum_{k \ge 2} (-1)^k u_k - \sum_{k \ge 2} (-1)^k v_k$
serait convergente\pause  : contradiction

\item\pause 
Conclusion : la série $\sum_{k \ge 2} \frac{(-1)^k}{\sqrt{k}+(-1)^k}$ diverge
\end{itemize}
\end{exemple}

\end{frame}



%%%%%%%%%%%%%%%%%%%%%%%%%%%%%%%%%%%%%%%%%%%%%%%%%%%%%%%%%%%%%%%%
\section{Mini-exercices}

\begin{frame}
\begin{miniexercice}
\begin{enumerate}
  \item Est-ce que le critère de Leibniz s'applique aux séries suivantes ?
  
  \centerline{$\displaystyle
  \sum_{k\ge 2} \frac{(-1)^k}{\sqrt k + \ln k} \qquad
  \sum_{k\ge 2} \frac{(-1)^k}{\frac{k+1}{k}} 
  $}
  
  \centerline{$\displaystyle
  \sum_{k\ge2} \frac{1}{(-1)^{k+1}(\sqrt k - \ln k)}  \qquad
  \sum_{k\ge 2} (-1)^k\big(\ln(k+1)-\ln(k)\big)$ }
  
  \centerline{$\displaystyle
  \sum_{k\ge2} \frac{1}{\sqrt k + (-1)^{k}\ln k} \qquad
  \sum_{k\ge 2} \frac{(-1)^k}{3k + (-1)^k}$}
  

  \item \`A partir de quel rang la somme partielle
  $S_n$ de la série $\displaystyle\sum_{k=1}^{+\infty} \frac{(-1)^k}{k^2}$ 
  est-elle une approximation à $0,1$ près de sa somme $S$ ?
  Et à $0,001$ près ?
  \`A l'aide d'une calculatrice ou d'un ordinateur, déterminer deux décimales
  exactes après la virgule de $S$.
  Mêmes questions avec  $\frac{(-1)^k}{2^k}$ ; $\frac{(-1)^k}{\sqrt{k}}$ ;
  $\frac{(-1)^k}{k!}$.
  
\end{enumerate}
\end{miniexercice}
\end{frame}

\end{document}