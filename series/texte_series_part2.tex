
%%%%%%%%%%%%%%%%%% PREAMBULE %%%%%%%%%%%%%%%%%%


\documentclass[12pt]{article}

\usepackage{amsfonts,amsmath,amssymb,amsthm}
\usepackage[utf8]{inputenc}
\usepackage[T1]{fontenc}
\usepackage[francais]{babel}


% packages
\usepackage{amsfonts,amsmath,amssymb,amsthm}
\usepackage[utf8]{inputenc}
\usepackage[T1]{fontenc}
%\usepackage{lmodern}

\usepackage[francais]{babel}
\usepackage{fancybox}
\usepackage{graphicx}

\usepackage{float}

%\usepackage[usenames, x11names]{xcolor}
\usepackage{tikz}
\usepackage{datetime}

\usepackage{mathptmx}
%\usepackage{fouriernc}
%\usepackage{newcent}
\usepackage[mathcal,mathbf]{euler}

%\usepackage{palatino}
%\usepackage{newcent}


% Commande spéciale prompteur

%\usepackage{mathptmx}
%\usepackage[mathcal,mathbf]{euler}
%\usepackage{mathpple,multido}

\usepackage[a4paper]{geometry}
\geometry{top=2cm, bottom=2cm, left=1cm, right=1cm, marginparsep=1cm}

\newcommand{\change}{{\color{red}\rule{\textwidth}{1mm}\\}}

\newcounter{mydiapo}

\newcommand{\diapo}{\newpage
\hfill {\normalsize  Diapo \themydiapo \quad \texttt{[\jobname]}} \\
\stepcounter{mydiapo}}


%%%%%%% COULEURS %%%%%%%%%%

% Pour blanc sur noir :
%\pagecolor[rgb]{0.5,0.5,0.5}
% \pagecolor[rgb]{0,0,0}
% \color[rgb]{1,1,1}



%\DeclareFixedFont{\myfont}{U}{cmss}{bx}{n}{18pt}
\newcommand{\debuttexte}{
%%%%%%%%%%%%% FONTES %%%%%%%%%%%%%
\renewcommand{\baselinestretch}{1.5}
\usefont{U}{cmss}{bx}{n}
\bfseries

% Taille normale : commenter le reste !
%Taille Arnaud
%\fontsize{19}{19}\selectfont

% Taille Barbara
%\fontsize{21}{22}\selectfont

%Taille François
\fontsize{25}{30}\selectfont

%Taille Pascal
%\fontsize{25}{30}\selectfont

%Taille Laura
%\fontsize{30}{35}\selectfont


%\myfont
%\usefont{U}{cmss}{bx}{n}

%\Huge
%\addtolength{\parskip}{\baselineskip}
}


% \usepackage{hyperref}
% \hypersetup{colorlinks=true, linkcolor=blue, urlcolor=blue,
% pdftitle={Exo7 - Exercices de mathématiques}, pdfauthor={Exo7}}


%section
% \usepackage{sectsty}
% \allsectionsfont{\bf}
%\sectionfont{\color{Tomato3}\upshape\selectfont}
%\subsectionfont{\color{Tomato4}\upshape\selectfont}

%----- Ensembles : entiers, reels, complexes -----
\newcommand{\Nn}{\mathbb{N}} \newcommand{\N}{\mathbb{N}}
\newcommand{\Zz}{\mathbb{Z}} \newcommand{\Z}{\mathbb{Z}}
\newcommand{\Qq}{\mathbb{Q}} \newcommand{\Q}{\mathbb{Q}}
\newcommand{\Rr}{\mathbb{R}} \newcommand{\R}{\mathbb{R}}
\newcommand{\Cc}{\mathbb{C}} 
\newcommand{\Kk}{\mathbb{K}} \newcommand{\K}{\mathbb{K}}

%----- Modifications de symboles -----
\renewcommand{\epsilon}{\varepsilon}
\renewcommand{\Re}{\mathop{\text{Re}}\nolimits}
\renewcommand{\Im}{\mathop{\text{Im}}\nolimits}
%\newcommand{\llbracket}{\left[\kern-0.15em\left[}
%\newcommand{\rrbracket}{\right]\kern-0.15em\right]}

\renewcommand{\ge}{\geqslant}
\renewcommand{\geq}{\geqslant}
\renewcommand{\le}{\leqslant}
\renewcommand{\leq}{\leqslant}

%----- Fonctions usuelles -----
\newcommand{\ch}{\mathop{\mathrm{ch}}\nolimits}
\newcommand{\sh}{\mathop{\mathrm{sh}}\nolimits}
\renewcommand{\tanh}{\mathop{\mathrm{th}}\nolimits}
\newcommand{\cotan}{\mathop{\mathrm{cotan}}\nolimits}
\newcommand{\Arcsin}{\mathop{\mathrm{Arcsin}}\nolimits}
\newcommand{\Arccos}{\mathop{\mathrm{Arccos}}\nolimits}
\newcommand{\Arctan}{\mathop{\mathrm{Arctan}}\nolimits}
\newcommand{\Argsh}{\mathop{\mathrm{Argsh}}\nolimits}
\newcommand{\Argch}{\mathop{\mathrm{Argch}}\nolimits}
\newcommand{\Argth}{\mathop{\mathrm{Argth}}\nolimits}
\newcommand{\pgcd}{\mathop{\mathrm{pgcd}}\nolimits} 

\newcommand{\Card}{\mathop{\text{Card}}\nolimits}
\newcommand{\Ker}{\mathop{\text{Ker}}\nolimits}
\newcommand{\id}{\mathop{\text{id}}\nolimits}
\newcommand{\ii}{\mathrm{i}}
\newcommand{\dd}{\mathrm{d}}
\newcommand{\Vect}{\mathop{\text{Vect}}\nolimits}
\newcommand{\Mat}{\mathop{\mathrm{Mat}}\nolimits}
\newcommand{\rg}{\mathop{\text{rg}}\nolimits}
\newcommand{\tr}{\mathop{\text{tr}}\nolimits}
\newcommand{\ppcm}{\mathop{\text{ppcm}}\nolimits}

%----- Structure des exercices ------

\newtheoremstyle{styleexo}% name
{2ex}% Space above
{3ex}% Space below
{}% Body font
{}% Indent amount 1
{\bfseries} % Theorem head font
{}% Punctuation after theorem head
{\newline}% Space after theorem head 2
{}% Theorem head spec (can be left empty, meaning ‘normal’)

%\theoremstyle{styleexo}
\newtheorem{exo}{Exercice}
\newtheorem{ind}{Indications}
\newtheorem{cor}{Correction}


\newcommand{\exercice}[1]{} \newcommand{\finexercice}{}
%\newcommand{\exercice}[1]{{\tiny\texttt{#1}}\vspace{-2ex}} % pour afficher le numero absolu, l'auteur...
\newcommand{\enonce}{\begin{exo}} \newcommand{\finenonce}{\end{exo}}
\newcommand{\indication}{\begin{ind}} \newcommand{\finindication}{\end{ind}}
\newcommand{\correction}{\begin{cor}} \newcommand{\fincorrection}{\end{cor}}

\newcommand{\noindication}{\stepcounter{ind}}
\newcommand{\nocorrection}{\stepcounter{cor}}

\newcommand{\fiche}[1]{} \newcommand{\finfiche}{}
\newcommand{\titre}[1]{\centerline{\large \bf #1}}
\newcommand{\addcommand}[1]{}
\newcommand{\video}[1]{}

% Marge
\newcommand{\mymargin}[1]{\marginpar{{\small #1}}}



%----- Presentation ------
\setlength{\parindent}{0cm}

%\newcommand{\ExoSept}{\href{http://exo7.emath.fr}{\textbf{\textsf{Exo7}}}}

\definecolor{myred}{rgb}{0.93,0.26,0}
\definecolor{myorange}{rgb}{0.97,0.58,0}
\definecolor{myyellow}{rgb}{1,0.86,0}

\newcommand{\LogoExoSept}[1]{  % input : echelle
{\usefont{U}{cmss}{bx}{n}
\begin{tikzpicture}[scale=0.1*#1,transform shape]
  \fill[color=myorange] (0,0)--(4,0)--(4,-4)--(0,-4)--cycle;
  \fill[color=myred] (0,0)--(0,3)--(-3,3)--(-3,0)--cycle;
  \fill[color=myyellow] (4,0)--(7,4)--(3,7)--(0,3)--cycle;
  \node[scale=5] at (3.5,3.5) {Exo7};
\end{tikzpicture}}
}



\theoremstyle{definition}
%\newtheorem{proposition}{Proposition}
%\newtheorem{exemple}{Exemple}
%\newtheorem{theoreme}{Théorème}
\newtheorem{lemme}{Lemme}
\newtheorem{corollaire}{Corollaire}
%\newtheorem*{remarque*}{Remarque}
%\newtheorem*{miniexercice}{Mini-exercices}
%\newtheorem{definition}{Définition}




%definition d'un terme
\newcommand{\defi}[1]{{\color{myorange}\textbf{\emph{#1}}}}
\newcommand{\evidence}[1]{{\color{blue}\textbf{\emph{#1}}}}



 %----- Commandes divers ------

\newcommand{\codeinline}[1]{\texttt{#1}}

%%%%%%%%%%%%%%%%%%%%%%%%%%%%%%%%%%%%%%%%%%%%%%%%%%%%%%%%%%%%%
%%%%%%%%%%%%%%%%%%%%%%%%%%%%%%%%%%%%%%%%%%%%%%%%%%%%%%%%%%%%%


\begin{document}

\debuttexte


%%%%%%%%%%%%%%%%%%%%%%%%%%%%%%%%%%%%%%%%%%%%%%%%%%%%%%%%%%%
\diapo

Nous poursuivons le chapitre consacré aux séries par une leçon sur les séries à termes positifs.

\change
Les séries à termes positifs ou nuls se comportent comme les suites croissantes
et sont donc plus faciles à étudier.

\change
Nous commencerons par étudier la convergence par les sommes partielles,

\change
puis nous verrons un théorème de comparaison

\change
et enfin, nous verrons le théorème des équivalents,

avec à chaque fois plusieurs exemples d'utilisation.

%%%%%%%%%%%%%%%%%%%%%%%%%%%%%%%%%%%%%%%%%%%%%%%%%%%%%%%%%%%
\diapo

Pour étudier la convergence des sommes partielles, commençons par rappeler un résultat sur les suites croissantes de nombres réels. Soit $(s_n)_{n\ge0}$ une telle suite. On a l'alternative suivante.

\change
Si la suite $(s_n)$ est majorée, alors la suite converge, c'est-à-dire qu'elle admet une limite finie.

\change
Sinon la suite $(s_n)$ tend vers $+\infty$.

\change
Appliquons ceci aux séries $\sum u_k$ à \defi{termes positifs}, 
c'est-à-dire $u_k\ge 0$ pour tout $k$.

\change
Dans ce cas la suite $(S_n)$ des sommes partielles, définie par 
$S_n = \sum_{k=0}^n u_k$, est une suite croissante.
En effet 

$S_{n}-S_{n-1} = u_n \ge 0.$

\change
Par les rappels sur les suites, nous avons donc la proposition suivante.

Une série à termes positifs est une série convergente si et seulement si la suite des sommes partielles est majorée.

\change
Autrement dit, la série $\sum u_k$ à termes positifs converge si et seulement s'il existe $M>0$
tel que, pour tout $n\ge 0$, $S_n \le M$.

\change
De plus, dans le cas de convergence, la somme de la série
$S$, qui vaut $\lim S_n$, est un majorant de la suite  $(S_n)$ des sommes partielles.

\change
Les deux situations convergence/divergence sont possibles. 

Par exemple la série géométrique
$\sum_{k\ge0} q^k$ converge si $0<q<1$, et diverge si $q \ge 1$.

%%%%%%%%%%%%%%%%%%%%%%%%%%%%%%%%%%%%%%%%%%%%%%%%%%%%%%%%%%%
\diapo

Quelle est la méthode générale pour trouver la nature d'une série à termes positifs ?
On la compare avec des séries classiques simples au moyen du théorème de comparaison suivant.

\change
Soient $\sum u_k$ et $\sum v_k$ deux séries à termes positifs ou nuls.

\change
On suppose qu'il existe $k_0\ge 0$ tel que, pour tout $k\ge k_0$, $u_k \le v_k$.

c-à-d la suite $u_k$ est plus petite que la suite $v_k$, éventuellement 
à partir d'un certain rang seulement.

\change
Si $\sum v_k$ converge alors $\sum u_k$ converge.

\change
Si $\sum u_k$ diverge alors $\sum v_k$ diverge.

\change
Démontrons ce théorème.

Comme nous l'avons observé, la convergence ne dépend pas des
premiers termes. Sans perte de généralité on peut donc supposer $k_0=0$.

\change
Notons $S_n=u_0+\cdots+u_n$ et $S'_n = v_0+\cdots+v_n$. 

\change
Les suites $(S_n)$ et $(S'_n)$ sont croissantes, 

\change
et de plus, pour tout $n \ge 0$, $S_n\le S'_n$. 

\change
Si la série $\sum v_k$ converge, alors la suite
$(S'_n)$ converge. Soit $S'$ sa limite. 

\change
La suite $(S_n)$ est croissante et majorée par $S'$, 

\change
donc elle converge, et ainsi la série $\sum u_k$
converge aussi. 

\change
Inversement, si la série $\sum u_k$ diverge, alors
la suite $(S_n)$ tend vers $+\infty$, 

\change
et il en est de même pour la
suite $(S'_n)$ et ainsi la série $\sum v_k$ diverge. 


%%%%%%%%%%%%%%%%%%%%%%%%%%%%%%%%%%%%%%%%%%%%%%%%%%%%%%%%%%%
\diapo

Nous allons à présent voir plusieurs exemples d'utilisation du théorème de comparaison.

\change
Nous avons déjà vu dans la leçon précédente que la
série $\displaystyle\sum_{k=0}^{+\infty} \frac{1}{(k+1)(k+2)}$ converge. 

Nous allons en déduire que $\displaystyle\sum_{k=1}^{+\infty} \frac{1}{k^2}$ converge.

\change
En effet, on a :
$
\displaystyle\lim_{k\to+\infty}
\dfrac{\dfrac{1}{2k^2}}{\dfrac{1}{(k+1)(k+2)}}=\frac{1}{2}. 
$

En particulier, on est donc sûr que ce rapport est plus petit que $1$ pour $k$ assez grand.

\change
Il existe $k_0$ tel que pour $k \ge k_0$ 
$$
\frac{1}{2k^2} \le \frac{1}{(k+1)(k+2)}
$$
En fait c'est vrai pour $k \ge 4$, mais il est inutile de calculer une
valeur précise de $k_0$. 

\change
On en déduit que la série de terme
général $\frac{1}{2k^2}$ converge,

\change
d'où le résultat par
linéarité.  


%%%%%%%%%%%%%%%%%%%%%%%%%%%%%%%%%%%%%%%%%%%%%%%%%%%%%%%%%%%
\diapo

Voici un autre exemple fondamental, la série exponentielle.

La série $\displaystyle\sum_{k\ge 0} \frac{1}{k!}$ converge.

Notons que $0!=1$ et que pour $k\ge 1$, $k!=1\cdot 2\cdot 3\dots \cdot k$.

\change
En effet 

$\frac{1}{k!}\le \frac{1}{k(k-1)}$ pour $k \ge 2$.

\change
Mais $\displaystyle\sum_{k \ge 2}\frac{1}{k(k-1)} =\displaystyle \sum_{k\ge0} \frac{1}{(k+1)(k+2)}$ par changement d'indice, qui est une série convergente.

\change
Donc la série exponentielle $\displaystyle\sum_{k\ge 0} \frac{1}{k!}$ converge.

\change
En fait, par définition, la somme $\displaystyle\sum_{k=0}^{+\infty} \frac{1}{k!}$ vaut le nombre d'Euler $e = \exp(1)$.


%%%%%%%%%%%%%%%%%%%%%%%%%%%%%%%%%%%%%%%%%%%%%%%%%%%%%%%%%%%
\diapo
Inversement, nous avons vu que la série $\displaystyle\sum_{k\ge1} \frac{1}{k}$
diverge. 

\change
On en déduit facilement que les séries 
$\displaystyle\sum_{k\ge1} \frac{\ln(k)}{k}$ et $\displaystyle\sum_{k\ge1} \frac{1}{\sqrt{k}}$ divergent également.  

%%%%%%%%%%%%%%%%%%%%%%%%%%%%%%%%%%%%%%%%%%%%%%%%%%%%%%%%%%%
\diapo

Continuons avec une application intéressante : le développement décimal d'un réel.

Soit $(a_k)_{k\ge 1}$ une suite d'entiers tous compris entre $0$ et $9$. 

Alors la série $\displaystyle\sum_{k\geq1}\frac{a_k}{10^k}$
converge.

\change
En effet, son terme général $u_k=\frac{a_k}{10^k}$ est majoré par $\frac{9}{10^k}$. 

\change
Mais la série géométrique $\sum \frac{1}{10^k}$ converge, car $\frac{1}{10}<1$. 

\change
La série 
$\sum \frac{9}{10^k}$ converge aussi par linéarité, d'où le résultat.

\change
Une telle somme $\displaystyle\sum_{k=1}^{+\infty} \frac{a_k}{10^k}$ est une écriture décimale d'un réel
$x$, avec ici $0 \le x \le 1$.

\change
Par exemple, si $a_k = 3$ pour tout $k$ : 

$\displaystyle\sum_{k=1}^{+\infty} \frac{3}{10^k} =$

\change
$ \frac{3}{10}+\frac{3}{100}+\frac{3}{1000}+\cdots$

\change
$= 0,3+0,03+0,003+\cdots$

\change
$ = 0,333\ldots$

\change
$= \frac13$

\change
On retrouve bien sûr le même résultat à l'aide de la série géométrique :

$\displaystyle\sum_{k=1}^{+\infty} \frac{3}{10^k} 
= \frac{3}{10} \sum_{k=0}^{+\infty} \frac{1}{10^k}$

\change
$
=  \frac{3}{10} \cdot \frac{1}{1-\frac{1}{10}}
=  \frac{3}{10} \cdot \frac{10}{9}$

qui redonne bien
$ \frac13$

%%%%%%%%%%%%%%%%%%%%%%%%%%%%%%%%%%%%%%%%%%%%%%%%%%%%%%%%%%%
\diapo

Nous allons améliorer le théorème de comparaison avec la notion de suites équivalentes.

Soient $(u_k)$ et $(v_k)$ deux suites strictement positives. 

\change
Alors les suites $(u_k)$ et $(v_k)$ sont dites équivalentes si 
$$\lim_{k\to+\infty} \frac{u_k}{v_k}=1.$$

\change
On note alors $u_k \sim v_k.$

\change
Théorème.

Soient $(u_k)$ et $(v_k)$ deux suites strictement positives. 

\change
Si $u_k \sim v_k$ alors les séries $\sum u_k$ et $\sum v_k$ sont de même nature.

\change
Autrement dit, si les suites sont équivalentes alors les séries sont soit toutes les deux convergentes,
soit toutes les deux divergentes. Bien sûr, en cas de convergence, il n'y a aucune 
raison que les sommes soient égales. 

\change
Enfin, si les suites sont toutes les deux strictement négatives, la
conclusion reste valable.

%%%%%%%%%%%%%%%%%%%%%%%%%%%%%%%%%%%%%%%%%%%%%%%%%%%%%%%%%%%
\diapo

Revenons sur un exemple qui montre que ce théorème est très pratique.

\change
Les suites $\frac{1}{k^2}$ et $\frac{1}{(k+1)(k+2)}=\frac{1}{k^2+3k+2}$
sont équivalentes car le leur quotient tend bien vers $1$.

\change
Comme la série $\sum \frac{1}{(k+1)(k+2)}$ converge (on l'a vu dans un exemple précédent), 

\change
alors cela implique que la série $\sum \frac{1}{k^2}$ converge.

\change
Démontrons le théorème des équivalents.

\change
Par hypothèse, comme le rapport $ \frac{u_k}{v_k}$ tend vers 1, pour tout $\epsilon>0$, il existe $k_0$ tel que, pour
tout $k \ge k_0$,
$$\left|\frac{u_k}{v_k} -1\right| < \epsilon,$$

\change
ce qui peut s'écrire par la double inégalité : 
$$(1-\epsilon)v_k < u_k <(1+\epsilon) v_k.$$

\change
Fixons un $\epsilon <1$.

\change
Si $\sum u_k$ converge, 

\change
alors par le théorème de comparaison, $\sum(1-\epsilon) v_k$ converge, 

\change
donc, en factorisant par $1-\epsilon$, $\sum v_k$ converge également. 

\change
Réciproquement, de la même manière, si $\sum u_k$ diverge, alors
$\sum (1+\epsilon)v_k$ diverge, et $\sum v_k$ aussi.

%%%%%%%%%%%%%%%%%%%%%%%%%%%%%%%%%%%%%%%%%%%%%%%%%%%%%%%%%%%
\diapo

Les deux séries 
$$
\sum \frac{k^2+3k+1}{k^4+2k^3+4}
\qquad \text{ et } \qquad 
\sum \frac{k +\ln(k)}{k^3}
$$
convergent.

\change
Dans les deux cas, le terme général est équivalent à $\frac{1}{k^2}$, 

\change
et nous savons que la série $\sum \frac{1}{k^2}$ converge.  

\change
Par contre
$$
\sum \frac{k^2+3k+1}{k^3+2k^2+4}
\qquad \text{ et } \qquad 
\sum \frac{k +\ln(k)}{k^2}$$
divergent.

\change
Dans les deux cas, le terme général est équivalent à $\frac{1}{k}$, 

\change
et nous avons vu que la série $\sum \frac{1}{k}$ diverge. 


%%%%%%%%%%%%%%%%%%%%%%%%%%%%%%%%%%%%%%%%%%%%%%%%%%%%%%%%%%%
\diapo

Voyons un exemple plus sophistiqué.

Est-ce que la série 
$$\sum_{k \ge 1}  \ln \big(\tanh k\big)$$
converge ?

La méthode est de chercher un équivalent simple du terme général.

\change
Remarquons tout d'abord que, pour $k >0$, $0<\tanh k<1$.

\change
Puis évaluons $\tanh k$ : par définition
$$\tanh k = \frac{\sh k}{\ch k} = \frac{e^k-e^{-k}}{e^k+e^{-k}}$$

\change
ce qui peut s'écrire
$$ 1+  \frac{-2e^{-k}}{e^k+e^{-k}} = 1+\frac{-2e^{-2k}}{1+e^{-2k}}$$

\change
Comme $\lim_{x\to 0} \frac{\ln(1+x)}{x}=1$, alors, si $u_k \to 0$, $\ln(1+u_k) \sim u_k$. 
  
\change
Ainsi $\ln(\tanh k) $, qui vaut $ \ln\left( 1+\frac{-2e^{-2k}}{1+e^{-2k}} \right)$ 

\change
est équivalent à  $  \frac{-2e^{-2k}}{1+e^{-2k}}$

\change
c'est-à-dire à $-2e^{-2k}$.

\change
Les suites $\ln(\tanh k)$ et $-2e^{-2k}$ sont deux suites strictement négatives et on vient de montrer qu'elles sont équivalentes.

\change
Or la série $\sum e^{-2k}= \sum (e^{-2})^k$ converge car c'est une série géométrique de raison $\frac{1}{e^2} < 1$.

\change
Alors, par le théorème des équivalents, la série $\sum\ln(\tanh k)$ converge également. 

(Si vous préférez, vous pouvez appliquer le théorème aux suites strictement positives $-\ln(\tanh k)$ et $+2e^{-2k}$.)

%%%%%%%%%%%%%%%%%%%%%%%%%%%%%%%%%%%%%%%%%%%%%%%%%%%%%%%%%%%
\diapo

Grâce aux deux théorèmes de cette leçon, vous pouvez 
à présent déterminer la nature de nombreuses séries.

\end{document}
