%%%%%%%%%%%%%%%%%% PREAMBULE %%%%%%%%%%%%%%%%%%

\documentclass[11pt]{report}

%----- Principaux packages -----
\usepackage{amsfonts,amsmath,amssymb,amsthm}
\usepackage[utf8]{inputenc}
\usepackage[T1]{fontenc}
\usepackage[francais]{babel}
\usepackage{fancybox}
\usepackage{graphicx}
\usepackage{float}
\usepackage[usenames, x11names]{xcolor}



%----- Ensembles : entiers, reels, complexes -----
\newcommand{\Nn}{\mathbb{N}} \newcommand{\N}{\mathbb{N}}
\newcommand{\Zz}{\mathbb{Z}} \newcommand{\Z}{\mathbb{Z}}
\newcommand{\Qq}{\mathbb{Q}} \newcommand{\Q}{\mathbb{Q}}
\newcommand{\Rr}{\mathbb{R}} \newcommand{\R}{\mathbb{R}}
\newcommand{\Cc}{\mathbb{C}} \newcommand{\C}{\mathbb{C}}
\newcommand{\Kk}{\mathbb{K}} \newcommand{\K}{\mathbb{K}}

%----- Modifications de symboles -----
\renewcommand{\epsilon}{\varepsilon}
\renewcommand{\Re}{\mathop{\text{Re}}\nolimits}
\renewcommand{\Im}{\mathop{\text{Im}}\nolimits}
\renewcommand{\ge}{\geqslant}
\renewcommand{\geq}{\geqslant}
\renewcommand{\le}{\leqslant}
\renewcommand{\leq}{\leqslant}
\renewcommand{\epsilon}{\varepsilon}

%----- Fonctions usuelles -----
\newcommand{\ch}{\mathop{\mathrm{ch}}\nolimits}
\newcommand{\sh}{\mathop{\mathrm{sh}}\nolimits}
\renewcommand{\tanh}{\mathop{\mathrm{th}}\nolimits}
\newcommand{\cotan}{\mathop{\mathrm{cotan}}\nolimits}
\newcommand{\Arcsin}{\mathop{\mathrm{Arcsin}}\nolimits}
\newcommand{\Arccos}{\mathop{\mathrm{Arccos}}\nolimits}
\newcommand{\Arctan}{\mathop{\mathrm{Arctan}}\nolimits}
\newcommand{\Argsh}{\mathop{\mathrm{Argsh}}\nolimits}
\newcommand{\Argch}{\mathop{\mathrm{Argch}}\nolimits}
\newcommand{\Argth}{\mathop{\mathrm{Argth}}\nolimits}
\newcommand{\pgcd}{\mathop{\mathrm{pgcd}}\nolimits} 

 %----- Commandes divers ------
\newcommand{\ii}{\mathrm{i}}
\newcommand{\dd}{\mathrm{d}}
\newcommand{\Ker}{\mathop{\text{Ker}}\nolimits}
\newcommand{\id}{\mathop{\text{id}}\nolimits}
\newcommand{\Card}{\mathop{\text{Card}}\nolimits}
\newcommand{\Vect}{\mathop{\text{Vect}}\nolimits}
\newcommand{\Mat}{\mathop{\mathrm{Mat}}\nolimits}
\newcommand{\rg}{\mathop{\text{Ker}}\nolimits}
\newcommand{\tr}{\mathop{\text{tr}}\nolimits}
\newcommand{\ppcm}{\mathop{\text{ppcm}}\nolimits}


%----- Definition d'un terme -----
\newcommand{\defi}[1]{{\color{myorange}\textbf{\emph{#1}}}}
\newcommand{\evidence}[1]{{\color{blue}\textbf{\emph{#1}}}}
\newcommand{\assertion}[1]{{\og\emph{#1}\fg}} % pour chapitre logique


%-----  Package liens hypertexts ----- 
\usepackage{hyperref}
\hypersetup{colorlinks=true, linkcolor=blue, urlcolor=blue,
pdftitle={Exo7 - Cours de mathématiques}, pdfauthor={Exo7}}

%----- Commandes tikz -----
\usepackage{tikz}
\usepackage{pgfplots}
\usetikzlibrary{calc}
\usetikzlibrary{shadows}
\usetikzlibrary{arrows}
\usetikzlibrary{patterns}
\usetikzlibrary{matrix}

%-----  Multi colonnes- ---- 
\usepackage{multicol}
\setlength{\columnseprule}{0.2mm}

%-----  Package d'importation ----- 
\usepackage{import}

%----- Format de la page ------
\setlength{\parindent}{0cm}




%Link to video Youtube

% variable myvideo : 0 no video, otherwise youtube reference
\newcommand{\video}[1]{\def\myvideo{#1}}
\newcommand{\insertvideo}[2]{\video{#1}%
{\small\texttt{\href{http://www.youtube.com/watch?v=\myvideo}{Vidéo $\blacksquare$ #2}}}}

% Liens vers les fiches d'exercices
\newcommand{\mafiche}[1]{\def\mymafiche{#1}}
\newcommand{\insertfiche}[2]{\mafiche{#1}%
{\small\texttt{\href{http://exo7.emath.fr/ficpdf/\mymafiche}{Fiche d'exercices $\blacklozenge$ #2}}}}



%----- Logo Exo7 ------
\definecolor{myred}{rgb}{0.93,0.26,0}
\definecolor{myorange}{rgb}{0.97,0.58,0}
\definecolor{myyellow}{rgb}{1,0.86,0}

\newcommand{\LogoExoSept}[1]{  % input : echelle
{\usefont{U}{cmss}{bx}{n}
\begin{tikzpicture}[scale=0.1*#1,transform shape]
  \fill[color=myorange] (0,0)--(4,0)--(4,-4)--(0,-4)--cycle;
  \fill[color=myred] (0,0)--(0,3)--(-3,3)--(-3,0)--cycle;
  \fill[color=myyellow] (4,0)--(7,4)--(3,7)--(0,3)--cycle;
  \node[scale=5] at (3.5,3.5) {Exo7};
\end{tikzpicture}}
}

%------ Titre livre -------------
\newcommand{\montitre}[1]{
\begin{center}
\Huge #1  
\end{center}
\begin{center}
\LogoExoSept{5}
\end{center}
\tableofcontents
\addtocontents{toc}{\setcounter{tocdepth}{1}}
}



%------ Chapitre -------------

\newcommand{\chapitre}[1]{          % pour chapitre dans livre
\chapter{#1}
}

\newcommand{\finchapitre}{}
% \newcommand{\finchapitre}{\end{document}} % pour chapitre seul



%----- Personnalisation pour les theoremes,... -----
\theoremstyle{definition}
\newtheorem{proposition}{Proposition}
\newtheorem*{propriete*}{Propriété}
\newtheorem{theoreme}{Théorème}
\newtheorem{definition}{Définition}
\newtheorem{corollaire}{Corollaire}
\newtheorem{exemple}{Exemple}
\newtheorem{exercicecours}{Exercice}
\newtheorem*{miniexercices}{Mini-exercices}
\newtheorem{lemme}{Lemme}
\newtheorem*{remarque*}{Remarque}
\newtheorem{tp}{Travaux pratiques}

%----- Commandes anti-beamer -----
\newcommand{\pause}{}  % permet de mettre des \pause dans beamer pas dans poly
\newcommand{\beameronly}[1]{}


%------ Figures ------
\def\myscale{1} % par défaut 
\newcommand{\myfigure}[2]{  % entrée : echelle, fichier figure
\def\myscale{#1}\begin{center}\footnotesize{#2}\end{center}}


%------ Encadrement des formules ------
\usepackage{fancybox}
%\setlength{\fboxsep}{7pt}
\newcommand{\mybox}[1]{\begin{center}\shadowbox{#1}\end{center}}
\newcommand{\myboxinline}[1]{\raisebox{-2ex}{\shadowbox{#1}}}


%------ Encadrement auteurs ------
\newcommand{\auteurs}[1]{\textbf{Auteurs} \\
 #1
}

%------ Algorithmes ------

\newcommand{\Python}{\texttt{Python}}
\newcommand{\Sage}{\texttt{Sage}}


% Pour afficher du code
\usepackage{listingsutf8}

\lstset{
  language=Python,
  upquote=true,
  columns=flexible,
  keepspaces=true,
  basicstyle=\ttfamily,
  commentstyle=\color{gray},
  showspaces=false
  showstringspaces=false
}


% \newcommand{\insertcode}[1]{\hrulefill\quad \texttt{#1}\newline
% \lstinputlisting[inputencoding=utf8/latin1]{Algos/#1}\hrulefill}
\newcommand{\insertcode}[2]{
{\hfill\texttt{\color{gray}#2}}
\begin{center}
\begin{minipage}{0.9\textwidth}
\lstinputlisting[inputencoding=utf8/latin1]{../#1} 
\end{minipage}  
\end{center}
}

\newcommand{\codeinline}[1]{\lstinline!#1!}