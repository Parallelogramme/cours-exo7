
%%%%%%%%%%%%%%%%%% PREAMBULE %%%%%%%%%%%%%%%%%%

\documentclass[aspectratio=169,utf8]{beamer}
%\documentclass[aspectratio=169,handout]{beamer}

\usetheme{Boadilla}
%\usecolortheme{seahorse}
\usecolortheme[RGB={245,66,24}]{structure}
\useoutertheme{infolines}

% packages
\usepackage{amsfonts,amsmath,amssymb,amsthm}
\usepackage[utf8]{inputenc}
\usepackage[T1]{fontenc}
\usepackage{lmodern}

\usepackage[francais]{babel}
\usepackage{fancybox}
\usepackage{graphicx}

\usepackage{float}
\usepackage{xfrac}

%\usepackage[usenames, x11names]{xcolor}
\usepackage{tikz}
\usepackage{pgfplots}
\usepackage{datetime}



%-----  Package unités -----
\usepackage{siunitx}
\sisetup{locale = FR,detect-all,per-mode = symbol}

%\usepackage{mathptmx}
%\usepackage{fouriernc}
%\usepackage{newcent}
%\usepackage[mathcal,mathbf]{euler}

%\usepackage{palatino}
%\usepackage{newcent}
% \usepackage[mathcal,mathbf]{euler}



% \usepackage{hyperref}
% \hypersetup{colorlinks=true, linkcolor=blue, urlcolor=blue,
% pdftitle={Exo7 - Exercices de mathématiques}, pdfauthor={Exo7}}


%section
% \usepackage{sectsty}
% \allsectionsfont{\bf}
%\sectionfont{\color{Tomato3}\upshape\selectfont}
%\subsectionfont{\color{Tomato4}\upshape\selectfont}

%----- Ensembles : entiers, reels, complexes -----
\newcommand{\Nn}{\mathbb{N}} \newcommand{\N}{\mathbb{N}}
\newcommand{\Zz}{\mathbb{Z}} \newcommand{\Z}{\mathbb{Z}}
\newcommand{\Qq}{\mathbb{Q}} \newcommand{\Q}{\mathbb{Q}}
\newcommand{\Rr}{\mathbb{R}} \newcommand{\R}{\mathbb{R}}
\newcommand{\Cc}{\mathbb{C}} 
\newcommand{\Kk}{\mathbb{K}} \newcommand{\K}{\mathbb{K}}

%----- Modifications de symboles -----
\renewcommand{\epsilon}{\varepsilon}
\renewcommand{\Re}{\mathop{\text{Re}}\nolimits}
\renewcommand{\Im}{\mathop{\text{Im}}\nolimits}
%\newcommand{\llbracket}{\left[\kern-0.15em\left[}
%\newcommand{\rrbracket}{\right]\kern-0.15em\right]}

\renewcommand{\ge}{\geqslant}
\renewcommand{\geq}{\geqslant}
\renewcommand{\le}{\leqslant}
\renewcommand{\leq}{\leqslant}
\renewcommand{\epsilon}{\varepsilon}

%----- Fonctions usuelles -----
\newcommand{\ch}{\mathop{\text{ch}}\nolimits}
\newcommand{\sh}{\mathop{\text{sh}}\nolimits}
\renewcommand{\tanh}{\mathop{\text{th}}\nolimits}
\newcommand{\cotan}{\mathop{\text{cotan}}\nolimits}
\newcommand{\Arcsin}{\mathop{\text{arcsin}}\nolimits}
\newcommand{\Arccos}{\mathop{\text{arccos}}\nolimits}
\newcommand{\Arctan}{\mathop{\text{arctan}}\nolimits}
\newcommand{\Argsh}{\mathop{\text{argsh}}\nolimits}
\newcommand{\Argch}{\mathop{\text{argch}}\nolimits}
\newcommand{\Argth}{\mathop{\text{argth}}\nolimits}
\newcommand{\pgcd}{\mathop{\text{pgcd}}\nolimits} 


%----- Commandes divers ------
\newcommand{\ii}{\mathrm{i}}
\newcommand{\dd}{\text{d}}
\newcommand{\id}{\mathop{\text{id}}\nolimits}
\newcommand{\Ker}{\mathop{\text{Ker}}\nolimits}
\newcommand{\Card}{\mathop{\text{Card}}\nolimits}
\newcommand{\Vect}{\mathop{\text{Vect}}\nolimits}
\newcommand{\Mat}{\mathop{\text{Mat}}\nolimits}
\newcommand{\rg}{\mathop{\text{rg}}\nolimits}
\newcommand{\tr}{\mathop{\text{tr}}\nolimits}


%----- Structure des exercices ------

\newtheoremstyle{styleexo}% name
{2ex}% Space above
{3ex}% Space below
{}% Body font
{}% Indent amount 1
{\bfseries} % Theorem head font
{}% Punctuation after theorem head
{\newline}% Space after theorem head 2
{}% Theorem head spec (can be left empty, meaning ‘normal’)

%\theoremstyle{styleexo}
\newtheorem{exo}{Exercice}
\newtheorem{ind}{Indications}
\newtheorem{cor}{Correction}


\newcommand{\exercice}[1]{} \newcommand{\finexercice}{}
%\newcommand{\exercice}[1]{{\tiny\texttt{#1}}\vspace{-2ex}} % pour afficher le numero absolu, l'auteur...
\newcommand{\enonce}{\begin{exo}} \newcommand{\finenonce}{\end{exo}}
\newcommand{\indication}{\begin{ind}} \newcommand{\finindication}{\end{ind}}
\newcommand{\correction}{\begin{cor}} \newcommand{\fincorrection}{\end{cor}}

\newcommand{\noindication}{\stepcounter{ind}}
\newcommand{\nocorrection}{\stepcounter{cor}}

\newcommand{\fiche}[1]{} \newcommand{\finfiche}{}
\newcommand{\titre}[1]{\centerline{\large \bf #1}}
\newcommand{\addcommand}[1]{}
\newcommand{\video}[1]{}

% Marge
\newcommand{\mymargin}[1]{\marginpar{{\small #1}}}

\def\noqed{\renewcommand{\qedsymbol}{}}


%----- Presentation ------
\setlength{\parindent}{0cm}

%\newcommand{\ExoSept}{\href{http://exo7.emath.fr}{\textbf{\textsf{Exo7}}}}

\definecolor{myred}{rgb}{0.93,0.26,0}
\definecolor{myorange}{rgb}{0.97,0.58,0}
\definecolor{myyellow}{rgb}{1,0.86,0}

\newcommand{\LogoExoSept}[1]{  % input : echelle
{\usefont{U}{cmss}{bx}{n}
\begin{tikzpicture}[scale=0.1*#1,transform shape]
  \fill[color=myorange] (0,0)--(4,0)--(4,-4)--(0,-4)--cycle;
  \fill[color=myred] (0,0)--(0,3)--(-3,3)--(-3,0)--cycle;
  \fill[color=myyellow] (4,0)--(7,4)--(3,7)--(0,3)--cycle;
  \node[scale=5] at (3.5,3.5) {Exo7};
\end{tikzpicture}}
}


\newcommand{\debutmontitre}{
  \author{} \date{} 
  \thispagestyle{empty}
  \hspace*{-10ex}
  \begin{minipage}{\textwidth}
    \titlepage  
  \vspace*{-2.5cm}
  \begin{center}
    \LogoExoSept{2.5}
  \end{center}
  \end{minipage}

  \vspace*{-0cm}
  
  % Astuce pour que le background ne soit pas discrétisé lors de la conversion pdf -> png
\begin{tikzpicture}
        \fill[opacity=0,green!60!black] (0,0)--++(0,0)--++(0,0)--++(0,0)--cycle; 
\end{tikzpicture}

% toc S'affiche trop tot :
% \tableofcontents[hideallsubsections, pausesections]
}

\newcommand{\finmontitre}{
  \end{frame}
  \setcounter{framenumber}{0}
} % ne marche pas pour une raison obscure

%----- Commandes supplementaires ------

% \usepackage[landscape]{geometry}
% \geometry{top=1cm, bottom=3cm, left=2cm, right=10cm, marginparsep=1cm
% }
% \usepackage[a4paper]{geometry}
% \geometry{top=2cm, bottom=2cm, left=2cm, right=2cm, marginparsep=1cm
% }

%\usepackage{standalone}


% New command Arnaud -- november 2011
\setbeamersize{text margin left=24ex}
% si vous modifier cette valeur il faut aussi
% modifier le decalage du titre pour compenser
% (ex : ici =+10ex, titre =-5ex

\theoremstyle{definition}
%\newtheorem{proposition}{Proposition}
%\newtheorem{exemple}{Exemple}
%\newtheorem{theoreme}{Théorème}
%\newtheorem{lemme}{Lemme}
%\newtheorem{corollaire}{Corollaire}
%\newtheorem*{remarque*}{Remarque}
%\newtheorem*{miniexercice}{Mini-exercices}
%\newtheorem{definition}{Définition}

% Commande tikz
\usetikzlibrary{calc}
\usetikzlibrary{patterns,arrows}
\usetikzlibrary{matrix}
\usetikzlibrary{fadings} 

%definition d'un terme
\newcommand{\defi}[1]{{\color{myorange}\textbf{\emph{#1}}}}
\newcommand{\evidence}[1]{{\color{blue}\textbf{\emph{#1}}}}
\newcommand{\assertion}[1]{\emph{\og#1\fg}}  % pour chapitre logique
%\renewcommand{\contentsname}{Sommaire}
\renewcommand{\contentsname}{}
\setcounter{tocdepth}{2}



%------ Figures ------

\def\myscale{1} % par défaut 
\newcommand{\myfigure}[2]{  % entrée : echelle, fichier figure
\def\myscale{#1}
\begin{center}
\footnotesize
{#2}
\end{center}}


%------ Encadrement ------

\usepackage{fancybox}


\newcommand{\mybox}[1]{
\setlength{\fboxsep}{7pt}
\begin{center}
\shadowbox{#1}
\end{center}}

\newcommand{\myboxinline}[1]{
\setlength{\fboxsep}{5pt}
\raisebox{-10pt}{
\shadowbox{#1}
}
}

%--------------- Commande beamer---------------
\newcommand{\beameronly}[1]{#1} % permet de mettre des pause dans beamer pas dans poly


\setbeamertemplate{navigation symbols}{}
\setbeamertemplate{footline}  % tiré du fichier beamerouterinfolines.sty
{
  \leavevmode%
  \hbox{%
  \begin{beamercolorbox}[wd=.333333\paperwidth,ht=2.25ex,dp=1ex,center]{author in head/foot}%
    % \usebeamerfont{author in head/foot}\insertshortauthor%~~(\insertshortinstitute)
    \usebeamerfont{section in head/foot}{\bf\insertshorttitle}
  \end{beamercolorbox}%
  \begin{beamercolorbox}[wd=.333333\paperwidth,ht=2.25ex,dp=1ex,center]{title in head/foot}%
    \usebeamerfont{section in head/foot}{\bf\insertsectionhead}
  \end{beamercolorbox}%
  \begin{beamercolorbox}[wd=.333333\paperwidth,ht=2.25ex,dp=1ex,right]{date in head/foot}%
    % \usebeamerfont{date in head/foot}\insertshortdate{}\hspace*{2em}
    \insertframenumber{} / \inserttotalframenumber\hspace*{2ex} 
  \end{beamercolorbox}}%
  \vskip0pt%
}


\definecolor{mygrey}{rgb}{0.5,0.5,0.5}
\setlength{\parindent}{0cm}
%\DeclareTextFontCommand{\helvetica}{\fontfamily{phv}\selectfont}

% background beamer
\definecolor{couleurhaut}{rgb}{0.85,0.9,1}  % creme
\definecolor{couleurmilieu}{rgb}{1,1,1}  % vert pale
\definecolor{couleurbas}{rgb}{0.85,0.9,1}  % blanc
\setbeamertemplate{background canvas}[vertical shading]%
[top=couleurhaut,middle=couleurmilieu,midpoint=0.4,bottom=couleurbas] 
%[top=fondtitre!05,bottom=fondtitre!60]



\makeatletter
\setbeamertemplate{theorem begin}
{%
  \begin{\inserttheoremblockenv}
  {%
    \inserttheoremheadfont
    \inserttheoremname
    \inserttheoremnumber
    \ifx\inserttheoremaddition\@empty\else\ (\inserttheoremaddition)\fi%
    \inserttheorempunctuation
  }%
}
\setbeamertemplate{theorem end}{\end{\inserttheoremblockenv}}

\newenvironment{theoreme}[1][]{%
   \setbeamercolor{block title}{fg=structure,bg=structure!40}
   \setbeamercolor{block body}{fg=black,bg=structure!10}
   \begin{block}{{\bf Th\'eor\`eme }#1}
}{%
   \end{block}%
}


\newenvironment{proposition}[1][]{%
   \setbeamercolor{block title}{fg=structure,bg=structure!40}
   \setbeamercolor{block body}{fg=black,bg=structure!10}
   \begin{block}{{\bf Proposition }#1}
}{%
   \end{block}%
}

\newenvironment{corollaire}[1][]{%
   \setbeamercolor{block title}{fg=structure,bg=structure!40}
   \setbeamercolor{block body}{fg=black,bg=structure!10}
   \begin{block}{{\bf Corollaire }#1}
}{%
   \end{block}%
}

\newenvironment{mydefinition}[1][]{%
   \setbeamercolor{block title}{fg=structure,bg=structure!40}
   \setbeamercolor{block body}{fg=black,bg=structure!10}
   \begin{block}{{\bf Définition} #1}
}{%
   \end{block}%
}

\newenvironment{lemme}[0]{%
   \setbeamercolor{block title}{fg=structure,bg=structure!40}
   \setbeamercolor{block body}{fg=black,bg=structure!10}
   \begin{block}{\bf Lemme}
}{%
   \end{block}%
}

\newenvironment{remarque}[1][]{%
   \setbeamercolor{block title}{fg=black,bg=structure!20}
   \setbeamercolor{block body}{fg=black,bg=structure!5}
   \begin{block}{Remarque #1}
}{%
   \end{block}%
}


\newenvironment{exemple}[1][]{%
   \setbeamercolor{block title}{fg=black,bg=structure!20}
   \setbeamercolor{block body}{fg=black,bg=structure!5}
   \begin{block}{{\bf Exemple }#1}
}{%
   \end{block}%
}


\newenvironment{miniexercice}[0]{%
   \setbeamercolor{block title}{fg=structure,bg=structure!20}
   \setbeamercolor{block body}{fg=black,bg=structure!5}
   \begin{block}{Mini-exercices}
}{%
   \end{block}%
}


\newenvironment{tp}[0]{%
   \setbeamercolor{block title}{fg=structure,bg=structure!40}
   \setbeamercolor{block body}{fg=black,bg=structure!10}
   \begin{block}{\bf Travaux pratiques}
}{%
   \end{block}%
}
\newenvironment{exercicecours}[1][]{%
   \setbeamercolor{block title}{fg=structure,bg=structure!40}
   \setbeamercolor{block body}{fg=black,bg=structure!10}
   \begin{block}{{\bf Exercice }#1}
}{%
   \end{block}%
}
\newenvironment{algo}[1][]{%
   \setbeamercolor{block title}{fg=structure,bg=structure!40}
   \setbeamercolor{block body}{fg=black,bg=structure!10}
   \begin{block}{{\bf Algorithme}\hfill{\color{gray}\texttt{#1}}}
}{%
   \end{block}%
}


\setbeamertemplate{proof begin}{
   \setbeamercolor{block title}{fg=black,bg=structure!20}
   \setbeamercolor{block body}{fg=black,bg=structure!5}
   \begin{block}{{\footnotesize Démonstration}}
   \footnotesize
   \smallskip}
\setbeamertemplate{proof end}{%
   \end{block}}
\setbeamertemplate{qed symbol}{\openbox}


\makeatother
\usecolortheme[RGB={153,0,0}]{structure}


%%%%%%%%%%%%%%%%%%%%%%%%%%%%%%%%%%%%%%%%%%%%%%%%%%%%%%%%%%%%%
%%%%%%%%%%%%%%%%%%%%%%%%%%%%%%%%%%%%%%%%%%%%%%%%%%%%%%%%%%%%%

\begin{document}

\title{{\bf Ensembles et applications}}
\subtitle{Relation d'équivalence}

\begin{frame}
  
  \debutmontitre

  \pause

{\footnotesize
\hfill
\setbeamercovered{transparent=50}
\begin{minipage}{0.6\textwidth}
  \begin{itemize}
    \item<3-> Définition
    \item<4-> Exemples
    \item<5-> Classes d'équivalence
    \item<6-> L’ensemble $\Zz/n\Zz$
  \end{itemize}
\end{minipage}
}

\end{frame}
\setcounter{framenumber}{0}

%---------------------------------------------------------------
\section{Relation d'équivalence}

\begin{frame}



Une \defi{relation} $\mathcal{R}$ sur un ensemble $E$, c'est associer à tout couple 
$(x,y)\in E \times E$ la valeur <<Vrai>> (s'ils sont en relation) et la valeur <<Faux>> sinon

\pause

On note $x\mathcal{R}y$ si $x$ et $y$ sont en relation 

\pause

\myfigure{1.5}{
\tikzinput{fig_ensembles18} 
}

\end{frame}

\begin{frame}

\begin{mydefinition}
$\mathcal{R}$ est une \defi{relation d'équivalence} si
\begin{itemize}
  \item $\forall x \in E$, $x \mathcal{R}x$ \qquad  (\defi{réflexivité})

 

\uncover<3->{
  \item $\forall x,y \in E$, $x \mathcal{R}y \implies y\mathcal{R}x$ \qquad  (\defi{symétrie})
}



\uncover<5->{
  \item $\forall x,y,z \in E$, $x \mathcal{R}y \text{ et }  y\mathcal{R}z \implies x\mathcal{R}z$ \qquad  (\defi{transitivité}) 
}
\end{itemize}
\end{mydefinition}

\bigskip

\begin{minipage}{0.5\textwidth}
\myfigure{1.3}{
\uncover<2->{\tikzinput{fig_ensembles19a} \;}
\uncover<4->{\tikzinput{fig_ensembles19b} \;}
\uncover<6->{\tikzinput{fig_ensembles19c} }
}  
\end{minipage}
\begin{minipage}{0.49\textwidth}
\myfigure{1}{
\uncover<7->{\tikzinput{fig_ensembles19} }
}  
\end{minipage}

\end{frame}

%---------------------------------------------------------------
\section{Exemples}

\begin{frame}

\begin{exemple}
La relation $\mathcal{R}$ <<\emph{être parallèle}>> est une relation d'équivalence 
pour l'ensemble $E$ des droites du plan

\pause

  \begin{itemize}
    \item réflexivité : une droite est parallèle à elle-même
\pause
    \item symétrie : si $D$ est parallèle à $D'$ alors $D'$ est parallèle à $D$
\pause
    \item transitivité : si $D$ parallèle à $D'$ et $D'$ parallèle à $D''$ 
alors $D$ est parallèle à $D''$
  \end{itemize}
\end{exemple}

\pause

\begin{exemple}
\begin{enumerate}
  \item La relation <<\emph{être du même âge}>> est une relation d'équivalence

\pause
  
  \item La relation <<\emph{être perpendiculaire}>> n'est pas une relation d'équivalence 
 
\pause

  \item La relation $\le$ (sur $E=\Rr$ par exemple) n'est pas une relation d'équivalence

\end{enumerate}
\end{exemple}

\end{frame}

%---------------------------------------------------------------
\section{Classes d'équivalence}

\begin{frame}

Soit $E$ un ensemble et $\mathcal{R}$ une relation d'équivalence sur $E$

\begin{mydefinition}
La \defi{classe d'équivalence} de $x \in E$ est 
\myboxinline{$\text{cl}(x) = \big\{ y  \in E \mid y\mathcal{R}x \big\}$}
\end{mydefinition}

\pause

\myfigure{1}{
\tikzinput{fig_ensembles20} 
}

\pause

\end{frame}

%---------------------------------------------------------------

\begin{frame}

L’ensemble $\text{cl}(x)$ est un sous-ensemble de $E$, on le note aussi $\overline{x}$

\pause
\medskip

Si $y\in\text{cl}(x)$, $y$ est un \defi{représentant} de $\text{cl}(x)$

\pause
\medskip

\begin{proposition}
\begin{enumerate}
  \item $x\mathcal{R}y  \Longleftrightarrow  \text{cl}(x)=\text{cl}(y)$

\pause

  \item Pour tout $x,y \in E$, $\text{cl}(x)=\text{cl}(y)$ ou 
$\text{cl}(x) \cap \text{cl}(y) = \varnothing$

\pause

  \item Soit $C$ un ensemble de représentants de toutes les classes 
alors $\big\{ \text{cl}(x) \mid x \in C \big\}$
forme une partition de $E$
\end{enumerate} 
\end{proposition}


\pause
\bigskip

\begin{minipage}{0.5\textwidth}
Une \defi{partition} de $E$ est un ensemble $\{E_i\}$ de parties de $E$ tel que \\
$E = \bigcup_i E_i$ et $E_i\cap E_j = \varnothing$ (si $i\neq j$)  
\end{minipage}\quad\quad
\begin{minipage}{0.39\textwidth}
\myfigure{1}{
\tikzinput{fig_ensembles21} 
}  
\end{minipage}



\end{frame}

%---------------------------------------------------------------

\begin{frame}

\begin{exemple}
Pour la relation <<\emph{être du même âge}>>, la classe d'équivalence d'une personne est l'ensemble
des personnes ayant le même âge 
\pause
  \begin{itemize}
    \item On est dans la même classe d'équivalence si et seulement si on est du même âge

\pause
    \item Deux personnes appartiennent soit à la même classe, soit à des classes disjointes

\pause
    \item Si on choisit une personne de chaque âge possible, cela forme un ensemble de représentants $C$

Une personne quelconque appartient à une et une seule classe d'un des représentants
  \end{itemize}

\end{exemple}

\pause

\begin{exemple}
Pour la relation <<\emph{être parallèle}>>, la classe d'équivalence d'une droite est l'ensemble des droites parallèles.
À chaque classe d'équivalence correspond une et une seule direction
\end{exemple}

\end{frame}


%---------------------------------------------------------------
\section{L'ensemble $\Zz/n\Zz$}

\begin{frame}
Soit $E = \Zz$, fixons $n\ge 2$. On définit la relation :
\mybox{$a \equiv b \pmod n \quad \Longleftrightarrow \quad a-b \text{ est un multiple de } n$}

\pause
\bigskip

Exemple ($n=7$) \quad
\begin{tabular}{ll}
$10 \equiv 3 \pmod 7$ & \pause $19 \equiv 5 \pmod 7$ \\ \pause
$77 \equiv 0 \pmod 7$ & \pause $-1 \equiv 20 \pmod 7$ \\
\end{tabular}

\pause
\bigskip

C'est une relation d'équivalence

\pause

\begin{itemize}
  \item $a\equiv a \pmod n$

\pause

  \item Si $a \equiv b \pmod n$ alors il existe $k\in \Zz$ tel que
$a-b=kn$ donc $b-a = (-k)n$ et ainsi $b\equiv a \pmod n$

\pause

  \item Si $a \equiv b \pmod n$ et $b \equiv c \pmod n$ 

on écrit $a-b=kn$ et $b-c=k'n$

D'où $a-c = (a-b) + (b-c) = (k+k')n$ donc $a \equiv c \pmod n$
\end{itemize}
\end{frame}


\begin{frame}
$$\text{cl}(a)=\overline a = \big\{ b \in \Zz \mid b \equiv a \pmod n \big\}$$
\pause
\vspace*{-3ex}
\mybox{$\overline a = \big\{ a+kn \mid k\in \Zz \big\}= a + n\Zz $}

\pause

$$\overline n = \overline 0,\quad  \overline {n+1} = \overline 1, \quad \overline {n+2} = \overline 2, \ldots$$

\pause

Il y a $n$ classes d'équivalence
\mybox{$\Zz/n\Zz = \big\{ \overline 0, \overline 1, \overline 2, \ldots, \overline{n-1} \big\}$}

\pause

Exemple $\Zz/7\Zz = \big\{  \overline 0, \overline 1, \overline 2,\ldots, \overline{6} \big\}$

\pause

\qquad $\overline 0 = \{\ldots,-7,0,7,14,21,\ldots\} = 7\Zz$ \only<9->{$= \overline 7$}

\pause

\qquad $\overline 1 = \{\ldots, -6, 1, 8, 15, \ldots \} = 1+ 7\Zz$

\pause

\qquad \ldots 

\qquad $\overline 6 = \{\ldots,-8, -1, 6, 13, \ldots \} = 6+7\Zz$

\pause % pour \only<5->


\end{frame}

%---------------------------------------------------------------
\section{Mini-exercices}

\begin{frame}
\begin{miniexercice}
\begin{enumerate}
  \item Montrer que la relation définie sur $\Nn$ par $x \mathcal{R} y \iff \frac{2x+y}{3} \in \Nn$ est une relation d'équivalence.
Montrer qu'il y a $3$ classes d'équivalence.

  \item Dans $\Rr^2$ montrer que la relation définie par $(x,y) \mathcal{R} (x',y') \iff x+y'=x'+y$ est une relation d'équivalence.
Montrer que deux points $(x,y)$ et $(x',y')$ sont dans une même classe si et seulement s'ils appartiennent à une même droite 
dont vous déterminerez la direction.

  \item On définit une addition sur $\Zz/n\Zz$ par $\overline{p} + \overline{q}= \overline{p + q}$.
Calculer la table d'addition dans $\Zz/6\Zz$ (c'est-à-dire toutes les sommes $\overline{p} + \overline{q}$
pour $\overline{p},\overline{q} \in \Zz/6\Zz$). Même chose avec la multiplication $\overline{p}\times \overline{q}= \overline{p\times q}$.
Mêmes questions avec $\Zz/5\Zz$, puis $\Zz/8\Zz$.
\end{enumerate}
\end{miniexercice}
\end{frame}



\end{document}