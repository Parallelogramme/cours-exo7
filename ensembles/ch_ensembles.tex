\documentclass[class=report,crop=false]{standalone}
\usepackage[screen]{../exo7book}

\begin{document}

%====================================================================
\chapitre{Ensembles et applications}
%====================================================================

\insertvideo{bPT6-g3B5wQ}{partie 1. Ensembles}

\insertvideo{Y8cV0zcFijs}{partie 2. Applications}

\insertvideo{1qax9qMxz4c}{partie 3. Injection, surjection, bijection}

\insertvideo{NElSI5_NIsk}{partie 4. Ensembles finis}

\insertvideo{WuJyP_7VIu8}{partie 5. Relation d'équivalence}

\insertfiche{fic00002.pdf}{Logique, ensembles, raisonnements}

\insertfiche{fic00003.pdf}{Injection, surjection, bijection}

\insertfiche{fic00005.pdf}{Dénombrement}

\insertfiche{fic00004.pdf}{Relation d'équivalence, relation d'ordre}


%%%%%%%%%%%%%%%%%%%%%%%%%%%%%%%%%%%%%%%%%%%%%%%%%%%%%%%%%%%%%%%%
\section*{Motivations}


Au début du \textsc{\romannumeral 20}\textsuperscript{e} siècle le professeur Frege peaufinait la rédaction
du second tome d'un ouvrage qui souhaitait refonder les mathématiques sur des bases logiques.
Il reçut une lettre d'un tout jeune mathématicien :
\emph{\og J'ai bien lu votre premier livre. Malheureusement vous supposez qu'il existe un ensemble
qui contient tous les ensembles. Un tel ensemble ne peut exister. \fg{}}
S'ensuit une démonstration de deux lignes. Tout le travail de Frege s'écroulait et
il ne s'en remettra jamais.
Le jeune Russell deviendra l'un des plus grands logiciens et philosophes de son temps.
Il obtient le prix Nobel de littérature en 1950.

Voici le \og paradoxe de Russell \fg{} pour montrer que l'ensemble de tous les ensembles ne peut exister.
C'est très bref, mais difficile à appréhender.
Par l'absurde, supposons qu'un tel ensemble $\mathcal{E}$ contenant tous les ensembles existe.
Considérons
$$F = \Big\lbrace E \in \mathcal{E} \mid E \notin E \Big\rbrace.$$
Expliquons l'écriture $E \notin E$ :
le $E$ de gauche est considéré comme un élément, en effet l'ensemble $\mathcal{E}$
est l'ensemble de tous les ensembles et $E$ est un élément de cet ensemble ;
le $E$ de droite est considéré comme un ensemble, en effet les élément de $\mathcal{E}$
sont des ensembles ! On peut donc s'interroger si l'élément $E$ appartient à l'ensemble $E$.
Si non, alors par définition on met $E$ dans l'ensemble $F$.

La contradiction arrive lorsque l'on se pose la question suivante : a-t-on $F \in F$ ou $F \notin F$ ?
L'une des deux affirmation doit être vraie. Et pourtant :
\begin{itemize}
  \item Si $F \in F$ alors par définition de $F$, $F$ est l'un des ensembles $E$ tel que $F \notin F$. Ce qui est contradictoire.
  \item Si $F \notin F$ alors $F$ vérifie bien la propriété définissant $F$ donc $F \in F$ ! Encore contradictoire.
\end{itemize}
Aucun des cas n'est possible. On en déduit qu'il ne peut exister un tel ensemble $\mathcal{E}$
contenant tous les ensembles.

Ce paradoxe a été popularisé par l’énigme suivante :
\emph{\og Dans une ville, le barbier rase tous ceux qui ne se rasent pas eux-mêmes.
Qui rase le barbier ? \fg{}}
La seule réponse valable est qu'une telle situation ne peut exister.

\bigskip


Ne vous inquiétez pas, Russell et d'autres ont fondé la logique et les ensembles sur des bases
solides. Cependant il n'est pas possible dans ce cours de tout redéfinir.
Heureusement, vous connaissez déjà quelques ensembles :
\begin{itemize}
  \item l'ensemble des entiers naturels $\Nn =\{0,1,2,3,\ldots\}$.
  \item l'ensemble des entiers relatifs $\Zz = \{\ldots, -2,-1,0,1,2,\ldots\}$.
  \item l'ensemble des rationnels $\Qq = \big\{ \frac{p}{q} \mid p \in \Zz, q \in \Nn\setminus \{ 0\} \big\}.$
  \item l'ensemble des réels $\Rr$, par exemple $1, \sqrt 2$, $\pi$, $\ln(2)$,\ldots
  \item l'ensemble des nombres complexes $\Cc$.
\end{itemize}

\bigskip

Nous allons essayer de voir les propriétés des ensembles, sans s'attacher à un exemple particulier.
Vous vous apercevrez assez rapidement que ce qui est au moins aussi important que les
ensembles, ce sont les relations entre ensembles : ce sera la notion d'application (ou fonction)
entre deux ensembles.



%%%%%%%%%%%%%%%%%%%%%%%%%%%%%%%%%%%%%%%%%%%%%%%%%%%%%%%%%%%%%%%%
\section{Ensembles}

%---------------------------------------------------------------
\subsection{Définir des ensembles}

\begin{itemize}
  \item On va définir informellement ce qu'est un ensemble :
un \defi{ensemble}\index{ensemble} est une collection d'éléments.
  \item Exemples :
$$\{ 0, 1 \},\quad
\{ \text{rouge}, \text{noir} \}, \quad
\{0, 1, 2, 3,\ldots\} = \Nn.$$
  \item Un ensemble particulier est l'\defi{ensemble vide}\index{ensemble vide}, noté $\varnothing$\index{$\varnothing$}
qui est l'ensemble ne contenant aucun élément.
  \item On note
\mybox{$x \in E$}
\index{$\in$}
si $x$ est un élément de $E$, et $x \notin E$\index{$\notin$} dans le cas contraire.
  \item Voici une autre façon de définir des ensembles : une collection d'éléments qui vérifient
une propriété.
  \item Exemples :
$$\big\{ x \in \Rr \mid  |x-2| < 1 \big\}, \quad \big\{ z \in \Cc \mid z^5=1 \big\}, \quad
\big\{ x \in \Rr \mid 0 \le x \le 1 \big\}=[0,1].$$
\end{itemize}







%---------------------------------------------------------------
\subsection{Inclusion, union, intersection, complémentaire}

\begin{itemize}

  \item L'\defi{inclusion}\index{inclusion}. $E \subset F$\index{$\subset$} si tout élément de $E$ est aussi un élément de $F$.
Autrement dit: $\forall x \in E \; (x \in F)$. On dit alors que $E$ est un \defi{sous-ensemble}\index{sous-ensemble} de $F$
ou une \defi{partie}\index{partie} de $F$.

  \item L'\defi{égalité}. $E = F$ si et seulement si $E \subset F$ et $F \subset E$.

  \item \defi{Ensemble des parties} de $E$. On note $\mathcal{P}(E)$\index{$\mathcal{P}(E)$} l'ensemble des parties de $E$.
Par exemple si $E= \{1,2,3\}$ :
$$\mathcal{P}(\{1,2,3\}) =
\big\{ \varnothing, \{1\}, \{2\}, \{3\}, \{1,2\}, \{1,3\}, \{2,3\}, \{1,2,3\} \big\}.$$

  \item \defi{Complémentaire}\index{complementaire@complémentaire}. Si $A \subset E$,
\mybox{$\complement_E A = \big\{ x \in E \mid x \notin A \big\}$}
On le note aussi $E \setminus A$\index{$E \setminus A$} et juste $\complement A$\index{$\complement$}
s'il n'y a pas d'ambigu\"ité (et parfois aussi $A^c$ ou $\overline{A}$).

\myfigure{1}{
\tikzinput{fig_ensembles01}
}

  \item \defi{Union}\index{union}\index{$\cup$}. Pour $A, B \subset E$,
\mybox{$A \cup B = \big\{ x \in E \mid x \in A \text{ ou } x \in B \big\}$}
Le \emph{\og ou \fg} n'est pas exclusif : $x$ peut appartenir à $A$ et à $B$ en même temps.
\medskip
\myfigure{1}{
\tikzinput{fig_ensembles02}
}

  \item \defi{Intersection}\index{intersection}\index{$\cap$}.
\mybox{$A \cap B = \big\{ x \in E \mid x \in A \text{ et } x \in B \big\}$}
\medskip
\myfigure{1}{
\tikzinput{fig_ensembles03}
}

\end{itemize}


%---------------------------------------------------------------
\subsection{Règles de calculs}

Soient $A, B, C$ des parties d'un ensemble $E$.

\begin{itemize}
  \item $A\cap B = B \cap A$
  \item $A \cap (B \cap C) = (A \cap B) \cap C$
  \qquad (on peut donc écrire $A\cap B \cap C$ sans ambigüité)
  \item $A \cap \varnothing = \varnothing$, \quad $A \cap A= A$,  \quad $A \subset B \Longleftrightarrow A\cap B = A$
\end{itemize}

\medskip

\begin{itemize}
  \item $A\cup B = B \cup A$
  \item $A \cup (B \cup C) = (A \cup B) \cup C$
  \qquad (on peut donc écrire $A\cup B \cup C$ sans ambiguïté)
  \item $A \cup \varnothing = A$, \quad  $A \cup A= A$, \quad  $A \subset B \Longleftrightarrow A\cup B = B$
\end{itemize}

\medskip

\begin{itemize}
  \item \myboxinline{$A \cap (B \cup C) = (A \cap B) \cup (A \cap C)$}
  \item \myboxinline{$A \cup (B \cap C) = (A \cup B) \cap (A \cup C)$}
\end{itemize}

\medskip

\begin{itemize}
  \item $\complement \left( \complement A \right) = A$ \quad et donc
  \quad $A \subset B \Longleftrightarrow \complement B \subset \complement A$
  \item $\complement \left( A \cap B \right) = \complement A \cup \complement B$
  \item $\complement \left( A \cup B \right) = \complement A \cap \complement B$
\end{itemize}

\medskip
Voici les dessins pour les deux dernières assertions.

\myfigure{1}{
\tikzinput{fig_ensembles04a}
\qquad
\tikzinput{fig_ensembles04b}
}

\myfigure{1}{
\tikzinput{fig_ensembles04c}
\qquad
\tikzinput{fig_ensembles04d}
}

Les preuves sont pour l'essentiel une reformulation des opérateurs logiques, en voici quelques-unes :
\begin{itemize}
  \item Preuve de $A \cap (B \cup C) = (A \cap B) \cup (A \cap C)$:
$x \in A \cap (B \cup C)
\iff x \in A \text{ et } x \in (B \cup C)
\iff  x \in A \text{ et } (x \in B \text{ ou } x \in C)
\iff (x \in A \text{ et } x\in B) \text{ ou } (x \in A \text{ et } x \in C)
\iff (x \in A \cap B) \text{ ou } (x \in A \cap C)
\iff x \in (A\cap B) \cup (A\cap C)$.

  \item Preuve de $\complement \left( A \cap B \right) = \complement A \cup \complement B$:
$x \in \complement \left( A \cap B \right)
\iff x \notin \left( A \cap B \right)
\iff \text{non} \big(x \in A \cap B\big)
\iff \text{non} \big(x \in A \text{ et } x \in B\big)
\iff \text{non} (x \in A) \text{ ou } \text{non} (x \in B)
\iff x \notin A \text{ ou } x\notin B
\iff x \in \complement A \cup \complement B$.
\end{itemize}
Remarquez que l'on repasse aux éléments pour les preuves.

%---------------------------------------------------------------
\subsection{Produit cartésien}

Soient $E$ et $F$ deux ensembles.
Le \defi{produit cartésien}\index{produit cartesien@produit cartésien}, noté $E \times F$, est l'ensemble des couples $(x,y)$ où $x \in E$ et $y \in F$.

\begin{exemple}
\sauteligne
\begin{enumerate}
  \item Vous connaissez $\Rr^2 = \Rr \times \Rr= \big\{ (x,y) \mid x,y \in \Rr \big\}.$
  \item Autre exemple $[0,1] \times \Rr = \big\{ (x,y) \mid 0 \le x \le 1, y \in \Rr \big\}$

\myfigure{1}{
\tikzinput{fig_ensembles05}
}

  \item $[0,1] \times [0,1] \times [0,1] = \big\{ (x,y,z) \mid 0 \le x,y,z \le 1 \big\}$

\myfigure{1}{
\tikzinput{fig_ensembles06}
}
\end{enumerate}

\end{exemple}

%---------------------------------------------------------------
%\subsection{Mini-exercices}

\begin{miniexercices}
\sauteligne
\begin{enumerate}
  \item En utilisant les définitions, montrer : $A \neq B$ si et seulement s'il existe $a \in A \setminus B$ ou
$b \in B \setminus A$.
  \item Énumérer $\mathcal{P}(\{1,2,3,4\})$.

  \item Montrer $A \cup (B \cap C) = (A \cup B) \cap (A \cup C)$ et
$\complement \left( A \cup B \right) = \complement A \cap \complement B$.

  \item Énumérer $\{1,2,3\} \times \{1,2,3,4\}$.
  \item Représenter les sous-ensembles de $\Rr^2$ suivants :
$\big(]0,1[ \cup [2,3[\big) \times [-1,1]$,
$\big( \Rr \setminus (]0,1[ \cup [2,3[)\big) \times \big( (\Rr \setminus[-1,1]) \cap [0,2]  \big)$.
\end{enumerate}
\end{miniexercices}




%%%%%%%%%%%%%%%%%%%%%%%%%%%%%%%%%%%%%%%%%%%%%%%%%%%%%%%%%%%%%%%%
\section{Applications}

%---------------------------------------------------------------
\subsection{Définitions}

\begin{itemize}
  \item Une \defi{application}\index{application} (ou une \defi{fonction}\index{fonction}) $f : E \to F$,
c'est la donnée pour chaque élément $x \in E$ d'un unique élément de $F$ noté $f(x)$.


Nous représenterons les applications par deux types d'illustrations :
les ensembles \og patates \fg{}, l'ensemble de départ (et celui d'arrivée) est
schématisé par un ovale ses éléments par des points. L'association $x \mapsto f(x)$ est représentée par une flèche.

\myfigure{1}{
\tikzinput{fig_ensembles07a}
}


L'autre représentation est celle des fonctions continues de $\Rr$ dans $\Rr$ (ou des sous-ensembles de $\Rr$).
L'ensemble de départ $\Rr$ est représenté par l'axe des abscisses et celui d'arrivée par l'axe des ordonnées.
L'association $x \mapsto f(x)$ est représentée par le point $(x,f(x))$.

\myfigure{1}{
\tikzinput{fig_ensembles07b}
}

  \item \defi{Égalité}. Deux applications $f,g : E \to F$ sont égales si et seulement
si pour tout $x \in E$, $f(x)=g(x)$. On note alors $f=g$.

  \item Le \defi{graphe}\index{graphe} de $f : E \to F$ est
\mybox{$\Gamma_f = \Big\{ \big(x,f(x)\big) \in E \times F  \mid x \in E \Big\}$}
\myfigure{1}{
\tikzinput{fig_ensembles08}
}
  \item \defi{Composition}\index{composition}. Soient $f : E \to F$ et $g : F \to G$ alors $g \circ f : E \to G$ est l'application définie par $g \circ f (x) = g\big( f(x) \big)$.
    $$
      \begin{tikzpicture}[node distance=2cm, every edge/.style={draw,thick,->}]
         \node(E){$E$};
         \node[right of=E](F){$F$};
         \node[right of=F](G){$G$};
         \draw (E) edge (F);\draw (E) edge[myred, bend left=45] node[above]{$f$} (F);
         \draw (F) edge (G);\draw (F) edge[myred, bend left=45] node[above]{$g$} (G);
         \draw (E) edge[myred, bend right] node[below]{$g\circ f$} (G);
      \end{tikzpicture}
    $$
    
     \item \defi{Restriction}\index{restriction}. Soient $f : E \to F$ et $A\subset E$ alors la restriction de $f$ à $A$ est l'application 
$$\begin {array}{rccc}
f_{|_A}\ : \ & A &\longrightarrow& F \\
        &x & \longmapsto & f(x)\\
\end{array}$$
 
 
\end{itemize}



\begin{exemple}
\sauteligne
\begin{enumerate}
  \item L'\defi{identité}\index{identite@identité}, $\id_E : E \to E$ est simplement définie par $x \mapsto x$ et sera très utile dans la suite.
  \item Définissons  $f,g$ ainsi
$$\begin {array}{rccc}
f \ : \ & ]0,+\infty[ &\longrightarrow& ]0,+\infty[ \\
        &x & \longmapsto & \frac 1x \\
\end{array}, \quad
\begin {array}{rccc}
g \ : \ & ]0,+\infty[ &\longrightarrow& \Rr \\
        & x & \longmapsto & \frac{x-1}{x+1} \\
\end{array}.
$$
Alors $g \circ f  : \ ]0,+\infty[ \to \Rr$ vérifie pour tout $x \in ]0,+\infty[$ :
$$g\circ f(x) = g \big(f(x)\big) = g \left(\frac 1x \right) = \frac{\frac 1x - 1}{\frac 1x + 1} = \frac{1-x}{1+x} = - g(x).$$
\end{enumerate}
\end{exemple}

%---------------------------------------------------------------

\subsection{Image directe, image réciproque}

Soient $E, F$ deux ensembles.

\begin{definition}
 Soit $A \subset E$ et $f : E \to F$, l'\defi{image directe}\index{image directe} de $A$ par $f$ est l'ensemble
\mybox{$f(A) = \big\{ f(x) \mid x \in A \big\}$}
\end{definition}
\myfigure{1}{
\tikzinput{fig_ensembles09a}
\qquad \qquad
\tikzinput{fig_ensembles09b}
}

\begin{definition}
Soit $B \subset F$ et $f : E \to F$, l'\defi{image réciproque}\index{image reciproque@image réciproque} de $B$ par $f$ est l'ensemble
\mybox{$f^{-1}(B) = \big\{ x \in E \mid f(x) \in B \big\}$}
\end{definition}

\myfigure{1}{
\tikzinput{fig_ensembles10a}
\qquad \qquad
\tikzinput{fig_ensembles10b}
}


\begin{remarque*}
Ces notions sont plus difficiles à maîtriser qu'il n'y paraît !
\begin{itemize}
  \item $f(A)$ est un sous-ensemble de $F$, $f^{-1}(B)$ est un sous-ensemble de $E$.

  \item La notation \og $f^{-1}(B)$ \fg{} est un tout, rien ne dit que $f$
est un fonction bijective (voir plus loin). L'image réciproque existe
quelque soit la fonction.

  \item L'image directe d'un singleton $f(\{x\}) = \big\{ f(x) \big\}$ est un singleton.
Par contre l'image réciproque d'un singleton $f^{-1}\big( \{ y \} \big)$ dépend de $f$.
Cela peut être un singleton, un ensemble à plusieurs éléments ; mais cela peut-être
$E$ tout entier (si $f$ est une fonction constante) ou même l'ensemble vide
(si aucune image par $f$ ne vaut $y$).

\end{itemize}
\end{remarque*}

%---------------------------------------------------------------
\subsection{Antécédents}


Fixons $y \in F$. Tout élément $x\in E$ tel que $f(x)=y$ est un \defi{antécédent}\index{antecedent@antécédent}
de $y$.

En termes d'image réciproque l'ensemble des antécédents de $y$ est $f^{-1}(\{y\})$.

\bigskip

Sur les dessins suivants, l'élément $y$ admet $3$ antécédents par $f$. Ce sont $x_1$, $x_2$, $x_3$.

\myfigure{1}{
\tikzinput{fig_ensembles12a}
\qquad \qquad
\tikzinput{fig_ensembles12b}
}

%---------------------------------------------------------------
%\subsection{Mini-exercices}

\begin{miniexercices}
\sauteligne
\begin{enumerate}
  \item Pour deux applications $f,g : E \to F$, quelle est la négation de $f=g$ ?
  \item Représenter le graphe de $f : \Nn \to \Rr$ définie par $n \mapsto \frac{4}{n+1}$.
  \item Soient $f,g,h : \Rr \to \Rr$ définies par $f(x)=x^2$, $g(x)=2x+1$, $h(x)=x^3-1$. Calculer $f\circ (g \circ h)$ et
$(f \circ g ) \circ h$.
  \item Pour la fonction $f : \Rr \to \Rr$ définie par $x \mapsto x^2$ représenter et calculer les ensembles suivants :
$f([0,1[)$, $f(\Rr)$, $f(]-1,2[)$,  $f^{-1}([1,2[)$, $f^{-1}([-1,1])$, $f^{-1}(\{3\})$, $f^{-1}(\Rr \setminus \Nn)$.
\end{enumerate}
\end{miniexercices}


%%%%%%%%%%%%%%%%%%%%%%%%%%%%%%%%%%%%%%%%%%%%%%%%%%%%%%%%%%%%%%%%
\section{Injection, surjection, bijection}

%---------------------------------------------------------------
\subsection{Injection, surjection}

Soit $E, F$ deux ensembles et $f : E \to F$ une application.

\begin{definition}
$f$ est \defi{injective}\index{injection} si pour tout $x,x' \in E$ avec $f(x)=f(x')$ alors $x=x'$.
Autrement dit :
\mybox{$\forall x, x' \in E \quad \big( f(x)=f(x') \implies x=x'\big)$}
\end{definition}

\begin{definition}
$f$ est \defi{surjective}\index{surjection} si pour tout $y \in F$, il existe $x \in E$ tel que $y=f(x)$.
Autrement dit :
\mybox{$\forall y \in F \quad \exists x \in E \quad \big( y = f(x) \big)$}
\end{definition}

Une autre formulation : $f$ est surjective si et seulement si $f(E)=F$.

Les applications $f$ représentées sont injectives :
\myfigure{1}{
\tikzinput{fig_ensembles11a}
\qquad \qquad
\tikzinput{fig_ensembles11b}
}

Les applications $f$ représentées sont surjectives :
\myfigure{1}{
\tikzinput{fig_ensembles11c}
\qquad
\tikzinput{fig_ensembles11d}
}




\begin{remarque*}
Encore une fois ce sont des notions difficiles à appréhender.
Une autre façon de formuler l'injectivité et la surjectivité est d'utiliser les antécédents.
\begin{itemize}
  \item $f$ est injective si et seulement si tout élément $y$ de $F$ a \emph{au plus} un antécédent (et éventuellement aucun).
  \item $f$ est surjective si et seulement si tout élément $y$ de $F$ a \emph{au moins} un antécédent.
\end{itemize}
\end{remarque*}


\begin{remarque*}
Voici deux fonctions non injectives :

\myfigure{1}{
\tikzinput{fig_ensembles12c}
\qquad
\tikzinput{fig_ensembles12d}
}

\bigskip

Ainsi que deux fonctions non surjectives :
\myfigure{1}{
\tikzinput{fig_ensembles11e}
 \qquad \qquad
\tikzinput{fig_ensembles11f}
}
\end{remarque*}



\begin{exemple}
\sauteligne
\begin{enumerate}
  \item Soit $f_1 : \Nn \to \Qq$ définie par $f_1(x)= \frac{1}{1+x}$.
Montrons que $f_1$ est injective : soit $x,x'\in \Nn$ tels que $f_1(x)=f_1(x')$.
Alors $\frac{1}{1+x}=\frac{1}{1+x'}$, donc $1+x=1+x'$ et donc $x=x'$.
Ainsi $f_1$ est injective.

Par contre $f_1$ n'est pas surjective. Il s'agit de trouver un élément $y$
qui n'a pas d'antécédent par $f_1$. Ici il est facile de voir que l'on a toujours
$f_1(x) \le 1$ et donc par exemple $y=2$ n'a pas d'antécédent.
Ainsi $f_1$ n'est pas surjective.

  \item Soit $f_2 : \Zz \to \Nn$ définie par $f_2(x)=x^2$. Alors $f_2$ n'est pas injective.
En effet on peut trouver deux éléments $x,x' \in \Zz$ différents tels que $f_2(x)=f_2(x')$.
Il suffit de prendre par exemple $x=2$, $x'=-2$.

$f_2$ n'est pas non plus surjective,
en effet il existe des éléments $y \in \Nn$ qui n'ont aucun antécédent. Par exemple
$y=3$ : si $y=3$ avait un antécédent $x$ par $f_2$, nous aurions $f_2(x)=y$, c'est-à-dire
$x^2=3$, d'où $x = \pm \sqrt 3$. Mais alors $x$ n'est pas un entier de $\Zz$.
Donc $y=3$ n'a pas d'antécédent et $f_2$ n'est pas surjective.
\end{enumerate}

\end{exemple}


%---------------------------------------------------------------
\subsection{Bijection}

\begin{definition}
$f$ est \defi{bijective}\index{bijection} si elle injective et surjective. Cela équivaut à :
pour tout $y \in F$ il existe un unique $x \in E$ tel que $y=f(x)$.
Autrement dit :
\mybox{$\forall y \in F \quad \exists! x \in E \quad \big( y = f(x) \big)$}
\end{definition}

L'existence du $x$ vient de la surjectivité et l'unicité de l'injectivité.
Autrement dit, tout élément de $F$ a un unique antécédent par $f$.

\myfigure{1}{
\tikzinput{fig_ensembles13a}
\qquad \qquad
\tikzinput{fig_ensembles13b}
}
\begin{proposition}
\label{prop:bij1}
Soit $E, F$ des ensembles et $f : E \to F$ une application.
\begin{enumerate}
  \item L'application $f$ est bijective si et seulement si il existe une application $g : F \to E$
telle que $f \circ g = \id_F$ et $g \circ f = \id_E$.
  \item Si $f$ est bijective alors l'application $g$ est unique et elle aussi est bijective.
L'application $g$ s'appelle la \defi{bijection réciproque}\index{bijection reciproque@bijection réciproque} de $f$ et est notée $f^{-1}$.
De plus $\left( f^{-1} \right)^{-1} = f$.
\end{enumerate}
\end{proposition}

\begin{remarque*}
\sauteligne
\begin{itemize}
  \item  $f \circ g = \id_F$ se reformule ainsi
$$\forall y \in F\quad  f\big(g(y)\big) = y.$$
  \item Alors que $g \circ f = \id_E$ s'écrit :
$$\forall x \in E\quad  g\big(f(x)\big) = x.$$
  \item Par exemple $f : \Rr \to ]0,+\infty[$
définie par $f(x)=\exp(x)$ est bijective, sa bijection réciproque
est $g : ]0,+\infty[ \to \Rr$ définie par $g(y)=\ln(y)$.
Nous avons bien $\exp\big( \ln(y) \big) = y$, pour tout $y \in ]0,+\infty[$
et $\ln\big( \exp(x) \big) = x$, pour tout $x\in \Rr$.
\end{itemize}

\end{remarque*}




\begin{proof}
\leavevmode
\begin{enumerate}
  \item
  \begin{itemize}
    \item Sens $\Rightarrow$. Supposons $f$ bijective.
Nous allons construire une application $g : F \to E$. Comme $f$ est surjective alors pour chaque $y \in F$, il existe un $x \in E$ tel que $y=f(x)$ et on pose $g(y)=x$.
On a $f\big( g(y) \big) = f(x)=y$, ceci pour tout $y \in F$ et donc $f \circ g = \id_F$.
On compose à droite avec $f$ donc $f \circ g \circ f = \id_F \circ f$.
Alors pour tout $x \in E$ on a $f\big( g\circ f(x) \big) = f(x)$ or $f$ est injective et donc
$g\circ f(x)=x$. Ainsi $g\circ f =\id_E$. Bilan: $f \circ g = \id_F$ et $g\circ f =\id_E$.

    \item Sens $\Leftarrow$. Supposons que $g$ existe et montrons que $f$ est bijective.
    \begin{itemize}
       \item $f$ est surjective : en effet soit $y \in F$ alors on note $x = g(y) \in E$ ; on a bien :
$f(x) = f\big( g(y) \big) = f \circ g(y) = \id_F(y)=y$, donc $f$ est bien surjective.
       \item $f$ est injective : soient $x,x' \in E$ tels que $f(x)=f(x')$. On compose par $g$ (à gauche)
alors $g\circ f(x)=g\circ f(x')$ donc $\id_E(x)=\id_E(x')$ donc $x=x'$ ; $f$ est bien injective.
    \end{itemize}
  \end{itemize}

   \item
   \begin{itemize}
     \item Si $f$ est bijective alors $g$ est aussi bijective car $g \circ f = \id_E$ et $f \circ g = \id_F$
et on applique ce que l'on vient de démontrer avec $g$ à la place de $f$. Ainsi $g^{-1}=f$.
     \item Si $f$ est bijective, $g$ est unique : en effet soit $h : F \to E$ une autre application
telle que $h \circ f = \id_E$ et $f \circ h = \id_F$ ; en particulier $f \circ h = \id_F = f \circ g$,
donc pour tout $y\in F$, $f\big( h(y) \big) = f\big( g(y) \big)$ or $f$ est injective alors $h(y)=g(y)$,
ceci pour tout $y\in F$ ; d'où $h=g$.
   \end{itemize}
\end{enumerate}

\end{proof}

\begin{proposition}
\label{prop:bij2}
Soient $f : E \to F$ et $g : F \to G$ des applications bijectives.
L'application $g \circ f$ est bijective et sa bijection réciproque est
\mybox{$(g\circ f)^{-1} = f^{-1} \circ g^{-1}$}
\end{proposition}

\begin{proof}
D'après la proposition \ref{prop:bij1}, il existe $u : F \to E$ tel que $u\circ f = \id_E$ et $f\circ u = \id_F$.
Il existe aussi $v : G \to F$ tel que $v \circ g = \id_F$ et $g \circ v = \id_G$. On a alors
$(g\circ f)\circ (u \circ v) = g \circ (f \circ u) \circ v = g \circ \id_F \circ v = g \circ v = \id_E$.
Et $(u \circ v) \circ (g\circ f) = u \circ (v \circ g) \circ f = u \circ \id_F \circ f = u \circ f = \id_E$.
Donc $g\circ f$ est bijective et son inverse est $u\circ v$.
Comme $u$ est la bijection réciproque de $f$ et $v$ celle de $g$ alors :
$u\circ v = f^{-1} \circ g^{-1}$.
\end{proof}

%---------------------------------------------------------------
%\subsection{Mini-exercices}

\begin{miniexercices}
\sauteligne
\begin{enumerate}
  \item Les fonctions suivantes sont-elles injectives, surjectives, bijectives ?
\begin{itemize}
  \item $f_1 : \Rr \to [0,+\infty[$, $x \mapsto x^2$.
  \item $f_2 : [0,+\infty[ \to [0,+\infty[$, $x \mapsto x^2$.
  \item $f_3 : \Nn \to \Nn$, $x \mapsto x^2$.
  \item $f_4 : \Zz \to \Zz$, $x \mapsto x-7$.
  \item $f_5 : \Rr \to [0, +\infty[$, $x \mapsto |x|$.
\end{itemize}
  \item Montrer que la fonction $f : \, ]1,+\infty[ \to ]0,+\infty[$ définie par $f(x)=\frac{1}{x-1}$ est bijective.
Calculer sa bijection réciproque.
\end{enumerate}
\end{miniexercices}


%%%%%%%%%%%%%%%%%%%%%%%%%%%%%%%%%%%%%%%%%%%%%%%%%%%%%%%%%%%%%%%%
\section{Ensembles finis}

%---------------------------------------------------------------
\subsection{Cardinal}

\begin{definition}
Un ensemble $E$ est \defi{fini} s'il existe un entier $n\in \Nn$ et
une bijection de $E$ vers $\{1,2,\ldots,n\}$.
Cet entier $n$ est unique et s'appelle le \defi{cardinal}\index{cardinal} de $E$ (ou le \defi{nombre d'éléments})
et est noté $\Card E$.
\end{definition}

Quelques exemples :
\begin{enumerate}
  \item $E=\{\text{rouge},\text{noir}\}$ est en bijection avec $\{1,2\}$ et donc est de cardinal $2$.
  \item $\Nn$ n'est pas un ensemble fini.
  \item Par définition le cardinal de l'ensemble vide est $0$.
\end{enumerate}

Enfin quelques propriétés :
\begin{enumerate}
  \item Si $A$ est un ensemble fini et $B \subset A$ alors $B$ est aussi un ensemble fini et $\Card B \le \Card A$.
  \item Si $A, B$ sont des ensembles finis disjoints (c'est-à-dire $A \cap B = \varnothing$) alors
$\Card (A \cup B) = \Card A + \Card B$.
  \item Si $A$ est un ensemble fini et $B \subset A$ alors $\Card (A \setminus B) = \Card A - \Card B$.
  En particulier si $B \subset A$ et  $\Card A = \Card B$ alors $A=B$.
  \item Enfin pour $A,B$ deux ensembles finis quelconques :
\mybox{$\Card (A \cup B) = \Card A + \Card B - \Card (A\cap B)$}
\end{enumerate}

\medskip

Voici une situation où s'applique la dernière propriété :
\myfigure{1.3}{
\tikzinput{fig_ensembles14}
}
La preuve de la dernière propriété utilise la décomposition
$$A \cup B = A \cup \big( B\setminus(A \cap B) \big)$$
Les ensembles $A$ et $B\setminus(A \cap B)$ sont disjoints,
donc
$$\Card (A \cup B) = \Card A + \Card \big( B\setminus(A \cap B) \big)
= \Card A + \Card B - \Card (A\cap B)$$
par la propriété 2, puis la propriété 3.

%---------------------------------------------------------------
\subsection{Injection, surjection, bijection et ensembles finis}

\begin{proposition}
Soit $E,F$ deux ensembles finis et $f : E \to F$ une application.
\begin{enumerate}
  \item \label{it:bij1} Si $f$ est injective alors $\Card E \le \Card F$.
  \item \label{it:bij2} Si $f$ est surjective alors $\Card E \ge \Card F$.
  \item \label{it:bij3} Si $f$ est bijective alors $\Card E = \Card F$.
\end{enumerate}
\end{proposition}

\begin{proof}
\leavevmode
\begin{enumerate}
  \item Supposons $f$ injective. Notons $F'=f(E) \subset F$ alors
la restriction $f_| : E \to F'$ (définie par $f_|(x) = f(x)$) est une
bijection. Donc pour chaque $y \in F'$ est associé un unique $x \in E$ tel que
$y = f(x)$. Donc $E$ et $F'$ ont le même nombre d'éléments. Donc $\Card F'= \Card E$.
Or $F' \subset F$, ainsi $\Card E = \Card F' \le \Card F$.

  \item Supposons $f$ surjective. Pour tout élément $y \in F$, il existe au moins un élément $x$ de $E$
tel que $y=f(x)$ et donc $\Card E \ge \Card F$.

  \item Cela découle de (\ref{it:bij1}) et (\ref{it:bij2}) (ou aussi de la preuve du (\ref{it:bij1})).
\end{enumerate}

\end{proof}


\begin{proposition}
Soit $E,F$ deux ensembles finis et $f : E \to F$ une application.
Si
$$\Card E = \Card F$$
alors les assertions suivantes sont équivalentes :
\begin{enumerate}
  \item[i.] $f$ est injective,
  \item[ii.] $f$ est surjective,
  \item[iii.] $f$ est bijective.
\end{enumerate}
\end{proposition}

\begin{proof}
Le schéma de la preuve est le suivant : nous allons montrer successivement les implications :
$$ (i) \implies (ii) \implies (iii) \implies (i) $$
ce qui prouvera bien toutes les équivalences.

\begin{itemize}
  \item $(i) \implies (ii) $. Supposons $f$ injective. Alors $\Card f(E)= \Card E = \Card F$.
Ainsi $f(E)$ est un sous-ensemble de $F$ ayant le même cardinal que $F$ ; cela entraîne $f(E)=F$ et
donc $f$ est surjective.

  \item $(ii) \implies (iii)$. Supposons $f$ surjective. Pour montrer que $f$ est bijective,
il reste à montrer que $f$ est injective. Raisonnons par l'absurde et supposons $f$ non injective.
Alors $\Card f(E) < \Card E$ (car au moins $2$ éléments ont la même image). Or $f(E)=F$ car $f$ surjective,
donc $\Card F  < \Card E$. C'est une  contradiction, donc $f$ doit être injective et ainsi
$f$ est bijective.

  \item $(iii) \implies (i)$. C'est clair : une fonction bijective est en particulier injective.
\end{itemize}

\end{proof}

Appliquez ceci pour montrer le \defi{principe des tiroirs}\index{principe des tiroirs} :
\begin{proposition}
Si l'on range dans $k$ tiroirs, $n > k$ paires de chaussettes
alors il existe (au moins) un tiroir contenant (au moins) deux paires
de chaussettes.
\end{proposition}

Malgré sa formulation amusante, c'est une proposition souvent utile.
Exemple : dans un amphi de $400$ étudiants, il y a au moins
deux étudiants nés le même jour !



%---------------------------------------------------------------
\subsection{Nombres d'applications}

Soient $E,F$ des ensembles finis, non vides. On note $\Card E=n$ et $\Card F=p$.

\begin{proposition}
Le nombre d'applications différentes de $E$ dans $F$ est :
\mybox{$p^n$}
\end{proposition}

Autrement dit c'est \myboxinline{$(\Card F)^{\Card E}$}.

\begin{exemple}
En particulier le nombre  d'applications de $E$ dans lui-même
est $n^n$.
Par exemple si $E=\{1,2,3,4,5\}$ alors ce nombre est $5^5 = 3125$.
\end{exemple}

\begin{proof}
Fixons $F$ et $p=\Card F$.
Nous allons effectuer une récurrence sur $n = \Card E$.
Soit $(P_n)$ l'assertion suivante : le nombre d'applications d'un ensemble à $n$ éléments
vers un ensemble à $p$ éléments est $p^n$.
\begin{itemize}
  \item \emph{Initialisation.} Pour $n=1$, une application de $E$ dans $F$ est définie par l'image de l'unique élément de $E$.
Il y a $p = \Card F$ choix possibles et donc $p^1$ applications distinctes. Ainsi $P_1$ est vraie.

  \item \emph{Hérédité.} Fixons $n \ge 1$ et supposons que $P_n$ est vraie.
Soit $E$ un ensemble à $n+1$ éléments. On choisit et fixe $a \in E$ ; soit alors $E' = E \setminus \{a\}$
qui a bien $n$ éléments. Le nombre d'applications de $E'$ vers $F$ est $p^n$, par l’hypothèse de récurrence $(P_n)$.
Pour chaque application $f : E' \to F$ on peut la prolonger en une application $f : E \to F$
en choisissant l'image de $a$. On a $p$ choix pour l'image de $a$ et donc $p^n \times p$ choix
pour les applications de $E$ vers $F$. Ainsi $P_{n+1}$ est vérifiée.

  \item \emph{Conclusion.} Par le principe de récurrence $P_n$ est vraie, pour tout $n \ge 1$.
\end{itemize}

\end{proof}

\begin{proposition}
\label{prop:nbinj}
Le nombre d'injections de $E$ dans $F$ est :
$$p\times(p-1)\times\cdots\times(p-(n-1)).$$
\end{proposition}

\begin{proof}
Supposons $E=\{a_1,a_2,\ldots,a_n\}$ ; pour l'image de $a_1$ nous avons
$p$ choix. Une fois ce choix fait, pour l'image de $a_2$ il reste $p-1$ choix (car $a_2$ ne doit pas avoir la même
image que $a_1$). Pour l'image de $a_3$ il y a $p-2$ possibilités. Ainsi de suite : pour l'image de $a_k$ il y a $p-(k-1)$ choix...
Il y a au final $p\times(p-1)\times\cdots\times(p-(n-1))$ applications injectives.
\end{proof}

Notation \defi{factorielle}\index{factorielle}\index{$n"!"$} : $n! = 1\times 2 \times 3 \times \cdots \times n$.
Avec $1!=1$ et par convention $0!=1$.

\begin{proposition}
Le nombre de bijections d'un ensemble $E$ de cardinal $n$ dans lui-même est :
\mybox{$n!$}
\end{proposition}



\begin{exemple}
Parmi les $3125$ applications de $\{1,2,3,4,5\}$ dans lui-même il y en a
$5! = 120$ qui sont bijectives.
\end{exemple}

\begin{proof}
Nous allons le prouver par récurrence sur $n$.
Soit $(P_n)$ l'assertion suivante : le nombre de bijections d'un ensemble à $n$ éléments
dans un ensemble à $n$ éléments est $n!$
\begin{itemize}
  \item $P_1$ est vraie. Il n'y a qu'une bijection d'un ensemble à $1$ élément
dans un ensemble à $1$ élément.

  \item Fixons $n \ge 1$ et supposons que $P_n$ est vraie.
Soit $E$ un ensemble à $n+1$ éléments. On fixe $a \in E$.
Pour chaque $b \in E$ il y a --par l'hypothèse de récurrence-- exactement
$n!$ applications bijectives de $E\setminus\{a\} \to E \setminus \{b\}$.
Chaque application se prolonge en une bijection de $E \to F$ en posant $a \mapsto b$.
Comme il y a $n+1$ choix de $b \in E$ alors nous obtenons $n! \times (n+1)$ bijections
de $E$ dans lui-même. Ainsi $P_{n+1}$ est vraie.


  \item Par le principe de récurrence le nombre de bijections d'un ensemble à $n$ éléments
est $n!$
\end{itemize}

On aurait aussi pu directement utiliser la proposition \ref{prop:nbinj} avec $n=p$
(sachant qu'alors les injections sont aussi des bijections).
\end{proof}


%---------------------------------------------------------------
\subsection{Nombres de sous-ensembles}

Soit $E$ un ensemble fini de cardinal $n$.

\begin{proposition}
Il y a $2^{\Card E}$ sous-ensembles de $E$ :
\mybox{$\Card \mathcal{P}(E) = 2^n$}
\end{proposition}

\begin{exemple}
Si $E= \{1,2,3,4,5\}$ alors
$\mathcal{P}(E)$ a $2^5 = 32$ parties.
C'est un bon exercice de les énumérer :
\begin{itemize}
  \item l'ensemble vide : $\varnothing$,
  \item $5$ singletons: $\{1\}, \{2\},\ldots$,
  \item $10$ paires: $\{1,2\}, \{1,3\}, \ldots, \{2,3\}, \ldots$,
  \item $10$ triplets: $\{1,2,3\},\ldots$,
  \item $5$ ensembles à $4$ éléments: $\{1,2,3,4\}, \{1,2,3,5\},\ldots$,
  \item et $E$ tout entier:  $\{1,2,3,4,5\}$.
\end{itemize}
\end{exemple}


\begin{proof}
Encore une récurrence sur $n = \Card E$.
\begin{itemize}
  \item Si $n=1$, $E = \{a\}$ est un singleton, les deux sous-ensembles sont : $\varnothing$ et $E$.

  \item Supposons que la proposition soit vraie pour $n\ge 1$ fixé. Soit $E$ un ensemble à $n+1$ éléments.
On fixe $a \in E$. Il y a deux sortes de sous-ensembles de $E$ :
  \begin{itemize}
     \item les sous-ensembles $A$ qui ne contiennent pas $a$ : ce sont les sous-ensembles  $A \subset E\setminus\{a\}$.
Par l'hypothèse de récurrence il y en a $2^{n}$.
     \item les sous-ensembles $A$ qui contiennent $a$ : ils sont de la forme $A = \{a\} \cup A'$ avec
$A' \subset E\setminus\{a\}$. Par l'hypothèse de récurrence il y a $2^n$ sous-ensembles $A'$ possibles et donc aussi $2^n$
sous-ensembles $A$.
  \end{itemize}
Le bilan : $2^n+2^n=2^{n+1}$ parties $A \subset E$.

  \item Par le principe de récurrence, nous avons prouvé que si $\Card E = n$ alors on a $\Card \mathcal{P}(E)=2^n$.

\end{itemize}

\end{proof}

%---------------------------------------------------------------
\subsection{Coefficients du binôme de Newton}

\begin{definition}
Le nombre de parties à $k$ éléments d'un ensemble à $n$ éléments est
noté $\binom{n}{k}$ ou $C_n^k$.\index{$\binom{n}{k}$}\index{$C_n^k$}
\end{definition}

\begin{exemple}
Les parties à deux éléments de $\{1,2,3\}$ sont $\{1,2\}$, $\{1,3\}$ et $\{2,3\}$ et
donc $\binom{3}{2} = 3$.
Nous avons déjà classé les parties de $\{1,2,3,4,5\}$ par nombre d'éléments et donc
\begin{itemize}
  \item $\binom{5}{0} = 1$ (la seule partie n'ayant aucun élément est l'ensemble vide),
  \item $\binom{5}{1} = 5$ (il y a $5$ singletons),
  \item $\binom{5}{2} = 10$ (il y a $10$ paires),
  \item $\binom{5}{3} = 10$,
  \item $\binom{5}{4} = 5$,
  \item $\binom{5}{5} = 1$ (la seule partie ayant $5$ éléments est l'ensemble tout entier).
\end{itemize}
\end{exemple}

Sans calculs on peut déjà remarquer les faits suivants :
\begin{proposition}
\sauteligne
\begin{itemize}
  \item $\binom{n}{0}=1$, $\binom{n}{1}=n$, $\binom{n}{n}=1$. \\
  \item \myboxinline{$\binom{n}{n-k} = \binom{n}{k}$} \\
  \item \myboxinline{$\binom{n}{0}+\binom{n}{1}+\cdots+\binom{n}{k}+\cdots+\binom{n}{n} = 2^n$}
\end{itemize}
\end{proposition}

\begin{proof}
\leavevmode
\begin{enumerate}
  \item Par exemple : $\binom{n}{1}=n$ car il y a $n$ singletons.
  \item Compter le nombre de parties $A \subset E$ ayant $k$ éléments revient aussi à compter
le nombre de parties de la forme $\complement A$ (qui ont donc $n-k$ éléments), ainsi $\binom{n}{n-k} = \binom{n}{k}$.
  \item La formule $\binom{n}{0}+\binom{n}{1}+\cdots+\binom{n}{k}+\cdots+\binom{n}{n} = 2^n$ exprime que
faire la somme du nombre de parties à $k$ éléments, pour $k=0,\ldots,n$, revient à compter toutes les parties de $E$.
\end{enumerate}
\end{proof}


\begin{proposition}\ \\[-2em]
\label{prop:bin}
\mybox{$\displaystyle \binom n k = \binom{n-1}{k} + \binom{n-1}{k-1} \qquad (0<k<n)$}
\end{proposition}

\begin{proof}
Soit $E$ un ensemble à $n$ éléments, $a \in E$ et $E' = E \setminus \{ a \}$.
Il y a deux sortes de parties $A \subset E$ ayant $k$ éléments:
\begin{itemize}
  \item celles qui ne contiennent pas $a$:
  ce sont donc des parties à $k$ éléments dans $E'$ qui a $n-1$ éléments. Il y a en a donc $\binom{n-1}{k}$,

  \item celles qui contiennent $a$ : elles sont de la forme $A = \{a\} \cup A'$
  avec $A'$ une partie à $k-1$ éléments dans $E'$ qui a $n-1$ éléments. Il y en a $\binom{n-1}{k-1} $.
\end{itemize}
Bilan : $\binom n k = \binom{n-1}{k-1} + \binom{n-1}{k}$.
\end{proof}


Le triangle de Pascal\index{triangle de Pascal} est un algorithme pour calculer ces coefficients $\binom{n}{k}$.
La ligne du haut correspond à $\binom 00$, la ligne suivante  à $\binom 10$ et $\binom 11$,
la ligne d'après à $\binom 20$, $\binom21$ et $\binom 22$.

La dernière ligne du triangle de gauche aux coefficients $\binom40$, $\binom41$, \ldots, $\binom44$.

Comment continuer ce triangle pour obtenir le triangle de droite ?
Chaque élément de la nouvelle ligne est obtenu en ajoutant les deux nombres qui lui sont
au-dessus à droite et au-dessus à gauche.

\myfigure{1}{
\tikzinput{fig_ensembles17}
}

Ce qui fait que cela fonctionne c'est bien sûr la proposition \ref{prop:bin}
qui se représente ainsi :
\myfigure{1}{
\tikzinput{fig_ensembles17b}
}

\bigskip


Une autre façon de calculer le coefficient du binôme de Newton repose sur la formule suivante :
\begin{proposition}\
\mybox{$\displaystyle \binom n k = \frac{n!}{k!(n-k)!}$}
\end{proposition}


\begin{proof}
Cela se fait par récurrence sur $n$.
C'est clair pour $n=1$.
Si c'est vrai au rang $n-1$ alors  écrivons $\binom n k = \binom{n-1}{k-1} + \binom{n-1}{k}$
et utilisons l'hypothèse de récurrence pour $\binom{n-1}{k-1}$ et $\binom{n-1}{k}$.
Ainsi
\begin{align*}
\binom n k & = \binom{n-1}{k-1} + \binom{n-1}{k} = \frac{(n-1)!}{(k-1)!(n-1-(k-1))!} + \frac{(n-1)!}{k!(n-1-k)!} \\
& = \frac{(n-1)!}{(k-1)!(n-k-1)!} \times \left( \frac{1}{n-k} + \frac{1}{k} \right) = \frac{(n-1)!}{(k-1)!(n-k-1)!} \times  \frac{n}{k(n-k)}\\
& = \frac{n!}{k!(n-k)!}
\end{align*}
\end{proof}


%---------------------------------------------------------------
\subsection{Formule du binôme de Newton}
\index{formule!du binôme de Newton}

\begin{theoreme}
Soient $a,b \in\Rr$ et $n$ un entier positif alors:
\mybox{$\displaystyle (a+b)^n = \sum_{k=0}^n \binom{n}{k} \ a^{n-k} \cdot b^{k}$}
\end{theoreme}

Autrement dit :
{\small
$$(a+b)^n = \binom{n}{0}\ a^n\cdot b^0 + \binom{n}{1}\ a^{n-1}\cdot b^{1}
+ \cdots + \binom{n}{k} \ a^{n-k} \cdot b^{k}+\cdots + \binom{n}{n}\ a^0\cdot b^n$$
}

Le théorème est aussi vrai si $a$ et $b$ sont des nombres complexes.

\begin{exemple}
\sauteligne
\begin{enumerate}
  \item Pour $n=2$ on retrouve la formule archi-connue : $(a+b)^2= a^2 + 2ab + b^2$.
  \item Il est aussi bon de connaître $(a+b)^3 = a^3 + 3a^2b + 3ab^2 + b^3$.
  \item Si $a=1$ et $b=1$ on retrouve la formule : $\sum_{k=0}^n \binom{n}{k} = 2^n$.
\end{enumerate}
\end{exemple}

\begin{proof}
Nous allons effectuer une récurrence sur $n$.
Soit $(P_n)$ l'assertion : $(a+b)^n = \sum_{k=0}^n \binom{n}{k} \ a^{n-k} \cdot b^{k}$.
\begin{itemize}
  \item \emph{Initialisation.} Pour $n=1$, $(a+b)^1 = \binom{1}{0} a^1b^0 + \binom{1}{1}a^0b^1$.
Ainsi $P_1$ est vraie.

  \item \emph{Hérédité.} Fixons $n \ge 2$ et supposons que $P_{n-1}$ est vraie.
\begin{eqnarray*}
(a+b)^{n}
  &=& (a+b)\cdot(a+b)^{n-1} \\
   & = & a\left(a^{n-1} +  \cdots + \binom{n-1}{k} a^{n-1-k}b^{k} +\cdots + b^{n-1} \right)  \\
   && +  b \left(a^{n-1} +  \cdots + \binom{n-1}{k-1} a^{n-1-(k-1)}b^{k-1} +\cdots + b^{n-1} \right)  \\
   &=& \cdots + \left(\binom{n-1}{k} + \binom{n-1}{k-1}\right) a^{n-k} b^{k} + \cdots \\
   &=& \cdots +\binom{n}{k} a^{n-k} b^{k} + \cdots \\
   &=&  \sum_{k=0}^n \binom{n}{k} \ a^{n-k} \cdot b^{k}
\end{eqnarray*}
 Ainsi $P_{n}$ est vérifiée.

  \item \emph{Conclusion.} Par le principe de récurrence $P_n$ est vraie, pour tout $n \ge 1$.
\end{itemize}
\end{proof}


%---------------------------------------------------------------
%\subsection{Mini-exercices}

\bigskip
\bigskip
\bigskip
\bigskip

\begin{miniexercices}
\sauteligne
\begin{enumerate}
  \item Combien y a-t-il d'applications injectives d'un ensemble à $n$ éléments dans un ensemble à $n+1$ éléments ?
  \item Combien y a-t-il d'applications surjectives d'un ensemble à $n+1$ éléments dans un ensemble à $n$ éléments ?
  \item Calculer le nombre de façons de choisir $5$ cartes dans un jeux de $32$ cartes.
  \item Calculer le nombre de listes à $k$ éléments dans un ensemble à $n$ éléments (les listes sont ordonnées :
par exemple $(1,2,3) \neq (1,3,2)$).
  \item Développer $(a-b)^4$, $(a+b)^5$.
  \item Que donne la formule du binôme pour $a=-1$, $b=+1$ ? En déduire que dans un ensemble à $n$ éléments il y a
autant de parties de cardinal pair que de cardinal impair.
\end{enumerate}
\end{miniexercices}



%%%%%%%%%%%%%%%%%%%%%%%%%%%%%%%%%%%%%%%%%%%%%%%%%%%%%%%%%%%%%%%%
\section{Relation d'équivalence}


%---------------------------------------------------------------
\subsection{Définition}

Une \defi{relation} sur un ensemble $E$, c'est la donnée pour tout couple
$(x,y)\in E \times E$ de \og Vrai \fg{} (s'ils sont en relation), ou de \og Faux \fg{} sinon.

Nous schématisons une relation ainsi : les éléments de $E$ sont des points,
une flèche de $x$ vers $y$ signifie que $x$ est en relation avec $y$, c'est-à-dire que l'on associe
\og Vrai\fg{} au couple $(x,y)$.
\myfigure{0.9}{
\tikzinput{fig_ensembles18}
}


\begin{definition}
Soit $E$ un ensemble et $\mathcal{R}$ une relation, c'est une
\defi{relation d'équivalence}\index{relation d equivalence@relation d'équivalence} si :
\begin{itemize}
  \item $\forall x \in E$, $x \mathcal{R}x$, \quad  (\defi{réflexivité}\index{reflexivite@réflexivité})
\myfigure{1}{
\tikzinput{fig_ensembles19a}
}
  \item $\forall x,y \in E$, $x \mathcal{R}y \implies y\mathcal{R}x$, \quad  (\defi{symétrie}\index{symetrie@symétrie})
\myfigure{1}{
\tikzinput{fig_ensembles19b}
}
  \item $\forall x,y,z \in E$, $x \mathcal{R}y \text{ et }  y\mathcal{R}z \implies x\mathcal{R}z$, \quad
  (\defi{transitivité}\index{transitivite@transitivité})
\myfigure{1}{
\tikzinput{fig_ensembles19c}
}
\end{itemize}
\end{definition}

Exemple de relation d'équivalence :
\myfigure{0.9}{
\tikzinput{fig_ensembles19}
}

%---------------------------------------------------------------
\subsection{Exemples}


\begin{exemple}
Voici des exemples basiques.
\begin{enumerate}
  \item La relation $\mathcal{R}$ \og être parallèle \fg{} est une relation d'équivalence
pour l'ensemble $E$ des droites affines du plan :
  \begin{itemize}
    \item réflexivité : une droite est parallèle à elle-même,
    \item symétrie : si $D$ est parallèle à $D'$ alors $D'$ est parallèle à $D$,
    \item transitivité : si $D$ parallèle à $D'$ et $D'$ parallèle à $D''$ alors $D$ est parallèle à $D''$.
  \end{itemize}
  \item La relation \og être du même âge \fg{} est une relation d'équivalence.
  \item La relation \og être perpendiculaire \fg{} n'est pas une relation d'équivalence
(ni la réflexivité, ni la transitivité ne sont vérifiées).
  \item La relation $\le$ (sur $E=\Rr$ par exemple) n'est pas une relation d'équivalence
(la symétrie n'est pas vérifiée).
\end{enumerate}
\end{exemple}


%---------------------------------------------------------------
\subsection{Classes d'équivalence}

\begin{definition}
Soit $\mathcal{R}$ une relation d'équivalence sur un ensemble $E$.
Soit $x\in E$, la \defi{classe d'équivalence}\index{classe d equivalence@classe d'équivalence} de $x$ est
\mybox{$\text{cl}(x) = \big\{ y  \in E \mid y\mathcal{R}x \big\}$}
\end{definition}



\myfigure{1}{
\tikzinput{fig_ensembles20}
}



$\text{cl}(x)$ est donc un sous-ensemble de $E$, on le note aussi $\overline{x}$.
Si $y \in \text{cl}(x)$, on dit que $y$ un \defi{représentant}\index{representant@représentant} de $\text{cl}(x)$.

\medskip

Soit $E$ un ensemble et $\mathcal{R}$ une relation d'équivalence.
\begin{proposition}
On a les propriétés suivantes :
\begin{enumerate}
  \item $\text{cl}(x)=\text{cl}(y) \Longleftrightarrow x\mathcal{R}y$.
  \item Pour tout $x,y \in E$, $\text{cl}(x)=\text{cl}(y)$ ou $\text{cl}(x) \cap \text{cl}(y) = \varnothing$.
  \item Soit $C$ un ensemble de représentants de toutes les classes alors $\big\{ \text{cl}(x) \mid x \in C \big\}$
constitue une partition de $E$.
\end{enumerate}
\end{proposition}

Une \defi{partition}\index{partition} de $E$ est un ensemble $\{E_i\}$ de parties de $E$ tel que
$E = \bigcup_i E_i$ et $E_i\cap E_j = \varnothing$ (si $i\neq j$).
\myfigure{1}{
\tikzinput{fig_ensembles21}
}

Exemples :
\begin{enumerate}
  \item Pour la relation \og être du même âge \fg{}, la classe d'équivalence d'une personne est l'ensemble
des personnes ayant le même âge. Il y a donc une classe d'équivalence formée des personnes
de $19$ ans, une autre formée des personnes de $20$ ans,...
Les trois assertions de la proposition se lisent ainsi :
  \begin{itemize}
    \item On est dans la même classe d'équivalence si et seulement si on est du même âge.
    \item Deux personnes appartiennent soit à la même classe, soit à des classes disjointes.
    \item Si on choisit une personne de chaque âge possible, cela forme un ensemble de représentants $C$.
Maintenant une personne quelconque appartient à une et une seule classe d'un des représentants.
  \end{itemize}

  \item Pour la relation \og être parallèle \fg{}, la classe d'équivalence d'une droite
  est l'ensemble des droites parallèles à cette droite.
\`A chaque classe d'équivalence correspond une et une seule direction.
\end{enumerate}


Voici un exemple que vous connaissez depuis longtemps :
\begin{exemple}
Définissons sur $E= \Zz \times \Nn^*$ la relation $\mathcal{R}$ par
$$(p,q) \mathcal{R} (p',q') \Longleftrightarrow pq'=p'q.$$

Tout d'abord $\mathcal{R}$ est une relation d'équivalence :
\begin{itemize}
  \item $\mathcal{R}$ est réflexive : pour tout $(p,q)$ on a bien $pq=pq$ et donc $(p,q) \mathcal{R} (p,q)$.
  \item $\mathcal{R}$ est symétrique : pour tout $(p,q)$, $(p',q')$ tels que $(p,q) \mathcal{R} (p',q')$
on a donc $pq'=p'q$ et donc $p'q=pq'$ d'où $(p',q') \mathcal{R} (p,q)$.
  \item $\mathcal{R}$ est transitive : pour tout $(p,q)$, $(p',q')$, $(p'',q'')$ tels que $(p,q) \mathcal{R} (p',q')$
et $(p',q') \mathcal{R} (p'',q'')$ on a donc $pq'=p'q$ et $p'q''=p''q'$.
Alors $(pq')q'' = (p'q)q'' = q(p'q'')=q(p''q')$. En divisant par $q' \neq 0$ on obtient
$pq'' = qp''$ et donc $(p,q) \mathcal{R} (p'',q'')$.
\end{itemize}


Nous allons noter $\frac pq=\text{cl}(p,q)$ la classe d'équivalence
d'un élément $(p,q) \in \Zz\times \Nn^*$.
Par exemple, comme $(2,3)\mathcal{R}(4,6)$ (car $2\times 6 = 3\times 4$) alors les classes de
$(2,3)$ et $(4,6)$ sont égales : avec notre notation cela s'écrit : $\frac 23=\frac 46$.

C'est ainsi que l'on définit les rationnels :
l'ensemble $\Qq$ des rationnels est l'ensemble de classes
d'équivalence de la relation $\mathcal{R}$.

Les nombres $\frac 23=\frac 46$ sont bien égaux (ce sont les mêmes classes)
mais les écritures sont différentes (les représentants sont distincts).
\end{exemple}




%---------------------------------------------------------------
\subsection{L'ensemble $\Zz/n\Zz$}

Soit $n\ge 2$ un entier fixé.
Définissons la relation suivante sur l'ensemble $E = \Zz$ :
\mybox{$a \equiv b \pmod n \quad \Longleftrightarrow \quad a-b \text{ est un multiple de } n$}

\medskip

Exemples pour $n=7$ : $10 \equiv 3 \pmod 7$,
$19 \equiv 5 \pmod 7$, $77 \equiv 0 \pmod 7$,
$-1 \equiv 20 \pmod 7$.

\medskip

Cette relation est bien une relation d'équivalence :
\begin{itemize}
  \item Pour tout $a \in \Zz$, $a-a=0 = 0\cdot n$ est un multiple de $n$ donc $a\equiv a \pmod n$.
  \item Pour $a,b \in \Zz$ tels que $a \equiv b \pmod n$ alors
$a-b$ est un multiple de $n$, autrement dit il existe $k\in \Zz$ tel que
$a-b=kn$ et donc $b-a = (-k)n$ et ainsi $b\equiv a \pmod n$.
  \item Si $a \equiv b \pmod n$ et $b \equiv c \pmod n$ alors il existe $k,k'\in \Zz$
tels que $a-b=kn$ et $b-c=k'n$. Alors $a-c = (a-b) + (b-c) = (k+k')n$ et donc $a \equiv c \pmod n$.
\end{itemize}

\medskip

La classe d'équivalence de $a\in\Zz$ est notée $\overline a$.
Par définition nous avons donc
$$\overline a = \text{cl}(a) = \big\{ b \in \Zz \mid b \equiv a \pmod n \big\}.$$
Comme un tel $b$ s'écrit $b=a+kn$ pour un certain $k\in \Zz$ alors
c'est aussi exactement
$$\overline a = a + n\Zz = \big\{ a+kn \mid k\in \Zz \big\}.$$

Comme $n \equiv 0 \pmod n$, $n+1 \equiv 1  \pmod n$, \ldots{} alors
$$\overline n = \overline 0, \quad \overline {n+1} = \overline 1,  \quad \overline {n+2} = \overline 2, \ldots$$
et donc l'ensemble des classes d'équivalence est l'ensemble
\mybox{$\Zz/n\Zz = \big\{ \overline 0, \overline 1, \overline 2, \ldots, \overline{n-1} \big\}$}
qui contient exactement $n$ éléments.

Par exemple, pour $n=7$ :
\begin{itemize}
  \item $\overline 0 = \{\ldots,-14,-7,0,7,14,21,\ldots\} = 7\Zz$

  \item $\overline 1 = \{\ldots, -13,-6, 1, 8, 15, \ldots \} = 1+ 7\Zz$

  \item \ldots

  \item $\overline 6 = \{\ldots, -8, -1, 6, 13, 20, \ldots \} = 6+7\Zz$
\end{itemize}
Mais ensuite
$\overline 7 = \{ \ldots -7, 0, 7, 14, 21, \ldots \} = \overline 0 = 7\Zz$.
Ainsi $\Zz/7\Zz = \big\{  \overline 0, \overline 1, \overline 2,\ldots, \overline{6} \big\}$
possède $7$ éléments.





\begin{remarque*}
Dans beaucoup de situations de la vie courante, nous raisonnons avec les modulos.
Par exemple pour l'heure : les minutes et les secondes sont modulo $60$ (après
59 minutes on repart à zéro), les heures modulo $24$ (ou modulo $12$ sur le cadran à aiguilles). Les jours de la semaine sont
modulo $7$, les mois modulo $12$,...
\end{remarque*}


%---------------------------------------------------------------
%\subsection{Mini-exercices}

\begin{miniexercices}
\sauteligne
\begin{enumerate}
  \item Montrer que la relation définie sur $\Nn$ par $x \mathcal{R} y \iff \frac{2x+y}{3} \in \Nn$ est une relation d'équivalence.
Montrer qu'il y a $3$ classes d'équivalence.

  \item Dans $\Rr^2$ montrer que la relation définie par $(x,y) \mathcal{R} (x',y') \iff x+y'=x'+y$ est une relation d'équivalence.
Montrer que deux points $(x,y)$ et $(x',y')$ sont dans une même classe si et seulement s'ils appartiennent à une même droite
dont vous déterminerez la direction.

  \item On définit une addition sur $\Zz/n\Zz$ par $\overline{p} + \overline{q}= \overline{p + q}$.
Calculer la table d'addition dans $\Zz/6\Zz$ (c'est-à-dire toutes les sommes $\overline{p} + \overline{q}$
pour $\overline{p},\overline{q} \in \Zz/6\Zz$). Même chose avec la multiplication $\overline{p}\times \overline{q}= \overline{p\times q}$.
Mêmes questions avec $\Zz/5\Zz$, puis $\Zz/8\Zz$.

\end{enumerate}
\end{miniexercices}


\auteurs{
Arnaud Bodin,
Benjamin Boutin,
Pascal Romon
}


\finchapitre
\end{document}
