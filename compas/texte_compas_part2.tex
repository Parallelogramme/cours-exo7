
%%%%%%%%%%%%%%%%%% PREAMBULE %%%%%%%%%%%%%%%%%%


\documentclass[12pt]{article}

\usepackage{amsfonts,amsmath,amssymb,amsthm}
\usepackage[utf8]{inputenc}
\usepackage[T1]{fontenc}
\usepackage[francais]{babel}


% packages
\usepackage{amsfonts,amsmath,amssymb,amsthm}
\usepackage[utf8]{inputenc}
\usepackage[T1]{fontenc}
%\usepackage{lmodern}

\usepackage[francais]{babel}
\usepackage{fancybox}
\usepackage{graphicx}

\usepackage{float}

%\usepackage[usenames, x11names]{xcolor}
\usepackage{tikz}
\usepackage{datetime}

\usepackage{mathptmx}
%\usepackage{fouriernc}
%\usepackage{newcent}
\usepackage[mathcal,mathbf]{euler}

%\usepackage{palatino}
%\usepackage{newcent}


% Commande spéciale prompteur

%\usepackage{mathptmx}
%\usepackage[mathcal,mathbf]{euler}
%\usepackage{mathpple,multido}

\usepackage[a4paper]{geometry}
\geometry{top=2cm, bottom=2cm, left=1cm, right=1cm, marginparsep=1cm}

\newcommand{\change}{{\color{red}\rule{\textwidth}{1mm}\\}}

\newcounter{mydiapo}

\newcommand{\diapo}{\newpage
\hfill {\normalsize  Diapo \themydiapo \quad \texttt{[\jobname]}} \\
\stepcounter{mydiapo}}


%%%%%%% COULEURS %%%%%%%%%%

% Pour blanc sur noir :
%\pagecolor[rgb]{0.5,0.5,0.5}
% \pagecolor[rgb]{0,0,0}
% \color[rgb]{1,1,1}



%\DeclareFixedFont{\myfont}{U}{cmss}{bx}{n}{18pt}
\newcommand{\debuttexte}{
%%%%%%%%%%%%% FONTES %%%%%%%%%%%%%
\renewcommand{\baselinestretch}{1.5}
\usefont{U}{cmss}{bx}{n}
\bfseries

% Taille normale : commenter le reste !
%Taille Arnaud
%\fontsize{19}{19}\selectfont

% Taille Barbara
%\fontsize{21}{22}\selectfont

%Taille François
\fontsize{25}{30}\selectfont

%Taille Pascal
%\fontsize{25}{30}\selectfont

%Taille Laura
%\fontsize{30}{35}\selectfont


%\myfont
%\usefont{U}{cmss}{bx}{n}

%\Huge
%\addtolength{\parskip}{\baselineskip}
}


% \usepackage{hyperref}
% \hypersetup{colorlinks=true, linkcolor=blue, urlcolor=blue,
% pdftitle={Exo7 - Exercices de mathématiques}, pdfauthor={Exo7}}


%section
% \usepackage{sectsty}
% \allsectionsfont{\bf}
%\sectionfont{\color{Tomato3}\upshape\selectfont}
%\subsectionfont{\color{Tomato4}\upshape\selectfont}

%----- Ensembles : entiers, reels, complexes -----
\newcommand{\Nn}{\mathbb{N}} \newcommand{\N}{\mathbb{N}}
\newcommand{\Zz}{\mathbb{Z}} \newcommand{\Z}{\mathbb{Z}}
\newcommand{\Qq}{\mathbb{Q}} \newcommand{\Q}{\mathbb{Q}}
\newcommand{\Rr}{\mathbb{R}} \newcommand{\R}{\mathbb{R}}
\newcommand{\Cc}{\mathbb{C}} 
\newcommand{\Kk}{\mathbb{K}} \newcommand{\K}{\mathbb{K}}

%----- Modifications de symboles -----
\renewcommand{\epsilon}{\varepsilon}
\renewcommand{\Re}{\mathop{\text{Re}}\nolimits}
\renewcommand{\Im}{\mathop{\text{Im}}\nolimits}
%\newcommand{\llbracket}{\left[\kern-0.15em\left[}
%\newcommand{\rrbracket}{\right]\kern-0.15em\right]}

\renewcommand{\ge}{\geqslant}
\renewcommand{\geq}{\geqslant}
\renewcommand{\le}{\leqslant}
\renewcommand{\leq}{\leqslant}

%----- Fonctions usuelles -----
\newcommand{\ch}{\mathop{\mathrm{ch}}\nolimits}
\newcommand{\sh}{\mathop{\mathrm{sh}}\nolimits}
\renewcommand{\tanh}{\mathop{\mathrm{th}}\nolimits}
\newcommand{\cotan}{\mathop{\mathrm{cotan}}\nolimits}
\newcommand{\Arcsin}{\mathop{\mathrm{Arcsin}}\nolimits}
\newcommand{\Arccos}{\mathop{\mathrm{Arccos}}\nolimits}
\newcommand{\Arctan}{\mathop{\mathrm{Arctan}}\nolimits}
\newcommand{\Argsh}{\mathop{\mathrm{Argsh}}\nolimits}
\newcommand{\Argch}{\mathop{\mathrm{Argch}}\nolimits}
\newcommand{\Argth}{\mathop{\mathrm{Argth}}\nolimits}
\newcommand{\pgcd}{\mathop{\mathrm{pgcd}}\nolimits} 

\newcommand{\Card}{\mathop{\text{Card}}\nolimits}
\newcommand{\Ker}{\mathop{\text{Ker}}\nolimits}
\newcommand{\id}{\mathop{\text{id}}\nolimits}
\newcommand{\ii}{\mathrm{i}}
\newcommand{\dd}{\mathrm{d}}
\newcommand{\Vect}{\mathop{\text{Vect}}\nolimits}
\newcommand{\Mat}{\mathop{\mathrm{Mat}}\nolimits}
\newcommand{\rg}{\mathop{\text{rg}}\nolimits}
\newcommand{\tr}{\mathop{\text{tr}}\nolimits}
\newcommand{\ppcm}{\mathop{\text{ppcm}}\nolimits}

%----- Structure des exercices ------

\newtheoremstyle{styleexo}% name
{2ex}% Space above
{3ex}% Space below
{}% Body font
{}% Indent amount 1
{\bfseries} % Theorem head font
{}% Punctuation after theorem head
{\newline}% Space after theorem head 2
{}% Theorem head spec (can be left empty, meaning ‘normal’)

%\theoremstyle{styleexo}
\newtheorem{exo}{Exercice}
\newtheorem{ind}{Indications}
\newtheorem{cor}{Correction}


\newcommand{\exercice}[1]{} \newcommand{\finexercice}{}
%\newcommand{\exercice}[1]{{\tiny\texttt{#1}}\vspace{-2ex}} % pour afficher le numero absolu, l'auteur...
\newcommand{\enonce}{\begin{exo}} \newcommand{\finenonce}{\end{exo}}
\newcommand{\indication}{\begin{ind}} \newcommand{\finindication}{\end{ind}}
\newcommand{\correction}{\begin{cor}} \newcommand{\fincorrection}{\end{cor}}

\newcommand{\noindication}{\stepcounter{ind}}
\newcommand{\nocorrection}{\stepcounter{cor}}

\newcommand{\fiche}[1]{} \newcommand{\finfiche}{}
\newcommand{\titre}[1]{\centerline{\large \bf #1}}
\newcommand{\addcommand}[1]{}
\newcommand{\video}[1]{}

% Marge
\newcommand{\mymargin}[1]{\marginpar{{\small #1}}}



%----- Presentation ------
\setlength{\parindent}{0cm}

%\newcommand{\ExoSept}{\href{http://exo7.emath.fr}{\textbf{\textsf{Exo7}}}}

\definecolor{myred}{rgb}{0.93,0.26,0}
\definecolor{myorange}{rgb}{0.97,0.58,0}
\definecolor{myyellow}{rgb}{1,0.86,0}

\newcommand{\LogoExoSept}[1]{  % input : echelle
{\usefont{U}{cmss}{bx}{n}
\begin{tikzpicture}[scale=0.1*#1,transform shape]
  \fill[color=myorange] (0,0)--(4,0)--(4,-4)--(0,-4)--cycle;
  \fill[color=myred] (0,0)--(0,3)--(-3,3)--(-3,0)--cycle;
  \fill[color=myyellow] (4,0)--(7,4)--(3,7)--(0,3)--cycle;
  \node[scale=5] at (3.5,3.5) {Exo7};
\end{tikzpicture}}
}



\theoremstyle{definition}
%\newtheorem{proposition}{Proposition}
%\newtheorem{exemple}{Exemple}
%\newtheorem{theoreme}{Théorème}
\newtheorem{lemme}{Lemme}
\newtheorem{corollaire}{Corollaire}
%\newtheorem*{remarque*}{Remarque}
%\newtheorem*{miniexercice}{Mini-exercices}
%\newtheorem{definition}{Définition}




%definition d'un terme
\newcommand{\defi}[1]{{\color{myorange}\textbf{\emph{#1}}}}
\newcommand{\evidence}[1]{{\color{blue}\textbf{\emph{#1}}}}



 %----- Commandes divers ------

\newcommand{\codeinline}[1]{\texttt{#1}}

%%%%%%%%%%%%%%%%%%%%%%%%%%%%%%%%%%%%%%%%%%%%%%%%%%%%%%%%%%%%%
%%%%%%%%%%%%%%%%%%%%%%%%%%%%%%%%%%%%%%%%%%%%%%%%%%%%%%%%%%%%%

\newcommand{\construc}{\mathcal{C}}
\newcommand{\plan}{\mathcal{P}}
\newcommand{\cercle}{\mathcal{C}}

\begin{document}

\debuttexte


%%%%%%%%%%%%%%%%%%%%%%%%%%%%%%%%%%%%%%%%%%%%%%%%%%%%%%%%%%%
\diapo

\change
Pour résoudre les trois problèmes grecs, il va falloir les transformer complètement.

\change
Après des définitions,

\change
nous passerons d'une question géométrique à une question algébrique.

\change
On ramène nos problèmes de la construction à la règle et au compas à la construction
de nombres réels.


%%%%%%%%%%%%%%%%%%%%%%%%%%%%%%%%%%%%%%%%%%%%%%%%%%%%%%%%%%%
\diapo


On définit des ensembles de points $\construc_i$ dans le plan euclidien, identifié à $\Rr^2$

Ces ensembles sont définis par récurrence.

\change

On se donne au départ seulement deux points : $\construc_0 = \{ O, I\}$ où $O=(0,0)$ et $I=(1,0)$. 

\change
Fixons un entier $i$, et supposons qu'un certain nombre de points $\construc_i$
  soient déjà construits.
  
\change
Alors on définit $\construc_{i+1}$ par récurrence,
comme l'ensemble des \defi{points élémentairement constructibles}
à partir de $\construc_i$. 

\change
Cela signifie que  $P \in \construc_{i+1}$ si et seulement si l'une des conditions est vérifiées :


\change
  tout d'abord $P$ peut déjà être un point de $\construc_i$,
  
voici les nouveaux points qui apparaissent : 

\change
(1) $P$ est l'intersection de deux droites $(AB)$ et $(A'B')$ avec $A, B, A', B'$ dans l'ensemble 
    initial $\construc_i$,
    
\change    
ou bien (2) $P$ est l'intersection d'une droite  $(AB)$ avec $A,B$ des points de $\construc_i$,
et d'un cercle centré en $A'$ et passant par $B'$ où $A'$ et $B'$ sont aussi des points de $\construc_i$.

\change
ou bien (3) $P$ est l'intersection de deux cercles, avec encore une fois les centres 
dans $\construc_i$ et passant chacun par un point de $\construc_i$.


%%%%%%%%%%%%%%%%%%%%%%%%%%%%%%%%%%%%%%%%%%%%%%%%%%%%%%%%%%%
\diapo

Il faut comprendre cette construction par récurrence ainsi : si 
$A,B, A', B'$ ont été construits et sont dans $\construc_i$ 
alors, à partir de ces points, on peut tracer plusieurs objets à
la règle et au compas : 

\change
par exemple la droite $(AB)$ 

\change
et la droite $(A'B')$.

\change
Par définition le point $P$ est dans $\construc_{i+1}$

\change
Partant de quatre points dans $\construc_i$,

\change
on pourrait tracer la droite $(AB)$ 

\change
et le cercle de centre $A'$ et passant $B'$.

\change
Cette droite et ce cercle s'intersectent en deux points, 
et chacun des ces points par exemple ce $P$ est par définition 
dans $\construc_{i+1}$


\change
Même construction avec deux cercles

\change

\change

\change


%%%%%%%%%%%%%%%%%%%%%%%%%%%%%%%%%%%%%%%%%%%%%%%%%%%%%%%%%%%
\diapo

Voici les premières étapes de nos ensembles de points $\construc_i$.

Je pars de $\construc_0$ qui est constitué de seulement deux points : l'origine et le point $(1,0)$.

\change
On peut tracer la droite qui passe par ces deux points

\change
et aussi le cercle centré à l'origine passant par l'autre point

\change
ou alors le cercle de centre $(1,0)$ passant par l'origine.

\change
L'ensemble $\construc_1$, est formé des points de $\construc_0$ en bleu et de
$4$ points supplémentaires, en rouge.


Pour $\construc_2$ on repartirait de tous les points (rouge ou bleu) 
de $\construc_1$, et on tracerait tous les cercles ou droites 
possibles (il y en a beaucoup !), et les points d'intersections 
formeraient l'ensemble $\construc_2$.


Par contre  si l'on a construit deux points $A$, $B$, on peut bien sûr tracer tracer la droite $(AB)$, 
pour autant les points de $(AB)$ ne sont pas tous constructibles. 
On peut seulement construire les points d'intersection de cette droite $(AB)$ avec d'autres objets déjà 
construits.


%%%%%%%%%%%%%%%%%%%%%%%%%%%%%%%%%%%%%%%%%%%%%%%%%%%%%%%%%%%
\diapo

L'union  des $\construc_i$ s'appelle l'ensemble des \defi{points constructibles}, on la note 
 $\construc$.

\change
C'est une union infinie, ce qui signifie simplement que 
$P \in \construc$ si et seulement s'il existe $i$ tel que $P \in \construc_i$.


\change

$\construc_\Rr$ est l'ensemble des abscisses des points constructibles, c'est donc un sous-ensemble de $\Rr$, 
que l'on appelle les \defi{nombres (réels) constructibles}.  


\change
Par exemple si ce point est un point constructible,
alors son abscisse est un nombre réel constructible.

\change
Notez que son ordonnée est aussi un réel constructible, on peut tracer l'horizontale passant par ce point,
elle recoupe l'axe des ordonnées en $(0,y)$.

\change
On trace ensuite le cercle centré à l'origine, passant par ce point.

\change
Ce cercle recoupe l'axe des abscisses en un point $(y,0)$, qui est bien un point constructible puisque l'on vient de le construire !
Donc $y$ est un nombre constructible.

\change
En fait déterminer les points constructibles du plan $\construc$ ou déterminer 
les nombres constructibles $\construc_\Rr$ sont deux problèmes équivalents.


En effet, réciproquement si on part de deux nombres constructibles $x,y \in \Rr$ alors on inverse
la construction que l'on vient de faire et il est facile de construire le point $(x,y)$.


%%%%%%%%%%%%%%%%%%%%%%%%%%%%%%%%%%%%%%%%%%%%%%%%%%%%%%%%%%%
\diapo

Si $x, x'$ sont des réels constructibles alors :

\change
$x+x'$ est constructible,
 
\change
$-x$ est constructible,

\change
$x \cdot x'$ est constructible.

\change
Si $x' \neq 0$, alors $x/x'$ est constructible.


\change
Comme application :
$$\Nn \subset \construc_\Rr \qquad\qquad \Zz \subset \construc_\Rr \qquad\qquad \Qq \subset \construc_\Rr$$


Autrement dit,  tous les rationnels sont des nombres réels constructibles.


La preuve de ces inclusions découle facilement de la proposition :
Puisque $1$ est un nombre constructible alors $2=1+1$ est constructible, mais alors 
$3=2+1$ est constructible et par récurrence tout entier $n\ge 0$ est un élément de $\construc_\Rr$.
 
Comme tout entier $n\ge 0$ est constructible alors $-n$ l'est aussi ; donc tous les entiers
 $n\in \Zz$ sont constructibles.
 
Enfin comme les entiers $p,q$ sont constructibles, alors le quotient 
$\frac pq$ est constructible et ainsi $\Qq \subset \construc_\Rr$.

%%%%%%%%%%%%%%%%%%%%%%%%%%%%%%%%%%%%%%%%%%%%%%%%%%%%%%%%%%%
\diapo

Passons à la preuve.
On part de deux nombres réels constructibles $x$ et $x'$.

La construction pour le réel $x+x'$ est facile en utilisant le report 
  du compas : on reporte la longueur $x'$ à partir de $x$. 
  
\change
Une autre méthode est de construire d'abord le milieu $\frac{x+x'}{2}$ puis le symétrique de 
  $0$ par rapport à ce milieu : c'est $x+x'$.
  
\change
L'opposé du réel $x$ s'obtient comme symétrique par rapport à l'origine : 
  tracez la droite passant par $0$ et $x$ ; tracez le cercle de centre 
  $0$ passant par $x$  ; ce cercle recoupe la droite en $-x$. 
  
  
%%%%%%%%%%%%%%%%%%%%%%%%%%%%%%%%%%%%%%%%%%%%%%%%%%%%%%%%%%%
\diapo 

On continue la preuve avec le produit de deux nombres réels constructibles.

\change
On suppose donc construits les points $(x,0)$ et $(0,x')$.

\change
On trace la droite $\mathcal{D}$ bleue passant par $(x,0)$ et $(0,1)$. 

\change
On construit ensuite --\,à la règle et au compas\,-- la droite rouge $\mathcal{D}'$
  parallèle à la droite bleue et passant par $(0,x')$. 

\change
  Le théorème de Thalès prouve que 
  cette dernière droite recoupe l'axe des abscisses en $(x\cdot x',0)$. 
  
\change
Pour le quotient la méthode est similaire, on part de $x$ et $x'$ et on construit
le quotient $x/x'$.


%%%%%%%%%%%%%%%%%%%%%%%%%%%%%%%%%%%%%%%%%%%%%%%%%%%%%%%%%%%
\diapo

Nous allons voir que l'ensemble des nombres réels constructibles
contient davantage de nombres que les rationnels.

Proposition : Si $x\ge 0$ est un nombre constructible, alors $\sqrt x$ est aussi un nombre constructible.


Remarque la réciproque est vraie car si $x'$ est un nombre constructible, alors $x'^2$ est constructible.

\change
Passons à la preuve : on a placé sur l'axe des abscisses 
les nombres constructibles $0,-1$ et aussi $x$ qui est constructible par hypothèse.

\change

\change
Traçons le cercle dont le diamètre est $[-1,x]$ 
(cela revient à construire le centre du cercle $\frac{x-1}{2}$).

\change
Ce cercle recoupe l'axe des ordonnées en $y\ge0$. 

Et nous allons montrer que $y = \sqrt x$.

\change

\change
On reprend la figure ici et on applique le théorème 
de Pythagore dans trois triangles rectangles, pour obtenir :

\change
$a^2+b^2 =(1+x)^2$

\change
puis $1+y^2 = a^2$

\change
et enfin $x^2+y^2 = b^2.$

\change
On réunit tout cela pour obtenir d'une part 
$a^2+b^2 = (1+x)^2 = 1 + x^2 + 2x$ 

\change
et d'autre part $a^2+b^2 = 1 + x^2 + 2y^2$.

\change
Ainsi $1+x^2+2y^2 = 1 + x^2+2x$ 

\change
d'où $y^2=x$. 

\change
Comme $y$ est positif alors $y = \sqrt{x}$.

%%%%%%%%%%%%%%%%%%%%%%%%%%%%%%%%%%%%%%%%%%%%%%%%%%%%%%%%%%%
\diapo
Avec le langage des nombres constructibles les problèmes historiques s'énoncent 
ainsi :
 
 \change
 Tout d'abord le problème de la duplication du cube devient : 
 Est-ce que le réel $\sqrt[3]{2}$ est un nombre constructible ?
  
  \change
  Pour la quadrature du cercle : Est-ce que le réel $\pi$ est un nombre constructible ?
  
  \change
  Enfin pour la trisection des angles :
  
  \'Etant donné un réel 
  constructible $\cos \theta$, est-ce que le réel $\cos \frac\theta3$ est aussi constructible ?

  \change
Si vous n'êtes pas convaincu de l'équivalence des problèmes voici la justification :

Si on a un cube de côté $a$ alors son volume est $a^3$, et il faut construire 
  un cube de volume $2a^3$.
  Ce qui revient à construire un segment de longueur $a\sqrt[3]{2}$ à partir d'un segment de longueur $a$.
  Ce qui équivaut à savoir si $\sqrt[3]{2}$ est un réel constructible ou pas.
  
  \change
  
  Si on a un cercle de rayon $r$, donc d'aire $\pi r^2$. 
  
  Je vous rappelle que $\pi$ soit constructible équivaut à $\sqrt\pi$ constructible. 
  
  Et construire un segment de longueur $\sqrt\pi r$,
  correspond à un carré d'aire $\pi r^2$, donc de même aire que le cercle initial.
  Nous aurions construit un carré de la même aire que le cercle !
  On aurait résolu la quadrature du cercle.


%%%%%%%%%%%%%%%%%%%%%%%%%%%%%%%%%%%%%%%%%%%%%%%%%%%%%%%%%%%
\diapo

Pour finir remarquons que construire un angle géométrique de mesure $\theta$
est équivalent à construire le nombre réel $\cos \theta$.
  
\change
  
  Partons d'un angle géométrique $\theta$, c'est-à-dire partons d'un réel
  $\cos \theta$ constructible. Construire $\cos \frac\theta3$
  est équivalent à construire un angle géométrique de mesure $\frac\theta 3$.
  Si on savait faire cela on aurait résolu la trisection des angles.

\end{document}
