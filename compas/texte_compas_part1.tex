
%%%%%%%%%%%%%%%%%% PREAMBULE %%%%%%%%%%%%%%%%%%


\documentclass[12pt]{article}

\usepackage{amsfonts,amsmath,amssymb,amsthm}
\usepackage[utf8]{inputenc}
\usepackage[T1]{fontenc}
\usepackage[francais]{babel}


% packages
\usepackage{amsfonts,amsmath,amssymb,amsthm}
\usepackage[utf8]{inputenc}
\usepackage[T1]{fontenc}
%\usepackage{lmodern}

\usepackage[francais]{babel}
\usepackage{fancybox}
\usepackage{graphicx}

\usepackage{float}

%\usepackage[usenames, x11names]{xcolor}
\usepackage{tikz}
\usepackage{datetime}

\usepackage{mathptmx}
%\usepackage{fouriernc}
%\usepackage{newcent}
\usepackage[mathcal,mathbf]{euler}

%\usepackage{palatino}
%\usepackage{newcent}


% Commande spéciale prompteur

%\usepackage{mathptmx}
%\usepackage[mathcal,mathbf]{euler}
%\usepackage{mathpple,multido}

\usepackage[a4paper]{geometry}
\geometry{top=2cm, bottom=2cm, left=1cm, right=1cm, marginparsep=1cm}

\newcommand{\change}{{\color{red}\rule{\textwidth}{1mm}\\}}

\newcounter{mydiapo}

\newcommand{\diapo}{\newpage
\hfill {\normalsize  Diapo \themydiapo \quad \texttt{[\jobname]}} \\
\stepcounter{mydiapo}}


%%%%%%% COULEURS %%%%%%%%%%

% Pour blanc sur noir :
%\pagecolor[rgb]{0.5,0.5,0.5}
% \pagecolor[rgb]{0,0,0}
% \color[rgb]{1,1,1}



%\DeclareFixedFont{\myfont}{U}{cmss}{bx}{n}{18pt}
\newcommand{\debuttexte}{
%%%%%%%%%%%%% FONTES %%%%%%%%%%%%%
\renewcommand{\baselinestretch}{1.5}
\usefont{U}{cmss}{bx}{n}
\bfseries

% Taille normale : commenter le reste !
%Taille Arnaud
%\fontsize{19}{19}\selectfont

% Taille Barbara
%\fontsize{21}{22}\selectfont

%Taille François
\fontsize{25}{30}\selectfont

%Taille Pascal
%\fontsize{25}{30}\selectfont

%Taille Laura
%\fontsize{30}{35}\selectfont


%\myfont
%\usefont{U}{cmss}{bx}{n}

%\Huge
%\addtolength{\parskip}{\baselineskip}
}


% \usepackage{hyperref}
% \hypersetup{colorlinks=true, linkcolor=blue, urlcolor=blue,
% pdftitle={Exo7 - Exercices de mathématiques}, pdfauthor={Exo7}}


%section
% \usepackage{sectsty}
% \allsectionsfont{\bf}
%\sectionfont{\color{Tomato3}\upshape\selectfont}
%\subsectionfont{\color{Tomato4}\upshape\selectfont}

%----- Ensembles : entiers, reels, complexes -----
\newcommand{\Nn}{\mathbb{N}} \newcommand{\N}{\mathbb{N}}
\newcommand{\Zz}{\mathbb{Z}} \newcommand{\Z}{\mathbb{Z}}
\newcommand{\Qq}{\mathbb{Q}} \newcommand{\Q}{\mathbb{Q}}
\newcommand{\Rr}{\mathbb{R}} \newcommand{\R}{\mathbb{R}}
\newcommand{\Cc}{\mathbb{C}} 
\newcommand{\Kk}{\mathbb{K}} \newcommand{\K}{\mathbb{K}}

%----- Modifications de symboles -----
\renewcommand{\epsilon}{\varepsilon}
\renewcommand{\Re}{\mathop{\text{Re}}\nolimits}
\renewcommand{\Im}{\mathop{\text{Im}}\nolimits}
%\newcommand{\llbracket}{\left[\kern-0.15em\left[}
%\newcommand{\rrbracket}{\right]\kern-0.15em\right]}

\renewcommand{\ge}{\geqslant}
\renewcommand{\geq}{\geqslant}
\renewcommand{\le}{\leqslant}
\renewcommand{\leq}{\leqslant}

%----- Fonctions usuelles -----
\newcommand{\ch}{\mathop{\mathrm{ch}}\nolimits}
\newcommand{\sh}{\mathop{\mathrm{sh}}\nolimits}
\renewcommand{\tanh}{\mathop{\mathrm{th}}\nolimits}
\newcommand{\cotan}{\mathop{\mathrm{cotan}}\nolimits}
\newcommand{\Arcsin}{\mathop{\mathrm{Arcsin}}\nolimits}
\newcommand{\Arccos}{\mathop{\mathrm{Arccos}}\nolimits}
\newcommand{\Arctan}{\mathop{\mathrm{Arctan}}\nolimits}
\newcommand{\Argsh}{\mathop{\mathrm{Argsh}}\nolimits}
\newcommand{\Argch}{\mathop{\mathrm{Argch}}\nolimits}
\newcommand{\Argth}{\mathop{\mathrm{Argth}}\nolimits}
\newcommand{\pgcd}{\mathop{\mathrm{pgcd}}\nolimits} 

\newcommand{\Card}{\mathop{\text{Card}}\nolimits}
\newcommand{\Ker}{\mathop{\text{Ker}}\nolimits}
\newcommand{\id}{\mathop{\text{id}}\nolimits}
\newcommand{\ii}{\mathrm{i}}
\newcommand{\dd}{\mathrm{d}}
\newcommand{\Vect}{\mathop{\text{Vect}}\nolimits}
\newcommand{\Mat}{\mathop{\mathrm{Mat}}\nolimits}
\newcommand{\rg}{\mathop{\text{rg}}\nolimits}
\newcommand{\tr}{\mathop{\text{tr}}\nolimits}
\newcommand{\ppcm}{\mathop{\text{ppcm}}\nolimits}

%----- Structure des exercices ------

\newtheoremstyle{styleexo}% name
{2ex}% Space above
{3ex}% Space below
{}% Body font
{}% Indent amount 1
{\bfseries} % Theorem head font
{}% Punctuation after theorem head
{\newline}% Space after theorem head 2
{}% Theorem head spec (can be left empty, meaning ‘normal’)

%\theoremstyle{styleexo}
\newtheorem{exo}{Exercice}
\newtheorem{ind}{Indications}
\newtheorem{cor}{Correction}


\newcommand{\exercice}[1]{} \newcommand{\finexercice}{}
%\newcommand{\exercice}[1]{{\tiny\texttt{#1}}\vspace{-2ex}} % pour afficher le numero absolu, l'auteur...
\newcommand{\enonce}{\begin{exo}} \newcommand{\finenonce}{\end{exo}}
\newcommand{\indication}{\begin{ind}} \newcommand{\finindication}{\end{ind}}
\newcommand{\correction}{\begin{cor}} \newcommand{\fincorrection}{\end{cor}}

\newcommand{\noindication}{\stepcounter{ind}}
\newcommand{\nocorrection}{\stepcounter{cor}}

\newcommand{\fiche}[1]{} \newcommand{\finfiche}{}
\newcommand{\titre}[1]{\centerline{\large \bf #1}}
\newcommand{\addcommand}[1]{}
\newcommand{\video}[1]{}

% Marge
\newcommand{\mymargin}[1]{\marginpar{{\small #1}}}



%----- Presentation ------
\setlength{\parindent}{0cm}

%\newcommand{\ExoSept}{\href{http://exo7.emath.fr}{\textbf{\textsf{Exo7}}}}

\definecolor{myred}{rgb}{0.93,0.26,0}
\definecolor{myorange}{rgb}{0.97,0.58,0}
\definecolor{myyellow}{rgb}{1,0.86,0}

\newcommand{\LogoExoSept}[1]{  % input : echelle
{\usefont{U}{cmss}{bx}{n}
\begin{tikzpicture}[scale=0.1*#1,transform shape]
  \fill[color=myorange] (0,0)--(4,0)--(4,-4)--(0,-4)--cycle;
  \fill[color=myred] (0,0)--(0,3)--(-3,3)--(-3,0)--cycle;
  \fill[color=myyellow] (4,0)--(7,4)--(3,7)--(0,3)--cycle;
  \node[scale=5] at (3.5,3.5) {Exo7};
\end{tikzpicture}}
}



\theoremstyle{definition}
%\newtheorem{proposition}{Proposition}
%\newtheorem{exemple}{Exemple}
%\newtheorem{theoreme}{Théorème}
\newtheorem{lemme}{Lemme}
\newtheorem{corollaire}{Corollaire}
%\newtheorem*{remarque*}{Remarque}
%\newtheorem*{miniexercice}{Mini-exercices}
%\newtheorem{definition}{Définition}




%definition d'un terme
\newcommand{\defi}[1]{{\color{myorange}\textbf{\emph{#1}}}}
\newcommand{\evidence}[1]{{\color{blue}\textbf{\emph{#1}}}}



 %----- Commandes divers ------

\newcommand{\codeinline}[1]{\texttt{#1}}

%%%%%%%%%%%%%%%%%%%%%%%%%%%%%%%%%%%%%%%%%%%%%%%%%%%%%%%%%%%%%
%%%%%%%%%%%%%%%%%%%%%%%%%%%%%%%%%%%%%%%%%%%%%%%%%%%%%%%%%%%%%



\begin{document}

\debuttexte


%%%%%%%%%%%%%%%%%%%%%%%%%%%%%%%%%%%%%%%%%%%%%%%%%%%%%%%%%%%
\diapo

\change
Vous avez à votre disposition une règle et un compas et bien sûr du papier 
et un crayon ! Avec si peu de matériel s'ouvre à vous un monde merveilleux 
rempli de géométrie et d'algèbre.

\change
Nous allons voir dans cette première partie que tout 
un tas de constructions sont possibles.

\change
Mais le but de ce cours est de répondre à trois problèmes 
qui datent des mathématiciens grecs :
la trisection des angles, 

\change
la duplication du cube 

\change
ainsi que le célèbre problème de la quadrature du cercle.

%%%%%%%%%%%%%%%%%%%%%%%%%%%%%%%%%%%%%%%%%%%%%%%%%%%%%%%%%%%
\diapo

Nous avons à notre disposition un compas et une règle (non graduée).
On démarre par des constructions élémentaires. 

\change
Si $A,B$ sont deux points donnés du plan, 

\change 
alors on peut construire, à la règle et au compas, 
  le \evidence{symétrique} de $B$ par rapport à $A$.
  
\change  
  Pour cela, il suffit juste de tracer avec la règle la droite $(AB)$ 
  
\change  
  puis le cercle de centre $A$ passant par $B$.
  
\change
  Cette droite et ce cercle se coupent en $B$ bien sûr et aussi en $B' =s_A(B)$,
  le symétrique de $B$ par rapport à $A$.
  
\change
Si $A,B$ sont deux points donnés du plan, alors on peut aussi construire la 
  \evidence{médiatrice} de $[AB]$.
  
\change
  Pour cela, tracer le cercle centré en $A$ passant par $B$ 
  
\change
  et aussi le cercle centré en $B$ passant par~$A$. 
  
\change   
  Ces deux cercles s'intersectent en deux points $C$, $D$. 
  Les points $C$, $D$ appartiennent à la médiatrice de $[AB]$.
  
\change
  Avec la règle on trace la droite $(CD)$ qui est la médiatrice de $[AB]$.
  
\change  
En particulier cela permet de construire le \evidence{milieu} 
  $I$ du segment $[AB]$. En effet, c'est l'intersection 
de la droite $(AB)$ et de la médiatrice $(CD)$ 
que l'on vient de construire. 



%%%%%%%%%%%%%%%%%%%%%%%%%%%%%%%%%%%%%%%%%%%%%%%%%%%%%%%%%%%
\diapo

Si $A, B, C$ sont trois points donnés 

\change
alors  on peut construire la \evidence{parallèle} à la droite $(AB)$ passant par $C$.

\change
  Tout d'abord je construis le milieu $I$ de $[AC]$. 
  
\change
  Puis je construis $D$ le symétrique de $B$ par rapport à $I$.
  
\change
  La figure $ABCD$ est un \evidence{parallélogramme}, 
  donc la droite $(CD)$ est bien la parallèle à la droite $(AB)$
  passant par $C$.
  
\change
Pour construire la \evidence{perpendiculaire} à $(AB)$ 
  passant par un point $C$, 
  
\change
on construit d'abord la médiatrice de $[AB]$, 

\change
puis la parallèle   à cette médiatrice passant par $C$. 

%%%%%%%%%%%%%%%%%%%%%%%%%%%%%%%%%%%%%%%%%%%%%%%%%%%%%%%%%%%
\diapo

Il est peut-être temps d'expliquer ce que l'on est autorisé à faire.
Voici les règles du jeu : partez de points sur une feuille.
Vous pouvez maintenant tracer d'autres points, à partir de cercles et de droites en respectant
les conditions suivantes :

\change
vous pouvez tracer une droite entre deux points déjà construits,


Par exemple à partir de ces points 

\change
on peut tracer cette droite

\change
Deuxième règle : vous pouvez tracer un cercle dont le centre est un point construit et qui passe
  par un autre point construit,
  
\change


Par exemple ici on a tracé le cercle dont ce point est le centre et qui passe par cet autre point.

\change
Dernière règle : 
vous pouvez utiliser pour vos constructions futures les points obtenus comme intersections de deux droites tracées,
ou bien intersections d'une droite et d'un cercle tracé,  ou bien intersections de deux cercles tracés.
  
\change

\change
Par exemple on trace une première droite

\change
puis une seconde

\change
on a construit le point d'intersection !

\change

\change
Autre exemple si on trace ce cercle 

\change
et cette droite,

\change
Alors on a construit ces points d'intersection.

\change

\change

\change

Enfin ces deux cercles sont centrés en des points déjà construits, les points bleus,
et passent par des points déjà construits. Ce qui nous permet de construire deux nouveaux points en rouge.



%%%%%%%%%%%%%%%%%%%%%%%%%%%%%%%%%%%%%%%%%%%%%%%%%%%%%%%%%%%
\diapo


On peut \evidence{conserver l'écartement du compas}. 
C'est une propriété importante qui simplifie les constructions.
 
\change 
Si l'on a placé des points $A,B,A'$

\change
  alors on peut placer la pointe en $A$ avec un écartement de longueur $AB$. 
  C'est-à-dire que l'on peut mesurer le segment $[AB]$, 
  
\change
puis soulever le compas en gardant l'écartement... 


\change
...pour tracer le cercle centré en $A'$ et d'écartement $AB$. 

\change
  Cette opération se justifie de la façon suivante : on pourrait construire le point 
  $B'$ tel que $A'ABB'$ soit un parallélogramme et ensuite tracer le cercle centré en $A'$
  passant par $B'$.


%%%%%%%%%%%%%%%%%%%%%%%%%%%%%%%%%%%%%%%%%%%%%%%%%%%%%%%%%%%
\diapo

En conservant l'écartement du compas, nous pouvons plus facilement construire 
  des parallélogrammes, avec seulement deux traits de compas. 
  
Donnons-nous trois points $A,B,C$.  
  
\change
On mesure l'écartement $[AB]$, on trace le cercle centré en $C$ de rayon $AB$.
  
\change  
Puis on mesure l'écartement $[BC]$ et on trace le cercle centré en $A$ de rayon $BC$.  
  
\change
Ces deux cercles se recoupent en deux points, dont l'un est $D$,...

\change
...tel que $ABCD$ est un parallélogramme.


%%%%%%%%%%%%%%%%%%%%%%%%%%%%%%%%%%%%%%%%%%%%%%%%%%%%%%%%%%%
\diapo


Voyons comment le théorème de Thalès nous permet de diviser un segment en 
$n$ morceaux.

Fixons $n$ un entier, pour nous ce sera $n=5$.
Voici les étapes pour diviser un segment $[AB]$ en $5$ parts égales.

\change
Tracer une droite $\mathcal{D}$ quelconque, passant par $A$, autre que la droite $(AB)$.
  
\change
Prendre un écartement quelconque du compas. Sur la droite $\mathcal{D}$ et en partant de $A$,
on trace un point $A_1$ correspondant à cet écartement.

\change
En gardant le même écartement de compas on trace $5$ segments de même longueur.

\change

\change

\change

\change

 On obtient des points $A_1,A_2,\ldots,A_5$.
 
\change 
On trace maintenant la droite reliant le dernier point $A_5$ à $B$.

\change
Puis on trace les parallèles à cette droite passant par chacun des points $A_i$.

\change

\change

\change

\change

\change
  Ces droites recoupent le segment $[AB]$ en des points $B_1,B_2,...$ qui découpent l'intervalle
  $[AB]$ en $5$ segments égaux.
  
Je vous laisse prouver que cette construction fonctionne grâce au théorème de Thalès.


%%%%%%%%%%%%%%%%%%%%%%%%%%%%%%%%%%%%%%%%%%%%%%%%%%%%%%%%%%%
\diapo

Je vous laisse réfléchir avec la construction suivante de certaines racines carrées.

Par exemple supposons que l'on parte d'un segment de longueur $1$,
il est facile de construire un segment de longueur $\sqrt2$,
c'est la longueur de la diagonale du carré de côté $1$.

\change
Je vous laisse comprendre et démontrer comment on peut construire $\sqrt3$,

\change
Puis $\sqrt 4$, $\sqrt5$, etc.

La clé de la démonstration c'est le théorème de Pythagore !


%%%%%%%%%%%%%%%%%%%%%%%%%%%%%%%%%%%%%%%%%%%%%%%%%%%%%%%%%%%
\diapo

Voici maintenant trois questions qui datent de la Grèce antique 
et qui vont nous occuper le reste du chapitre.

Considérons un angle $\theta$, c'est-à-dire la donnée d'un point $A$ et de deux demi-droites 
issues de ce point.

\change
Nous savons diviser cet angle en deux à l'aide d'une règle et d'un compas :
il suffit de tracer la bissectrice. 

\change
Pour cela on fixe un écartement de compas et on trace un cercle
centré en $A$ : 

\change
il recoupe les demi-droites en des points $B$ et $C$. 

\change
On trace maintenant deux cercles centrés en $B$ puis $C$ (avec le même rayon pour les deux cercles). 


Si $D$ est un point de l'intersection de ces
deux cercles ...

\change
...alors la droite $(AD)$ est la bissectrice de l'angle.

\change
Et on a bien divisé notre angle en deux.


%%%%%%%%%%%%%%%%%%%%%%%%%%%%%%%%%%%%%%%%%%%%%%%%%%%%%%%%%%%
\diapo

Voici notre premier problème : c'est le problème de la trisection.

Peut-on diviser un angle donné en trois 
angles égaux à l'aide de la règle et du compas ?




%%%%%%%%%%%%%%%%%%%%%%%%%%%%%%%%%%%%%%%%%%%%%%%%%%%%%%%%%%%
\diapo

Avant d'aborder le problème suivant : commençons par un problème assez simple.

étant donné un carré, construire (à la règle et au compas)
un carré dont l'aire est le double. 

\change
Voici le premier carré avec  un coté de longueur $a$

Sa surface est donc $a^2$.

On souhaite donc construire un carré de surface $2 \times a^2$.

\change
Cela revient à savoir tracer un côté de longueur
$a\sqrt2$ car alors l'aire du nouveau carré est $(a\sqrt2)^2$ qui est bien $2 \times a^2$.


\change
C'est facile de construire un tel carré car à partir du premier carré.

En effet la diagonale de notre carré original 
a bien la longueur voulue $a\sqrt 2$. 

Partant de cette longueur, on construit un carré dont l'aire est
est bien le double du carré de départ.


%%%%%%%%%%%%%%%%%%%%%%%%%%%%%%%%%%%%%%%%%%%%%%%%%%%%%%%%%%%
\diapo

Posons nous la question dans l'espace : étant donné un cube, peut-on construire
un second cube dont le volume est le double du premier ?

\change
Si le premier cube a ses côtés de longueur $a$, alors le volume est $a^3$. 

\change
Ainsi on veut que le second cube est un volume de $2 \times a^3$, donc doit
avoir ses côtés de longueur $a\sqrt[3]{2}$. 

\change
Le problème de la duplication du cube se formule alors de la manière suivante :


\'Etant donné un segment de longueur $1$, peut-on 
construire à la règle et au compas 
un segment de longueur $\sqrt[3]{2}$ ?

%%%%%%%%%%%%%%%%%%%%%%%%%%%%%%%%%%%%%%%%%%%%%%%%%%%%%%%%%%%
\diapo

On termine par le célèbre problème de la quadrature du cercle !


\'Etant donné un cercle, 
peut-on construire à la règle et au compas 
un carré de même aire ?


Si le cercle est de rayon $r$ alors sa surface est $\pi r^2$,

Donc pour que le carré ait la même surface il faudrait construire
un carré de longueur $\sqrt{\pi} \cdot r$.

\change
En d'autres termes cela revient à savoir construire un segment de 
longueur $\sqrt{\pi}$ à la règle et au compas, à
partir d'un segment de longueur $1$.

\end{document}
