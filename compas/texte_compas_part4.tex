
%%%%%%%%%%%%%%%%%% PREAMBULE %%%%%%%%%%%%%%%%%%


\documentclass[12pt]{article}

\usepackage{amsfonts,amsmath,amssymb,amsthm}
\usepackage[utf8]{inputenc}
\usepackage[T1]{fontenc}
\usepackage[francais]{babel}


% packages
\usepackage{amsfonts,amsmath,amssymb,amsthm}
\usepackage[utf8]{inputenc}
\usepackage[T1]{fontenc}
%\usepackage{lmodern}

\usepackage[francais]{babel}
\usepackage{fancybox}
\usepackage{graphicx}

\usepackage{float}

%\usepackage[usenames, x11names]{xcolor}
\usepackage{tikz}
\usepackage{datetime}

\usepackage{mathptmx}
%\usepackage{fouriernc}
%\usepackage{newcent}
\usepackage[mathcal,mathbf]{euler}

%\usepackage{palatino}
%\usepackage{newcent}


% Commande spéciale prompteur

%\usepackage{mathptmx}
%\usepackage[mathcal,mathbf]{euler}
%\usepackage{mathpple,multido}

\usepackage[a4paper]{geometry}
\geometry{top=2cm, bottom=2cm, left=1cm, right=1cm, marginparsep=1cm}

\newcommand{\change}{{\color{red}\rule{\textwidth}{1mm}\\}}

\newcounter{mydiapo}

\newcommand{\diapo}{\newpage
\hfill {\normalsize  Diapo \themydiapo \quad \texttt{[\jobname]}} \\
\stepcounter{mydiapo}}


%%%%%%% COULEURS %%%%%%%%%%

% Pour blanc sur noir :
%\pagecolor[rgb]{0.5,0.5,0.5}
% \pagecolor[rgb]{0,0,0}
% \color[rgb]{1,1,1}



%\DeclareFixedFont{\myfont}{U}{cmss}{bx}{n}{18pt}
\newcommand{\debuttexte}{
%%%%%%%%%%%%% FONTES %%%%%%%%%%%%%
\renewcommand{\baselinestretch}{1.5}
\usefont{U}{cmss}{bx}{n}
\bfseries

% Taille normale : commenter le reste !
%Taille Arnaud
%\fontsize{19}{19}\selectfont

% Taille Barbara
%\fontsize{21}{22}\selectfont

%Taille François
\fontsize{25}{30}\selectfont

%Taille Pascal
%\fontsize{25}{30}\selectfont

%Taille Laura
%\fontsize{30}{35}\selectfont


%\myfont
%\usefont{U}{cmss}{bx}{n}

%\Huge
%\addtolength{\parskip}{\baselineskip}
}


% \usepackage{hyperref}
% \hypersetup{colorlinks=true, linkcolor=blue, urlcolor=blue,
% pdftitle={Exo7 - Exercices de mathématiques}, pdfauthor={Exo7}}


%section
% \usepackage{sectsty}
% \allsectionsfont{\bf}
%\sectionfont{\color{Tomato3}\upshape\selectfont}
%\subsectionfont{\color{Tomato4}\upshape\selectfont}

%----- Ensembles : entiers, reels, complexes -----
\newcommand{\Nn}{\mathbb{N}} \newcommand{\N}{\mathbb{N}}
\newcommand{\Zz}{\mathbb{Z}} \newcommand{\Z}{\mathbb{Z}}
\newcommand{\Qq}{\mathbb{Q}} \newcommand{\Q}{\mathbb{Q}}
\newcommand{\Rr}{\mathbb{R}} \newcommand{\R}{\mathbb{R}}
\newcommand{\Cc}{\mathbb{C}} 
\newcommand{\Kk}{\mathbb{K}} \newcommand{\K}{\mathbb{K}}

%----- Modifications de symboles -----
\renewcommand{\epsilon}{\varepsilon}
\renewcommand{\Re}{\mathop{\text{Re}}\nolimits}
\renewcommand{\Im}{\mathop{\text{Im}}\nolimits}
%\newcommand{\llbracket}{\left[\kern-0.15em\left[}
%\newcommand{\rrbracket}{\right]\kern-0.15em\right]}

\renewcommand{\ge}{\geqslant}
\renewcommand{\geq}{\geqslant}
\renewcommand{\le}{\leqslant}
\renewcommand{\leq}{\leqslant}

%----- Fonctions usuelles -----
\newcommand{\ch}{\mathop{\mathrm{ch}}\nolimits}
\newcommand{\sh}{\mathop{\mathrm{sh}}\nolimits}
\renewcommand{\tanh}{\mathop{\mathrm{th}}\nolimits}
\newcommand{\cotan}{\mathop{\mathrm{cotan}}\nolimits}
\newcommand{\Arcsin}{\mathop{\mathrm{Arcsin}}\nolimits}
\newcommand{\Arccos}{\mathop{\mathrm{Arccos}}\nolimits}
\newcommand{\Arctan}{\mathop{\mathrm{Arctan}}\nolimits}
\newcommand{\Argsh}{\mathop{\mathrm{Argsh}}\nolimits}
\newcommand{\Argch}{\mathop{\mathrm{Argch}}\nolimits}
\newcommand{\Argth}{\mathop{\mathrm{Argth}}\nolimits}
\newcommand{\pgcd}{\mathop{\mathrm{pgcd}}\nolimits} 

\newcommand{\Card}{\mathop{\text{Card}}\nolimits}
\newcommand{\Ker}{\mathop{\text{Ker}}\nolimits}
\newcommand{\id}{\mathop{\text{id}}\nolimits}
\newcommand{\ii}{\mathrm{i}}
\newcommand{\dd}{\mathrm{d}}
\newcommand{\Vect}{\mathop{\text{Vect}}\nolimits}
\newcommand{\Mat}{\mathop{\mathrm{Mat}}\nolimits}
\newcommand{\rg}{\mathop{\text{rg}}\nolimits}
\newcommand{\tr}{\mathop{\text{tr}}\nolimits}
\newcommand{\ppcm}{\mathop{\text{ppcm}}\nolimits}

%----- Structure des exercices ------

\newtheoremstyle{styleexo}% name
{2ex}% Space above
{3ex}% Space below
{}% Body font
{}% Indent amount 1
{\bfseries} % Theorem head font
{}% Punctuation after theorem head
{\newline}% Space after theorem head 2
{}% Theorem head spec (can be left empty, meaning ‘normal’)

%\theoremstyle{styleexo}
\newtheorem{exo}{Exercice}
\newtheorem{ind}{Indications}
\newtheorem{cor}{Correction}


\newcommand{\exercice}[1]{} \newcommand{\finexercice}{}
%\newcommand{\exercice}[1]{{\tiny\texttt{#1}}\vspace{-2ex}} % pour afficher le numero absolu, l'auteur...
\newcommand{\enonce}{\begin{exo}} \newcommand{\finenonce}{\end{exo}}
\newcommand{\indication}{\begin{ind}} \newcommand{\finindication}{\end{ind}}
\newcommand{\correction}{\begin{cor}} \newcommand{\fincorrection}{\end{cor}}

\newcommand{\noindication}{\stepcounter{ind}}
\newcommand{\nocorrection}{\stepcounter{cor}}

\newcommand{\fiche}[1]{} \newcommand{\finfiche}{}
\newcommand{\titre}[1]{\centerline{\large \bf #1}}
\newcommand{\addcommand}[1]{}
\newcommand{\video}[1]{}

% Marge
\newcommand{\mymargin}[1]{\marginpar{{\small #1}}}



%----- Presentation ------
\setlength{\parindent}{0cm}

%\newcommand{\ExoSept}{\href{http://exo7.emath.fr}{\textbf{\textsf{Exo7}}}}

\definecolor{myred}{rgb}{0.93,0.26,0}
\definecolor{myorange}{rgb}{0.97,0.58,0}
\definecolor{myyellow}{rgb}{1,0.86,0}

\newcommand{\LogoExoSept}[1]{  % input : echelle
{\usefont{U}{cmss}{bx}{n}
\begin{tikzpicture}[scale=0.1*#1,transform shape]
  \fill[color=myorange] (0,0)--(4,0)--(4,-4)--(0,-4)--cycle;
  \fill[color=myred] (0,0)--(0,3)--(-3,3)--(-3,0)--cycle;
  \fill[color=myyellow] (4,0)--(7,4)--(3,7)--(0,3)--cycle;
  \node[scale=5] at (3.5,3.5) {Exo7};
\end{tikzpicture}}
}



\theoremstyle{definition}
%\newtheorem{proposition}{Proposition}
%\newtheorem{exemple}{Exemple}
%\newtheorem{theoreme}{Théorème}
\newtheorem{lemme}{Lemme}
\newtheorem{corollaire}{Corollaire}
%\newtheorem*{remarque*}{Remarque}
%\newtheorem*{miniexercice}{Mini-exercices}
%\newtheorem{definition}{Définition}




%definition d'un terme
\newcommand{\defi}[1]{{\color{myorange}\textbf{\emph{#1}}}}
\newcommand{\evidence}[1]{{\color{blue}\textbf{\emph{#1}}}}



 %----- Commandes divers ------

\newcommand{\codeinline}[1]{\texttt{#1}}

%%%%%%%%%%%%%%%%%%%%%%%%%%%%%%%%%%%%%%%%%%%%%%%%%%%%%%%%%%%%%
%%%%%%%%%%%%%%%%%%%%%%%%%%%%%%%%%%%%%%%%%%%%%%%%%%%%%%%%%%%%%

\newcommand{\construc}{\mathcal{C}}
\newcommand{\plan}{\mathcal{P}}
\newcommand{\cercle}{\mathcal{C}}

\begin{document}

\debuttexte


%%%%%%%%%%%%%%%%%%%%%%%%%%%%%%%%%%%%%%%%%%%%%%%%%%%%%%%%%%%
\diapo

\change
Cette partie est la charnière de ce chapitre. Nous expliquons à quoi 
correspondent algébriquement les opérations géométriques effectuées à la règle et au compas.

\change
On commence par le théorème de Wantzel qui relie les nombres
constructibles avec les extensions quadratiques

\change
La principale application sera que les nombres constructibles doivent vérifier certaines conditions algébriques

\change
On expliquera le théorème de Wantzel sur des exemples...

\change
... avant d'en détailler la preuve.


%%%%%%%%%%%%%%%%%%%%%%%%%%%%%%%%%%%%%%%%%%%%%%%%%%%%%%%%%%%
\diapo

Voici le résultat théorique le plus important de ce chapitre.
C'est Pierre-Laurent Wantzel qui a démontré ce théorème en 1837, à l'âge de 23 ans.

Théorème : 
Un nombre réel $x$ est constructible si et seulement s'il existe 
des extensions quadratiques 
$$K_0 \subset K_1 \subset \cdots \subset K_r$$

où le corps de départ est $K_0 = \Qq$ 

et $K_r$ le corps d'arrivée contient notre nombre constructible $x$.



\change
Ce qui est important c'est que chacune des extensions $K_{i+1}$ est une extension quadratique de $K_i$,

\change
c'est-à-dire $[K_{i+1}:K_i]=2$.

\change
Dire que chaque extension est une extension quadratique de la précédente,

c'est dire que $K_{i+1} = K_i(\sqrt{\delta_i})$ pour un certain $\delta_i \in K_i$. 

\change
Donc en partant $K_0 = \Qq$, les extensions sont de la forme :
$$\Qq \subset \Qq(\sqrt \delta_0) \subset \Qq(\sqrt \delta_0)(\sqrt \delta_1) \subset \cdots$$



%%%%%%%%%%%%%%%%%%%%%%%%%%%%%%%%%%%%%%%%%%%%%%%%%%%%%%%%%%%
\diapo


La conséquence la plus importante du théorème de Wantzel
est donnée par l'énoncé suivant. C'est ce résultat que l'on utilisera 
dans la pratique.

Corollaire :
Tout nombre réel constructible est un nombre algébrique
dont le degré algébrique est de la forme $2^n$, $n\ge 0$. 

Il y a donc deux résultats en un seul : 

(1) un nombre constructible est un nombre algébrique

et (2)  il y a une contrainte forte sur son degré algébrique qui est nécessairement une puissance de $2$.

\change
On commence par expliquer comment le théorème de Wantzel permet d'obtenir ce corollaire.
Pour cela partons de $x$ un nombre réel constructible. 

\change
Par le théorème de Wantzel, il existe des extensions quadratiques 
$\Qq = K_0 \subset K_1 \subset \cdots \subset K_r$
telles que $x$ appartienne à la dernière extension $K_r$. 

\change
Puisque chacune des extensions est de degré $2$ sur la précédente 
alors $K_r$ est une extension finie de $\Qq$.

Ainsi $x$ appartient à une extension de $\Qq$ de degré fini,

\change
Et on sait que tout élément d'une extension finie est un nombre algébrique !

\change

On va maintenant calculer le degré algébrique.

Comme le degré de l'extension  $K_1$ sur $K_0$ est $2$
de même pour $K_2$ sur $K_1$, alors le degré de $K_2$ sur $K_0$ est $4$.

\change

Par récurrence, le degré de l'extension $K_r$ sur $K_0$ est de $2^r$, mais je vous rappelle que $K_0 = \Qq$.

\change
Il nous reste à en déduire le degré algébrique $[\Qq(x):\Qq]$.
On sait que $\Qq(x) \subset K_r$ 

et nous savons qu'alors  $[K_r:\Qq(x)] \times [\Qq(x):\Qq]=[K_r:\Qq]=2^r$.

\change
Cela implique que $[\Qq(x):\Qq]$ divise $2^r$...

\change
...et est donc nécessairement une puissance de $2$.

%%%%%%%%%%%%%%%%%%%%%%%%%%%%%%%%%%%%%%%%%%%%%%%%%%%%%%%%%%%
\diapo


Donnons des exemples des extensions nécessaires dans les trois cas qui vont 
être les trois cas de la preuve du théorème de Wantzel.


\change
Le premier cas est lorsque l'on construit un point $P$ 
comme l'intersection de deux droites.

\change
On a donc deux droites $(AB)$ et $(A'B')$.

\change
et les coordonnées des points $A, B, A', B'$ sont des points à 
coordonnées rationnelles (et même ici entières).


\change
On calcule les équations des droites :
  $2x+y=1$ et $3x-y=2$ et bien sûr les coefficients sont des nombres rationnels.
  
\change  
Le point d'intersection $P$ a pour coordonnées les solutions du système linéaire
formé par ces deux équations : 

\change
on trouve $(\frac 35, -\frac 15)$,

\change
L'abscisse et l'ordonnée sont bien des nombres rationnels. 

\change
Dans ce cas il n'y a pas besoin d'extension de corps. Le corps original
$K = \Qq$ qui contenait les coordonnées de nos points de départ, contient aussi les coordonnées
du point d'intersection.


%%%%%%%%%%%%%%%%%%%%%%%%%%%%%%%%%%%%%%%%%%%%%%%%%%%%%%%%%%%
\diapo

Voici maintenant un exemple où $P$ est obtenu comme l'intersection d'une droite
et d'un cercle.

\change
La droite est définie par les deux points $A$ et $B$ à coordonnées rationnelles.

\change
Son équation est $2x+y=1$.

\change
Le cercle a pour centre le point $A'$ de coordonnées $(2,1)$
et passe par le point $B'$ de coordonnées $(-1,1)$. Encore une fois les abscisses et ordonnées 
de départ sont toutes des nombres rationnels.

\change
Le rayon du cercle est $3$ et l'équation du cercle est 
$(x-2)^2+(y-1)^2=9$.

\change

Les coordonnées des points d'intersections sont obtenues en résolvant le systèmes formé de ces deux équations.

On trouve bien sûr deux solutions, tout d'abord le point $P$ de coordonnées :
  $$\left(\frac 1 5 \left(2- \sqrt{29}\right),\frac 1 5 \left(1+2\sqrt{29}\right) \right)$$

\change
et aussi le point $P'$ avec un changement de signe ici et là.
  
\change
Les coordonnées ne sont pas des nombres rationnels
mais elles sont de la forme $\alpha+\beta\sqrt{\delta}$ avec $\delta = 29$
et $\alpha,\beta$ des rationnels.

\change
Il faut donc étendre le corps, et le corps qui contient les coordonnées
des points d'intersections est $\Qq(\sqrt{29})$,
qui est bien une extension quadratique de $\Qq$.

%%%%%%%%%%%%%%%%%%%%%%%%%%%%%%%%%%%%%%%%%%%%%%%%%%%%%%%%%%%
\diapo

Le dernier exemple est l'intersection de deux cercles.

\change
Le premier cercle est centré en  $(-1,0)$ et passe par $(1,0)$ ; son rayon est $2$
et son équation est $(x+1)^2+y^2=4$.

\change 
Le second cercle est centré en $(2,1)$ et passe par l'origine, son rayon est $\sqrt{5}$ mais son
équation est $(x-2)^2+(y-1)^2=5$ qui est bien à coefficients rationnels.

\change

Il y a deux points d'intersections,
Le premier a pour coordonnées 
  $$\left(\frac{1}{20}\left( 7 - \sqrt{79}\right),\frac{3}{20}\left( 3 + \sqrt{79}\right)\right)$$
  
\change
Pour le second on change les signes.

\change
Encore une fois les abscisses et ordonnées sont de la forme $\alpha+\beta\sqrt\delta$ avec 
$\delta = 79$ et $\alpha,\beta \in \Qq$.

\change
L'extension qui convient est l'extension quadratique $\Qq(\sqrt{79})$.


%%%%%%%%%%%%%%%%%%%%%%%%%%%%%%%%%%%%%%%%%%%%%%%%%%%%%%%%%%%
\diapo

Nous terminons cette séquence par la preuve du théorème de Wantzel.

Je vous rappelle l'énoncé. 

\change

Il y a un sens facile : 

\change
comme on sait construire les racines carrées des nombres constructibles

\change
alors on sait construire tout élément d'une extension quadratique 
$K_1 = \Qq(\sqrt \delta_0)$, 

\change
puis par récurrence tout élément de $K_2$, $K_3$,...


%%%%%%%%%%%%%%%%%%%%%%%%%%%%%%%%%%%%%%%%%%%%%%%%%%%%%%%%%%%
\diapo
\change
Rappelons nous que les points constructibles sont construits
par étapes $\construc_0$ qui contient seulement deux points,
à partir desquels on construit $\construc_1$, puis on construit $\construc_2$, etc.

\change
L'ensemble $\construc_{j+1}$ s'obtient 
à partir de $\construc_j$ en ajoutant les intersections des droites et des cercles
que l'on peut tracer à partir de $\construc_j$.


Nous allons voir que ce passage correspond à une extension quadratique.

\change
Notons $K$ le plus petit corps contenant les coordonnées des points de $\construc_j$.

\change
Soit maintenant $P$ un point de l'étape d'après, donc un point de $\construc_{j+1}$.

Il s'agit de montrer que les coordonnées de $P$ sont dans une extension quadratique de $K$.

Ce point $P$ est l'intersection de deux objets (deux droites ; une droite et un cercle ;
ou deux cercles). On va faire une étude au cas par cas.

\change
On commence par le plus simple : $P$ est l'intersection de deux droites.

\change
On part de deux points $A$ et $B$ dont les coordonnées sont dans le corps $K$.

\change
A partir de ces coordonnées, on calcule une équation de la droite $(AB)$ :
$y = \frac{y_B-y_A}{x_B-x_A} (x-x_A)+ y_A$,

\change
il est important de noter que cette équation est de la forme $ax+by=c$ 
 avec $a,b,c$ des éléments toujours du même corps $K$.
 
\change
De même pour l'autre droite, elle a une équation $a'x+b'y=c'$, avec  $a',b',c'$ dans $K$.

\change 
Les coordonnées du point d'intersection $P$ sont donc solution du système linéaire formé par ces deux équations, l'unique solution est
$$\left(\frac{cb'-c'b}{ab'-a'b} , \frac{ac'-a'c}{ab'-a'b} \right).$$

\change
Comme $K$ est un corps alors l'abscisse et l'ordonnée de ce $P$ sont encore dans $K$.

\change
Dans ce cas il n'y a pas besoin d'extension : 
le plus petit corps contenant les coordonnées des points de $\construc_j$ et de $P$ est $K$.


%%%%%%%%%%%%%%%%%%%%%%%%%%%%%%%%%%%%%%%%%%%%%%%%%%%%%%%%%%%
\diapo

Passons au deuxième cas ; c'est le cas où $P$ est intersection d'une droite et d'un cercle.


\change
Notons l'équation de la droite $ax+by=c$ avec $a,b,c$ des éléments du corps $K$.

\change
On a l'équation du cercle qui est $(x-x_0)^2+(y-y_0)^2=r^2$. 

$x_0,y_0$ sont les coordonnées du centre donc sont dans $K$,

le rayon au carré mesure la distance au carré entre deux points à coordonnées dans $K$ donc est dans $K$.

Le rayon $r$ lui n'est pas toujours dans $K$ mais ce n'est pas important.

\change
Il reste à calculer les intersections de la droite et du cercle,

\change
pour cela on pose $\delta$ cette quantité là.

Notez que c'est bien une somme et produit d'éléments de $K$, donc 
$\delta$ appartient à $K$.

\change

Le premier point d'intersection a pour abscisse $x$ qui vaut ceci.
Notez que cette fois tous les termes sont dans $K$, sauf ici où apparaît $\sqrt{\delta}$.

Comme ce point est sur la droite d'équation $ax+by=c$, une fois que l'on $x$, on en déduit $y$, qui contiendra aussi
du $\sqrt{\delta}$.

\change
Les formules sont similaires pour le second point, il suffit de changer ici un signe en "moins".

\change
Les coordonnées sont bien de la forme $\alpha+\beta\sqrt{\delta}$ avec $\alpha,\beta \in K$
et c'est le même $\sqrt{\delta}$ pour $x,y,x',y'$.

\change
Donc les coordonnées de $P$ sont bien dans l'extension $K(\sqrt \delta)$, 
qui est bien une extension quadratique de $K$.


%%%%%%%%%%%%%%%%%%%%%%%%%%%%%%%%%%%%%%%%%%%%%%%%%%%%%%%%%%%
\diapo

La preuve est presque terminée !

Il reste le cas où $P$ est l'intersection de deux cercles.

\change
Pour ce cas on trouve deux points dont les coordonnées sont 
aussi de la forme $\alpha+\beta\sqrt\delta$
pour un certain $\delta \in K$ qui est le même pour les deux points.

Je vous épargne les formules complètes qui sont plus longues à écrire
et je vous renvoie à l'exemple vu précédemment.

\change
En résumé, dans tous les cas, les coordonnées de $P$ sont dans $K$ ou dans 
une extension quadratique du corps $K$, 
où $K$ est le corps qui contient les coefficients qui servent à construire $P$.


Par récurrence cela donne la preuve du théorème de Wantzel.

\end{document}
