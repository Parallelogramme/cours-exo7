
%%%%%%%%%%%%%%%%%% PREAMBULE %%%%%%%%%%%%%%%%%%

\documentclass[aspectratio=169,utf8]{beamer}
%\documentclass[aspectratio=169,handout]{beamer}

\usetheme{Boadilla}
%\usecolortheme{seahorse}
\usecolortheme[RGB={245,66,24}]{structure}
\useoutertheme{infolines}

% packages
\usepackage{amsfonts,amsmath,amssymb,amsthm}
\usepackage[utf8]{inputenc}
\usepackage[T1]{fontenc}
\usepackage{lmodern}

\usepackage[francais]{babel}
\usepackage{fancybox}
\usepackage{graphicx}

\usepackage{float}
\usepackage{xfrac}

%\usepackage[usenames, x11names]{xcolor}
\usepackage{tikz}
\usepackage{pgfplots}
\usepackage{datetime}



%-----  Package unités -----
\usepackage{siunitx}
\sisetup{locale = FR,detect-all,per-mode = symbol}

%\usepackage{mathptmx}
%\usepackage{fouriernc}
%\usepackage{newcent}
%\usepackage[mathcal,mathbf]{euler}

%\usepackage{palatino}
%\usepackage{newcent}
% \usepackage[mathcal,mathbf]{euler}



% \usepackage{hyperref}
% \hypersetup{colorlinks=true, linkcolor=blue, urlcolor=blue,
% pdftitle={Exo7 - Exercices de mathématiques}, pdfauthor={Exo7}}


%section
% \usepackage{sectsty}
% \allsectionsfont{\bf}
%\sectionfont{\color{Tomato3}\upshape\selectfont}
%\subsectionfont{\color{Tomato4}\upshape\selectfont}

%----- Ensembles : entiers, reels, complexes -----
\newcommand{\Nn}{\mathbb{N}} \newcommand{\N}{\mathbb{N}}
\newcommand{\Zz}{\mathbb{Z}} \newcommand{\Z}{\mathbb{Z}}
\newcommand{\Qq}{\mathbb{Q}} \newcommand{\Q}{\mathbb{Q}}
\newcommand{\Rr}{\mathbb{R}} \newcommand{\R}{\mathbb{R}}
\newcommand{\Cc}{\mathbb{C}} 
\newcommand{\Kk}{\mathbb{K}} \newcommand{\K}{\mathbb{K}}

%----- Modifications de symboles -----
\renewcommand{\epsilon}{\varepsilon}
\renewcommand{\Re}{\mathop{\text{Re}}\nolimits}
\renewcommand{\Im}{\mathop{\text{Im}}\nolimits}
%\newcommand{\llbracket}{\left[\kern-0.15em\left[}
%\newcommand{\rrbracket}{\right]\kern-0.15em\right]}

\renewcommand{\ge}{\geqslant}
\renewcommand{\geq}{\geqslant}
\renewcommand{\le}{\leqslant}
\renewcommand{\leq}{\leqslant}
\renewcommand{\epsilon}{\varepsilon}

%----- Fonctions usuelles -----
\newcommand{\ch}{\mathop{\text{ch}}\nolimits}
\newcommand{\sh}{\mathop{\text{sh}}\nolimits}
\renewcommand{\tanh}{\mathop{\text{th}}\nolimits}
\newcommand{\cotan}{\mathop{\text{cotan}}\nolimits}
\newcommand{\Arcsin}{\mathop{\text{arcsin}}\nolimits}
\newcommand{\Arccos}{\mathop{\text{arccos}}\nolimits}
\newcommand{\Arctan}{\mathop{\text{arctan}}\nolimits}
\newcommand{\Argsh}{\mathop{\text{argsh}}\nolimits}
\newcommand{\Argch}{\mathop{\text{argch}}\nolimits}
\newcommand{\Argth}{\mathop{\text{argth}}\nolimits}
\newcommand{\pgcd}{\mathop{\text{pgcd}}\nolimits} 


%----- Commandes divers ------
\newcommand{\ii}{\mathrm{i}}
\newcommand{\dd}{\text{d}}
\newcommand{\id}{\mathop{\text{id}}\nolimits}
\newcommand{\Ker}{\mathop{\text{Ker}}\nolimits}
\newcommand{\Card}{\mathop{\text{Card}}\nolimits}
\newcommand{\Vect}{\mathop{\text{Vect}}\nolimits}
\newcommand{\Mat}{\mathop{\text{Mat}}\nolimits}
\newcommand{\rg}{\mathop{\text{rg}}\nolimits}
\newcommand{\tr}{\mathop{\text{tr}}\nolimits}


%----- Structure des exercices ------

\newtheoremstyle{styleexo}% name
{2ex}% Space above
{3ex}% Space below
{}% Body font
{}% Indent amount 1
{\bfseries} % Theorem head font
{}% Punctuation after theorem head
{\newline}% Space after theorem head 2
{}% Theorem head spec (can be left empty, meaning ‘normal’)

%\theoremstyle{styleexo}
\newtheorem{exo}{Exercice}
\newtheorem{ind}{Indications}
\newtheorem{cor}{Correction}


\newcommand{\exercice}[1]{} \newcommand{\finexercice}{}
%\newcommand{\exercice}[1]{{\tiny\texttt{#1}}\vspace{-2ex}} % pour afficher le numero absolu, l'auteur...
\newcommand{\enonce}{\begin{exo}} \newcommand{\finenonce}{\end{exo}}
\newcommand{\indication}{\begin{ind}} \newcommand{\finindication}{\end{ind}}
\newcommand{\correction}{\begin{cor}} \newcommand{\fincorrection}{\end{cor}}

\newcommand{\noindication}{\stepcounter{ind}}
\newcommand{\nocorrection}{\stepcounter{cor}}

\newcommand{\fiche}[1]{} \newcommand{\finfiche}{}
\newcommand{\titre}[1]{\centerline{\large \bf #1}}
\newcommand{\addcommand}[1]{}
\newcommand{\video}[1]{}

% Marge
\newcommand{\mymargin}[1]{\marginpar{{\small #1}}}

\def\noqed{\renewcommand{\qedsymbol}{}}


%----- Presentation ------
\setlength{\parindent}{0cm}

%\newcommand{\ExoSept}{\href{http://exo7.emath.fr}{\textbf{\textsf{Exo7}}}}

\definecolor{myred}{rgb}{0.93,0.26,0}
\definecolor{myorange}{rgb}{0.97,0.58,0}
\definecolor{myyellow}{rgb}{1,0.86,0}

\newcommand{\LogoExoSept}[1]{  % input : echelle
{\usefont{U}{cmss}{bx}{n}
\begin{tikzpicture}[scale=0.1*#1,transform shape]
  \fill[color=myorange] (0,0)--(4,0)--(4,-4)--(0,-4)--cycle;
  \fill[color=myred] (0,0)--(0,3)--(-3,3)--(-3,0)--cycle;
  \fill[color=myyellow] (4,0)--(7,4)--(3,7)--(0,3)--cycle;
  \node[scale=5] at (3.5,3.5) {Exo7};
\end{tikzpicture}}
}


\newcommand{\debutmontitre}{
  \author{} \date{} 
  \thispagestyle{empty}
  \hspace*{-10ex}
  \begin{minipage}{\textwidth}
    \titlepage  
  \vspace*{-2.5cm}
  \begin{center}
    \LogoExoSept{2.5}
  \end{center}
  \end{minipage}

  \vspace*{-0cm}
  
  % Astuce pour que le background ne soit pas discrétisé lors de la conversion pdf -> png
\begin{tikzpicture}
        \fill[opacity=0,green!60!black] (0,0)--++(0,0)--++(0,0)--++(0,0)--cycle; 
\end{tikzpicture}

% toc S'affiche trop tot :
% \tableofcontents[hideallsubsections, pausesections]
}

\newcommand{\finmontitre}{
  \end{frame}
  \setcounter{framenumber}{0}
} % ne marche pas pour une raison obscure

%----- Commandes supplementaires ------

% \usepackage[landscape]{geometry}
% \geometry{top=1cm, bottom=3cm, left=2cm, right=10cm, marginparsep=1cm
% }
% \usepackage[a4paper]{geometry}
% \geometry{top=2cm, bottom=2cm, left=2cm, right=2cm, marginparsep=1cm
% }

%\usepackage{standalone}


% New command Arnaud -- november 2011
\setbeamersize{text margin left=24ex}
% si vous modifier cette valeur il faut aussi
% modifier le decalage du titre pour compenser
% (ex : ici =+10ex, titre =-5ex

\theoremstyle{definition}
%\newtheorem{proposition}{Proposition}
%\newtheorem{exemple}{Exemple}
%\newtheorem{theoreme}{Théorème}
%\newtheorem{lemme}{Lemme}
%\newtheorem{corollaire}{Corollaire}
%\newtheorem*{remarque*}{Remarque}
%\newtheorem*{miniexercice}{Mini-exercices}
%\newtheorem{definition}{Définition}

% Commande tikz
\usetikzlibrary{calc}
\usetikzlibrary{patterns,arrows}
\usetikzlibrary{matrix}
\usetikzlibrary{fadings} 

%definition d'un terme
\newcommand{\defi}[1]{{\color{myorange}\textbf{\emph{#1}}}}
\newcommand{\evidence}[1]{{\color{blue}\textbf{\emph{#1}}}}
\newcommand{\assertion}[1]{\emph{\og#1\fg}}  % pour chapitre logique
%\renewcommand{\contentsname}{Sommaire}
\renewcommand{\contentsname}{}
\setcounter{tocdepth}{2}



%------ Figures ------

\def\myscale{1} % par défaut 
\newcommand{\myfigure}[2]{  % entrée : echelle, fichier figure
\def\myscale{#1}
\begin{center}
\footnotesize
{#2}
\end{center}}


%------ Encadrement ------

\usepackage{fancybox}


\newcommand{\mybox}[1]{
\setlength{\fboxsep}{7pt}
\begin{center}
\shadowbox{#1}
\end{center}}

\newcommand{\myboxinline}[1]{
\setlength{\fboxsep}{5pt}
\raisebox{-10pt}{
\shadowbox{#1}
}
}

%--------------- Commande beamer---------------
\newcommand{\beameronly}[1]{#1} % permet de mettre des pause dans beamer pas dans poly


\setbeamertemplate{navigation symbols}{}
\setbeamertemplate{footline}  % tiré du fichier beamerouterinfolines.sty
{
  \leavevmode%
  \hbox{%
  \begin{beamercolorbox}[wd=.333333\paperwidth,ht=2.25ex,dp=1ex,center]{author in head/foot}%
    % \usebeamerfont{author in head/foot}\insertshortauthor%~~(\insertshortinstitute)
    \usebeamerfont{section in head/foot}{\bf\insertshorttitle}
  \end{beamercolorbox}%
  \begin{beamercolorbox}[wd=.333333\paperwidth,ht=2.25ex,dp=1ex,center]{title in head/foot}%
    \usebeamerfont{section in head/foot}{\bf\insertsectionhead}
  \end{beamercolorbox}%
  \begin{beamercolorbox}[wd=.333333\paperwidth,ht=2.25ex,dp=1ex,right]{date in head/foot}%
    % \usebeamerfont{date in head/foot}\insertshortdate{}\hspace*{2em}
    \insertframenumber{} / \inserttotalframenumber\hspace*{2ex} 
  \end{beamercolorbox}}%
  \vskip0pt%
}


\definecolor{mygrey}{rgb}{0.5,0.5,0.5}
\setlength{\parindent}{0cm}
%\DeclareTextFontCommand{\helvetica}{\fontfamily{phv}\selectfont}

% background beamer
\definecolor{couleurhaut}{rgb}{0.85,0.9,1}  % creme
\definecolor{couleurmilieu}{rgb}{1,1,1}  % vert pale
\definecolor{couleurbas}{rgb}{0.85,0.9,1}  % blanc
\setbeamertemplate{background canvas}[vertical shading]%
[top=couleurhaut,middle=couleurmilieu,midpoint=0.4,bottom=couleurbas] 
%[top=fondtitre!05,bottom=fondtitre!60]



\makeatletter
\setbeamertemplate{theorem begin}
{%
  \begin{\inserttheoremblockenv}
  {%
    \inserttheoremheadfont
    \inserttheoremname
    \inserttheoremnumber
    \ifx\inserttheoremaddition\@empty\else\ (\inserttheoremaddition)\fi%
    \inserttheorempunctuation
  }%
}
\setbeamertemplate{theorem end}{\end{\inserttheoremblockenv}}

\newenvironment{theoreme}[1][]{%
   \setbeamercolor{block title}{fg=structure,bg=structure!40}
   \setbeamercolor{block body}{fg=black,bg=structure!10}
   \begin{block}{{\bf Th\'eor\`eme }#1}
}{%
   \end{block}%
}


\newenvironment{proposition}[1][]{%
   \setbeamercolor{block title}{fg=structure,bg=structure!40}
   \setbeamercolor{block body}{fg=black,bg=structure!10}
   \begin{block}{{\bf Proposition }#1}
}{%
   \end{block}%
}

\newenvironment{corollaire}[1][]{%
   \setbeamercolor{block title}{fg=structure,bg=structure!40}
   \setbeamercolor{block body}{fg=black,bg=structure!10}
   \begin{block}{{\bf Corollaire }#1}
}{%
   \end{block}%
}

\newenvironment{mydefinition}[1][]{%
   \setbeamercolor{block title}{fg=structure,bg=structure!40}
   \setbeamercolor{block body}{fg=black,bg=structure!10}
   \begin{block}{{\bf Définition} #1}
}{%
   \end{block}%
}

\newenvironment{lemme}[0]{%
   \setbeamercolor{block title}{fg=structure,bg=structure!40}
   \setbeamercolor{block body}{fg=black,bg=structure!10}
   \begin{block}{\bf Lemme}
}{%
   \end{block}%
}

\newenvironment{remarque}[1][]{%
   \setbeamercolor{block title}{fg=black,bg=structure!20}
   \setbeamercolor{block body}{fg=black,bg=structure!5}
   \begin{block}{Remarque #1}
}{%
   \end{block}%
}


\newenvironment{exemple}[1][]{%
   \setbeamercolor{block title}{fg=black,bg=structure!20}
   \setbeamercolor{block body}{fg=black,bg=structure!5}
   \begin{block}{{\bf Exemple }#1}
}{%
   \end{block}%
}


\newenvironment{miniexercice}[0]{%
   \setbeamercolor{block title}{fg=structure,bg=structure!20}
   \setbeamercolor{block body}{fg=black,bg=structure!5}
   \begin{block}{Mini-exercices}
}{%
   \end{block}%
}


\newenvironment{tp}[0]{%
   \setbeamercolor{block title}{fg=structure,bg=structure!40}
   \setbeamercolor{block body}{fg=black,bg=structure!10}
   \begin{block}{\bf Travaux pratiques}
}{%
   \end{block}%
}
\newenvironment{exercicecours}[1][]{%
   \setbeamercolor{block title}{fg=structure,bg=structure!40}
   \setbeamercolor{block body}{fg=black,bg=structure!10}
   \begin{block}{{\bf Exercice }#1}
}{%
   \end{block}%
}
\newenvironment{algo}[1][]{%
   \setbeamercolor{block title}{fg=structure,bg=structure!40}
   \setbeamercolor{block body}{fg=black,bg=structure!10}
   \begin{block}{{\bf Algorithme}\hfill{\color{gray}\texttt{#1}}}
}{%
   \end{block}%
}


\setbeamertemplate{proof begin}{
   \setbeamercolor{block title}{fg=black,bg=structure!20}
   \setbeamercolor{block body}{fg=black,bg=structure!5}
   \begin{block}{{\footnotesize Démonstration}}
   \footnotesize
   \smallskip}
\setbeamertemplate{proof end}{%
   \end{block}}
\setbeamertemplate{qed symbol}{\openbox}


\makeatother
\usecolortheme[RGB={151,53,151}]{structure}

%%%%%%%%%%%%%%%%%%%%%%%%%%%%%%%%%%%%%%%%%%%%%%%%%%%%%%%%%%%%%
%%%%%%%%%%%%%%%%%%%%%%%%%%%%%%%%%%%%%%%%%%%%%%%%%%%%%%%%%%%%%

\title{{\bf Logique et raisonnements}}
\subtitle{Logique}
\author{}

\date{}

\begin{document}


\begin{frame}
  
  \debutmontitre

  \pause

{\footnotesize
\hfill
\setbeamercovered{transparent=50}
\begin{minipage}{0.6\textwidth}
  \begin{itemize}
    \item<3-> Assertions
    \item<4-> Quantificateurs
  \end{itemize}
\end{minipage}
}

\end{frame}

\setcounter{framenumber}{0}




%%%%%%%%%%%%%%%%%%%%%%%%%%%%%%%%%%%%%%%%%%%%%%%%%%%%%%%%%%%%%%%%
\section{Motivation}

\begin{frame}

\begin{itemize}
  \item Ambigu\"ité de la langue française
     \begin{itemize}
        \item \assertion{fromage ou dessert}
        \item \assertion{un as ou un c\oe ur}
 
\pause

        \item \evidence{langage rigoureux}
     \end{itemize}

        
\pause      
\bigskip

  \item Exprimer des choses complexes
     \begin{itemize}
        \item \assertion{on trace le graphe sans lever le crayon}

\pause

        \item $f : I \to \Rr$ est continue en un point $x_0\in I$
$$\forall \epsilon > 0 \quad \exists \delta >0 \quad \forall x \in I \quad 
  (|x-x_0|<\delta \implies |f(x)-f(x_0)|< \epsilon)$$

\pause

        \item langage de la \evidence{logique}
     \end{itemize}

\pause
\bigskip

  \item Distinguer le vrai du faux
     \begin{itemize}
        \item \assertion{Est-ce qu'une augmentation de $20\%$, puis de $30\%$ est 
plus intéressante qu'une augmentation de $50\%$ ?}

\pause

        \item démarche logique qui mène à la conclusion
        \item vous convaincre vous-même et convaincre les autres

\pause

        \item \evidence{raisonnement}
     \end{itemize}
\end{itemize}

\end{frame}




%%%%%%%%%%%%%%%%%%%%%%%%%%%%%%%%%%%%%%%%%%%%%%%%%%%%%%%%%%%%%%%%
%\section{Logique}

%---------------------------------------------------------------
\section{Assertions}

\begin{frame}
Une \defi{assertion} est une phrase soit vraie, soit fausse, pas les deux en même temps

\pause
\bigskip
\begin{itemize}
  \item<2-> \assertion{Il pleut.}
  \item<2-> \assertion{Je suis plus grand que toi.}
  \item<3-> \assertion{$2+2=4$}
  \item<3-> \assertion{$2\times 3 = 7$}
  \item<4-> \assertion{Pour tout $x \in \Rr$, on a $x^2 \ge 0$.}
  \item<4-> \assertion{Pour tout $z\in \Cc$, on a $|z| = 1$.}
\end{itemize}
\end{frame}





%--------------
\section{\og et\fg}


\begin{frame}
L'assertion \assertion{$P$ \defi{et} $Q$} est vraie si $P$ est vraie et $Q$ est vraie,

l'assertion \assertion{$P$ et $Q$} est fausse sinon

\pause

\bigskip

Table de vérité 
\begin{figure}[H]
\centering
\begin{tabular}{c|c|c}
\textcolor{olive}{$P$} $\backslash$ \textcolor{blue}{$Q$}  & \textcolor{blue}{V} & \textcolor{blue}{F} \\ \hline
\textcolor{olive}{V} & \uncover<3->{\textcolor{red}{V}} & \uncover<4->{\textcolor{red}{F}} \\ \hline
\textcolor{olive}{F} & \uncover<5->{\textcolor{red}{F}} & \uncover<6->{\textcolor{red}{F}} \\ 
\end{tabular}
\end{figure}

\pause\pause\pause\pause\pause

Exemple 
\begin{itemize}
  \item $P$ est l'assertion \assertion{Cette carte est un as}
  \item $Q$ l'assertion \assertion{Cette carte est c\oe ur}
  \item \assertion{$P$ et $Q$} est vraie si la carte est l'as de c\oe ur 
  \item  \assertion{$P$ et $Q$} est fausse pour toute autre carte
\end{itemize}

\end{frame}






%--------------
\section{\og ou\fg}


\begin{frame}
L'assertion \assertion{$P$ \defi{ou} $Q$} est vraie si l'une au moins des deux assertions $P$ ou $Q$ est vraie

L'assertion \assertion{$P$ ou $Q$} est fausse si les deux assertions $P$ et $Q$ sont fausses


\pause

\bigskip

\begin{figure}[H]
\centering
\begin{tabular}{c|c|c}
\textcolor{olive}{$P$} $\backslash$ \textcolor{blue}{$Q$}  & \textcolor{blue}{V} & \textcolor{blue}{F} \\ \hline
\textcolor{olive}{V} & \uncover<3->{\textcolor{red}{V}} & \uncover<4->{\textcolor{red}{V}} \\ \hline
\textcolor{olive}{F} & \uncover<5->{\textcolor{red}{V}} & \uncover<6->{\textcolor{red}{F}} \\ 
\end{tabular}
\end{figure}

\pause\pause\pause\pause\pause

\begin{itemize}
  \item \assertion{$P$ ou $Q$} est vraie si la carte est un as ou bien un c\oe ur
  \item  \assertion{$P$ ou $Q$} est fausse pour toute autre carte
\end{itemize}

\end{frame}





%--------------
\section{La négation \og non\fg}

\begin{frame}
L'assertion \assertion{\defi{non} $P$} est vraie si $P$ est fausse, et fausse si $P$ est vraie

\bigskip

\begin{figure}[H]
\centering
\begin{tabular}{c|c|c}
 \textcolor{blue}{$P$}  &  \textcolor{blue}{V} &  \textcolor{blue}{F} \\ \hline
 \textcolor{red}{non $P$}    & \textcolor{red}{F} & \textcolor{red}{V} \\  
\end{tabular}
\end{figure}
\end{frame}




%--------------
\section{L'implication $\implies$}

\begin{frame}
L'\defi{implication}
\mybox{
L'assertion \assertion{(non $P$) ou $Q$} est notée \assertion{$P \implies Q$}
}

\pause
\bigskip

\begin{figure}[H]
\centering
\begin{tabular}{c|c|c}
\textcolor{olive}{$P$} $\backslash$ \textcolor{blue}{$Q$}  & \textcolor{blue}{V} & \textcolor{blue}{F} \\ \hline
\textcolor{olive}{V} & \uncover<3->{\textcolor{red}{V}} & \uncover<4->{\textcolor{red}{F}} \\ \hline
\textcolor{olive}{F} & \uncover<5->{\textcolor{red}{V}} & \uncover<5->{\textcolor{red}{V}} \\ 
\end{tabular}
\end{figure}

\pause\pause\pause\pause
\bigskip
 
\begin{itemize}
  \item \assertion{$0 \le x \le 25 \implies \sqrt x \le 5$}  est vraie  

\pause

  \item \assertion{$x \in ]-\infty, -4[ \implies x^2+3x-4 > 0$} est vraie 

\pause

  \item \assertion{$\sin(\theta)=0 \implies \theta = 0$} est fausse 


\pause

  \item si $P$ est fausse alors l'assertion \assertion{$P \implies Q$} est toujours vraie
\end{itemize} 
\end{frame}





%--------------
\section{L'équivalence $\iff$}

\begin{frame}
L'\defi{équivalence} 
\mybox{
\assertion{$P \iff Q$} est l'assertion \assertion{($P \implies Q$) \  et \  ($Q \implies P$)}
}

\pause
\bigskip

\begin{figure}[H]
\centering
\begin{tabular}{c|c|c}
\textcolor{olive}{$P$} $\backslash$ \textcolor{blue}{$Q$}  & \textcolor{blue}{V} & \textcolor{blue}{F} \\ \hline
\textcolor{olive}{V} & \textcolor{red}{V} & \textcolor{red}{F} \\ \hline
\textcolor{olive}{F} & \textcolor{red}{F} & \textcolor{red}{V} \\ 
\end{tabular}
\end{figure}


\pause
\bigskip

Pour $x,x' \in \Rr$, l'assertion est vraie :

\centerline{\assertion{$x\cdot x'=0 \iff (x=0 \text{ ou } x'=0)$}} 

\end{frame}


\begin{frame}
\begin{proposition}
\label{prop:log}
\begin{enumerate}
  \item<1-> $P \iff \text{ non}(\text{non}(P))$
  \item<2-> $(P \text{ et } Q) \iff (Q \text{ et } P)$
  \item<3-> $(P \text{ ou } Q) \iff (Q \text{ ou } P)$
  \item<4-> $\text{non}(P \text{ et } Q)  \iff  (\text{non } P)  \text{ ou } (\text{non }Q)$
  \item<5-> $\text{non}(P \text{ ou } Q)  \iff  (\text{non } P)  \text{ et } (\text{non }Q)$
  \item<6-> $\big(P \text{ et } (Q \text{ ou } R)  \big)   \iff 
(P \text{ et } Q) \text{ ou } (P \text{ et }  R)$
  \item<7-> $\big(P \text{ ou } (Q \text{ et } R)  \big)   \iff 
(P \text{ ou } Q) \text{ et } (P \text{ ou }  R)$
  \item<8->  \assertion{$P \implies Q$} $\iff$ \assertion{$\text{non}(Q) \implies \text{non}(P)$}
\end{enumerate}
\end{proposition}
\end{frame}



\begin{frame} 
\begin{proof}

Table de vérité de  \assertion{$\text{non}(P\text{ et }Q)$} 
\begin{figure}[H]
\centering
\begin{tabular}{c|c|c}
\textcolor{olive}{$P$} $\backslash$ \textcolor{blue}{$Q$}  & \textcolor{blue}{V} & \textcolor{blue}{F} \\ \hline
\textcolor{olive}{V} & \uncover<2->{\textcolor{red}{F}} & \uncover<3->{\textcolor{red}{V}} \\ \hline
\textcolor{olive}{F} & \uncover<4->{\textcolor{red}{V}} & \uncover<4->{\textcolor{red}{V}} \\ 
\end{tabular}
\end{figure}
\pause
\pause\pause\pause
\bigskip

Table de vérité \assertion{$(\text{non } P)\text{ ou }(\text{non }Q)$} 
\begin{figure}[H]
\centering
\begin{tabular}{c|c|c}
\textcolor{olive}{$P$} $\backslash$ \textcolor{blue}{$Q$}  & \textcolor{blue}{V} & \textcolor{blue}{F} \\ \hline
\textcolor{olive}{V} & \uncover<6->{\textcolor{red}{F}} & \uncover<7->{\textcolor{red}{V}} \\ \hline
\textcolor{olive}{F} & \uncover<8->{\textcolor{red}{V}} & \uncover<8->{\textcolor{red}{V}} \\ 
\end{tabular}
\end{figure}
\pause\pause\pause
\end{proof}
 
\end{frame} 






%--------------
\section{$\forall$ : \og pour tout\fg}


\begin{frame}
Une assertion $P$ peut dépendre d'un paramètre $x$\\[0.5em]

Exemple : \assertion{$x^2 \ge 1$}, \  
$P(x)$ est vraie ou fausse selon la valeur de $x$

\pause
\bigskip

L'assertion $$\forall x \in E \quad P(x)$$
est vraie lorsque les assertions $P(x)$ sont vraies pour tous les éléments $x$
de l'ensemble $E$

\pause
\bigskip

\begin{itemize}
  \item \assertion{$\forall x \in [1,+\infty[ \quad (x^2\ge 1)$} \  est une assertion vraie
\pause

  \item \assertion{$\forall x \in \Rr \quad (x^2\ge 1)$} \  est une assertion fausse
\pause

  \item \assertion{$\forall n \in \Nn \quad n(n+1) \text{ est divisible par } 2$} \ est vraie
\end{itemize} 

\end{frame}



%--------------
\section{$\exists$ : \og il existe\fg}

\begin{frame}
L'assertion 
$$\exists x \in E \quad P(x)$$
est une assertion vraie lorsque l'on peut trouver au moins un $x$ de $E$ pour lequel $P(x)$ est vraie

\pause
\bigskip

\begin{itemize}
  \item \assertion{$\exists x \in \Rr \quad (x(x-1)<0)$} \  est vraie
  \item \assertion{$\exists n \in \Nn \quad n^2-n > n$} \  est vraie
  \item \assertion{$\exists x \in \Rr \quad (x^2=-1)$} \ est fausse
\end{itemize} 


\end{frame}


%--------------
\section{Négation des quantificateurs}

\begin{frame}

\mybox{
La négation de \assertion{$\forall x \in E \quad P(x)$} \ est \  \assertion{$\exists x \in E \quad \text{non } P(x)$}
}

\pause
\bigskip

Exemple : la négation de \assertion{$\forall x \in [1,+\infty[ \quad (x^2\ge 1)$}
est l'assertion \assertion{$\exists x \in [1,+\infty[ \quad (x^2 < 1)$}

\pause
\bigskip
\mybox{
La négation de \assertion{$\exists x \in E \quad P(x)$} \ est \ \assertion{$\forall x \in E \quad \text{non } P(x)$}
}

\pause
\bigskip

{\small
\begin{itemize}
  \item 
$\exists z \in \Cc \quad (z^2+z+1 = 0)$

$\forall z \in \Cc \quad (z^2+z+1 \neq 0)$

\pause

  \item 
$\forall x \in \Rr \quad  (x+1 \in \Zz)$

$\exists x \in \Rr \quad (x+1 \notin \Zz)$

\pause

  \item 

$\forall x \in \Rr \quad \exists y >0 \quad (x+y > 10)$ 

$\exists x \in \Rr \quad \forall y > 0 \quad (x+y \le 10)$
\end{itemize} 
}

\end{frame}



%--------------
\section{Remarques}

\begin{frame}

Remarques
\begin{itemize}
  \item<1-> 
 L'\evidence{ordre des quantificateurs} est très important

\qquad \assertion{$\forall x \in \Rr \quad  \exists y \in \Rr \quad (x+y>0)$} \quad est vraie

\qquad \assertion{$\exists y \in \Rr \quad \forall x \in \Rr \quad (x+y>0)$} \quad est fausse

  \item<2-> \assertion{$\exists x \in \Rr \quad (f(x)=0)$} : il existe au moins un réel
pour lequel $f$ s'annule

\assertion{$\exists \alert{\mathbf{!}}\, x \in \Rr \quad (f(x)=0)$} : \defi{il existe un unique} 
réel pour lequel $f$ s'annule


  \item<3-> la négation de l'inégalité stricte \assertion{$<$} 
est l'inégalité large \assertion{$\ge$}

  \item<4-> Les quantificateurs ne sont pas des abréviations 
  \begin{itemize}
     \item \assertion{Pour tout réel $x$, si $f(x)=1$ alors $x \ge 0$.}
     \item \assertion{$\forall x \in \Rr \quad (f(x)=1 \implies x \ge 0)$}
     \item N'écrivez pas \assertion{$\forall x$ réel, si $f(x)=1 \implies x$ positif ou nul}
  \end{itemize}


  \item<5-> Ces symboles n'existent pas $\not\!\exists$, $\not\!\!\!\implies$

\end{itemize} 
\end{frame}




%---------------------------------------------------------------
\section{Mini-exercices}

\begin{frame}

\begin{miniexercice}
\begin{enumerate}
  \item \'Ecrire la table de vérité du \assertion{ou exclusif}. (C'est le \emph{ou} dans la phrase
\assertion{fromage ou dessert}, l'un ou l'autre mais pas les deux.) 
  \item \'Ecrire la table de vérité de \assertion{non ($P$ et $Q$)}. Que remarquez-vous ?
  \item \'Ecrire la négation de \assertion{$P \implies Q$}.
  \item Démontrer les assertions restantes de la proposition, page 8.
  \item \'Ecrire la négation de \assertion{$\big(P \text{ et } (Q \text{ ou } R)  \big)$}.
  \item \'Ecrire à l'aide des quantificateurs la phrase suivante :
\assertion{Pour tout nombre réel, son carré est positif}. Puis écrire la négation.
  \item Mêmes questions avec les phrases : \assertion{Pour chaque réel, je peux trouver un entier
relatif tel que leur produit soit strictement plus grand que $1$}. Puis
\assertion{Pour tout entier $n$, il existe un unique réel $x$ tel que $\exp(x)$ égale $n$}.
\end{enumerate}  
\end{miniexercice}

\end{frame}

\end{document}