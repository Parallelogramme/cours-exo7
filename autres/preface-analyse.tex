
\pagestyle{empty}\thispagestyle{empty}
\vspace*{\fill}
\begin{center}
\fontsize{52}{52}\selectfont
\textsc{ANALYSE}
\end{center}
\vfill
\begin{center}
\huge
\textsc{Cours de mathématiques}\\ 
\textsc{Première année}
\end{center}

\begin{center}
\LogoExoSept{3}
\end{center}
\clearemptydoublepage
\thispagestyle{empty}

\vspace*{\fill}
\section*{À la découverte de l'analyse}
 
 
Les mathématiques, vous les avez bien sûr manipulées au lycée. Dans le supérieur, il s'agit d'apprendre à les construire ! La première année pose les bases et introduit les outils dont vous aurez besoin par la suite. Elle est aussi l'occasion de découvrir la beauté des mathématiques, de l'infiniment grand (les limites) à l'infiniment petit (le calcul de dérivée).


L'outil central abordé dans ce tome d'analyse, ce sont les fonctions. Vous en connaissez déjà beaucoup, racine carrée, sinus et cosinus, logarithme, exponentielle... Elles interviennent dès que l'on s'intéresse à des phénomènes qui varient en fonction de certains paramètres. Position d'une comète en fonction du temps, variation du volume d'un gaz en fonction de la température et de la pression, nombre de bactéries en fonction de la nourriture disponible : physique, chimie, biologie ou encore économie, autant de domaines dans lesquels le formalisme mathématique s'applique et permet de résoudre des problèmes.

\medskip

Ce tome débute par l'étude des nombres réels, puis des suites. Les chapitres suivants sont consacrés aux fonctions : limite, continuité, dérivabilité sont des notions essentielles, qui reposent sur des définitions et des preuves minutieuses. Toutes ces notions ont une interprétation géométrique, qu'on lit sur le graphe de la fonction, et c'est pourquoi vous trouverez dans ce livre de nombreux dessins pour vous aider à comprendre l'intuition cachée derrière les énoncés. En fin de volume, deux chapitres explorent les applications des études de fonctions au tracé de courbes paramétrées et à la résolution d'équations différentielles.

\medskip


Les efforts que vous devrez fournir sont importants: tout d'abord comprendre le cours, ensuite connaître par c\oe ur les définitions, les théorèmes, les propositions... sans oublier de travailler les exemples et les démonstrations, qui permettent de bien assimiler les notions nouvelles et les mécanismes de raisonnement. Enfin, vous devrez passer autant de temps à pratiquer les mathématiques: il est  indispensable de résoudre activement par vous-même des exercices, sans regarder les solutions! Pour vous aider, vous trouverez sur le site Exo7 toutes les vidéos correspondant à ce cours, ainsi que des exercices corrigés. 


Alors n'hésitez plus : manipulez, calculez, raisonnez, et dessinez, à vous de jouer !


\vspace*{\fill}

\newpage
\addtocontents{toc}{\protect\setcounter{tocdepth}{1}}
\tableofcontents

