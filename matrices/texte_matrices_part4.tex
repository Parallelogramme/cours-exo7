
%%%%%%%%%%%%%%%%%% PREAMBULE %%%%%%%%%%%%%%%%%%


\documentclass[12pt]{article}

\usepackage{amsfonts,amsmath,amssymb,amsthm}
\usepackage[utf8]{inputenc}
\usepackage[T1]{fontenc}
\usepackage[francais]{babel}


% packages
\usepackage{amsfonts,amsmath,amssymb,amsthm}
\usepackage[utf8]{inputenc}
\usepackage[T1]{fontenc}
%\usepackage{lmodern}

\usepackage[francais]{babel}
\usepackage{fancybox}
\usepackage{graphicx}

\usepackage{float}

%\usepackage[usenames, x11names]{xcolor}
\usepackage{tikz}
\usepackage{datetime}

\usepackage{mathptmx}
%\usepackage{fouriernc}
%\usepackage{newcent}
\usepackage[mathcal,mathbf]{euler}

%\usepackage{palatino}
%\usepackage{newcent}


% Commande spéciale prompteur

%\usepackage{mathptmx}
%\usepackage[mathcal,mathbf]{euler}
%\usepackage{mathpple,multido}

\usepackage[a4paper]{geometry}
\geometry{top=2cm, bottom=2cm, left=1cm, right=1cm, marginparsep=1cm}

\newcommand{\change}{{\color{red}\rule{\textwidth}{1mm}\\}}

\newcounter{mydiapo}

\newcommand{\diapo}{\newpage
\hfill {\normalsize  Diapo \themydiapo \quad \texttt{[\jobname]}} \\
\stepcounter{mydiapo}}


%%%%%%% COULEURS %%%%%%%%%%

% Pour blanc sur noir :
%\pagecolor[rgb]{0.5,0.5,0.5}
% \pagecolor[rgb]{0,0,0}
% \color[rgb]{1,1,1}



%\DeclareFixedFont{\myfont}{U}{cmss}{bx}{n}{18pt}
\newcommand{\debuttexte}{
%%%%%%%%%%%%% FONTES %%%%%%%%%%%%%
\renewcommand{\baselinestretch}{1.5}
\usefont{U}{cmss}{bx}{n}
\bfseries

% Taille normale : commenter le reste !
%Taille Arnaud
%\fontsize{19}{19}\selectfont

% Taille Barbara
%\fontsize{21}{22}\selectfont

%Taille François
\fontsize{25}{30}\selectfont

%Taille Pascal
%\fontsize{25}{30}\selectfont

%Taille Laura
%\fontsize{30}{35}\selectfont


%\myfont
%\usefont{U}{cmss}{bx}{n}

%\Huge
%\addtolength{\parskip}{\baselineskip}
}


% \usepackage{hyperref}
% \hypersetup{colorlinks=true, linkcolor=blue, urlcolor=blue,
% pdftitle={Exo7 - Exercices de mathématiques}, pdfauthor={Exo7}}


%section
% \usepackage{sectsty}
% \allsectionsfont{\bf}
%\sectionfont{\color{Tomato3}\upshape\selectfont}
%\subsectionfont{\color{Tomato4}\upshape\selectfont}

%----- Ensembles : entiers, reels, complexes -----
\newcommand{\Nn}{\mathbb{N}} \newcommand{\N}{\mathbb{N}}
\newcommand{\Zz}{\mathbb{Z}} \newcommand{\Z}{\mathbb{Z}}
\newcommand{\Qq}{\mathbb{Q}} \newcommand{\Q}{\mathbb{Q}}
\newcommand{\Rr}{\mathbb{R}} \newcommand{\R}{\mathbb{R}}
\newcommand{\Cc}{\mathbb{C}} 
\newcommand{\Kk}{\mathbb{K}} \newcommand{\K}{\mathbb{K}}

%----- Modifications de symboles -----
\renewcommand{\epsilon}{\varepsilon}
\renewcommand{\Re}{\mathop{\text{Re}}\nolimits}
\renewcommand{\Im}{\mathop{\text{Im}}\nolimits}
%\newcommand{\llbracket}{\left[\kern-0.15em\left[}
%\newcommand{\rrbracket}{\right]\kern-0.15em\right]}

\renewcommand{\ge}{\geqslant}
\renewcommand{\geq}{\geqslant}
\renewcommand{\le}{\leqslant}
\renewcommand{\leq}{\leqslant}

%----- Fonctions usuelles -----
\newcommand{\ch}{\mathop{\mathrm{ch}}\nolimits}
\newcommand{\sh}{\mathop{\mathrm{sh}}\nolimits}
\renewcommand{\tanh}{\mathop{\mathrm{th}}\nolimits}
\newcommand{\cotan}{\mathop{\mathrm{cotan}}\nolimits}
\newcommand{\Arcsin}{\mathop{\mathrm{Arcsin}}\nolimits}
\newcommand{\Arccos}{\mathop{\mathrm{Arccos}}\nolimits}
\newcommand{\Arctan}{\mathop{\mathrm{Arctan}}\nolimits}
\newcommand{\Argsh}{\mathop{\mathrm{Argsh}}\nolimits}
\newcommand{\Argch}{\mathop{\mathrm{Argch}}\nolimits}
\newcommand{\Argth}{\mathop{\mathrm{Argth}}\nolimits}
\newcommand{\pgcd}{\mathop{\mathrm{pgcd}}\nolimits} 

\newcommand{\Card}{\mathop{\text{Card}}\nolimits}
\newcommand{\Ker}{\mathop{\text{Ker}}\nolimits}
\newcommand{\id}{\mathop{\text{id}}\nolimits}
\newcommand{\ii}{\mathrm{i}}
\newcommand{\dd}{\mathrm{d}}
\newcommand{\Vect}{\mathop{\text{Vect}}\nolimits}
\newcommand{\Mat}{\mathop{\mathrm{Mat}}\nolimits}
\newcommand{\rg}{\mathop{\text{rg}}\nolimits}
\newcommand{\tr}{\mathop{\text{tr}}\nolimits}
\newcommand{\ppcm}{\mathop{\text{ppcm}}\nolimits}

%----- Structure des exercices ------

\newtheoremstyle{styleexo}% name
{2ex}% Space above
{3ex}% Space below
{}% Body font
{}% Indent amount 1
{\bfseries} % Theorem head font
{}% Punctuation after theorem head
{\newline}% Space after theorem head 2
{}% Theorem head spec (can be left empty, meaning ‘normal’)

%\theoremstyle{styleexo}
\newtheorem{exo}{Exercice}
\newtheorem{ind}{Indications}
\newtheorem{cor}{Correction}


\newcommand{\exercice}[1]{} \newcommand{\finexercice}{}
%\newcommand{\exercice}[1]{{\tiny\texttt{#1}}\vspace{-2ex}} % pour afficher le numero absolu, l'auteur...
\newcommand{\enonce}{\begin{exo}} \newcommand{\finenonce}{\end{exo}}
\newcommand{\indication}{\begin{ind}} \newcommand{\finindication}{\end{ind}}
\newcommand{\correction}{\begin{cor}} \newcommand{\fincorrection}{\end{cor}}

\newcommand{\noindication}{\stepcounter{ind}}
\newcommand{\nocorrection}{\stepcounter{cor}}

\newcommand{\fiche}[1]{} \newcommand{\finfiche}{}
\newcommand{\titre}[1]{\centerline{\large \bf #1}}
\newcommand{\addcommand}[1]{}
\newcommand{\video}[1]{}

% Marge
\newcommand{\mymargin}[1]{\marginpar{{\small #1}}}



%----- Presentation ------
\setlength{\parindent}{0cm}

%\newcommand{\ExoSept}{\href{http://exo7.emath.fr}{\textbf{\textsf{Exo7}}}}

\definecolor{myred}{rgb}{0.93,0.26,0}
\definecolor{myorange}{rgb}{0.97,0.58,0}
\definecolor{myyellow}{rgb}{1,0.86,0}

\newcommand{\LogoExoSept}[1]{  % input : echelle
{\usefont{U}{cmss}{bx}{n}
\begin{tikzpicture}[scale=0.1*#1,transform shape]
  \fill[color=myorange] (0,0)--(4,0)--(4,-4)--(0,-4)--cycle;
  \fill[color=myred] (0,0)--(0,3)--(-3,3)--(-3,0)--cycle;
  \fill[color=myyellow] (4,0)--(7,4)--(3,7)--(0,3)--cycle;
  \node[scale=5] at (3.5,3.5) {Exo7};
\end{tikzpicture}}
}



\theoremstyle{definition}
%\newtheorem{proposition}{Proposition}
%\newtheorem{exemple}{Exemple}
%\newtheorem{theoreme}{Théorème}
\newtheorem{lemme}{Lemme}
\newtheorem{corollaire}{Corollaire}
%\newtheorem*{remarque*}{Remarque}
%\newtheorem*{miniexercice}{Mini-exercices}
%\newtheorem{definition}{Définition}




%definition d'un terme
\newcommand{\defi}[1]{{\color{myorange}\textbf{\emph{#1}}}}
\newcommand{\evidence}[1]{{\color{blue}\textbf{\emph{#1}}}}



 %----- Commandes divers ------

\newcommand{\codeinline}[1]{\texttt{#1}}

%%%%%%%%%%%%%%%%%%%%%%%%%%%%%%%%%%%%%%%%%%%%%%%%%%%%%%%%%%%%%
%%%%%%%%%%%%%%%%%%%%%%%%%%%%%%%%%%%%%%%%%%%%%%%%%%%%%%%%%%%%%



\begin{document}

\debuttexte


%%%%%%%%%%%%%%%%%%%%%%%%%%%%%%%%%%%%%%%%%%%%%%%%%%%%%%%%%%%
\diapo

\change
Dans ce chapitre nous allons voir une méthode pour calculer l'inverse 
d'une matrice quelconque de manière efficace. 
Cette méthode est une reformulation de la méthode du pivot de Gauss pour les systèmes linéaires.

\change
Mais tout d'abord, nous commencerons par une formule directe dans le cas simple des matrices $2\times 2$.

\change
Puis nous verrons la méthode de Gauss dans le cadre général.

\change
Enfin, nous étudierons en détail un exemple de détermination d'une inverse.


%%%%%%%%%%%%%%%%%%%%%%%%%%%%%%%%%%%%%%%%%%%%%%%%%%%%%%%%%%
\diapo
Commençons par le cas particulier des matrices $2\times 2$. Considérons la matrice $2\times 2$ :
$A = \begin{pmatrix}
 a & b\\
 c & d       
     \end{pmatrix}.
$ 

\change
On a alors la proposition suivante.

Si $ad - bc \not= 0$,  alors, d'une part, la matrice $A$ est inversible et, d'autre part, on connaît son inverse :
$A^{-1} = \frac{1}{ad-bc} \begin{pmatrix}
d & -b\\
 -c & a
\end{pmatrix}$
 
\change
La démonstration est directe. 
On appelle $B=\frac{1}{ad-bc}  \left(\begin{smallmatrix}
d & -b\\
 -c & a
\end{smallmatrix}\right)$, qui est notre candidat pour l'inverse de $A$. 

\change 
On vérifie alors facilement que le produit $AB$ est égal à la matrice identité d'ordre $2$,

\change
de même que le produit $BA$. 

Ma matrice $B$ vérifie $AB=BA=I$ : la matrice $A$ est donc inversible, et son inverse est $B$.

%%%%%%%%%%%%%%%%%%%%%%%%%%%%%%%%%%%%%%%%%%%%%%%%%%%%%%%%%%%
\diapo

Plaçons nous à présent dans le cas général.

\change
La méthode pour inverser une matrice quelconque $A$ consiste à faire des opérations élémentaires sur les lignes de la matrice $A$ jusqu'à la transformer en la matrice identité $I$. Nous allons voir dans un instant en quoi consistent ces opérations élémentaires sur les lignes.

\change
On fait *simultanément* les mêmes opérations élémentaires en partant de la matrice identité $I$. 

\change
A la fin, lorsque la matrice $A$ est transformée en $I$, alors la matrice $I$ initiale est transformée en $A^{-1}$. 
%La démonstration sera vue dans la section suivante.

\change
En pratique, on fait les deux opérations en même temps sur $A$ et sur $I$ en adoptant la disposition suivante : à côté de la matrice $A$ que l'on veut inverser, on ajoute la matrice identité pour former un tableau $(A\ |\ I)$. 

\change
Sur les lignes de cette matrice augmentée, on effectue des opérations élémentaires jusqu'à obtenir le tableau $(I\ |\ B)$. 

\change
Et alors $B=A^{-1}$.

\change
Ces opérations élémentaires sur les lignes sont les suivantes:

\change
on peut remplacer une ligne $L_i$ par $\lambda L_i$ où $\lambda\neq 0$, c'est-à-dire qu'on peut multiplier une ligne par un réel non nul,
    
\change
on peut aussi ajouter à la ligne $L_i$ un multiple d'une autre ligne $L_j$.
  
\change
enfin on peut échanger deux lignes $L_i$ et $L_j$.


N'oubliez pas : tout ce que vous faites sur la partie gauche de la matrice augmentée, vous devez aussi le faire sur la partie droite.


%%%%%%%%%%%%%%%%%%%%%%%%%%%%%%%%%%%%%%%%%%%%%%%%%%%%%%%%%%
\diapo

Passons tout de suite à un exemple, et calculons l'inverse de la matrice $A$ suivante.

\change
Voici la matrice augmentée, ici la matrice $A$ que l'on souhaite inverser
et ici la matrice identité et aussi les lignes numérotées.

Il s'agit donc de transformer de ce coté, la matrice $A$ en l'identité, 
par des opérations élémentaires sur les lignes. 
On commence par la première colonne. On a déjà un $1$ ici. 
On applique donc la méthode de Gauss pour faire apparaître des $0$ en dessous.

\change
On s'occupe d'abord de la deuxième ligne. On conserve la première ligne 
et la troisième ligne.

\change
Pour faire apparaître un $0$ en début de deuxième ligne on fait 
l'opération élémentaire
$L_2 \leftarrow L_2 - 4 L_1$ 


\change
Sur la partie gauche on obtient 

$4-4\times1=0$ (c'est ce que l'on voulait), 

$0-4\times2 =-8$, 

$-1-4\times1 =-5$, 

\change
Et on fait la même opération sur la partie droite $L_2 \leftarrow L_2 - 4 L_1$ 
ce qui donne 

$0-4\times1 =-4$, 

$1-4\times0 =1$, 

et $0-4\times 0=0$.

\change
On souhaite à présent faire apparaître un $0$ sur la première colonne, à la troisième ligne, 
on conserve telles quelles nos deux premières lignes.

\change
Et on remplace $L_3$ par $L_3 + L_1$.

\change
Ce qui donne a gauche $0,4,3$ 

\change
et à droite $1 0 1$.

Ce n'est bien sûr pas encore fini.

%%%%%%%%%%%%%%%%%%%%%%%%%%%%%%%%%%%%%%%%%%%%%%%%%%%%%%%%%%%
\diapo

Voilà où nous nous en sommes arrêté. Passons à présent à la deuxième colonne.
On veut un $1$ ici et des zéros ici et là.

\change
On multiplie la ligne $L_2$ afin qu'elle commence par $1$, c'est-à-dire qu'on la multiplie par $-1/8$

\change
et on obtient donc ceci.

\change
On continue afin de faire apparaître un $0$ sous le $1$ que l'on vient d'obtenir.
Par la substitution $L_3 \leftarrow L_3 -4L_2$.

\change
On obtient à gauche $0, 0, \frac12$.


\change
et on multiplie la ligne $L_3$ par $2$ pour obtenir un $1$ en bas à droite.

\change
Ce qui termine la première partie de la méthode de Gauss.


%%%%%%%%%%%%%%%%%%%%%%%%%%%%%%%%%%%%%%%%%%%%%%%%%%%%%%%%%%%
\diapo

A l'étape précédente, nous avons obtenu cette matrice, c'est-à-dire qu'à gauche de notre matrice augmentée, nous avons à présent des $1$ sur la diagonale et des $0$ en dessous. Il ne reste plus qu'à "remonter" pour faire apparaître des zéros au-dessus de la diagonale.

\change
On fait apparaître un $0$ ici en remplaçant $L_2$ par $L_2-\frac{5}{8} L_3$

\change
et enfin, on fait simultanément apparaître des $0$ sur la première ligne en remplaçant $L_1$ par $L_1-2L_2-L_3$.

On a juste combiné deux opérations élémentaires.

C'est terminé ! 
On a bien obtenu à gauche la matrice identité. Ainsi l'inverse de $A$ est la matrice obtenue à droite,

\change
et après avoir factorisé tous les coefficients par $\frac14$, on a obtenu ceci.

\change
Pour se rassurer sur ses calculs, on n'oublie pas de vérifier rapidement que $A \times A^{-1} = I$.


%%%%%%%%%%%%%%%%%%%%%%%%%%%%%%%%%%%%%%%%%%%%%%%%%%%%%%%%%%%
\diapo
Mettez immédiatement ce que vous venez d'apprendre en pratique !

\end{document}
