
%%%%%%%%%%%%%%%%%% PREAMBULE %%%%%%%%%%%%%%%%%%


\documentclass[12pt]{article}

\usepackage{amsfonts,amsmath,amssymb,amsthm}
\usepackage[utf8]{inputenc}
\usepackage[T1]{fontenc}
\usepackage[francais]{babel}


% packages
\usepackage{amsfonts,amsmath,amssymb,amsthm}
\usepackage[utf8]{inputenc}
\usepackage[T1]{fontenc}
%\usepackage{lmodern}

\usepackage[francais]{babel}
\usepackage{fancybox}
\usepackage{graphicx}

\usepackage{float}

%\usepackage[usenames, x11names]{xcolor}
\usepackage{tikz}
\usepackage{datetime}

\usepackage{mathptmx}
%\usepackage{fouriernc}
%\usepackage{newcent}
\usepackage[mathcal,mathbf]{euler}

%\usepackage{palatino}
%\usepackage{newcent}


% Commande spéciale prompteur

%\usepackage{mathptmx}
%\usepackage[mathcal,mathbf]{euler}
%\usepackage{mathpple,multido}

\usepackage[a4paper]{geometry}
\geometry{top=2cm, bottom=2cm, left=1cm, right=1cm, marginparsep=1cm}

\newcommand{\change}{{\color{red}\rule{\textwidth}{1mm}\\}}

\newcounter{mydiapo}

\newcommand{\diapo}{\newpage
\hfill {\normalsize  Diapo \themydiapo \quad \texttt{[\jobname]}} \\
\stepcounter{mydiapo}}


%%%%%%% COULEURS %%%%%%%%%%

% Pour blanc sur noir :
%\pagecolor[rgb]{0.5,0.5,0.5}
% \pagecolor[rgb]{0,0,0}
% \color[rgb]{1,1,1}



%\DeclareFixedFont{\myfont}{U}{cmss}{bx}{n}{18pt}
\newcommand{\debuttexte}{
%%%%%%%%%%%%% FONTES %%%%%%%%%%%%%
\renewcommand{\baselinestretch}{1.5}
\usefont{U}{cmss}{bx}{n}
\bfseries

% Taille normale : commenter le reste !
%Taille Arnaud
%\fontsize{19}{19}\selectfont

% Taille Barbara
%\fontsize{21}{22}\selectfont

%Taille François
\fontsize{25}{30}\selectfont

%Taille Pascal
%\fontsize{25}{30}\selectfont

%Taille Laura
%\fontsize{30}{35}\selectfont


%\myfont
%\usefont{U}{cmss}{bx}{n}

%\Huge
%\addtolength{\parskip}{\baselineskip}
}


% \usepackage{hyperref}
% \hypersetup{colorlinks=true, linkcolor=blue, urlcolor=blue,
% pdftitle={Exo7 - Exercices de mathématiques}, pdfauthor={Exo7}}


%section
% \usepackage{sectsty}
% \allsectionsfont{\bf}
%\sectionfont{\color{Tomato3}\upshape\selectfont}
%\subsectionfont{\color{Tomato4}\upshape\selectfont}

%----- Ensembles : entiers, reels, complexes -----
\newcommand{\Nn}{\mathbb{N}} \newcommand{\N}{\mathbb{N}}
\newcommand{\Zz}{\mathbb{Z}} \newcommand{\Z}{\mathbb{Z}}
\newcommand{\Qq}{\mathbb{Q}} \newcommand{\Q}{\mathbb{Q}}
\newcommand{\Rr}{\mathbb{R}} \newcommand{\R}{\mathbb{R}}
\newcommand{\Cc}{\mathbb{C}} 
\newcommand{\Kk}{\mathbb{K}} \newcommand{\K}{\mathbb{K}}

%----- Modifications de symboles -----
\renewcommand{\epsilon}{\varepsilon}
\renewcommand{\Re}{\mathop{\text{Re}}\nolimits}
\renewcommand{\Im}{\mathop{\text{Im}}\nolimits}
%\newcommand{\llbracket}{\left[\kern-0.15em\left[}
%\newcommand{\rrbracket}{\right]\kern-0.15em\right]}

\renewcommand{\ge}{\geqslant}
\renewcommand{\geq}{\geqslant}
\renewcommand{\le}{\leqslant}
\renewcommand{\leq}{\leqslant}

%----- Fonctions usuelles -----
\newcommand{\ch}{\mathop{\mathrm{ch}}\nolimits}
\newcommand{\sh}{\mathop{\mathrm{sh}}\nolimits}
\renewcommand{\tanh}{\mathop{\mathrm{th}}\nolimits}
\newcommand{\cotan}{\mathop{\mathrm{cotan}}\nolimits}
\newcommand{\Arcsin}{\mathop{\mathrm{Arcsin}}\nolimits}
\newcommand{\Arccos}{\mathop{\mathrm{Arccos}}\nolimits}
\newcommand{\Arctan}{\mathop{\mathrm{Arctan}}\nolimits}
\newcommand{\Argsh}{\mathop{\mathrm{Argsh}}\nolimits}
\newcommand{\Argch}{\mathop{\mathrm{Argch}}\nolimits}
\newcommand{\Argth}{\mathop{\mathrm{Argth}}\nolimits}
\newcommand{\pgcd}{\mathop{\mathrm{pgcd}}\nolimits} 

\newcommand{\Card}{\mathop{\text{Card}}\nolimits}
\newcommand{\Ker}{\mathop{\text{Ker}}\nolimits}
\newcommand{\id}{\mathop{\text{id}}\nolimits}
\newcommand{\ii}{\mathrm{i}}
\newcommand{\dd}{\mathrm{d}}
\newcommand{\Vect}{\mathop{\text{Vect}}\nolimits}
\newcommand{\Mat}{\mathop{\mathrm{Mat}}\nolimits}
\newcommand{\rg}{\mathop{\text{rg}}\nolimits}
\newcommand{\tr}{\mathop{\text{tr}}\nolimits}
\newcommand{\ppcm}{\mathop{\text{ppcm}}\nolimits}

%----- Structure des exercices ------

\newtheoremstyle{styleexo}% name
{2ex}% Space above
{3ex}% Space below
{}% Body font
{}% Indent amount 1
{\bfseries} % Theorem head font
{}% Punctuation after theorem head
{\newline}% Space after theorem head 2
{}% Theorem head spec (can be left empty, meaning ‘normal’)

%\theoremstyle{styleexo}
\newtheorem{exo}{Exercice}
\newtheorem{ind}{Indications}
\newtheorem{cor}{Correction}


\newcommand{\exercice}[1]{} \newcommand{\finexercice}{}
%\newcommand{\exercice}[1]{{\tiny\texttt{#1}}\vspace{-2ex}} % pour afficher le numero absolu, l'auteur...
\newcommand{\enonce}{\begin{exo}} \newcommand{\finenonce}{\end{exo}}
\newcommand{\indication}{\begin{ind}} \newcommand{\finindication}{\end{ind}}
\newcommand{\correction}{\begin{cor}} \newcommand{\fincorrection}{\end{cor}}

\newcommand{\noindication}{\stepcounter{ind}}
\newcommand{\nocorrection}{\stepcounter{cor}}

\newcommand{\fiche}[1]{} \newcommand{\finfiche}{}
\newcommand{\titre}[1]{\centerline{\large \bf #1}}
\newcommand{\addcommand}[1]{}
\newcommand{\video}[1]{}

% Marge
\newcommand{\mymargin}[1]{\marginpar{{\small #1}}}



%----- Presentation ------
\setlength{\parindent}{0cm}

%\newcommand{\ExoSept}{\href{http://exo7.emath.fr}{\textbf{\textsf{Exo7}}}}

\definecolor{myred}{rgb}{0.93,0.26,0}
\definecolor{myorange}{rgb}{0.97,0.58,0}
\definecolor{myyellow}{rgb}{1,0.86,0}

\newcommand{\LogoExoSept}[1]{  % input : echelle
{\usefont{U}{cmss}{bx}{n}
\begin{tikzpicture}[scale=0.1*#1,transform shape]
  \fill[color=myorange] (0,0)--(4,0)--(4,-4)--(0,-4)--cycle;
  \fill[color=myred] (0,0)--(0,3)--(-3,3)--(-3,0)--cycle;
  \fill[color=myyellow] (4,0)--(7,4)--(3,7)--(0,3)--cycle;
  \node[scale=5] at (3.5,3.5) {Exo7};
\end{tikzpicture}}
}



\theoremstyle{definition}
%\newtheorem{proposition}{Proposition}
%\newtheorem{exemple}{Exemple}
%\newtheorem{theoreme}{Théorème}
\newtheorem{lemme}{Lemme}
\newtheorem{corollaire}{Corollaire}
%\newtheorem*{remarque*}{Remarque}
%\newtheorem*{miniexercice}{Mini-exercices}
%\newtheorem{definition}{Définition}




%definition d'un terme
\newcommand{\defi}[1]{{\color{myorange}\textbf{\emph{#1}}}}
\newcommand{\evidence}[1]{{\color{blue}\textbf{\emph{#1}}}}



 %----- Commandes divers ------

\newcommand{\codeinline}[1]{\texttt{#1}}

%%%%%%%%%%%%%%%%%%%%%%%%%%%%%%%%%%%%%%%%%%%%%%%%%%%%%%%%%%%%%
%%%%%%%%%%%%%%%%%%%%%%%%%%%%%%%%%%%%%%%%%%%%%%%%%%%%%%%%%%%%%



\begin{document}

\debuttexte


%%%%%%%%%%%%%%%%%%%%%%%%%%%%%%%%%%%%%%%%%%%%%%%%%%%%%%%%%%%
\diapo

Dans ce chapitre, nous allons aborder la notion d'intégrale impropre. 
Il s'agit d'intégrer une fonction sur un domaine de longueur infinie, 
ou bien d'intégrer une fonction qui tend vers l'infini.


\change

\change

 Nous commencerons par préciser la notion de  points incertains, 
 qui sont les points où l'intégration pose problème

\change

 et nous définirons la convergence des intégrales impropres.

\change

Dans un deuxième temps, nous nous intéresserons aux propriétés de 
l'intégrale impropre, en commençant par la relation de Chasles et la linéarité,

\change 

puis la positivité

\change

et nous donnerons un critère de convergence dû à Cauchy.



%%%%%%%%%%%%%%%%%%%%%%%%%%%%%%%%%%%%%%%%%%%%%%%%%%%%%%%%%%%
\diapo

 On considère comme exemple la fonction $f$ qui à 
 $t$ non nul, associe $\displaystyle \frac{\sin |t|}{|t|^{3/2}}$, 
 et dont voici le graphe. On veut donner un sens à l'intégrale de 
 $f$ sur tout le domaine de définition.

 \change
 On commence  par identifier les \emph{points incertains} :
 
 qui sont dans cet exemple $-\infty$ et $+\infty$, d'une part,
 qui de toute façon seront toujours des points incertains, 
 et d'autre part le point $t=0$ au voisinage duquel 
 la fonction n'est pas bornée.


 \change
 On découpe ensuite chaque intervalle d'intégration 
 (ici les deux intervalles $]-\infty,0[$ et $]0,+\infty[$) en autant 
 d'intervalles qu'il faut pour que chacun d'eux ne contienne 
 qu'\evidence{un seul point incertain} à une de ses bornes. 
 On emploie pour cela la relation de Chasles.

 %%%%%%%%%%%%%%%%%%%%%%%%%%%%%%%%%%%%%%%%%%%%%%%%%%%%%%%%%%%
 \diapo

Après découpage, les intégrales impropres peuvent être de quatre types, 
selon la nature (finie ou infinie) de la borne
et selon que la fonction est étudiée à gauche ou à droite de cette borne. 
Voici la définition complète pour deux des types.

\change

Dans ce cours, on considérera toujours des intégrales de fonctions continues.

\change
Par définition, on dit que l'intégrale $\int_a^{+\infty} f(t)\;\dd t$ \defi{converge} si
la limite, lorsque $x$ tend vers $+\infty$, de la primitive 
$\int_a^{x} f(t)\;\dd t$ existe et est finie. 

\change

Si c'est le cas, on pose :
\begin{equation}
%\label{intcv1}
\int_a^{+\infty} f(t)\;\dd t = \lim_{x\rightarrow+\infty} 
\int_a^x f(t)\;\dd t\;. 
\end{equation}

\change

On définit de manière analogue l'intégrale d'une fonction continue 
$f$ définie sur l'intervalle $]a,b]$ ouvert en $a$.

\change
Dans ce cas on dit que $\int_a^b f(t)\;\dd t$ \defi{converge} si
 $\lim_{x\rightarrow *a^+*}  \int_x^b f(t)\;\dd t\;$ existe et est finie.
 
\change 
 
On appelle alors $\int_a^{b} f(t)\;\dd t $
la  limite de l'intégrale de $f$ sur des intervalles fermés $[x,b]$ où la limite  
est prise lorsque $x$ tend vers $a$ vers la droite.

\change

Si l'intégrale ne converge pas, on dit que l'intégrale \defi{diverge}.

Les définitions analogues pour les intervalles du type 
$]-\infty,a]$ ou $[a,b[$ s'obtiennent grâce au changement 
de variable $t\mapsto -t$.


%%%%%%%%%%%%%%%%%%%%%%%%%%%%%%%%%%%%%%%%%%%%%%%%%%%%%%%%%%
\diapo

L'étude des intégrales impropres 
consiste à intégrer une fonction sur un domaine où elle est définie, 
puis à passer à la limite. Voici un premier exemple.

L'intégrale
$$
\int_0^{+\infty} \frac{1}{1+t^2}\;\dd t\qquad \text{ converge.}
$$

\change


En effet une primitive de $\frac{1}{1+t^2}$ est $\arctan t$, 

\change
lorsqu'on intègre sur le domaine fini $[0,x]$ 
on obtient donc $\arctan x-\arctan 0$, 

\change
donc $\arctan x$.

\change
Puis le passage à la limite lorsque $x$ tend vers $+\infty$ 
donne $\frac{\pi}{2}$.

Comme la limite est finie alors l'intégrale est convergente.

\change
La valeur de cette intégrale impropre est donc $\frac{\pi}{2}$.

\change
Cela prouve que le domaine sous la courbe, qui n'est bien sûr pas borné,
a cependant une aire qui est finie !


%%%%%%%%%%%%%%%%%%%%%%%%%%%%%%%%%%%%%%%%%%%%%%%%%%%%%%%%%%
\diapo

Voici deux autres exemples pour développer notre intuition.

\change

$$
\int_0^1 \frac{1}{t}\;\dd t
$$
dont le point incertain est en $0$, diverge.
\change




en effet une primitive de $\frac{1}{t}$ 

\change
est $\ln t$, 


et l'intégration sur le domaine $[x,1]$ pour $x>0$ 

\change 
donne  $-\ln x$, 

\change
une quantité qui tend vers $+\infty$ lorsque $x$ tend vers $0$. 

Dans le cas d'une intégrale divergente, 

on ne parle pas de la valeur de l'intégrale.


\change

Par contre 
$$
\int_0^1 \ln t\;\dd t\qquad \text{ converge.}
$$ 
le point incertain est encore en $0$
\change

Une primitive de $\ln t$ 

\change
est $t\ln t -t$, 

\change
et l'intégration sur le domaine $[x,1]$ donne $x-x\ln x-1$, 

\change
qui tend vers $-1$ lorsque $x$ tend vers $0$ 
(car $x\ln x$ tend vers $0$). 

On a bien prouvé que cette intégrale converge et vaut $-1$.



%%%%%%%%%%%%%%%%%%%%%%%%%%%%%%%%%%%%%%%%%%%%%%%%%%%%%%%%%%
\diapo

Lorsqu'elle converge, l'intégrale impropre vérifie les mêmes 
propriétés que l'intégrale de Riemann usuelle.

Commençons par la relation de Chasles.

\change

Soit $f : [a,+\infty[ \to \Rr$ une fonction continue et soit
  $a' \in [a,+\infty[$. 
  Alors les intégrales impropres $\int_a^{+\infty} f(t) \;\dd t$ 
  et $\int_{a'}^{+\infty} f(t) \;\dd t $ sont de même nature. 

\og{}\^Etre de même nature  \fg{}  signifie que les deux intégrales 
sont convergentes en même temps ou bien divergentes en même temps.

\change

Si elles convergent, la relation de Chasles est vérifiée :
$\int_a^{+\infty}  = \int_{a}^{a'} + \int_{a'}^{+\infty}$ 

La preuve résulte de la relation de Chasles pour les intégrales 
usuelles et d'un passage à la limite.

La relation de Chasles implique donc que la convergence ne dépend
pas du comportement de la fonction sur des intervalles bornés,
mais seulement de son comportement au voisinage de $+\infty$.



%%%%%%%%%%%%%%%%%%%%%%%%%%%%%%%%%%%%%%%%%%%%%%%%%%%%%%%%%%
\diapo
Pour ce qui est de la linéarité de l'intégrale impropre, on procède exactement 
de la même façon en se ramenant à des intégrales usuelles grâce à un passage 
à la limite utilisant la linéarité des limites.

L'énoncé de la linéarité est le suivant :

* $f$ et $g$ sont deux fonctions continues 
ayant des intégrales convergentes.

* $\lambda,\mu$ sont deux réels. 

* Alors d'une part l'intégrale 
$\int \lambda f + \mu g$ converge 

* et d'autre part 
$\int \lambda f + \mu g = \lambda\int  f  + \mu \int  g $

Bien sûr, on a des résultats analogues à la relation de Chasles 
et la linéarité 
pour les autres types d'intervalle d'étude.

Remarque : la réciprocité dans la linéarité est fausse,
il est possible de trouver deux fonctions $f,g$ telles que
$\int_a^{+\infty} f+g$ converge, sans que $\int_a^{+\infty} f$, ni
$\int_a^{+\infty} g$ convergent. Trouvez un tel exemple !

%%%%%%%%%%%%%%%%%%%%%%%%%%%%%%%%%%%%%%%%%%%%%%%%%%%%%%%%%%
\diapo

L'intégrale impropre 
est compatible avec la relation d'ordre sur les fonctions, 
au sens suivant : 
si $f\leq g$, alors l'intégrale impropre de $f$ sur $[a,+\infty[$ est 
inférieure à l'intégrale impropre de $g$ sur $[a,+\infty[$.

Attention ici on suppose que les intégrales convergent ! 
La preuve utilise le fait que le passage à la limite est 
compatible avec les inégalités larges.

\change

Voici en particulier, un résultat très utile : 
l'intégrale impropre d'une fonction positive est positive :

Si \quad $f\ge 0$ \quad alors \quad 
$\displaystyle \int_a^{+\infty} f(t)\;\dd t \ge 0$

On a de nouveau des résultats analogues pour les autres types 
d'intervalle d'étude, 
il faut seulement faire attention à respecter 
l'ordre des bornes d'intégration, 
comme pour les intégrales usuelles.


%%%%%%%%%%%%%%%%%%%%%%%%%%%%%%%%%%%%%%%%%%%%%%%%%%%%%%%%%%
\diapo

On dispose d'un critère de convergence très utile pour les intégrales impropres, mais plus technique.

Rappelons d'abord le critère de Cauchy pour les limites.

Une fonction $f$ admet une limite finie lorsque $x\to +\infty$ 

si et seulement si la fonction $f(u)-f(v)$ tend vers $0$ lorsque $u$ et $v$ tendent tous les deux vers $+\infty$ :

$$\forall \epsilon>0\quad\exists M \ge a \qquad 
\left( u,v \ge M \implies \big|f(u)-f(v)\big|<\epsilon\right)$$



\change

Le critère de Cauchy pour les intégrales impropres dit de même que 
$\int_a^{+\infty} f(t) \; \dd t$ converge si et seulement si 
$\int_u^v f(t) \; \dd t$ tend vers $0$ lorsque $u$ et $v$ tendent 
tous les deux vers $+\infty$. Mathématiquement :
$$\forall \epsilon>0  \quad \exists  M \ge a \qquad
\left( u,v \ge M \implies \Bigl|\int_u^v f(t) \;\dd t\Bigr|<\epsilon\right)$$

\change

Le critère de Cauchy pour les intégrales impropres 
se prouve en utilisant le critère de Cauchy pour les fonctions, 
appliqué à la primitive de la fonction dont on considère 
l'intégrale impropre,

\change

et en utilisant l'égalité 
$\big|F(u)-F(v)\big|=\big|\int_u^vf(t)\; \dd t\big|$.



%%%%%%%%%%%%%%%%%%%%%%%%%%%%%%%%%%%%%%%%%%%%%%%%%%%%%%%%%%
\diapo

On peut considérer des intégrales qui sont doublement impropres, 
c-à-d définies sur un intervalle dont les *deux* bornes sont des points 
incertains. On doit d'abord découper cet 
intervalle en deux pour se ramener à deux  intégrales ayant chacune 
un seul point incertain. \\

On considère une fonction $f$ définie et continue sur un 
intervalle ouvert $]a,b[$, où les points $a$ et $b$ sont deux points incertains.

\change

On dit que l'intégrale $\int_a^b f(t)\;\dd t$ \defi{converge} 
s'il existe
$c \in ]a,b[$, 
tel que les \evidence{deux} intégrales
impropres $\int_a^c f(t)\;\dd t$ et $\int_c^b f(t)\;\dd t$ convergent.

\change

En cas de convergence, la valeur de cette intégrale doublement impropre 
est alors la somme des 
deux intégrales impropres convergentes, conformément à ce qu'on attend 
d'après la relation de Chasles. Cette même relation implique que la nature 
et la valeur de cette intégrale doublement impropre ne dépendent pas du 
choix de $c$ dans l'intervalle ouvert $]a,b[$ :

quelque soit $c$
$$\int_a^b f(t)\;\dd t = \int_a^c f(t)\;\dd t+\int_c^b f(t)\;\dd t$$


Notez que si l'une des deux intégrales $\int_a^c f(t)\;\dd t$ ou 
$\int_c^b f(t)\;\dd t$ diverge, alors l'intégrale 
$\int_a^b f(t)\;\dd t$ diverge par définition.



%%%%%%%%%%%%%%%%%%%%%%%%%%%%%%%%%%%%%%%%%%%%%%%%%%%%%%%%%%
\diapo

Examinons un premier exemple : 
$\int_{-\infty}^{+\infty} \frac{1}{1+t^2}\;\dd t$ 
converge 

\change

Il résulte d'un calcul précédent que 
$ \int_0^{+\infty} \frac{1}{1+t^2}\;\dd t$ converge et vaut 
$\frac{\pi}{2}$, 

\change
le même calcul sur l'intervalle $]-\infty,0[$, 
ou un raisonnement de parité prouve
que $\int_{-\infty}^{0} \frac{1}{1+t^2}\;\dd t$
converge et vaut aussi $\frac{\pi}{2}$, 

\change
Conclusion : l'intégrale doublement impropre :
$\int_{-\infty}^{+\infty} \frac{1}{1+t^2}\;\dd t$ converge et vaut $\pi$

\change

Voici un exemple qui vous oblige a bien comprendre la définition :
$\int_{-\infty}^{+\infty}t\;\dd t$ diverge. 

\change
En effet  $\int_{0}^{x} t\;\dd t = \frac{x^2}{2}$ 
tend vers $+\infty$ lorsque $x$ tend vers  $+\infty$. 

\change
Donc $\int_0^{+\infty}t\;\dd t$ diverge

\change
D'après la définition de l'intégrale impropre alors
$\int_{-\infty}^{+\infty}t\;\dd t$ est une intégrale divergente.

\change
Il y a cependant un phénomène surprenant : 
pour tout $x$ on a $\int_{-x}^{+x} t\;\dd t=0$ (par parité). 

Conclusion : il ne faut jamais examiner la limite des intégrales lorsque 
les deux bornes tendent simultanément vers des points incertains, mais bel et bien revenir à la définition.



%%%%%%%%%%%%%%%%%%%%%%%%%%%%%%%%%%%%%%%%%%%%%%%%%%%%%%%%%%
\diapo

Voici quelques exercices pour vous familiariser avec 
les intégrales impropres.

\end{document}
