
%%%%%%%%%%%%%%%%%% PREAMBULE %%%%%%%%%%%%%%%%%%


\documentclass[12pt]{article}

\usepackage{amsfonts,amsmath,amssymb,amsthm}
\usepackage[utf8]{inputenc}
\usepackage[T1]{fontenc}
\usepackage[francais]{babel}


% packages
\usepackage{amsfonts,amsmath,amssymb,amsthm}
\usepackage[utf8]{inputenc}
\usepackage[T1]{fontenc}
%\usepackage{lmodern}

\usepackage[francais]{babel}
\usepackage{fancybox}
\usepackage{graphicx}

\usepackage{float}

%\usepackage[usenames, x11names]{xcolor}
\usepackage{tikz}
\usepackage{datetime}

\usepackage{mathptmx}
%\usepackage{fouriernc}
%\usepackage{newcent}
\usepackage[mathcal,mathbf]{euler}

%\usepackage{palatino}
%\usepackage{newcent}


% Commande spéciale prompteur

%\usepackage{mathptmx}
%\usepackage[mathcal,mathbf]{euler}
%\usepackage{mathpple,multido}

\usepackage[a4paper]{geometry}
\geometry{top=2cm, bottom=2cm, left=1cm, right=1cm, marginparsep=1cm}

\newcommand{\change}{{\color{red}\rule{\textwidth}{1mm}\\}}

\newcounter{mydiapo}

\newcommand{\diapo}{\newpage
\hfill {\normalsize  Diapo \themydiapo \quad \texttt{[\jobname]}} \\
\stepcounter{mydiapo}}


%%%%%%% COULEURS %%%%%%%%%%

% Pour blanc sur noir :
%\pagecolor[rgb]{0.5,0.5,0.5}
% \pagecolor[rgb]{0,0,0}
% \color[rgb]{1,1,1}



%\DeclareFixedFont{\myfont}{U}{cmss}{bx}{n}{18pt}
\newcommand{\debuttexte}{
%%%%%%%%%%%%% FONTES %%%%%%%%%%%%%
\renewcommand{\baselinestretch}{1.5}
\usefont{U}{cmss}{bx}{n}
\bfseries

% Taille normale : commenter le reste !
%Taille Arnaud
%\fontsize{19}{19}\selectfont

% Taille Barbara
%\fontsize{21}{22}\selectfont

%Taille François
\fontsize{25}{30}\selectfont

%Taille Pascal
%\fontsize{25}{30}\selectfont

%Taille Laura
%\fontsize{30}{35}\selectfont


%\myfont
%\usefont{U}{cmss}{bx}{n}

%\Huge
%\addtolength{\parskip}{\baselineskip}
}


% \usepackage{hyperref}
% \hypersetup{colorlinks=true, linkcolor=blue, urlcolor=blue,
% pdftitle={Exo7 - Exercices de mathématiques}, pdfauthor={Exo7}}


%section
% \usepackage{sectsty}
% \allsectionsfont{\bf}
%\sectionfont{\color{Tomato3}\upshape\selectfont}
%\subsectionfont{\color{Tomato4}\upshape\selectfont}

%----- Ensembles : entiers, reels, complexes -----
\newcommand{\Nn}{\mathbb{N}} \newcommand{\N}{\mathbb{N}}
\newcommand{\Zz}{\mathbb{Z}} \newcommand{\Z}{\mathbb{Z}}
\newcommand{\Qq}{\mathbb{Q}} \newcommand{\Q}{\mathbb{Q}}
\newcommand{\Rr}{\mathbb{R}} \newcommand{\R}{\mathbb{R}}
\newcommand{\Cc}{\mathbb{C}} 
\newcommand{\Kk}{\mathbb{K}} \newcommand{\K}{\mathbb{K}}

%----- Modifications de symboles -----
\renewcommand{\epsilon}{\varepsilon}
\renewcommand{\Re}{\mathop{\text{Re}}\nolimits}
\renewcommand{\Im}{\mathop{\text{Im}}\nolimits}
%\newcommand{\llbracket}{\left[\kern-0.15em\left[}
%\newcommand{\rrbracket}{\right]\kern-0.15em\right]}

\renewcommand{\ge}{\geqslant}
\renewcommand{\geq}{\geqslant}
\renewcommand{\le}{\leqslant}
\renewcommand{\leq}{\leqslant}

%----- Fonctions usuelles -----
\newcommand{\ch}{\mathop{\mathrm{ch}}\nolimits}
\newcommand{\sh}{\mathop{\mathrm{sh}}\nolimits}
\renewcommand{\tanh}{\mathop{\mathrm{th}}\nolimits}
\newcommand{\cotan}{\mathop{\mathrm{cotan}}\nolimits}
\newcommand{\Arcsin}{\mathop{\mathrm{Arcsin}}\nolimits}
\newcommand{\Arccos}{\mathop{\mathrm{Arccos}}\nolimits}
\newcommand{\Arctan}{\mathop{\mathrm{Arctan}}\nolimits}
\newcommand{\Argsh}{\mathop{\mathrm{Argsh}}\nolimits}
\newcommand{\Argch}{\mathop{\mathrm{Argch}}\nolimits}
\newcommand{\Argth}{\mathop{\mathrm{Argth}}\nolimits}
\newcommand{\pgcd}{\mathop{\mathrm{pgcd}}\nolimits} 

\newcommand{\Card}{\mathop{\text{Card}}\nolimits}
\newcommand{\Ker}{\mathop{\text{Ker}}\nolimits}
\newcommand{\id}{\mathop{\text{id}}\nolimits}
\newcommand{\ii}{\mathrm{i}}
\newcommand{\dd}{\mathrm{d}}
\newcommand{\Vect}{\mathop{\text{Vect}}\nolimits}
\newcommand{\Mat}{\mathop{\mathrm{Mat}}\nolimits}
\newcommand{\rg}{\mathop{\text{rg}}\nolimits}
\newcommand{\tr}{\mathop{\text{tr}}\nolimits}
\newcommand{\ppcm}{\mathop{\text{ppcm}}\nolimits}

%----- Structure des exercices ------

\newtheoremstyle{styleexo}% name
{2ex}% Space above
{3ex}% Space below
{}% Body font
{}% Indent amount 1
{\bfseries} % Theorem head font
{}% Punctuation after theorem head
{\newline}% Space after theorem head 2
{}% Theorem head spec (can be left empty, meaning ‘normal’)

%\theoremstyle{styleexo}
\newtheorem{exo}{Exercice}
\newtheorem{ind}{Indications}
\newtheorem{cor}{Correction}


\newcommand{\exercice}[1]{} \newcommand{\finexercice}{}
%\newcommand{\exercice}[1]{{\tiny\texttt{#1}}\vspace{-2ex}} % pour afficher le numero absolu, l'auteur...
\newcommand{\enonce}{\begin{exo}} \newcommand{\finenonce}{\end{exo}}
\newcommand{\indication}{\begin{ind}} \newcommand{\finindication}{\end{ind}}
\newcommand{\correction}{\begin{cor}} \newcommand{\fincorrection}{\end{cor}}

\newcommand{\noindication}{\stepcounter{ind}}
\newcommand{\nocorrection}{\stepcounter{cor}}

\newcommand{\fiche}[1]{} \newcommand{\finfiche}{}
\newcommand{\titre}[1]{\centerline{\large \bf #1}}
\newcommand{\addcommand}[1]{}
\newcommand{\video}[1]{}

% Marge
\newcommand{\mymargin}[1]{\marginpar{{\small #1}}}



%----- Presentation ------
\setlength{\parindent}{0cm}

%\newcommand{\ExoSept}{\href{http://exo7.emath.fr}{\textbf{\textsf{Exo7}}}}

\definecolor{myred}{rgb}{0.93,0.26,0}
\definecolor{myorange}{rgb}{0.97,0.58,0}
\definecolor{myyellow}{rgb}{1,0.86,0}

\newcommand{\LogoExoSept}[1]{  % input : echelle
{\usefont{U}{cmss}{bx}{n}
\begin{tikzpicture}[scale=0.1*#1,transform shape]
  \fill[color=myorange] (0,0)--(4,0)--(4,-4)--(0,-4)--cycle;
  \fill[color=myred] (0,0)--(0,3)--(-3,3)--(-3,0)--cycle;
  \fill[color=myyellow] (4,0)--(7,4)--(3,7)--(0,3)--cycle;
  \node[scale=5] at (3.5,3.5) {Exo7};
\end{tikzpicture}}
}



\theoremstyle{definition}
%\newtheorem{proposition}{Proposition}
%\newtheorem{exemple}{Exemple}
%\newtheorem{theoreme}{Théorème}
\newtheorem{lemme}{Lemme}
\newtheorem{corollaire}{Corollaire}
%\newtheorem*{remarque*}{Remarque}
%\newtheorem*{miniexercice}{Mini-exercices}
%\newtheorem{definition}{Définition}




%definition d'un terme
\newcommand{\defi}[1]{{\color{myorange}\textbf{\emph{#1}}}}
\newcommand{\evidence}[1]{{\color{blue}\textbf{\emph{#1}}}}



 %----- Commandes divers ------

\newcommand{\codeinline}[1]{\texttt{#1}}

%%%%%%%%%%%%%%%%%%%%%%%%%%%%%%%%%%%%%%%%%%%%%%%%%%%%%%%%%%%%%
%%%%%%%%%%%%%%%%%%%%%%%%%%%%%%%%%%%%%%%%%%%%%%%%%%%%%%%%%%%%%



\begin{document}

\debuttexte


%%%%%%%%%%%%%%%%%%%%%%%%%%%%%%%%%%%%%%%%%%%%%%%%%%%%%%%%%%%


%%%%%%%%%%%%%%%%%%%%%%%%%%%%%%%%%%%%%%%%%%%%%%%%%%%%%%%%%%%
\diapo

Dans cette partie, nous nous concentrons sur les intégrales 
impropres des fonctions oscillantes.

\change

\change

Nous  expliquerons ce qu'est la convergence absolue

\change

puis nous définirons la semi-convergence

\change

et nous terminerons avec le Théorème d'Abel 
qui est un critère de semi-convergence.


%%%%%%%%%%%%%%%%%%%%%%%%%%%%%%%%%%%%%%%%%%%%%%%%%%%%%%%%%%%
\diapo

Nous considérons ici $$\int_a^{+\infty} f(t)\;\dd t\; ,$$ où $f(t)$
oscille jusqu'à l'infini entre des valeurs positives 
et négatives, comme sur le graphe suivant.

\change

La définition de l'intégrale impropre reste la même :

$$\int_a^{+\infty} f(t)\;\dd t = \lim_{x\rightarrow +\infty} \int_a^x f(t)\;\dd t\;.$$
 
%%%%%%%%%%%%%%%%%%%%%%%%%%%%%%%%%%%%%%%%%%%%%%%%%%%%%%%%%%%
\diapo

Le cas le plus favorable est celui où la *\emph{valeur absolue}* 
de $f$ converge. 

On dit que $\int f$ est \defi{absolument convergente} si 
$\int \big|f\big|\;\dd t$ converge.

\change 
Le théorème suivant est souvent utilisé pour démontrer la
convergence d'une intégrale. Malheureusement, il ne permet pas de
calculer la valeur de cette intégrale.

Théorème : si l'intégrale de $f$ sur $[a,+\infty[$ est 
absolument convergente, alors l'intégrale de $f$ sur 
$[a,+\infty[$ est convergente.

Autrement dit, être absolument convergent est 
plus fort qu'être convergent.

\change

C'est une conséquence du critère de Cauchy appliqué à $|f|$, 
puis à $f$.

\change
En effet, comme l'intégrale de $|f|$ sur $[a,+\infty[$ est 
convergente, on sait d'après le critère de Cauchy pour $|f|$, 
vu comme condition nécessaire de convergence, que 
$\int_u^v \big|f(t)\big| \;\dd t$ est aussi petit que l'on veut, ici 
$<\epsilon$, lorsque $u$ et $v$ tendent vers $+\infty$.

\change 

Mais comme d'après l'inégalité triangulaire 
$\left|\int_u^v f(t) \;\dd t\right| 
\le \int_u^v \big|f(t)\big| \;\dd t$, 
on a aussi que $\int_u^v f(t) \;\dd t < \epsilon$ en valeur absolue.

\change
En appliquant le critère de Cauchy pour $f$ vu 
comme condition suffisante de convergence, 
on peut conclure que l'intégrale de $f$ sur $[a,+\infty[$ 
est convergente.

%%%%%%%%%%%%%%%%%%%%%%%%%%%%%%%%%%%%%%%%%%%%%%%%%%%%%%%%%%%
\diapo

Voyons un exemple : 
$$\int_1^{+\infty} \frac{\sin t}{t^2}\;\dd t$$
 est absolument convergente.
 En particulier, elle converge.


\change

La preuve est très simple : 

\change
on peut majorer la valeur absolue de 
$\frac{\sin t}{t^2}$ par $\frac{1}{t^2}$, 

\change
et comme l'intégrale de Riemann
$$\int_1^{+\infty} \frac{1}{t^2}\;\dd t$$ converge , 

\change
alors on déduit du théorème de comparaison que 
$$\int_1^{+\infty} \frac{|\sin t|}{t^2}\;\dd t$$ est 
aussi convergente. 

Ainsi $\int_1^{+\infty} \frac{\sin t}{t^2}\;\dd t$
est absolument convergente.

Donc par le théorème précédent, cette intégrale est convergente.


%%%%%%%%%%%%%%%%%%%%%%%%%%%%%%%%%%%%%%%%%%%%%%%%%%%%%%%%%%%
\diapo

On dit que $\int_a^{+\infty} f(t)\;\dd t$ est \defi{semi-convergente} si 
elle est convergente mais *pas* absolument convergente.

\change
Voici un exemple :
$\displaystyle \int_1^{+\infty} \frac{\sin t }{t}\;\dd t$ est semi-convergente.

\change
Montrons d'abord la convergence et pour cela 
faisons une intégration par parties, ce qui donne :
$$\int_1^x \frac{\sin t}{t}\;\dd t  = \left[\frac{-\cos t}{t}\right]_1^x
-  \int_1^x \frac{\cos t}{t^2}\;\dd t\;.$$

\change
Examinons les deux termes :
$$\left[\frac{-\cos t}{t}\right]_1^x 
= -\frac{\cos x}{x}+\cos 1 \; .$$
    

Or la fonction $\frac{\cos x}{x}$ tend vers $0$ 
(lorsque $x \to +\infty$).

\change
Donc $\left[\frac{-\cos t}{t}\right]_1^x$ admet une 
limite finie, à savoir $\cos 1$. 

\change
Pour l'autre terme, notons que 
$\int_1^{+\infty} \frac{\cos t}{t^2}\;\dd t$ est 
une intégrale absolument convergente, 
par le même argument qu'on vient de voir 
pour $\int_1^{+\infty} \frac{\sin t}{t^2}\;\dd t$.

Conclusion : $\int_1^x \frac{\sin t}{t}\;\dd t$ admet 
une limite finie (lorsque $x\to+\infty$), et 
donc par définition $\int_1^{+\infty} \frac{\sin t}{t}\;\dd t$ converge.


\change
Montrons à présent qu'il n'y a pas convergence absolue. 

\change
On emploie la minoration suivante :
$\frac{|\sin t|}{t} \ge \frac{\sin^2 t}{t}$
car $\big| \sin t \big|\le 1$

\change
et $\frac{\sin^2 t}{t}= \frac{1-\cos(2t)}{2t}$

\change
A présent le théorème de comparaison permet 
de conclure, en effet $\int_1^{+\infty}\frac{1-\cos(2t)}{2t} \dd t$ 
est somme de deux intégrales :

(.../...) \newpage

\change
tout d'abord une intégrale de Riemann divergente 
$\int_1^{+\infty}\frac{1}{2t} \dd t$,

et ensuite d'une intégrale $\int_1^{+\infty}\frac{-\cos(2t)}{2t} \dd t$ 
dont on prouve la convergence comme dans cette première partie.

\change
Une intégrale convergente + une intégrale divergente donne une intégrale divergente.
Ce qui termine cet exemple.


%%%%%%%%%%%%%%%%%%%%%%%%%%%%%%%%%%%%%%%%%%%%%%%%%%%%%%%%%%%
\diapo

Pour montrer qu'une intégrale converge, 
quand elle n'est pas absolument convergente, 
on dispose d'un critère plus sophistiqué : le théorème d'Abel.

On considère une fonction $f$,  $\mathcal{C}^1$ sur $[a,+\infty[$, positive,
décroissante, ayant une limite nulle en $+\infty$.

On considère de plus une fonction continue $g$ sur $[a,+\infty[$, telle que la
primitive $\int_a^x g(t)\;\dd t$ soit bornée.

Alors l'intégrale du produit des deux fonctions $f$ et $g$ converge.

Avec $f(t)=\frac1t$ et $g(t)=\sin t$, on retrouve que 
l'intégrale $\displaystyle \int_1^{+\infty} \frac{\sin t }{t}\;\dd t$ converge.
La preuve du théorème d'Abel est d'ailleurs analogue à celle de la convergence 
de $\displaystyle \int_1^{+\infty} \frac{\sin t }{t}\;\dd t$ 
que nous venons de voir.

\change

Le théorème d'Abel permet de généraliser l'exemple précédent.

Si $\alpha$ est un réel strictement
positif, et $k$ un entier positif \emph{impair}, alors l'intégrale 
$$
\int_1^{+\infty} \frac{\sin^k(t)}{t^\alpha}\;\dd t\quad\text{ converge.}
$$


Attention, cette intégrale n'est absolument convergente que 
pour $\alpha>1$. 

Mais elle converge pour tout $\alpha>0$.

\change
On vérifie que les hypothèses du théorème d'Abel sont
satisfaites pour $f(t) = \frac{1}{t^\alpha}$ et $g(t)=\sin^k(t)$. Pour
s'assurer que la primitive de $\sin^k$ est bornée, il suffit de
trouver une linéarisation, qui transformera $\sin^k(t)$ en une
combinaison linéaire des $\sin(\ell t)$ dont la
primitive sera toujours bornée. 
C'est dans cette étape qu'on utilise le fait que $k$ est impair.

Si $k$ est pair alors la linéarisation de $\sin^k$ fait apparaitre une terme constant
dont l'intégrale ne sera pas bornée.



%%%%%%%%%%%%%%%%%%%%%%%%%%%%%%%%%%%%%%%%%%%%%%%%%%%%%%%%%%%
\diapo

Voici quelques exercices pour vous entraîner avec 
les intégrales absolument convergente et semi-convergentes.

\end{document}
