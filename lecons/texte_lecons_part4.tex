
%%%%%%%%%%%%%%%%%% PREAMBULE %%%%%%%%%%%%%%%%%%


\documentclass[12pt]{article}

\usepackage{amsfonts,amsmath,amssymb,amsthm}
\usepackage[utf8]{inputenc}
\usepackage[T1]{fontenc}
\usepackage[francais]{babel}


% packages
\usepackage{amsfonts,amsmath,amssymb,amsthm}
\usepackage[utf8]{inputenc}
\usepackage[T1]{fontenc}
%\usepackage{lmodern}

\usepackage[francais]{babel}
\usepackage{fancybox}
\usepackage{graphicx}

\usepackage{float}

%\usepackage[usenames, x11names]{xcolor}
\usepackage{tikz}
\usepackage{datetime}

\usepackage{mathptmx}
%\usepackage{fouriernc}
%\usepackage{newcent}
\usepackage[mathcal,mathbf]{euler}

%\usepackage{palatino}
%\usepackage{newcent}


% Commande spéciale prompteur

%\usepackage{mathptmx}
%\usepackage[mathcal,mathbf]{euler}
%\usepackage{mathpple,multido}

\usepackage[a4paper]{geometry}
\geometry{top=2cm, bottom=2cm, left=1cm, right=1cm, marginparsep=1cm}

\newcommand{\change}{{\color{red}\rule{\textwidth}{1mm}\\}}

\newcounter{mydiapo}

\newcommand{\diapo}{\newpage
\hfill {\normalsize  Diapo \themydiapo \quad \texttt{[\jobname]}} \\
\stepcounter{mydiapo}}


%%%%%%% COULEURS %%%%%%%%%%

% Pour blanc sur noir :
%\pagecolor[rgb]{0.5,0.5,0.5}
% \pagecolor[rgb]{0,0,0}
% \color[rgb]{1,1,1}



%\DeclareFixedFont{\myfont}{U}{cmss}{bx}{n}{18pt}
\newcommand{\debuttexte}{
%%%%%%%%%%%%% FONTES %%%%%%%%%%%%%
\renewcommand{\baselinestretch}{1.5}
\usefont{U}{cmss}{bx}{n}
\bfseries

% Taille normale : commenter le reste !
%Taille Arnaud
%\fontsize{19}{19}\selectfont

% Taille Barbara
%\fontsize{21}{22}\selectfont

%Taille François
\fontsize{25}{30}\selectfont

%Taille Pascal
%\fontsize{25}{30}\selectfont

%Taille Laura
%\fontsize{30}{35}\selectfont


%\myfont
%\usefont{U}{cmss}{bx}{n}

%\Huge
%\addtolength{\parskip}{\baselineskip}
}


% \usepackage{hyperref}
% \hypersetup{colorlinks=true, linkcolor=blue, urlcolor=blue,
% pdftitle={Exo7 - Exercices de mathématiques}, pdfauthor={Exo7}}


%section
% \usepackage{sectsty}
% \allsectionsfont{\bf}
%\sectionfont{\color{Tomato3}\upshape\selectfont}
%\subsectionfont{\color{Tomato4}\upshape\selectfont}

%----- Ensembles : entiers, reels, complexes -----
\newcommand{\Nn}{\mathbb{N}} \newcommand{\N}{\mathbb{N}}
\newcommand{\Zz}{\mathbb{Z}} \newcommand{\Z}{\mathbb{Z}}
\newcommand{\Qq}{\mathbb{Q}} \newcommand{\Q}{\mathbb{Q}}
\newcommand{\Rr}{\mathbb{R}} \newcommand{\R}{\mathbb{R}}
\newcommand{\Cc}{\mathbb{C}} 
\newcommand{\Kk}{\mathbb{K}} \newcommand{\K}{\mathbb{K}}

%----- Modifications de symboles -----
\renewcommand{\epsilon}{\varepsilon}
\renewcommand{\Re}{\mathop{\text{Re}}\nolimits}
\renewcommand{\Im}{\mathop{\text{Im}}\nolimits}
%\newcommand{\llbracket}{\left[\kern-0.15em\left[}
%\newcommand{\rrbracket}{\right]\kern-0.15em\right]}

\renewcommand{\ge}{\geqslant}
\renewcommand{\geq}{\geqslant}
\renewcommand{\le}{\leqslant}
\renewcommand{\leq}{\leqslant}

%----- Fonctions usuelles -----
\newcommand{\ch}{\mathop{\mathrm{ch}}\nolimits}
\newcommand{\sh}{\mathop{\mathrm{sh}}\nolimits}
\renewcommand{\tanh}{\mathop{\mathrm{th}}\nolimits}
\newcommand{\cotan}{\mathop{\mathrm{cotan}}\nolimits}
\newcommand{\Arcsin}{\mathop{\mathrm{Arcsin}}\nolimits}
\newcommand{\Arccos}{\mathop{\mathrm{Arccos}}\nolimits}
\newcommand{\Arctan}{\mathop{\mathrm{Arctan}}\nolimits}
\newcommand{\Argsh}{\mathop{\mathrm{Argsh}}\nolimits}
\newcommand{\Argch}{\mathop{\mathrm{Argch}}\nolimits}
\newcommand{\Argth}{\mathop{\mathrm{Argth}}\nolimits}
\newcommand{\pgcd}{\mathop{\mathrm{pgcd}}\nolimits} 

\newcommand{\Card}{\mathop{\text{Card}}\nolimits}
\newcommand{\Ker}{\mathop{\text{Ker}}\nolimits}
\newcommand{\id}{\mathop{\text{id}}\nolimits}
\newcommand{\ii}{\mathrm{i}}
\newcommand{\dd}{\mathrm{d}}
\newcommand{\Vect}{\mathop{\text{Vect}}\nolimits}
\newcommand{\Mat}{\mathop{\mathrm{Mat}}\nolimits}
\newcommand{\rg}{\mathop{\text{rg}}\nolimits}
\newcommand{\tr}{\mathop{\text{tr}}\nolimits}
\newcommand{\ppcm}{\mathop{\text{ppcm}}\nolimits}

%----- Structure des exercices ------

\newtheoremstyle{styleexo}% name
{2ex}% Space above
{3ex}% Space below
{}% Body font
{}% Indent amount 1
{\bfseries} % Theorem head font
{}% Punctuation after theorem head
{\newline}% Space after theorem head 2
{}% Theorem head spec (can be left empty, meaning ‘normal’)

%\theoremstyle{styleexo}
\newtheorem{exo}{Exercice}
\newtheorem{ind}{Indications}
\newtheorem{cor}{Correction}


\newcommand{\exercice}[1]{} \newcommand{\finexercice}{}
%\newcommand{\exercice}[1]{{\tiny\texttt{#1}}\vspace{-2ex}} % pour afficher le numero absolu, l'auteur...
\newcommand{\enonce}{\begin{exo}} \newcommand{\finenonce}{\end{exo}}
\newcommand{\indication}{\begin{ind}} \newcommand{\finindication}{\end{ind}}
\newcommand{\correction}{\begin{cor}} \newcommand{\fincorrection}{\end{cor}}

\newcommand{\noindication}{\stepcounter{ind}}
\newcommand{\nocorrection}{\stepcounter{cor}}

\newcommand{\fiche}[1]{} \newcommand{\finfiche}{}
\newcommand{\titre}[1]{\centerline{\large \bf #1}}
\newcommand{\addcommand}[1]{}
\newcommand{\video}[1]{}

% Marge
\newcommand{\mymargin}[1]{\marginpar{{\small #1}}}



%----- Presentation ------
\setlength{\parindent}{0cm}

%\newcommand{\ExoSept}{\href{http://exo7.emath.fr}{\textbf{\textsf{Exo7}}}}

\definecolor{myred}{rgb}{0.93,0.26,0}
\definecolor{myorange}{rgb}{0.97,0.58,0}
\definecolor{myyellow}{rgb}{1,0.86,0}

\newcommand{\LogoExoSept}[1]{  % input : echelle
{\usefont{U}{cmss}{bx}{n}
\begin{tikzpicture}[scale=0.1*#1,transform shape]
  \fill[color=myorange] (0,0)--(4,0)--(4,-4)--(0,-4)--cycle;
  \fill[color=myred] (0,0)--(0,3)--(-3,3)--(-3,0)--cycle;
  \fill[color=myyellow] (4,0)--(7,4)--(3,7)--(0,3)--cycle;
  \node[scale=5] at (3.5,3.5) {Exo7};
\end{tikzpicture}}
}



\theoremstyle{definition}
%\newtheorem{proposition}{Proposition}
%\newtheorem{exemple}{Exemple}
%\newtheorem{theoreme}{Théorème}
\newtheorem{lemme}{Lemme}
\newtheorem{corollaire}{Corollaire}
%\newtheorem*{remarque*}{Remarque}
%\newtheorem*{miniexercice}{Mini-exercices}
%\newtheorem{definition}{Définition}




%definition d'un terme
\newcommand{\defi}[1]{{\color{myorange}\textbf{\emph{#1}}}}
\newcommand{\evidence}[1]{{\color{blue}\textbf{\emph{#1}}}}



 %----- Commandes divers ------

\newcommand{\codeinline}[1]{\texttt{#1}}

%%%%%%%%%%%%%%%%%%%%%%%%%%%%%%%%%%%%%%%%%%%%%%%%%%%%%%%%%%%%%
%%%%%%%%%%%%%%%%%%%%%%%%%%%%%%%%%%%%%%%%%%%%%%%%%%%%%%%%%%%%%



\begin{document}

\debuttexte

%%%%%%%%%%%%%%%%%%%%%%%%%%%%%%%%%%%%%%%%%%%%%%%%%%%%%%%%%%%
\diapo

Je vais vous présenter dans cette leçon les formules de trigonométrie,
qui sont supposées connues en entrant à l'université.


%%%%%%%%%%%%%%%%%%%%%%%%%%%%%%%%%%%%%%%%%%%%%%%%%%%%%%%%%%%
\diapo

Commençons simplement par le cercle trigonométrique, qui est juste le cercle
centré à l'origine et de rayon $1$, sur lequel on lit les angles.

On part de $0$, puis on tourne dans le sens inverse des aiguilles d'une montre.
On a représenté ici quelques angles remarquables, $30$ degrés,
$45$ degrés, $60$ degrés, $90$ degrés pour un angle droit,
etc

$180$ degrés pour un angle plat,
et avec $360$ degrés on revient au départ.

On préfère utiliser les radians,

$\frac \pi 6$ pour $30$ degrés,

$\frac \pi 4$ pour $45$ degrés,

$\frac\pi 3$ pour $60$ degrés,

$\frac \pi 2$ pour $90$ degrés,

$\pi$ pour $180$ degrés,

et $2\pi$ pour $360$ degrés.

Enfin au lieu de parler de l'angle $\frac{11\pi}{6}$
il est plus facile de le noter $-\frac\pi6$.

Autre exemple $+\frac{3\pi}{2}$ est aussi l'angle $-\frac\pi2$.

%%%%%%%%%%%%%%%%%%%%%%%%%%%%%%%%%%%%%%%%%%%%%%%%%%%%%%%%%%%
\diapo

Un point du cercle correspondant à un angle $x$

a pour abscisse $\cos x$

et pour ordonnée $\sin x$.

\change

La droite $(OM)$ recoupe la droite d'équation $(x=1)$ en un point $T$

l'ordonnée de $T$ est $\tan x$.

\change

Voici trois formules de base :

$$ \cos^2 x + \sin^2 x = 1$$

qui exprime le fait que le point $M$ appartient au cercle trigonométrique (de rayon $1$)

Et $$\cos(x+2\pi)=\cos x,  \sin(x+2\pi)=\sin x$$

qui exprime le fait qu'après un tour complet on retombe sur le même point.


%%%%%%%%%%%%%%%%%%%%%%%%%%%%%%%%%%%%%%%%%%%%%%%%%%%%%%%%%%%
\diapo

$\cos (-x) = \cos x$

$\sin (-x) = -\sin x$ 

Cela se mémorise aussi avec le dessin.  

Les points correspondants à l'angle
$+x$ et $-x$ sont symétriques par rapport à l'axe des abscisses 

Ils ont donc la même abscisse $\cos x$
et des ordonnées opposées $+\sin x$ et $-\sin x$.


%%%%%%%%%%%%%%%%%%%%%%%%%%%%%%%%%%%%%%%%%%%%%%%%%%%%%%%%%%%
\diapo

Il en est de même pour les formules suivantes 
qui se retrouvent rapidement en faisant un petit croquis.

\begin{align*}
\cos (\pi + x) &= -\cos x \\
\sin (\pi + x) &= -\sin x \\  
\end{align*}  

\begin{align*}
\cos (\pi - x) &= -\cos x \\
\sin (\pi - x) &= \sin x \\  
\end{align*}  

\begin{align*}
\cos (\frac\pi2 - x) &= \sin x \\
\sin (\frac\pi2 - x) &= \cos x \\  
\end{align*}  
Le complément à $\frac\pi2$ échange donc les sinus et cosinus.


%%%%%%%%%%%%%%%%%%%%%%%%%%%%%%%%%%%%%%%%%%%%%%%%%%%%%%%%%%%
\diapo

Il est indispensable de connaître les valeurs des cosinus et sinus
prisent en certains angles particuliers : $0$, $\frac\pi6$,
$\frac\pi4$, $\frac\pi3$ et $\frac\pi2$.

Je vous laisse lire et apprendre ce tableau.

Bien sûr on retrouve ces valeurs
dans les coordonnées des points du cercle trigonométrique.


%%%%%%%%%%%%%%%%%%%%%%%%%%%%%%%%%%%%%%%%%%%%%%%%%%%%%%%%%%%
\diapo

Intéressons-nous maintenant aux fonctions sinus et cosinus.

Ce sont des fonctions périodiques de période $2\pi$.

\change

Voici un zoom sur l'intervalle $]-\pi,+\pi]$.



Leurs valeurs sont comprises entre $-1$ et $+1$.

La fonctions cosinus est paire (donc symétrique par rapport à l'axe des ordonnées).

La fonction sinus est impaire (donc symétrique par rapport à l'origine).

\change

Enfin pour les dérivées :

$\cos' x = -\sin x \qquad \sin'x=\cos x$



%%%%%%%%%%%%%%%%%%%%%%%%%%%%%%%%%%%%%%%%%%%%%%%%%%%%%%%%%%%
\diapo

Passons à la fonction tangente.

$\tan x = \dfrac{\sin x}{\cos x}$

Mais attention la tangente n'est pas définie lorsque le cosinus s'annule.
Il faut donc exclure les valeurs 
$-\frac\pi2, \frac\pi2, -\frac{3\pi}{2}, \frac{3\pi}{2},\ldots $

Voici le graphe de la fonction tangente, c'est une fonction 
impaire et périodique de période $\pi$. Aux points d'indétermination
la limite à gauche est $+\infty$, la limite à droite est $-\infty$.


Enfin la dérivée de la fonction tangente est 

$\tan' x = 1+\tan^2x$ qui s'écrit aussi $\dfrac{1}{\cos^2x}$.


%%%%%%%%%%%%%%%%%%%%%%%%%%%%%%%%%%%%%%%%%%%%%%%%%%%%%%%%%%%
\diapo

Passons au formules d'addition, qu'il faut connaître par c\oe ur.

\begin{align*}
\cos(a+b) &= \cos a \cdot \cos b - \sin a \cdot \sin b \\
\sin(a+b) &= \sin a\cdot \cos b  +  \sin b\cdot\cos a \\
\tan (a+b) &=\dfrac{\tan a + \tan b}{1-\tan a \cdot \tan b}\\
\end{align*}

On en déduirait immédiatement des formules pour $\cos(a-b)$, $\sin(a-b)$,
$\tan(a-b)$.

\change

Mais ce qui est plus utile ce sont les formules d'additions
obtenues en faisant $a=b$, qu'il faut aussi apprendre par c\oe ur.

\begin{align*}
\cos 2a &= 2\cos^2a-1\\
    &= 1-2\sin^2a\\
    &=\cos^2a-\sin^2a\\
\sin 2a &= 2\sin a\cdot \cos a\\
\tan 2a &= \frac{2\tan a}{1-\tan^2 a}
\end{align*}  


%%%%%%%%%%%%%%%%%%%%%%%%%%%%%%%%%%%%%%%%%%%%%%%%%%%%%%%%%%%
\diapo

Des formules d'additions découlent plusieurs formules :

commençons par des produits de cosinus et de sinus qui peuvent s'exprimer
comme somme de cosinus ou de sinus.

Par exemple $\cos a\cdot\cos b = \frac{1}{2}\big[ \cos(a+b)+\cos(a-b)\big]$.

\change

Une autre formulation est de passer d'une somme à un produit ;
par exemple :

$\cos p+\cos q = 2\cos \frac{p+q}{2}\cdot\cos\frac{p-q}{2}$

%%%%%%%%%%%%%%%%%%%%%%%%%%%%%%%%%%%%%%%%%%%%%%%%%%%%%%%%%%%
\diapo

Les formules de la <<tangente de l'arc moitié>> permettent d'exprimer sinus, cosinus et tangente
en fonction de $\tan \frac x2$.

En posant $t=\tan \frac{x}{2}$ on a les relations :

$\cos x = \frac {1-t^2}{1+t^2}$

$\sin x = \frac{2t}{1+t^2}$
    
$\tan x = \frac{2t}{1-t^2}$

Ces formules sont utiles pour le calcul de certaines intégrales par changement de variable, 
on utilise alors en plus la relation $dx=\dfrac{2dt}{1+t^2}$.

%%%%%%%%%%%%%%%%%%%%%%%%%%%%%%%%%%%%%%%%%%%%%%%%%%%%%%%%%%%
\diapo

Une fois que vous connaissez bien ces expressions, faites ces exercices sans
consulter les formules !

\end{document}