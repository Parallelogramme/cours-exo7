\documentclass[class=report,crop=false]{standalone}
\usepackage[screen]{../exo7book}

\begin{document}

%====================================================================
\chapitre{Systèmes linéaires}
%====================================================================

\insertvideo{0uYJ3RNL5SU}{partie 1. Introduction aux systèmes d'équations linéaires}

\insertvideo{h62zbpi39YY}{partie 2. Théorie des systèmes linéaires}

\insertvideo{WDHDv55LS-I}{partie 3. Résolution par la méthode du pivot de Gauss}

\insertfiche{fic00160.pdf}{Systèmes d'équations linéaires}

%%%%%%%%%%%%%%%%%%%%%%%%%%%%%%%%%%%%%%%%%%%%%%%%%%%%%%%%%%%%%%%%
\section{Introduction aux systèmes d'équations linéaires}


L'algèbre linéaire est un outil essentiel pour toutes les branches des mathématiques,
en particulier lorsqu'il s'agit de modéliser puis résoudre numériquement des problèmes issus de
divers domaines : des sciences physiques ou mécaniques, des sciences du vivant, de la chimie,
de l'économie, des sciences de l'ingénieur ...

Les systèmes linéaires interviennent 
%dans de nombreux contextes d'applications
à travers leurs applications dans de nombreux contextes,
car ils forment la base calculatoire
de l'algèbre linéaire. Ils permettent également de traiter
une bonne partie de la théorie de l'algèbre linéaire en dimension finie.
C'est pourquoi 
%le présent cours 
ce cours
commence avec une étude des équations
linéaires et de leur résolution.

Le but de ce chapitre est
%Ce chapitre a un but 
essentiellement pratique : il s'agit de résoudre des systèmes linéaires.
La partie théorique sera revue et prouvée dans le chapitre \og Matrices \fg.

%---------------------------------------------------------------
\subsection{Exemple : deux droites dans le plan}

L'équation d'une droite dans le plan $(Oxy)$ s'écrit
$$a x + b y = e$$
où $a, b$ et $e$ sont des paramètres réels, $a$ et $b$ n'étant pas simultanément nuls. Cette équation s'appelle
\defi{équation linéaire}\index{equation lineaire@équation linéaire} dans les variables
(ou inconnues) $x$ et $y$.\\

Par exemple, $2x + 3y = 6$ est une équation linéaire, alors
que les équations suivantes ne sont pas des équations linéaires :
 $$ 2x + y^2 = 1 \quad \text { ou }  \quad y = \sin (x)  \quad \text { ou } \quad  x = \sqrt{y}.$$


\bigskip

Considérons maintenant deux droites $D_1$ et $D_2$ et cherchons les points qui
sont simultanément sur ces deux droites. Un point $(x,y)$ est dans l'intersection
$D_1 \cap D_2$ s'il est solution du système :
\begin{equation}
\left\{\begin{array}{rcl}
a x + b y & = & e\\
c x + d y & = & f
\end{array}\right.
\tag{$S$}
\label{eq:syslin1}
\end{equation}


Trois cas se présentent alors:
\begin{enumerate}
  \item Les droites $D_1$ et $D_2$ se coupent en un seul point.
Dans ce cas, illustré par la figure de gauche,
le système (\ref{eq:syslin1}) a une seule solution.


  \item Les droites $D_1$ et $D_2$ sont parallèles. Alors le système
  (\ref{eq:syslin1}) n'a pas de solution.
 La figure du centre illustre cette situation.


  \item Les droites $D_1$ et $D_2$ sont confondues et, dans ce cas,
  le système  (\ref{eq:syslin1}) a une infinité de solutions.
\end{enumerate}


\myfigure{0.6}{
\tikzinput{fig_syslin01}
\ \ 
\tikzinput{fig_syslin02}
\ \ 
\tikzinput{fig_syslin03}
}


Nous verrons plus loin que ces trois cas de figure (une seule solution,
aucune solution, une infinité de solutions)
sont les seuls cas qui peuvent se présenter pour n'importe quel système
d'équations linéaires.

%---------------------------------------------------------------
\subsection{Résolution par substitution}

Pour savoir s'il existe une ou plusieurs solutions à un système linéaire,
et les calculer, une première méthode est la
\defi{substitution}\index{substitution}.
Par exemple pour le système :
\begin{equation}
\left\{\begin{array}{rcl}
3 x + 2 y & = & 1\\
2 x - 7 y & = & -2
\end{array}\right.
\tag{$S$}
\label{eq:syslin2}
\end{equation}
Nous réécrivons la première ligne $3 x + 2 y  =  1$ sous la forme
$y = \frac12 - \frac32x$. Et nous remplaçons (nous \emph{substituons}) le $y$
de la seconde équation, par l'expression $\frac12 - \frac32x$. Nous obtenons un système
équivalent :
$$\left\{\begin{array}{rcl}
y & = & \frac12 - \frac32x\\
2 x - 7 (\frac12 - \frac32x) & = & -2
\end{array}\right.$$
La seconde équation est maintenant une expression qui ne contient que des $x$, et
on peut la résoudre :
$$\left\{\begin{array}{rcl}
y & = & \frac12 - \frac32x\\
(2+7\times\frac32)x  & = & -2 +\frac72
\end{array}\right.
\iff
\left\{\begin{array}{rcl}
y & = & \frac12 - \frac32x\\
x & = & \frac{3}{25}
\end{array}\right.$$
Il ne reste plus qu'à remplacer dans la première ligne la valeur de $x$ obtenue :
$$\left\{\begin{array}{rcl}
y & = & \frac{8}{25}\\
x & = & \frac{3}{25}
\end{array}\right.$$
Le système (\ref{eq:syslin2}) admet donc une  solution unique $(\frac{3}{25},\frac{8}{25})$.
L'ensemble des solutions est donc
$$\mathcal{S} = \left\lbrace \left(\frac{3}{25},\frac{8}{25}\right) \right\rbrace.$$


%---------------------------------------------------------------
\subsection{Exemple : deux plans dans l'espace}

Dans l'espace $(Oxyz)$, une équation linéaire est l'équation d'un plan :
$$a x + b y  + c z = d$$
(on suppose ici que $a$, $b$ et $c$ ne sont pas simultanément nuls).

L'intersection de deux plans dans l'espace correspond au système suivant
à $2$ équations et à $3$ inconnues :
$$\left\{\begin{array}{rcl}
a x + b y + c z    & = & d \\
a' x + b' y + c' z & = & d'
\end{array}\right.$$

Trois cas se présentent alors:
\begin{itemize}
\item les plans sont parallèles (et distincts) et il n'y a alors aucune solution au système,
\item les plans sont confondus et il y a une infinité de solutions au système,
\item les plans se coupent en une droite et il y a une infinité de solutions.
\end{itemize}


\begin{exemple}
\sauteligne
\begin{enumerate}
  \item Le système $\left\{\begin{array}{rcl}
2 x + 3 y - 4 z  & = & 7 \\
4 x + 6 y - 8 z  & = & -1
\end{array}\right.$ n'a pas de solution. En effet, en divisant par $2$ la seconde équation,
on obtient le système équivalent :
$\left\{\begin{array}{rcl}
2 x + 3 y - 4 z  & = & 7 \\
2 x + 3 y - 4 z  & = & -\frac12
\end{array}\right.$. Les deux lignes sont clairement incompatibles : aucun $(x,y,z)$
ne peut vérifier à la fois $2 x + 3 y - 4 z  = 7$ et $2 x + 3 y - 4 z  =  -\frac12$.
L'ensemble des solutions est donc $\mathcal{S}=\varnothing$.

  \item Pour le système $\left\{\begin{array}{rcl}
2 x + 3 y - 4 z  & = & 7 \\
4 x + 6 y - 8 z  & = & 14
 \end{array}\right.$, les deux équations définissent le même plan !
 Le système est donc équivalent à une seule équation : $2 x + 3 y - 4 z = 7$.
 Si on réécrit cette équation sous la forme $z=\frac12 x + \frac34 y - \frac74$,
 alors on peut décrire l'ensemble des solutions sous la forme :
 $\mathcal{S}= \big\lbrace (x,y,\frac12 x + \frac34 y - \frac74) \mid x,y\in \Rr \big\rbrace$.

  \item Soit le système $\left\{\begin{array}{rcl}
7 x + 2 y - 2 z  & = & 1 \\
2 x + 3 y + 2 z  & = & 1
 \end{array}\right.$.
 Par substitution :
$$\left\{\begin{array}{l}
7 x + 2 y - 2 z   =  1 \\
2 x + 3 y + 2 z   =  1
 \end{array}\right. \iff
\left\{\begin{array}{l}
z   = \frac72 x + y -\frac12 \\
2 x + 3 y + 2 \big( \frac72 x + y -\frac12 \big)   =  1
 \end{array}\right. $$
$$\iff
\left\{\begin{array}{l}
z   = \frac72 x + y -\frac12 \\
9 x + 5 y    =  2
 \end{array}\right.   \iff
\left\{\begin{array}{l}
z = \frac72 x + y -\frac12 \\
y =  -\frac{9}{5} x +  \frac25
 \end{array}\right.
 \iff
\left\{\begin{array}{l}
z = \frac{17}{10}x-\frac{1}{10} \\
y =  -\frac{9}{5} x +  \frac25
 \end{array}\right.
 $$
Pour décrire l’ensemble des solutions, on peut choisir $x$ comme paramètre :
 $$\mathcal{S}= \left\lbrace \left(x,-\frac{9}{5} x +  \frac25,\frac{17}{10}x-\frac{1}{10}\right) \mid x\in \Rr \right\rbrace.$$
Géométriquement : nous avons trouvé une équation paramétrique de la droite
définie par l'intersection de deux plans.
\end{enumerate}


\end{exemple}

Du point de vue du nombre de solutions, nous constatons qu'il n'y a
que deux possibilités,
à savoir aucune solution ou une infinité de solutions. Mais les deux
derniers cas ci-dessus sont néanmoins très différents géométriquement et
il semblerait que dans le second cas (plans confondus), l'infinité de solutions
soit plus grande que dans le troisième cas. Les chapitres suivants nous permettront
de rendre rigoureuse cette impression.

Si on considère trois plans dans l'espace, une autre possibilité apparaît :
il se peut que les trois plans s'intersectent en un seul point.


%---------------------------------------------------------------
\subsection{Résolution par la méthode de Cramer}

\index{regle@règle!de Cramer}

On note $\left| \begin{smallmatrix} a & b \\ c & d \end{smallmatrix}\right|=ad-bc$
le \defi{déterminant}\index{determinant@déterminant}. On considère le cas d'un système de $2$ équations à $2$ inconnues :
$$
\left\{\begin{array}{rcl}
a x + b y & = & e\\
c x + d y & = & f
\end{array}\right.
$$
Si $ad-bc\neq 0$, on trouve une unique solution dont les coordonnées $(x,y)$ sont :
$$x = \frac{\begin{vmatrix} e & b \\ f & d \end{vmatrix}}{\begin{vmatrix} a & b \\ c & d \end{vmatrix}} \qquad
y = \frac{\begin{vmatrix} a & e \\ c & f \end{vmatrix}}{\begin{vmatrix} a & b \\ c & d \end{vmatrix}}$$

Notez que le dénominateur égale le déterminant pour les deux coordonnées et est donc non nul.
Pour le numérateur de la première coordonnée $x$, on remplace la première colonne par le second membre ;
pour la seconde coordonnée $y$, on remplace la seconde colonne par le second membre.

\begin{exemple}
Résolvons le système
$
\left\{\begin{array}{rcl}
t x - 2y   & = & 1\\
3x + t y & = & 1
\end{array}\right.
$
suivant la valeur du paramètre $t\in \Rr$.

Le déterminant associé au système est
$\left| \begin{smallmatrix} t & -2 \\ 3 & t \end{smallmatrix}\right|= t^2+6$
et ne s'annule jamais. Il existe donc une unique solution $(x,y)$ et elle vérifie :
$$x = \frac{\begin{vmatrix} 1 & -2 \\ 1 & t \end{vmatrix}}{t^2+6} = \frac{t+2}{t^2+6}, \qquad
y = \frac{\begin{vmatrix} t & 1 \\ 3 & 1 \end{vmatrix}}{t^2+6} = \frac{t-3}{t^2+6}.$$

Pour chaque $t$, l'ensemble des solutions est
$\mathcal{S}= \left\lbrace \left(\frac{t+2}{t^2+6},\frac{t-3}{t^2+6}\right) \right\rbrace$.
\end{exemple}

%---------------------------------------------------------------
\subsection{Résolution par inversion de matrice}

%Pour ceux qui connaissent les matrices,
En termes matriciels, le système linéaire
$$
\left\{\begin{array}{rcl}
a x + b y & = & e\\
c x + d y & = & f
\end{array}\right.
$$
est équivalent à
$$AX = Y \quad \text{ où } \quad A = \begin{pmatrix} a & b \\ c & d \end{pmatrix},
\quad  X = \begin{pmatrix} x \\ y \end{pmatrix}, \quad Y = \begin{pmatrix} e \\ f \end{pmatrix}.$$

Si le déterminant de la matrice $A$ est non nul, c'est-à-dire si $ad-bc \neq 0$,
alors la matrice $A$ est inversible et
$$A^{-1} = \frac{1}{ad-bc} \begin{pmatrix} d & -b \\ -c & a \end{pmatrix}$$
et l'unique solution $X=\left( \begin{smallmatrix} x \\ y \end{smallmatrix}\right)$
du système est donnée par
$$X = A^{-1} Y.$$

\begin{exemple}
Résolvons le système
$
\left\{\begin{array}{rcl}
 x + y  & = & 1\\
 x + t^2 y & = & t
\end{array}\right.
$
suivant la valeur du paramètre $t\in \Rr$.

Le déterminant du système est $\left| \begin{smallmatrix} 1 & 1 \\ 1 & t^2 \end{smallmatrix}\right|= t^2-1$.

\textbf{Premier cas. $t\neq+1$ et $t\neq-1$.}
Alors $t^2-1\neq 0$. La matrice $A=\left( \begin{smallmatrix} 1 & 1 \\ 1 & t^2 \end{smallmatrix}\right)$
est inversible d'inverse $A^{-1} = \frac{1}{t^2-1}\left( \begin{smallmatrix} t^2 & -1 \\ -1 & 1 \end{smallmatrix}\right)$.
Et la solution $X=\left( \begin{smallmatrix} x \\ y \end{smallmatrix}\right)$ est
$$X = A^{-1} Y = \frac{1}{t^2-1}\begin{pmatrix} t^2 & -1 \\ -1 & 1 \end{pmatrix} \begin{pmatrix} 1 \\ t \end{pmatrix}
= \frac{1}{t^2-1}\begin{pmatrix} t^2-t\\ t-1 \end{pmatrix}
= \begin{pmatrix} \frac{t}{t+1} \\ \frac{1}{t+1} \end{pmatrix}.$$
Pour chaque $t\neq \pm1$, l'ensemble des solutions est
$\mathcal{S}= \big\lbrace \left(\frac{t}{t+1},\frac{1}{t+1}\right) \big\rbrace$.

\textbf{Deuxième cas. $t=+1$.}
Le système s'écrit alors :$
\left\{\begin{array}{rcl}
 x + y  & = & 1\\
 x + y & = & 1
\end{array}\right.
$ et les deux équations sont identiques. Il y a une infinité de solutions :
$\mathcal{S}= \big\lbrace (x,1-x) \mid x\in \Rr \big\rbrace$.


\textbf{Troisième cas. $t=-1$.}
Le système s'écrit alors :
$
\left\{\begin{array}{rcl}
 x + y  & = & 1\\
 x + y & = & -1
\end{array}\right.$,
les deux équations sont clairement incompatibles et donc $\mathcal{S}= \varnothing$.
\end{exemple}





%---------------------------------------------------------------
%\subsection{Mini-exercices}

\begin{miniexercices}
\sauteligne
\begin{enumerate}
  \item Tracer les droites d'équations $\left\{\begin{array}{rcl}
x-2y  & = & -1\\
-x+3y & = & 3
\end{array}\right.$ et résoudre le système linéaire
 de trois façons différentes :
substitution, méthode de Cramer, inverse d'une matrice.
Idem avec $\left\{\begin{array}{rcl}
2x-y  & = & 4\\
3x+3y & = & -5
\end{array}\right..$

  \item Résoudre suivant la valeur du paramètre $t\in \Rr$ :
$\left\{\begin{array}{rcl}
4x-3y & = & t\\
2x-y  & = & t^2
\end{array}\right..$

  \item  Discuter et résoudre suivant la valeur du paramètre $t\in \Rr$ :
$\left\{\begin{array}{rcl}
tx-y     & = & 1\\
x+(t-2)y & = & -1
\end{array}\right..$
Idem avec $\left\{\begin{array}{rcl}
(t-1)x+y & = & 1\\
2x+ty    & = & -1
\end{array}\right..$

\end{enumerate}
\end{miniexercices}




%%%%%%%%%%%%%%%%%%%%%%%%%%%%%%%%%%%%%%%%%%%%%%%%%%%%%%%%%%%%%%%%
\section{Théorie des systèmes linéaires}

%---------------------------------------------------------------
\subsection{Définitions}

\begin{definition}
On appelle \defi{équation linéaire}\index{equation lineaire@équation linéaire} dans les variables (ou \defi{inconnues})
$x_1,\ldots,x_p$ toute relation de la forme
\begin{equation}
  a_1 x_1 + \cdots + a_p x_p = b,
\end{equation}
où $a_1, \ldots, a_p $ et $b$ sont des nombres réels donnés.
\end{definition}

\begin{remarque*}
\sauteligne
\begin{itemize}
  \item Il importe d'insister ici sur le fait que ces équations linéaires sont
\emph{implicites}, c'est-à-dire qu'elles décrivent des relations
entre les variables, mais ne donnent pas directement les valeurs que
peuvent prendre les variables.

  \item \emph{Résoudre} une équation signifie donc la rendre
\emph{explicite}, c'est-à-dire rendre plus apparentes les valeurs
que les variables peuvent prendre.

  \item On peut aussi considérer des équations linéaires de nombres rationnels ou de nombres complexes.
\end{itemize}
\end{remarque*}



Soit $n\ge 1$ un entier.
\begin{definition}
Un \defi{système de $n$ équations linéaires à $p$ inconnues}\index{systeme@système!linéaire} 
est une liste de $n$ équations linéaires.
\end{definition}

On écrit usuellement de tels systèmes en $n$ lignes placées les unes sous les autres.

\begin{exemple}
\label{exsyslin1}
 Le système suivant a $2$ équations et $3$ inconnues :
\[ \left\{  \begin{array}{ccccccc}
 x_1 &- &3x_2 &+ &x_3 & =  &1\\
 -2x_1 &+ &4x_2 &- &3x_3 & =  &9
 \end{array} \right .
\]
\end{exemple}

La forme générale d'un système linéaire de $n$ équations à $p$ inconnues est la suivante :
$$\left\{
\begin{array}{cccccccl}
 a_{11}x_1 & +a_{12}x_2  &+a_{13}x_3&+&\cdots&+a_{1p}x_p&=&b_1 \qquad
(\leftarrow\hbox{équation  $1$})   \\
 a_{21}x_1 & +a_{22}x_2  &+a_{23}x_3&+&\cdots&+a_{2p}x_p&=&b_2 \qquad
(\leftarrow\hbox{équation  $2$})  \\
 \vdots &  \vdots  & \vdots&&& \vdots&=& \vdots   \\
 a_{i1}x_1 & +a_{i2}x_2  &+a_{i3}x_3&+&\cdots&+a_{ip}x_p&=&b_i \qquad
(\leftarrow\hbox{équation  $i$}) \\
 \vdots &  \vdots  & \vdots&&& \vdots&=& \vdots   \\
 a_{n1}x_1 & +a_{n2}x_2  &+a_{n3}x_3&+&\cdots&+a_{np}x_p&=&b_n \qquad
(\leftarrow\hbox{équation  $n$})  \\
   \end{array}
\right.
$$

Les nombres $a_{ij}$, $i=1,\ldots, n$, $j=1,\ldots, p$, sont les \defi{coefficients} du système.
Ce sont des données. Les nombres $b_{i}$, $i=1,\ldots, n$, constituent le
\defi{second membre}\index{second membre} du système et sont également des données.

Il convient de bien observer comment on a rangé le système en lignes
(une ligne par équation) numérotées de $1$ à $n$ par l'indice $i$, et en colonnes :
les termes correspondant à une même inconnue $x_j$ sont alignés verticalement
les uns sous les autres. L'indice $j$ varie de $1$ à $p$. Il y a donc $p$ colonnes
à gauche des signes d'égalité, plus une colonne supplémentaire à droite pour le
second membre. La notation avec double indice $a_{ij}$ correspond à ce rangement :
le premier indice (ici $i$) est le numéro de \emph{ligne} et le second
indice (ici $j$) est le numéro de \emph{colonne}. Il est extrêmement important
de toujours respecter cette convention.

Dans l'exemple \ref{exsyslin1}, on a $n=2$ (nombre d'équations = nombre de lignes),
$p=3$ (nombre d'inconnues = nombre de colonnes à gauche du signe =) et
$a_{11}=1$, $a_{12}=-3$, $a_{13}=1$, $a_{21}=-2$, $a_{22}=4$, $a_{23}=-3$, $b_1=1$ et $b_2=9$.

\begin{definition}
Une \defi{solution} du système linéaire est une liste de
$p$ nombres réels $(s_1,s_2,\ldots,s_p)$ (un $p$-uplet) tels que si
l'on substitue $s_1$ pour $x_1$, $s_2$ pour $x_2$, etc., dans le système
linéaire, on obtient une égalité. L'\,\defi{ensemble des solutions du système}
est l'ensemble de tous ces $p$-uplets.
\end{definition}

\begin{exemple}
 Le système
\[ \left\{  \begin{array}{ccccccc}
 x_1 &- &3x_2 &+ &x_3 & =  &1\\
 -2x_1 &+ &4x_2 &- &3x_3 & =  &9
 \end{array} \right .
\]
admet comme solution $(-18,-6,1)$, c'est-à-dire
$$ x_1=-18\, , \qquad\qquad x_2 = -6\, , \qquad\qquad x_3 = 1\, .$$

Par contre, $(7,2,0)$ ne satisfait que la première équation.
Ce n'est donc pas une solution du système.
\end{exemple}

En règle générale, on s'attache à déterminer l'ensemble
des solutions d'un système linéaire. C'est ce que l'on appelle \defi{résoudre}
le système linéaire.
Ceci amène à poser la définition suivante.

\begin{definition}
\label{systemes equivalents}
On dit que deux systèmes linéaires sont \defi{équivalents}\index{systeme@système!équivalent}
s'ils ont le même ensemble de solutions.
\end{definition}


{\`A} partir de là, le jeu pour résoudre un système linéaire
donné consistera à le transformer en un système équivalent dont
la résolution sera plus simple que celle du système de départ.
Nous verrons plus loin comment procéder de façon systématique pour arriver à ce but.

%---------------------------------------------------------------
\subsection{Différents types de systèmes}

Voici un résultat théorique important pour les systèmes linéaires.
\begin{theoreme}
Un système d'équations linéaires n'a soit aucune solution,
soit une seule solution, soit une infinité de solutions.
\end{theoreme}


En particulier, si vous trouvez $2$ solutions différentes à un système
linéaire, alors c'est que vous pouvez en trouver une infinité !
Un système linéaire qui n'a aucune solution est dit \defi{incompatible}\index{systeme@système!incompatible}.
La preuve de ce théorème sera vue dans un chapitre ultérieur
(\og Matrices \fg).

%---------------------------------------------------------------
\subsection{Systèmes homogènes}

Un cas particulier important est celui des \defi{systèmes homogènes}\index{systeme@système!homogène}, pour lesquels
$b_1=b_2=\cdots=b_n=0$, c'est-à-dire dont le second membre est nul.
De tels systèmes sont toujours compatibles car ils admettent toujours
la solution $s_1=s_2=\cdots=s_p=0$. Cette solution est appelée
\emph{solution triviale}. Géométriquement, dans le cas $2\times2$,
un système homogène correspond à deux droites qui passent par l'origine,
$(0,0)$ étant donc toujours solution.



%---------------------------------------------------------------
%\subsection{Mini-exercices}

\begin{miniexercices}
\sauteligne
\begin{enumerate}
  \item \'Ecrire  un système linéaire de $4$ équations et $3$
  inconnues qui n'a aucune solution. Idem avec une infinité de solution.
  Idem avec une solution unique.

  \item Résoudre le système à $n$ équations et $n$ inconnues dont les équations sont
  $(L_i)$ : $x_i - x_{i+1} = 1$ pour $i=1,\ldots, n-1$ et $(L_n)$ : $x_n=1$.

  \item Résoudre les systèmes suivants : \\
$\left\{  \begin{array}{cccccc}
 x_1 & +2x_2 & +3x_3 & +4x_4 &=& 0\\
     & x_2   & +2x_3 & +3x_4 &=& 9\\
     &       & x_3   & +2x_4 &=& 0
 \end{array} \right.
$
$\left\{  \begin{array}{ccccc}
 x_1 & +2x_2 & +3x_3 &=& 1\\
 x_1 & +x_2  & +x_3  &=& 2\\
 x_1 & -x_2  & +x_3  &=& 3
 \end{array} \right.
$
$\left\{  \begin{array}{cccccc}
 x_1 & + x_2 &       &       &=& 1\\
     & x_2   & + x_3 &       &=& 2\\
     &       & x_3   & + x_4 &=& 3\\
 x_1 & +2x_2 & +2x_3 & + x_4 &=& 0\\
 \end{array} \right.
$

  \item Montrer que si un système linéaire \emph{homogène} a une solution
  $(x_1,\ldots,x_p) \neq (0,\ldots,0)$, alors il admet une infinité de solutions.

\end{enumerate}
\end{miniexercices}



%%%%%%%%%%%%%%%%%%%%%%%%%%%%%%%%%%%%%%%%%%%%%%%%%%%%%%%%%%%%%%%%
\section{Résolution par la méthode du pivot de Gauss}

%---------------------------------------------------------------
\subsection{Systèmes échelonnés}

\begin{definition}
Un système est \defi{échelonné}\index{systeme@système!échelonné} si :
\begin{itemize}
  \item le nombre de coefficients nuls commençant une ligne croît strictement ligne après ligne.
\end{itemize}

Il est \defi{échelonné réduit}\index{systeme@système!réduit} si en plus :
\begin{itemize}
\setcounter{enumi}{1}
  \item le premier coefficient non nul d'une ligne vaut $1$ ;

  \item et c'est le seul élément non nul de sa colonne.
\end{itemize}
\end{definition}


\begin{exemple}
\sauteligne
\begin{itemize}
  \item $\left\{\begin{array}{cccccc}
2x_1&+3x_2&+2x_3&-x_4&=&5\\
&-x_2&-2x_3&&=&4\\
&&&3x_4&=&1
\end{array}\right.
$ est échelonné (mais pas réduit).

  \item $\left\{\begin{array}{cccccc}
2x_1&+3x_2&+2x_3&-x_4&=&5\\
&&-2x_3&&=&4\\
&&x_3&+x_4&=&1
\end{array}\right.
$ n'est pas échelonné (la dernière ligne commence
avec la même variable que la ligne au-dessus).


\end{itemize}

\end{exemple}


Il se trouve que les systèmes linéaires sous une forme échelonnée réduite
sont particulièrement simples à résoudre.

\begin{exemple}
\label{syslin:exsysechel}
Le système linéaire suivant à $3$ équations et $4$ inconnues est échelonné et réduit.
$$\left\{\begin{array}{cccccc}
x_1&&+2x_3&&=&25\\
&x_2&-2x_3&&=&16\\
&&&x_4&=&1
\end{array}\right.
$$
Ce système se résout trivialement en
$$\left\{\begin{array}{ccc}
x_1&=&25-2x_3\\
x_2&=&16+2x_3\\
x_4&=&1.
\end{array}\right.
$$
En d'autres termes, pour toute valeur de $x_3$ réelle,
les valeurs de $x_1$, $x_2$ et $x_4$ calculées ci-dessus fournissent
une solution du système, et on les a ainsi toutes obtenues.
On peut donc décrire entièrement l'ensemble des solutions :
$$\mathcal{S}=\big\{(25-2x_3,16+2x_3,x_3,1) \mid x_3\in \Rr \big\}.$$
\end{exemple}


%---------------------------------------------------------------
\subsection{Opérations sur les équations d'un système}

Nous allons utiliser trois opérations
élémentaires sur les équations (c'est-à-dire sur les lignes) qui sont :
\begin{enumerate}
  \item $L_i \leftarrow \lambda L_i$ avec $\lambda \neq 0$ :
  on peut multiplier une équation par un réel non nul.

  \item $L_i \leftarrow L_i+\lambda L_j$ avec $\lambda \in \Rr$ (et $j\neq i$) :
  on peut ajouter à l'équation $L_i$ un multiple d'une autre équation $L_j$.

  \item $L_i \leftrightarrow L_j$ : on peut échanger deux équations.
\end{enumerate}

Ces trois opérations élémentaires ne changent pas les solutions d'un système linéaire ;
autrement dit ces opérations transforment un système linéaire en un système linéaire équivalent.


\begin{exemple}
Utilisons ces opérations élémentaires pour résoudre le système suivant.
\[ \left\{
\begin{array}{cccccr}
x  & + y & +7z & = & -1 & {\scriptstyle \quad (L_1)}\\
2x & - y & +5z & = & -5 & {\scriptstyle \quad (L_2)}\\
-x & -3y & -9z & = & -5 & {\scriptstyle \quad (L_3)}
\end{array} \right.
\]

Commençons par l'opération $L_2 \leftarrow L_2 - 2L_1$ :
on soustrait à la deuxième équation deux fois la première équation.
On obtient un système équivalent avec une nouvelle deuxième ligne (plus simple) :

\[ \left\{
\begin{array}{cccccr}
x  & + y & +7z & = & -1 & \\
   & -3y & -9z & = & -3 & {\scriptstyle \quad L_2 \leftarrow L_2 - 2L_1}\\
-x & -3y & -9z & = & -5 &
\end{array} \right.
\]

Puis $L_3 \leftarrow L_3 + L_1$ :
\[ \left\{
\begin{array}{cccccr}
x  & + y & +7z & = & -1 & \\
   & -3y & -9z & = & -3 & \\
   & -2y & -2z & = & -6 & {\scriptstyle \quad L_3 \leftarrow L_3 + L_1}
\end{array} \right.
\]


On continue pour faire apparaître un coefficient $1$ en tête de la deuxième ligne ;
pour cela on divise la ligne $L_2$ par $-3$ :
\[ \left\{
\begin{array}{cccccr}
x  & + y & +7z & = & -1 & \\
   &  y  & +3z & = & 1 & {\scriptstyle \  L_2 \leftarrow \ -\frac13 L_2}\\
   & -2y & -2z & = & -6 &
\end{array} \right.
\]

On continue ainsi
\[ \left\{
\begin{array}{cccccr}
x  & + y & +7z & = & -1 & \\
   &  y  & +3z & = & 1 & \\
   &     & 4z  & = & -4& {\scriptstyle \  L_3 \leftarrow L_3+2 L_2}
\end{array} \right.
\qquad
\left\{
\begin{array}{cccccr}
x  & + y & +7z & = & -1 & \\
   &  y  & +3z & = & 1 & \\
   &     & z  & = & -1& {\scriptstyle \ L_3 \leftarrow \frac14L_3}
\end{array} \right.
\]

\[ \left\{
\begin{array}{cccccr}
x  & + y & +7z & = & -1 & \\
   &  y  &     & = & 4  & {\scriptstyle \  L_2 \leftarrow L_2-3L_3}\\
   &     &  z  & = & -1 &
\end{array} \right.
\qquad
\left\{
\begin{array}{cccccr}
x  & + y &     & = & 6 & {\scriptstyle \  L_1 \leftarrow L_1 -7L_3}\\
   &  y  &     & = & 4 & \\
   &     & z   & = & -1 &
\end{array} \right.
\]

On aboutit à un système réduit et échelonné :
\[\left\{
\begin{array}{cccccr}
x  &   &      & = & 2 & {\scriptstyle \quad L_1 \leftarrow L_1 -L_2}\\
   &  y  &    & = & 4 & \\
   &     & z  & = & -1 &
\end{array} \right.
\]

On obtient ainsi $x=2$, $y=4$ et $z=-1$ et
l'unique solution du système est $(2,4,-1)$.
\end{exemple}

La méthode utilisée pour cet exemple est reprise et généralisée dans le paragraphe suivant.

%---------------------------------------------------------------
\subsection{Méthode du pivot de Gauss}

La méthode du pivot de Gauss permet de trouver les solutions de n'importe
quel système linéaire.
Nous allons décrire cet algorithme sur un exemple.
Il s'agit d'une description précise d'une suite d'opérations à effectuer,
qui dépendent de la situation et d'un ordre précis. Ce processus
aboutit toujours (et en plus assez rapidement)
à un système échelonné puis réduit, qui conduit immédiatement aux solutions du système.


\textbf{Partie A. Passage à une forme échelonnée.}

Soit le système suivant à résoudre :
 \[\left\{\begin{array}{ccccccr}
 & -x_2 &+2x_3& +13x_4 &=& 5  & \\
x_1 &-2x_2&+3x_3& +17x_4 &=& 4 & \\
-x_1& +3x_2    &-3x_3  & -20x_4  &=& -1 &
\end{array} \right.\]

Pour appliquer la méthode du pivot de Gauss, il faut
d'abord que le premier coefficient de la première ligne soit non nul.
Comme ce n'est pas le cas ici, on échange les deux premières lignes par l'opération élémentaire
$L_1 \leftrightarrow L_2$ :
\[\left\{\begin{array}{ccccccr}
x_1 &-2x_2&+3x_3& +17x_4 &=& 4 & {\scriptstyle \quad L_1 \leftrightarrow L_2} \\
    & -x_2 &+2x_3& +13x_4 &=& 5  &  \\
-x_1& +3x_2    &-3x_3  & -20x_4  &=& -1 &
\end{array} \right.\]

Nous avons déjà un coefficient $1$ devant le $x_1$ de la première ligne.
On dit que nous avons un \defi{pivot} en position $(1,1)$ (première ligne, première colonne).
Ce pivot sert de base pour éliminer tous les autres termes sur la même colonne.

Il n'y a pas de terme $x_1$ sur le deuxième ligne.
Faisons disparaître le terme $x_1$ de la troisième ligne ;
pour cela on fait l'opération élémentaire $L_3 \leftarrow L_3+L_1$ :
\[\left\{\begin{array}{ccccccr}
x_1 &-2x_2&+3x_3& +17x_4 &=& 4 & \\
    & -x_2 &+2x_3& +13x_4 &=& 5  &  \\
    & x_2    &  & -3x_4  &=& 3 & {\scriptstyle \quad L_3 \leftarrow L_3+L_1}
\end{array} \right.\]
\qquad

On change le signe de la seconde ligne ($L_2 \leftarrow -L_2$) pour faire apparaître
$1$ au coefficient du pivot $(2,2)$ (deuxième ligne, deuxième colonne) :
\[
\left\{\begin{array}{ccccccr}
x_1 &-2x_2&+3x_3& +17x_4 &=& 4 & \\
    & x_2 &-2x_3& -13x_4 &=& -5  &  {\scriptstyle \quad L_2 \leftarrow \ -L_2}\\
    & x_2    &  & -3x_4  &=& 3 &
\end{array} \right.
 \]


 On fait disparaître le terme $x_2$ de la troisième ligne,
 puis on fait apparaître un coefficient $1$ pour le pivot de la position $(3,3)$ :
\[
\left\{\begin{array}{ccccccr}
x_1 &-2x_2&+3x_3& +17x_4 &=& 4 & \\
    & x_2 &-2x_3& -13x_4 &=& -5  & \\
    &     & 2x_3 & +10x_4  &=& 8 & {\scriptstyle \quad L_3 \leftarrow L_3 -L_2}
\end{array} \right.
\]\[
\left\{\begin{array}{ccccccr}
x_1 &-2x_2&+3x_3& +17x_4 &=& 4 & \\
    & x_2 &-2x_3& -13x_4 &=& -5  & \\
    &     & x_3 & +5x_4  &=& 4  & {\scriptstyle \quad L_3 \leftarrow \frac12 L_3}
\end{array} \right. \]
Le système est maintenant sous forme échelonnée.

\bigskip

\textbf{Partie B. Passage à une forme réduite.}

Il reste à le mettre sous la forme échelonnée réduite.
Pour cela, on ajoute à une ligne des multiples adéquats des lignes situées
au-dessous d'elle, en allant du bas à droite vers le haut à gauche.


On fait apparaître des $0$ sur la troisième colonne en utilisant le pivot de la troisième ligne :
\[\left\{\begin{array}{ccccccr}
x_1 &-2x_2&+3x_3& +17x_4 &=& 4 & \\
    & x_2 &     & -3x_4  &=& 3 & {\scriptstyle \quad L_2 \leftarrow L_2 + 2L_3}\\
    &     & x_3 & +5x_4  &=& 4  &
\end{array} \right.
\]\[
\left\{\begin{array}{ccccccr}
x_1 &-2x_2&     & 2x_4 &=& -8 & {\scriptstyle \quad L_1 \leftarrow L_1 -3 L_3}\\
    & x_2 &     & -3x_4 &=& 3  & \\
    &     & x_3 & +5x_4  &=& 4  &
\end{array} \right. \]

On fait apparaître des $0$ sur la deuxième colonne (en utilisant le pivot de la deuxième ligne) :
\[\left\{\begin{array}{ccccccr}
x_1 &     &     & -4x_4 &=& -2 & {\scriptstyle \quad L_1 \leftarrow L_1 + 2L_2}\\
    & x_2 &     & -3x_4 &=& 3  & \\
    &     & x_3 & +5x_4 &=& 4  &
\end{array} \right. \]

Le système est sous forme échelonnée réduite.

\bigskip

\textbf{Partie C. Solutions.}
Le système est maintenant très simple à résoudre. En choisissant $x_4$ comme variable libre, on peut exprimer
$x_1,x_2,x_3$ en fonction de $x_4$ :
$$x_1=4x_4-2, \quad x_2 = 3x_4+3,\quad x_3=-5x_4+4.$$
Ce qui permet d'obtenir toutes les solutions du système :
$$\mathcal{S}= \big\lbrace (4x_4-2,3x_4+3,-5x_4+4,x_4) \mid x_4 \in \Rr \big\rbrace.$$


% A ne pas mettre dans les diapos
% \begin{remarque*}
% \begin{itemize}
%   \item Arrêtons nous quelque peu sur la notion d'algorithme. Il s'agit d'une description
% précise d'une suite d'opérations à effectuer, dans quel ordre et dans quel cas,
% qui aboutit au bout d'un nombre fini d'étapes si possible connu à l'avance au
% résultat voulu.
%
%   \item La première raison pour utiliser cet algorithme du pivot  Gauss est
%   que l'on peut certes résoudre les systèmes à $2$ ou $3$ inconnues par des
%   manipulations sur les équations menées au petit bonheur la chance et qui aboutissent
% à un résultat après un plus ou moins grand nombre d'opérations. Or l'expérience montre
% que ces opérations sont le plus souvent inutiles, redondantes, et surtout cause
% d'erreurs de calculs. Il est bien préférable de se laisser guider par une méthode
% stricte dont l'application garantit un nombre minimal de calculs (en général).
%
%   \item La seconde raison est que dans les applications pratiques de l'algèbre linéaire,
% lesquelles sont extrêmement nombreuses et importantes, les systèmes à résoudre
% sont énormes (des milliers, voire des millions d'équations et d'inconnues) et
% qu'il n'est pas question d'effectuer les calculs à la main. Ce sont des ordinateurs
% qui s'en chargent, et ces derniers ont besoin de programmes, lesquels sont la
% traduction en tel ou tel langage d'un algorithme.
% \end{itemize}
%
% \end{remarque*}




%---------------------------------------------------------------
\subsection{Systèmes homogènes}
\index{systeme@système!homogène}

Le fait que l'on puisse toujours se ramener à un système échelonné
réduit implique le résultat suivant :
\begin{theoreme}
Tout système homogène d'équations linéaires dont le nombre d'inconnues est
strictement plus grand que le nombre d'équations a une infinité de solutions.
\end{theoreme}

\begin{exemple} Considérons le système homogène
\[\left\{
 \begin{array}{ccccccccccc}
3x_1 &+ &3x_2 &- &2x_3 &&&- &x_5 & = & 0\\
 -x_1 &- &x_2 &+ &x_3 &+ &3x_4 &+ &x_5 & = & 0\\
 2x_1 &+ &2x_2 &- &x_3 &+ &2x_4 &+ &2x_5 & = & 0\\
 &&&&x_3 &+ &8x_4 &+ &4x_5 & = & 0.
\end{array} \right.
\] \end{exemple}

Sa forme échelonnée réduite est

\[ \left\{
\begin{array}{ccccccccc}
x_1 &+ &x_2 &&& + &13x_5 & = & 0\\
&&&x_3 & &+ &20x_5 & = & 0\\
&&&&x_4  &-&2x_5 & = & 0.
\end{array} \right.
\]

On pose comme variables libres $x_2$ et $x_5$
pour avoir
$$x_1 = -x_2-13x_5,\qquad x_3 = -20x_5, \qquad x_4 = 2x_5,$$
et l'ensemble des solutions :
$$\mathcal{S}= \big\lbrace (-x_2-13x_5, x_2,-20x_5, 2x_5, x_5) \mid x_2,x_5 \in \Rr \big\rbrace$$
qui est bien infini.

%---------------------------------------------------------------
%\subsection{Mini-exercices}

\begin{miniexercices}
\sauteligne
\begin{enumerate}
  \item \'Ecrire un système linéaire à $4$ équations et $5$ inconnues qui soit
  échelonné mais pas réduit. Idem avec échelonné, non réduit, dont tous les coefficients sont
  $0$ ou $+1$. Idem avec échelonné et réduit.

  \item Résoudre les systèmes échelonnés suivants :
 $\left \{ \begin{array}{cccccc}
2x_1 & -x_2&      & +x_4 &=& 1 \\
     & x_2 & +x_3 &-2x_4 &=& 3 \\
     &     & 2x_3 & +x_4 &=& 4 \\
     &     &      & x_4  &=& -2 \\
\end{array} \right.$\qquad
 $\left \{ \begin{array}{cccccc}
x_1 & +x_2&      & +x_4 &=& 0 \\
     & x_2 & +x_3 &     &=& 0 \\
     &     & 2x_3 & +x_4 &=& 0 \\
\end{array} \right.$ 

 $\left \{ \begin{array}{cccccc}
x_1 & +2x_2&      & +x_4   &=& 0 \\
     &     & 2x_3 & -3x_4 &=& 0 \\
\end{array} \right.$

  \item Si l'on passe d'un système $(S)$ par une des trois opérations élémentaires à un système
  $(S')$, alors quelle opération permet de passer de $(S')$ à $(S)$ ?

  \item Résoudre les systèmes linéaires suivants par la méthode du pivot de Gauss :
  $$\left\{
\begin{array}{ccccccc}
2x   &+&  y    &+&   z  &=&  3 \\
x    &-&  y    &+&   3z &=&  8 \\
x    &+&  2y   &-&   z  &=&  -3
\end{array}\right.$$
$$
\left\{
\begin{array}{ccccccccc}
2x_1 & + & 4x_2 & - & 6x_3 & - & 2x_4 & = & 2\\
3x_1 & + & 6x_2 & - & 7x_3 & + & 4x_4 &= & 2\\
5x_1 & + & 10 x_2 & - & 11 x_3 & + & 6x_4 &   = & 3
\end{array}
 \right.
$$


  \item Résoudre le système suivant, selon les valeurs de $a,b \in \Rr$ :
$$\left \{ \begin{array}{ccccc}
x  & +y &-z   &=& a \\
-x &    & +2z &=& b \\
   & 2y & +2z &=& 4 \\
\end{array} \right.$$

\end{enumerate}
\end{miniexercices}



%%%%%%%%%%%%%%%%%%%%%%%%%%%%%%%%%%%%%%%%%%%%%%%%%%%%%%%%%%%%%%%%
%%%%%%%%%%%%%%%%%%%%%%%%%%%%%%%%%%%%%%%%%%%%%%%%%%%%%%%%%%%%%%%%

\bigskip
\bigskip

\auteurs{
\begin{itemize}
  \item D'après un cours de Eva Bayer-Fluckiger, Philippe Chabloz, Lara Thomas
  de l'\'Ecole Polytechnique Fédérale de Lausanne,

  \item et un cours de Sophie Chemla de l'université Pierre et Marie Curie,
  reprenant des parties d'un cours de H. Ledret et d'une équipe de l'université de
  Bordeaux animée par J. Queyrut,

  \item mixés et révisés par Arnaud Bodin, relu par Vianney Combet.
\end{itemize}
}

\finchapitre
\end{document}
