
%%%%%%%%%%%%%%%%%% PREAMBULE %%%%%%%%%%%%%%%%%%


\documentclass[12pt]{article}

\usepackage{amsfonts,amsmath,amssymb,amsthm}
\usepackage[utf8]{inputenc}
\usepackage[T1]{fontenc}
\usepackage[francais]{babel}


% packages
\usepackage{amsfonts,amsmath,amssymb,amsthm}
\usepackage[utf8]{inputenc}
\usepackage[T1]{fontenc}
%\usepackage{lmodern}

\usepackage[francais]{babel}
\usepackage{fancybox}
\usepackage{graphicx}

\usepackage{float}

%\usepackage[usenames, x11names]{xcolor}
\usepackage{tikz}
\usepackage{datetime}

\usepackage{mathptmx}
%\usepackage{fouriernc}
%\usepackage{newcent}
\usepackage[mathcal,mathbf]{euler}

%\usepackage{palatino}
%\usepackage{newcent}


% Commande spéciale prompteur

%\usepackage{mathptmx}
%\usepackage[mathcal,mathbf]{euler}
%\usepackage{mathpple,multido}

\usepackage[a4paper]{geometry}
\geometry{top=2cm, bottom=2cm, left=1cm, right=1cm, marginparsep=1cm}

\newcommand{\change}{{\color{red}\rule{\textwidth}{1mm}\\}}

\newcounter{mydiapo}

\newcommand{\diapo}{\newpage
\hfill {\normalsize  Diapo \themydiapo \quad \texttt{[\jobname]}} \\
\stepcounter{mydiapo}}


%%%%%%% COULEURS %%%%%%%%%%

% Pour blanc sur noir :
%\pagecolor[rgb]{0.5,0.5,0.5}
% \pagecolor[rgb]{0,0,0}
% \color[rgb]{1,1,1}



%\DeclareFixedFont{\myfont}{U}{cmss}{bx}{n}{18pt}
\newcommand{\debuttexte}{
%%%%%%%%%%%%% FONTES %%%%%%%%%%%%%
\renewcommand{\baselinestretch}{1.5}
\usefont{U}{cmss}{bx}{n}
\bfseries

% Taille normale : commenter le reste !
%Taille Arnaud
%\fontsize{19}{19}\selectfont

% Taille Barbara
%\fontsize{21}{22}\selectfont

%Taille François
\fontsize{25}{30}\selectfont

%Taille Pascal
%\fontsize{25}{30}\selectfont

%Taille Laura
%\fontsize{30}{35}\selectfont


%\myfont
%\usefont{U}{cmss}{bx}{n}

%\Huge
%\addtolength{\parskip}{\baselineskip}
}


% \usepackage{hyperref}
% \hypersetup{colorlinks=true, linkcolor=blue, urlcolor=blue,
% pdftitle={Exo7 - Exercices de mathématiques}, pdfauthor={Exo7}}


%section
% \usepackage{sectsty}
% \allsectionsfont{\bf}
%\sectionfont{\color{Tomato3}\upshape\selectfont}
%\subsectionfont{\color{Tomato4}\upshape\selectfont}

%----- Ensembles : entiers, reels, complexes -----
\newcommand{\Nn}{\mathbb{N}} \newcommand{\N}{\mathbb{N}}
\newcommand{\Zz}{\mathbb{Z}} \newcommand{\Z}{\mathbb{Z}}
\newcommand{\Qq}{\mathbb{Q}} \newcommand{\Q}{\mathbb{Q}}
\newcommand{\Rr}{\mathbb{R}} \newcommand{\R}{\mathbb{R}}
\newcommand{\Cc}{\mathbb{C}} 
\newcommand{\Kk}{\mathbb{K}} \newcommand{\K}{\mathbb{K}}

%----- Modifications de symboles -----
\renewcommand{\epsilon}{\varepsilon}
\renewcommand{\Re}{\mathop{\text{Re}}\nolimits}
\renewcommand{\Im}{\mathop{\text{Im}}\nolimits}
%\newcommand{\llbracket}{\left[\kern-0.15em\left[}
%\newcommand{\rrbracket}{\right]\kern-0.15em\right]}

\renewcommand{\ge}{\geqslant}
\renewcommand{\geq}{\geqslant}
\renewcommand{\le}{\leqslant}
\renewcommand{\leq}{\leqslant}

%----- Fonctions usuelles -----
\newcommand{\ch}{\mathop{\mathrm{ch}}\nolimits}
\newcommand{\sh}{\mathop{\mathrm{sh}}\nolimits}
\renewcommand{\tanh}{\mathop{\mathrm{th}}\nolimits}
\newcommand{\cotan}{\mathop{\mathrm{cotan}}\nolimits}
\newcommand{\Arcsin}{\mathop{\mathrm{Arcsin}}\nolimits}
\newcommand{\Arccos}{\mathop{\mathrm{Arccos}}\nolimits}
\newcommand{\Arctan}{\mathop{\mathrm{Arctan}}\nolimits}
\newcommand{\Argsh}{\mathop{\mathrm{Argsh}}\nolimits}
\newcommand{\Argch}{\mathop{\mathrm{Argch}}\nolimits}
\newcommand{\Argth}{\mathop{\mathrm{Argth}}\nolimits}
\newcommand{\pgcd}{\mathop{\mathrm{pgcd}}\nolimits} 

\newcommand{\Card}{\mathop{\text{Card}}\nolimits}
\newcommand{\Ker}{\mathop{\text{Ker}}\nolimits}
\newcommand{\id}{\mathop{\text{id}}\nolimits}
\newcommand{\ii}{\mathrm{i}}
\newcommand{\dd}{\mathrm{d}}
\newcommand{\Vect}{\mathop{\text{Vect}}\nolimits}
\newcommand{\Mat}{\mathop{\mathrm{Mat}}\nolimits}
\newcommand{\rg}{\mathop{\text{rg}}\nolimits}
\newcommand{\tr}{\mathop{\text{tr}}\nolimits}
\newcommand{\ppcm}{\mathop{\text{ppcm}}\nolimits}

%----- Structure des exercices ------

\newtheoremstyle{styleexo}% name
{2ex}% Space above
{3ex}% Space below
{}% Body font
{}% Indent amount 1
{\bfseries} % Theorem head font
{}% Punctuation after theorem head
{\newline}% Space after theorem head 2
{}% Theorem head spec (can be left empty, meaning ‘normal’)

%\theoremstyle{styleexo}
\newtheorem{exo}{Exercice}
\newtheorem{ind}{Indications}
\newtheorem{cor}{Correction}


\newcommand{\exercice}[1]{} \newcommand{\finexercice}{}
%\newcommand{\exercice}[1]{{\tiny\texttt{#1}}\vspace{-2ex}} % pour afficher le numero absolu, l'auteur...
\newcommand{\enonce}{\begin{exo}} \newcommand{\finenonce}{\end{exo}}
\newcommand{\indication}{\begin{ind}} \newcommand{\finindication}{\end{ind}}
\newcommand{\correction}{\begin{cor}} \newcommand{\fincorrection}{\end{cor}}

\newcommand{\noindication}{\stepcounter{ind}}
\newcommand{\nocorrection}{\stepcounter{cor}}

\newcommand{\fiche}[1]{} \newcommand{\finfiche}{}
\newcommand{\titre}[1]{\centerline{\large \bf #1}}
\newcommand{\addcommand}[1]{}
\newcommand{\video}[1]{}

% Marge
\newcommand{\mymargin}[1]{\marginpar{{\small #1}}}



%----- Presentation ------
\setlength{\parindent}{0cm}

%\newcommand{\ExoSept}{\href{http://exo7.emath.fr}{\textbf{\textsf{Exo7}}}}

\definecolor{myred}{rgb}{0.93,0.26,0}
\definecolor{myorange}{rgb}{0.97,0.58,0}
\definecolor{myyellow}{rgb}{1,0.86,0}

\newcommand{\LogoExoSept}[1]{  % input : echelle
{\usefont{U}{cmss}{bx}{n}
\begin{tikzpicture}[scale=0.1*#1,transform shape]
  \fill[color=myorange] (0,0)--(4,0)--(4,-4)--(0,-4)--cycle;
  \fill[color=myred] (0,0)--(0,3)--(-3,3)--(-3,0)--cycle;
  \fill[color=myyellow] (4,0)--(7,4)--(3,7)--(0,3)--cycle;
  \node[scale=5] at (3.5,3.5) {Exo7};
\end{tikzpicture}}
}



\theoremstyle{definition}
%\newtheorem{proposition}{Proposition}
%\newtheorem{exemple}{Exemple}
%\newtheorem{theoreme}{Théorème}
\newtheorem{lemme}{Lemme}
\newtheorem{corollaire}{Corollaire}
%\newtheorem*{remarque*}{Remarque}
%\newtheorem*{miniexercice}{Mini-exercices}
%\newtheorem{definition}{Définition}




%definition d'un terme
\newcommand{\defi}[1]{{\color{myorange}\textbf{\emph{#1}}}}
\newcommand{\evidence}[1]{{\color{blue}\textbf{\emph{#1}}}}



 %----- Commandes divers ------

\newcommand{\codeinline}[1]{\texttt{#1}}

%%%%%%%%%%%%%%%%%%%%%%%%%%%%%%%%%%%%%%%%%%%%%%%%%%%%%%%%%%%%%
%%%%%%%%%%%%%%%%%%%%%%%%%%%%%%%%%%%%%%%%%%%%%%%%%%%%%%%%%%%%%



\begin{document}

\debuttexte


%%%%%%%%%%%%%%%%%%%%%%%%%%%%%%%%%%%%%%%%%%%%%%%%%%%%%%%%%%%
\diapo

Beaucoup de problèmes issus de la physique conduisent 
naturellement à une équation différentielle, nous allons
en voir quelques-uns :

\change

\change
Tout d'abord nous étudierons le mouvement d'un parachutiste,

\change
Nous étudierons la notion de demi-vie des noyaux radioactifs,

\change
Nous étudierons deux modèles d'évolutions de population de bactéries,

\change
et enfin comment se déplacent une masse attachée par un ressort.


%%%%%%%%%%%%%%%%%%%%%%%%%%%%%%%%%%%%%%%%%%%%%%%%%%%%%%%%%%%
\diapo

Reprenons l'exemple introductifs du chapitre,

on lâche un corps dans le ciel.
La seule force qui s'applique à l'objet est son poids $P$

\change
Le principe fondamental de la mécanique affirme alors que 
la somme des forces égale la masse fois l’accélération.

Le poids est la seule force et s'écrit $mg$ sur l'axe verticale,
l'accélération est la dérivée de la vitesse. On obtient
cette équation qu'il est facile d'intégrer pour trouver $v(t)$.

\change
Pour le cas d'un parchutiste il faut tenir compte des force de frottements,
qui s'oppose aux mouvement : 

\change
on suppose que le frottement est proportionnelle à la vitesse.

Le signe moins signifie que le frottement s'oppose au déplacement.

Le principe fondamental de la mécanique
conduit alors l'équation différentielle du premier ordre :
$$m\frac{\dd v(t)}{\dd t}= mg - fm v(t)$$


\change
et on on simplifie l'équation en divisant par la masse.

\change
$g$ est la constante de gravitation

\change
et $f$ le coefficient de frottement.

\change
Nous avons tous les ingrédients pour trouver la fonction $v(t)$.

\change
On commence  par résoudre l'équation homogène.
$v'(t)=-fv(t)$,

\change
on sait que les solutions sont les $v(t)= ke^{-ft}$, 
où $k$ est une constante réelle.

\change
On cherche une solution particulière 
de l'équation de départ $v'= g - f v$

\change
On la cherche sous la forme $v_p(t)=k(t)e^{-ft}$
   par la méthode de variation de la constante. 
   
 \change
 On calcule $v_p'(t)$
 
 \change
 Et lorsque l'on injecte $v_p$ et $v'_p$ dans l'équation différentielle
 on trouve que $v_p$ est solution de l'équation différentielle 
  ssi   $k'(t)e^{-ft}=g$. 
  
  
  \change
  Ainsi $k'(t)=ge^{ft}$ donc, par exemple, $k(t)=\frac{g}{f}e^{ft}$.
  
  \change
  On trouve tout simplement pour la solution particulière $v_p(t)$, la fonction constante 
  $=\frac{g}{f}$.
  
  
  \change
  Conclusion 
  La solution générale de l'équation est 
  $v(t)=\frac{g}{f}+k e^{-ft}$, $k\in\Rr$.
  "une solution particulière plus les solutions de l'équation homogène."

%%%%%%%%%%%%%%%%%%%%%%%%%%%%%%%%%%%%%%%%%%%%%%%%%%%%%%%%%%%
\diapo

Continuons avec l'équation du parachutiste.
On a trouvé une infinité de solutions

\change
  Si à l'instant $t=0$ le parachute se lance 
  avec une vitesse initiale nulle, c'est-à-dire
  $v(0)=0$, alors cela permet de déterminer la constante $k$ et
  l'unique solution  vitesse est :
  $v(t)=\frac{g}{f}-\frac{g}{f}e^{-ft}.$
  
\change
Dont voici le graphe.

A l'instant initial la vitesse est nulle, 

la vitesse croit,

\change
et tend vers une vitesse limite $v_\infty$
que le parachutiste ne peut dépasser


\change
qui par le calcul vaut $g/f$.

\change
  Expérimentalement, on mesure que $v_\infty$ 
  vaut environ 20 km/h,
  
  et comme on connait $g$  cela permet de calculer 
  le coefficient de frottement $f$.
  
  \change
  
Une fois que l'on a la vitesse il n'est pas dur d'en déduire la position.

\change
  Comme  $v(t) = \frac{\dd x(t)}{\dd t}$, 
  
  \change
  trouver la position $x$ revient à trouver 
  une primitive de $v$ :
  
  \change
  Ainsi 
  $$x(t) = \frac{g}{f} t + \frac{g}{f^2}(e^{-ft}-1)$$
  en prenant comme convention $x(0)=0$.


Ceci n'est bien sûr qu'un \evidence{modèle} qui ne correspond 
pas parfaitement à la réalité, mais permet de mettre en évidence
des propriétés vérifiées pas les conditions expérimentales, 
comme la vitesse limite par exemple.


%%%%%%%%%%%%%%%%%%%%%%%%%%%%%%%%%%%%%%%%%%%%%%%%%%%%%%%%%%%
\diapo

Voyons une nouvelle application.

\change
Dans un tissu radioactif la vitesse de désintégration 
des noyaux radioactifs est proportionnelle 
au nombre de noyaux radioactifs $N(t)$ présents dans le
tissus à l'instant $t$. 

"plus il y a de noyaux, plus ils se désintègrent vite"

\change
Il existe donc une constante $\lambda$ strictement positive telle que :
$$N'(t) = -\lambda N(t)$$


$\lambda$ dépend des paramètres du tissus.

Le signe \og$-$\fg\ de cette équation différentielle traduit la
décroissance du nombre de noyaux.

\change
On sait intégrer cette équation différentielle !

Si $N_0$ désigne le nombre de noyaux à l'instant initial, on a donc :
$$N(t) = N_0 e^{-\lambda t}$$

\change
Voici l'évolution du nombre de noyaux radioactifs,

leur nombre tend vers $0$, la décroissance est exponentielle.

; on va maintenant s'interesser
à savoir quand est-ce que la moitié des noyaux se sont désintégrés : c'est la période de demi-vie.


%%%%%%%%%%%%%%%%%%%%%%%%%%%%%%%%%%%%%%%%%%%%%%%%%%%%%%%%%%%
\diapo


Dans ce contexte apparaissent souvent une grandeur: la 
\defi{période de demi-vie}

\change
la \defi{période de demi-vie}, notée $\tau_{1/2}$, 
  est la période au bout de laquelle la moitié des noyaux se
  sont désintégrés. 
  
\change
Ainsi entre l'instant $t=0$ et $t=\tau_{1/2}$ on a la relation :
  $$N(\tau_{1/2}) = \frac{N_0}{2}$$
  
\change  
  Donc $N_0 e^{ -\lambda \tau_{1/2} } =   \frac{N_0}{2}$, d'où 
  $\lambda \tau_{1/2} = \ln 2$.
  
\change
  Ainsi :
  $$\tau_{1/ 2} = \frac{\ln 2}{\lambda}$$

  \change         
  On peut aussi exprimer $N(t)$ en fonction de la période de demi-vie, qui est un parmètre
  physique plus facilement mesurable :
  $N(t) = N_0 e^{-\lambda t}$
  
  \change
  $=  N_0 e^{-\frac{t}{\tau_{1/2}} \ln 2}$
  
  \change
  $= N_0 2^{-\frac{t}{\tau_{1/2}}}$
  
  
  \change
    Notez que $\tau_{1/ 2}$ ne dépend pas de $N_0$, et c'est bien le
    temps nécessaire pour que la moitié des noyaux se
  soient désintégrés, ce quel que soit l'instant initial :
  
  et pas seulement à partir de $t=0$.
  
  \change  
  Cela se vérifie par le calcul
  $N(t+\tau_{1/ 2})$
  
  \change
  $= N_0 2^{-\frac{t+\tau_{1/2}}{\tau_{1/2}}}$
  
  \change
  $= N_0 2^{-\frac{t}{\tau_{1/2}}-1}$
  
  \change
  $  = \frac12 N_0 2^{-\frac{t}{\tau_{1/2}}}$
  
  \change
  $= \frac{N(t)}{2}.$
  
  A l'instant $t+\tau_{1/2}$ on retrouve bien la moitié des noyau par rapport
  à l'instant $t$, ceci qq soit $t$.
  

%%%%%%%%%%%%%%%%%%%%%%%%%%%%%%%%%%%%%%%%%%%%%%%%%%%%%%%%%%%
\diapo

On continue avec des équations différentielles issues de la biologie.

\change
On considère une culture de bactéries en milieu clos.


\change
Soit $N_0$ le nombre de bactéries introduites 
dans la culture à l'instant initial $t = 0$.


\change
\change


Un premier modèle, appelé Loi de Malthus, 
est de supposer que comme les bactéries se reproduisent, 
alors plus il y a de bactéries plus il y a de naissances.
Ainsi la vitesse d'accroissement des bactéries 
est proportionnelle au nombre de bactéries en présence :


\change
Cela signifie que le nombre $N(t)$ de bactéries vérifie l'équation différentielle :
$$y' = ay$$

\change
où $a>0$ est une constante dépendant des conditions expérimentales.

\change
Nous savons résoudre cette équation ! Ainsi selon ce modèle
$$N(t) = N_0 e^{at}$$

\change
Le milieu étant limité (en volume, en éléments nutritifs,...), 
le nombre de bactéries ne peut pas croître indéfiniment 
de façon exponentielle. Ce modèle ne peut donc pas 
s'appliquer sur une longue période.


%%%%%%%%%%%%%%%%%%%%%%%%%%%%%%%%%%%%%%%%%%%%%%%%%%%%%%%%%%%
\diapo


Je vous présente un autre modèle d'évolution plus 
réaliste le modèle de Verhulst.


\change
On suppose que le nombre $N(t)$ de bactéries vérifie 
l'équation différentielle : 
\begin{equation}
  y' = ay(M - y)
  \label{eq:eqdiffverhulst}
  \tag{$E$}
\end{equation}
où $a>0$ et $M>0$ sont des constantes.

\change
On va chercher les solutions $y$, $>0$.


\change
On suppose qu'une telle solution $y$ existe.
et on va effectuer un changement de fonction,
afin de transformer notre équation en une équation 
plus facile à résoudre.

\change
  Pour cela on pose $z(x) = \frac{1}{y(x)}$
  
  \change
  La fonction $z$ est dérivable et :
  $$z'=-\frac{y'}{y^2} = \frac{ay(y-M)}{y^2} = a - \frac{aM}{y} = a-aMz$$

  \change
  \change
  Ainsi la fonction $z$ doit vérifier l'équation différentielle :
  $$z' = a-aMz$$  
  qui est tout simplement une équation différentielle linaire d'ordre $1$ 
  à coefficients constants avec second membre constant.
  
  \change
  On sait alors que 
  $$z(x) = k e^{-aMx} + \frac1M$$
  où $k \in \Rr$ est une constante

  \change
  \change
  Cela permet d'obtenir $y$ :
  $$y(x) = \frac{1}{z(x)} = \frac{1}{k e^{-aMx} + \frac1M} = \frac{M}{kM e^{-aMx}+1}$$
  
  \change
  La constante $k$ est déterminée par la condition initiale,
  en effet 
  $y(0) = \frac{M}{kM+1}=N_0$, 
  
  et ainsi $k = \frac{1}{N_0}-\frac{1}{M}$.
  

%%%%%%%%%%%%%%%%%%%%%%%%%%%%%%%%%%%%%%%%%%%%%%%%%%%%%%%%%%%
\diapo

Voyons un exemple.

\change
prenons $N_0 = 0,01$ (en million de bactéries) et $M=1$, $a=1$.

L'équation différentielle devient $y' = y(1 - y)$

\change
La constante $k$ de la solution est 
$k = \frac{1}{N_0}-\frac{1}{M} = 99$. 

\change
  Ainsi selon ce modèle :
  $$N(t) = \frac{1}{1 + 99 e^{-t}}$$ 
  
\change
Etudions plus en détails cette solution.

  Il est clair que $0<N(t)<1$, pour tout $t\ge0$ 
  
\change
et   $N(t) \to 1$ lorsque $t\to+\infty$.

\change   
  Pour connaître les variations de la fonction $N$, 
  nul besoin de calculs 
  

\change  
  car on sait déjà que 
  $N$ est solution de l'équation différentielle $y' = y(1 - y)$, donc
  $N'(t) = N(t)(1 - N(t))$.
  
  \change
  Ainsi $N'(t)>0$, donc la fonction $N$ est croissante.
  
  \change    
  Le second modèle de Verhulst ici en rouge a l'avantage de
bien faire apparaître un comportement asymptotique particulier :
le nombre de bactéries finit par se stabiliser. Alors qu'avec
le modèle de Malthus le nombre de bactéries croissait indéfiniment.

%%%%%%%%%%%%%%%%%%%%%%%%%%%%%%%%%%%%%%%%%%%%%%%%%%%%%%%%%%%
\diapo

Dernière application que nous allons voir des équations différentielles :
le mouvement d'une masse attachée à un ressort.

\change
Pour établir l'équation du mouvement nous allons d'abord chercher 
une équation différentielle, qui une nouvelle fois va découler
d'une étude des forces en présence.

\change
Il y a le poids $\vec P$,

\change
la force de réaction $\vec R=-\vec P$ qui s'oppose au poids,

\change
une force de rappel $\vec T$ à cause du ressort,

\change
et eventuellement une force de frottement $\vec F$.

\change
Le principe fondamental de la mécanique s'écrit :


\change
somme des forces = masse fois accélération :

$$\vec P + \vec R + \vec T + \vec F = m\vec a.$$

\change
Mais comme la réaction s'oppose au poids, on a une équation plus simple
$$\vec T + \vec F = m\vec a.$$




%%%%%%%%%%%%%%%%%%%%%%%%%%%%%%%%%%%%%%%%%%%%%%%%%%%%%%%%%%%
\diapo

Modélisons la force de rappel.

\change
La force de rappel est une force horizontale, 

elle est nulle 
à la position d'équilibre, qui sera pour nous l'origine $x=0$.

\change
Si on écarte davantage la masse du mur, 
la force de rappel est un vecteur horizontal qui pointe 
vers la position d'équilibre (vers la gauche sur le dessin).
Si on rapproche la masse du mur, le ressort se comprime,
et la force de rappel est un vecteur horizontal qui pointe 
encore vers la position d'équilibre (cette fois vers la droite 
sur le dessin). 

Dans tous les cas la force de rappel tend à ramener 
la masse vers la position d'équilibre.


\change
On modélise ceci par 
$$\vec T = -k x \vec i$$

\change
où $x$ est la position de la masse 
(on peut avoir $x\ge0$, ou $x\le 0$), 

\change
$k>0$ est une constante qui dépend du ressort.

%%%%%%%%%%%%%%%%%%%%%%%%%%%%%%%%%%%%%%%%%%%%%%%%%%%%%%%%%%%
\diapo

On commence par calculer le mouvement lorsqu'il n'y a pas frottements.

\change
$\vec F = \vec 0$.

\change
Le principe fondamental de la mécanique, 
se réduit à : la force de rappel = m fois l'accelaration.

Considéré uniquement sur l'axe horizontal il s'écrit :
$$-kx(t) = m\frac{\dd^2 x(t)}{\dd t^2}$$

\change
Il s'agit donc de résoudre l'équation différentielle du second ordre :
$$y'' + \frac{k}{m} y = 0.$$

\change
L'équation caractéristique est $r^2+\frac{k}{m} = 0$,

\change
Nous sommes dans le cas $\Delta <0$, les solutions sont les nombres complexes 
$r_1 = +\ii\sqrt{\frac{k}{m}}$ et $r_2 = -\ii\sqrt{\frac{k}{m}}$.

\change
ce qui fait que les solutions de l'équation différentielle sont
les :
$$y(x) = e^{\alpha x}\big(\lambda\cos (\beta x)+\mu\sin (\beta x)\big)$$

où $\alpha$ est la partie réelle de nos racines, donc ici est nul.
Et $\beta$ la partie imaginaire donc $\sqrt{\frac{k}{m}}$

\change
Dans notre situation (la fonction inconnue est $x$ et la variable $t$) :
$$x(t) = \lambda\cos \left(\sqrt{\tfrac{k}{m}}t\right)
+\mu\sin \left(\sqrt{\tfrac{k}{m}}t\right)
\qquad \lambda, \mu \in \Rr$$

%%%%%%%%%%%%%%%%%%%%%%%%%%%%%%%%%%%%%%%%%%%%%%%%%%%%%%%%%%%
\diapo

Par exemple 

\change
On lâche la masse au point d'abscisse $1$, sans vitesse initiale. 

\change
Cela nous donne les conditions initiales 
$x(0)=1$ et $x'(0)=0$.

\change
Comme $x(0)=1$ alors $\lambda=1$. 

\change
Et comme $x'(0) = 0$ alors $\mu = 0$.

\change
Ainsi on trouve une solution périodique :
$$x(t) = \cos \left(\sqrt{\tfrac{k}{m}}t\right)$$

qui oscille périodiquement autour de la position d'équilibre $x=0$.

%%%%%%%%%%%%%%%%%%%%%%%%%%%%%%%%%%%%%%%%%%%%%%%%%%%%%%%%%%%
\diapo

Voyons les oscillations avec faibles frottements.

\change
On rajoute maintenant une force de frottement qui s'oppose au déplacement 
et est proportionnelle à la vitesse.

\change
en terme d'équation  $\vec F =  -fm \frac{\dd x(t)}{\dd t}$
où $f$ est le coefficient de frottement.

\change
Le principe fondamental de la mécanique devient :
$$-kx(t)-fm \frac{\dd x(t)}{\dd t} = m\frac{\dd^2 x(t)}{\dd t^2}$$

\change
Il s'agit donc de résoudre l'équation différentielle que l'on noterait
$$y'' + f y' + \frac{k}{m} y = 0.$$  

\change
L'équation caractéristique est cette fois $r^2+fr+\frac{k}{m}=0$.

\change
Son discriminant est  $\Delta = f^2-4\frac{k}{m}$.
et comme le coefficient de frottement $f$ est faible, 
alors $\Delta$ est encore $<0$, comme dans le cas sans frottement.

\change
Les deux racines sont 
$r_1 = \alpha + \ii \beta$ et $r_2 = \alpha - \ii \beta$

\change
La partie réelle est $\alpha = -\frac{f}{2}$ et est non nulle


[petit $\delta$/grand $\Delta$]

la partie imaginaire est plus ou moins
$\beta = \frac{\delta}{2}$

où $\delta = \sqrt{|\Delta|}$ [montrer $\Delta$].

\change
On sait alors que les solutions de l'équation différentielle 
sont les 
$$y(x) = e^{\alpha x}\big(\lambda\cos (\beta x)+\mu\sin (\beta x)\big)$$

\change
Ce qui donne ici :
$$x(t) = e^{-\frac{f}{2} t}\left(\lambda\cos \left(\tfrac{\delta}{2}t\right)
+\mu\sin \left(\tfrac{\delta}{2}t\right)\right)\qquad \lambda, \mu \in \Rr$$


%%%%%%%%%%%%%%%%%%%%%%%%%%%%%%%%%%%%%%%%%%%%%%%%%%%%%%%%%%%
\diapo

Cette fois la solution n'est plus périodique, 
mais correspond à un mouvement 
oscillant amorti, qui tend vers la position d'équilibre $x=0$.


%%%%%%%%%%%%%%%%%%%%%%%%%%%%%%%%%%%%%%%%%%%%%%%%%%%%%%%%%%%
\diapo

\change



\end{document}
