
%%%%%%%%%%%%%%%%%% PREAMBULE %%%%%%%%%%%%%%%%%%

\documentclass[aspectratio=169,utf8]{beamer}
%\documentclass[aspectratio=169,handout]{beamer}

\usetheme{Boadilla}
%\usecolortheme{seahorse}
\usecolortheme[RGB={245,66,24}]{structure}
\useoutertheme{infolines}

% packages
\usepackage{amsfonts,amsmath,amssymb,amsthm}
\usepackage[utf8]{inputenc}
\usepackage[T1]{fontenc}
\usepackage{lmodern}

\usepackage[francais]{babel}
\usepackage{fancybox}
\usepackage{graphicx}

\usepackage{float}
\usepackage{xfrac}

%\usepackage[usenames, x11names]{xcolor}
\usepackage{tikz}
\usepackage{pgfplots}
\usepackage{datetime}



%-----  Package unités -----
\usepackage{siunitx}
\sisetup{locale = FR,detect-all,per-mode = symbol}

%\usepackage{mathptmx}
%\usepackage{fouriernc}
%\usepackage{newcent}
%\usepackage[mathcal,mathbf]{euler}

%\usepackage{palatino}
%\usepackage{newcent}
% \usepackage[mathcal,mathbf]{euler}



% \usepackage{hyperref}
% \hypersetup{colorlinks=true, linkcolor=blue, urlcolor=blue,
% pdftitle={Exo7 - Exercices de mathématiques}, pdfauthor={Exo7}}


%section
% \usepackage{sectsty}
% \allsectionsfont{\bf}
%\sectionfont{\color{Tomato3}\upshape\selectfont}
%\subsectionfont{\color{Tomato4}\upshape\selectfont}

%----- Ensembles : entiers, reels, complexes -----
\newcommand{\Nn}{\mathbb{N}} \newcommand{\N}{\mathbb{N}}
\newcommand{\Zz}{\mathbb{Z}} \newcommand{\Z}{\mathbb{Z}}
\newcommand{\Qq}{\mathbb{Q}} \newcommand{\Q}{\mathbb{Q}}
\newcommand{\Rr}{\mathbb{R}} \newcommand{\R}{\mathbb{R}}
\newcommand{\Cc}{\mathbb{C}} 
\newcommand{\Kk}{\mathbb{K}} \newcommand{\K}{\mathbb{K}}

%----- Modifications de symboles -----
\renewcommand{\epsilon}{\varepsilon}
\renewcommand{\Re}{\mathop{\text{Re}}\nolimits}
\renewcommand{\Im}{\mathop{\text{Im}}\nolimits}
%\newcommand{\llbracket}{\left[\kern-0.15em\left[}
%\newcommand{\rrbracket}{\right]\kern-0.15em\right]}

\renewcommand{\ge}{\geqslant}
\renewcommand{\geq}{\geqslant}
\renewcommand{\le}{\leqslant}
\renewcommand{\leq}{\leqslant}
\renewcommand{\epsilon}{\varepsilon}

%----- Fonctions usuelles -----
\newcommand{\ch}{\mathop{\text{ch}}\nolimits}
\newcommand{\sh}{\mathop{\text{sh}}\nolimits}
\renewcommand{\tanh}{\mathop{\text{th}}\nolimits}
\newcommand{\cotan}{\mathop{\text{cotan}}\nolimits}
\newcommand{\Arcsin}{\mathop{\text{arcsin}}\nolimits}
\newcommand{\Arccos}{\mathop{\text{arccos}}\nolimits}
\newcommand{\Arctan}{\mathop{\text{arctan}}\nolimits}
\newcommand{\Argsh}{\mathop{\text{argsh}}\nolimits}
\newcommand{\Argch}{\mathop{\text{argch}}\nolimits}
\newcommand{\Argth}{\mathop{\text{argth}}\nolimits}
\newcommand{\pgcd}{\mathop{\text{pgcd}}\nolimits} 


%----- Commandes divers ------
\newcommand{\ii}{\mathrm{i}}
\newcommand{\dd}{\text{d}}
\newcommand{\id}{\mathop{\text{id}}\nolimits}
\newcommand{\Ker}{\mathop{\text{Ker}}\nolimits}
\newcommand{\Card}{\mathop{\text{Card}}\nolimits}
\newcommand{\Vect}{\mathop{\text{Vect}}\nolimits}
\newcommand{\Mat}{\mathop{\text{Mat}}\nolimits}
\newcommand{\rg}{\mathop{\text{rg}}\nolimits}
\newcommand{\tr}{\mathop{\text{tr}}\nolimits}


%----- Structure des exercices ------

\newtheoremstyle{styleexo}% name
{2ex}% Space above
{3ex}% Space below
{}% Body font
{}% Indent amount 1
{\bfseries} % Theorem head font
{}% Punctuation after theorem head
{\newline}% Space after theorem head 2
{}% Theorem head spec (can be left empty, meaning ‘normal’)

%\theoremstyle{styleexo}
\newtheorem{exo}{Exercice}
\newtheorem{ind}{Indications}
\newtheorem{cor}{Correction}


\newcommand{\exercice}[1]{} \newcommand{\finexercice}{}
%\newcommand{\exercice}[1]{{\tiny\texttt{#1}}\vspace{-2ex}} % pour afficher le numero absolu, l'auteur...
\newcommand{\enonce}{\begin{exo}} \newcommand{\finenonce}{\end{exo}}
\newcommand{\indication}{\begin{ind}} \newcommand{\finindication}{\end{ind}}
\newcommand{\correction}{\begin{cor}} \newcommand{\fincorrection}{\end{cor}}

\newcommand{\noindication}{\stepcounter{ind}}
\newcommand{\nocorrection}{\stepcounter{cor}}

\newcommand{\fiche}[1]{} \newcommand{\finfiche}{}
\newcommand{\titre}[1]{\centerline{\large \bf #1}}
\newcommand{\addcommand}[1]{}
\newcommand{\video}[1]{}

% Marge
\newcommand{\mymargin}[1]{\marginpar{{\small #1}}}

\def\noqed{\renewcommand{\qedsymbol}{}}


%----- Presentation ------
\setlength{\parindent}{0cm}

%\newcommand{\ExoSept}{\href{http://exo7.emath.fr}{\textbf{\textsf{Exo7}}}}

\definecolor{myred}{rgb}{0.93,0.26,0}
\definecolor{myorange}{rgb}{0.97,0.58,0}
\definecolor{myyellow}{rgb}{1,0.86,0}

\newcommand{\LogoExoSept}[1]{  % input : echelle
{\usefont{U}{cmss}{bx}{n}
\begin{tikzpicture}[scale=0.1*#1,transform shape]
  \fill[color=myorange] (0,0)--(4,0)--(4,-4)--(0,-4)--cycle;
  \fill[color=myred] (0,0)--(0,3)--(-3,3)--(-3,0)--cycle;
  \fill[color=myyellow] (4,0)--(7,4)--(3,7)--(0,3)--cycle;
  \node[scale=5] at (3.5,3.5) {Exo7};
\end{tikzpicture}}
}


\newcommand{\debutmontitre}{
  \author{} \date{} 
  \thispagestyle{empty}
  \hspace*{-10ex}
  \begin{minipage}{\textwidth}
    \titlepage  
  \vspace*{-2.5cm}
  \begin{center}
    \LogoExoSept{2.5}
  \end{center}
  \end{minipage}

  \vspace*{-0cm}
  
  % Astuce pour que le background ne soit pas discrétisé lors de la conversion pdf -> png
\begin{tikzpicture}
        \fill[opacity=0,green!60!black] (0,0)--++(0,0)--++(0,0)--++(0,0)--cycle; 
\end{tikzpicture}

% toc S'affiche trop tot :
% \tableofcontents[hideallsubsections, pausesections]
}

\newcommand{\finmontitre}{
  \end{frame}
  \setcounter{framenumber}{0}
} % ne marche pas pour une raison obscure

%----- Commandes supplementaires ------

% \usepackage[landscape]{geometry}
% \geometry{top=1cm, bottom=3cm, left=2cm, right=10cm, marginparsep=1cm
% }
% \usepackage[a4paper]{geometry}
% \geometry{top=2cm, bottom=2cm, left=2cm, right=2cm, marginparsep=1cm
% }

%\usepackage{standalone}


% New command Arnaud -- november 2011
\setbeamersize{text margin left=24ex}
% si vous modifier cette valeur il faut aussi
% modifier le decalage du titre pour compenser
% (ex : ici =+10ex, titre =-5ex

\theoremstyle{definition}
%\newtheorem{proposition}{Proposition}
%\newtheorem{exemple}{Exemple}
%\newtheorem{theoreme}{Théorème}
%\newtheorem{lemme}{Lemme}
%\newtheorem{corollaire}{Corollaire}
%\newtheorem*{remarque*}{Remarque}
%\newtheorem*{miniexercice}{Mini-exercices}
%\newtheorem{definition}{Définition}

% Commande tikz
\usetikzlibrary{calc}
\usetikzlibrary{patterns,arrows}
\usetikzlibrary{matrix}
\usetikzlibrary{fadings} 

%definition d'un terme
\newcommand{\defi}[1]{{\color{myorange}\textbf{\emph{#1}}}}
\newcommand{\evidence}[1]{{\color{blue}\textbf{\emph{#1}}}}
\newcommand{\assertion}[1]{\emph{\og#1\fg}}  % pour chapitre logique
%\renewcommand{\contentsname}{Sommaire}
\renewcommand{\contentsname}{}
\setcounter{tocdepth}{2}



%------ Figures ------

\def\myscale{1} % par défaut 
\newcommand{\myfigure}[2]{  % entrée : echelle, fichier figure
\def\myscale{#1}
\begin{center}
\footnotesize
{#2}
\end{center}}


%------ Encadrement ------

\usepackage{fancybox}


\newcommand{\mybox}[1]{
\setlength{\fboxsep}{7pt}
\begin{center}
\shadowbox{#1}
\end{center}}

\newcommand{\myboxinline}[1]{
\setlength{\fboxsep}{5pt}
\raisebox{-10pt}{
\shadowbox{#1}
}
}

%--------------- Commande beamer---------------
\newcommand{\beameronly}[1]{#1} % permet de mettre des pause dans beamer pas dans poly


\setbeamertemplate{navigation symbols}{}
\setbeamertemplate{footline}  % tiré du fichier beamerouterinfolines.sty
{
  \leavevmode%
  \hbox{%
  \begin{beamercolorbox}[wd=.333333\paperwidth,ht=2.25ex,dp=1ex,center]{author in head/foot}%
    % \usebeamerfont{author in head/foot}\insertshortauthor%~~(\insertshortinstitute)
    \usebeamerfont{section in head/foot}{\bf\insertshorttitle}
  \end{beamercolorbox}%
  \begin{beamercolorbox}[wd=.333333\paperwidth,ht=2.25ex,dp=1ex,center]{title in head/foot}%
    \usebeamerfont{section in head/foot}{\bf\insertsectionhead}
  \end{beamercolorbox}%
  \begin{beamercolorbox}[wd=.333333\paperwidth,ht=2.25ex,dp=1ex,right]{date in head/foot}%
    % \usebeamerfont{date in head/foot}\insertshortdate{}\hspace*{2em}
    \insertframenumber{} / \inserttotalframenumber\hspace*{2ex} 
  \end{beamercolorbox}}%
  \vskip0pt%
}


\definecolor{mygrey}{rgb}{0.5,0.5,0.5}
\setlength{\parindent}{0cm}
%\DeclareTextFontCommand{\helvetica}{\fontfamily{phv}\selectfont}

% background beamer
\definecolor{couleurhaut}{rgb}{0.85,0.9,1}  % creme
\definecolor{couleurmilieu}{rgb}{1,1,1}  % vert pale
\definecolor{couleurbas}{rgb}{0.85,0.9,1}  % blanc
\setbeamertemplate{background canvas}[vertical shading]%
[top=couleurhaut,middle=couleurmilieu,midpoint=0.4,bottom=couleurbas] 
%[top=fondtitre!05,bottom=fondtitre!60]



\makeatletter
\setbeamertemplate{theorem begin}
{%
  \begin{\inserttheoremblockenv}
  {%
    \inserttheoremheadfont
    \inserttheoremname
    \inserttheoremnumber
    \ifx\inserttheoremaddition\@empty\else\ (\inserttheoremaddition)\fi%
    \inserttheorempunctuation
  }%
}
\setbeamertemplate{theorem end}{\end{\inserttheoremblockenv}}

\newenvironment{theoreme}[1][]{%
   \setbeamercolor{block title}{fg=structure,bg=structure!40}
   \setbeamercolor{block body}{fg=black,bg=structure!10}
   \begin{block}{{\bf Th\'eor\`eme }#1}
}{%
   \end{block}%
}


\newenvironment{proposition}[1][]{%
   \setbeamercolor{block title}{fg=structure,bg=structure!40}
   \setbeamercolor{block body}{fg=black,bg=structure!10}
   \begin{block}{{\bf Proposition }#1}
}{%
   \end{block}%
}

\newenvironment{corollaire}[1][]{%
   \setbeamercolor{block title}{fg=structure,bg=structure!40}
   \setbeamercolor{block body}{fg=black,bg=structure!10}
   \begin{block}{{\bf Corollaire }#1}
}{%
   \end{block}%
}

\newenvironment{mydefinition}[1][]{%
   \setbeamercolor{block title}{fg=structure,bg=structure!40}
   \setbeamercolor{block body}{fg=black,bg=structure!10}
   \begin{block}{{\bf Définition} #1}
}{%
   \end{block}%
}

\newenvironment{lemme}[0]{%
   \setbeamercolor{block title}{fg=structure,bg=structure!40}
   \setbeamercolor{block body}{fg=black,bg=structure!10}
   \begin{block}{\bf Lemme}
}{%
   \end{block}%
}

\newenvironment{remarque}[1][]{%
   \setbeamercolor{block title}{fg=black,bg=structure!20}
   \setbeamercolor{block body}{fg=black,bg=structure!5}
   \begin{block}{Remarque #1}
}{%
   \end{block}%
}


\newenvironment{exemple}[1][]{%
   \setbeamercolor{block title}{fg=black,bg=structure!20}
   \setbeamercolor{block body}{fg=black,bg=structure!5}
   \begin{block}{{\bf Exemple }#1}
}{%
   \end{block}%
}


\newenvironment{miniexercice}[0]{%
   \setbeamercolor{block title}{fg=structure,bg=structure!20}
   \setbeamercolor{block body}{fg=black,bg=structure!5}
   \begin{block}{Mini-exercices}
}{%
   \end{block}%
}


\newenvironment{tp}[0]{%
   \setbeamercolor{block title}{fg=structure,bg=structure!40}
   \setbeamercolor{block body}{fg=black,bg=structure!10}
   \begin{block}{\bf Travaux pratiques}
}{%
   \end{block}%
}
\newenvironment{exercicecours}[1][]{%
   \setbeamercolor{block title}{fg=structure,bg=structure!40}
   \setbeamercolor{block body}{fg=black,bg=structure!10}
   \begin{block}{{\bf Exercice }#1}
}{%
   \end{block}%
}
\newenvironment{algo}[1][]{%
   \setbeamercolor{block title}{fg=structure,bg=structure!40}
   \setbeamercolor{block body}{fg=black,bg=structure!10}
   \begin{block}{{\bf Algorithme}\hfill{\color{gray}\texttt{#1}}}
}{%
   \end{block}%
}


\setbeamertemplate{proof begin}{
   \setbeamercolor{block title}{fg=black,bg=structure!20}
   \setbeamercolor{block body}{fg=black,bg=structure!5}
   \begin{block}{{\footnotesize Démonstration}}
   \footnotesize
   \smallskip}
\setbeamertemplate{proof end}{%
   \end{block}}
\setbeamertemplate{qed symbol}{\openbox}


\makeatother
\usecolortheme[RGB={51,102,51}]{structure}

% Commande spécifique à ce chapitre
\newcommand{\alenvers}[1]{\rotatebox[origin=c]{180}{#1}}

%%%%%%%%%%%%%%%%%%%%%%%%%%%%%%%%%%%%%%%%%%%%%%%%%%%%%%%%%%%%%
%%%%%%%%%%%%%%%%%%%%%%%%%%%%%%%%%%%%%%%%%%%%%%%%%%%%%%%%%%%%%


\begin{document}


\title{{\bf \'Equations différentielles}}
\subtitle{Définition}

\begin{frame}
  
  \debutmontitre

  \pause

{\footnotesize
\hfill
\setbeamercovered{transparent=50}
\begin{minipage}{0.6\textwidth}
  \begin{itemize}
    \item<3-> Motivation
    \item<4-> Introduction
    \item<5-> Définition
    \item<6-> \'Equation différentielle linéaire
  \end{itemize}
\end{minipage}
}

\end{frame}

\setcounter{framenumber}{0}

%%%%%%%%%%%%%%%%%%%%%%%%%%%%%%%%%%%%%%%%%%%%%%%%%%%%%%%%%%%%%%%%
\section*{Motivation}

\begin{frame}

\begin{minipage}{0.59\textwidth}
\uncover<3->{Un parachutiste est soumis a deux forces }
\begin{itemize}
  \item \uncover<2->{Poids $\vec{P}$}
\uncover<4->{  \begin{itemize}
    \item force verticale 
    \item $P = mg$
  \end{itemize}
}
\uncover<3->{  \item Force de frottement $\vec{F}$}
\uncover<5->{  \begin{itemize}
    \item opposée à sa vitesse
    \item $F = -f m v$
  \end{itemize}
}
\end{itemize}  
\end{minipage}
\begin{minipage}{0.39\textwidth}
\myfigure{0.9}{
    \begin{tikzpicture}


  \draw[->,>=latex,thick,gray] (-0.5,1.5) -- (-0.5,-2.5) node[above right,black] {$x$};
  \draw[thick,gray] (-0.6,0)--(-0.4,0) node[left=5pt] {$0$};
  \coordinate (P) at (1,0) ;

  \draw[->,>=latex, ultra thick, green!60!black] (P)-- + (0,-2)node[midway, right,  black] {$\vec P$};

  \beameronly{\uncover<3->}{
  \draw[->,>=latex, ultra thick, green!60!black] (P)-- + (0,1)node[midway, right,  black] {$\vec F$};
 \draw (P)--+(60:1.5);
 \draw (P)--+(80:1.5);
 \draw (P)--+(100:1.5);
 \draw (P)--+(120:1.5);

  \draw[ultra thick, color=myred]
  ($(P)+(60:1.5)$) .. controls +(-0.1,0.4)  and +(0.1,0.4) ..
  ($(P)+(80:1.5)$) .. controls +(-0.1,0.4)  and +(0.1,0.4)  ..
  ($(P)+(100:1.5)$) .. controls +(-0.1,0.4)  and +(0.1,0.4) ..
  ($(P)+(120:1.5)$);
  }

  \fill[red]  (P) circle (5pt);
\end{tikzpicture}

  } 
\end{minipage}

\pause\pause\pause\pause\pause
\vspace*{-1ex}
\begin{itemize}[<+->]
  \item Principe fondamental de la mécanique $\vec{P}+\vec{F} = m\vec{a}$
  \item $mg - fmv = ma$
  \item \myboxinline{$\frac{\dd v(t)}{\dd t} = g - fv(t)$}
  \item C'est une relation entre la fonction vitesse $v$ et sa dérivée 
  \item C'est une \evidence{équation différentielle}
  \item Résoudre l'équation différentielle c'est trouver $v(t)$
  \item Comme $v(t) = \frac{\dd x(t)}{\dd t}$ en déduire $x(t)$
\end{itemize}

\end{frame}





%%%%%%%%%%%%%%%%%%%%%%%%%%%%%%%%%%%%%%%%%%%%%%%%%%%%%%%%%%%%%%%%
\section*{Introduction}

\begin{frame}
Une \evidence{équation différentielle} est une équation :
\pause
\begin{itemize}
\item dont l'inconnue est une fonction 
(généralement notée $y(x)$ ou simplement $y$) 

\pause

\item dans laquelle apparaissent certaines des dérivées de la
  fonction (dérivée première $y'$, 
  ou dérivées d'ordres supérieurs $y''$, $y^{(3)},\ldots$)
\end{itemize}
\end{frame}


\begin{frame}

\begin{exemple}
De tête, trouver au moins une fonction, solution des 
équations différentielles suivantes :
$$\begin{array}{lr}
y' = \sin x & \qquad \qquad \uncover<2->{\alenvers{$y = -\cos x + k \quad\text{ où } k \in \Rr$}} \\
y' = 1 + e^x & \qquad \qquad \uncover<2->{\alenvers{$y = x + e^x + k \quad\text{ où } k \in \Rr$}} \\
y' = y & \qquad \qquad \uncover<2->{\alenvers{$y = k e^x \quad\text{ où } k \in \Rr$}} \\
y' = 3y & \qquad \qquad \uncover<2->{\alenvers{$y = k e^{3x} \quad\text{ où } k \in \Rr$}} \\
y'' = \cos x & \qquad \qquad \uncover<2->{\alenvers{$y = -\cos x + ax + b \quad\text{ où } a, b \in \Rr$}} \\
y'' = y & \qquad \qquad \uncover<2->{\alenvers{$y = a e^x + b e^{-x} \quad\text{ où }  a, b \in \Rr$}}  \\  
\end{array}$$
\end{exemple}
\end{frame}


\begin{frame}

\begin{exemple}
\begin{enumerate}
  \item $y' = 2xy+4x$
  
  \pause
  
  Vérifier que $y(x)=k\exp(x^2)-2$ est une solution sur $\Rr$, 
  pour tout $k\in \Rr$

  \pause
  
  \item $x^2y''-2y+2x=0$
  
  \pause
  
  Vérifier que $y(x)=kx^2+x$ est une solution sur $\Rr$, 
  pour tout $k \in \Rr$
\end{enumerate}
\end{exemple}

\end{frame}



%%%%%%%%%%%%%%%%%%%%%%%%%%%%%%%%%%%%%%%%%%%%%%%%%%%%%%%%%%%%%%%%
\section*{Définition}

\begin{frame}

\begin{mydefinition}
\begin{itemize}
  \item Une \defi{équation différentielle} d'ordre $n$ est une équation de la forme
  \begin{equation}
    F\left(x,y,y',\dots ,y^{(n)}\right)=0 
    \label{eq:eqdiff}
    \tag{$E$}
  \end{equation}
  où $F$ est une fonction de $(n+2)$ variables
  
  \pause
  
  \item Une \defi{solution} d'une telle équation sur un intervalle $I\subset \Rr$ 
  est une fonction $y :I \to \Rr$ :
  \begin{itemize}
    \item qui est $n$ fois dérivable 
    \item et qui vérifie l'équation (\ref{eq:eqdiff})
  \end{itemize}
\end{itemize}
\end{mydefinition}
\end{frame}


\begin{frame}

\begin{itemize}
  \item Notation : $y$ au lieu de $y(x)$, 
  $y'$ au lieu $y'(x)$,\ldots
  \pause
  
  Exemple : \og$y' = \sin x$\fg\ signifie \og$y'(x) = \sin x$\fg
  
    \pause
  \item Changement de nom pour les fonctions et les variables 
  
  \pause
  $(x'')^3+t(x')^3+(\sin t) x^4=e^t$
  l'inconnue est une fonction $x$ qui dépend de la variable $t$ :
  $(x''(t))^3+t(x'(t))^3+(\sin t) (x(t))^4=e^t$

  \pause
  
  \item Rechercher une primitive c'est résoudre $y'= f(x)$
   
  \pause
  
  \item La notion d'intervalle est fondamentale : 
  si on change d'intervalle, on peut très bien obtenir d'autres solutions
  
  \pause
  
  \begin{itemize}
    \item Sur $I_1 = ]0, +\infty[$, $y' = 1/x$ a pour solutions $y(x) = \ln(x) + k$
    \pause
    \item Sur $I_2 = ]-\infty, 0[$, $y' = 1/x$ a pour solutions $y(x) = \ln(-x) + k$
  \end{itemize} 
  
  \pause
  \item Si aucune précision n'est donnée, il s'agit de $I = \Rr$
\end{itemize}  
\end{frame}


\begin{frame}

\evidence{\'Equation à variables séparées}
\pause
$$y'=g(x)/f(y) \pause\qquad \text{ ou } \qquad y'f(y)=g(x)$$
\pause
\vspace*{-3ex}
\begin{itemize}
  \item $G(x)$ primitive de $g(x)$ \pause : $G'(x)=g(x)$
  \pause
  \item $F(x)$ primitive de $f(x)$ \pause : $\big(F(y(x))\big)' = y'(x) F'(y(x)) \pause = y' f(y)$
\end{itemize}
\pause
$$y'f(y)=g(x)
\pause
\iff \big(F(y(x))\big)' = G'(x)
\pause
\iff F(y(x))=G(x)+c$$

\pause
\vspace*{-2ex}

\begin{exemple}

$$x^2y' = e^{-y}$$
\pause\vspace*{-3ex}
\begin{itemize}
  \item On sépare les variables $y'e^{y} = \frac{1}{x^2}$
  \pause
  \item On intègre des deux côtés $e^{y} = -\frac{1}{x}+c$, $c\in\Rr$
  \pause
  \item Solution $y(x) = \ln\left(-\frac{1}{x}+c\right)$
  \pause
  \item Intervalle $I$ où solution est définie et dérivable dépend de la constante $c$ :
  il faut $x\neq0$ et $-\frac{1}{x}+c>0$
  \pause
  \item si $c<0$, $I=\,]\frac1c,0[$  ;  si $c=0$, $I=\,]-\infty,0[$ ;  si $c>0$, $I=\,]\frac1c,+\infty[$
\end{itemize}

\end{exemple}

\end{frame}



%%%%%%%%%%%%%%%%%%%%%%%%%%%%%%%%%%%%%%%%%%%%%%%%%%%%%%%%%%%%%%%%
\section*{\'Equation différentielle linéaire}


\begin{frame}

\begin{itemize}
  \item Une équation différentielle d'ordre $n$ est \defi{linéaire} si elle est de la forme
$$a_0(x)y+a_1(x)y'+\dots +a_n(x)y^{(n)} = g(x)$$
où les $a_i$ et $g$ sont des fonctions réelles continues sur un intervalle $I\subset \Rr$
\pause
  
  \item Une équation différentielle linéaire est \defi{homogène}, ou \defi{sans second membre},
  si la fonction $g$ est la fonction nulle :
  $$a_0(x)y+a_1(x)y'+\dots +a_n(x)y^{(n)} = 0$$
\pause  
  \item Une équation différentielle linéaire est \defi{à coefficients constants} si
  $$a_0y+a_1y'+\dots +a_ny^{(n)} = g(x)$$
  où les $a_i$ sont des constantes réelles et $g$ une fonction continue
\end{itemize}
\end{frame}


\begin{frame}
\begin{exemple}
\begin{enumerate}
  \item $y' + 5xy = e^x$ est une équation différentielle linéaire du premier ordre avec second membre
\pause  
  \item $y' + 5xy = 0$ est l'équation différentielle homogène associée à la précédente
\pause  
  \item $2y'' - 3y' + 5y = 0$ est une équation différentielle linéaire 
  du second ordre à coefficients constants,
  sans second membre
\pause    
  \item $y'^2 - y = x$ ou $y'' \cdot  y' - y = 0$ \textbf{ne sont pas} 
  des équations différentielles linéaires 
\end{enumerate}
\end{exemple}
\end{frame}



\begin{frame}

\begin{proposition}[Principe de linéarité]  
Si $y_1$ et $y_2$ sont solutions de l'équation différentielle linéaire homogène
\begin{equation}
  a_0(x)y+a_1(x)y'+\dots +a_n(x)y^{(n)} = 0
  \label{eq:eqdifflin}
 \tag{$E_0$}
\end{equation}
alors, quels que soient $\lambda,\mu \in \Rr$, $\lambda y_1 + \mu y_2$ est aussi solution de cette équation
\end{proposition}

\end{frame}


\begin{frame}

Pour résoudre une équation différentielle linéaire avec second membre 
\begin{equation}
  a_0(x)y+a_1(x)y'+\dots +a_n(x)y^{(n)} = g(x)
  \label{eq:eqdifflinscnd}
 \tag{$E$}
\end{equation}

\pause
\begin{enumerate}
  \item on trouve une solution particulière $y_0$ de l'équation (\ref{eq:eqdifflinscnd})
  \pause
  \item on trouve l'ensemble $\mathcal{S}_h$ des solutions $y$ de l'équation homogène associée
\begin{equation}
  a_0(x)y+a_1(x)y'+\dots +a_n(x)y^{(n)} = 0
  \label{eq:eqdifflin}
 \tag{$E_0$}
\end{equation}  
\end{enumerate}

\pause

\begin{proposition}[Principe de superposition]
L'ensemble des solutions $\mathcal{S}$ de (\ref{eq:eqdifflinscnd}) est formé 
des 
$$y_0 + y \quad \text{ avec } \quad  y \in \mathcal{S}_h$$
\end{proposition}

\end{frame}





%%%%%%%%%%%%%%%%%%%%%%%%%%%%%%%%%%%%%%%%%%%%%%%%%%%%%%%%%%%%%%%
 \section*{Mini-exercices}

\begin{frame}
\begin{miniexercice}
\begin{enumerate}
  \item Chercher une solution \og simple \fg\ de l'équation différentielle
  $y'=2y$. Même question avec $y''=-y$ ; $y''+\cos(2x)=0$ ; $xy''=y'$.
  
  \item Résoudre l'équation différentielle à variables séparées $y'y^2=x$.
  Même question avec $y'= y \ln x$ ; $y' = \frac{1}{y^n}$ ($n\ge1$).
  
  \item Soit l'équation $y' = y(1 - y)$. Montrer que si 
  $y$ est une solution non nulle de cette équation, alors $z = 2y$
  n'est pas solution. Que peut-on en conclure ?

\end{enumerate}
\end{miniexercice}
\end{frame}




\end{document}
