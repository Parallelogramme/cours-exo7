
%%%%%%%%%%%%%%%%%% PREAMBULE %%%%%%%%%%%%%%%%%%

\documentclass[aspectratio=169,utf8]{beamer}
%\documentclass[aspectratio=169,handout]{beamer}

\usetheme{Boadilla}
%\usecolortheme{seahorse}
\usecolortheme[RGB={245,66,24}]{structure}
\useoutertheme{infolines}

% packages
\usepackage{amsfonts,amsmath,amssymb,amsthm}
\usepackage[utf8]{inputenc}
\usepackage[T1]{fontenc}
\usepackage{lmodern}

\usepackage[francais]{babel}
\usepackage{fancybox}
\usepackage{graphicx}

\usepackage{float}
\usepackage{xfrac}

%\usepackage[usenames, x11names]{xcolor}
\usepackage{tikz}
\usepackage{pgfplots}
\usepackage{datetime}



%-----  Package unités -----
\usepackage{siunitx}
\sisetup{locale = FR,detect-all,per-mode = symbol}

%\usepackage{mathptmx}
%\usepackage{fouriernc}
%\usepackage{newcent}
%\usepackage[mathcal,mathbf]{euler}

%\usepackage{palatino}
%\usepackage{newcent}
% \usepackage[mathcal,mathbf]{euler}



% \usepackage{hyperref}
% \hypersetup{colorlinks=true, linkcolor=blue, urlcolor=blue,
% pdftitle={Exo7 - Exercices de mathématiques}, pdfauthor={Exo7}}


%section
% \usepackage{sectsty}
% \allsectionsfont{\bf}
%\sectionfont{\color{Tomato3}\upshape\selectfont}
%\subsectionfont{\color{Tomato4}\upshape\selectfont}

%----- Ensembles : entiers, reels, complexes -----
\newcommand{\Nn}{\mathbb{N}} \newcommand{\N}{\mathbb{N}}
\newcommand{\Zz}{\mathbb{Z}} \newcommand{\Z}{\mathbb{Z}}
\newcommand{\Qq}{\mathbb{Q}} \newcommand{\Q}{\mathbb{Q}}
\newcommand{\Rr}{\mathbb{R}} \newcommand{\R}{\mathbb{R}}
\newcommand{\Cc}{\mathbb{C}} 
\newcommand{\Kk}{\mathbb{K}} \newcommand{\K}{\mathbb{K}}

%----- Modifications de symboles -----
\renewcommand{\epsilon}{\varepsilon}
\renewcommand{\Re}{\mathop{\text{Re}}\nolimits}
\renewcommand{\Im}{\mathop{\text{Im}}\nolimits}
%\newcommand{\llbracket}{\left[\kern-0.15em\left[}
%\newcommand{\rrbracket}{\right]\kern-0.15em\right]}

\renewcommand{\ge}{\geqslant}
\renewcommand{\geq}{\geqslant}
\renewcommand{\le}{\leqslant}
\renewcommand{\leq}{\leqslant}
\renewcommand{\epsilon}{\varepsilon}

%----- Fonctions usuelles -----
\newcommand{\ch}{\mathop{\text{ch}}\nolimits}
\newcommand{\sh}{\mathop{\text{sh}}\nolimits}
\renewcommand{\tanh}{\mathop{\text{th}}\nolimits}
\newcommand{\cotan}{\mathop{\text{cotan}}\nolimits}
\newcommand{\Arcsin}{\mathop{\text{arcsin}}\nolimits}
\newcommand{\Arccos}{\mathop{\text{arccos}}\nolimits}
\newcommand{\Arctan}{\mathop{\text{arctan}}\nolimits}
\newcommand{\Argsh}{\mathop{\text{argsh}}\nolimits}
\newcommand{\Argch}{\mathop{\text{argch}}\nolimits}
\newcommand{\Argth}{\mathop{\text{argth}}\nolimits}
\newcommand{\pgcd}{\mathop{\text{pgcd}}\nolimits} 


%----- Commandes divers ------
\newcommand{\ii}{\mathrm{i}}
\newcommand{\dd}{\text{d}}
\newcommand{\id}{\mathop{\text{id}}\nolimits}
\newcommand{\Ker}{\mathop{\text{Ker}}\nolimits}
\newcommand{\Card}{\mathop{\text{Card}}\nolimits}
\newcommand{\Vect}{\mathop{\text{Vect}}\nolimits}
\newcommand{\Mat}{\mathop{\text{Mat}}\nolimits}
\newcommand{\rg}{\mathop{\text{rg}}\nolimits}
\newcommand{\tr}{\mathop{\text{tr}}\nolimits}


%----- Structure des exercices ------

\newtheoremstyle{styleexo}% name
{2ex}% Space above
{3ex}% Space below
{}% Body font
{}% Indent amount 1
{\bfseries} % Theorem head font
{}% Punctuation after theorem head
{\newline}% Space after theorem head 2
{}% Theorem head spec (can be left empty, meaning ‘normal’)

%\theoremstyle{styleexo}
\newtheorem{exo}{Exercice}
\newtheorem{ind}{Indications}
\newtheorem{cor}{Correction}


\newcommand{\exercice}[1]{} \newcommand{\finexercice}{}
%\newcommand{\exercice}[1]{{\tiny\texttt{#1}}\vspace{-2ex}} % pour afficher le numero absolu, l'auteur...
\newcommand{\enonce}{\begin{exo}} \newcommand{\finenonce}{\end{exo}}
\newcommand{\indication}{\begin{ind}} \newcommand{\finindication}{\end{ind}}
\newcommand{\correction}{\begin{cor}} \newcommand{\fincorrection}{\end{cor}}

\newcommand{\noindication}{\stepcounter{ind}}
\newcommand{\nocorrection}{\stepcounter{cor}}

\newcommand{\fiche}[1]{} \newcommand{\finfiche}{}
\newcommand{\titre}[1]{\centerline{\large \bf #1}}
\newcommand{\addcommand}[1]{}
\newcommand{\video}[1]{}

% Marge
\newcommand{\mymargin}[1]{\marginpar{{\small #1}}}

\def\noqed{\renewcommand{\qedsymbol}{}}


%----- Presentation ------
\setlength{\parindent}{0cm}

%\newcommand{\ExoSept}{\href{http://exo7.emath.fr}{\textbf{\textsf{Exo7}}}}

\definecolor{myred}{rgb}{0.93,0.26,0}
\definecolor{myorange}{rgb}{0.97,0.58,0}
\definecolor{myyellow}{rgb}{1,0.86,0}

\newcommand{\LogoExoSept}[1]{  % input : echelle
{\usefont{U}{cmss}{bx}{n}
\begin{tikzpicture}[scale=0.1*#1,transform shape]
  \fill[color=myorange] (0,0)--(4,0)--(4,-4)--(0,-4)--cycle;
  \fill[color=myred] (0,0)--(0,3)--(-3,3)--(-3,0)--cycle;
  \fill[color=myyellow] (4,0)--(7,4)--(3,7)--(0,3)--cycle;
  \node[scale=5] at (3.5,3.5) {Exo7};
\end{tikzpicture}}
}


\newcommand{\debutmontitre}{
  \author{} \date{} 
  \thispagestyle{empty}
  \hspace*{-10ex}
  \begin{minipage}{\textwidth}
    \titlepage  
  \vspace*{-2.5cm}
  \begin{center}
    \LogoExoSept{2.5}
  \end{center}
  \end{minipage}

  \vspace*{-0cm}
  
  % Astuce pour que le background ne soit pas discrétisé lors de la conversion pdf -> png
\begin{tikzpicture}
        \fill[opacity=0,green!60!black] (0,0)--++(0,0)--++(0,0)--++(0,0)--cycle; 
\end{tikzpicture}

% toc S'affiche trop tot :
% \tableofcontents[hideallsubsections, pausesections]
}

\newcommand{\finmontitre}{
  \end{frame}
  \setcounter{framenumber}{0}
} % ne marche pas pour une raison obscure

%----- Commandes supplementaires ------

% \usepackage[landscape]{geometry}
% \geometry{top=1cm, bottom=3cm, left=2cm, right=10cm, marginparsep=1cm
% }
% \usepackage[a4paper]{geometry}
% \geometry{top=2cm, bottom=2cm, left=2cm, right=2cm, marginparsep=1cm
% }

%\usepackage{standalone}


% New command Arnaud -- november 2011
\setbeamersize{text margin left=24ex}
% si vous modifier cette valeur il faut aussi
% modifier le decalage du titre pour compenser
% (ex : ici =+10ex, titre =-5ex

\theoremstyle{definition}
%\newtheorem{proposition}{Proposition}
%\newtheorem{exemple}{Exemple}
%\newtheorem{theoreme}{Théorème}
%\newtheorem{lemme}{Lemme}
%\newtheorem{corollaire}{Corollaire}
%\newtheorem*{remarque*}{Remarque}
%\newtheorem*{miniexercice}{Mini-exercices}
%\newtheorem{definition}{Définition}

% Commande tikz
\usetikzlibrary{calc}
\usetikzlibrary{patterns,arrows}
\usetikzlibrary{matrix}
\usetikzlibrary{fadings} 

%definition d'un terme
\newcommand{\defi}[1]{{\color{myorange}\textbf{\emph{#1}}}}
\newcommand{\evidence}[1]{{\color{blue}\textbf{\emph{#1}}}}
\newcommand{\assertion}[1]{\emph{\og#1\fg}}  % pour chapitre logique
%\renewcommand{\contentsname}{Sommaire}
\renewcommand{\contentsname}{}
\setcounter{tocdepth}{2}



%------ Figures ------

\def\myscale{1} % par défaut 
\newcommand{\myfigure}[2]{  % entrée : echelle, fichier figure
\def\myscale{#1}
\begin{center}
\footnotesize
{#2}
\end{center}}


%------ Encadrement ------

\usepackage{fancybox}


\newcommand{\mybox}[1]{
\setlength{\fboxsep}{7pt}
\begin{center}
\shadowbox{#1}
\end{center}}

\newcommand{\myboxinline}[1]{
\setlength{\fboxsep}{5pt}
\raisebox{-10pt}{
\shadowbox{#1}
}
}

%--------------- Commande beamer---------------
\newcommand{\beameronly}[1]{#1} % permet de mettre des pause dans beamer pas dans poly


\setbeamertemplate{navigation symbols}{}
\setbeamertemplate{footline}  % tiré du fichier beamerouterinfolines.sty
{
  \leavevmode%
  \hbox{%
  \begin{beamercolorbox}[wd=.333333\paperwidth,ht=2.25ex,dp=1ex,center]{author in head/foot}%
    % \usebeamerfont{author in head/foot}\insertshortauthor%~~(\insertshortinstitute)
    \usebeamerfont{section in head/foot}{\bf\insertshorttitle}
  \end{beamercolorbox}%
  \begin{beamercolorbox}[wd=.333333\paperwidth,ht=2.25ex,dp=1ex,center]{title in head/foot}%
    \usebeamerfont{section in head/foot}{\bf\insertsectionhead}
  \end{beamercolorbox}%
  \begin{beamercolorbox}[wd=.333333\paperwidth,ht=2.25ex,dp=1ex,right]{date in head/foot}%
    % \usebeamerfont{date in head/foot}\insertshortdate{}\hspace*{2em}
    \insertframenumber{} / \inserttotalframenumber\hspace*{2ex} 
  \end{beamercolorbox}}%
  \vskip0pt%
}


\definecolor{mygrey}{rgb}{0.5,0.5,0.5}
\setlength{\parindent}{0cm}
%\DeclareTextFontCommand{\helvetica}{\fontfamily{phv}\selectfont}

% background beamer
\definecolor{couleurhaut}{rgb}{0.85,0.9,1}  % creme
\definecolor{couleurmilieu}{rgb}{1,1,1}  % vert pale
\definecolor{couleurbas}{rgb}{0.85,0.9,1}  % blanc
\setbeamertemplate{background canvas}[vertical shading]%
[top=couleurhaut,middle=couleurmilieu,midpoint=0.4,bottom=couleurbas] 
%[top=fondtitre!05,bottom=fondtitre!60]



\makeatletter
\setbeamertemplate{theorem begin}
{%
  \begin{\inserttheoremblockenv}
  {%
    \inserttheoremheadfont
    \inserttheoremname
    \inserttheoremnumber
    \ifx\inserttheoremaddition\@empty\else\ (\inserttheoremaddition)\fi%
    \inserttheorempunctuation
  }%
}
\setbeamertemplate{theorem end}{\end{\inserttheoremblockenv}}

\newenvironment{theoreme}[1][]{%
   \setbeamercolor{block title}{fg=structure,bg=structure!40}
   \setbeamercolor{block body}{fg=black,bg=structure!10}
   \begin{block}{{\bf Th\'eor\`eme }#1}
}{%
   \end{block}%
}


\newenvironment{proposition}[1][]{%
   \setbeamercolor{block title}{fg=structure,bg=structure!40}
   \setbeamercolor{block body}{fg=black,bg=structure!10}
   \begin{block}{{\bf Proposition }#1}
}{%
   \end{block}%
}

\newenvironment{corollaire}[1][]{%
   \setbeamercolor{block title}{fg=structure,bg=structure!40}
   \setbeamercolor{block body}{fg=black,bg=structure!10}
   \begin{block}{{\bf Corollaire }#1}
}{%
   \end{block}%
}

\newenvironment{mydefinition}[1][]{%
   \setbeamercolor{block title}{fg=structure,bg=structure!40}
   \setbeamercolor{block body}{fg=black,bg=structure!10}
   \begin{block}{{\bf Définition} #1}
}{%
   \end{block}%
}

\newenvironment{lemme}[0]{%
   \setbeamercolor{block title}{fg=structure,bg=structure!40}
   \setbeamercolor{block body}{fg=black,bg=structure!10}
   \begin{block}{\bf Lemme}
}{%
   \end{block}%
}

\newenvironment{remarque}[1][]{%
   \setbeamercolor{block title}{fg=black,bg=structure!20}
   \setbeamercolor{block body}{fg=black,bg=structure!5}
   \begin{block}{Remarque #1}
}{%
   \end{block}%
}


\newenvironment{exemple}[1][]{%
   \setbeamercolor{block title}{fg=black,bg=structure!20}
   \setbeamercolor{block body}{fg=black,bg=structure!5}
   \begin{block}{{\bf Exemple }#1}
}{%
   \end{block}%
}


\newenvironment{miniexercice}[0]{%
   \setbeamercolor{block title}{fg=structure,bg=structure!20}
   \setbeamercolor{block body}{fg=black,bg=structure!5}
   \begin{block}{Mini-exercices}
}{%
   \end{block}%
}


\newenvironment{tp}[0]{%
   \setbeamercolor{block title}{fg=structure,bg=structure!40}
   \setbeamercolor{block body}{fg=black,bg=structure!10}
   \begin{block}{\bf Travaux pratiques}
}{%
   \end{block}%
}
\newenvironment{exercicecours}[1][]{%
   \setbeamercolor{block title}{fg=structure,bg=structure!40}
   \setbeamercolor{block body}{fg=black,bg=structure!10}
   \begin{block}{{\bf Exercice }#1}
}{%
   \end{block}%
}
\newenvironment{algo}[1][]{%
   \setbeamercolor{block title}{fg=structure,bg=structure!40}
   \setbeamercolor{block body}{fg=black,bg=structure!10}
   \begin{block}{{\bf Algorithme}\hfill{\color{gray}\texttt{#1}}}
}{%
   \end{block}%
}


\setbeamertemplate{proof begin}{
   \setbeamercolor{block title}{fg=black,bg=structure!20}
   \setbeamercolor{block body}{fg=black,bg=structure!5}
   \begin{block}{{\footnotesize Démonstration}}
   \footnotesize
   \smallskip}
\setbeamertemplate{proof end}{%
   \end{block}}
\setbeamertemplate{qed symbol}{\openbox}


\makeatother
\usecolortheme[RGB={153, 204, 0}]{structure}

% Commande spécifique à ce chapitre
\newcounter{saveenumi}

%%%%%%%%%%%%%%%%%%%%%%%%%%%%%%%%%%%%%%%%%%%%%%%%%%%%%%%%%%%%%
%%%%%%%%%%%%%%%%%%%%%%%%%%%%%%%%%%%%%%%%%%%%%%%%%%%%%%%%%%%%%



\begin{document}



\title{{\bf Arithmétique}}
\subtitle{Division euclidienne et pgcd}

\begin{frame}
  
  \debutmontitre

  \pause

{\footnotesize
\hfill
\setbeamercovered{transparent=50}
\begin{minipage}{0.6\textwidth}
  \begin{itemize}
    \item<3-> Divisibilité et division euclidienne
    \item<4-> pgcd
    \item<5-> Algorithme d'Euclide
    \item<6-> Nombres premiers entre eux
  \end{itemize}
\end{minipage}
}

\end{frame}

\setcounter{framenumber}{0}


%%%%%%%%%%%%%%%%%%%%%%%%%%%%%%%%%%%%%%%%%%%%%%%%%%%%%%%%%%%%%%%%


\section*{Motivation}


\begin{frame}
Une motivation : l'arithmétique appliquée à la cryptographie
\pause
\begin{itemize}
  \item Avec deux \evidence{nombres premiers} $p$ et $q$ secrets on calcule $n=p\times q$
\pause
Même connaissant $n$ il est très difficile de retrouver $p$ et $q$ 
\pause
  \item La clé secrète et la clé publique se calculent à l'aide de l'\evidence{algorithme d'Euclide} 
et des \evidence{coefficients de Bézout}
\pause
  \item Les calculs se font \evidence{modulo} $n$
\pause
  \item Le décodage fonctionne grâce au \evidence{petit théorème de Fermat}
\end{itemize}
\end{frame}


%---------------------------------------------------------------
\section{Divisibilité et division euclidienne}

%---------------------------------------------------------------

\begin{frame}

Soient $a,b \in \Zz$
\begin{mydefinition}
$b$ \defi{divise} $a$ s'il existe $q \in \Zz$ tel que \myboxinline{$a = b \times q$}

\pause
\smallskip

\centerline{on note $b | a$}
\end{mydefinition}

\pause

\begin{exemple}
\begin{itemize}
  \item $7 | 21$ \pause\qquad $6 | 48$  \pause\qquad  $a$ est pair si et seulement si $2|a$
\pause
  \item Pour tout $a \in \Zz$ on a \ \ $a | 0$ \ \   \pause et aussi \ \  $1|a$
\pause
  \item Si $a|1$ alors $a=+1$ ou $a=-1$
\pause
  \item $(a|b \text{ et } b|a) \implies b= \pm a$
\pause
  \item $(a|b \text{ et } b|c) \implies a | c$
\pause
  \item $(a|b \text{ et } a|c) \implies a | b+c$
\end{itemize}
\end{exemple}
\end{frame}

\begin{frame}
Soit $a\in \Zz$ et $b\in \Nn\setminus \{0\}$
\begin{theoreme}[Division euclidienne]
Il \evidence{existe} des entiers $q,r \in \Zz$ \uncover<3->{\evidence{uniques}} tels que
\mybox{$a=bq+r \qquad\pause \text{ et } \qquad  0 \le r < b$
\pause\pause}
\end{theoreme}

\pause

\begin{exemple}
$a=6789$ et $b=34$ \qquad $6789= 34 \times 199 + 23$
\pause
\myfigure{1}{
\tikzinput{fig_arithmetique02} \qquad \qquad \qquad
\pause
\tikzinput{fig_arithmetique01}
}

\end{exemple}
\end{frame}


%---------------------------------------------------------------
\section{pgcd de deux entiers}

\begin{frame}
Soient $a,b\in\Zz$ deux entiers, non tous les deux nuls

\begin{mydefinition}
Le \defi{plus grand diviseur commun} de $a$ et $b$ 
est le plus grand entier qui divise à la fois $a$ et $b$
\end{mydefinition}

\pause

Il se note $\pgcd(a,b)$

\pause
\bigskip

\begin{exemple}
\begin{itemize}
  \item $\pgcd(21,14)=7$ \quad\pause $\pgcd(12,32)=4$ \quad\pause $\pgcd(21,26)=1$
\pause
  \item $\pgcd(a,ka)=a$ \quad pour tout $k \in \Zz$ et $a \ge 0$
\pause
  \item Pour tout $a\ge 0$ : $\pgcd(a,0)=a$ \pause \quad et \quad  $\pgcd(a,1)=1$
\end{itemize}
\end{exemple}
\end{frame}

%---------------------------------------------------------------
\section{Algorithme d'Euclide}

\begin{frame}

Soient $a,b \in \Nn^*$. \'Ecrivons la division euclidienne $a=bq+r$
\begin{lemme}
\label{lem:algoeuclide}
\mybox{$\pgcd(a,b)=\pgcd(b,r)$} 
\end{lemme}

\pause
\medskip

En fait on a $\pgcd(a,b) = \pgcd(b,a-qb)$ pour tout $q\in \Zz$

\pause
\medskip

\begin{proof}

\begin{itemize}
  \item Soit $d$ un diviseur de $a$ et de $b$. Alors $d$ divise $bq-a=r$
\pause
  \item Soit $d$ un diviseur de $b$ et de $r$. Alors $d$ divise aussi $bq+r=a$
\end{itemize}
\pause
Les diviseurs de $a$ et de $b$ sont exactement les mêmes que les diviseurs de $b$ et $r$
\end{proof}
\end{frame}


\begin{frame}
\evidence{Algorithme d'Euclide} : calcul de $\pgcd(a,b)$

\medskip
\pause

On calcule des divisions euclidiennes successives

\pause

Le pgcd sera le dernier reste non nul

\pause
\medskip

\begin{exemple}
Le pgcd de $a=600$ et $b=124$


\pause

$$
\begin{array}{rclclcl}
600 & = & 124\tikz[remember picture]\coordinate(eucun); & \times & 4 & + & 104\tikz[remember picture]\coordinate(eucdeux); \\ 
\uncover<7->{124\tikz[remember picture]\coordinate(euctrois); & = & 104\tikz[remember picture]\coordinate(eucquatre); 
  & \times & 1 & + & 20\tikz[remember picture]\coordinate(euccinq);} \\
\uncover<9->{104\tikz[remember picture]\coordinate(eucsix); & = & 20\tikz[remember picture]\coordinate(eucsept);  
  & \times & 5 & + & \alert<11->{4}} \\
\pause
\uncover<10->{20  & = & 4   & \times & 5 & + & 0} \\
\end{array}
$$
\begin{tikzpicture}[x=1mm,y=1mm, remember picture, overlay]

\uncover<5->{
   \coordinate (myeucun) at ($(eucun)+(-6,0)$);
   \coordinate (myeucdeux) at ($(eucdeux)+(-6,0)$);
   \coordinate (myeuctrois) at ($(euctrois)+(0,2)$);
   \coordinate (myeucquatre) at ($(eucquatre)+(0,2)$);
   \draw[->, myred, very thick] (myeucun) to[thick] (myeuctrois);
   \draw[->, myred, very thick] (myeucdeux) to[thick] (myeucquatre);
}

\uncover<8->{
   \coordinate (myeucquatrebis) at ($(eucquatre)+(-6,0)$);
   \coordinate (myeuccinq) at ($(euccinq)+(-4,0)$);
   \coordinate (myeucsix) at ($(eucsix)+(0,2)$);
   \coordinate (myeucsept) at ($(eucsept)+(0,2)$);
   \draw[->, myorange, very thick] (myeucquatrebis) to[thick] (myeucsix);
   \draw[->, myorange, very thick] (myeuccinq) to[thick] (myeucsept);
}
\end{tikzpicture}

\pause \pause \pause \pause \pause \pause
Ainsi $\pgcd(600,124)=4$
\end{exemple}



\end{frame}



\begin{frame}

\begin{exemple}
Calculons $\pgcd(9945,3003)$
\pause
$$
\begin{array}{rclclcl}
9945 & = & 3003 & \times & 3 & + & 936 \\ 
\pause
3003 & = & 936  & \times & 3 & + & 195 \\
\pause
936  & = & 195  & \times & 4 & + & 156 \\
\pause
195  & = & 156  & \times & 1 & + & \alt<5-6>{39}{\textcolor{red}{39}} \\
\pause
156  & = & 39   & \times & 4 & + & 0 \\
\end{array}
$$
\pause\pause
Ainsi $\pgcd(9945,3003) = 39$
\end{exemple}
\end{frame}

%---------------------------------------------------------------
\section{Nombres premiers entre eux}

\begin{frame}
\begin{mydefinition}
$a$ et $b$ sont \defi{premiers entre eux} si $\pgcd(a,b)=1$ 
\end{mydefinition}

\pause

\begin{exemple}
Pour tout $a\in \Zz$, $a$ et $a+1$ sont premiers entre eux

\pause 
{\footnotesize
\emph{Preuve :} soit $d$ un diviseur commun à $a$ et à $a+1$

\pause

$d|a \text{ et } d|a+1 \pause\implies d| a+1-a \pause\implies d| 1 \pause\implies d=\pm1 \pause\implies \pgcd(a,a+1)=1$
}
\end{exemple}

\pause

\begin{exemple}
Pour $a,b \in \Zz$ quelconques, notons $d = \pgcd(a,b)$

La décomposition suivante est souvent utile :
\mybox{$\left\{\begin{array}{ll} a &= \ a'd \\ b &= \ b'd \\ \end{array} \quad 
\text{ avec }\quad  \pgcd(a',b')=1 \right.$}
\end{exemple}


\end{frame}



%---------------------------------------------------------------
\section*{Mini-exercices}

\begin{frame}
\begin{miniexercice}
\begin{enumerate}
  \item \'Ecrire la division euclidienne de $111\,111$ par $20xx$, où $20xx$ est l'année en cours.
  \item Montrer qu'un diviseur positif de $10\,008$ et de $10\,014$ appartient nécessairement à $\{1,2,3,6\}$.
  \item Calculer $\pgcd(560,133)$, $\pgcd(12\,121,789)$, $\pgcd(99\,999,1110)$.
  \item Trouver tous les entiers $1\le a \le 50$ tels que $a$ et $50$ soient premiers entre eux.
Même question avec $52$.
\end{enumerate}  
\end{miniexercice}
\end{frame}


\end{document}