
%%%%%%%%%%%%%%%%%% PREAMBULE %%%%%%%%%%%%%%%%%%


\documentclass[12pt]{article}

\usepackage{amsfonts,amsmath,amssymb,amsthm}
\usepackage[utf8]{inputenc}
\usepackage[T1]{fontenc}
\usepackage[francais]{babel}


% packages
\usepackage{amsfonts,amsmath,amssymb,amsthm}
\usepackage[utf8]{inputenc}
\usepackage[T1]{fontenc}
%\usepackage{lmodern}

\usepackage[francais]{babel}
\usepackage{fancybox}
\usepackage{graphicx}

\usepackage{float}

%\usepackage[usenames, x11names]{xcolor}
\usepackage{tikz}
\usepackage{datetime}

\usepackage{mathptmx}
%\usepackage{fouriernc}
%\usepackage{newcent}
\usepackage[mathcal,mathbf]{euler}

%\usepackage{palatino}
%\usepackage{newcent}


% Commande spéciale prompteur

%\usepackage{mathptmx}
%\usepackage[mathcal,mathbf]{euler}
%\usepackage{mathpple,multido}

\usepackage[a4paper]{geometry}
\geometry{top=2cm, bottom=2cm, left=1cm, right=1cm, marginparsep=1cm}

\newcommand{\change}{{\color{red}\rule{\textwidth}{1mm}\\}}

\newcounter{mydiapo}

\newcommand{\diapo}{\newpage
\hfill {\normalsize  Diapo \themydiapo \quad \texttt{[\jobname]}} \\
\stepcounter{mydiapo}}


%%%%%%% COULEURS %%%%%%%%%%

% Pour blanc sur noir :
%\pagecolor[rgb]{0.5,0.5,0.5}
% \pagecolor[rgb]{0,0,0}
% \color[rgb]{1,1,1}



%\DeclareFixedFont{\myfont}{U}{cmss}{bx}{n}{18pt}
\newcommand{\debuttexte}{
%%%%%%%%%%%%% FONTES %%%%%%%%%%%%%
\renewcommand{\baselinestretch}{1.5}
\usefont{U}{cmss}{bx}{n}
\bfseries

% Taille normale : commenter le reste !
%Taille Arnaud
%\fontsize{19}{19}\selectfont

% Taille Barbara
%\fontsize{21}{22}\selectfont

%Taille François
\fontsize{25}{30}\selectfont

%Taille Pascal
%\fontsize{25}{30}\selectfont

%Taille Laura
%\fontsize{30}{35}\selectfont


%\myfont
%\usefont{U}{cmss}{bx}{n}

%\Huge
%\addtolength{\parskip}{\baselineskip}
}


% \usepackage{hyperref}
% \hypersetup{colorlinks=true, linkcolor=blue, urlcolor=blue,
% pdftitle={Exo7 - Exercices de mathématiques}, pdfauthor={Exo7}}


%section
% \usepackage{sectsty}
% \allsectionsfont{\bf}
%\sectionfont{\color{Tomato3}\upshape\selectfont}
%\subsectionfont{\color{Tomato4}\upshape\selectfont}

%----- Ensembles : entiers, reels, complexes -----
\newcommand{\Nn}{\mathbb{N}} \newcommand{\N}{\mathbb{N}}
\newcommand{\Zz}{\mathbb{Z}} \newcommand{\Z}{\mathbb{Z}}
\newcommand{\Qq}{\mathbb{Q}} \newcommand{\Q}{\mathbb{Q}}
\newcommand{\Rr}{\mathbb{R}} \newcommand{\R}{\mathbb{R}}
\newcommand{\Cc}{\mathbb{C}} 
\newcommand{\Kk}{\mathbb{K}} \newcommand{\K}{\mathbb{K}}

%----- Modifications de symboles -----
\renewcommand{\epsilon}{\varepsilon}
\renewcommand{\Re}{\mathop{\text{Re}}\nolimits}
\renewcommand{\Im}{\mathop{\text{Im}}\nolimits}
%\newcommand{\llbracket}{\left[\kern-0.15em\left[}
%\newcommand{\rrbracket}{\right]\kern-0.15em\right]}

\renewcommand{\ge}{\geqslant}
\renewcommand{\geq}{\geqslant}
\renewcommand{\le}{\leqslant}
\renewcommand{\leq}{\leqslant}

%----- Fonctions usuelles -----
\newcommand{\ch}{\mathop{\mathrm{ch}}\nolimits}
\newcommand{\sh}{\mathop{\mathrm{sh}}\nolimits}
\renewcommand{\tanh}{\mathop{\mathrm{th}}\nolimits}
\newcommand{\cotan}{\mathop{\mathrm{cotan}}\nolimits}
\newcommand{\Arcsin}{\mathop{\mathrm{Arcsin}}\nolimits}
\newcommand{\Arccos}{\mathop{\mathrm{Arccos}}\nolimits}
\newcommand{\Arctan}{\mathop{\mathrm{Arctan}}\nolimits}
\newcommand{\Argsh}{\mathop{\mathrm{Argsh}}\nolimits}
\newcommand{\Argch}{\mathop{\mathrm{Argch}}\nolimits}
\newcommand{\Argth}{\mathop{\mathrm{Argth}}\nolimits}
\newcommand{\pgcd}{\mathop{\mathrm{pgcd}}\nolimits} 

\newcommand{\Card}{\mathop{\text{Card}}\nolimits}
\newcommand{\Ker}{\mathop{\text{Ker}}\nolimits}
\newcommand{\id}{\mathop{\text{id}}\nolimits}
\newcommand{\ii}{\mathrm{i}}
\newcommand{\dd}{\mathrm{d}}
\newcommand{\Vect}{\mathop{\text{Vect}}\nolimits}
\newcommand{\Mat}{\mathop{\mathrm{Mat}}\nolimits}
\newcommand{\rg}{\mathop{\text{rg}}\nolimits}
\newcommand{\tr}{\mathop{\text{tr}}\nolimits}
\newcommand{\ppcm}{\mathop{\text{ppcm}}\nolimits}

%----- Structure des exercices ------

\newtheoremstyle{styleexo}% name
{2ex}% Space above
{3ex}% Space below
{}% Body font
{}% Indent amount 1
{\bfseries} % Theorem head font
{}% Punctuation after theorem head
{\newline}% Space after theorem head 2
{}% Theorem head spec (can be left empty, meaning ‘normal’)

%\theoremstyle{styleexo}
\newtheorem{exo}{Exercice}
\newtheorem{ind}{Indications}
\newtheorem{cor}{Correction}


\newcommand{\exercice}[1]{} \newcommand{\finexercice}{}
%\newcommand{\exercice}[1]{{\tiny\texttt{#1}}\vspace{-2ex}} % pour afficher le numero absolu, l'auteur...
\newcommand{\enonce}{\begin{exo}} \newcommand{\finenonce}{\end{exo}}
\newcommand{\indication}{\begin{ind}} \newcommand{\finindication}{\end{ind}}
\newcommand{\correction}{\begin{cor}} \newcommand{\fincorrection}{\end{cor}}

\newcommand{\noindication}{\stepcounter{ind}}
\newcommand{\nocorrection}{\stepcounter{cor}}

\newcommand{\fiche}[1]{} \newcommand{\finfiche}{}
\newcommand{\titre}[1]{\centerline{\large \bf #1}}
\newcommand{\addcommand}[1]{}
\newcommand{\video}[1]{}

% Marge
\newcommand{\mymargin}[1]{\marginpar{{\small #1}}}



%----- Presentation ------
\setlength{\parindent}{0cm}

%\newcommand{\ExoSept}{\href{http://exo7.emath.fr}{\textbf{\textsf{Exo7}}}}

\definecolor{myred}{rgb}{0.93,0.26,0}
\definecolor{myorange}{rgb}{0.97,0.58,0}
\definecolor{myyellow}{rgb}{1,0.86,0}

\newcommand{\LogoExoSept}[1]{  % input : echelle
{\usefont{U}{cmss}{bx}{n}
\begin{tikzpicture}[scale=0.1*#1,transform shape]
  \fill[color=myorange] (0,0)--(4,0)--(4,-4)--(0,-4)--cycle;
  \fill[color=myred] (0,0)--(0,3)--(-3,3)--(-3,0)--cycle;
  \fill[color=myyellow] (4,0)--(7,4)--(3,7)--(0,3)--cycle;
  \node[scale=5] at (3.5,3.5) {Exo7};
\end{tikzpicture}}
}



\theoremstyle{definition}
%\newtheorem{proposition}{Proposition}
%\newtheorem{exemple}{Exemple}
%\newtheorem{theoreme}{Théorème}
\newtheorem{lemme}{Lemme}
\newtheorem{corollaire}{Corollaire}
%\newtheorem*{remarque*}{Remarque}
%\newtheorem*{miniexercice}{Mini-exercices}
%\newtheorem{definition}{Définition}




%definition d'un terme
\newcommand{\defi}[1]{{\color{myorange}\textbf{\emph{#1}}}}
\newcommand{\evidence}[1]{{\color{blue}\textbf{\emph{#1}}}}



 %----- Commandes divers ------

\newcommand{\codeinline}[1]{\texttt{#1}}

%%%%%%%%%%%%%%%%%%%%%%%%%%%%%%%%%%%%%%%%%%%%%%%%%%%%%%%%%%%%%
%%%%%%%%%%%%%%%%%%%%%%%%%%%%%%%%%%%%%%%%%%%%%%%%%%%%%%%%%%%%%



\begin{document}

\debuttexte

%%%%%%%%%%%%%%%%%%%%%%%%%%%%%%%%%%%%%%%%%%%%%%%%%%%%%%%%%%%
\diapo

\change

Dans cette première partie du chapitre sur les intégrales nous n'allons pas en calculer beaucoup !
Nous allons bien comprendre comment est définie l'intégrale, plus précisément l'intégrale de Riemann.


\change

Nous allons d'abord définir l'intégrale pour des fonctions très simples : les fonctions en escalier.

\change

Cela nous permettra ensuite de définir l'intégrale de fonctions beaucoup plus générales

\change

Puis nous verrons quelques propriétés élémentaires

\change

Le théorème principal de cette leçon sera que les fonctions continues admettent des intégrales

\change

Et nous en donneront la preuve.

%%%%%%%%%%%%%%%%%%%%%%%%%%%%%%%%%%%%%%%%%%%%%%%%%%%%%%%%%%%
\diapo



Nous allons introduire l'intégrale à l'aide d'un exemple.

Considérons la fonction exponentielle $f(x)=e^x$. 


\change


On souhaite calculer
l'aire $\mathcal{A}$ en-dessous du graphe de $f$ et 
entre les droites d'équation $(x=0)$, $(x=1)$ et l'axe $(O x)$.


\change


Nous approchons cette aire par des sommes d'aires de rectangles situés sous la courbe. 


Plus précisément découpons notre intervalle $[0,1]$ en $n$ morceaux.

Sur ce dessin $n=4$.


On considère les <<rectangles inférieurs>> $\mathcal{R}_i^-$, 
chacun ayant pour base le segment $\big[\frac{i-1}{n},\frac{i}{n}\big]$ de longueur $1/n$.

On choisit la hauteur de chacun des rectangles pour qu'il soit juste en dessous de la courbe.
Ici la hauteur est donc la valeur prise sur la gauche de l'intervalle qui vaut ici $e^{(i-1)/n}$


\change

Calculons l'aire verte de nos rectangles inférieurs.

L'aire d'un rectangle est <<base $\times$ hauteur>> : 
donc l'aire vaut $\frac1n \times e^{(i-1)/n}$

Et il faut faire la somme sur tous les rectangles, c'est-à-dire pour $i$ variant de $1$
à $n$.

\change

On récrit cette somme ainsi

Et on reconnaît la somme d'une suite géométrique

dont la raison est $e^{\frac 1n}$.

\change

On calcule donc cette somme 

\change

Et on la réécrit ainsi.

$\frac{\frac{1}{n}}{e^{\frac 1n}-1}\big(e-1\big) $


Que se passe-t-il lorsque $n$ tend vers $+\infty$ ?

On sait que $\frac{e^x-1}{x} \to 1$ lorsque $x$ tend vers $0$,

donc $\frac{\frac{1}{n}}{e^{\frac 1n}-1}$ tend aussi vers $1$ lorsque $n$ tend vers $+\infty$.

\change

Notre somme tend donc vers $e-1$.

\bigskip

[split]

\change





Regardons maintenant les <<rectangles supérieurs>> 

ils ont la même base mais sont au-dessus de la courbe. Pour notre fonction
ils ont donc la hauteur qui correspond à la valeur à la droite du segment de base.


Un calcul tout à fait similaire montre que lorsque $n$ tend vers $+\infty$ la somme des aires rouges des rectangles supérieurs
tend aussi vers $e-1$.

\change


L'aire $\mathcal{A}$ de notre région est plus grande que l'aire verte des rectangles inférieurs  ;
et plus petite que l'aire rouge des rectangles supérieurs. 


Lorsque l'on considère des subdivisions de plus en plus petites
(c'est-à-dire lorsque l'on fait tendre $n$ vers $+\infty$) 

alors on obtient à la limite que l'aire $\mathcal{A}$
de notre région est encadrée par deux aires qui tendent vers $e-1$. 

\change

Donc l'aire de notre région est $\mathcal{A} = e-1$.


\bigskip

[split]


Résumons tout cela : nous avons encadrer l'aire sous la courbe 
par des rectangles en vert qui restent en dessous et des rectangles
rouge qui sont au-dessus. 

Cela nous donne un encadrement de notre aire sous la courbe.

Notre intervalle de départ est coupé en $n$ morceaux, donc chaque rectangle
à une base de longueur $1/n$.

Sur ce dessin $n=4$,

\change

lorsque $n$ grandit [n=5]

\change

[n=6]

\change

[n=7]

\change

[n=10]

\change

[n=15]

\change

[n=20]

Alors la base de nos rectangles diminue et ils approchent de mieux en mieux 
l'aire sous la courbe.


Si l'on arrive à justifier que l'aire verte s'approche de l'aire rouge (comme on la fait ici par un calcul)
alors à la limite cet encadrement nous donne l'aire sous la courbe



%%%%%%%%%%%%%%%%%%%%%%%%%%%%%%%%%%%%%%%%%%%%%%%%%%%%%%%%%%%
\diapo

Nous allons reprendre la construction que l'on vient de faire mais cette fois pour une fonction $f$ quelconque.


\change

Ce qui va remplacer les rectangles seront des \evidence{fonctions en escalier}.

\change

Si la limite des aires en-dessous égale la limite des aires au-dessus on appelle cette limite commune 
\evidence{l'intégrale} de $f$ 

\change

Que l'on note $\int_a^b f(x) \; dx$. 

\change

Cependant il n'est pas toujours vrai que ces limites soient égales, l'intégrale n'est donc 
définie que pour les fonctions \evidence{intégrables}.

\change

Heureusement nous verrons 
que si la fonction $f$ est continue alors elle est intégrable.


%%%%%%%%%%%%%%%%%%%%%%%%%%%%%%%%%%%%%%%%%%%%%%%%%%%%%%%%%%%
\diapo


Soit $[a,b]$ un intervalle de $\Rr$.

On appelle une subdivision de $[a,b]$ une suite finie,
strictement croissante, de nombres $x_i$ tels que :
$a=x_0< x_1<\ldots< x_n=b$.


\change

Une fonction $f$ d'un intervalle $[a,b]$ dans $\Rr$ 
s'appelle  une fonction en escalier

s'il existe  une subdivision de l'intervalle $[a,b]$ 
et des valeurs $c_1,\ldots,c_n$ tels que $f(x)=c_i$
pour tous les $x$ dans le sous-intervalle $]x_{i-1},x_i[$.

\change

Autrement dit $f$ est une fonction constante sur chacun des sous-intervalles de la subdivision.

\change

Nous allons définir l'intégrale pour ces fonctions très spéciales :

Pour une fonction en escalier comme ci-dessus, son intégrale
est le réel $\sum_{i=1}^n c_i(x_i-x_{i-1})$

que l'on note $\int_a^b f(x) \; dx$.


(pause)

\change

Notez que chaque terme $c_i(x_i-x_{i-1})$ est l'aire du rectangle compris entre les abscisses
$x_{i-1}$ et $x_i$ et de hauteur $c_i$. Il faut juste prendre garde que l'on compte l'aire 
avec un signe <<$+$>> si $c_i>0$ et un signe <<$-$>> si $c_i<0$.


L'intégrale d'une fonction en escalier est l'aire  de la partie -en rouge- située au-dessus
de l'axe des abscisses  moins l'aire de la partie -en bleu-
située en-dessous.


L'intégrale d'une fonction en escalier est bien un nombre réel qui mesure l'aire algébrique 
(c'est-à-dire avec signe) entre la courbe de $f$ et l'axe des abscisses.


%%%%%%%%%%%%%%%%%%%%%%%%%%%%%%%%%%%%%%%%%%%%%%%%%%%%%%%%%%%
\diapo

Avant de définir l'intégrale pour une fonction quelconque rappelons qu'une fonction
 $f : [a,b] \to \Rr$ est \defi{bornée} s'il existe un réel $M\ge0$ tel que :
$-M \le f(x) \le M$ pour tout $x$ de $[a,b]$

\change

Rappelons aussi que si l'on a deux fonctions $f,g : [a,b] \to \Rr$, alors on dira 
$f$ est inférieur à $g$ si et seulement si $ f(x) \le g(x)$ pour tout $x\in [a,b]$.

\change

On suppose à présent que $f : [a,b] \to \Rr$ est une fonction bornée quelconque.

\change

Pour définir l'intégrale nous aurons besoin de deux nombres.

Le premier $I^-(f)$ est défini ainsi : (pointer du doigt).

On prend toute les fonctions $\phi$ en escalier qui sont inférieures à $f$.
On calcule à chaque fois l'intégrale de $\phi$, c'est-à-dire l'aire sous la fonction en escalier.

On prend l'aire la plus grande parmi toutes ces fonctions en escalier,
mais comme on n'est pas sûr que ce maximum existe on prend la borne supérieure.


\change

De même on définit un second réel $I^+(f)$ :

c'est le même principe mais les fonctions $\phi$ en escalier sont supérieures à $f$
et on cherche l'aire la plus petite possible.


\change

Il est intuitif que l'on a $I^-(f) \le I^+(f)$


\change

Nous dirons d'une fonction bornée $f :[a,b] \to \Rr$ qu'elle est \defi{intégrable}
si les deux nombres $I^-(f)$ et $I^+(f)$ sont égaux.

\change

Si c'est le cas alors cette valeur commune s'appelle l'intégrale de $f$ sur $[a,b]$ 

et on la note $\int_a^b f(x)\; dx$.



%%%%%%%%%%%%%%%%%%%%%%%%%%%%%%%%%%%%%%%%%%%%%%%%%%%%%%%%%%%
\diapo

La définition de l'intégrale repose donc sur la même idée qu'avec les rectangles :
on a des fonctions en escalier sous la courbe et on cherche l'aire la plus grande possible, pour cela on s'autorise 
à prendre des bases de rectangles aussi petites que l'on veut.

On a aussi des fonctions en escalier au-dessus de la courbe, et on cherche l'aire la plus petite possible.

\change

Au lieu de s'autoriser seulement un découpage régulier de l'intervalle avec des hauteurs correspondant
aux valeurs aux extrémités on considères toutes les subdivisions possibles, avec toutes les valeurs possibles
du moment que l'on reste en dessous pour les rectangles verts et au dessus pour les rectangles rouges.

\change

Évidemment c'est lorsque les subdivisions sont de plus en plus fines 

\change

que l'on peut trouver des fonctions en escaliers
qui approchent très bien le graphe de notre fonction.

Si les aires vertes et les aires rouge sont aussi aussi proche que l'on veut alors 
on dit que la fonction $f$ est intégrable et la limite
s'appelle l'intégrale de $f$ sur $[a,b]$.


%%%%%%%%%%%%%%%%%%%%%%%%%%%%%%%%%%%%%%%%%%%%%%%%%%%%%%%%%%%
\diapo

Voyons quelques exemples.

Tout d'abord les fonctions en escalier sont intégrables ! En effet si $f$ est une fonction en escalier
alors la borne inférieure $I^-(f)$ et la borne supérieure $I^+(f)$ sont atteintes avec la fonction 
$\phi=f$. 


\change

Nous verrons bientôt que toutes les fonctions continues sont des fonctions intégrables.


\change


Attention cependant, il existe des fonctions qui ne sont pas intégrables, par exemple celle-ci.

\change

Si $x$ est rationnel alors la fonction vaut $1$, sinon la fonction vaut $0$.

\change

C'est un bon exercice d'écrire proprement que $f$ n'est pas intégrable :
par la densité des rationnels si une fonction en escalier est supérieure à $f$ alors elle est supérieure à $1$.
Et par densité des irrationnels une fonction en escalier inférieure à $f$ est toujours négative.

Sur cet exemple $I^-(f)=0$ et $I^+(f)=1$, ce sont des valeurs distinctes et donc $f$ n'est pas intégrable.



%%%%%%%%%%%%%%%%%%%%%%%%%%%%%%%%%%%%%%%%%%%%%%%%%%%%%%%%%%%
\diapo

Mettons en pratique la définition de l'intégrale.

Regardons la fonction $f$ qui a $x$ associe $x^2$.

\change

$f$ est-elle intégrable ? 

\change

Et si oui que vaut son intégrale entre $0$ et $1$ ?


\change

On prend la subdivision régulière de $[0,1]$ suivante
$(0,\frac1n,\frac2n,\ldots,\frac in,\ldots, \frac{n-1}{n},1)$.


\change


Alors sur chaque intervalle $\big[\frac{i-1}{n},\frac in\big]$ nous avons
$\big(\tfrac{i-1}{n}\big)^2 \le x^2 \le \big(\tfrac in\big)^2$.


\change

Nous construisons une fonction en escalier $\phi^-$ en-dessous de $f$  par 
$\phi^-(x) = \frac{(i-1)^2}{n^2}$ si $x \in \big[\frac{i-1}{n},\frac in\big[$

\change

De même nous construisons une fonction en escalier $\phi^+$ au-dessus de $f$ définie
par $\phi^+(x) = \frac{i^2}{n^2}$ si $x \in \big[\frac{i-1}{n},\frac in\big[$

\change 

Ici nous avons découpé notre intervalle en $5$, 

\change

voici les dessins pour $n=6$ 

\change

et $n=7$.



%%%%%%%%%%%%%%%%%%%%%%%%%%%%%%%%%%%%%%%%%%%%%%%%%%%%%%%%%%%
\diapo


Nous allons maintenant calculer chacune des aires vertes et rouges
et faire tendre $n$ vers $+\infty$.

[n=7]

\change

[n=10]

\change

[n=20]

\change


\change


$\phi^-$ et $\phi^+$ sont des fonctions en escalier et l'on a $\phi^- \le f \le \phi^+$.

\change

L'intégrale de la fonction en escalier $\phi^+$ est par définition

 la somme de aires calculées par la formule 
base fois hauteur.

Ce qui se récrit :
$ \frac{1}{n^3} \sum_{i=1}^n i^2.$

\change

On se souvient de la formule pour la somme $\sum_{i=1}^n i^2$

\change

Et donc l'intégrale de la fonction en escalier $\phi^+(x)$ vaut 
$\frac{(n + 1)(2n + 1)}{6n^2}$


\change

On calcule de même l'intégrale de la fonction en escalier $\phi^-(x)$

\change

Maintenant $I^-(f)$ est la borne supérieure sur toutes les fonctions en escalier inférieures à $f$
donc en particulier $I^-(f) \ge \int_0^1 \phi^-(x)\; dx$.

\change

De même $I^+(f) \le \int_0^1 \phi^+(x)\; dx$.

\change

On connaît donc chacune des extrémités et elles tendent 
toutes les deux vers $1/3$ lorsque $n$ tend vers $+\infty$.

Par le théorème des gendarmes alors d'une part cela implique que $I^-(f)= I^+(f)$
et en plus cette valeur vaut $1/3$.

\change

Donc $f$ est bien intégrable et $\int_0^1 x^2\; dx = \frac13$.




%%%%%%%%%%%%%%%%%%%%%%%%%%%%%%%%%%%%%%%%%%%%%%%%%%%%%%%%%%%
\diapo

Voici deux résultats qui découlent de la définition :

Premièrement si $f : [a,b] \to \Rr$ est intégrable et si l'on change les valeurs de $f$
en un nombre fini de points alors la fonction $f$ est 
toujours intégrable et la valeur de $\int_a^b f(x)\; dx$ ne change pas.

\change

Deuxièmement si $f  : [a,b] \to \Rr$ est intégrable alors la restriction de $f$

à tout intervalle $[a',b'] \subset [a,b]$ est encore intégrable.


%%%%%%%%%%%%%%%%%%%%%%%%%%%%%%%%%%%%%%%%%%%%%%%%%%%%%%%%%%%
\diapo

Voici le résultat théorique le plus important de cette leçon :

Théorème : Si $f : [a,b] \to \Rr$ est une fonction continue alors $f$ est une fonction intégrable.  


\change

Ce résultat s'étend à une classe de fonction plus générale.

Une fonction $f : [a,b] \to \Rr$ est dite \defi{continue par morceaux} s'il existe 
une subdivision $(x_0,\ldots,x_n)$ telle que la restriction de $f$ soit continue sur chacun des sous-intervalles.

\change

Une conséquence du théorème est que les fonctions continues par morceaux sont intégrables.

%%%%%%%%%%%%%%%%%%%%%%%%%%%%%%%%%%%%%%%%%%%%%%%%%%%%%%%%%%%
\diapo

Voici une preuve du théorème principale, preuve que vous pouvez passer lors d'une première lecture.

Nous allons prouver une version un peu plus faible du théorème.  

Rappelons qu'une fonction $f$ est dite de 
classe $\mathcal{C}^1$ si $f$ est continue, dérivable et que $f'$ est aussi continue.

\change

Nous allons montrer le théorème suivant :

Si $f$ est de classe $\mathcal{C}^1$ alors $f$ est intégrable. 


L'idée est que les fonctions continues (ou ici de classe $\mathcal{C}^1$) 
peuvent être approchées d'aussi près que l'on veut par des fonctions en escalier, 
tout en gardant un contrôle d'erreur uniforme sur l'intervalle

\change

Comme $f$ est de classe  $\mathcal{C}^1$ alors $f'$ est une fonction continue sur l'intervalle fermé et borné $[a,b]$ ;
$f'$ est donc une fonction bornée : il existe $M\ge 0$ tel que pour tout $x \in [a,b]$ on ait $|f'(x)|\le M$.


\change

L'inégalité des accroissements finis nous dit alors que 
$\forall x,y \in [a,b] \quad |f(x)-f(y)| \le M |x-y|$


\change

Fixons un $\epsilon>0$ et considérons une subdivision $(x_0,x_1,\ldots,x_n)$ de l'intervalle 
$[a,b]$ suffisamment fine pour que $x_i-x_{i-1} \le \epsilon$


\change


Nous allons construire deux fonctions
en escalier $\phi^-, \phi^+ : [a,b] \to \Rr$.

$\phi^-$ est définie de la façon suivante 
sur l'intervalle $[x_{i-1},x_i[$,
$\phi^-(x)$ est la fonction constante $c_i$ 
où $c_i=\inf_{t\in [x_{i-1},x_i[} f(t)$.

\change

Voici la valeur $c_i$ pour ce sous-intervalle.

\change


De même pour $\phi^+(x)$ est constante sur chaque intervalle $[x_{i-1},x_i[$,
où elle prend la valeur $d_i = \sup_{t\in [x_{i-1},x_i[} f(t)$


$\phi^-$ et $\phi^+$ sont bien deux fonctions en escalier


%%%%%%%%%%%%%%%%%%%%%%%%%%%%%%%%%%%%%%%%%%%%%%%%%%%%%%%%%%%
\diapo


On a donc construit deux fonctions en escalier telles que $\phi^-\le f \le \phi^+$


\change

$\int_a^b \phi^-(x)\;dx \le I^-(f) \le I^+(f) \le \int_a^b \phi^+(x)\;dx$

\change

En utilisant la continuité de $f$ sur l'intervalle $[x_{i - 1} - x_i]$, 
alors le maximum et le minimum sont atteint sur cet intervalle,

c-a-d il existe
de $a_i,b_i \in [x_{i -1},x_i]$ tels que $f(a_i)=c_i$ et $f(b_i)=d_i$.

\change

Mais nous avons vu que grâce au théorème des accroissements finis 

$d_i-c_i$ qui vaut $ f(b_i) - f(a_i)$ est plus petit que $M |b_i - c_i|$

donc plus petit que $M (x_i-x_{i-1})$ et par le choix de notre subdivision $ \le M\epsilon$ 

\change

On va utiliser ces inégalités pour majorer la différence 
$\int_a^b \phi^+(x)\;dx - \int_a^b \phi^-(x)\;dx$

sur chaque intervalle  l'écart est inférieur à $M\epsilon(x_i-x_{i-1})$ 
(on a majorer la différence de hauteur par $M\epsilon$ et on multiplie par la base).

Lorsque l'on somme pour $i$ allant de $1$ à $n$ cela vaut 
$M\epsilon(b-a)$
car les $x_i$ forment un subdivision de $[a,b]$.

\change

Cela implique l'inégalité $I^+(f) - I^-(f) \le M\epsilon(b-a)$ 


\change

$M$ et $b-a$ sont des constantes fixées

lorsque l'on fait tendre $\epsilon \to 0$

on obtient que $I^+(f) = I^-(f)$, 

\change

ce qui prouve que $f$ est intégrable.

%%%%%%%%%%%%%%%%%%%%%%%%%%%%%%%%%%%%%%%%%%%%%%%%%%%%%%%%%%%
\diapo

C'est le moment de vérifier votre compréhension du cours !


\end{document}