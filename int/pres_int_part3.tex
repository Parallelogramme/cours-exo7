
%%%%%%%%%%%%%%%%%% PREAMBULE %%%%%%%%%%%%%%%%%%

\documentclass[aspectratio=169,utf8]{beamer}
%\documentclass[aspectratio=169,handout]{beamer}

\usetheme{Boadilla}
%\usecolortheme{seahorse}
\usecolortheme[RGB={245,66,24}]{structure}
\useoutertheme{infolines}

% packages
\usepackage{amsfonts,amsmath,amssymb,amsthm}
\usepackage[utf8]{inputenc}
\usepackage[T1]{fontenc}
\usepackage{lmodern}

\usepackage[francais]{babel}
\usepackage{fancybox}
\usepackage{graphicx}

\usepackage{float}
\usepackage{xfrac}

%\usepackage[usenames, x11names]{xcolor}
\usepackage{tikz}
\usepackage{pgfplots}
\usepackage{datetime}



%-----  Package unités -----
\usepackage{siunitx}
\sisetup{locale = FR,detect-all,per-mode = symbol}

%\usepackage{mathptmx}
%\usepackage{fouriernc}
%\usepackage{newcent}
%\usepackage[mathcal,mathbf]{euler}

%\usepackage{palatino}
%\usepackage{newcent}
% \usepackage[mathcal,mathbf]{euler}



% \usepackage{hyperref}
% \hypersetup{colorlinks=true, linkcolor=blue, urlcolor=blue,
% pdftitle={Exo7 - Exercices de mathématiques}, pdfauthor={Exo7}}


%section
% \usepackage{sectsty}
% \allsectionsfont{\bf}
%\sectionfont{\color{Tomato3}\upshape\selectfont}
%\subsectionfont{\color{Tomato4}\upshape\selectfont}

%----- Ensembles : entiers, reels, complexes -----
\newcommand{\Nn}{\mathbb{N}} \newcommand{\N}{\mathbb{N}}
\newcommand{\Zz}{\mathbb{Z}} \newcommand{\Z}{\mathbb{Z}}
\newcommand{\Qq}{\mathbb{Q}} \newcommand{\Q}{\mathbb{Q}}
\newcommand{\Rr}{\mathbb{R}} \newcommand{\R}{\mathbb{R}}
\newcommand{\Cc}{\mathbb{C}} 
\newcommand{\Kk}{\mathbb{K}} \newcommand{\K}{\mathbb{K}}

%----- Modifications de symboles -----
\renewcommand{\epsilon}{\varepsilon}
\renewcommand{\Re}{\mathop{\text{Re}}\nolimits}
\renewcommand{\Im}{\mathop{\text{Im}}\nolimits}
%\newcommand{\llbracket}{\left[\kern-0.15em\left[}
%\newcommand{\rrbracket}{\right]\kern-0.15em\right]}

\renewcommand{\ge}{\geqslant}
\renewcommand{\geq}{\geqslant}
\renewcommand{\le}{\leqslant}
\renewcommand{\leq}{\leqslant}
\renewcommand{\epsilon}{\varepsilon}

%----- Fonctions usuelles -----
\newcommand{\ch}{\mathop{\text{ch}}\nolimits}
\newcommand{\sh}{\mathop{\text{sh}}\nolimits}
\renewcommand{\tanh}{\mathop{\text{th}}\nolimits}
\newcommand{\cotan}{\mathop{\text{cotan}}\nolimits}
\newcommand{\Arcsin}{\mathop{\text{arcsin}}\nolimits}
\newcommand{\Arccos}{\mathop{\text{arccos}}\nolimits}
\newcommand{\Arctan}{\mathop{\text{arctan}}\nolimits}
\newcommand{\Argsh}{\mathop{\text{argsh}}\nolimits}
\newcommand{\Argch}{\mathop{\text{argch}}\nolimits}
\newcommand{\Argth}{\mathop{\text{argth}}\nolimits}
\newcommand{\pgcd}{\mathop{\text{pgcd}}\nolimits} 


%----- Commandes divers ------
\newcommand{\ii}{\mathrm{i}}
\newcommand{\dd}{\text{d}}
\newcommand{\id}{\mathop{\text{id}}\nolimits}
\newcommand{\Ker}{\mathop{\text{Ker}}\nolimits}
\newcommand{\Card}{\mathop{\text{Card}}\nolimits}
\newcommand{\Vect}{\mathop{\text{Vect}}\nolimits}
\newcommand{\Mat}{\mathop{\text{Mat}}\nolimits}
\newcommand{\rg}{\mathop{\text{rg}}\nolimits}
\newcommand{\tr}{\mathop{\text{tr}}\nolimits}


%----- Structure des exercices ------

\newtheoremstyle{styleexo}% name
{2ex}% Space above
{3ex}% Space below
{}% Body font
{}% Indent amount 1
{\bfseries} % Theorem head font
{}% Punctuation after theorem head
{\newline}% Space after theorem head 2
{}% Theorem head spec (can be left empty, meaning ‘normal’)

%\theoremstyle{styleexo}
\newtheorem{exo}{Exercice}
\newtheorem{ind}{Indications}
\newtheorem{cor}{Correction}


\newcommand{\exercice}[1]{} \newcommand{\finexercice}{}
%\newcommand{\exercice}[1]{{\tiny\texttt{#1}}\vspace{-2ex}} % pour afficher le numero absolu, l'auteur...
\newcommand{\enonce}{\begin{exo}} \newcommand{\finenonce}{\end{exo}}
\newcommand{\indication}{\begin{ind}} \newcommand{\finindication}{\end{ind}}
\newcommand{\correction}{\begin{cor}} \newcommand{\fincorrection}{\end{cor}}

\newcommand{\noindication}{\stepcounter{ind}}
\newcommand{\nocorrection}{\stepcounter{cor}}

\newcommand{\fiche}[1]{} \newcommand{\finfiche}{}
\newcommand{\titre}[1]{\centerline{\large \bf #1}}
\newcommand{\addcommand}[1]{}
\newcommand{\video}[1]{}

% Marge
\newcommand{\mymargin}[1]{\marginpar{{\small #1}}}

\def\noqed{\renewcommand{\qedsymbol}{}}


%----- Presentation ------
\setlength{\parindent}{0cm}

%\newcommand{\ExoSept}{\href{http://exo7.emath.fr}{\textbf{\textsf{Exo7}}}}

\definecolor{myred}{rgb}{0.93,0.26,0}
\definecolor{myorange}{rgb}{0.97,0.58,0}
\definecolor{myyellow}{rgb}{1,0.86,0}

\newcommand{\LogoExoSept}[1]{  % input : echelle
{\usefont{U}{cmss}{bx}{n}
\begin{tikzpicture}[scale=0.1*#1,transform shape]
  \fill[color=myorange] (0,0)--(4,0)--(4,-4)--(0,-4)--cycle;
  \fill[color=myred] (0,0)--(0,3)--(-3,3)--(-3,0)--cycle;
  \fill[color=myyellow] (4,0)--(7,4)--(3,7)--(0,3)--cycle;
  \node[scale=5] at (3.5,3.5) {Exo7};
\end{tikzpicture}}
}


\newcommand{\debutmontitre}{
  \author{} \date{} 
  \thispagestyle{empty}
  \hspace*{-10ex}
  \begin{minipage}{\textwidth}
    \titlepage  
  \vspace*{-2.5cm}
  \begin{center}
    \LogoExoSept{2.5}
  \end{center}
  \end{minipage}

  \vspace*{-0cm}
  
  % Astuce pour que le background ne soit pas discrétisé lors de la conversion pdf -> png
\begin{tikzpicture}
        \fill[opacity=0,green!60!black] (0,0)--++(0,0)--++(0,0)--++(0,0)--cycle; 
\end{tikzpicture}

% toc S'affiche trop tot :
% \tableofcontents[hideallsubsections, pausesections]
}

\newcommand{\finmontitre}{
  \end{frame}
  \setcounter{framenumber}{0}
} % ne marche pas pour une raison obscure

%----- Commandes supplementaires ------

% \usepackage[landscape]{geometry}
% \geometry{top=1cm, bottom=3cm, left=2cm, right=10cm, marginparsep=1cm
% }
% \usepackage[a4paper]{geometry}
% \geometry{top=2cm, bottom=2cm, left=2cm, right=2cm, marginparsep=1cm
% }

%\usepackage{standalone}


% New command Arnaud -- november 2011
\setbeamersize{text margin left=24ex}
% si vous modifier cette valeur il faut aussi
% modifier le decalage du titre pour compenser
% (ex : ici =+10ex, titre =-5ex

\theoremstyle{definition}
%\newtheorem{proposition}{Proposition}
%\newtheorem{exemple}{Exemple}
%\newtheorem{theoreme}{Théorème}
%\newtheorem{lemme}{Lemme}
%\newtheorem{corollaire}{Corollaire}
%\newtheorem*{remarque*}{Remarque}
%\newtheorem*{miniexercice}{Mini-exercices}
%\newtheorem{definition}{Définition}

% Commande tikz
\usetikzlibrary{calc}
\usetikzlibrary{patterns,arrows}
\usetikzlibrary{matrix}
\usetikzlibrary{fadings} 

%definition d'un terme
\newcommand{\defi}[1]{{\color{myorange}\textbf{\emph{#1}}}}
\newcommand{\evidence}[1]{{\color{blue}\textbf{\emph{#1}}}}
\newcommand{\assertion}[1]{\emph{\og#1\fg}}  % pour chapitre logique
%\renewcommand{\contentsname}{Sommaire}
\renewcommand{\contentsname}{}
\setcounter{tocdepth}{2}



%------ Figures ------

\def\myscale{1} % par défaut 
\newcommand{\myfigure}[2]{  % entrée : echelle, fichier figure
\def\myscale{#1}
\begin{center}
\footnotesize
{#2}
\end{center}}


%------ Encadrement ------

\usepackage{fancybox}


\newcommand{\mybox}[1]{
\setlength{\fboxsep}{7pt}
\begin{center}
\shadowbox{#1}
\end{center}}

\newcommand{\myboxinline}[1]{
\setlength{\fboxsep}{5pt}
\raisebox{-10pt}{
\shadowbox{#1}
}
}

%--------------- Commande beamer---------------
\newcommand{\beameronly}[1]{#1} % permet de mettre des pause dans beamer pas dans poly


\setbeamertemplate{navigation symbols}{}
\setbeamertemplate{footline}  % tiré du fichier beamerouterinfolines.sty
{
  \leavevmode%
  \hbox{%
  \begin{beamercolorbox}[wd=.333333\paperwidth,ht=2.25ex,dp=1ex,center]{author in head/foot}%
    % \usebeamerfont{author in head/foot}\insertshortauthor%~~(\insertshortinstitute)
    \usebeamerfont{section in head/foot}{\bf\insertshorttitle}
  \end{beamercolorbox}%
  \begin{beamercolorbox}[wd=.333333\paperwidth,ht=2.25ex,dp=1ex,center]{title in head/foot}%
    \usebeamerfont{section in head/foot}{\bf\insertsectionhead}
  \end{beamercolorbox}%
  \begin{beamercolorbox}[wd=.333333\paperwidth,ht=2.25ex,dp=1ex,right]{date in head/foot}%
    % \usebeamerfont{date in head/foot}\insertshortdate{}\hspace*{2em}
    \insertframenumber{} / \inserttotalframenumber\hspace*{2ex} 
  \end{beamercolorbox}}%
  \vskip0pt%
}


\definecolor{mygrey}{rgb}{0.5,0.5,0.5}
\setlength{\parindent}{0cm}
%\DeclareTextFontCommand{\helvetica}{\fontfamily{phv}\selectfont}

% background beamer
\definecolor{couleurhaut}{rgb}{0.85,0.9,1}  % creme
\definecolor{couleurmilieu}{rgb}{1,1,1}  % vert pale
\definecolor{couleurbas}{rgb}{0.85,0.9,1}  % blanc
\setbeamertemplate{background canvas}[vertical shading]%
[top=couleurhaut,middle=couleurmilieu,midpoint=0.4,bottom=couleurbas] 
%[top=fondtitre!05,bottom=fondtitre!60]



\makeatletter
\setbeamertemplate{theorem begin}
{%
  \begin{\inserttheoremblockenv}
  {%
    \inserttheoremheadfont
    \inserttheoremname
    \inserttheoremnumber
    \ifx\inserttheoremaddition\@empty\else\ (\inserttheoremaddition)\fi%
    \inserttheorempunctuation
  }%
}
\setbeamertemplate{theorem end}{\end{\inserttheoremblockenv}}

\newenvironment{theoreme}[1][]{%
   \setbeamercolor{block title}{fg=structure,bg=structure!40}
   \setbeamercolor{block body}{fg=black,bg=structure!10}
   \begin{block}{{\bf Th\'eor\`eme }#1}
}{%
   \end{block}%
}


\newenvironment{proposition}[1][]{%
   \setbeamercolor{block title}{fg=structure,bg=structure!40}
   \setbeamercolor{block body}{fg=black,bg=structure!10}
   \begin{block}{{\bf Proposition }#1}
}{%
   \end{block}%
}

\newenvironment{corollaire}[1][]{%
   \setbeamercolor{block title}{fg=structure,bg=structure!40}
   \setbeamercolor{block body}{fg=black,bg=structure!10}
   \begin{block}{{\bf Corollaire }#1}
}{%
   \end{block}%
}

\newenvironment{mydefinition}[1][]{%
   \setbeamercolor{block title}{fg=structure,bg=structure!40}
   \setbeamercolor{block body}{fg=black,bg=structure!10}
   \begin{block}{{\bf Définition} #1}
}{%
   \end{block}%
}

\newenvironment{lemme}[0]{%
   \setbeamercolor{block title}{fg=structure,bg=structure!40}
   \setbeamercolor{block body}{fg=black,bg=structure!10}
   \begin{block}{\bf Lemme}
}{%
   \end{block}%
}

\newenvironment{remarque}[1][]{%
   \setbeamercolor{block title}{fg=black,bg=structure!20}
   \setbeamercolor{block body}{fg=black,bg=structure!5}
   \begin{block}{Remarque #1}
}{%
   \end{block}%
}


\newenvironment{exemple}[1][]{%
   \setbeamercolor{block title}{fg=black,bg=structure!20}
   \setbeamercolor{block body}{fg=black,bg=structure!5}
   \begin{block}{{\bf Exemple }#1}
}{%
   \end{block}%
}


\newenvironment{miniexercice}[0]{%
   \setbeamercolor{block title}{fg=structure,bg=structure!20}
   \setbeamercolor{block body}{fg=black,bg=structure!5}
   \begin{block}{Mini-exercices}
}{%
   \end{block}%
}


\newenvironment{tp}[0]{%
   \setbeamercolor{block title}{fg=structure,bg=structure!40}
   \setbeamercolor{block body}{fg=black,bg=structure!10}
   \begin{block}{\bf Travaux pratiques}
}{%
   \end{block}%
}
\newenvironment{exercicecours}[1][]{%
   \setbeamercolor{block title}{fg=structure,bg=structure!40}
   \setbeamercolor{block body}{fg=black,bg=structure!10}
   \begin{block}{{\bf Exercice }#1}
}{%
   \end{block}%
}
\newenvironment{algo}[1][]{%
   \setbeamercolor{block title}{fg=structure,bg=structure!40}
   \setbeamercolor{block body}{fg=black,bg=structure!10}
   \begin{block}{{\bf Algorithme}\hfill{\color{gray}\texttt{#1}}}
}{%
   \end{block}%
}


\setbeamertemplate{proof begin}{
   \setbeamercolor{block title}{fg=black,bg=structure!20}
   \setbeamercolor{block body}{fg=black,bg=structure!5}
   \begin{block}{{\footnotesize Démonstration}}
   \footnotesize
   \smallskip}
\setbeamertemplate{proof end}{%
   \end{block}}
\setbeamertemplate{qed symbol}{\openbox}


\makeatother
\usecolortheme[RGB={0,45,179}]{structure}

%%%%%%%%%%%%%%%%%%%%%%%%%%%%%%%%%%%%%%%%%%%%%%%%%%%%%%%%%%%%%
%%%%%%%%%%%%%%%%%%%%%%%%%%%%%%%%%%%%%%%%%%%%%%%%%%%%%%%%%%%%%



\begin{document}



\title{{\bf Intégrales}}
\subtitle{Primitive d'une fonction}

\begin{frame}
  
  \debutmontitre

  \pause

{\footnotesize
\hfill
\setbeamercovered{transparent=50}
\begin{minipage}{0.6\textwidth}
  \begin{itemize}
    \item<3-> Définition
    \item<4-> Primitives des fonctions usuelles
    \item<5-> Relation primitive-intégrale
    \item<6-> Sommes de Riemann
  \end{itemize}
\end{minipage}
}

\end{frame}

\setcounter{framenumber}{0}



%---------------------------------------------------------------
\section*{Définition}


\begin{frame}

\begin{mydefinition}
Soit $f:I \to \Rr$ une fonction définie sur un intervalle $I$ \\
$F : I \to \Rr$ est une \defi{primitive} de $f$ si \\
$F$ est dérivable et $F'(x)=f(x)$ pour tout $x \in I$  
\end{mydefinition}

\pause

\begin{exemple}
\begin{enumerate}
\item 
   \begin{itemize}
     \item Soit $f: \Rr \to \Rr$, $f(x) = x^2$
\pause
     \item Alors $F: \Rr \to \Rr$ définie par $F(x) = \frac{x^3}{3}$ est une primitive de $f$
\pause
     \item Et $F(x)= \frac{x^3}{3}+1$ est aussi une primitive de $f$
   \end{itemize}
\pause
\item 
   \begin{itemize}
     \item Soit $g : [0,+\infty[ \to \Rr$, $g(x)=\sqrt x$
\pause
     \item $G : [0,+\infty[ \to\Rr$ définie par $G(x)=\frac{2}{3} x^{\frac{3}{2}}$
est une primitive de $g$
\pause
     \item Pour tout $c\in \Rr$, la fonction $G+c$ est aussi une primitive de $g$
   \end{itemize}
\end{enumerate}  
\end{exemple}

\end{frame}



\begin{frame}


\begin{proposition}
Si $F$ est une primitive de $f$ alors
toute primitive de $f$ s'écrit \\
\hfil\hfil $G=F+c$ \quad où $c\in \Rr$
\end{proposition}

\pause

\begin{proof}
\begin{itemize}
  \item Si $G(x)=F(x)+c$ alors $G'(x)=F'(x)$, \pause donc $G'(x)=f(x)$\\ \pause
ainsi $G$ est bien une primitive de $f$
\pause
  \item Si $G$ est une primitive quelconque de $f$ alors  \\
\hfil \hfil $(G-F)'(x)\pause=G'(x)-F'(x)\pause=f(x)-f(x)=0$ \\
\pause donc $G-F$ est une fonction constante. \pause Il existe $c\in \Rr$ tel que $(G-F)(x)=c$.
Ainsi $G(x)=F(x)+c$
\end{itemize}
\vspace*{-2ex}
\end{proof}

\pause
\textbf{Notations} $\int f(t) \; dt$ désigne une primitive

\pause
\begin{itemize}
  \item Autres notations : $\int f(x) \; dx$,  $\int f(u) \; du$, $\int f$
\pause
  \item Si $F$ est une primitive de $f$ alors $F=\int f(t) \; dt + c$
\pause
  \item $\int f(t)\;dt$ est une fonction de $I$ dans $\Rr$
\pause
  \item $\int_a^b f(t) \; dt$ désigne un nombre réel
\end{itemize}
\end{frame}



\begin{frame}
\begin{proposition}
Soient $F$ une primitive de $f$ et $G$ une primitive de $g$ et $\lambda \in \Rr$
\begin{itemize}
  \item $F+G$ est une primitive de $f+g$
\pause
  \item $\lambda F$ est une primitive de $\lambda f$
\end{itemize}
\end{proposition}
  
\bigskip
\pause


\mybox{$\displaystyle\int\big(\lambda f(t)+ \mu g(t)\big) \; dt=\lambda \int f(t) \; dt+\mu \int g(t)\; dt$}
\end{frame}





%---------------------------------------------------------------
\section*{Primitives des fonctions usuelles}


\begin{frame}
\begin{center}
%\noindent
\setlength{\arrayrulewidth}{0.05mm}
\begin{tabular}{c@{\vrule depth 3ex height 4ex width 0mm \ }}  
\hline
   $\int e^x \; dx  = e^x + c$  \quad sur $\Rr$ \\ \hline
\pause
   $\int \cos x \; dx  = \sin x  + c$  \quad sur $\Rr$ \\ \hline
\pause
   $\int \sin x \; dx  = -\cos x  + c$  \quad sur $\Rr$ \\ \hline
\pause
   $\int x^n \; dx = \frac{x^{n+1}}{n+1} + c$  \quad ($n \in \Nn$)  \quad sur $\Rr$ \\ \hline
\pause
   $\int x^\alpha \; dx = \frac{x^{\alpha+1}}{\alpha+1} + c$  \quad ($\alpha \in \Rr\setminus\{-1\}$)  sur $]0,+\infty[$\\ \hline
\pause
   $\int \frac 1x \; dx  = \ln |x|  + c$  \quad sur $]0,+\infty[$ ou $]-\infty,0[$ \\ \hline
\end{tabular} 
\end{center}
\end{frame}



\begin{frame}

\begin{center}
\begin{tabular}{c@{\vrule depth 3ex height 4ex width 0mm \ }} 
\hline
 $\int\sh x \; dx=\ch x+c$ \quad  $\int \ch x \; dx=\sh x+c$ \quad sur $\Rr$ \\ \hline
\pause
   $\int \frac{dx}{1+x^2}= \arctan x+c$ \quad sur $\Rr$ \\ \hline
\pause
   $\int\frac{dx}{\sqrt{1-x^2}} = \left\{ \begin{array}{l}
   \arcsin x + c \\ \frac\pi2-\arccos x +c \end{array} \right.$ \quad  sur $]-1,1[$ \\ \hline
\pause
   $\int \frac{dx}{\sqrt {x^2+1}}=  \left\{ \begin{array}{l} \mbox{argsh} x+c \\
   \ln\big(x+\sqrt{x^2+1}\big)+c  \end{array} \right.$ \quad sur $\Rr$ \\ \hline
\pause
   $\int \frac{dx}{\sqrt {x^2-1}} = \left\{ \begin{array}{l} \mbox{argch} x+c \\ 
   \ln\big(x+\sqrt{x^2-1}\big)+c \end{array} \right.$ \quad sur $x\in ]1,+ \infty[$\\ \hline
\end{tabular} 
\end{center}

\end{frame}




%---------------------------------------------------------------
\section*{Relation primitive-intégrale}


\begin{frame}
\begin{theoreme}
Soit $f : [a,b] \to \Rr$ continue. Alors $F:I \to \Rr$ définie par
\vspace*{-0.7ex}
\mybox{$\displaystyle F(x)=\int_a^x f(t) \; dt$}
\vspace*{-0.7ex}
est une primitive de $f$, c'est-à-dire $F$ est dérivable et $F'(x)=f(x)$

\pause

Ainsi pour une primitive $F$ quelconque de $f$ :
\vspace*{-0.7ex}
\mybox{$\displaystyle \int_a^b f(t) \; dt = F(b)-F(a)$}
\end{theoreme}

\pause

\textbf{Notation.}  $\big[F(x)\big]_a^b=F(b)-F(a)$

\pause

\begin{itemize}
  \item $F(x)=\int_a^x f(t) \; dt$ est \evidence{l'unique primitive de $f$ qui s'annule en $a$}

\pause

  \item Si $F$ est de classe $\mathcal{C}^1$ alors \myboxinline{$\int_a^b F'(t) \; dt = F(b)-F(a)$}
\end{itemize}  
\end{frame}



\begin{frame}
\begin{exemple}

\begin{enumerate}
  \item $f(x)=e^x$,\quad \pause $F(x)=e^x$,\quad \pause $\int_0^1 e^x \; dx =  \big[e^x\big]_0^1=e^1-e^0=e-1$
\pause
\medskip

  \item $g(x)=x^2$,\quad \pause $G(x)=\frac{x^3}{3}$,\quad \pause $\int_0^1 x^2 \; dx =  \big[\tfrac{x^3}{3}\big]_0^1=\tfrac{1}{3}$
\pause
\medskip

  \item $\int_a^x \cos t \; dt \pause= \big[ \sin t\big]_{t=a}^{t=x} = \sin x - \sin a$
\pause
\medskip

  \item Si $f$ est impaire alors ses primitives sont paires. \pause Et $\int_{-a}^a f(t)  \; dt = 0$
\end{enumerate}
\end{exemple}

\end{frame}


%---------------------------------------------------------------
\section*{Sommes de Riemann}


\begin{frame}

\begin{theoreme}[Somme de Riemann]
\ 
\mybox{$\displaystyle S_n = \tfrac{b-a}{n} \sum_{k=1}^{n} f\big(a+k\tfrac{b-a}{n} \big) 
\qquad \xrightarrow[n\to+\infty]{} \qquad \int_a^b f(x) \; dx$}
\end{theoreme}


\pause
\bigskip 

\begin{minipage}{0.6\textwidth}
\vspace*{0.8cm}
Cas fréquent :  $a=0$, $b=1$ 
$$S_n = \tfrac{1}{n} \sum_{k=1}^{n} f\big(\tfrac kn \big) 
\quad \xrightarrow[n\to+\infty]{} \quad \int_0^1 f(x) \; dx$$  

\end{minipage}
\begin{minipage}{0.39\textwidth}
\myfigure{0.9}{
\tikzinput{fig_int10} 
}   
\end{minipage}

 
\end{frame}



\begin{frame}
\begin{exemple}
Calculer la limite de $S_n= \sum_{k=1}^{n} \frac1{n+k}$

\pause
\begin{itemize}
  \item $S_1=\frac12$, \pause $S_2=\frac13+\frac14$, \pause  $S_3=\frac14+\frac15+\frac16$, \pause
$S_4=\frac15+\frac16+\frac17+\frac18$, \ldots
\pause
\smallskip
  \item $S_n = \frac{1}{n}  \sum_{k=1}^{n} \frac1{1+\frac kn}$
\pause
\smallskip
  \item  $S_n$ est une somme de Riemann pour $f(x)=\frac{1}{1+x}$, $a=0$ et $b=1$
\pause
\smallskip
  \item $\displaystyle S_n=\tfrac{1}{n} \sum_{k=1}^{n} \frac1{1+\frac kn}=\tfrac{1}{n} \sum_{k=1}^{n} f\big(\tfrac kn\big)\pause
\xrightarrow[n\to+\infty]{} \int_a^b f(x) \; dx $
\pause
\smallskip
  \item $\int_a^b f(x) \; dx =\int_0^1 \frac{1}{1+x} \; dx \pause =\big[\ln|1+x|\big]_0^1 = \ln 2-\ln 1 = \ln 2$
\pause
\smallskip
  \item $S_n\xrightarrow[n\to+\infty]{} \ln 2$
\end{itemize}
\end{exemple}
\end{frame}





%%%%%%%%%%%%%%%%%%%%%%%%%%%%%%%%%%%%%%%%%%%%%%%%%%%%%%%%%%%%%%%%
\section*{Mini-exercices}


\begin{frame}
\begin{miniexercice}
\begin{enumerate}
  \item Trouver les primitives des fonctions : $x^3-x^7$, $\cos x+\exp x$, $\sin(2x)$, $1+\sqrt{x}+x$, 
$\frac{1}{\sqrt x}$, $\sqrt[3]{x}$, $\frac{1}{x+1}$.
  \item Trouver les primitives des fonctions : $\ch (x)-\sh (\frac{x}{2})$, $\frac{1}{1+4x^2}$,
$\frac{1}{\sqrt {1+x^2}} - \frac{1}{\sqrt {1-x^2}}$.
  \item Trouver une primitive de $x^2e^x$ sous la forme $(a x^2+b x+c)e^x$.
  \item Trouver toutes les primitives de $x\mapsto \frac{1}{x^2}$ (préciser les intervalles et les constantes).
  \item Calculer les intégrales $\int_0^1 x^n \; dx$, $\int_0^{\frac\pi4} \frac{dx}{1+x^2}$, 
$\int_1^e \frac{1-x}{x^2}\; dx$, $\int_0^{\frac12} \frac{dx}{x^2-1}$.
  \item Calculer la limite (lorsque $n\to+\infty$) de la somme
$S_n = \sum_{k=0}^n \frac{e^{k/n}}{n}$. Idem avec $S'_n = \sum_{k=0}^n \frac{n}{(n+k)^2}$.
\end{enumerate}
\end{miniexercice}
\end{frame}


\end{document}