
%%%%%%%%%%%%%%%%%% PREAMBULE %%%%%%%%%%%%%%%%%%


\documentclass[12pt]{article}

\usepackage{amsfonts,amsmath,amssymb,amsthm}
\usepackage[utf8]{inputenc}
\usepackage[T1]{fontenc}
\usepackage[francais]{babel}


% packages
\usepackage{amsfonts,amsmath,amssymb,amsthm}
\usepackage[utf8]{inputenc}
\usepackage[T1]{fontenc}
%\usepackage{lmodern}

\usepackage[francais]{babel}
\usepackage{fancybox}
\usepackage{graphicx}

\usepackage{float}

%\usepackage[usenames, x11names]{xcolor}
\usepackage{tikz}
\usepackage{datetime}

\usepackage{mathptmx}
%\usepackage{fouriernc}
%\usepackage{newcent}
\usepackage[mathcal,mathbf]{euler}

%\usepackage{palatino}
%\usepackage{newcent}


% Commande spéciale prompteur

%\usepackage{mathptmx}
%\usepackage[mathcal,mathbf]{euler}
%\usepackage{mathpple,multido}

\usepackage[a4paper]{geometry}
\geometry{top=2cm, bottom=2cm, left=1cm, right=1cm, marginparsep=1cm}

\newcommand{\change}{{\color{red}\rule{\textwidth}{1mm}\\}}

\newcounter{mydiapo}

\newcommand{\diapo}{\newpage
\hfill {\normalsize  Diapo \themydiapo \quad \texttt{[\jobname]}} \\
\stepcounter{mydiapo}}


%%%%%%% COULEURS %%%%%%%%%%

% Pour blanc sur noir :
%\pagecolor[rgb]{0.5,0.5,0.5}
% \pagecolor[rgb]{0,0,0}
% \color[rgb]{1,1,1}



%\DeclareFixedFont{\myfont}{U}{cmss}{bx}{n}{18pt}
\newcommand{\debuttexte}{
%%%%%%%%%%%%% FONTES %%%%%%%%%%%%%
\renewcommand{\baselinestretch}{1.5}
\usefont{U}{cmss}{bx}{n}
\bfseries

% Taille normale : commenter le reste !
%Taille Arnaud
%\fontsize{19}{19}\selectfont

% Taille Barbara
%\fontsize{21}{22}\selectfont

%Taille François
\fontsize{25}{30}\selectfont

%Taille Pascal
%\fontsize{25}{30}\selectfont

%Taille Laura
%\fontsize{30}{35}\selectfont


%\myfont
%\usefont{U}{cmss}{bx}{n}

%\Huge
%\addtolength{\parskip}{\baselineskip}
}


% \usepackage{hyperref}
% \hypersetup{colorlinks=true, linkcolor=blue, urlcolor=blue,
% pdftitle={Exo7 - Exercices de mathématiques}, pdfauthor={Exo7}}


%section
% \usepackage{sectsty}
% \allsectionsfont{\bf}
%\sectionfont{\color{Tomato3}\upshape\selectfont}
%\subsectionfont{\color{Tomato4}\upshape\selectfont}

%----- Ensembles : entiers, reels, complexes -----
\newcommand{\Nn}{\mathbb{N}} \newcommand{\N}{\mathbb{N}}
\newcommand{\Zz}{\mathbb{Z}} \newcommand{\Z}{\mathbb{Z}}
\newcommand{\Qq}{\mathbb{Q}} \newcommand{\Q}{\mathbb{Q}}
\newcommand{\Rr}{\mathbb{R}} \newcommand{\R}{\mathbb{R}}
\newcommand{\Cc}{\mathbb{C}} 
\newcommand{\Kk}{\mathbb{K}} \newcommand{\K}{\mathbb{K}}

%----- Modifications de symboles -----
\renewcommand{\epsilon}{\varepsilon}
\renewcommand{\Re}{\mathop{\text{Re}}\nolimits}
\renewcommand{\Im}{\mathop{\text{Im}}\nolimits}
%\newcommand{\llbracket}{\left[\kern-0.15em\left[}
%\newcommand{\rrbracket}{\right]\kern-0.15em\right]}

\renewcommand{\ge}{\geqslant}
\renewcommand{\geq}{\geqslant}
\renewcommand{\le}{\leqslant}
\renewcommand{\leq}{\leqslant}

%----- Fonctions usuelles -----
\newcommand{\ch}{\mathop{\mathrm{ch}}\nolimits}
\newcommand{\sh}{\mathop{\mathrm{sh}}\nolimits}
\renewcommand{\tanh}{\mathop{\mathrm{th}}\nolimits}
\newcommand{\cotan}{\mathop{\mathrm{cotan}}\nolimits}
\newcommand{\Arcsin}{\mathop{\mathrm{Arcsin}}\nolimits}
\newcommand{\Arccos}{\mathop{\mathrm{Arccos}}\nolimits}
\newcommand{\Arctan}{\mathop{\mathrm{Arctan}}\nolimits}
\newcommand{\Argsh}{\mathop{\mathrm{Argsh}}\nolimits}
\newcommand{\Argch}{\mathop{\mathrm{Argch}}\nolimits}
\newcommand{\Argth}{\mathop{\mathrm{Argth}}\nolimits}
\newcommand{\pgcd}{\mathop{\mathrm{pgcd}}\nolimits} 

\newcommand{\Card}{\mathop{\text{Card}}\nolimits}
\newcommand{\Ker}{\mathop{\text{Ker}}\nolimits}
\newcommand{\id}{\mathop{\text{id}}\nolimits}
\newcommand{\ii}{\mathrm{i}}
\newcommand{\dd}{\mathrm{d}}
\newcommand{\Vect}{\mathop{\text{Vect}}\nolimits}
\newcommand{\Mat}{\mathop{\mathrm{Mat}}\nolimits}
\newcommand{\rg}{\mathop{\text{rg}}\nolimits}
\newcommand{\tr}{\mathop{\text{tr}}\nolimits}
\newcommand{\ppcm}{\mathop{\text{ppcm}}\nolimits}

%----- Structure des exercices ------

\newtheoremstyle{styleexo}% name
{2ex}% Space above
{3ex}% Space below
{}% Body font
{}% Indent amount 1
{\bfseries} % Theorem head font
{}% Punctuation after theorem head
{\newline}% Space after theorem head 2
{}% Theorem head spec (can be left empty, meaning ‘normal’)

%\theoremstyle{styleexo}
\newtheorem{exo}{Exercice}
\newtheorem{ind}{Indications}
\newtheorem{cor}{Correction}


\newcommand{\exercice}[1]{} \newcommand{\finexercice}{}
%\newcommand{\exercice}[1]{{\tiny\texttt{#1}}\vspace{-2ex}} % pour afficher le numero absolu, l'auteur...
\newcommand{\enonce}{\begin{exo}} \newcommand{\finenonce}{\end{exo}}
\newcommand{\indication}{\begin{ind}} \newcommand{\finindication}{\end{ind}}
\newcommand{\correction}{\begin{cor}} \newcommand{\fincorrection}{\end{cor}}

\newcommand{\noindication}{\stepcounter{ind}}
\newcommand{\nocorrection}{\stepcounter{cor}}

\newcommand{\fiche}[1]{} \newcommand{\finfiche}{}
\newcommand{\titre}[1]{\centerline{\large \bf #1}}
\newcommand{\addcommand}[1]{}
\newcommand{\video}[1]{}

% Marge
\newcommand{\mymargin}[1]{\marginpar{{\small #1}}}



%----- Presentation ------
\setlength{\parindent}{0cm}

%\newcommand{\ExoSept}{\href{http://exo7.emath.fr}{\textbf{\textsf{Exo7}}}}

\definecolor{myred}{rgb}{0.93,0.26,0}
\definecolor{myorange}{rgb}{0.97,0.58,0}
\definecolor{myyellow}{rgb}{1,0.86,0}

\newcommand{\LogoExoSept}[1]{  % input : echelle
{\usefont{U}{cmss}{bx}{n}
\begin{tikzpicture}[scale=0.1*#1,transform shape]
  \fill[color=myorange] (0,0)--(4,0)--(4,-4)--(0,-4)--cycle;
  \fill[color=myred] (0,0)--(0,3)--(-3,3)--(-3,0)--cycle;
  \fill[color=myyellow] (4,0)--(7,4)--(3,7)--(0,3)--cycle;
  \node[scale=5] at (3.5,3.5) {Exo7};
\end{tikzpicture}}
}



\theoremstyle{definition}
%\newtheorem{proposition}{Proposition}
%\newtheorem{exemple}{Exemple}
%\newtheorem{theoreme}{Théorème}
\newtheorem{lemme}{Lemme}
\newtheorem{corollaire}{Corollaire}
%\newtheorem*{remarque*}{Remarque}
%\newtheorem*{miniexercice}{Mini-exercices}
%\newtheorem{definition}{Définition}




%definition d'un terme
\newcommand{\defi}[1]{{\color{myorange}\textbf{\emph{#1}}}}
\newcommand{\evidence}[1]{{\color{blue}\textbf{\emph{#1}}}}



 %----- Commandes divers ------

\newcommand{\codeinline}[1]{\texttt{#1}}

%%%%%%%%%%%%%%%%%%%%%%%%%%%%%%%%%%%%%%%%%%%%%%%%%%%%%%%%%%%%%
%%%%%%%%%%%%%%%%%%%%%%%%%%%%%%%%%%%%%%%%%%%%%%%%%%%%%%%%%%%%%



\begin{document}

\debuttexte

%%%%%%%%%%%%%%%%%%%%%%%%%%%%%%%%%%%%%%%%%%%%%%%%%%%%%%%%%%%
\diapo

Nous allons voir que le moyen le plus simple pour calculer une intégrale
est de trouver d'abord une primitive, lorsque cela est possible bien sûr !

\change

\change

On commence par la définition de ce qu'est une primitive,

\change

Puis nous verrons la liste des primitives des fonctions usuelles

\change

Ensuite on montre comment calculer une intégrale à partir d'une primitive.

\change

On termine avec les sommes de Riemann
qui sont des limites de sommes obtenues à partir d'un calcul d'intégrale.


%%%%%%%%%%%%%%%%%%%%%%%%%%%%%%%%%%%%%%%%%%%%%%%%%%%%%%%%%%%
\diapo

[[petit f, grand F]]

Soit $f:I \to \Rr$ une fonction définie sur un intervalle $I$ quelconque. 


On dit que $F : I \to \Rr$ est une \defi{primitive} de $f$ sur $I$ si

tout d'abord 
$F$ est une fonction dérivable sur $I$ 

et ensuite $F'(x)=f(x)$ pour tout $x \in I$. 

La dérivée de $F$ est $f$.

\change


Voyons deux exemples :

Notons $f$ la fonction $x^2$, ici l'intervalle de définition est $\Rr$ tout entier.

\change

Alors la fonction  $F$ définie par $F(x) = \frac{x^3}{3}$ est une primitive de $f$

en effet lorsque l'on dérive $F$ on retrouve $f$.

\change

Notez que l'on aurait aussi pu prendre $F(x)= \frac{x^3}{3}+1$ comme primitive de $f$


\change

Soit maintenant $g$ la fonction racine carrée, définie sur l'intervalle $[0,+\infty[$.

\change

Alors  $G(x)=\frac{2}{3} x^{\frac{3}{2}}$ est une primitive de $g$

car encore une fois lorsque l'on dérive $G$ on trouve $g$.

\change

Bien sûr si on prend la fonction $G+c$ où $c$ est une constante,
alors comme la dérivée d'une constante est nulle,
lorsque l'on dérive on retrouve $g$.  

Ainsi $G+c$ est aussi une primitive de $g$.


%%%%%%%%%%%%%%%%%%%%%%%%%%%%%%%%%%%%%%%%%%%%%%%%%%%%%%%%%%%
\diapo

Nous allons voir que trouver une primitive permet de les trouver toutes.

[[petit f, grand F]]

Soit $f$ une fonction et soit $F$ une primitive de $f$. 

Toute primitive de $f$ s'écrit sous la forme $G=F+c$ où $c$ est une constante réelle.

\change

La preuve n'est pas difficile,

remarquons d'abord que 
 si l'on note $G$ la fonction définie par $G(x)=F(x)+c$ alors
$G'(x)=F'(x)$ 

\change

mais comme $F'(x)=f(x)$ alors $G'(x)=f(x)$ 

\change

et $G$ est bien une primitive de $f$.

\change

Pour la réciproque supposons que $G$ soit une primitive quelconque de $f$.
Alors $(G-F)'(x)$

\change

$=G'(x)-F'(x)$

Mais comme $G$ et $F$ sont des primitives de $f$ alors ces deux dérivées valent $f(x)$

\change

\change

ainsi la fonction $G-F$ a une dérivée nulle sur un intervalle, c'est donc une fonction constante!

\change

Il existe donc $c\in \Rr$ tel que $G(x)=F(x)+c$ (pour tout $x\in I$).


(pause)


\change


On notera une primitive de $f$ par $\int f(t) \; dt$  

\change

Cette notation est standard mais elle prête un peu à confusion.

Tout d'abord on peut indifféremment noter cette primitive par 
$\int f(x) \; dx$ ou $\int f(u) \; du$ ou simplement par~$\int f$

car les lettres $t, x, u, ...$ sont  des variables muettes
c'est-à-dire interchangeables. 

\change


Cette proposition [[montrer]] nous dit que si $F$ est une primitive de $f$ 
alors il existe un réel $c$, tel que $F=\int f(t) \; dt + c$. 

\change

Voici une dernière mise en garde : $\int f(t)\;dt$ désigne une fonction de $I$ dans $\Rr$

\change

alors que l'intégrale $\int_a^b f(t) \; dt$ désigne un nombre réel.

Nous verrons dans très peu de temps que bien sûr primitives et intégrales sont étroitement liées.

%%%%%%%%%%%%%%%%%%%%%%%%%%%%%%%%%%%%%%%%%%%%%%%%%%%%%%%%%%%
\diapo

Par dérivation on prouve facilement le résultat suivant :

Soit $F$ une primitive de $f$ et $G$ une primitive de $g$. 

Alors $F+G$ est une
primitive de $f+g$.

\change

Si maintenant $\lambda$ est un réel alors $\lambda F$ est une primitive de $\lambda f$.

\change

Une autre formulation est de dire que pour tous réels $\lambda,\mu$ on a 
: 

$\int\big(\lambda f(t)+ \mu g(t)\big) \; dt=\lambda \int f(t) \; dt+\mu \int g(t)\; dt$

Qui bien sûr est à mettre en parallèle avec la linéarité de l'intégrale !


%%%%%%%%%%%%%%%%%%%%%%%%%%%%%%%%%%%%%%%%%%%%%%%%%%%%%%%%%%%
\diapo

Voici maintenant une liste de primitives de fonctions classiques

   $\int e^x \; dx  = e^x + c$

\change

   $\int \cos x \; dx  = \sin x  + c$

\change

   $\int \sin x \; dx  = -\cos x  + c$  

\change

   $\int x^n \; dx = \frac{x^{n+1}}{n+1} + c$  où $n$ est un entier positif

\change

Et plus généralement si $\alpha$ est un réel différent de $-1$

   $\int x^\alpha \; dx = \frac{x^{\alpha+1}}{\alpha+1} + c$  


\change

Le cas $\alpha=-1$ est fondamentalement différent et essentiel :

   $\int \frac 1x \; dx  = \ln |x|  + c$  



(pause)

Ces formules sont à savoir par coeur mais en fait elles découlent 
directement des formules de dérivation que vous connaissez déjà.

(pause)


Voici comment bien lire ce tableau. Si par exemple $f$ est la fonction définie sur $\Rr$ par $f(x)=x^n$
alors la fonction : $x \mapsto \frac{x^{n+1}}{n+1}$ est une primitive de $f$ sur $\Rr$. Les primitives
de $f$ sont les fonctions définies par $x \mapsto \frac{x^{n+1}}{n+1}+c$ (pour $c$ une constante réelle quelconque).

% (pause)
% 
% Souvenez vous que la variable sous le symbole intégrale est une variable muette. On peut aussi bien écrire
% $\int t^n \; dt = \frac{x^{n+1}}{n+1}+c$.


(pause)

La constante est définie pour un intervalle. Si l'on a deux intervalles, il y a 
deux constantes qui peuvent être différentes. Par exemple pour $\int \frac 1x \; dx$ nous avons deux domaines de validité :
et une constante sur chacun des intervalles.




%%%%%%%%%%%%%%%%%%%%%%%%%%%%%%%%%%%%%%%%%%%%%%%%%%%%%%%%%%%
\diapo

On continue avec 

   $\int\sh x \; dx=\ch x+c$

et 

   $\int \ch x \; dx=\sh x+c$ 

\change

   $\int \frac{dx}{1+x^2}= \arctan x+c$

\change

   $\int\frac{dx}{\sqrt{1-x^2}} = \left\{ \begin{array}{l}
   \arcsin x + c \\ \frac\pi2-\arccos x +c \end{array} \right.$ 

\change

   $\int \frac{dx}{\sqrt {x^2+1}}=  \left\{ \begin{array}{l} \mbox{argsh} x+c \\
   \ln\big(x+\sqrt{x^2+1}\big)+c  \end{array} \right.$ 

\change

   $\int \frac{dx}{\sqrt {x^2-1}} = \left\{ \begin{array}{l} \mbox{argch} x+c \\ 
   \ln\big(x+\sqrt{x^2-1}\big)+c \end{array} \right.$

Vous remarquez que l'on peut trouver des primitives aux allures très différentes par exemple $x\mapsto \arcsin x$ et 
$x\mapsto \frac{\pi}{2}-\arccos x$ sont deux primitives de la même fonction $x\mapsto \frac{1}{\sqrt{1-x^2}}$.

Mais bien sûr on sait que $\arcsin x + \arccos x = \frac\pi2$, donc les primitives diffèrent bien d'une constante !

%%%%%%%%%%%%%%%%%%%%%%%%%%%%%%%%%%%%%%%%%%%%%%%%%%%%%%%%%%%
\diapo

[[petit f, grand F]]

Voici le théorème le plus important de cette séquence : 
le lien entre l'intégrale d'une fonction et ses primtives.


Soit $f : [a,b] \to \Rr$ une fonction continue.
La fonction $F:I \to \Rr$ définie par
$\displaystyle F(x)=\int_a^x f(t) \; dt$
est une primitive de $f$, c'est-à-dire $F$ est dérivable et $F'(x)=f(x)$.


\change

Une fois que l'on a une primitive on les a toutes, donc une autre formulation de ce résultat
est que si maintenant $F$ est une primitive *quelconque* de $f$
alors on a 
$\int_a^b f(t) \; dt = F(b)-F(a)$


\change

On va noter $\big[F(x)\big]_a^b$ la valeur $F(b)-F(a)$.
Donc l'intégrale de petit $f$ entre $a$ et $b$ est le crochet de grand $F$ entre $a$ et $b$.

\change

On peut même être un peu plus précis 
$F(x)=\int_a^x f(t) \; dt$ est même l'unique primitive de $f$ qui s'annule en $a$.


\change

Enfin une dernière formulation du théorème dit que si $F$ est une fonction de classe $\mathcal{C}^1$ alors

$\int_a^b F'(t) \; dt = F(b)-F(a)$


Cela exprime bien qu'intégrer une fonction est l'opération inverse de dériver.


%%%%%%%%%%%%%%%%%%%%%%%%%%%%%%%%%%%%%%%%%%%%%%%%%%%%%%%%%%%
\diapo



Le théorème précédent est un fantastique moyen de calculer des intégrales :
 Recalculons d'abord des intégrales déjà rencontrées.

[petit f, grand F]

\change

 Pour $f(x)=e^x$ 

\change

une primitive est $F(x)=e^x$ donc

\change

$\int_0^1 e^x \; dx =  \big[e^x\big]_0^1=e^1-e^0=e-1.$

\change

Pour $g(x)=x^2$ 

\change

une primitive est $G(x)=\frac{x^3}{3}$ 


\change

donc $\int_0^1 x^2 \; dx =  \big[\tfrac{x^3}{3}\big]_0^1=\tfrac{1}{3}.$

\change 

Continuons avec de nouveaux exemples :

\change

Une primitive de $\cos t$ est $\sin t$.

Donc 
$\int_a^x \cos t \; dt = \big[ \sin t\big]_{t=a}^{t=x}$


c'à-d $= \sin x - \sin a$

% En tant que fonction de $x$, $\int_a^x \cos t \; dt$ est une primitive de $\cos x$.

\change

Enfin je vous laisse vérifier que si un fonction $f$ est impaire alors 
ses primitives sont paires.

\change

Et que l'on en plus $\int_{-a}^a f(t)  \; dt = 0$.


%%%%%%%%%%%%%%%%%%%%%%%%%%%%%%%%%%%%%%%%%%%%%%%%%%%%%%%%%%%
\diapo

Terminons avec la formule des sommes de Riemann.

Nous avons défini l'intégrale comme étant une limite de sommes (rappelez-vous les aires de rectangles 
avec une base de plus en plus petites)


Mais maintenant que nous savons calculer des intégrales à l'aide des primitives 
on peut faire le cheminement inverse : calculer des limites de sommes à partir d'intégrales.

Dans cette formule on part d'une fonction $f$ sur un intervalle $a,b$.
On l'évalue en $n$ points répartis régulièrement, les $a+k\tfrac{b-a}{n}$
alors $S_n = \tfrac{b-a}{n} \sum_{k=1}^{n} f\big(a+k\tfrac{b-a}{n} \big) 
\qquad \xrightarrow[n\to+\infty]{} \qquad \int_a^b f(x) \; dx$

On applique cette formule lorsque l'on sait calculer l'intégrale et que l'on cherche à connaître la limite des sommes.


\change

Le cas le plus courant c'est pour $a=0$, $b=1$ alors $\frac{b-a}{n}=\frac1n$ et 
$f\big(a+k\frac{b-a}{n}\big) = f\big(\frac kn\big)$.


La somme $S_n$ est alors $\tfrac{1}{n} \sum_{k=1}^{n} f\big(\tfrac kn \big)$
et tend vers $\int_0^1 f(x) \; dx$



%%%%%%%%%%%%%%%%%%%%%%%%%%%%%%%%%%%%%%%%%%%%%%%%%%%%%%%%%%%
\diapo

Voyons les sommes de Riemann en action  !

On souhaite calculer la limite de la somme $S_n= \sum_{k=1}^{n} \frac1{n+k}$.


\change

Tout d'abord calculons les premiers termes  $S_1=\frac12$, 

$S_2=\frac13+\frac14$,

\change

$S_3=\frac14+\frac15+\frac16$,

\change
 
$S_4=\frac15+\frac16+\frac17+\frac18$,\ldots

Il n'est pas du tout évident de deviner la limite !

\change

Mais réécrivons cette somme sous la forme  $S_n = \frac{1}{n}  \sum_{k=1}^{n} \frac1{1+\frac kn}$.

Cela ressemble furieusement à une somme de Riemann.

\change


En effet en posant $f(x)=\frac{1}{1+x}$, $a=0$ et $b=1$,
on reconnaît que $S_n$ est une somme de Riemann.

\change

$S_n=\tfrac{1}{n} \sum_{k=1}^{n} f\big(\tfrac kn\big)$

\change

Donc lorsque $n \to +\infty$ alors 
$S_n$ tend vers $\int_a^b f(x) \; dx$

\change

Il ne reste plus qu'a calculer l'intégrale 
ici $a=0$ et $b=1$, $f(x)=\frac{1}{1+x}$

Donc la limite est $\int_0^1 \frac{1}{1+x} \; dx$

\change 

Une primitive de $\frac{1}{1+x}$ est $\ln |1+x|$

Donc l'intégrale vaut 
$\big[\ln|1+x|\big]_0^1 = \ln 2-\ln 1 = \ln 2.$

\change 

Ainsi, lorsque $n\to +\infty$, $S_n \to \ln 2$.


%%%%%%%%%%%%%%%%%%%%%%%%%%%%%%%%%%%%%%%%%%%%%%%%%%%%%%%%%%%
\diapo

Voici une série de petit exercices pour vérifier si vous avez assimiler le cours.


\end{document}