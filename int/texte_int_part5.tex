
%%%%%%%%%%%%%%%%%% PREAMBULE %%%%%%%%%%%%%%%%%%


\documentclass[12pt]{article}

\usepackage{amsfonts,amsmath,amssymb,amsthm}
\usepackage[utf8]{inputenc}
\usepackage[T1]{fontenc}
\usepackage[francais]{babel}


% packages
\usepackage{amsfonts,amsmath,amssymb,amsthm}
\usepackage[utf8]{inputenc}
\usepackage[T1]{fontenc}
%\usepackage{lmodern}

\usepackage[francais]{babel}
\usepackage{fancybox}
\usepackage{graphicx}

\usepackage{float}

%\usepackage[usenames, x11names]{xcolor}
\usepackage{tikz}
\usepackage{datetime}

\usepackage{mathptmx}
%\usepackage{fouriernc}
%\usepackage{newcent}
\usepackage[mathcal,mathbf]{euler}

%\usepackage{palatino}
%\usepackage{newcent}


% Commande spéciale prompteur

%\usepackage{mathptmx}
%\usepackage[mathcal,mathbf]{euler}
%\usepackage{mathpple,multido}

\usepackage[a4paper]{geometry}
\geometry{top=2cm, bottom=2cm, left=1cm, right=1cm, marginparsep=1cm}

\newcommand{\change}{{\color{red}\rule{\textwidth}{1mm}\\}}

\newcounter{mydiapo}

\newcommand{\diapo}{\newpage
\hfill {\normalsize  Diapo \themydiapo \quad \texttt{[\jobname]}} \\
\stepcounter{mydiapo}}


%%%%%%% COULEURS %%%%%%%%%%

% Pour blanc sur noir :
%\pagecolor[rgb]{0.5,0.5,0.5}
% \pagecolor[rgb]{0,0,0}
% \color[rgb]{1,1,1}



%\DeclareFixedFont{\myfont}{U}{cmss}{bx}{n}{18pt}
\newcommand{\debuttexte}{
%%%%%%%%%%%%% FONTES %%%%%%%%%%%%%
\renewcommand{\baselinestretch}{1.5}
\usefont{U}{cmss}{bx}{n}
\bfseries

% Taille normale : commenter le reste !
%Taille Arnaud
%\fontsize{19}{19}\selectfont

% Taille Barbara
%\fontsize{21}{22}\selectfont

%Taille François
\fontsize{25}{30}\selectfont

%Taille Pascal
%\fontsize{25}{30}\selectfont

%Taille Laura
%\fontsize{30}{35}\selectfont


%\myfont
%\usefont{U}{cmss}{bx}{n}

%\Huge
%\addtolength{\parskip}{\baselineskip}
}


% \usepackage{hyperref}
% \hypersetup{colorlinks=true, linkcolor=blue, urlcolor=blue,
% pdftitle={Exo7 - Exercices de mathématiques}, pdfauthor={Exo7}}


%section
% \usepackage{sectsty}
% \allsectionsfont{\bf}
%\sectionfont{\color{Tomato3}\upshape\selectfont}
%\subsectionfont{\color{Tomato4}\upshape\selectfont}

%----- Ensembles : entiers, reels, complexes -----
\newcommand{\Nn}{\mathbb{N}} \newcommand{\N}{\mathbb{N}}
\newcommand{\Zz}{\mathbb{Z}} \newcommand{\Z}{\mathbb{Z}}
\newcommand{\Qq}{\mathbb{Q}} \newcommand{\Q}{\mathbb{Q}}
\newcommand{\Rr}{\mathbb{R}} \newcommand{\R}{\mathbb{R}}
\newcommand{\Cc}{\mathbb{C}} 
\newcommand{\Kk}{\mathbb{K}} \newcommand{\K}{\mathbb{K}}

%----- Modifications de symboles -----
\renewcommand{\epsilon}{\varepsilon}
\renewcommand{\Re}{\mathop{\text{Re}}\nolimits}
\renewcommand{\Im}{\mathop{\text{Im}}\nolimits}
%\newcommand{\llbracket}{\left[\kern-0.15em\left[}
%\newcommand{\rrbracket}{\right]\kern-0.15em\right]}

\renewcommand{\ge}{\geqslant}
\renewcommand{\geq}{\geqslant}
\renewcommand{\le}{\leqslant}
\renewcommand{\leq}{\leqslant}

%----- Fonctions usuelles -----
\newcommand{\ch}{\mathop{\mathrm{ch}}\nolimits}
\newcommand{\sh}{\mathop{\mathrm{sh}}\nolimits}
\renewcommand{\tanh}{\mathop{\mathrm{th}}\nolimits}
\newcommand{\cotan}{\mathop{\mathrm{cotan}}\nolimits}
\newcommand{\Arcsin}{\mathop{\mathrm{Arcsin}}\nolimits}
\newcommand{\Arccos}{\mathop{\mathrm{Arccos}}\nolimits}
\newcommand{\Arctan}{\mathop{\mathrm{Arctan}}\nolimits}
\newcommand{\Argsh}{\mathop{\mathrm{Argsh}}\nolimits}
\newcommand{\Argch}{\mathop{\mathrm{Argch}}\nolimits}
\newcommand{\Argth}{\mathop{\mathrm{Argth}}\nolimits}
\newcommand{\pgcd}{\mathop{\mathrm{pgcd}}\nolimits} 

\newcommand{\Card}{\mathop{\text{Card}}\nolimits}
\newcommand{\Ker}{\mathop{\text{Ker}}\nolimits}
\newcommand{\id}{\mathop{\text{id}}\nolimits}
\newcommand{\ii}{\mathrm{i}}
\newcommand{\dd}{\mathrm{d}}
\newcommand{\Vect}{\mathop{\text{Vect}}\nolimits}
\newcommand{\Mat}{\mathop{\mathrm{Mat}}\nolimits}
\newcommand{\rg}{\mathop{\text{rg}}\nolimits}
\newcommand{\tr}{\mathop{\text{tr}}\nolimits}
\newcommand{\ppcm}{\mathop{\text{ppcm}}\nolimits}

%----- Structure des exercices ------

\newtheoremstyle{styleexo}% name
{2ex}% Space above
{3ex}% Space below
{}% Body font
{}% Indent amount 1
{\bfseries} % Theorem head font
{}% Punctuation after theorem head
{\newline}% Space after theorem head 2
{}% Theorem head spec (can be left empty, meaning ‘normal’)

%\theoremstyle{styleexo}
\newtheorem{exo}{Exercice}
\newtheorem{ind}{Indications}
\newtheorem{cor}{Correction}


\newcommand{\exercice}[1]{} \newcommand{\finexercice}{}
%\newcommand{\exercice}[1]{{\tiny\texttt{#1}}\vspace{-2ex}} % pour afficher le numero absolu, l'auteur...
\newcommand{\enonce}{\begin{exo}} \newcommand{\finenonce}{\end{exo}}
\newcommand{\indication}{\begin{ind}} \newcommand{\finindication}{\end{ind}}
\newcommand{\correction}{\begin{cor}} \newcommand{\fincorrection}{\end{cor}}

\newcommand{\noindication}{\stepcounter{ind}}
\newcommand{\nocorrection}{\stepcounter{cor}}

\newcommand{\fiche}[1]{} \newcommand{\finfiche}{}
\newcommand{\titre}[1]{\centerline{\large \bf #1}}
\newcommand{\addcommand}[1]{}
\newcommand{\video}[1]{}

% Marge
\newcommand{\mymargin}[1]{\marginpar{{\small #1}}}



%----- Presentation ------
\setlength{\parindent}{0cm}

%\newcommand{\ExoSept}{\href{http://exo7.emath.fr}{\textbf{\textsf{Exo7}}}}

\definecolor{myred}{rgb}{0.93,0.26,0}
\definecolor{myorange}{rgb}{0.97,0.58,0}
\definecolor{myyellow}{rgb}{1,0.86,0}

\newcommand{\LogoExoSept}[1]{  % input : echelle
{\usefont{U}{cmss}{bx}{n}
\begin{tikzpicture}[scale=0.1*#1,transform shape]
  \fill[color=myorange] (0,0)--(4,0)--(4,-4)--(0,-4)--cycle;
  \fill[color=myred] (0,0)--(0,3)--(-3,3)--(-3,0)--cycle;
  \fill[color=myyellow] (4,0)--(7,4)--(3,7)--(0,3)--cycle;
  \node[scale=5] at (3.5,3.5) {Exo7};
\end{tikzpicture}}
}



\theoremstyle{definition}
%\newtheorem{proposition}{Proposition}
%\newtheorem{exemple}{Exemple}
%\newtheorem{theoreme}{Théorème}
\newtheorem{lemme}{Lemme}
\newtheorem{corollaire}{Corollaire}
%\newtheorem*{remarque*}{Remarque}
%\newtheorem*{miniexercice}{Mini-exercices}
%\newtheorem{definition}{Définition}




%definition d'un terme
\newcommand{\defi}[1]{{\color{myorange}\textbf{\emph{#1}}}}
\newcommand{\evidence}[1]{{\color{blue}\textbf{\emph{#1}}}}



 %----- Commandes divers ------

\newcommand{\codeinline}[1]{\texttt{#1}}

%%%%%%%%%%%%%%%%%%%%%%%%%%%%%%%%%%%%%%%%%%%%%%%%%%%%%%%%%%%%%
%%%%%%%%%%%%%%%%%%%%%%%%%%%%%%%%%%%%%%%%%%%%%%%%%%%%%%%%%%%%%



\begin{document}

\debuttexte

%%%%%%%%%%%%%%%%%%%%%%%%%%%%%%%%%%%%%%%%%%%%%%%%%%%%%%%%%%%
\diapo

\change

Dans cette leçon nous allons intégrer une famille complète de fonctions : les fractions rationnelles.

\change

On se concentre dans un premier temps sur trois types de fractions qui sont les plus fréquentes.

\change

On passe ensuite au cas général : l'intégration des éléments simples.

\change

Et on termine avec une application : l'intégration des fonctions qui ont la forme d'un polynôme ou d'une fraction rationnelle
en sinus et cosinus.



%%%%%%%%%%%%%%%%%%%%%%%%%%%%%%%%%%%%%%%%%%%%%%%%%%%%%%%%%%%
\diapo

Avant de passer aux fractions rationnelles remarquons que 
nous savons déjà intégrer beaucoup de fonctions simples. 

Par exemple pour une fonction polynomiale : si $f(x)=a_0+a_1x+\cdots+ a_n x^n$

\change

alors les primitives sont $a_0x+a_1\frac{x^2}{2}+\cdots+a_n\frac{x^{n+1}}{n+1}+c$.

\change

On pourrait croire que toutes les fonctions s'intègrent facilement, ou du moins que toute fonction <<simple>>
s'intègrent. Il n'en est rien !



Prenons par exemple la fonction $f(t)=\sqrt{a^2\cos^2 t+ b^2 \sin^2 t}$ 

\change

alors
l'intégrale $\int_0^{2\pi} f(t) \; dt$ ne peut *pas* s'exprimer 
comme somme, produit, inverse ou composition de fonctions que vous connaissez.

\change

En fait cette intégrale vaut la longueur d'une ellipse
d'équation  $(a\cos t, b\sin t)$ ; 

il n'y a donc pas de formule 
pour le périmètre d'une ellipse 
(sauf bien sûr si $a=b$ auquel cas l'ellipse est un cercle !).

\change

Dans cette exemple c'est la racines carrées qui pose problème.


Qu'en est-t-il pour les fractions rationnelles ?

De façon remarquable nous allons voir maintenant comment intégrer toutes les fractions rationnelles.


%%%%%%%%%%%%%%%%%%%%%%%%%%%%%%%%%%%%%%%%%%%%%%%%%%%%%%%%%%%
\diapo

Avant de passer au cas général on s'occupe de la fraction rationnelle 
 $\frac{\alpha x + \beta}{a x^2+b x+c}$

ou $\alpha, \beta, a, b, c$ sont des constantes réelles et  $a$ est non nul.

On distingues trois cas.


\change

Premier cas. Le dénominateur $a x^2+b x+c$ possède deux racines réelles distinctes $x_1,x_ 2\in \Rr$.
C'est-à-dire le discriminant $\Delta=b^2-4ac$ est strictement positif.


\change

On factorise le dénominateur et on décompose la fractions en éléments simples :

\change

il existe deux constantes $A$ et $B$ telles que $f(x)=\frac{A}{x - x_1}+\frac{B}{x -x_2}$.

Il est maintenant facile d'intégrer $f$ avec des logarithmes :

\change

Les primitives de $f$ sont de la forme $A \ln|x - x_1|+B\ln|x -x_2|+cst$.

Chaque primitive est définie sur un intervalle ouvert $]-\infty,x_1[$ ou  $]x_1,x_2[$
ou bien $]x_2,+\infty[$.

\change

Considérons maintenant le deuxième cas dans lequel $\Delta=0$ c'est-à-dire que $a x^2+b x+c$ possède une racine 
double réelle $x_0$.

\change

La fraction s'écrit alors $\frac{\alpha x + \beta}{a(x -x_0)^2}$ et 

\change

se décompose sous la forme 
$\frac{A}{(x - x_0)^2}+\frac{B}{x - x_0}$. 

\change

Les primitives de $f$ sont alors les
$$-\frac{A}{x - x_0} + B\ln|x - x_0|+cst$$ 

Elles sont définies sur l'intervalle
$]-\infty,x_0[$ ou bien sur l'intervalle $]x_0,+\infty[$.

%%%%%%%%%%%%%%%%%%%%%%%%%%%%%%%%%%%%%%%%%%%%%%%%%%%%%%%%%%%
\diapo


Passons au troisième et dernier cas qui est plus compliqué.

C'est le cas ou le discriminant est strictement négatif 
et donc le dénominateur $a x^2+b x+c$ ne possède pas de racine réelle.


\change

Voyons comment calculer primitives et intégrales dans cette situation
en considérant l'exemple de la fraction 
$f(x)=\frac{x+1}{2x^2+x+1}$.

On vérifie d'abord que le discriminant $\Delta$ est strictement négatif ;
il n'y a donc pas de racines réelles.

\change

Dans un premier temps on fait apparaître une fraction du type $\frac{u'}{u}$ 

Pourquoi ? Tout simplement parce que l'on sait intégrer ce type de fraction :
une primitive de $\frac{u'}{u}$ est $\ln|u|$.

\change

Le dénominateur est $2x^2+x+1$, je vais donc faire apparaître $4x+1$ 
(la dérivée de $2x^2+x+1$) au numérateur.


Bien sûr je suis obligé de compenser.

\change

Je sépare ma fraction en deux.

\change

J'ai tout fais pour que la première fraction soit du type $u'/u$.

\change

On peut donc intégrer la première moitié de la fraction :
$\frac{4x+1}{2x^2+x+1} \; dx$

\change

qui est de la forme $\frac{u'(x)}{u(x)}$

\change

Dont les  primitives sont les 
$\ln \big| 2x^2+x+1 \big|+c$

On a ainsi intégrer la première moitié.


%%%%%%%%%%%%%%%%%%%%%%%%%%%%%%%%%%%%%%%%%%%%%%%%%%%%%%%%%%%
\diapo

On continue avec notre fraction.
On a intégré la première partie de la décomposition.

Occupons nous maintenant de l'autre partie $\frac{1}{2x^2+x+1}$, 

\change

nous allons l'écrire sous la forme
$\frac{1}{u^2+1}$.

\change

Pourquoi ? Car nous connaissons un primitive de 
$\frac{1}{u^2+1}$ c'est $\arctan u$.

\change

On écrit le dénominateur sous la forme canonique est faisant
artificiellement apparaître le début d'un carré.

\change

$2x^2+x+1 = 2(x+\frac 14)^2+\frac78$.

\change

On écrit donc la fraction sous la forme $1/(u^2+1)$

\change 

où $u = \frac{4}{\sqrt7}(x+\frac 14)$

\change

C'est notre changement de variable !

\change

On a $du = \frac{4}{\sqrt7} dx$

\change

On calcule la primitive de notre seconde fraction :
$\frac{dx}{2x^2+x+1} $

\change

\change

s'écrit $du/(u^2+1)$

\change

Une primitive est $\arctan u$

\change 

On remplace $u$ par son expression en fonction de $x$.

Et c'est tout bon !


\change

Il ne reste plus qu'à conclure,
nous avons calculer les primitives de chacune des deux fractions.

\change

\change

Nous obtenons donc les primitives de $f$ !



%%%%%%%%%%%%%%%%%%%%%%%%%%%%%%%%%%%%%%%%%%%%%%%%%%%%%%%%%%%
\diapo

Nous savons donc comment intégrer un type de fraction rationnelle,

voyons comment intégrer toutes les fractions rationnelles !


Prenons  $\frac{P(x)}{Q(x)}$ une fraction rationnelle quelconque.

$P(x),Q(x)$ sont des polynômes à coefficients réels.

La première choses à faire est de décomposer la fraction $\frac{P(x)}{Q(x)}$ en éléments simples 

\change

$\frac{P(x)}{Q(x)}$ 
s'écrit comme somme d'un polynôme $E(x)$ (qui s'appelle la partie entière)

\change

et d'éléments simples :

Il y a deux sortes d'éléments simples :

\change

ceux de la forme $\frac{\gamma}{(x - x_0)^k}$

\change

et ceux de la forme  $\frac{\alpha x+\beta}{(a x^2+b x+c)^k}$
avec $b^2-4ac < 0$

Il n'y a bien sûr pas de problème pour intégrer le polynôme.


\change

Intégrons l'élément simple $\frac{\gamma}{(x - x_0)^k}$.

\change

Si $k=1$ alors c'est du type $1/u$ donc $\int \frac{\gamma \; dx}{x - x_0} = \gamma \ln|x - x_0|+c$.

\change

Si $k\ge 2$ alors c'est du type $1/u^k$ donc  $\int \frac{\gamma \; dx}{(x - x_0)^k} = 
\frac{\gamma}{-k+1}(x - x_0)^{-k+1}+c$. 


\change

Passons à l'intégration de l'élément simple $\frac{\alpha x+\beta}{(a x^2+b x+c)^k}$.

\change

On écrit cette fraction comme somme de deux fractions 

\change

Nous nous sommes débrouillés pour que la première fraction 

\change

soit de la forme $u'/u^k$

\change

qui s'intègre en $\frac{1}{-k+1}u(x)^{-k+1}$.

\change

\change

Passons à la deuxième fraction :

Si $k=1$, calculons de $\int \frac{1}{a x^2+b x+c}\; dx$. On écrit le dénominateur sous la forme canonique. 

\change

Puis par un changement de variable $u= px+q$ on se ramène à 
calculer une primitive du type 
$\int \frac{du}{u^2+1}=\arctan u + c$.


\change

Si $k\ge 2$, c'est plus délicat :

\change

 On effectue le même changement de variable 

\change

qu'au dessus pour
 se ramener au calcul de $I_k =\int \frac{du}{(u^2+1)^k}$.

\change

On calcule ces primitives par récurrence, en effet une
 intégration par parties permet de passer de $I_k$ à $I_{k-1}$.


%%%%%%%%%%%%%%%%%%%%%%%%%%%%%%%%%%%%%%%%%%%%%%%%%%%%%%%%%%%
\diapo

Voyons comment calculer la primitive $I_2= \int \frac{du}{(u^2+1)^2}$

\change

\`A partir de  $I_1=\int \frac{du}{u^2+1}$.

\change

On effectue une IPP pour laquelle on pose 
$f=\frac{1}{u^2+1}$ et $g'=1$. 

\change

On a donc $f'=-\frac{2u}{(u^2+1)^2}$ et $g=u$.

\change 

La formule d'intégration par parties $\int f g'=[f g]-\int f' g$ donne

\change

$I_1 =  \left[ \frac{u}{u^2+1} \right] + \int \frac{2u^2 \; du}{(u^2+1)^2}$

\change

On sépare la primitive qui apparaît en deux afin de faire apparaître $I_2$.

\change

\change

On reconnaît $2I_1-2I_2$.

\change

Cela nous donne une relation entre $I_1$ et $I_2$ qui fournit l'égalité :
$I_2 = \frac12 I_1 + \frac12\frac{u}{u^2+1} + c$.

\change


Mais comme $I_1=\arctan u$ alors

\change

$I_2  =  \frac12 \arctan u + \frac12\frac{u}{u^2+1} + c.$


%%%%%%%%%%%%%%%%%%%%%%%%%%%%%%%%%%%%%%%%%%%%%%%%%%%%%%%%%%%
\diapo

Après les fractions rationnelles nous allons
calculer les primitives de la forme $\int P(\cos x,\sin x)\;dx$ 

où $P$ est un polynôme en cosinus et sinus par exemple $\cos^2x*\sin^3x$.

\change


et aussi les primitives des fonctions
$ \frac{P(\cos x,\sin x)}{Q(\cos x, \sin x)}\;dx$
où $P$ et $Q$ sont des polynômes,

Par exemple $\cos x/(\sin x + \cos x)$

 en se ramenant à intégrer une fraction rationnelle.



Il existe deux méthodes : 
\begin{itemize}
  \item les règles de Bioche sont assez efficaces mais ne fonctionnent pas toujours ;


  \item le changement de variable $t = \tan \frac x2$ fonctionne tout le temps mais conduit à davantage de calculs.
\end{itemize}

\change

Commençons par les règles de Bioche.

\change

On note $\omega(x) = f(x)\;dx$.

\change

On a alors $\omega(-x)= f(-x)\;d(-x)=-f(-x)\;dx$ et

\change

$\omega (\pi-x)= f(\pi-x)\;d(\pi-x)=-f(\pi-x)\;dx$.

\change

Voici trois règles simples pour savoir quel changement de variable 
va permettre de calculer la primitive :

\begin{itemize}
  \item Si $\omega(-x)=\omega(x)$ 

\change

alors on effectue le changement de variable $u=\cos x$.

\change

  \item Si $\omega(\pi-x)=\omega(x)$ alors on effectue le changement de variable $u=\sin x$.

\change

  \item Si $\omega(\pi + x)=\omega(x)$ alors on effectue le changement de variable $u=\tan x$.
\end{itemize}

\change

Voyons comment cela marche sur un exemple.

avec $\int \frac{\cos x \; dx}{2-\sin^2 x}$

\change

On note $\omega(x)= \frac{\cos x \; dx}{2-\cos^2 x}$ (on n'oublie pas le $dx$)

\change

On teste chacune des possibilité et on remarque que 


$\omega(\pi-x)=\frac{\cos(\pi-x) \; d(\pi-x)}{2-\cos^2 (\pi-x)} = \frac{(-\cos x) \; (-dx)}{2-\cos^2 x} 
= \omega(x)$

\change

Le changement de variable qui convient est donc $u = \sin x$

\change

pour lequel $du= \cos x \; dx$. 

\change

Notre fraction rationnelle 
$\int \frac{\cos x \; dx}{2-\cos^2 x}$

\change

se réécrit $= \int \frac{\cos x \; dx}{2-(1-\sin^2 x)}$

\change

on a donc fait apparaître $du$ au numérateur et le dénominateur est 
$1+u^2$.


\change

C'est une fraction rationnelle que l'on sait intégrer 

en $\arctan u$

\change

Donc en remplaçant $u$ par $\sin x$ on trouve que notre primitive est 
$\arctan (\sin x) + cst$

%%%%%%%%%%%%%%%%%%%%%%%%%%%%%%%%%%%%%%%%%%%%%%%%%%%%%%%%%%%
\diapo


Les règles de Bioche sont très pratiques mais ne fonctionnent pas toujours.

Le changement de variable $t=\tan \frac x2$
transforme toujours une fractionnelle en cosinus(x) et sinus(x) en une fraction rationnelle en t,
mais le calculs peuvent être plus compliqués.

\change


Les formules de la <<tangente de l'arc moitié>> permettent d'exprimer sinus, cosinus et tangente
en fonction de $\tan \frac x2$. 

Si on pose $ t=\tan \frac{x}{2}$  

alors on a 
$\cos x = \frac {1-t^2}{1+t^2}$

\change

$\sin x = \frac{2t}{1+t^2}$

\change

$\tan x = \frac{2t}{1-t^2}$

\change

$dx=\dfrac{2\;dt}{1+t^2}$

\change

\change

Calculons par exemple $\int_{-\pi/2}^0 \frac{dx}{1-\sin x}$.

\change

On pose le changement de variable $t=\tan \frac{x}{2}$ 

\change

définit une bijection de $[-\frac\pi2,0]$ vers $[-1,0]$ 

(pour $x=-\frac\pi2$, $t=-1$ et pour $x=0$, $t=0$). 

\change

Les formules nous donnent $\sin x = \frac{2t}{1+t^2}$
et $dx=\frac{2\;dt}{1+t^2}$.

[montrer du doigt]

\change

\change

\change

Ce qui se simplifie en $2\int_{-1}^0 \frac{dt}{(1-t)^2} $

\change

que l'on intègre ainsi :

\change

\change

Et donc l'intégrale vaut $1$.


%%%%%%%%%%%%%%%%%%%%%%%%%%%%%%%%%%%%%%%%%%%%%%%%%%%%%%%%%%%
\diapo

N'oubliez pas que vous devez passer beaucoup 
de temps à travailler par vous même devant une feuille blanche
à résoudre des exercices.


\end{document}