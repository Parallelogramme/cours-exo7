
%%%%%%%%%%%%%%%%%% PREAMBULE %%%%%%%%%%%%%%%%%%

\documentclass[aspectratio=169,utf8]{beamer}
%\documentclass[aspectratio=169,handout]{beamer}

\usetheme{Boadilla}
%\usecolortheme{seahorse}
\usecolortheme[RGB={245,66,24}]{structure}
\useoutertheme{infolines}

% packages
\usepackage{amsfonts,amsmath,amssymb,amsthm}
\usepackage[utf8]{inputenc}
\usepackage[T1]{fontenc}
\usepackage{lmodern}

\usepackage[francais]{babel}
\usepackage{fancybox}
\usepackage{graphicx}

\usepackage{float}
\usepackage{xfrac}

%\usepackage[usenames, x11names]{xcolor}
\usepackage{tikz}
\usepackage{pgfplots}
\usepackage{datetime}



%-----  Package unités -----
\usepackage{siunitx}
\sisetup{locale = FR,detect-all,per-mode = symbol}

%\usepackage{mathptmx}
%\usepackage{fouriernc}
%\usepackage{newcent}
%\usepackage[mathcal,mathbf]{euler}

%\usepackage{palatino}
%\usepackage{newcent}
% \usepackage[mathcal,mathbf]{euler}



% \usepackage{hyperref}
% \hypersetup{colorlinks=true, linkcolor=blue, urlcolor=blue,
% pdftitle={Exo7 - Exercices de mathématiques}, pdfauthor={Exo7}}


%section
% \usepackage{sectsty}
% \allsectionsfont{\bf}
%\sectionfont{\color{Tomato3}\upshape\selectfont}
%\subsectionfont{\color{Tomato4}\upshape\selectfont}

%----- Ensembles : entiers, reels, complexes -----
\newcommand{\Nn}{\mathbb{N}} \newcommand{\N}{\mathbb{N}}
\newcommand{\Zz}{\mathbb{Z}} \newcommand{\Z}{\mathbb{Z}}
\newcommand{\Qq}{\mathbb{Q}} \newcommand{\Q}{\mathbb{Q}}
\newcommand{\Rr}{\mathbb{R}} \newcommand{\R}{\mathbb{R}}
\newcommand{\Cc}{\mathbb{C}} 
\newcommand{\Kk}{\mathbb{K}} \newcommand{\K}{\mathbb{K}}

%----- Modifications de symboles -----
\renewcommand{\epsilon}{\varepsilon}
\renewcommand{\Re}{\mathop{\text{Re}}\nolimits}
\renewcommand{\Im}{\mathop{\text{Im}}\nolimits}
%\newcommand{\llbracket}{\left[\kern-0.15em\left[}
%\newcommand{\rrbracket}{\right]\kern-0.15em\right]}

\renewcommand{\ge}{\geqslant}
\renewcommand{\geq}{\geqslant}
\renewcommand{\le}{\leqslant}
\renewcommand{\leq}{\leqslant}
\renewcommand{\epsilon}{\varepsilon}

%----- Fonctions usuelles -----
\newcommand{\ch}{\mathop{\text{ch}}\nolimits}
\newcommand{\sh}{\mathop{\text{sh}}\nolimits}
\renewcommand{\tanh}{\mathop{\text{th}}\nolimits}
\newcommand{\cotan}{\mathop{\text{cotan}}\nolimits}
\newcommand{\Arcsin}{\mathop{\text{arcsin}}\nolimits}
\newcommand{\Arccos}{\mathop{\text{arccos}}\nolimits}
\newcommand{\Arctan}{\mathop{\text{arctan}}\nolimits}
\newcommand{\Argsh}{\mathop{\text{argsh}}\nolimits}
\newcommand{\Argch}{\mathop{\text{argch}}\nolimits}
\newcommand{\Argth}{\mathop{\text{argth}}\nolimits}
\newcommand{\pgcd}{\mathop{\text{pgcd}}\nolimits} 


%----- Commandes divers ------
\newcommand{\ii}{\mathrm{i}}
\newcommand{\dd}{\text{d}}
\newcommand{\id}{\mathop{\text{id}}\nolimits}
\newcommand{\Ker}{\mathop{\text{Ker}}\nolimits}
\newcommand{\Card}{\mathop{\text{Card}}\nolimits}
\newcommand{\Vect}{\mathop{\text{Vect}}\nolimits}
\newcommand{\Mat}{\mathop{\text{Mat}}\nolimits}
\newcommand{\rg}{\mathop{\text{rg}}\nolimits}
\newcommand{\tr}{\mathop{\text{tr}}\nolimits}


%----- Structure des exercices ------

\newtheoremstyle{styleexo}% name
{2ex}% Space above
{3ex}% Space below
{}% Body font
{}% Indent amount 1
{\bfseries} % Theorem head font
{}% Punctuation after theorem head
{\newline}% Space after theorem head 2
{}% Theorem head spec (can be left empty, meaning ‘normal’)

%\theoremstyle{styleexo}
\newtheorem{exo}{Exercice}
\newtheorem{ind}{Indications}
\newtheorem{cor}{Correction}


\newcommand{\exercice}[1]{} \newcommand{\finexercice}{}
%\newcommand{\exercice}[1]{{\tiny\texttt{#1}}\vspace{-2ex}} % pour afficher le numero absolu, l'auteur...
\newcommand{\enonce}{\begin{exo}} \newcommand{\finenonce}{\end{exo}}
\newcommand{\indication}{\begin{ind}} \newcommand{\finindication}{\end{ind}}
\newcommand{\correction}{\begin{cor}} \newcommand{\fincorrection}{\end{cor}}

\newcommand{\noindication}{\stepcounter{ind}}
\newcommand{\nocorrection}{\stepcounter{cor}}

\newcommand{\fiche}[1]{} \newcommand{\finfiche}{}
\newcommand{\titre}[1]{\centerline{\large \bf #1}}
\newcommand{\addcommand}[1]{}
\newcommand{\video}[1]{}

% Marge
\newcommand{\mymargin}[1]{\marginpar{{\small #1}}}

\def\noqed{\renewcommand{\qedsymbol}{}}


%----- Presentation ------
\setlength{\parindent}{0cm}

%\newcommand{\ExoSept}{\href{http://exo7.emath.fr}{\textbf{\textsf{Exo7}}}}

\definecolor{myred}{rgb}{0.93,0.26,0}
\definecolor{myorange}{rgb}{0.97,0.58,0}
\definecolor{myyellow}{rgb}{1,0.86,0}

\newcommand{\LogoExoSept}[1]{  % input : echelle
{\usefont{U}{cmss}{bx}{n}
\begin{tikzpicture}[scale=0.1*#1,transform shape]
  \fill[color=myorange] (0,0)--(4,0)--(4,-4)--(0,-4)--cycle;
  \fill[color=myred] (0,0)--(0,3)--(-3,3)--(-3,0)--cycle;
  \fill[color=myyellow] (4,0)--(7,4)--(3,7)--(0,3)--cycle;
  \node[scale=5] at (3.5,3.5) {Exo7};
\end{tikzpicture}}
}


\newcommand{\debutmontitre}{
  \author{} \date{} 
  \thispagestyle{empty}
  \hspace*{-10ex}
  \begin{minipage}{\textwidth}
    \titlepage  
  \vspace*{-2.5cm}
  \begin{center}
    \LogoExoSept{2.5}
  \end{center}
  \end{minipage}

  \vspace*{-0cm}
  
  % Astuce pour que le background ne soit pas discrétisé lors de la conversion pdf -> png
\begin{tikzpicture}
        \fill[opacity=0,green!60!black] (0,0)--++(0,0)--++(0,0)--++(0,0)--cycle; 
\end{tikzpicture}

% toc S'affiche trop tot :
% \tableofcontents[hideallsubsections, pausesections]
}

\newcommand{\finmontitre}{
  \end{frame}
  \setcounter{framenumber}{0}
} % ne marche pas pour une raison obscure

%----- Commandes supplementaires ------

% \usepackage[landscape]{geometry}
% \geometry{top=1cm, bottom=3cm, left=2cm, right=10cm, marginparsep=1cm
% }
% \usepackage[a4paper]{geometry}
% \geometry{top=2cm, bottom=2cm, left=2cm, right=2cm, marginparsep=1cm
% }

%\usepackage{standalone}


% New command Arnaud -- november 2011
\setbeamersize{text margin left=24ex}
% si vous modifier cette valeur il faut aussi
% modifier le decalage du titre pour compenser
% (ex : ici =+10ex, titre =-5ex

\theoremstyle{definition}
%\newtheorem{proposition}{Proposition}
%\newtheorem{exemple}{Exemple}
%\newtheorem{theoreme}{Théorème}
%\newtheorem{lemme}{Lemme}
%\newtheorem{corollaire}{Corollaire}
%\newtheorem*{remarque*}{Remarque}
%\newtheorem*{miniexercice}{Mini-exercices}
%\newtheorem{definition}{Définition}

% Commande tikz
\usetikzlibrary{calc}
\usetikzlibrary{patterns,arrows}
\usetikzlibrary{matrix}
\usetikzlibrary{fadings} 

%definition d'un terme
\newcommand{\defi}[1]{{\color{myorange}\textbf{\emph{#1}}}}
\newcommand{\evidence}[1]{{\color{blue}\textbf{\emph{#1}}}}
\newcommand{\assertion}[1]{\emph{\og#1\fg}}  % pour chapitre logique
%\renewcommand{\contentsname}{Sommaire}
\renewcommand{\contentsname}{}
\setcounter{tocdepth}{2}



%------ Figures ------

\def\myscale{1} % par défaut 
\newcommand{\myfigure}[2]{  % entrée : echelle, fichier figure
\def\myscale{#1}
\begin{center}
\footnotesize
{#2}
\end{center}}


%------ Encadrement ------

\usepackage{fancybox}


\newcommand{\mybox}[1]{
\setlength{\fboxsep}{7pt}
\begin{center}
\shadowbox{#1}
\end{center}}

\newcommand{\myboxinline}[1]{
\setlength{\fboxsep}{5pt}
\raisebox{-10pt}{
\shadowbox{#1}
}
}

%--------------- Commande beamer---------------
\newcommand{\beameronly}[1]{#1} % permet de mettre des pause dans beamer pas dans poly


\setbeamertemplate{navigation symbols}{}
\setbeamertemplate{footline}  % tiré du fichier beamerouterinfolines.sty
{
  \leavevmode%
  \hbox{%
  \begin{beamercolorbox}[wd=.333333\paperwidth,ht=2.25ex,dp=1ex,center]{author in head/foot}%
    % \usebeamerfont{author in head/foot}\insertshortauthor%~~(\insertshortinstitute)
    \usebeamerfont{section in head/foot}{\bf\insertshorttitle}
  \end{beamercolorbox}%
  \begin{beamercolorbox}[wd=.333333\paperwidth,ht=2.25ex,dp=1ex,center]{title in head/foot}%
    \usebeamerfont{section in head/foot}{\bf\insertsectionhead}
  \end{beamercolorbox}%
  \begin{beamercolorbox}[wd=.333333\paperwidth,ht=2.25ex,dp=1ex,right]{date in head/foot}%
    % \usebeamerfont{date in head/foot}\insertshortdate{}\hspace*{2em}
    \insertframenumber{} / \inserttotalframenumber\hspace*{2ex} 
  \end{beamercolorbox}}%
  \vskip0pt%
}


\definecolor{mygrey}{rgb}{0.5,0.5,0.5}
\setlength{\parindent}{0cm}
%\DeclareTextFontCommand{\helvetica}{\fontfamily{phv}\selectfont}

% background beamer
\definecolor{couleurhaut}{rgb}{0.85,0.9,1}  % creme
\definecolor{couleurmilieu}{rgb}{1,1,1}  % vert pale
\definecolor{couleurbas}{rgb}{0.85,0.9,1}  % blanc
\setbeamertemplate{background canvas}[vertical shading]%
[top=couleurhaut,middle=couleurmilieu,midpoint=0.4,bottom=couleurbas] 
%[top=fondtitre!05,bottom=fondtitre!60]



\makeatletter
\setbeamertemplate{theorem begin}
{%
  \begin{\inserttheoremblockenv}
  {%
    \inserttheoremheadfont
    \inserttheoremname
    \inserttheoremnumber
    \ifx\inserttheoremaddition\@empty\else\ (\inserttheoremaddition)\fi%
    \inserttheorempunctuation
  }%
}
\setbeamertemplate{theorem end}{\end{\inserttheoremblockenv}}

\newenvironment{theoreme}[1][]{%
   \setbeamercolor{block title}{fg=structure,bg=structure!40}
   \setbeamercolor{block body}{fg=black,bg=structure!10}
   \begin{block}{{\bf Th\'eor\`eme }#1}
}{%
   \end{block}%
}


\newenvironment{proposition}[1][]{%
   \setbeamercolor{block title}{fg=structure,bg=structure!40}
   \setbeamercolor{block body}{fg=black,bg=structure!10}
   \begin{block}{{\bf Proposition }#1}
}{%
   \end{block}%
}

\newenvironment{corollaire}[1][]{%
   \setbeamercolor{block title}{fg=structure,bg=structure!40}
   \setbeamercolor{block body}{fg=black,bg=structure!10}
   \begin{block}{{\bf Corollaire }#1}
}{%
   \end{block}%
}

\newenvironment{mydefinition}[1][]{%
   \setbeamercolor{block title}{fg=structure,bg=structure!40}
   \setbeamercolor{block body}{fg=black,bg=structure!10}
   \begin{block}{{\bf Définition} #1}
}{%
   \end{block}%
}

\newenvironment{lemme}[0]{%
   \setbeamercolor{block title}{fg=structure,bg=structure!40}
   \setbeamercolor{block body}{fg=black,bg=structure!10}
   \begin{block}{\bf Lemme}
}{%
   \end{block}%
}

\newenvironment{remarque}[1][]{%
   \setbeamercolor{block title}{fg=black,bg=structure!20}
   \setbeamercolor{block body}{fg=black,bg=structure!5}
   \begin{block}{Remarque #1}
}{%
   \end{block}%
}


\newenvironment{exemple}[1][]{%
   \setbeamercolor{block title}{fg=black,bg=structure!20}
   \setbeamercolor{block body}{fg=black,bg=structure!5}
   \begin{block}{{\bf Exemple }#1}
}{%
   \end{block}%
}


\newenvironment{miniexercice}[0]{%
   \setbeamercolor{block title}{fg=structure,bg=structure!20}
   \setbeamercolor{block body}{fg=black,bg=structure!5}
   \begin{block}{Mini-exercices}
}{%
   \end{block}%
}


\newenvironment{tp}[0]{%
   \setbeamercolor{block title}{fg=structure,bg=structure!40}
   \setbeamercolor{block body}{fg=black,bg=structure!10}
   \begin{block}{\bf Travaux pratiques}
}{%
   \end{block}%
}
\newenvironment{exercicecours}[1][]{%
   \setbeamercolor{block title}{fg=structure,bg=structure!40}
   \setbeamercolor{block body}{fg=black,bg=structure!10}
   \begin{block}{{\bf Exercice }#1}
}{%
   \end{block}%
}
\newenvironment{algo}[1][]{%
   \setbeamercolor{block title}{fg=structure,bg=structure!40}
   \setbeamercolor{block body}{fg=black,bg=structure!10}
   \begin{block}{{\bf Algorithme}\hfill{\color{gray}\texttt{#1}}}
}{%
   \end{block}%
}


\setbeamertemplate{proof begin}{
   \setbeamercolor{block title}{fg=black,bg=structure!20}
   \setbeamercolor{block body}{fg=black,bg=structure!5}
   \begin{block}{{\footnotesize Démonstration}}
   \footnotesize
   \smallskip}
\setbeamertemplate{proof end}{%
   \end{block}}
\setbeamertemplate{qed symbol}{\openbox}


\makeatother
\usecolortheme[RGB={205,100,0}]{structure}

%%%%%%%%%%%%%%%%%%%%%%%%%%%%%%%%%%%%%%%%%%%%%%%%%%%%%%%%%%%%%
%%%%%%%%%%%%%%%%%%%%%%%%%%%%%%%%%%%%%%%%%%%%%%%%%%%%%%%%%%%%%


\begin{document}


\title{{\bf L'espace vectoriel $\Rr^n$}}
\subtitle{Exemples d'applications linéaires}

\begin{frame}
  
  \debutmontitre

  \pause

{\footnotesize
\hfill
\setbeamercovered{transparent=50}
\begin{minipage}{0.6\textwidth}
  \begin{itemize}
    \item<3-> Applications linéaires
    \item<4-> Exemples géométriques 
  \end{itemize}
\end{minipage}
}

\end{frame}

\setcounter{framenumber}{0}


%%%%%%%%%%%%%%%%%%%%%%%%%%%%%%%%%%%%%%%%%%%%%%%%%%%%%%%%%%%%%%%%
\section{Applications linéaires}

\begin{frame}

\begin{itemize}[<+->]\setlength{\itemsep}{8pt}
  \item Soient $n$ fonctions de $p$ variables réelles à valeurs réelles
  
  \item $f_1 : \Rr^p \longrightarrow \Rr \qquad f_2 : \Rr^p \longrightarrow \Rr \qquad \cdots \qquad f_n : \Rr^p \longrightarrow \Rr$
  
  \item $f_i : \Rr^p \longrightarrow \Rr, \qquad (x_1,x_2,\ldots,x_p) \mapsto f_i(x_1,\ldots,x_p)$
  
  \item On construit une application $f : \Rr^p \longrightarrow \Rr^n$
  
  \item Définie par $f(x_1,\dots , x_p)  = \left(f_1 (x_1, \dots , x_p), \dots , f_n (x_1, \dots , x_p)\right)$
  
\end{itemize}




\end{frame}


\begin{frame}
\begin{mydefinition}
Une application $f: \Rr^p  \longrightarrow \Rr^n$ définie par $f(x_1, \dots , x_p) = (y_1, \dots , y_n)$
est dite une \defi{application linéaire} si 
\vspace*{-2ex}
 $$  \left\{
\begin{array}{ccccccccc}
y_1 & = &a_{11}x_1 &+ &a_{12} x_2 & + & \cdots & + & a_{1p}x_p\\
y_2 & = & a_{21} x_1 & + & a_{22} x_2 & + & \cdots & + & a_{2p} x_p\\
\vdots &&\vdots &&\vdots & & & &\vdots\\
y_n & = & a_{n1}x_1 & + & a_{n2}x_2 &+&\cdots & +& a_{np} x_p\, 
\end{array}\right.
$$
\end{mydefinition}

\pause

\begin{itemize}[<+->]
  \item $f\left(\begin{smallmatrix} x_1\\x_2 \\ \vdots \\x_p \end{smallmatrix}\right)= 
\left(\begin{smallmatrix} y_1\\ y_2\\ \vdots\\ y_n  \end{smallmatrix}\right) = 
\left(\begin{smallmatrix}
a_{11} & a_{12} & \cdots & a_{1p}\\
a_{21} & a_{22} & \cdots & a_{2p}\\
\vdots & \vdots & & \vdots\\
a_{n1}& a_{n2} & \cdots & a_{np}  
\end{smallmatrix}\right)
\left(\begin{smallmatrix} x_1\\x_2 \\ \vdots \\x_p \end{smallmatrix}\right)
$
  
  \item $X= \left(\begin{smallmatrix} x_1\\ \vdots \\x_p \end{smallmatrix}\right)$ et 
$A = (a_{ij}) \in M_{n,p}(\Rr)$ 
\myboxinline{$f(X)=AX$}
  
  \item Une application linéaire $\Rr^p \to \Rr^n$ s'écrit $X \mapsto A X$
  
  \item $A\in M_{n,p}(\Rr)$ est la \defi{matrice de l'application linéaire} $f$
  
\end{itemize}
\end{frame}


\begin{frame}
\begin{exemple}
La fonction $f: \Rr^4 \longrightarrow \Rr^3$ définie par
$$\left\{\begin{array}{rcccccccc}
y_1 & = & -2x_1 &+& 5x_2 &+& 2x_3 &-& 7x_4\\
y_2 & = & 4x_1 &+& 2x_2 &-& 3x_3 &+& 3x_4\\
y_3 & = & 7x_1 &-& 3x_2 &+& 9x_3 && 
\end{array}\right.$$
s'exprime sous forme matricielle comme suit : 
$$\begin{pmatrix} y_1\\ y_2\\ y_3 \end{pmatrix} =
\begin{pmatrix}
-2 & 5 & 2 & -7\\
4 & 2 & -3 & 3\\
7 & -3 & 9 & 0  
\end{pmatrix}\quad
\begin{pmatrix}x_1\\ x_2\\ x_3\\ x_4 \end{pmatrix}$$
\end{exemple}
\end{frame}


\begin{frame}
\begin{exemple}
\begin{itemize}\setlength{\itemsep}{8pt}
  \item Pour l'application linéaire identité $\Rr^n \to \Rr^n$, $(x_1,\ldots,x_n) \mapsto (x_1,\ldots,x_n)$, sa matrice associée
  est l'identité $I_n$ (car $I_n X= X$)

  \pause
  
  \item Pour l'application linéaire nulle $\Rr^p \to \Rr^n$, $(x_1,\ldots,x_p) \mapsto (0,\ldots,0)$, sa matrice associée
  est la matrice nulle $0_{n,p}$ (car $0_{n,p} X= 0$)
\end{itemize}
\end{exemple}
\end{frame}


%%%%%%%%%%%%%%%%%%%%%%%%%%%%%%%%%%%%%%%%%%%%%%%%%%%%%%%%%%%%%%%%
\section{Exemples géométriques}

\begin{frame}
\evidence{Réflexion par rapport à l'axe $(Oy)$} 


La fonction
$$f : \Rr^2 \longrightarrow \Rr^2 \qquad \begin{pmatrix} x \\ y \end{pmatrix} \mapsto \begin{pmatrix} -x \\ y \end{pmatrix}$$
est la réflexion par rapport à l'axe des ordonnées $(Oy)$

\medskip

Sa matrice est \quad
$\begin{pmatrix}-1 & 0 \\ 0 & 1 \end{pmatrix}
\quad \text{ car } \quad 
\begin{pmatrix} -1 & 0 \\ 0 & 1 \end{pmatrix}
\begin{pmatrix} x \\ y \end{pmatrix}
= \begin{pmatrix} -x \\ y \end{pmatrix}$


\myfigure{1}{
\tikzinput{fig_erene05} 
}
\end{frame}


\begin{frame}
\evidence{Réflexion par rapport à l'axe $(Ox)$}

La réflexion par rapport à l'axe des abscisses $(Ox)$ est donnée par la matrice
$\begin{pmatrix} 1 & 0\\ 0 & -1 \end{pmatrix}$

\myfigure{0.9}{
\tikzinput{fig_erene07} 
}
\end{frame}


\begin{frame}
\evidence{Réflexion par rapport à la droite $(y = x)$}


La réflexion par rapport à la droite $(y=x)$ est donnée par
$$f: \Rr^2 \longrightarrow \Rr^2, \qquad 
\begin{pmatrix}x\\y\end{pmatrix} \mapsto \begin{pmatrix}y\\x\end{pmatrix}$$
Sa matrice est
$\begin{pmatrix}0 & 1\\1 & 0\end{pmatrix}$

\myfigure{0.9}{
\tikzinput{fig_erene06} 
}
\end{frame}


\begin{frame}
\evidence{Homothéties} 

L'homothétie de rapport $\lambda$ centrée à l'origine est : 
$$f: \Rr^2 \longrightarrow \Rr^2, \qquad 
\begin{pmatrix}x\\y\end{pmatrix} \mapsto \begin{pmatrix} \lambda x \\ \lambda y \end{pmatrix}$$

Ainsi 
$f\left(\begin{smallmatrix} x \\ y \end{smallmatrix}\right) = 
\left(\begin{smallmatrix} \lambda & 0 \\ 0 & \lambda \end{smallmatrix}\right)
\left(\begin{smallmatrix} x \\ y \end{smallmatrix}\right)$
et la matrice de l'homothétie est
$\begin{pmatrix} \lambda & 0 \\ 0 & \lambda \end{pmatrix}$
 
\myfigure{1}{
\tikzinput{fig_erene08} 
}

\end{frame}


\begin{frame}
\evidence{Rotations}


Soit $f: \Rr^2 \longrightarrow \Rr^2$ la rotation d'angle $\theta$, centrée à l'origine.

Si $\left(\begin{smallmatrix}x' \\ y' \end{smallmatrix}\right)$ dénote l'image 
de $\left(\begin{smallmatrix}x \\ y \end{smallmatrix}\right)$ par la rotation d'angle
$\theta$, on obtient :
$$\left\{\begin{array}{rcl}
x' & = & x \cos  \theta - y \sin\theta\\
y' & = & x\sin \, \theta + y \cos\theta
\end{array}\right.
\quad \text{ donc }\quad 
\begin{pmatrix}x' \\ y' \end{pmatrix}
= \begin{pmatrix}
\cos\theta & -\sin\theta\\
\sin\theta & \cos\theta
\end{pmatrix}
\begin{pmatrix}x \\ y \end{pmatrix}
$$ 


\myfigure{0.9}{
\tikzinput{fig_erene09pres} 
}



\end{frame}


\begin{frame}
\evidence{Projections orthogonales}

\begin{enumerate}
  \item L'application linéaire
$f: \Rr^2  \longrightarrow \Rr^2, \qquad \left(\begin{smallmatrix}x\\y\end{smallmatrix}\right) \mapsto 
\left(\begin{smallmatrix}x\\0\end{smallmatrix}\right)$
est la projection orthogonale sur l'axe $(Ox)$.
Sa matrice est
$\left(\begin{smallmatrix} 1 & 0 \\ 0 & 0 \end{smallmatrix}\right)$


 \uncover<2->{

  \item L'application linéaire 
  $f: \Rr^3  \longrightarrow \Rr^3, \qquad
\left(\begin{smallmatrix}x\\y\\z\end{smallmatrix}\right)  
\mapsto \left(\begin{smallmatrix}x\\y\\0\end{smallmatrix}\right)$
est la projection orthogonale sur le plan $(Oxy)$ et sa matrice est
$\left(\begin{smallmatrix}
1 & 0 & 0\\
0 & 1 & 0\\
0 & 0 & 0
\end{smallmatrix}\right)$
}
\end{enumerate}


\myfigure{0.7}{

\begin{minipage}{0.49\textwidth}
\tikzinput{fig_erene10}  
\end{minipage}
\begin{minipage}{0.49\textwidth}
\uncover<2->{\tikzinput{fig_erene11}}
\end{minipage}
}


\end{frame}


\begin{frame}
\evidence{Réflexions dans l'espace}

\bigskip

L'application 
$$
f: \Rr^3  \longrightarrow \Rr^3, \qquad \begin{pmatrix}x\\y\\z\end{pmatrix} \mapsto \begin{pmatrix}x\\y\\-z\end{pmatrix}$$
est la réflexion par rapport au plan $(Oxy)$. C'est une application linéaire et sa matrice est
\[ \begin{pmatrix}
1 & 0 & 0\\
0 & 1 & 0\\
0 & 0 & -1 
\end{pmatrix}
 \]
\end{frame}


%%%%%%%%%%%%%%%%%%%%%%%%%%%%%%%%%%%%%%%%%%%%%%%%%%%%%%%%%%%%%%%%
\section{Mini-exercices}

\begin{frame}
\begin{miniexercice}
\begin{enumerate}
  \item Soit $A=\left(\begin{smallmatrix} 1 & 2 \\ 1 & 3 \end{smallmatrix} \right)$
  et soit $f$ l'application linéaire associée. Calculer et dessiner l'image par $f$ de 
  $\left(\begin{smallmatrix} 1 \\ 0 \end{smallmatrix} \right)$, puis
  $\left(\begin{smallmatrix} 0 \\ 1 \end{smallmatrix} \right)$
et plus généralement de $\left(\begin{smallmatrix} x \\ y \end{smallmatrix} \right)$.
Dessiner l'image par $f$ du carré de sommets 
$\left(\begin{smallmatrix} 0 \\ 0 \end{smallmatrix} \right)$
$\left(\begin{smallmatrix} 1 \\ 0 \end{smallmatrix} \right)$
$\left(\begin{smallmatrix} 1 \\ 1 \end{smallmatrix} \right)$
$\left(\begin{smallmatrix} 0 \\ 1 \end{smallmatrix} \right)$.
Dessiner l'image par $f$ du cercle inscrit dans ce carré.
  
  \item Soit $A=\left(\begin{smallmatrix} 1 & 2 & -1 \\ 0 & 1 & 0 \\ 2 & 1 & 1 \end{smallmatrix} \right)$
  et soit $f$ l'application linéaire associée. Calculer l'image par $f$ de 
  $\left(\begin{smallmatrix} 1 \\ 0 \\ 0 \end{smallmatrix} \right)$,
  $\left(\begin{smallmatrix} 0 \\ 1 \\ 0 \end{smallmatrix} \right)$,
  $\left(\begin{smallmatrix} 0 \\ 0 \\ 1 \end{smallmatrix} \right)$ 
et plus généralement de $\left(\begin{smallmatrix} x \\ y \\z \end{smallmatrix} \right)$.

  
  \item \'Ecrire la matrice de la rotation du plan d'angle $\frac\pi4$ centrée à l'origine.
 Idem dans l'espace avec la rotation d'angle $\frac\pi4$ d'axe $(Ox)$.
  
  \item \'Ecrire la matrice de la réflexion du plan par rapport à la droite $(y=-x)$.
 Idem dans l'espace avec la réflexion par rapport au plan d'équation $(y=-x)$.
 
 \item\'Ecrire la matrice de la projection orthogonale de l'espace sur l'axe $(Oy)$.
 
\end{enumerate}
\end{miniexercice}
\end{frame}

\end{document}
