
%%%%%%%%%%%%%%%%%% PREAMBULE %%%%%%%%%%%%%%%%%%

\documentclass[aspectratio=169,utf8]{beamer}
%\documentclass[aspectratio=169,handout]{beamer}

\usetheme{Boadilla}
%\usecolortheme{seahorse}
\usecolortheme[RGB={245,66,24}]{structure}
\useoutertheme{infolines}

% packages
\usepackage{amsfonts,amsmath,amssymb,amsthm}
\usepackage[utf8]{inputenc}
\usepackage[T1]{fontenc}
\usepackage{lmodern}

\usepackage[francais]{babel}
\usepackage{fancybox}
\usepackage{graphicx}

\usepackage{float}
\usepackage{xfrac}

%\usepackage[usenames, x11names]{xcolor}
\usepackage{tikz}
\usepackage{pgfplots}
\usepackage{datetime}



%-----  Package unités -----
\usepackage{siunitx}
\sisetup{locale = FR,detect-all,per-mode = symbol}

%\usepackage{mathptmx}
%\usepackage{fouriernc}
%\usepackage{newcent}
%\usepackage[mathcal,mathbf]{euler}

%\usepackage{palatino}
%\usepackage{newcent}
% \usepackage[mathcal,mathbf]{euler}



% \usepackage{hyperref}
% \hypersetup{colorlinks=true, linkcolor=blue, urlcolor=blue,
% pdftitle={Exo7 - Exercices de mathématiques}, pdfauthor={Exo7}}


%section
% \usepackage{sectsty}
% \allsectionsfont{\bf}
%\sectionfont{\color{Tomato3}\upshape\selectfont}
%\subsectionfont{\color{Tomato4}\upshape\selectfont}

%----- Ensembles : entiers, reels, complexes -----
\newcommand{\Nn}{\mathbb{N}} \newcommand{\N}{\mathbb{N}}
\newcommand{\Zz}{\mathbb{Z}} \newcommand{\Z}{\mathbb{Z}}
\newcommand{\Qq}{\mathbb{Q}} \newcommand{\Q}{\mathbb{Q}}
\newcommand{\Rr}{\mathbb{R}} \newcommand{\R}{\mathbb{R}}
\newcommand{\Cc}{\mathbb{C}} 
\newcommand{\Kk}{\mathbb{K}} \newcommand{\K}{\mathbb{K}}

%----- Modifications de symboles -----
\renewcommand{\epsilon}{\varepsilon}
\renewcommand{\Re}{\mathop{\text{Re}}\nolimits}
\renewcommand{\Im}{\mathop{\text{Im}}\nolimits}
%\newcommand{\llbracket}{\left[\kern-0.15em\left[}
%\newcommand{\rrbracket}{\right]\kern-0.15em\right]}

\renewcommand{\ge}{\geqslant}
\renewcommand{\geq}{\geqslant}
\renewcommand{\le}{\leqslant}
\renewcommand{\leq}{\leqslant}
\renewcommand{\epsilon}{\varepsilon}

%----- Fonctions usuelles -----
\newcommand{\ch}{\mathop{\text{ch}}\nolimits}
\newcommand{\sh}{\mathop{\text{sh}}\nolimits}
\renewcommand{\tanh}{\mathop{\text{th}}\nolimits}
\newcommand{\cotan}{\mathop{\text{cotan}}\nolimits}
\newcommand{\Arcsin}{\mathop{\text{arcsin}}\nolimits}
\newcommand{\Arccos}{\mathop{\text{arccos}}\nolimits}
\newcommand{\Arctan}{\mathop{\text{arctan}}\nolimits}
\newcommand{\Argsh}{\mathop{\text{argsh}}\nolimits}
\newcommand{\Argch}{\mathop{\text{argch}}\nolimits}
\newcommand{\Argth}{\mathop{\text{argth}}\nolimits}
\newcommand{\pgcd}{\mathop{\text{pgcd}}\nolimits} 


%----- Commandes divers ------
\newcommand{\ii}{\mathrm{i}}
\newcommand{\dd}{\text{d}}
\newcommand{\id}{\mathop{\text{id}}\nolimits}
\newcommand{\Ker}{\mathop{\text{Ker}}\nolimits}
\newcommand{\Card}{\mathop{\text{Card}}\nolimits}
\newcommand{\Vect}{\mathop{\text{Vect}}\nolimits}
\newcommand{\Mat}{\mathop{\text{Mat}}\nolimits}
\newcommand{\rg}{\mathop{\text{rg}}\nolimits}
\newcommand{\tr}{\mathop{\text{tr}}\nolimits}


%----- Structure des exercices ------

\newtheoremstyle{styleexo}% name
{2ex}% Space above
{3ex}% Space below
{}% Body font
{}% Indent amount 1
{\bfseries} % Theorem head font
{}% Punctuation after theorem head
{\newline}% Space after theorem head 2
{}% Theorem head spec (can be left empty, meaning ‘normal’)

%\theoremstyle{styleexo}
\newtheorem{exo}{Exercice}
\newtheorem{ind}{Indications}
\newtheorem{cor}{Correction}


\newcommand{\exercice}[1]{} \newcommand{\finexercice}{}
%\newcommand{\exercice}[1]{{\tiny\texttt{#1}}\vspace{-2ex}} % pour afficher le numero absolu, l'auteur...
\newcommand{\enonce}{\begin{exo}} \newcommand{\finenonce}{\end{exo}}
\newcommand{\indication}{\begin{ind}} \newcommand{\finindication}{\end{ind}}
\newcommand{\correction}{\begin{cor}} \newcommand{\fincorrection}{\end{cor}}

\newcommand{\noindication}{\stepcounter{ind}}
\newcommand{\nocorrection}{\stepcounter{cor}}

\newcommand{\fiche}[1]{} \newcommand{\finfiche}{}
\newcommand{\titre}[1]{\centerline{\large \bf #1}}
\newcommand{\addcommand}[1]{}
\newcommand{\video}[1]{}

% Marge
\newcommand{\mymargin}[1]{\marginpar{{\small #1}}}

\def\noqed{\renewcommand{\qedsymbol}{}}


%----- Presentation ------
\setlength{\parindent}{0cm}

%\newcommand{\ExoSept}{\href{http://exo7.emath.fr}{\textbf{\textsf{Exo7}}}}

\definecolor{myred}{rgb}{0.93,0.26,0}
\definecolor{myorange}{rgb}{0.97,0.58,0}
\definecolor{myyellow}{rgb}{1,0.86,0}

\newcommand{\LogoExoSept}[1]{  % input : echelle
{\usefont{U}{cmss}{bx}{n}
\begin{tikzpicture}[scale=0.1*#1,transform shape]
  \fill[color=myorange] (0,0)--(4,0)--(4,-4)--(0,-4)--cycle;
  \fill[color=myred] (0,0)--(0,3)--(-3,3)--(-3,0)--cycle;
  \fill[color=myyellow] (4,0)--(7,4)--(3,7)--(0,3)--cycle;
  \node[scale=5] at (3.5,3.5) {Exo7};
\end{tikzpicture}}
}


\newcommand{\debutmontitre}{
  \author{} \date{} 
  \thispagestyle{empty}
  \hspace*{-10ex}
  \begin{minipage}{\textwidth}
    \titlepage  
  \vspace*{-2.5cm}
  \begin{center}
    \LogoExoSept{2.5}
  \end{center}
  \end{minipage}

  \vspace*{-0cm}
  
  % Astuce pour que le background ne soit pas discrétisé lors de la conversion pdf -> png
\begin{tikzpicture}
        \fill[opacity=0,green!60!black] (0,0)--++(0,0)--++(0,0)--++(0,0)--cycle; 
\end{tikzpicture}

% toc S'affiche trop tot :
% \tableofcontents[hideallsubsections, pausesections]
}

\newcommand{\finmontitre}{
  \end{frame}
  \setcounter{framenumber}{0}
} % ne marche pas pour une raison obscure

%----- Commandes supplementaires ------

% \usepackage[landscape]{geometry}
% \geometry{top=1cm, bottom=3cm, left=2cm, right=10cm, marginparsep=1cm
% }
% \usepackage[a4paper]{geometry}
% \geometry{top=2cm, bottom=2cm, left=2cm, right=2cm, marginparsep=1cm
% }

%\usepackage{standalone}


% New command Arnaud -- november 2011
\setbeamersize{text margin left=24ex}
% si vous modifier cette valeur il faut aussi
% modifier le decalage du titre pour compenser
% (ex : ici =+10ex, titre =-5ex

\theoremstyle{definition}
%\newtheorem{proposition}{Proposition}
%\newtheorem{exemple}{Exemple}
%\newtheorem{theoreme}{Théorème}
%\newtheorem{lemme}{Lemme}
%\newtheorem{corollaire}{Corollaire}
%\newtheorem*{remarque*}{Remarque}
%\newtheorem*{miniexercice}{Mini-exercices}
%\newtheorem{definition}{Définition}

% Commande tikz
\usetikzlibrary{calc}
\usetikzlibrary{patterns,arrows}
\usetikzlibrary{matrix}
\usetikzlibrary{fadings} 

%definition d'un terme
\newcommand{\defi}[1]{{\color{myorange}\textbf{\emph{#1}}}}
\newcommand{\evidence}[1]{{\color{blue}\textbf{\emph{#1}}}}
\newcommand{\assertion}[1]{\emph{\og#1\fg}}  % pour chapitre logique
%\renewcommand{\contentsname}{Sommaire}
\renewcommand{\contentsname}{}
\setcounter{tocdepth}{2}



%------ Figures ------

\def\myscale{1} % par défaut 
\newcommand{\myfigure}[2]{  % entrée : echelle, fichier figure
\def\myscale{#1}
\begin{center}
\footnotesize
{#2}
\end{center}}


%------ Encadrement ------

\usepackage{fancybox}


\newcommand{\mybox}[1]{
\setlength{\fboxsep}{7pt}
\begin{center}
\shadowbox{#1}
\end{center}}

\newcommand{\myboxinline}[1]{
\setlength{\fboxsep}{5pt}
\raisebox{-10pt}{
\shadowbox{#1}
}
}

%--------------- Commande beamer---------------
\newcommand{\beameronly}[1]{#1} % permet de mettre des pause dans beamer pas dans poly


\setbeamertemplate{navigation symbols}{}
\setbeamertemplate{footline}  % tiré du fichier beamerouterinfolines.sty
{
  \leavevmode%
  \hbox{%
  \begin{beamercolorbox}[wd=.333333\paperwidth,ht=2.25ex,dp=1ex,center]{author in head/foot}%
    % \usebeamerfont{author in head/foot}\insertshortauthor%~~(\insertshortinstitute)
    \usebeamerfont{section in head/foot}{\bf\insertshorttitle}
  \end{beamercolorbox}%
  \begin{beamercolorbox}[wd=.333333\paperwidth,ht=2.25ex,dp=1ex,center]{title in head/foot}%
    \usebeamerfont{section in head/foot}{\bf\insertsectionhead}
  \end{beamercolorbox}%
  \begin{beamercolorbox}[wd=.333333\paperwidth,ht=2.25ex,dp=1ex,right]{date in head/foot}%
    % \usebeamerfont{date in head/foot}\insertshortdate{}\hspace*{2em}
    \insertframenumber{} / \inserttotalframenumber\hspace*{2ex} 
  \end{beamercolorbox}}%
  \vskip0pt%
}


\definecolor{mygrey}{rgb}{0.5,0.5,0.5}
\setlength{\parindent}{0cm}
%\DeclareTextFontCommand{\helvetica}{\fontfamily{phv}\selectfont}

% background beamer
\definecolor{couleurhaut}{rgb}{0.85,0.9,1}  % creme
\definecolor{couleurmilieu}{rgb}{1,1,1}  % vert pale
\definecolor{couleurbas}{rgb}{0.85,0.9,1}  % blanc
\setbeamertemplate{background canvas}[vertical shading]%
[top=couleurhaut,middle=couleurmilieu,midpoint=0.4,bottom=couleurbas] 
%[top=fondtitre!05,bottom=fondtitre!60]



\makeatletter
\setbeamertemplate{theorem begin}
{%
  \begin{\inserttheoremblockenv}
  {%
    \inserttheoremheadfont
    \inserttheoremname
    \inserttheoremnumber
    \ifx\inserttheoremaddition\@empty\else\ (\inserttheoremaddition)\fi%
    \inserttheorempunctuation
  }%
}
\setbeamertemplate{theorem end}{\end{\inserttheoremblockenv}}

\newenvironment{theoreme}[1][]{%
   \setbeamercolor{block title}{fg=structure,bg=structure!40}
   \setbeamercolor{block body}{fg=black,bg=structure!10}
   \begin{block}{{\bf Th\'eor\`eme }#1}
}{%
   \end{block}%
}


\newenvironment{proposition}[1][]{%
   \setbeamercolor{block title}{fg=structure,bg=structure!40}
   \setbeamercolor{block body}{fg=black,bg=structure!10}
   \begin{block}{{\bf Proposition }#1}
}{%
   \end{block}%
}

\newenvironment{corollaire}[1][]{%
   \setbeamercolor{block title}{fg=structure,bg=structure!40}
   \setbeamercolor{block body}{fg=black,bg=structure!10}
   \begin{block}{{\bf Corollaire }#1}
}{%
   \end{block}%
}

\newenvironment{mydefinition}[1][]{%
   \setbeamercolor{block title}{fg=structure,bg=structure!40}
   \setbeamercolor{block body}{fg=black,bg=structure!10}
   \begin{block}{{\bf Définition} #1}
}{%
   \end{block}%
}

\newenvironment{lemme}[0]{%
   \setbeamercolor{block title}{fg=structure,bg=structure!40}
   \setbeamercolor{block body}{fg=black,bg=structure!10}
   \begin{block}{\bf Lemme}
}{%
   \end{block}%
}

\newenvironment{remarque}[1][]{%
   \setbeamercolor{block title}{fg=black,bg=structure!20}
   \setbeamercolor{block body}{fg=black,bg=structure!5}
   \begin{block}{Remarque #1}
}{%
   \end{block}%
}


\newenvironment{exemple}[1][]{%
   \setbeamercolor{block title}{fg=black,bg=structure!20}
   \setbeamercolor{block body}{fg=black,bg=structure!5}
   \begin{block}{{\bf Exemple }#1}
}{%
   \end{block}%
}


\newenvironment{miniexercice}[0]{%
   \setbeamercolor{block title}{fg=structure,bg=structure!20}
   \setbeamercolor{block body}{fg=black,bg=structure!5}
   \begin{block}{Mini-exercices}
}{%
   \end{block}%
}


\newenvironment{tp}[0]{%
   \setbeamercolor{block title}{fg=structure,bg=structure!40}
   \setbeamercolor{block body}{fg=black,bg=structure!10}
   \begin{block}{\bf Travaux pratiques}
}{%
   \end{block}%
}
\newenvironment{exercicecours}[1][]{%
   \setbeamercolor{block title}{fg=structure,bg=structure!40}
   \setbeamercolor{block body}{fg=black,bg=structure!10}
   \begin{block}{{\bf Exercice }#1}
}{%
   \end{block}%
}
\newenvironment{algo}[1][]{%
   \setbeamercolor{block title}{fg=structure,bg=structure!40}
   \setbeamercolor{block body}{fg=black,bg=structure!10}
   \begin{block}{{\bf Algorithme}\hfill{\color{gray}\texttt{#1}}}
}{%
   \end{block}%
}


\setbeamertemplate{proof begin}{
   \setbeamercolor{block title}{fg=black,bg=structure!20}
   \setbeamercolor{block body}{fg=black,bg=structure!5}
   \begin{block}{{\footnotesize Démonstration}}
   \footnotesize
   \smallskip}
\setbeamertemplate{proof end}{%
   \end{block}}
\setbeamertemplate{qed symbol}{\openbox}


\makeatother
\usecolortheme[RGB={205,100,0}]{structure}

% Commande spécifique à ce chapitre
%\newcommand{\Mat}{\mathop{\text{Mat}}\nolimits}

%%%%%%%%%%%%%%%%%%%%%%%%%%%%%%%%%%%%%%%%%%%%%%%%%%%%%%%%%%%%%
%%%%%%%%%%%%%%%%%%%%%%%%%%%%%%%%%%%%%%%%%%%%%%%%%%%%%%%%%%%%%


\begin{document}


\title{{\bf L'espace vectoriel $\Rr^n$}}
\subtitle{Propriétés des applications linéaires}

\begin{frame}
  
  \debutmontitre

  \pause

{\footnotesize
\hfill
\setbeamercovered{transparent=50}
\begin{minipage}{0.6\textwidth}
  \begin{itemize}
    \item<3-> Composition d'applications linéaires et produit de matrices
    \item<4-> Application linéaire bijective et matrice inversible
    \item<5-> Caractérisation des applications linéaires
  \end{itemize}
\end{minipage}
}

\end{frame}

\setcounter{framenumber}{0}


%%%%%%%%%%%%%%%%%%%%%%%%%%%%%%%%%%%%%%%%%%%%%%%%%%%%%%%%%%%%%%%%
\section{Composition et produit}

\begin{frame}
Soient deux applications linéaires
$$f : \Rr^p \longrightarrow \Rr^n \qquad \text{ et } \qquad g : \Rr^q \longrightarrow \Rr^p$$
\pause
Et leur composition :
$$\Rr^q \overset{g}{\longrightarrow} \Rr^p\overset{f}{\longrightarrow} \Rr^n
 \qquad \qquad
f \circ g :  \Rr^q \longrightarrow \Rr^n
$$

\pause

L'application $f \circ g$ est une application linéaire.
\pause
Notons : 
\begin{itemize}
  \item $A = \Mat(f) \in M_{n,p}(\Rr)$ la matrice associée à $f$ : $X \mapsto AX$
  \pause
  \item $B = \Mat(g) \in M_{p,q}(\Rr)$ la matrice associée à $g$ : $X \mapsto BX$
  \pause
  \item $C = \Mat(f \circ g) \in M_{n,q}(\Rr)$ la matrice associée à $f \circ g$
\end{itemize}
\pause
Alors $C = AB$
\pause
\mybox{$\Mat(f \circ g) = \Mat (f) \times \Mat (g)$}

\pause
Preuve. Pour $X \in \Rr^q$ :
$$(f \circ g)(X)  =  f \big(g(X)\big) = f\big( BX \big) = A(BX) = (AB)X$$

\end{frame}


\begin{frame}
\begin{exemple}
\begin{itemize}[<+->]\setlength{\itemsep}{8pt}
  \item Soit $f: \Rr^2 \longrightarrow \Rr^2$ 
  la réflexion par rapport à la droite $(y = x)$
  
  \item $A = \Mat(f) = \begin{pmatrix} 0 & 1 \\ 1 & 0 \end{pmatrix}$
  
  \item Soit $g: \Rr^2 \longrightarrow \Rr^2$ 
la rotation d'angle $\theta=\frac\pi3$ (centrée à l'origine)
  
  \item $B = \Mat(g) =  
\begin{pmatrix}
\cos\theta & -\sin\theta\\
\sin\theta & \cos\theta
\end{pmatrix}
=
\begin{pmatrix} 
\frac12 & -\frac{\sqrt3}{2} \\  
\frac{\sqrt3}{2} & \frac12
\end{pmatrix}
$
  
  \item $C = \Mat (f\circ g)  = \Mat (f) \times \Mat (g)$

  \item $C =\begin{pmatrix} 0 & 1 \\ 1 & 0 \end{pmatrix} \times  
\begin{pmatrix} \frac12 & -\frac{\sqrt3}{2} \\ \frac{\sqrt3}{2} & \frac12 \end{pmatrix}
= \begin{pmatrix} \frac{\sqrt3}{2} & \frac12  \\ \frac12 & -\frac{\sqrt3}{2}\end{pmatrix}
$
\end{itemize}

\end{exemple} 
\end{frame}


\begin{frame}
\begin{exemple}
Voici pour $X=\left(\begin{smallmatrix}1\\0\end{smallmatrix} \right)$ les images
$f(X)$, $g(X)$, $f\circ g(X)$, $g\circ f(X)$ :
\myfigure{1.2}{
\tikzinput{fig_erene13} 
} 


\end{exemple} 
\end{frame}


\begin{frame}
\begin{exemple}
\begin{itemize}[<+->]\setlength{\itemsep}{8pt}
  \item Pour $g\circ f$ : $D = \Mat (g \circ f)  = \Mat (g) \times \Mat (f)$
  \item $D = \begin{pmatrix} \frac12 & -\frac{\sqrt3}{2} \\ \frac{\sqrt3}{2} & \frac12 \end{pmatrix}
 \times  \begin{pmatrix} 0 & 1 \\ 1 & 0 \end{pmatrix}
= \begin{pmatrix}-\frac{\sqrt3}{2} & \frac12 \\  \frac12 & \frac{\sqrt3}{2} \end{pmatrix}
$
  \item $C=AB$ différent de $D=BA$
  \item La composition d'applications linéaires, 
  comme la multiplication des matrices, n'est pas commutative
\end{itemize}
\end{exemple} 
\end{frame}

%%%%%%%%%%%%%%%%%%%%%%%%%%%%%%%%%%%%%%%%%%%%%%%%%%%%%%%%%%%%%%%%
\section{Bijection et inverse}

\begin{frame}

\begin{theoreme}
\label{th:matinv}
Une application linéaire $f : \Rr^n \to \Rr^n$ est bijective
si et seulement si sa matrice associée $A= \Mat(f) \in M_n(\Rr)$ 
est inversible
\end{theoreme}

\pause
\mybox{$\Mat(f^{-1}) = \big(\Mat (f) \big)^{-1}$}

\pause

\begin{itemize}[<+->]
  \item $f$ est définie par $f(X) = AX$
  \item Supposons $f$ est bijective
  \item $f(X)=Y \iff X = f^{-1}(Y)$
  \item $AX=Y \iff X = A^{-1}Y$
  \item Donc $f^{-1}(Y)=A^{-1}Y$
  \item Conséquence : la matrice de $f^{-1}$ est $A^{-1}$
\end{itemize}
\end{frame}


\begin{frame}

\begin{exemple}
\begin{itemize}[<+->]\setlength{\itemsep}{8pt}
  \item Soit $f: \Rr^2 \longrightarrow \Rr^2$ la rotation d'angle $\theta$
  
  \item $\Mat(f) = 
\begin{pmatrix}
\cos\, \theta & & -\sin \theta\\
\sin\, \theta &&\cos\, \theta  
\end{pmatrix}$
  
  \item Alors $f^{-1}: \Rr^2 \longrightarrow \Rr^2$ est la rotation d'angle $-\theta$
  
  \item $\Mat(f^{-1}) = \big(\Mat (f) \big)^{-1}$

  \item $\Mat(f^{-1}) = \begin{pmatrix}
\cos\, \theta &&\sin \,\theta\\
-\sin\, \theta && \cos\,\theta    
\end{pmatrix}
= \begin{pmatrix}
\cos\, (-\theta ) && -\sin\, (-\theta )\\
\sin\, (-\theta ) && \cos (-\theta )    
  \end{pmatrix}
$
\end{itemize}
\end{exemple}

\end{frame}


% \begin{frame}
% \begin{exemple}
% \myfigure{1}{
% \begin{tikzpicture}

      \draw[->,>=latex,thick, gray] (-1,0)--(4,0) node[below,black] {$x$};
       \draw[->,>=latex,thick, gray] (0,-1)--(0,3) node[right,black] {$y$};


       \draw[->,>=latex, gray] (0,0)--(3,1.5);
       \fill[myred] (3,1.5) circle (2pt);

       \draw[->,>=latex, gray] (0,0)--(3,0.7);
       \fill[myred] (3,0.7) circle (2pt);

       \draw[dashed,thick, gray] (3,2.5)--(3,0);
       \draw[->,>=latex, gray] (0,0)--(3,2.5);
       \fill[myred] (3,2.5) circle (2pt);

       \fill[myred] (3,0) circle (2pt);
      \node[right] at (3,1.5) {$\begin{pmatrix}x\\y\end{pmatrix}$};
      \node[below] at (3,0) {$\begin{pmatrix}x\\0\end{pmatrix}$};
           \node[below left] at (0,0) {$O$};

\end{tikzpicture}
 
% }
% \vspace*{-4ex}
% \begin{itemize}\setlength{\itemsep}{4pt}
%   \item Soit $f: \Rr^2 \longrightarrow \Rr^2$ la projection sur l'axe $(Ox)$
%   
%   \item
%   \begin{itemize}
%     \item $f$ n'est pas injective
%     \item Pour $x$ fixé et tout $y\in \Rr$,
% $f\left(\begin{smallmatrix}x\\y\end{smallmatrix}\right) 
% = \left(\begin{smallmatrix}x\\0\end{smallmatrix}\right)$
%     \item $f$ n'est pas non plus surjective
%   \end{itemize}
%   
%   \item La matrice de $f$ est 
% $\left(\begin{smallmatrix}
% 1 & 0\\
% 0 & 0\end{smallmatrix}\right)$
% 
%   \item La matrice n'est pas inversible
%   
% \end{itemize}
% 
% \end{exemple}
% \end{frame}



%%%%%%%%%%%%%%%%%%%%%%%%%%%%%%%%%%%%%%%%%%%%%%%%%%%%%%%%%%%%%%%%
\section{Caractérisation}


\begin{frame}
\begin{theoreme}
\label{th:applinrn}
Une application $f: \Rr^p \longrightarrow \Rr^n$ est linéaire si et seulement si  
pour tous les vecteurs $u$, $v$ de $\Rr^p$ et pour tout scalaire $\lambda \in \Rr$,
on a
\begin{enumerate}
  \item[(i)]  $f(u+v) = f(u) + f(v)$
  \item[(ii)] $ f(\lambda u) = \lambda f(u)$
\end{enumerate} 
\end{theoreme}

\end{frame}


\begin{frame}
\begin{mydefinition}
Les \defi{vecteurs de la base canonique} de $\Rr^p$ sont
$$
e_1 = \begin{pmatrix} 1\\0\\0\\\vdots\\0\end{pmatrix}\qquad 
e_2 = \begin{pmatrix} 0\\1\\0\\\vdots\\0\end{pmatrix}\qquad\cdots\qquad
e_p = \begin{pmatrix} 0\\0\\\vdots\\0\\1\end{pmatrix}
$$
\end{mydefinition}

\pause

\begin{corollaire}
Soit $f: \Rr^p \longrightarrow \Rr^n$ une application linéaire. 
Alors la matrice de $f$ (dans les bases canoniques 
de $\Rr^p$ vers $\Rr^n$) est donnée par 
$$\Mat(f) = \begin{pmatrix} f(e_1)&f(e_2)& \cdots& f(e_p)\end{pmatrix}$$
\end{corollaire}

\end{frame}


\begin{frame}
\begin{exemple}
\begin{itemize}[<+->]\setlength{\itemsep}{4pt}
  \item Considérons l'application linéaire $f : \Rr^3 \to \Rr^4$ définie par 
$$ \left\{\begin{array}{ccccccc}
y_1 & = & 2x_1  &+ x_2   & -  x_3\\
y_2 & = & -x_1 &-  4x_2 &         \\
y_3 & = & 5x_1  &+  x_2  & +  x_3 \\
y_4 & = &       & 3x_2  & +  2x_3
\end{array}\right.
$$
  
  \item La base canonique   
  $\left(\begin{smallmatrix} 1 \\ 0 \\ 0 \end{smallmatrix} \right)$,
  $\left(\begin{smallmatrix} 0 \\ 1 \\ 0 \end{smallmatrix} \right)$,
  $\left(\begin{smallmatrix} 0 \\ 0 \\ 1 \end{smallmatrix} \right)$
  
  \item $f\left(\begin{smallmatrix} 1 \\ 0 \\ 0 \end{smallmatrix}\right) = \left(\begin{smallmatrix} 2 \\ -1 \\ 5 \\ 0 \end{smallmatrix}\right)\qquad 
  f\left(\begin{smallmatrix} 0 \\ 1 \\ 0 \end{smallmatrix}\right) = \left(\begin{smallmatrix} 1 \\ -4 \\ 1 \\ 3 \end{smallmatrix}\right)\qquad
  f\left(\begin{smallmatrix} 0 \\ 0 \\ 1 \end{smallmatrix}\right) = \left(\begin{smallmatrix} -1 \\ 0 \\ 1 \\ 2 \end{smallmatrix}\right)
  $
  
  \item Donc la matrice de $f$ est 
$\Mat(f) = 
\begin{pmatrix}
2 & 1 & -1 \\
-1& -4& 0  \\
 5& 1 & 1 \\
 0 & 3 & 2 \\              
\end{pmatrix}$
\end{itemize}

  

\end{exemple}
\end{frame}


\begin{frame}
\begin{exemple}
\begin{itemize}[<+->]\setlength{\itemsep}{4pt}
  \item Soit $f : \Rr^2 \to \Rr^2$ la réflexion par rapport à la droite $(y=x)$
  
  \item Soit $g$ la rotation du plan d'angle $\frac\pi3$ centrée à l'origine
  
  \item Calculons la matrice de l'application $f\circ g$
  
  \item La base canonique de $\Rr^2$ est formée des vecteurs
$\left(\begin{smallmatrix} 1 \\ 0 \end{smallmatrix} \right)$ 
et $\left(\begin{smallmatrix} 0 \\ 1 \end{smallmatrix} \right)$
  
  \item $f\circ g \begin{pmatrix} 1 \\ 0 \end{pmatrix}
= f \begin{pmatrix} \frac12 \\ \frac{\sqrt{3}}{2} \end{pmatrix}
= \begin{pmatrix} \frac{\sqrt{3}}{2} \\ \frac12 \end{pmatrix}$

  \item $f\circ g \begin{pmatrix} 0 \\ 1 \end{pmatrix}
= f \begin{pmatrix}  -\frac{\sqrt{3}}{2} \\ \frac12  \end{pmatrix}
= \begin{pmatrix} \frac12 \\ -\frac{\sqrt{3}}{2}  \end{pmatrix}$

  \item La matrice de $f \circ g$ est :
$\Mat(f) = 
\begin{pmatrix}
\frac{\sqrt{3}}{2} &  \frac12  \\
\frac12& -\frac{\sqrt{3}}{2}\\            
\end{pmatrix}$

\end{itemize}
\end{exemple}
\end{frame}


%%%%%%%%%%%%%%%%%%%%%%%%%%%%%%%%%%%%%%%%%%%%%%%%%%%%%%%%%%%%%%%%
\section{Mini-exercices}

\begin{frame}
\begin{miniexercice}
\begin{enumerate}
  \item Soit $f$ la réflexion du plan par rapport à l'axe $(Ox)$ et soit $g$
  la rotation d'angle $\frac{2\pi}{3}$ centrée à l'ori\-gine. Calculer la matrice de 
  $f\circ g$ de deux façons différentes (produit de matrices et image de la base canonique).
  Cette matrice est-elle inversible ? Si oui, calculer l'inverse. Interprétation géométrique.
  Même question avec $g\circ f$.
  
  \item Soit $f$ la projection orthogonale de l'espace sur le plan $(Oxz)$ et soit $g$
  la rotation d'angle $\frac{\pi}{2}$ d'axe $(Oy)$. Calculer la matrice de 
  $f\circ g$ de deux façons différentes (produit de matrices et image de la base canonique).
  Cette matrice est-elle inversible ? Si oui, calculer l'inverse. Interprétation géométrique.
  Même question avec $g\circ f$.  
\end{enumerate}
\end{miniexercice}
\end{frame}

\end{document}
