
%%%%%%%%%%%%%%%%%% PREAMBULE %%%%%%%%%%%%%%%%%%


\documentclass[12pt]{article}

\usepackage{amsfonts,amsmath,amssymb,amsthm}
\usepackage[utf8]{inputenc}
\usepackage[T1]{fontenc}
\usepackage[francais]{babel}


% packages
\usepackage{amsfonts,amsmath,amssymb,amsthm}
\usepackage[utf8]{inputenc}
\usepackage[T1]{fontenc}
%\usepackage{lmodern}

\usepackage[francais]{babel}
\usepackage{fancybox}
\usepackage{graphicx}

\usepackage{float}

%\usepackage[usenames, x11names]{xcolor}
\usepackage{tikz}
\usepackage{datetime}

\usepackage{mathptmx}
%\usepackage{fouriernc}
%\usepackage{newcent}
\usepackage[mathcal,mathbf]{euler}

%\usepackage{palatino}
%\usepackage{newcent}


% Commande spéciale prompteur

%\usepackage{mathptmx}
%\usepackage[mathcal,mathbf]{euler}
%\usepackage{mathpple,multido}

\usepackage[a4paper]{geometry}
\geometry{top=2cm, bottom=2cm, left=1cm, right=1cm, marginparsep=1cm}

\newcommand{\change}{{\color{red}\rule{\textwidth}{1mm}\\}}

\newcounter{mydiapo}

\newcommand{\diapo}{\newpage
\hfill {\normalsize  Diapo \themydiapo \quad \texttt{[\jobname]}} \\
\stepcounter{mydiapo}}


%%%%%%% COULEURS %%%%%%%%%%

% Pour blanc sur noir :
%\pagecolor[rgb]{0.5,0.5,0.5}
% \pagecolor[rgb]{0,0,0}
% \color[rgb]{1,1,1}



%\DeclareFixedFont{\myfont}{U}{cmss}{bx}{n}{18pt}
\newcommand{\debuttexte}{
%%%%%%%%%%%%% FONTES %%%%%%%%%%%%%
\renewcommand{\baselinestretch}{1.5}
\usefont{U}{cmss}{bx}{n}
\bfseries

% Taille normale : commenter le reste !
%Taille Arnaud
%\fontsize{19}{19}\selectfont

% Taille Barbara
%\fontsize{21}{22}\selectfont

%Taille François
\fontsize{25}{30}\selectfont

%Taille Pascal
%\fontsize{25}{30}\selectfont

%Taille Laura
%\fontsize{30}{35}\selectfont


%\myfont
%\usefont{U}{cmss}{bx}{n}

%\Huge
%\addtolength{\parskip}{\baselineskip}
}


% \usepackage{hyperref}
% \hypersetup{colorlinks=true, linkcolor=blue, urlcolor=blue,
% pdftitle={Exo7 - Exercices de mathématiques}, pdfauthor={Exo7}}


%section
% \usepackage{sectsty}
% \allsectionsfont{\bf}
%\sectionfont{\color{Tomato3}\upshape\selectfont}
%\subsectionfont{\color{Tomato4}\upshape\selectfont}

%----- Ensembles : entiers, reels, complexes -----
\newcommand{\Nn}{\mathbb{N}} \newcommand{\N}{\mathbb{N}}
\newcommand{\Zz}{\mathbb{Z}} \newcommand{\Z}{\mathbb{Z}}
\newcommand{\Qq}{\mathbb{Q}} \newcommand{\Q}{\mathbb{Q}}
\newcommand{\Rr}{\mathbb{R}} \newcommand{\R}{\mathbb{R}}
\newcommand{\Cc}{\mathbb{C}} 
\newcommand{\Kk}{\mathbb{K}} \newcommand{\K}{\mathbb{K}}

%----- Modifications de symboles -----
\renewcommand{\epsilon}{\varepsilon}
\renewcommand{\Re}{\mathop{\text{Re}}\nolimits}
\renewcommand{\Im}{\mathop{\text{Im}}\nolimits}
%\newcommand{\llbracket}{\left[\kern-0.15em\left[}
%\newcommand{\rrbracket}{\right]\kern-0.15em\right]}

\renewcommand{\ge}{\geqslant}
\renewcommand{\geq}{\geqslant}
\renewcommand{\le}{\leqslant}
\renewcommand{\leq}{\leqslant}

%----- Fonctions usuelles -----
\newcommand{\ch}{\mathop{\mathrm{ch}}\nolimits}
\newcommand{\sh}{\mathop{\mathrm{sh}}\nolimits}
\renewcommand{\tanh}{\mathop{\mathrm{th}}\nolimits}
\newcommand{\cotan}{\mathop{\mathrm{cotan}}\nolimits}
\newcommand{\Arcsin}{\mathop{\mathrm{Arcsin}}\nolimits}
\newcommand{\Arccos}{\mathop{\mathrm{Arccos}}\nolimits}
\newcommand{\Arctan}{\mathop{\mathrm{Arctan}}\nolimits}
\newcommand{\Argsh}{\mathop{\mathrm{Argsh}}\nolimits}
\newcommand{\Argch}{\mathop{\mathrm{Argch}}\nolimits}
\newcommand{\Argth}{\mathop{\mathrm{Argth}}\nolimits}
\newcommand{\pgcd}{\mathop{\mathrm{pgcd}}\nolimits} 

\newcommand{\Card}{\mathop{\text{Card}}\nolimits}
\newcommand{\Ker}{\mathop{\text{Ker}}\nolimits}
\newcommand{\id}{\mathop{\text{id}}\nolimits}
\newcommand{\ii}{\mathrm{i}}
\newcommand{\dd}{\mathrm{d}}
\newcommand{\Vect}{\mathop{\text{Vect}}\nolimits}
\newcommand{\Mat}{\mathop{\mathrm{Mat}}\nolimits}
\newcommand{\rg}{\mathop{\text{rg}}\nolimits}
\newcommand{\tr}{\mathop{\text{tr}}\nolimits}
\newcommand{\ppcm}{\mathop{\text{ppcm}}\nolimits}

%----- Structure des exercices ------

\newtheoremstyle{styleexo}% name
{2ex}% Space above
{3ex}% Space below
{}% Body font
{}% Indent amount 1
{\bfseries} % Theorem head font
{}% Punctuation after theorem head
{\newline}% Space after theorem head 2
{}% Theorem head spec (can be left empty, meaning ‘normal’)

%\theoremstyle{styleexo}
\newtheorem{exo}{Exercice}
\newtheorem{ind}{Indications}
\newtheorem{cor}{Correction}


\newcommand{\exercice}[1]{} \newcommand{\finexercice}{}
%\newcommand{\exercice}[1]{{\tiny\texttt{#1}}\vspace{-2ex}} % pour afficher le numero absolu, l'auteur...
\newcommand{\enonce}{\begin{exo}} \newcommand{\finenonce}{\end{exo}}
\newcommand{\indication}{\begin{ind}} \newcommand{\finindication}{\end{ind}}
\newcommand{\correction}{\begin{cor}} \newcommand{\fincorrection}{\end{cor}}

\newcommand{\noindication}{\stepcounter{ind}}
\newcommand{\nocorrection}{\stepcounter{cor}}

\newcommand{\fiche}[1]{} \newcommand{\finfiche}{}
\newcommand{\titre}[1]{\centerline{\large \bf #1}}
\newcommand{\addcommand}[1]{}
\newcommand{\video}[1]{}

% Marge
\newcommand{\mymargin}[1]{\marginpar{{\small #1}}}



%----- Presentation ------
\setlength{\parindent}{0cm}

%\newcommand{\ExoSept}{\href{http://exo7.emath.fr}{\textbf{\textsf{Exo7}}}}

\definecolor{myred}{rgb}{0.93,0.26,0}
\definecolor{myorange}{rgb}{0.97,0.58,0}
\definecolor{myyellow}{rgb}{1,0.86,0}

\newcommand{\LogoExoSept}[1]{  % input : echelle
{\usefont{U}{cmss}{bx}{n}
\begin{tikzpicture}[scale=0.1*#1,transform shape]
  \fill[color=myorange] (0,0)--(4,0)--(4,-4)--(0,-4)--cycle;
  \fill[color=myred] (0,0)--(0,3)--(-3,3)--(-3,0)--cycle;
  \fill[color=myyellow] (4,0)--(7,4)--(3,7)--(0,3)--cycle;
  \node[scale=5] at (3.5,3.5) {Exo7};
\end{tikzpicture}}
}



\theoremstyle{definition}
%\newtheorem{proposition}{Proposition}
%\newtheorem{exemple}{Exemple}
%\newtheorem{theoreme}{Théorème}
\newtheorem{lemme}{Lemme}
\newtheorem{corollaire}{Corollaire}
%\newtheorem*{remarque*}{Remarque}
%\newtheorem*{miniexercice}{Mini-exercices}
%\newtheorem{definition}{Définition}




%definition d'un terme
\newcommand{\defi}[1]{{\color{myorange}\textbf{\emph{#1}}}}
\newcommand{\evidence}[1]{{\color{blue}\textbf{\emph{#1}}}}



 %----- Commandes divers ------

\newcommand{\codeinline}[1]{\texttt{#1}}

%%%%%%%%%%%%%%%%%%%%%%%%%%%%%%%%%%%%%%%%%%%%%%%%%%%%%%%%%%%%%
%%%%%%%%%%%%%%%%%%%%%%%%%%%%%%%%%%%%%%%%%%%%%%%%%%%%%%%%%%%%%



\begin{document}

\debuttexte


%%%%%%%%%%%%%%%%%%%%%%%%%%%%%%%%%%%%%%%%%%%%%%%%%%%%%%%%%%%
\diapo

Ce chapitre est consacré à l'ensemble $\Rr^n$ vu comme espace vectoriel.
Il peut être vu de plusieurs façons :

(1) un cours minimal sur les espaces vectoriels pour ceux qui n'auraient besoin que
  de $\Rr^n$,
  
(2) une introduction avant d'attaquer le cours détaillé sur les espaces vectoriels,
  
(3) une source d'exemples à lire en parallèle du cours sur les espaces vectoriels. 


\change

\change


Nous commençons dans cette partie par définir les opérations sur les vecteurs.

\change

Puis nous parlerons de la représentation des vecteurs de $\Rr^n$.

\change

On termine par le produit scalaire .



%%%%%%%%%%%%%%%%%%%%%%%%%%%%%%%%%%%%%%%%%%%%%%%%%%%%%%%%%%%
\diapo

L'ensemble des nombres réels $\Rr$ est souvent représenté par une droite. C'est un
espace de dimension~$1$.  
  

Le plan est formé des couples 
$\left(\begin{smallmatrix}x_1\\  x_2\end{smallmatrix}\right)$ de nombres réels. 
Il est noté $\Rr^2$. C'est un espace à deux dimensions.
  
\change
L'espace de dimension $3$ est constitué des triplets de nombres réels
$\left(\begin{smallmatrix}x_1\\x_2\\x_3\end{smallmatrix}\right)$. 
Il est noté $\Rr^3$.

\change
Le symbole $\left(\begin{smallmatrix}x_1\\x_2\\x_3\end{smallmatrix}\right)$ 
a deux interprétations géométriques : soit comme un point de l'espace ,
soit comme un vecteur.


%%%%%%%%%%%%%%%%%%%%%%%%%%%%%%%%%%%%%%%%%%%%%%%%%%%%%%%%%%%
\diapo

On généralise ces notions en considérant des espaces de dimension $n$ 
pour tout entier positif $n = 1,\, 2,\, 3,\, 4,\, \dots$

\change
Les éléments de l'espace de dimension $n$ sont les $n$-uplets
$\left(\begin{smallmatrix} x_1\\ x_2 \\ \vdots \\ x_n \end{smallmatrix}\right)$ 
de nombres réels. 

\change
L'espace de dimension $n$ est noté $\Rr^n$. 

\change
Comme en dimensions $2$ et $3$, le $n$-uplet
$\left(\begin{smallmatrix} x_1\\x_2 \\  \vdots \\ x_n \end{smallmatrix}\right)$
note aussi bien un point qu'un vecteur de l'espace de dimension $n$. 



%%%%%%%%%%%%%%%%%%%%%%%%%%%%%%%%%%%%%%%%%%%%%%%%%%%%%%%%%%%
\diapo

Soient $u$ et $v$ deux vecteurs de $\Rr^n$. 

\change

Leur somme est par définition le vecteur 
$u + v$ dont les coordonnées sont 
$\begin{pmatrix}u_1 + v_1 \\ \vdots \\ u_n + v_n\end{pmatrix}.$

\change
Pour $\lambda$ un réel, la multiplication du vecteur $u$ par $\lambda$
est le vecteur de coordonnées 
$\begin{pmatrix}\lambda u_1 \\ \vdots \\ \lambda u_n \end{pmatrix}.$

\change
Le \defi{vecteur nul} est tout simplement le vecteur 
$0 = \left(\begin{smallmatrix} 0 \\ \vdots \\ 0 \end{smallmatrix}\right)$
dont toutes les coordonnées sont nulles.
  
\change
Enfin l'\defi{opposé} du vecteur $u$
est le vecteur notée $-u$
dont les coordonnées sont 
$\left(\begin{smallmatrix} -u_1\\ \vdots \\ -u_n \end{smallmatrix}\right)$.


%%%%%%%%%%%%%%%%%%%%%%%%%%%%%%%%%%%%%%%%%%%%%%%%%%%%%%%%%%%
\diapo

Voici des vecteurs dans le plan $\Rr^2$ 

Voici $u$ et $v$.

Voici la somme $u+v$.

Voici $\lambda\cdot u$, pour cet exemple $\lambda = 2$.

Le vecteur nul est ici.

Et l'opposé du vecteur $u$ est le vecteur $-u$.

Dans un premier temps, vous pouvez noter les vecteurs avec des flèches au-dessus de la lettre,
Mais il faudra rapidement s'habituer à s'en passer.

%%%%%%%%%%%%%%%%%%%%%%%%%%%%%%%%%%%%%%%%%%%%%%%%%%%%%%%%%%%
\diapo


  Soient $u$, $v$ et $w$ des vecteurs de $\Rr^n$
  et soient aussi $\lambda, \mu$ deux réels. Alors, on a 
  une liste de huit propriétés.

 \change 
 $u + v = v + u$
 
 \change 
 $u + (v+w) = (u+v) +w$
 
 \change 
 $u + 0 = 0 + u = u$
 
 \change 
 $u + (-u) = 0$
 
 \change 
 $1 \cdot u = u$
 
 \change 
 $\lambda \cdot (\mu \cdot u) = (\lambda\mu )\cdot u$
 
 \change 
 $\lambda \cdot (u+v) = \lambda \cdot u + \lambda \cdot v$
 
 \change 
 $(\lambda + \mu ) \cdot u = \lambda \cdot u + \mu \cdot u$


Chacune de ces propriétés découle directement de la définition de la somme 
et de la multiplication par un scalaire. Ces huit propriétés font de $\Rr^n$ un
\defi{espace vectoriel}. Dans le cadre général, ce sont ces huit propriétés qui définissent 
ce qu'est un espace vectoriel.

%%%%%%%%%%%%%%%%%%%%%%%%%%%%%%%%%%%%%%%%%%%%%%%%%%%%%%%%%%%
\diapo

Voici une figure, qui illustre la première propriété : 
la commutativité de l'addition.

En effet la somme $u+v$ et la somme $v+u$ donne le même vecteur.

La somme se construit en traçant un parallélogramme.

%%%%%%%%%%%%%%%%%%%%%%%%%%%%%%%%%%%%%%%%%%%%%%%%%%%%%%%%%%%
\diapo

Soit 
$u = \left(\begin{smallmatrix} u_1\\ \vdots \\ u_n \end{smallmatrix}\right)$ 
un vecteur de $\Rr^n$. 

\change
\'Ecrit ainsi on l'appelle un \defi{vecteur colonne} 

\change
et on considère naturellement $u$ comme une matrice de taille $n\times 1$. 

\change
Parfois, on rencontre aussi des \defi{vecteurs lignes} : 

\change
on peut voir le vecteur $u$ comme une matrice $1\times n$, de la forme
$(u_1,\dots,u_n)$. 

\change
En fait, le vecteur ligne correspondant à $u$
est le transposé $u^T$ du vecteur colonne $u$.

Les opérations de somme et de produit par un scalaire définies ci-dessus 
pour les vecteurs coïncident parfaitement avec les opérations définies sur
les matrices.


%%%%%%%%%%%%%%%%%%%%%%%%%%%%%%%%%%%%%%%%%%%%%%%%%%%%%%%%%%%
\diapo

Soient $u = \left(\begin{smallmatrix} u_1\\ \vdots \\ u_n \end{smallmatrix}\right)$ 
et $v = \left(\begin{smallmatrix} v_1\\ \vdots \\ v_n \end{smallmatrix}\right)$
deux vecteurs de $\Rr^n$. 


\change
On définit le \defi{produit scalaire} de $u$ et $v$  par
$$\langle u \mid v \rangle = u_1 v_1 + u_2 v_2 + \dots + u_nv_n.$$

\change
C'est un scalaire, c-à-d un nombre réel. 

Remarquons que cette définition généralise la notion 
de produit scalaire dans le plan $\Rr^2$ et dans l'espace $\Rr^3$.

\change

Une autre façon de voir le produit scalaire
est de représenter $u$ comme un vecteur ligne, c-à-d on prend $u^T$.
,
et on représente $v$ comme un vecteur colonne.
Alors $\langle u \mid v \rangle$ est le produit de matrice $u^T \times v$ 

qui est une matrice $1\times 1$ que l'on considère bien sûr comme un réel.


%%%%%%%%%%%%%%%%%%%%%%%%%%%%%%%%%%%%%%%%%%%%%%%%%%%%%%%%%%%
\diapo

Voici quelques exercices pour vous familiariser avec les vecteurs de $\Rr^n$.

\end{document}
