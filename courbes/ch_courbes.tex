\documentclass[class=report,crop=false]{standalone}
\usepackage[screen]{../exo7book}

\begin{document}

%====================================================================
\chapitre{Courbes paramétrées}
%====================================================================


\insertvideo{QV-567w4zBg}{partie 1. Notions de base}

\insertvideo{zcZCMAaNmKc}{partie 2. Tangente à une courbe paramétrée}

\insertvideo{X1N4CAVeHZI}{partie 3. Points singuliers -- Branches infinies}

\insertvideo{vBWsG1XD6F4}{partie 4. Plan d'étude d'une courbe paramétrée}

\insertvideo{XMIAlkOWy4o}{partie 5. Courbes en polaires : théorie}

\insertvideo{rLJQQ6WeWWQ}{partie 6. Courbes en polaires : exemples}

\insertfiche{fic00164.pdf}{Courbes planes}

\bigskip

Dans ce chapitre nous allons voir les propriétés fondamentales des courbes paramétrées.
Commençons par présenter une courbe particulièrement intéressante.
La \defi{cycloïde} est la courbe que parcourt un point choisi de la roue d'un vélo,
lorsque le vélo avance. Les coordonnées $(x,y)$ de ce point $M$ varient
en fonction du temps :
$$\left\{\begin{array}{rcl}
x(t) &=& r(t-\sin t) \\
y(t) &=& r(1-\cos t)
\end{array} \right.$$
où $r$ est le rayon de la roue.

\myfigure{0.7}{
\tikzinput{fig_courbes_part0_01}
}

La cycloïde a des propriétés remarquables. Par exemple, la
cycloïde renversée est une courbe \emph{brachistochrone} :
c'est-à-dire que c'est la courbe
qui permet à une bille d'arriver le plus vite possible d'un point $A$ à un point $B$.
Contrairement à ce que l'on pourrait croire ce n'est pas
une ligne droite, mais bel et bien la cycloïde.
Sur le dessin suivant les deux billes sont lâchées en $A$ à l'instant $t_0$,
l'une sur le segment $[AB]$ ; elle aura donc une accélération constante.
La seconde parcourt la cycloïde renversée, ayant une tangente verticale en $A$ et passant par $B$.
La bille accélère beaucoup au début et elle atteint $B$ bien avant l'autre bille
(à l'instant $t_4$ sur le dessin).
Notez que la bille passe même par des positions en-dessous de $B$ (par exemple en $t_3$).

\myfigure{1}{
\tikzinput{fig_courbes_part0_03}
}

% La cycloïde renversée est aussi une courbe \emph{tautochrone} : si on pose deux billes
% avec des positions initiales différentes sur la courbes, elles vont
% cependant arriver en même temps au point le plus bas.
% En effet la bille qui part de plus loin, prend plus de vitesse au démarrage
% car la courbe y a une pente plus forte.


%%%%%%%%%%%%%%%%%%%%%%%%%%%%%%%%%%%%%%%%%%%%%%%%%%%%%%%%%%%%%%%%
\section{Notions de base}

%---------------------------------------------------------------
\subsection{Définition d'une courbe paramétrée}


\begin{definition}
Une \defi{courbe paramétrée plane}\index{courbe!parametree@paramétrée} est une application
$$\begin{array}[t]{cccc}
f :&D\subset\Rr&\rightarrow&\Rr^2\\
 &t&\mapsto&f(t)
\end{array}$$ d'un sous-ensemble $D$ de $\Rr$ dans $\Rr^2$.
\end{definition}



Ainsi, une \emph{courbe paramétrée} est une application qui,
à un réel $t$ (le \emph{paramètre}), associe \emph{un point}
du plan. On parle aussi d'\defi{arc paramétré}.
On peut aussi la noter
$\begin{array}[t]{cccc}
f :&D\subset\Rr&\rightarrow&\Rr^2\\
 &t&\mapsto&M(t)
\end{array}$ ou écrire en abrégé
$t\mapsto M(t)$ ou $t\mapsto\left(
\begin{smallmatrix}
x(t)\\
y(t)
\end{smallmatrix}
\right)$.
Enfin en identifiant $\Cc$ avec $\Rr^2$, on note aussi
$t\mapsto z(t)=x(t)+\ii y(t)$ avec l'identification
usuelle entre le point $M(t)=\left(\begin{smallmatrix}
x(t)\\
y(t)
\end{smallmatrix}
\right)$ et son affixe $z(t)=x(t)+\ii y(t)$.

\myfigure{0.8}{
\tikzinput{fig_courbes_part1_00}
}

Par la suite, une courbe sera
fréquemment décrite de manière très synthétique sous une forme du type
$$\left\{
\begin{array}{l}
x(t) = 3\ln t\\
y(t) = 2t^2+1
\end{array}
\right., \quad t\in]0,+\infty[ \quad \text{ ou } \quad z(t) = e^{\ii t}, \quad t\in[0,2\pi].$$
Il faut comprendre que $x$ et $y$ désignent des fonctions de $D$ dans $\Rr$
ou que $z$ désigne une fonction de $D$ dans $\Cc$.
Nous connaissons déjà des exemples de paramétrisations.
\begin{exemple}
\sauteligne
\begin{itemize}
\item $t\mapsto (\cos t,\sin t)$, $t\in[0,2\pi[$ : une paramétrisation du cercle trigonométrique.

\item $t\mapsto(2t-3,3t+1)$, $t\in\Rr$ : une paramétrisation de la droite
passant par le point $A(-3,1)$ et de vecteur directeur $\vec{u}(2,3)$.

\item $\lambda\mapsto\big((1-\lambda)x_A+\lambda x_B,(1-\lambda)y_A+\lambda y_B\big)$,
$\lambda\in[0,1]$ : une paramétrisation du segment $[AB]$.

\item Si $f$ est une fonction d'un domaine $D$ de
$\Rr$ à valeurs dans $\Rr$, une paramétrisation du graphe de $f$,
c'est-à-dire de la courbe d'équation $y=f(x)$, est $\left\{
\begin{array}{l}
x(t)=t\\
y(t)=f(t)
\end{array}
\right.$.
\end{itemize}


\myfigure{0.65}{
\tikzinput{fig_courbes_part1_01a} \qquad
\tikzinput{fig_courbes_part1_01b}
}
\myfigure{0.65}{
\tikzinput{fig_courbes_part1_01c} \qquad
\tikzinput{fig_courbes_part1_01d} 
}
\end{exemple}

Il est important de comprendre qu'une courbe paramétrée
ne se réduit pas au dessin, malgré le vocabulaire utilisé,
mais c'est bel et bien \emph{une application}.
Le graphe de la courbe porte le nom suivant :
\begin{definition}
Le \defi{support d'une courbe paramétrée}\index{courbe!support} $\begin{array}[t]{cccc}
f :&D\subset\Rr&\rightarrow&\Rr^2\\
 &t&\mapsto&f(t)
\end{array}$ est l'ensemble des points $M(t)$ où $t$ décrit $D$.
\end{definition}

Néanmoins par la suite, quand cela ne pose pas de problème, nous identifierons
ces deux notions en employant le mot \emph{courbe}
pour désigner indifféremment à la fois l'application et son graphe.
Des courbes paramétrées différentes peuvent avoir un même support. C'est par exemple
le cas des courbes :
$$\begin{array}[t]{ccc}
[0,2\pi[&\rightarrow&\Rr^2\\
t&\mapsto&(\cos t,\sin t)
\end{array} \qquad \text{ et } \qquad \begin{array}[t]{ccc}
[0,4\pi[&\rightarrow&\Rr^2\\
t&\mapsto&(\cos t,\sin t)
\end{array}$$
dont le support est un cercle, parcouru une seule fois pour la première paramétrisation
et deux fois pour l'autre (figure de gauche).
\myfigure{0.65}{
\tikzinput{fig_courbes_part1_02} \qquad
\tikzinput{fig_courbes_part1_03}
}


Plus surprenant, la courbe
$$t\mapsto \left(\frac{1-t^2}{1+t^2},\frac{2t}{1+t^2}\right), \qquad t\in\Rr,$$
est une paramétrisation du cercle privé du point $(-1,0)$,
avec des coordonnées qui sont des fractions rationnelles (figure de droite).


Ainsi, la seule donnée du support ne suffit pas à définir un arc
paramétré, qui est donc plus qu'un simple dessin.
C'est une \emph{courbe munie d'un mode de parcours}. Sur cette courbe, on avance
mais on peut revenir en arrière, on peut la parcourir une ou plusieurs fois, au gré du
paramètre, celui-ci n'étant d'ailleurs jamais visible sur le dessin.
On \og voit \fg{} $x(t)$, $y(t)$, mais pas $t$.

\bigskip

\evidence{Interprétation cinématique.} La cinématique est l'étude des mouvements.
Le paramètre $t$ s'interprète comme le \emph{temps}. On affine alors
le vocabulaire : la courbe paramétrée s'appelle plutôt \emph{point en mouvement}
et le support de cette courbe porte le nom de \emph{trajectoire}. Dans ce cas,
on peut dire que $M(t)$ est la \emph{position} du point $M$ à \emph{l'instant} $t$.



%---------------------------------------------------------------
\subsection{Réduction du domaine d'étude}


Rappelons tout d'abord l'effet de quelques transformations
géométriques usuelles sur le point $M(x,y)$ ($x$ et $y$ désignant
les coordonnées de $M$ dans un repère orthonormé $(O,\vec{i},\vec{j})$ donné).

\begin{itemize}
  \item Translation de vecteur $\vec{u}(a,b)$ : $t_{\vec{u}}(M)=(x+a,y+b)$.
  \item Réflexion d'axe $(Ox)$ : $s_{(Ox)}(M)=(x,-y)$.
  \item Réflexion d'axe $(Oy)$ : $s_{(Oy)}(M)=(-x,y)$.
  \item Symétrie centrale de centre $O$ : $s_O(M)=(-x,-y)$.
  \item Symétrie centrale de centre $I(a,b)$ : $s_I(M)=(2a-x,2b-y)$.
  \item Réflexion d'axe la droite $(D)$ d'équation $y=x$ : $s_D(M)=(y,x)$.
  \item Réflexion d'axe la droite $(D')$ d'équation $y=-x$ : $s_{D'}(M)=(-y,-x)$.
  \item Rotation d'angle $\frac{\pi}{2}$ autour de $O$ : $\text{rot}_{O,\pi/2}(M)=(-y,x)$.
  \item Rotation d'angle $-\frac{\pi}{2}$ autour de $O$ : $\text{rot}_{O,-\pi/2}(M)=(y,-x)$.
\end{itemize}

Voici la représentation graphique de quelques-unes de ces transformations.

\myfigure{0.5}{
\tikzinput{fig_courbes_part1_04a} \qquad
\tikzinput{fig_courbes_part1_04b}
} 
\myfigure{0.5}{
\tikzinput{fig_courbes_part1_04c} \qquad 
\tikzinput{fig_courbes_part1_04d} 
}

On utilise ces transformations pour réduire le domaine
d'étude d'une courbe paramétrée. Nous le ferons à travers quatre exercices.

\begin{exemple}
Déterminer un domaine d'étude le plus simple possible de la courbe 
$$\left\{
\begin{array}{l}
x(t)=t-\frac32\sin t\\
y(t)=1-\frac32\cos t
\end{array}
\right.$$

\medskip
\textbf{Solution.}

Pour $t\in\Rr$,
\begin{eqnarray*}
M(t+2\pi)
  &=& \big(t+2\pi-\tfrac32\sin(t+2\pi),1-\tfrac32\cos(t+2\pi)\big)\\
  &=& (t-\tfrac32\sin t,1-\tfrac32\cos t)+(2\pi,0)=t_{\vec{u}} \big(M(t)\big)  
\end{eqnarray*}
où $\vec{u}=(2\pi,0)$. Donc, on étudie l'arc
et on en trace le support sur un intervalle de longueur $2\pi$ au choix, comme
$[-\pi,\pi]$ par exemple, puis on obtient la
courbe complète par translations de vecteurs $k \cdot (2\pi,0)=(2k\pi,0)$, $k\in\Zz$.

Pour $t\in[-\pi,\pi]$,
$$M(-t)=\big(-(t-\tfrac32\sin t),1-\tfrac32\cos t \big)=s_{(Oy)}\big(M(t)\big).$$
On étudie la courbe et on en trace le support sur
$[0,\pi]$ (première figure), ensuite on effectue la
réflexion d'axe $(Oy)$ (deuxième figure), puis on obtient la courbe complète
par translations de vecteurs $k\vec{u}$, $k\in\Zz$ (troisième figure).


\myfigure{0.8}{
\tikzinput{fig_courbes_part1_05a} \qquad\qquad
\tikzinput{fig_courbes_part1_05b}
}
\myfigure{0.8}{
\tikzinput{fig_courbes_part1_05}
}
\end{exemple}



\begin{exemple}
Déterminer un domaine d'étude le plus simple possible d'une \defi{courbe de Lissajous}
$\left\{
\begin{array}{l}
x(t)=\sin(2t)\\
y(t)=\sin(3t)
\end{array}
\right.$

\medskip
\textbf{Solution.}

\begin{itemize}
  \item Pour $t\in\Rr$, $M(t+2\pi)=M(t)$ et on obtient la courbe
complète quand $t$ décrit $[-\pi,\pi]$.

  \item Pour $t\in[-\pi,\pi]$,
$M(-t)= \big(-\sin(2t),-\sin(3t)\big)=s_O\big(M(t)\big)$. On étudie et on
construit la courbe pour $t\in[0,\pi]$, puis on obtient
la courbe complète par
symétrie centrale de centre $O$.

  \item Pour $t\in[0,\pi]$,
$M(\pi-t)=\big(\sin(2\pi-2t),\sin(3\pi-3t)\big)=\big(\sin(-2t),\sin(\pi-3t)\big)
=\big(-\sin(2t),\sin(3t)\big)=s_{(Oy)}\big(M(t)\big)$.
On étudie et on construit la courbe pour
$t\in[0,\frac{\pi}{2}]$ (première figure),
on effectue la réflexion d'axe $(Oy)$ (deuxième figure),
puis on obtient la courbe complète par symétrie centrale de centre $O$ (troisième figure).
\end{itemize}

\myfigure{0.85}{
\tikzinput{fig_courbes_part1_06a} \quad
\tikzinput{fig_courbes_part1_06b}\quad
\tikzinput{fig_courbes_part1_06}
}

\end{exemple}




\begin{exemple}
Déterminer un domaine d'étude le plus simple possible de l'arc
$\left\{
\begin{array}{l}
x(t)=\dfrac{t}{1+t^4}\\
y(t)=\dfrac{t^3}{1+t^4}
\end{array}
\right.$

Indication : on pourra, entre autres, considérer la transformation $t\mapsto 1/t$.

\medskip
\textbf{Solution.}

Pour tout réel $t$, $M(t)$ est bien défini.

\begin{itemize}
  \item Pour $t\in\Rr$, $M(-t)=s_O\big(M(t)\big)$.
On étudie et on construit l'arc quand $t$ décrit $[0,+\infty[$, puis on obtient la courbe
complète par symétrie centrale de centre $O$.

  \item Pour $t\in]0,+\infty[$,
\begin{eqnarray*}
M\left(\frac{1}{t}\right)
  &=& \left(\frac{1/t}{1+1/t^4},\frac{1/t^3}{1+1/t^4}\right)=
\left(\frac{t^3}{1+t^4},\frac{t}{1+t^4}\right) \\
  &=& \big(y(t),x(t)\big)=s_{(y=x)}\big(M(t)\big).
\end{eqnarray*}
Autrement dit, $M(t_2) = s_{(y=x)}\big(M(t_1)\big)$ avec $t_2=1/t_1$,
et si $t_1\in]0,1]$ alors $t_2\in[1,+\infty[$.
Puisque la
fonction $t\mapsto\frac{1}{t}$ réalise une
bijection de $[1,+\infty[$ sur $]0,1]$, alors on étudie et on construit
la courbe quand $t$ décrit $]0,1]$ (première figure),
puis on effectue la réflexion d'axe
la première bissectrice (deuxième figure)
puis on obtient la courbe complète par symétrie centrale de centre $O$ et
enfin en plaçant le point $M(0)=(0,0)$ (troisième figure).
\end{itemize}

\myfigure{1}{
\tikzinput{fig_courbes_part1_07a}\quad
\tikzinput{fig_courbes_part1_07b}\quad
\tikzinput{fig_courbes_part1_07}
}

\end{exemple}


\begin{exemple}
Déterminer un domaine d'étude le plus simple possible
de l'arc $z=\frac{1}{3}\big(2e^{\ii t}+e^{-2\ii t}\big)$. En calculant $z(t+\frac{2\pi}{3})$,
trouver une transformation géométrique
simple laissant la courbe globalement invariante.

\medskip
\textbf{Solution.}

\begin{itemize}
  \item Pour $t\in\Rr$,
$z(t+2\pi)=\frac{1}{3}\left(2e^{\ii(t+2\pi)}+e^{-2\ii(t+2\pi)}\right)
=\frac{1}{3}\left(2e^{\ii t}+e^{-2\ii t}\right)=z(t)$.
La courbe complète est obtenue quand $t$ décrit $[-\pi,\pi]$.

  \item Pour $t\in[-\pi,\pi]$,
$z(-t)=\frac{1}{3}\left(2e^{-\ii t}+e^{2\ii t}\right)
=\overline{\frac{1}{3}\left(2e^{\ii t}+e^{-2\ii t}\right)}=\overline{z(t)}$.
Donc, on étudie et on construit la courbe
quand $t$ décrit $[0,\pi]$, la courbe complète étant alors
obtenue par réflexion d'axe $(Ox)$ (qui correspond à la conjugaison).

  \item Pour $t\in\Rr$,
\begin{eqnarray*}
z(t+\frac{2\pi}{3})
&=&\frac{1}{3}\left(2e^{\ii(t+2\pi/3)}+e^{-2\ii(t+2\pi/3)}\right) \\
&=&\frac{1}{3}\left(2e^{2\ii \pi/3}e^{\ii t}+e^{-4\ii \pi/3}e^{-2\ii t}\right)=e^{2\ii \pi/3}z(t).
\end{eqnarray*}
Le point
$M(t+2\pi/3)$ est donc l'image du point $M(t)$ par la rotation
de centre $O$ et d'angle $\frac{2\pi}{3}$. La courbe complète est
ainsi invariante par la rotation de centre $O$
et d'angle $\frac{2\pi}{3}$.
\end{itemize}

\myfigure{1.1}{
\tikzinput{fig_courbes_part1_08}
}


\end{exemple}


%---------------------------------------------------------------
\subsection{Points simples, points multiples}


\begin{definition}
Soit $f : t\mapsto M(t)$ une courbe paramétrée et soit $A$ un point du plan.
La \defi{multiplicité}\index{courbe!multiplicte@multiplicité} du point $A$ par rapport à la courbe $f$ est le nombre
de réels $t$ pour lesquels $M(t)=A$.
\end{definition}

En termes plus savants : la multiplicité du point $A$ par rapport à l'arc $f$
est $\text{Card}\big(f^{-1}(A)\big)$.

\myfigure{0.7}{
\tikzinput{fig_courbes_part1_09a}\qquad\qquad
\tikzinput{fig_courbes_part1_09b}\qquad\qquad
\tikzinput{fig_courbes_part1_09c}
}


\begin{itemize}
\item Si $A$ est atteint une et une seule fois, sa multiplicité est $1$
et on dit que le point $A$ est un \defi{point simple} de la courbe (première figure).

\item Si $A$ est atteint pour deux valeurs distinctes du paramètre et
deux seulement, on dit que $A$ est un \defi{point double}\index{point!double} de la courbe (deuxième figure).

\item On parle de même de \defi{points triples} (troisième figure), \defi{quadruples},
\ldots, \defi{multiples} (dès que le point est atteint au moins deux fois).

%\item Si $A$ n'est pas sur le support de la courbe, par convention sa multiplicité est $0$.

\item Une courbe dont tous les points sont simples est une
\defi{courbe paramétrée simple}\index{courbe!simple}. Il revient au même de dire que l'application
$t\mapsto M(t)$ est injective.
\end{itemize}

\bigskip

Comment trouve-t-on les points multiples ?
\mybox{
\begin{minipage}{0.7\textwidth}
\begin{center}
Pour trouver les points multiples d'une courbe, \\
on cherche les couples $(t,u)\in D^2$ tels que $t>u$ et \\
$M(t)=M(u)$.
\end{center}
\end{minipage}
}

On se limite au couple $(t,u)$ avec $t>u$ afin de ne
pas compter la solution redondante $(u,t)$ en plus de $(t,u)$.


\begin{exemple}
Trouver les points multiples de l'arc
$\left\{
\begin{array}{l}
x(t)=2t+t^2\\
y(t)=2t-\frac{1}{t^2}
\end{array}
\right.$, $t\in\Rr^*$.

\myfigure{0.6}{
\tikzinput{fig_courbes_part1_10}
}

\medskip
\textbf{Solution.}

Soit $(t,u)\in(\Rr^*)^2$ tel que $t>u$.
\begin{align*}
M(t)=M(u)
&\iff\left\{
\begin{array}{l}
2t+t^2=2u+u^2\\
2t-\frac{1}{t^2}=2u-\frac{1}{u^2}
\end{array}
\right.\iff\left\{
\begin{array}{l}
(t^2-u^2)+2(t-u)=0\\
2(t-u)-\big(\frac{1}{t^2}-\frac{1}{u^2}\big)=0
\end{array}
\right.\\
&\iff\left\{
\begin{array}{l}
(t-u)(t+u+2)=0\\
(t-u)\big(2+\frac{t+u}{t^2u^2}\big)=0
\end{array}
\right.\\
&\iff\left\{
\begin{array}{l}
t+u+2=0\\
2+\frac{t+u}{t^2u^2}=0
\end{array}
\right.\;(\text{car}\;t-u\neq0) \\
&\iff\left\{
\begin{array}{l}
S+2=0\\
2+\frac{S}{P^2}=0
\end{array}
\right.\;(\text{en posant}\;S=t+u\;\text{et}\;P=tu)
\\
 &\iff\left\{
\begin{array}{l}
S=-2\\
P^2=1
\end{array}
\right.\iff\left\{
\begin{array}{l}
S=-2\\
P=1
\end{array}
\right.\;\text{ou}\;\left\{
\begin{array}{l}
S=-2\\
P=-1
\end{array}
\right.
\\
 &\iff t\ \text{ et }\ u\ \text{ sont les deux solutions de }
 \left\{\begin{array}{ll}
  & X^2+2X+1=0 \\
  \text{ ou } & X^2+2X-1=0 \\
\end{array}\right.
\\
 &\iff t=-1+\sqrt{2}\quad\text{et}\quad u=-1-\sqrt{2}\quad(\text{car}\;t>u).
\end{align*}




Il nous reste à déterminer où est ce point double $M(t)=M(u)$.
Fixons $t=-1+\sqrt{2}$ et $u=-1-\sqrt{2}$.
De plus, $x(t)=t^2+2t=1$ (puisque pour cette valeur de $t$, $t^2+2t-1=0$). Ensuite, en
divisant les deux membres de l'égalité $t^2+2t=1$
par $t^2$, nous déduisons  $\frac{1}{t^2}=1+\frac{2}{t}$,
puis, en divisant les deux membres de l'égalité $t^2+2t=1$ par $t$,
nous déduisons $\frac{1}{t}=t+2$. Par suite, $y(t)=2t-(1+2(t+2))=-5$.
La courbe admet un point double, le point de coordonnées $(1,-5)$.
\end{exemple}

\begin{remarque*}
Dans cet exercice, les expressions utilisées sont des fractions
rationnelles, ou encore, une fois réduites au même dénominateur,
puis une fois les dénominateurs éliminés, les expressions sont polynomiales.
Or, à $u$ donné, l'équation $M(t)=M(u)$, d'inconnue $t$, admet bien sûr la
solution $t=u$. En conséquence, on doit systématiquement pouvoir mettre en
facteur $(t-u)$, ce que nous avons fait en regroupant les termes analogues :
nous avons écrit tout de suite $(t^2-u^2)+2(t-u)=0$ et non pas $t^2+2t-u^2-2u=0$.
Le facteur $t-u$ se simplifie alors car il est non nul.
\end{remarque*}


\begin{miniexercices}
\sauteligne
\begin{enumerate}
  \item Représenter graphiquement chacune des transformations du plan
  qui servent à réduire l'intervalle d'étude.

  \item Pour la courbe de Lissajous définie par $x(t)=\sin(2t)$ et
  $y(t)=\sin(3t)$, montrer que la courbe est symétrique par rapport à
  l'axe $(Ox)$. Exprimer cette symétrie en fonction de celles déjà trouvées :
  $s_O$ et $s_{(Oy)}$.



  \item Trouver les symétries et les points multiples de la courbe définie par
  $x(t) = \frac{1-t^2}{1+t^2}$ et $y(t) = t \frac{1-t^2}{1+t^2}$.

  \item Trouver un intervalle d'étude pour l'astroïde définie par $x(t) = \cos^3 t$,
  $y(t) = \sin^3 t$.

  \item Trouver un intervalle d'étude pour la cycloïde définie par $x(t) = r(t-\sin t)$,
  $y(t) = r(1-\cos t)$. Montrer que la cycloïde n'a pas de points multiples.

\end{enumerate}
\end{miniexercices}




%%%%%%%%%%%%%%%%%%%%%%%%%%%%%%%%%%%%%%%%%%%%%%%%%%%%%%%%%%%%%%%%
\section{Tangente à une courbe paramétrée}


%---------------------------------------------------------------
\subsection{Tangente à une courbe}

Soit $f : t\mapsto M(t)$, $t\in D\subset\Rr$, une courbe.
Soit $t_0\in D$. On veut définir la tangente en $M(t_0)$.


\myfigure{0.7}{
\tikzinput{fig_courbes_part2_01}\qquad
\tikzinput{fig_courbes_part2_02}
}

On doit déjà prendre garde au fait que lorsque ce point $M(t_0)$ est un point multiple de la courbe,
alors la courbe peut tout à fait avoir plusieurs tangentes en ce point (figure de droite).
Pour éviter cela, on supposera que la courbe est
\defi{localement simple en $t_0$}, c'est-à-dire qu'il existe un intervalle
ouvert non vide $I$ de centre $t_0$ tel que l'équation $M(t)=M(t_0)$
admette une et une seule solution dans $D\cap I$, à savoir $t=t_0$ (figure de gauche). Il
revient au même de dire que l'application $t\mapsto M(t)$ est
\emph{localement injective}. Dans tout ce paragraphe, nous supposerons
systématiquement que cette condition est réalisée.


\bigskip

Soit $f : t\mapsto M(t)$, $t\in D\subset\Rr$, une courbe paramétrée et
soit $t_0\in D$. On suppose que la courbe est localement simple en $t_0$.
\begin{definition}[Tangente]
On dit que la courbe admet une tangente en $M(t_0)$ si la droite $(M(t_0)M(t))$
admet une position limite quand $t$ tend vers $t_0$. Dans ce cas,
la droite limite est la \defi{tangente}\index{tangente}\index{courbe!tangente} en $M(t_0)$.
\end{definition}

\myfigure{1.2}{
\tikzinput{fig_courbes_part2_03}
}

%---------------------------------------------------------------
\subsection{Vecteur dérivé}

On sait déjà que la tangente en $M(t_0)$, quand elle existe, passe
par le point $M(t_0)$. Mais il nous manque sa direction. Pour
$t\neq t_0$, un vecteur directeur de la droite $(M(t_0)M(t))$ est
le vecteur $\overrightarrow{M(t_0)M(t)}
=\left(\begin{smallmatrix}
x(t)-x(t_0)\\
y(t)-y(t_0)
\end{smallmatrix}\right)$
(rappelons que ce vecteur est supposé non nul pour $t$ proche
de $t_0$ et distinct de $t_0$). Quand $t$ tend vers $t_0$, les coordonnées
de ce vecteur tendent vers $0$ ; autrement dit le vecteur $\overrightarrow{M(t_0)M(t)}$
tend (malheureusement) vers $\overrightarrow{0}$. Le vecteur nul n'indique
aucune direction particulière et nous ne connaissons toujours pas la
direction limite de la droite $(M(t_0)M(t))$. Profitons-en néanmoins
pour définir la notion de limite et de continuité d'une fonction à valeurs dans $\Rr^2$.

\begin{definition}
Soit $t\mapsto M(t)=\big(x(t),y(t)\big)$, $t\in D\subset\Rr$, une courbe paramétrée
et soit $t_0\in D$.
La courbe est \defi{continue en $t_0$} si et seulement si les fonctions $x$ et
$y$ sont continues en $t_0$.
La courbe est \defi{continue sur $D$} si et seulement si elle est continue en tout point de $D$.
\end{definition}


En d'autres termes la courbe est continue en $t_0$ si et seulement si
$x(t) \to x(t_0)$ \evidence{et} $y(t) \to y(t_0)$, lorsque $t\to t_0$.



Revenons maintenant à notre tangente. Un autre vecteur directeur
de la droite $(M(t_0)M(t))$ est le vecteur
$$\frac{1}{t-t_0}\overrightarrow{M(t_0)M(t)}=\left(\begin{array}{c}
\frac{x(t)-x(t_0)}{t-t_0}\\
\frac{y(t)-y(t_0)}{t-t_0}
\end{array}
\right).$$

On a multiplié le vecteur $\overrightarrow{M(t_0)M(t)}$
par le réel $\frac{1}{t-t_0}$.
Remarquons que chaque coordonnée de ce vecteur est un taux d'accroissement,
dont on cherche la limite.
D'où la définition :

\begin{definition}
Soient $t\mapsto M(t)=(x(t),y(t))$, $t\in D\subset\Rr$, une courbe paramétrée
et $t_0\in D$. La courbe est \defi{dérivable en $t_0$} si et seulement si les fonctions
$x$ et $y$ le sont. Dans ce cas, le \defi{vecteur dérivé}\index{vecteur derive@vecteur dérivé} de la courbe en $t_0$ est le vecteur
$\left(
\begin{matrix}
x'(t_0)\\
y'(t_0)
\end{matrix}
\right)$.
Ce vecteur se note $\overrightarrow{\dfrac{\dd M}{\dd t}}(t_0)$.
\end{definition}



Cette notation se justifie car dans le vecteur
$\frac{1}{t-t_0}\overrightarrow{M(t_0)M(t)}$,
dont on cherche la limite, $\overrightarrow{M(t_0)M(t)}$
peut s'écrire $M(t) - M(t_0)$ (on rappelle
qu'une différence de deux points $B-A$ est un vecteur $\overrightarrow{AB}$).
Ainsi :
$$\overrightarrow{\dfrac{\dd M}{\dd t}}(t_0)
=\text{\og}
\overrightarrow{\dfrac{\text{\small{différence infinitésimale de}}\;M}
{\text{\small{différence infinitésimale de}}\;t}}\;\text{en}\;t_0\text{\fg}$$

\myfigure{1.3}{
\tikzinput{fig_courbes_part2_04}
}

%---------------------------------------------------------------
\subsection{Tangente en un point régulier}


Si le vecteur dérivé $\overrightarrow{\frac{\dd M}{\dd t}}(t_0)$ n'est pas nul,
celui-ci indique effectivement la direction limite de la droite
$(M(t_0)M(t))$. Nous étudierons plus tard le cas où le
vecteur dérivé est nul.

\begin{definition}
Soit $t\mapsto M(t)$, $t\in D\subset\Rr$, une courbe dérivable sur
$D$ et soit $t_0$ un réel de $D$.
\begin{itemize}
  \item Si $\overrightarrow{\frac{\dd M}{\dd t}}(t_0)\neq\vec{0}$,
le point $M(t_0)$ est dit \defi{régulier}.

  \item Si $\overrightarrow{\frac{\dd M}{\dd t}}(t_0)=\vec{0}$,
  le point $M(t_0)$ est dit \defi{singulier}.

  \item Une courbe dont tous les points sont réguliers est
appelée \defi{courbe régulière}\index{courbe!reguliere@régulière}.
\end{itemize}
\end{definition}



\textbf{Interprétation cinématique.} Si $t$ est le temps,
le vecteur dérivé $\overrightarrow{\frac{\dd M}{\dd t}}(t_0)$ est
le \emph{vecteur vitesse} au point $M(t_0)$.
Un point singulier, c'est-à-dire un point en lequel
la vitesse est nulle, s’appellera alors plus volontiers
\emph{point stationnaire}.
D'un point de vue cinématique, il est logique
que le vecteur vitesse en un point, quand il est non nul,
dirige la tangente à la trajectoire en ce point.
C'est ce qu'exprime le théorème suivant, qui découle directement
de notre étude du vecteur dérivé :

\begin{theoreme}
En tout point régulier d'une courbe dérivable, cette courbe admet une tangente.
La tangente en un point régulier est dirigée par le vecteur dérivé en ce point.
\end{theoreme}

\myfigure{1}{
\tikzinput{fig_courbes_part2_05}
}


Si $\overrightarrow{\frac{\dd M}{\dd t}}(t_0)\neq\vec{0}$,
une équation de la tangente $T_0$ en $M(t_0)$ est donc fournie par :
\mybox{
$M(x,y)\in T_0 \iff \left|
\begin{array}{cc}
x-x(t_0)&x'(t_0)\\
y-y(t_0)&y'(t_0)
\end{array}
\right|=0 \iff y'(t_0)\big(x-x(t_0)\big)-x'(t_0)\big(y-y(t_0)\big)=0.$
}

\begin{exemple}
Trouver les points où la tangente à la courbe de Lissajous
$\left\{
\begin{array}{l}
x(t)=\sin(2t)\\
y(t)=\sin(3t)
\end{array}
\right.$, $t\in [-\pi,\pi]$,
est verticale, puis horizontale.


\medskip
\textbf{Solution.}

Tout d'abord, par symétries, on limite notre étude sur $t\in[0,\frac\pi2]$.
Or au point $M(t) = \left(\begin{smallmatrix} \sin(2t) \\ \sin(3t)\end{smallmatrix}\right)$,
le vecteur dérivé est
$$\overrightarrow{\frac{\dd M}{\dd t}}
= \left(\begin{matrix} x'(t) \\ y'(t) \end{matrix}\right)
= \left(\begin{matrix} 2\cos(2t) \\ 3\cos(3t)\end{matrix}\right).$$

Quand est-ce que la première coordonnée de ce vecteur dérivé est nul
(sur $t\in[0,\frac\pi2]$) ?
$$x'(t) = 0 \iff 2\cos(2t) = 0 \iff t = \frac\pi4$$
Et pour la seconde :
$$y'(t) = 0 \iff 3\cos(3t) = 0 \iff t = \frac\pi6  \quad\text{ ou }\quad t=\frac\pi2$$

Les deux coordonnées ne s'annulent jamais en même temps, donc
le vecteur dérivé n'est jamais nul, ce qui prouve que tous les points sont réguliers,
et le vecteur dérivé dirige la tangente.

La tangente est verticale lorsque le vecteur dérivé est vertical,
ce qui équivaut à $x'(t) = 0$, autrement dit en $M(\frac\pi4)$.
La tangente est horizontale lorsque le vecteur dérivé est horizontal,
ce qui équivaut à $y'(t) = 0$, autrement dit en $M(\frac\pi6)$
et en $M(\frac\pi2)$.

\myfigure{1}{
\tikzinput{fig_courbes_part2_06}
}

On trouve les autres tangentes horizontales et verticales par symétrie.
\end{exemple}


\begin{remarque*}
\sauteligne
\begin{itemize}
\item Une courbe peut avoir une tangente verticale, contrairement à ce
à quoi on est habitué pour les graphes de fonctions du type $y=f(x)$.

\item Par contre dans le cas d'une paramétrisation cartésienne du type $\left\{
\begin{array}{l}
x(t)=t\\
y(t)=f(t)
\end{array}
\right.$ qui est une paramétrisation du graphe de la fonction (dérivable)
$f$ (où cette fois-ci $f$ est à valeurs dans $\Rr$),
le vecteur dérivé en $t_0=x_0$ est $\left(
\begin{smallmatrix}
1\\
f'(x_0)
\end{smallmatrix}
\right)$. Celui-ci n'est jamais nul puisque sa première coordonnée
est non nulle. Ainsi, une paramétrisation cartésienne dérivable
est toujours régulière. De plus, pour la même raison, ce vecteur n'est jamais vertical.
\end{itemize}

\end{remarque*}


%---------------------------------------------------------------
\subsection{Dérivation d'expressions usuelles}

On généralise un peu l'étude précédente.
Voici comment dériver le produit scalaire de deux fonctions vectorielles ainsi
que la norme.

\begin{theoreme}
Soient $f$ et $g$ deux applications définies sur un domaine $D$ de $\Rr$
à valeurs dans $\Rr^2$ et soit $t_0\in D$. On suppose que $f$ et $g$ sont dérivables en $t_0$. Alors :
\begin{enumerate}
\item L'application
$t\mapsto \big\langle f(t) \mid {g}(t) \big\rangle$
est dérivable en $t_0$ et
$$\frac{\dd \big\langle {f} \mid {g} \big\rangle}{\dd t}(t_0)
=\big\langle {\frac{\dd {f}}{\dd t}}(t_0) \mid {g}(t_0)\big\rangle+
\big\langle {f}(t_0) \mid {\frac{\dd {g}}{\dd t}}(t_0)\big\rangle.$$
\item Si ${f}(t_0)\neq\vec{0}$, l'application
$t\mapsto\|{f}(t)\|$ est dérivable en $t_0$ et, dans ce cas,
$$\frac{\dd\|{f}\|}{\dd t}(t_0)
=\frac{\big\langle {f}(t_0) \mid
{\frac{\dd {f}}{\dd t}}(t_0)\big\rangle}{\|{f}(t_0)\|}.$$
\end{enumerate}
\end{theoreme}

\begin{proof}
Le produit scalaire et la norme sont des fonctions de $D$ dans $\Rr$.
\begin{enumerate}
\item Posons ${f}=(x_1,y_1)$ et ${g}=(x_2,y_2)$.
Alors $\langle{f}\mid{g}\rangle=x_1x_2+y_1y_2$ est dérivable en $t_0$ et
$$\big\langle{f}\mid{g}\big\rangle'(t_0)
=(x_1'x_2+x_1x_2'+y_1'y_2+y_1y_2')(t_0)
=\big\langle{f}'\mid{g}\big\rangle(t_0)+
\big\langle{f}\mid{g}'\big\rangle(t_0).$$

\item La fonction $\big\langle{f}\mid{f}\big\rangle$ est positive,
strictement positive en $t_0$ et est dérivable en $t_0$. D'après
le théorème de dérivation des fonctions composées, la fonction
$\|{f}\|=\sqrt{\big\langle{f}\mid{f}\big\rangle}$
est dérivable en $t_0$ et
$$\big(\|{f}\|\big)'(t_0)
=\frac{1}{2\sqrt{\big\langle{f}\mid{f}\big\rangle}}
\left(\big\langle{f}'\mid{f}\big\rangle
+\big\langle{f}\mid{f}'\big\rangle\right)(t_0)
=\frac{\big\langle{f}\mid{f}'\big\rangle}
{\|{f}\|}(t_0).$$


\end{enumerate}
\end{proof}

\begin{exemple}
Soit  $t\mapsto M(t)=(\cos t,\sin t)$ une
paramétrisation du cercle de centre $O$ et de rayon $1$. Pour tout
réel $t$, on a $OM(t)=1$ ou encore $\|\overrightarrow{OM(t)}\|=1$.
En dérivant cette fonction constante, on obtient : $\forall t\in\Rr$,
$\big\langle \overrightarrow{OM(t)} \mid \overrightarrow{\frac{\dd M}{\dd t}}(t) \big\rangle =0$ et on
retrouve le fait que la tangente au cercle au point $M(t)$ est
orthogonale au rayon $\overrightarrow{OM(t)}$.

\myfigure{1}{
\tikzinput{fig_courbes_part2_07}
}
\end{exemple}

\begin{theoreme}
Soient ${f}$, ${g}$ deux applications définies sur un domaine $D$
de $\Rr$ à valeurs dans $\Rr^2$ et $\lambda$ une application de $D$
dans $\Rr$. Soit $t_0\in D$. On suppose que ${f}$, ${g}$ et $\lambda$ sont
dérivables en $t_0$. Alors, ${f}+{g}$ et $\lambda {f}$ sont dérivables en $t_0$, et
$${\frac{\dd({f}+{g})}{\dd t}}(t_0)
={\frac{\dd {f}}{\dd t}}(t_0)+{\frac{\dd {g}}{\dd t}}(t_0)$$
$$\text{ et } \qquad {\frac{\dd(\lambda \cdot {f})}{\dd t}}(t_0)
=\lambda'(t_0){f}(t_0)+\lambda(t_0){\frac{\dd {f}}{\dd t}}(t_0).$$
\end{theoreme}



\begin{proof}
Posons ${f}=(x_1,y_1)$ et ${g}=(x_2,y_2)$. Alors
$$({f}+{g})'(t_0)=(x_1+x_2,y_1+y_2)'(t_0)
=(x_1'+x_2',y_1'+y_2')(t_0)={f}'(t_0)+{g}'(t_0),$$
et aussi
\begin{multline*}
(\lambda{f})'(t_0)=(\lambda x_1,\lambda y_1)'(t_0)
=(\lambda'x_1+\lambda x_1',\lambda'y_1+\lambda y_1')(t_0)\\
=\lambda'(x_1,y_1)(t_0)+\lambda(x_1',y_1')(t_0)
=(\lambda'{f}+\lambda{f}')(t_0).
\end{multline*}
\end{proof}


De même, toujours en travaillant sur les coordonnées, on établit aisément que :
\begin{theoreme}
Soient $t\mapsto\theta(t)$ une application dérivable sur un domaine
$D$ de $\Rr$ à valeurs dans un domaine $D'$ de $\Rr$ et $u\mapsto {f}(u)$
une application dérivable sur $D'$ à valeurs dans $\Rr^2$. Alors
${f}\circ\theta$ est dérivable sur $D$ et, pour $t_0\in D$,
$${\frac{\dd({f}\circ\theta)}{\dd t}}(t_0)
=\theta'(t_0) \cdot {\dfrac{\dd {f}}{\dd t}}\big(\theta(t_0)\big).$$
\end{theoreme}



\begin{miniexercices}
\sauteligne
\begin{enumerate}
  \item Soit la courbe définie par $x(t)= t^5-4t^3$, $y(t)=t^2$.
  Calculer le vecteur dérivé en chaque point. Déterminer le point singulier.
  Calculer une équation de la tangente au point $(3,1)$.
  Calculer les équations de deux tangentes au point $(0,4)$.

  \item Soit $f$ une fonction dérivable de $D \subset \Rr$ dans $\Rr^2$.
  Calculer la dérivée de l'application $t \mapsto \| f(t) \|^2$.

  \item Calculer le vecteur dérivé en tout point de l'astroïde définie par $x(t) = \cos^3 t$,
  $y(t) = \sin^3 t$. Quels sont les points singuliers ? Trouver une expression simple pour
  la pente de tangente en un point régulier.

  \item Calculer le vecteur dérivé en tout point de la cycloïde définie par
  $x(t) = r(t-\sin t)$,  $y(t) = r(1-\cos t)$. Quels sont les points singuliers ?
  En quels points la tangente est-elle horizontale ? En quels points la tangente est-elle
  parallèle à la bissectrice d'équation $(y=x)$ ?

\end{enumerate}
\end{miniexercices}



%%%%%%%%%%%%%%%%%%%%%%%%%%%%%%%%%%%%%%%%%%%%%%%%%%%%%%%%%%%%%%%%
\section{Points singuliers -- Branches infinies}


%---------------------------------------------------------------
\subsection{Tangente en un point singulier}


Rappelons qu'un point $M(t_0)$ d'une courbe paramétrée $M(t) = \big( x(t),y(t)\big)$
est dit \defi{point singulier}\index{point!singulier} si le vecteur dérivé en ce point est nul,
c'est-à-dire si $\overrightarrow{\frac{\dd M}{\dd t}}(t_0)=\vec{0}$,
ou autrement dit si $\big( x'(t_0), y'(t_0)\big) = (0,0)$.
Lorsque le vecteur dérivé est nul, il n'est d'aucune utilité pour
la recherche d'une tangente.
Pour obtenir une éventuelle tangente en un point singulier,
le plus immédiat est de revenir à la définition en étudiant
la direction limite de la droite $(M(t_0)M(t))$, par exemple
en étudiant la limite du coefficient directeur de cette droite
dans le cas où cette droite n'est pas parallèle à $(Oy)$.
En supposant que c'est le cas :

\mybox
{
\begin{minipage}{0.7\textwidth}
\begin{center}
En un point $M(t_0)$ singulier, on étudie
$\displaystyle \lim_{t\rightarrow t_0}\frac{y(t)-y(t_0)}{x(t)-x(t_0)}$.\\
Si cette limite est un réel $\ell$, la tangente en $M(t_0)$
existe et a pour coefficient directeur $\ell$.\\
Si cette limite existe mais est infinie, la tangente en $M(t_0)$
existe et est verticale.
\end{center}
\end{minipage}
}

\begin{exemple}
Trouver les points singuliers de la courbe
$\left\{
\begin{array}{l}
x(t)=3t^2\\
y(t)=2t^3\\
\end{array}\right.$.
Donner une équation cartésienne de la tangente en tout point de la courbe.

\medskip
\textbf{Solution.}

\begin{itemize}
  \item \textbf{Calcul du vecteur dérivé.}
  Pour $t\in\Rr$,
  $\overrightarrow{\frac{\dd M}{\dd t}}(t)=\left(\begin{smallmatrix}
6t\\6t^2\end{smallmatrix}\right)$.
Ce vecteur est nul si et seulement si $6t=6t^2=0$ ou encore $t=0$. Tous
les points de la courbe sont réguliers, à l'exception de $M(0)$.

  \item \textbf{Tangente en $M(0)$.}
  Pour $t\neq 0$,
$\frac{y(t)-y(0)}{x(t)-x(0)}=\frac{2t^3}{3t^2}=\frac{2t}{3}$.
Quand $t$ tend vers $0$, cette expression tend vers $0$. L'arc
admet une tangente en $M(0)$ et cette tangente est la droite
passant par $M(0)=(0,0)$ et de pente $0$ : c'est l'axe $(Ox)$ (d'équation $y=0$).


  \item \textbf{Tangente en $M(t)$, $t\neq0$.}
  Pour $t\in\Rr^*$, la courbe admet en $M(t)$ une tangente dirigée
  par $\overrightarrow{\frac{\dd M}{\dd t}}(t)
=\left(\begin{smallmatrix}6t\\6t^2\end{smallmatrix}\right)$
ou aussi par le vecteur
$\frac{1}{6t}\left(\begin{smallmatrix}
6t\\6t^2\end{smallmatrix}\right)
=\left(\begin{smallmatrix}1\\t\end{smallmatrix}\right)$.
Une équation de la tangente en $M(t)$ est donc $t(x-3t^2)-(y-2t^3)=0$
ou encore $y=tx-t^3$ (ce qui reste valable en $t=0$).


\end{itemize}

\myfigure{1}{
\tikzinput{fig_courbes_part3_01}
}

\end{exemple}



%---------------------------------------------------------------
\subsection{Position d'une courbe par rapport à sa tangente}


Quand la courbe arrive en $M(t_0)$, le long de sa tangente, on a
plusieurs possibilités :
\begin{itemize}
  \item la courbe continue dans le même sens, sans traverser la tangente :
  c'est un \defi{point d'allure ordinaire}\index{point!d'allure ordinaire},

  \item la courbe continue dans le même sens, en traversant la tangente :
  c'est un \defi{point d'inflexion}\index{point!d'inflexion},

  \item la courbe rebrousse chemin le long de
cette tangente en la traversant, c'est un
\defi{point de rebroussement de première espèce}\index{point!de rebroussement},

  \item la courbe rebrousse chemin le long de
cette tangente sans la traverser, c'est un
\defi{point de rebroussement de seconde espèce}.

\end{itemize}

\myfigure{1}{
\begin{tabular}{cc}
\tikzinput{fig_courbes_part3_02}&
\tikzinput{fig_courbes_part3_03}\\[3mm]
\tikzinput{fig_courbes_part3_04}&
\tikzinput{fig_courbes_part3_05}\\
\end{tabular}
}

Intuitivement, on ne peut rencontrer des points de rebroussement qu'en
un point stationnaire, car en un point où la vitesse est non nulle,
on continue son chemin dans le même sens.

\bigskip

Pour déterminer de façon systématique la position de la courbe par rapport à sa tangente
en un point singulier $M(t_0)$,
on effectue un développement limité des coordonnées de $M(t) = \big(x(t),y(t)\big)$
au voisinage de $t=t_0$.
Pour simplifier l'expression on suppose $t_0=0$.
On écrit
$$M(t) = M(0) + t^p \vec{v} + t^q \vec{w} +t^q \vec{\epsilon}(t)$$
où :
\begin{itemize}
  \item $p<q$ sont des entiers,
  \item $\vec{v}$ et $\vec{w}$ sont des vecteurs non colinéaires,
  \item $\vec{\epsilon}(t)$ est un vecteur, tel que $\|\vec{\epsilon}(t)\| \to 0$
  lorsque $t\to t_0$.
\end{itemize}

En un tel point $M(0)$, la courbe $\mathcal{C}$ admet une tangente, dont un vecteur directeur est
$\vec{v}$. La position de la courbe $\mathcal{C}$ par rapport à cette tangente
est donnée par la parité de $p$ et $q$ :

\myfigure{1}{
\begin{tabular}{cc}
\tikzinput{fig_courbes_part3_06}&
\tikzinput{fig_courbes_part3_07}\\[3mm]
\tikzinput{fig_courbes_part3_08}&
\tikzinput{fig_courbes_part3_09}\\
\end{tabular}
}

Prenons par exemple $M(t) = t^2 \vec{v} + t^5 \vec{w}$.
Donc $p=2$ et $q=5$.
Lorsque $t$ passe d'une valeur négative à positive,
$t^2$ s'annule mais en restant positif,
donc la courbe arrive au point le long de la tangente (dirigée par $\vec{v}$)
et rebrousse chemin en sens inverse.
Par contre $t^5$ change de signe, donc la courbe franchit la tangente au point singulier.
C'est bien un point de rebroussement de première espèce.
\myfigure{1}{
\tikzinput{fig_courbes_part3_01bis}
}

\bigskip


Voyons un exemple de chaque situation.

\begin{exemple}
Étudier le point singulier à l'origine de
$\left\{\begin{array}{l} x(t) = t^5\\ y(t) = t^3\end{array}\right..$

\medskip
\textbf{Solution.}

En $M(0)=(0,0)$, il y a bien un point singulier.
On écrit
$$M(t) = t^3 \begin{pmatrix}0\\1\end{pmatrix} + t^5
\begin{pmatrix}1\\0\end{pmatrix}.$$
Ainsi $p=3$, $q=5$,
$\vec{v}=\left(\begin{smallmatrix}0\\1\end{smallmatrix}\right)$,
$\vec{w}=\left(\begin{smallmatrix}1\\0\end{smallmatrix}\right)$.
La tangente, dirigée par $\vec{v}$, est verticale à l'origine.
Comme $p=3$ est impair alors $t^3$ change de signe en $0$,
donc la courbe continue le long de la tangente, et comme $q=5$ est aussi
impair, la courbe franchit la tangente au point singulier.
C'est un point d'inflexion.


\myfigure{1}{
\tikzinput{fig_courbes_part3_10}
}

\end{exemple}


\begin{exemple}
Étudier le point singulier à l'origine de
$\left\{\begin{array}{l} x(t) = 2t^2\\ y(t) = t^2 - t^3\end{array}\right..$

\medskip
\textbf{Solution.}

En $M(0)=(0,0)$, il y a bien un point singulier.
On écrit
$$M(t) = t^2 \begin{pmatrix}2\\1\end{pmatrix} + t^3
\begin{pmatrix}0\\-1\end{pmatrix}.$$
Ainsi $p=2$, $q=3$,
$\vec{v}=\left(\begin{smallmatrix}2\\1\end{smallmatrix}\right)$,
$\vec{w}=\left(\begin{smallmatrix}0\\-1\end{smallmatrix}\right)$.
C'est un point de rebroussement de première espèce.

\myfigure{1}{
\tikzinput{fig_courbes_part3_11}
}

\end{exemple}



\begin{exemple}
Étudier le point singulier en $(1,0)$ de
$\left\{\begin{array}{l} x(t) = 1+t^2+\frac12t^3 \\ y(t) = t^2 + \frac12t^3 + 2t^4 \end{array}\right..$

\medskip
\textbf{Solution.}

On écrit
$$M(t) = \begin{pmatrix}1\\0\end{pmatrix}
+ t^2 \begin{pmatrix}1\\1\end{pmatrix}
+ t^3 \begin{pmatrix}\frac12\\\frac12\end{pmatrix}
+ t^4 \begin{pmatrix}0\\2\end{pmatrix}.$$
On a donc $p=2$ avec $\vec{v}=\left(\begin{smallmatrix}1\\1\end{smallmatrix}\right)$,
par contre le vecteur $\vec{v}'=\left(\begin{smallmatrix}\frac12\\\frac12\end{smallmatrix}\right)$
est colinéaire à $\vec{v}$, donc $q=4$ et
$\vec{w}=\left(\begin{smallmatrix}0\\2\end{smallmatrix}\right)$.
C'est un point de rebroussement de seconde espèce.

\myfigure{1}{
\tikzinput{fig_courbes_part3_12}
}

\end{exemple}



\begin{exemple}
Étudier le point singulier à l'origine de
$\left\{\begin{array}{l} x(t) = t^2\ln(1+t) \\ y(t) = t^2\left(\exp(t^2) -1\right) \end{array}\right..$

\medskip
\textbf{Solution.}

On écrit
$$x(t) = t^3 - \frac{t^4}{2} + t^4 \epsilon_1(t)
\qquad
y(t) = t^4 + t^4 \epsilon_2(t)$$
et ainsi
$$M(t) =  t^3 \begin{pmatrix}1\\0\end{pmatrix} + t^4
\begin{pmatrix}-1/2\\1\end{pmatrix}+t^4 \vec{\epsilon}(t).$$
On a donc $p=3$, $q=4$ et c'est un point d'allure ordinaire.

\myfigure{1.2}{
\tikzinput{fig_courbes_part3_13}
}

\end{exemple}

%---------------------------------------------------------------
\subsection{Branches infinies}

Dans ce paragraphe, la courbe $f : t\mapsto M(t)$ est définie sur un intervalle
$I$ de $\Rr$. On note aussi $\mathcal{C}$ la courbe
et $t_0$ désigne l'une des bornes de $I$ et n'est pas dans $I$
($t_0$ est soit un réel, soit $-\infty$, soit $+\infty$).


\begin{definition}
Il y a \defi{branche infinie}\index{branche infinie} en $t_0$ dès que l'une au moins des deux
fonctions $|x|$ ou $|y|$ tend vers l'infini quand $t$ tend vers $t_0$.
Il revient au même de dire que $\lim_{t \to t_0}\|f(t)\|=+\infty$.
\end{definition}

Pour chaque branche infinie, on cherche s'il existe une asymptote,
c'est-à-dire une droite qui approxime cette branche infinie.
La droite d'équation $y = ax+b$ est \defi{asymptote}\index{asymptote} à $\mathcal{C}$
si $y(t) - \big( a x(t) + b \big) \to 0$, lorsque $t \to t_0$.


Dans la pratique, on mène l'étude suivante :
\begin{enumerate}
\item Si, quand $t$ tend vers $t_0$, $x(t)$ tend vers $+\infty$ (ou
$-\infty$) et $y(t)$ tend vers un réel $\ell$, la droite d'équation
$y=\ell$ est \defi{asymptote horizontale} à $\mathcal{C}$.

\item Si, quand $t$ tend vers $t_0$, $y(t)$ tend vers $+\infty$ (ou
$-\infty$) et $x(t)$ tend vers un réel $\ell$, la droite d'équation
$x=\ell$ est \defi{asymptote verticale} à $\mathcal{C}$.
\end{enumerate}



Si, quand $t$ tend vers $t_0$, $x(t)$ et $y(t)$ tendent vers
$+\infty$ (ou $-\infty$), il faut affiner. Le cas le plus important est le suivant :
\begin{definition}
La droite d'équation
$y=ax+b$ est \defi{asymptote oblique} à la courbe $\big(x(t),y(t) \big)$ si :
\begin{enumerate}
  \item $\frac{y(t)}{x(t)}$ tend vers un réel non nul $a$,
  \item $y(t)-ax(t)$ tend vers un réel $b$ (nul ou pas).
\end{enumerate}
\end{definition}

% \begin{enumerate}
% \item Si $\frac{y(t)}{x(t)}$ tend vers $0$, la courbe
% admet une \evidence{branche parabolique} de direction asymptotique
% d'équation $y=0$ (mais il n'y a pas d'asymptote).
%
% \item Si $\frac{y(t)}{x(t)}$ tend vers $+\infty$
% (ou $-\infty$), la courbe admet une \evidence{branche parabolique}
% de direction asymptotique d'équation $x=0$ (mais il n'y a pas d'asymptote).
%
% \item Si $\frac{y(t)}{x(t)}$ tend vers un réel
% \textbf{non nul} $a$, la courbe admet une \evidence{branche parabolique}
% de direction asymptotique d'équation $y=ax$.
%
% Il faut encore affiner l'étude. On étudie
% alors $\lim_{t\rightarrow t_0}\big(y(t)-ax(t)\big)$ avec les deux sous-cas :
% \begin{itemize}
% \item[(i)] Si $y(t)-ax(t)$ tend vers un réel $b$ (nul ou pas),
% alors $\lim_{t \to t_0}(y(t)-(ax(t)+b))=0$ et la droite d'équation
% $y=ax+b$ est \evidence{asymptote oblique} à la courbe.
% \item[(ii)] Si $y(t)-ax(t)$ tend vers $+\infty$ ou $-\infty$,
% ou n'a pas de limite, la courbe n'a qu'une direction asymptotique
% d'équation $y=ax$, mais n'admet pas de droite asymptote.
% \end{itemize}
% \end{enumerate}

De gauche à droite : asymptote verticale, horizontale, oblique.
\myfigure{0.37}{
\tikzinput{fig_courbes_part3_14}
\tikzinput{fig_courbes_part3_15}
\tikzinput{fig_courbes_part3_16}
}

Attention ! Une branche infinie peut ne pas admettre de droite asymptote, comme dans le cas d'une parabole :
\myfigure{0.4}{
\tikzinput{fig_courbes_part3_17}
}

\begin{exemple}
Étudier les asymptotes de la courbe
$\left\{\begin{array}{l} x(t) = \frac{t}{t-1} \\ y(t) = \frac{3t}{t^2-1} \end{array}\right..$
Déterminer la position de la courbe par rapport à ses asymptotes.

\medskip
\textbf{Solution.}

\begin{itemize}
  \item \textbf{Branches infinies.}
  La courbe est définie sur $\Rr \setminus \{-1,+1\}$.
  $|x(t)| \to +\infty$ uniquement lorsque $t \to +1^-$ ou $t\to +1^+$.
  $|y(t)| \to +\infty$ lorsque $t \to -1^-$ ou $t\to -1^+$ ou
  $t \to +1^-$ ou $t \to +1^+$.
  Il y a donc $4$ branches infinies, correspondant à
  $-1^-$, $-1^+$, $+1^-$, $+1^+$.

  \item \textbf{Étude en $-1^-$.}
  Lorsque $t\to-1^-$, $x(t) \to \frac12$ et $y(t) \to -\infty$ :
  la droite verticale $(x=\frac12)$ est donc asymptote
  pour cette branche infinie (qui part vers le bas).


  \item \textbf{Étude en $-1^+$.}
  Lorsque $t\to-1^+$, $x(t) \to \frac12$ et $y(t) \to +\infty$ :
  la même droite verticale d'équation $(x=\frac12)$ est  asymptote
  pour cette branche infinie (qui part cette fois vers le haut).

   \item \textbf{Étude en $+1^-$.}
  Lorsque $t\to +1^-$, $x(t) \to -\infty$ et $y(t) \to -\infty$.
  On cherche une asymptote oblique en calculant la limite de
  $\frac{y(t)}{x(t)}$ :
  $$\frac{y(t)}{x(t)} = \frac{\frac{3t}{t^2-1}}{\frac{t}{t-1}}
  = \frac{3}{t+1} \longrightarrow \frac{3}{2} \quad \text{ lorsque }t\to +1^-.$$

  On cherche ensuite si $y(t)-\frac32 x(t)$ admet une limite finie, lorsque $x\to +1^-$ :
  \begin{eqnarray*}
  y(t)-\frac32 x(t) 
  &=& \frac{3t}{t^2-1} - \frac32 \frac{t}{t-1}
  = \frac{3t - \frac32 t (t+1)}{t^2-1} \\
  &=& \frac{- \frac32 t (t-1)}{(t-1)(t+1)}
  = \frac{- \frac32 t }{t+1}
  \longrightarrow -\frac{3}{4} \quad \text{ lorsque }t\to +1^-.
  \end{eqnarray*}

  Ainsi la droite d'équation $y = \frac32 x -\frac34$ est asymptote à cette branche infinie.

  \item \textbf{Étude en $+1^+$.}  Les calculs sont similaires
  et la même droite d'équation $y = \frac32 x -\frac34$ est asymptote
  à cette autre branche infinie.

  \item \textbf{Position par rapport à l'asymptote verticale.}
  Il s'agit de déterminer le signe de $x(t) - \frac12$ lorsque
  $x \to -1^-$ (puis $x \to -1^+$).
  Une étude de signe montre que $x(t)-\frac12 > 0$ pour $t < -1$ et $t>+1$,
  et la courbe est alors à droite de l'asymptote verticale ; par contre
  $x(t)-\frac12 < 0$ pour $-1 < t < +1$, et la courbe est alors à gauche
  de l'asymptote verticale.

  \item \textbf{Position par rapport à l'asymptote oblique.}
  Il s'agit de déterminer le signe de $y(t) - \big( \frac32 x(t) -\frac34 \big)$.
  La courbe est au-dessus de l'asymptote oblique pour $-1 < t < +1$;
  et en-dessous de l'asymptote ailleurs.

  \item \textbf{Point à l'infini.} Lorsque $t\to +\infty$ (ou bien $t\to-\infty$) alors
  $x(t) \to 1$ et $y(t) \to 0$. Le point $(1,0)$ est donc un point limite de la courbe
  paramétrée.

\end{itemize}


\myfigure{1}{
\tikzinput{fig_courbes_part3_18}
}

\end{exemple}

On trouvera d'autres exemples d'études de branches infinies dans
les exercices de la section suivante.




\begin{miniexercices}
\sauteligne
\begin{enumerate}
  \item Déterminer la tangente et le type de point singulier à l'origine
  dans chacun des cas :
  $(t^5,t^3+t^4)$, $(t^2-t^3,t^2+t^3)$, $(t^2+t^3,t^4)$, $(t^3,t^6+t^7)$.

  \item Trouver les branches infinies de la courbe définie par
  $x(t) = 1-\frac{1}{1+t^2}$, $y(t)=t$. Déterminer l'asymptote, ainsi que la position
  de la courbe par rapport à cette asymptote.

  \item Mêmes questions pour les asymptotes de la courbe définie par
  $x(t) = \frac{1}{t}+\frac{1}{t-1}$, $y(t)=\frac{1}{t-1}$.

  \item Déterminer le type de point singulier de l'astroïde définie par $x(t) = \cos^3 t$,
  $y(t) = \sin^3 t$. Pourquoi l'astroïde n'a-t-elle pas de branche infinie ?

  \item Déterminer le type de point singulier de la cycloïde définie par
  $x(t) = r(t-\sin t)$,  $y(t) = r(1-\cos t)$. Pourquoi la cycloïde
  n'a-t-elle pas d'asymptote ?
\end{enumerate}
\end{miniexercices}




%%%%%%%%%%%%%%%%%%%%%%%%%%%%%%%%%%%%%%%%%%%%%%%%%%%%%%%%%%%%%%%%
\section{Plan d'étude d'une courbe paramétrée}


Dans la pratique, les courbes sont traitées de manière différente
à l'écrit et à l'oral. À l'écrit, l'étude d'une courbe est souvent
détaillée en un grand nombre de petites questions. Par contre,
à l'oral, un énoncé peut simplement prendre la forme
\og construire la courbe \fg.
Dans ce cas, on peut adopter le plan d'étude qui suit. Ce plan
n'est pas universel et n'est qu'une proposition. Aussi, pour
deux courbes  différentes, il peut être utile d'adopter
deux plans d'étude différents.

%---------------------------------------------------------------
\subsection{Plan d'étude}


\begin{enumerate}
\item \evidence{Domaine de définition de la courbe.}

Le point $M(t)$ est défini si et seulement si $x(t)$ et $y(t)$ sont définis.
Il faut ensuite déterminer un \evidence{domaine d'étude} (plus petit que le domaine de définition)
grâce aux symétries, périodicités\ldots

\item \evidence{Vecteur dérivé.}

Calcul des dérivées des coordonnées de $t\mapsto M(t)$.
Les valeurs de $t$ pour lesquelles $x'(t)=0$ (et $y'(t)\neq0$)
fournissent les points à tangente verticale et les valeurs de
$t$ pour lesquelles $y'(t)=0$ (et $x'(t)\neq0$) fournissent les
points à tangente horizontale. Enfin, les valeurs de $t$ pour
lesquelles $x'(t)=y'(t)=0$ fournissent les points singuliers,
en lesquels on n'a encore aucun renseignement sur la tangente.


\item \evidence{Tableau de variations conjointes.}

L'étude de $x'$ et $y'$ permet de connaître les variations de $x$ et $y$.
Reporter les résultats obtenus des \evidence{variations conjointes} des fonctions $x$ et $y$
dans un tableau. Cela donne alors un tableau à compléter :

$$\begin{array}{|c|ccccc|}
\hline
t&\qquad&\qquad&\qquad&\qquad&\qquad\\
\hline
x'(t)& & & & &\rule{5cm}{0cm}\\
\hline
 & & & & & \\
x& & & & & \\
 & & & & & \\
\hline
 & & & & & \\
y& & & & & \\
 & & & & & \\
\hline
y'(t)& & & & & \\
\hline
\end{array}
$$

Ce tableau est le tableau des variations des
deux fonctions $x$ et $y$ \emph{ensemble}. Il nous montre
l'évolution du point $M(t)$.
Par suite, pour une valeur de $t$ donnée, on doit lire
verticalement des résultats concernant et $x$, et $y$. Par
exemple, $x$ tend vers $+\infty$, pendant que $y$ \og vaut \fg{} $3$.


\item \evidence{Étude des points singuliers.}

\item \evidence{Étude des branches infinies.}

\item \evidence{Construction méticuleuse de la courbe.}

On
place dans l'ordre les deux axes et les unités. On construit
ensuite toutes les droites asymptotes. On place ensuite les points
importants avec leur tangente (points à tangente verticale,
horizontale, points singuliers, points d'intersection
avec une droite asymptote,\ldots). Tout est alors en place
pour la construction et on peut tracer l'arc grâce aux règles suivantes :


\mybox{
\begin{tabular}{c}
\textbf{Tracé de la courbe paramétrée $\big( x(t),y(t) \big)$}\\[2mm]
Si $x$ croît et $y$ croît, on va vers la droite et vers le haut.\\
Si $x$ croît et $y$ décroît, on va vers la droite et vers le bas.\\
Si $x$ décroît et $y$ croît, on va vers la gauche et vers le haut.\\
Si $x$ décroît et $y$ décroît, on va vers la gauche et vers le bas.
\end{tabular}
}




\item \evidence{Points multiples.}

On cherche les points multiples s'il y a lieu.
On attend souvent de commencer la construction de la courbe pour
voir s'il y a des points multiples et si on doit les chercher.
\end{enumerate}

%---------------------------------------------------------------
\subsection{Une étude complète}


\begin{exemple}
Construire la courbe
$$\left\{
\begin{array}{l}
x(t)=\dfrac{t^3}{t^2-1}\\[3mm]
y(t)=\dfrac{t(3t-2)}{3(t-1)}
\end{array}
\right..$$

\textbf{Solution.} On note $\mathcal{C}$ la courbe à construire.

\begin{itemize}
  \item \textbf{Domaine d'étude.}

Pour $t\in\Rr$, le point
$M(t)$ est défini si et seulement si $t\neq\pm1$. Aucune réduction intéressante
du domaine n'apparaît clairement et on étudie donc sur $D=\Rr\setminus\{-1,1\}$.


  \item \textbf{Variations conjointes des coordonnées.}

La fonction $x$ est dérivable sur $D$ et, pour $t\in D$,
$$x'(t)=\frac{3t^2(t^2-1)-t^3(2t)}{(t^2-1)^2}=\frac{t^2(t^2-3)}{(t^2-1)^2}.$$
La fonction $x$ est donc strictement croissante sur $]-\infty,-\sqrt{3}]$
et sur $[\sqrt{3},+\infty[$ et strictement décroissante sur $[-\sqrt{3},-1[$,
sur $]-1,1[$ et sur $]1,+\sqrt{3}[$.

\medskip

La fonction $y$ est dérivable sur $D\cup\{-1\}$ et, pour $t\in D\cup\{-1\}$,
$$y'(t)=\frac{(6t-2)(t-1)-(3t^2-2t)}{3(t-1)^2}=\frac{3t^2-6t+2}{3(t-1)^2}.$$
La fonction $y$ est donc strictement croissante sur
$]-\infty,1-\frac{1}{\sqrt{3}}]$ et sur $[1+\frac{1}{\sqrt{3}},+\infty[$,
strictement décroissante sur
$[1-\frac{1}{\sqrt{3}},1[$ et sur $]1,1+\frac{1}{\sqrt{3}}]$.

\medskip

Les fonctions $x'$ et $y'$ ne s'annulent jamais simultanément et
la courbe est donc régulière. La tangente en un point $M(t)$ est dirigée
par le vecteur dérivé $\left(\frac{t^2(t^2-3)}{(t^2-1)^2},\frac{3t^2-6t+2}{3(t-1)^2}\right)$ ou encore
par le vecteur $\left(\frac{3t^2(t^2-3)}{(t+1)^2},3t^2-6t+2\right)$.


\item \textbf{Tangentes parallèles aux axes.}

$y'$ s'annule en $1-\frac{1}{\sqrt{3}}$ et $1+\frac{1}{\sqrt{3}}$.
En les points $M(1-\frac{1}{\sqrt{3}})$ et $M(1+\frac{1}{\sqrt{3}})$, la
courbe admet une tangente parallèle à $(Ox)$. On a
\begin{align*}
x\left(1-\tfrac{1}{\sqrt{3}}\right)&=\left(1-\tfrac{1}{\sqrt{3}}\right)^3/\Big(\left(1-\tfrac{1}{\sqrt{3}}\right)^2-1\Big)
=\left(1-\tfrac{3}{\sqrt{3}}+\tfrac{3}{3}-\tfrac{1}{3\sqrt{3}}\right)/\left(-\tfrac{2}{\sqrt{3}}+\tfrac{1}{3}\right)\\
%&=\frac{6\sqrt{3}-10}{-6+\sqrt{3}}\\
 &=\tfrac{1}{33}\left(6\sqrt{3}-10\right)\left(-6-\sqrt{3}\right)=\frac{42-26\sqrt{3}}{33}=-0,09\ldots,
\end{align*}
et de même,
\begin{align*}
y\left(1-\tfrac{1}{\sqrt{3}}\right)
&=\tfrac{1}{3}\left(1-\tfrac{1}{\sqrt{3}}\right)\left(3-\sqrt{3}-2\right)/\left(-\tfrac{1}{\sqrt{3}}\right)\\
&=-\tfrac{1}{3}\left(\sqrt{3}-1\right)\left(1-\sqrt{3}\right)=\frac{4-2\sqrt{3}}{3}=0,17\ldots
\end{align*}

Puis, par un calcul conjugué (c'est-à-dire en remplaçant $\sqrt{3}$
par $-\sqrt{3}$ au début de calcul), on a
$x(1+\frac{1}{\sqrt{3}})=\frac{42+26\sqrt{3}}{33}
=2,63\ldots$ et $y(1+\frac{1}{\sqrt{3}})=\frac{4+2\sqrt{3}}{3}=2,48\ldots$

\medskip

$x'$ s'annule en $0$, $\sqrt{3}$ et $-\sqrt{3}$. Au point
$M(0)=(0,0)$, $M(\sqrt{3})=(\frac{3\sqrt{3}}{2},\frac{3+7\sqrt{3}}{6})
=(2,59\ldots,2,52\ldots)$ et en
$M(-\sqrt{3})=(-\frac{3\sqrt{3}}{2},\frac{3-7\sqrt{3}}{6})
=(-2,59\ldots,-1,52\ldots)$, il y a une tangente parallèle à $(Oy)$.


\item \textbf{Étude en l'infini.}

Quand $t$ tend vers $+\infty$, $x(t)$ et $y(t)$ tendent
toutes deux vers $+\infty$ et il y a donc une branche infinie.
Même chose quand $t$ tend vers $-\infty$.

Étudions $\lim_{t \to \pm\infty}\frac{y(t)}{x(t)}$.
Pour $t\in D\setminus\{0\}$,
$$\frac{y(t)}{x(t)}
=\frac{t(3t-2)}{3(t-1)}\times\frac{t^2-1}{t^3}=\frac{(3t-2)(t+1)}{3t^2}.$$
Cette expression tend vers $1$ quand $t$ tend vers $+\infty$ ou
vers $-\infty$.

Pour $t\in D$,
$$y(t)-x(t)=\frac{t(3t-2)}{3(t-1)}-\frac{t^3}{t^2-1}
=\frac{t(3t-2)(t+1)-3t^3}{3(t-1)(t+1)}=\frac{t^2-2t}{3(t-1)(t+1)}.$$
Cette expression tend vers $\frac{1}{3}$
quand $t$ tend vers $+\infty$ ou vers $-\infty$. Ainsi,
$$\lim_{t\rightarrow\pm\infty}\big(y(t)-(x(t)+\tfrac{1}{3})\big)=0.$$
Quand $t$ tend vers $+\infty$ ou vers $-\infty$, la droite
$\Delta$ d'équation $y=x+\frac{1}{3}$ est donc asymptote à la courbe.

Étudions la position relative de $\mathcal{C}$ et $\Delta$.
Pour $t\in D$, $y(t)-\big(x(t)+\frac{1}{3}\big)=\frac{t^2-2t}{3(t-1)(t+1)}-\frac{1}{3}
=\frac{-2t+1}{3(t-1)(t+1)}$.

{\scriptsize
$$\begin{array}{|c|cr||lcr|lcr||lc|}
\hline
t&-\infty\qquad\qquad &\multicolumn{2}{c}{-1}& &\multicolumn{2}{c}{\frac12}& &\multicolumn{2}{c}{1}& \qquad\qquad+\infty\\
\hline
\begin{array}{c}\text{signe de}\\\;y(t)-\big(x(t)+\frac{1}{3}\big)\end{array}& +& & &-& & &+& & &- \\
\hline
\text{position}& \mathcal{C}\;\text{au-dessus}& & &\mathcal{C}\;\text{en-dessous}& & &\mathcal{C}\;\text{au-dessus}& & &\mathcal{C}\;\text{en-dessous} \\
\text{relative}& \text{de}\;\Delta& & &\text{de}\;\Delta& & &\text{de}\;\Delta& & &\text{de}\;\Delta \\
\hline
\end{array}
$$
}

$\mathcal{C}$ et $\Delta$ se coupent au point $M(1/2)=(-1/6,1/6)=(-0,16\ldots,0,16\ldots)$.


  \item \textbf{Étude en $t=-1$.}

  Quand $t$ tend vers $-1$, $y(t)$ tend vers $-5/6$, et $x(t)$ tend vers $-\infty$ en $-1^-$
et vers $+\infty$ en $-1^+$. La
droite d'équation $y=-\frac{5}{6}$ est asymptote à $\mathcal{C}$.
La position relative est fournie par le signe de
$y(t)+\frac{5}{6}=\frac{6t^2+t-5}{6(t-1)} = \frac{(t+1)(6t-5)}{6(t-1)}$.


 \item \textbf{Étude en $t=1$.}

 Quand $t$ tend vers $1$, $x$ et $y$ tendent vers l'infini, $\frac{y(t)}{x(t)}
=\frac{(3t-2)(t+1)}{3t^2}$ tend vers $\frac{2}{3}$ et
$y(t)-\frac{2}{3}x(t)=\frac{t^3+t^2-2t}{3(t^2-1)}=\frac{t(t+2)}{3(t+1)}$
tend vers $\frac{1}{2}$. La droite d'équation
$y=\frac{2}{3}x+\frac{1}{2}$ est asymptote à la courbe.
La position relative est fournie par le signe de
$y(t)-\big(\frac{2}{3}x(t)+\frac{1}{2}\big)
=\frac{2t^2+t-3}{6(t+1)} = \frac{(t-1)(2t+3)}{6(t+1)}$.

  \item \textbf{Tableau de variations conjointes.}

\myfigure{0.7}{
\scriptsize
\tikzinput{fig_courbes_part4_01}
}



  \item \textbf{Intersection avec les axes.}

$x(t)=0$ équivaut à $t=0$.
La courbe coupe $(Oy)$ au point $M(0)=(0,0)$. $y(t)=0$
équivaut à $t=0$ ou $t=\frac{2}{3}$. La courbe coupe $(Ox)$ au point $M(0)=(0,0)$ et
$M(\frac{2}{3})=(-\frac{8}{15},0)$.


  \item \textbf{Tracé de la courbe.}
\end{itemize}

\myfigure{0.7}{
\tikzinput{fig_courbes_part4_02}
}

Le tracé fait apparaître un \textbf{point double}. Je vous laisse le chercher (et le trouver).
\end{exemple}




%---------------------------------------------------------------
\subsection{Une courbe de Lissajous}

\begin{exemple}
Construire la courbe $\left\{
\begin{array}{l}
x=\sin(2t)\\
y=\sin(3t)
\end{array}
\right.$ de la famille des \defi{courbes de Lissajous}.

\medskip
\textbf{Solution.}


\begin{itemize}
  \item \textbf{Domaine d'étude.}
  Pour tout réel $t$, $M(t)$ existe et $M(t+2\pi)=M(t)$.
On obtient la courbe complète quand $t$ décrit $[-\pi,\pi]$.

Pour $t\in[-\pi,\pi]$, $M(-t)=s_O(M(t))$, puis pour $t\in[0,\pi]$,
$M(\pi-t)=s_{(Oy)}(M(t))$. On étudie et on construit l'arc
quand $t$ décrit $[0,\frac{\pi}{2}]$, puis on obtient la courbe
complète par réflexion d'axe $(Oy)$ puis par symétrie centrale de centre
$O$. Puisque, pour tout réel $t$, $M(t+\pi)=s_{(Ox)}(M(t))$,
l'axe $(Ox)$ est également axe de symétrie de la courbe.

  \item \textbf{Localisation.}
  Pour tout réel $t$, $|x(t)|\leq1$ et $|y(t)|\leq1$.
Le support de la courbe est donc contenu dans le carré
$\big\{(x,y)\in\Rr^2 \ \  \mid \ \ |x|\leq1\ \text{et}\ |y|\leq1\big\}$.

  \item \textbf{Variations conjointes.}
  D'après les propriétés usuelles de la fonction sinus,
la fonction $x$ est croissante sur $[0,\frac{\pi}{4}]$ et décroissante
sur $[\frac{\pi}{4},\frac{\pi}{2}]$ ;
et de même, la fonction $y$ croît sur $[0,\frac{\pi}{6}]$ et décroît
sur $[\frac{\pi}{6},\frac{\pi}{2}]$.

  \item \textbf{Vecteur dérivé et tangente.}
  \begin{itemize}
    \item Pour $t\in[0,\frac{\pi}{2}]$, $\overrightarrow{\frac{\dd M}{\dd t}}(t)
=\big(2\cos(2t),3\cos(3t)\big)$. Par suite :
$$\overrightarrow{\frac{\dd M}{\dd t}}(t)=\vec{0}
\iff\cos(2t)=\cos(3t)=0
\iff t\in\big(\tfrac{\pi}{4}+\tfrac{\pi}{2}\Zz\big)\cap\big(\tfrac{\pi}{6}+\tfrac{\pi}{3}\Zz\big)=\varnothing.$$

Donc $\overrightarrow{\frac{\dd M}{\dd t}}$ ne s'annule pas et la courbe est régulière.
La tangente en tout point est dirigée par le vecteur $\big(2\cos(2t),3\cos(3t)\big)$.

    \item Cette tangente est parallèle à $(Ox)$ si et seulement si $\cos(3t)=0$ ou
encore $t\in\frac{\pi}{6}+\frac{\pi}{3}\Zz$ ou enfin
$t=\frac{\pi}{6}$ et $t=\frac{\pi}{2}$, et
cette tangente est parallèle à $(Oy)$ si et seulement si $\cos(2t)=0$
ou encore $t\in\frac{\pi}{4}+\frac{\pi}{2}\Zz$ ou enfin $t=\frac{\pi}{4}$.

    \item La tangente en $M(0)$ est dirigée par le vecteur $(2,3)$
et a donc pour coefficient directeur $3/2$.

    \item Pour $t\in[0,\frac{\pi}{2}]$, $M(t)\in(Ox)$ si et
seulement si $\sin(3t)=0$ ou encore $t\in\frac{\pi}{3}\Zz$ ou
enfin $t=0$ ou $t=\frac{\pi}{3}$. La tangente en
$M(\pi/3)$ est dirigée par le vecteur $(-1,-3)$ et a donc pour
coefficient directeur $3$.
    \end{itemize}
  \end{itemize}


\myfigure{1.2}{
\tikzinput{fig_courbes_part4_03}
}

\end{exemple}



%---------------------------------------------------------------
\subsection{Le folium de Descartes}


Il existe d'autres façons de définir une courbe, par exemple par une équation
cartésienne du type $f(x,y)=0$.
Par exemple, $(x^2+y^2-1=0)$ définit le cercle de rayon $1$, centré à l'origine.

Pour étudier les équations $f(x,y)=0$, il nous manque un cours sur les
fonctions de deux variables. Néanmoins, il est possible dès à présent
de construire de telles courbes en trouvant une paramétrisation.
Une idée (parmi tant d'autres), fréquemment utilisée en pratique,
est de chercher \emph{l'intersection de la courbe avec toutes les droites
passant par l'origine} comme le montre le dessin suivant. Ceci revient
en gros à prendre comme paramètre le réel $t=\frac{y}{x}$.

\myfigure{0.9}{
\tikzinput{fig_courbes_part4_05}
}

\begin{exemple}
Construire le \defi{folium de Descartes} $\mathcal{C}$
d'équation $x^3+y^3-3axy=0$, $a$ étant un réel strictement positif donné.

\medskip
\textbf{Solution.}

Commençons par montrer que l'intersection de la courbe avec
l'axe des ordonnées est réduite à l'origine :
$$M(x,y)\in\mathcal{C}\cap(Oy) \iff x^3+y^3-3axy=0
\ \text{ et }\ x=0 \iff x=y=0.$$


Soient $t\in\Rr$ et $D_t$ la droite d'équation $(y=tx)$.
Cherchons l'intersection de cette droite $D_t$ avec notre courbe $\mathcal{C}$ :

$$\begin{array}{l}
M(x,y)\in D_t\cap\mathcal{C}\setminus(Oy) \\
\iff\left\{
\begin{array}{l}
x\neq0\\
y=tx\\
x^3+t^3x^3-3atx^2=0
\end{array}
\right.
\iff\left\{
\begin{array}{l}
x\neq0\\
y=tx\\
(1+t^3)x-3at=0
\end{array}
\right.\\
\iff\left\{
\begin{array}{l}
x\neq0\\
x=\frac{3at}{1+t^3}\\
y=\frac{3at^2}{1+t^3}
\end{array}
\right.\;\text{pour}\;t\notin\{-1\}\iff\left\{
\begin{array}{l}
x=\frac{3at}{1+t^3}\\
y=\frac{3at^2}{1+t^3}
\end{array}
\right.\;\text{pour}\;t\notin\{-1,0\}.
\end{array}$$



Ainsi $\mathcal{C}$ est la réunion
de $\{O\}$ et de l'ensemble des points $\big(\frac{3at}{1+t^3},\frac{3at^2}{1+t^3}\big)$,
$t\notin\{-1,0\}$.
D'autre part les droites $D_{-1}$ et $D_0$ n'ont qu'un point commun avec $\mathcal{C}$,
à savoir le point $O$.
Comme $t=0$ refournit le point $O$, on a plus simplement :
$$\mathcal{C}=\left\{\left(\frac{3at}{1+t^3},\frac{3at^2}{1+t^3}\right)\ \mid \ t\in\Rr\setminus\{-1\}\right\}.$$
Une paramétrisation de la courbe est donc
$$t\mapsto\left\{
\begin{array}{l}
x(t)=\dfrac{3at}{1+t^3}\\
y(t)=\dfrac{3at^2}{1+t^3}
\end{array}
\right..$$

Après étude, on obtient le graphe suivant :
\myfigure{1.2}{
\tikzinput{fig_courbes_part4_04}
}

\end{exemple}

\begin{miniexercices}
\sauteligne
\begin{enumerate}
  \item  Faire une étude complète et le tracé de la courbe définie par
  $x(t) = \tan \left(\frac t 3\right)$,  $y(t) = \sin (t)$.

  \item Faire une étude complète et le tracé de
  l'astroïde définie par $x(t) = \cos^3 t$,  $y(t) = \sin^3 t$.


  \item Faire une étude complète et le tracé de
  la cycloïde définie par   $x(t) = r(t-\sin t)$,  $y(t) = r(1-\cos t)$.
\end{enumerate}
\end{miniexercices}


%%%%%%%%%%%%%%%%%%%%%%%%%%%%%%%%%%%%%%%%%%%%%%%%%%%%%%%%%%%%%%%%
\section{Courbes en polaires : théorie}
\index{courbe!en polaires}
%---------------------------------------------------------------
\subsection{Coordonnées polaires}

Rappelons tout d'abord la définition précise des coordonnées polaires.
Le plan est rapporté à un repère orthonormé
$(O,\overrightarrow{i},\overrightarrow{j})$. Pour $\theta$ réel,
on pose
$$\overrightarrow{u_\theta}=\cos\theta\overrightarrow{i}+\sin\theta\overrightarrow{j}
\quad \text{ et  } \quad
\overrightarrow{v_\theta}=-\sin\theta\overrightarrow{i}+\cos\theta\overrightarrow{j}
=\overrightarrow{u_{\theta+\pi/2}}.$$
$M$ étant un point du plan, on dit que $[r:\theta]$ est un
couple de \defi{coordonnées polaires} du point $M$ si et seulement
si $\overrightarrow{OM}=r\overrightarrow{u_\theta}$.



\mybox{
$M = [r:\theta]
\iff \overrightarrow{OM} =r\overrightarrow{u_\theta}
\iff M=O+r\overrightarrow{u_\theta}.$}

\myfigure{0.8}{
\tikzinput{fig_courbes_part5_01} \ 
\tikzinput{fig_courbes_part5_02}
}


%---------------------------------------------------------------
\subsection{Courbes d'équations polaires}

La courbe d'\defi{équation polaire} $r=f(\theta)$
est l'application suivante, où les coordonnées des
points sont données en coordonnées polaires :
$$\begin{array}{cccc}
F :&D&\rightarrow&\Rr^2\\
 &\theta&\mapsto&M(\theta) = \big[r(\theta):\theta\big] = O+r(\theta)\vec{u}_\theta
\end{array}
$$
ou encore, sous forme complexe, $\theta \mapsto r(\theta)e^{\ii\theta}$.




\begin{exemple}
Voici une spirale d'équation polaire $r = \sqrt{\theta}$,
définie pour $\theta \in [0,+\infty[$.

Par exemple pour $\theta = 0$, $r(\theta)= 0$,
donc l'origine appartient à la courbe $\mathcal{C}$.
Pour $\theta=\frac\pi2$, $r(\theta)=\sqrt{\frac\pi2}$, donc
$M(\frac\pi2) = \big[\sqrt{\frac\pi2} : \frac\pi2\big]$,
soit en coordonnées cartésiennes
$M(\frac\pi2) = (0,\  1,25\ldots) \in \mathcal{C}$.
Puis $M(\pi) = \big[\sqrt\pi : \pi\big] = (-1,77\ldots,0) \in \mathcal{C}$,
$M(2\pi) = \big[\sqrt{2\pi} : 2\pi\big]= \big[\sqrt{2\pi} : 0\big]
= (2,50\ldots,0) \in \mathcal{C}$,
\ldots


\myfigure{1}{
\tikzinput{fig_courbes_part5_03}
}
\end{exemple}



Une telle équation ($r=f(\theta)$) ressemble à une équation cartésienne
($y=f(x)$). Mais la non unicité d'un couple de coordonnées polaires
en fait un objet plus compliqué.
Reprenons l'exemple de la spirale d'équation polaire $r=\sqrt\theta$.
Le point de coordonnées polaires $[\sqrt{\pi}:\pi]$
est sur la spirale, mais aussi le point
$[-\sqrt{\pi}:2\pi]$ (car $[-\sqrt{\pi}:2\pi]=[\sqrt{\pi}:\pi]$).
Ainsi, si en cartésien on peut écrire
$M(x,y)\in\mathcal{C}_f\iff y=f(x)$,
ce n'est pas le cas en polaires, où l'on a seulement
$r=f(\theta) \implies M[r:\theta]\in\mathcal{C}$.

Pour avoir une équivalence, avec $\mathcal{C}$
la courbe d'équation polaire $r=f(\theta)$ et $M$ un point du plan,
il faut écrire :
\mybox{
$M\in\mathcal{C}\iff$
\begin{tabular}{c}
il existe un couple $[r:\theta]$ de coordonnées \\
polaires de $M$ tel que $r=f(\theta)$.
\end{tabular}
}

\begin{remarque*}
\sauteligne
\begin{itemize}
\item Dans cette présentation, la lettre $r$ désigne à la fois la première
des deux coordonnées polaires du point $[r:\theta]$ et aussi la
fonction $\theta\mapsto r(\theta)$, cette confusion des notations
étant résumée dans l'égalité $r=r(\theta)$.
% Cette notation unique
% pour deux objets différents se révèle parfois pratique, pour énoncer
% et mémoriser des formules par exemple. Mais elle est parfois source
% d'incompréhension. Dans ce dernier cas, il faut différencier les
% notations en notant plutôt $\theta \mapsto f(\theta)$ la fonction.

\item $r(\theta)$ n'est pas nécessairement la distance $OM(\theta)$ car la
fonction $r$ peut tout à fait prendre des valeurs strictement négatives.
La formule générale est  $OM(\theta)=|r(\theta)|$.

\item Grâce aux relations usuelles entre les coordonnées
cartésiennes et les coordonnées polaires d'un point, on peut à
tout moment écrire une représentation polaire sous la forme
d'une représentation paramétrique classique :
$$\theta\mapsto\left\{
\begin{array}{l}
x(\theta)=r(\theta)\cos(\theta)\\
y(\theta)=r(\theta)\sin(\theta)
\end{array}
\right..$$
\end{itemize}
\end{remarque*}






%---------------------------------------------------------------
\subsection{Calcul de la vitesse en polaires}

Pour pouvoir dériver un arc en coordonnées polaires, il faut d'abord savoir
dériver le vecteur
$\overrightarrow{u_\theta}=\cos\theta\overrightarrow{i}+\sin\theta\overrightarrow{j}$
en tant que fonction de $\theta$ :

$$\frac{\dd\overrightarrow{u_\theta}}{\dd\theta}(\theta)
=-\sin\theta\overrightarrow{i}+\cos\theta\overrightarrow{j}
=\overrightarrow{v_\theta}=\overrightarrow{u_{\theta+\pi/2}}$$
$$
\quad\text{ et }\quad
\frac{\dd\overrightarrow{v_\theta}}{\dd\theta}(\theta)
=\overrightarrow{u_{\theta+\pi/2+\pi/2}}
=\overrightarrow{u_{\theta+\pi}}=-\overrightarrow{u_\theta}.$$


En résumé, ils s'obtiennent par rotation d'angle $+\frac \pi2$  :
\mybox{
$\displaystyle\frac{\dd\overrightarrow{u_\theta}}{\dd\theta}
=\overrightarrow{v_\theta} \qquad\frac{\dd\overrightarrow{v_\theta}}{\dd\theta}
=-\overrightarrow{u_\theta}$
}

% \bigskip
%
%
% Soient maintenant $r$ et $\theta$ deux fonctions dérivables, dépendant d'un paramètre
% $t \in D \subset \Rr$, à valeurs dans $\Rr$. Alors la fonction
% $t\mapsto M(t)= O+r(t)\overrightarrow{u_{\theta(t)}}$
% est dérivable sur $D$ et
% \mybox{
% $\displaystyle\overrightarrow{\frac{\dd M}{\dd t}}
% =r'\overrightarrow{u_{\theta}}+r\theta'\overrightarrow{v_{\theta}}$
% }


%---------------------------------------------------------------
\subsection{Tangente en un point distinct de l'origine}

Soient $r$ une fonction dérivable sur un domaine $D$ de $\Rr$ à
valeurs dans $\Rr$ et $\mathcal{C}$ la courbe d'équation polaire
$r=r(\theta)$ ou encore de représentation polaire
$\theta\mapsto O+r(\theta)\overrightarrow{u_\theta}$.

\begin{theoreme}[Tangente en un point distinct de l'origine]
\index{courbe!tangente}
\sauteligne
\begin{enumerate}
\item Tout point de $\mathcal{C}$ distinct de l'origine $O$ est un point régulier.

\item Si $M(\theta)\neq O$, la tangente en $M(\theta)$ est dirigée
par le vecteur
\myboxinline{$\overrightarrow{\dfrac{\dd M}{\dd \theta}}(\theta)
=r'(\theta)\overrightarrow{u_\theta}+r(\theta)\overrightarrow{v_\theta}$}

\item L'angle $\beta$ entre le vecteur $\overrightarrow{u_\theta}$ et la tangente
en $M(\theta)$ vérifie \myboxinline{$\tan(\beta)=\frac{r}{r'}$} si $r'\neq 0$,
et $\beta=\frac\pi2 \pmod \pi$ sinon.

\end{enumerate}
\end{theoreme}


\myfigure{1.3}{
\tikzinput{fig_courbes_part5_08}
}

Le repère $(M(\theta),\overrightarrow{u_\theta},\overrightarrow{v_\theta})$ est le \defi{repère polaire}
en $M(\theta)$. Dans ce repère, les coordonnées du vecteur
$\overrightarrow{\frac{\dd M}{\dd\theta}}$ sont donc $(r',r)$.
On note $\beta$ l'angle $(\overrightarrow{u_\theta},\overrightarrow{\frac{\dd M}{\dd\theta}})$
et $\alpha$ l'angle $(\overrightarrow{i},\overrightarrow{\frac{\dd M}{\dd\theta}})$ de sorte
que $\alpha=\beta+\theta$.


\begin{proof}
~
\begin{itemize}
  \item Comme $M(\theta) = O+r(\theta) \overrightarrow{u_\theta}$, alors par la formule de dérivation d'un produit :
$$\overrightarrow{\frac{\dd M}{\dd \theta}}(\theta)
= \frac{\dd r(\theta)}{\dd \theta}\overrightarrow{u_\theta}
+ r(\theta)\frac{\dd \overrightarrow{u_\theta}}{\dd \theta}
=r'(\theta)\overrightarrow{u_\theta}+r(\theta)\overrightarrow{v_\theta}$$

  \item Déterminons alors les éventuels points singuliers. Puisque les
vecteurs $\overrightarrow{u_\theta}$ et $\overrightarrow{v_\theta}$
ne sont pas colinéaires,
$$\overrightarrow{\frac{\dd M}{\dd\theta}}(\theta)
=\overrightarrow{0} \iff r(\theta)=0 \text{ et } r'(\theta)=0$$

Maintenant, comme $r(\theta)=0 \iff M(\theta)=O$, on en déduit que
tout point distinct de l'origine est un point régulier.

  \item Comme $\overrightarrow{\frac{\dd M}{\dd \theta}}(\theta)
=r'(\theta)\overrightarrow{u_\theta}+r(\theta)\overrightarrow{v_\theta}$
alors, dans le repère polaire
$(M(\theta),\overrightarrow{u_\theta},\overrightarrow{v_\theta})$,
les coordonnées de $\overrightarrow{\frac{\dd M}{\dd\theta}}$ sont $(r',r)$.
On a alors
$$\cos\beta=\frac{r'}{\sqrt{r^2+r'^2}} \quad \text{ et } \quad \sin\beta = \frac{r}{\sqrt{r^2+r'^2}}.$$

Ces égalités définissent $\beta$ modulo $2\pi$. Ensuite, (puisque $r\neq0$) on a
$\frac{1}{\tan\beta}=\frac{r'}{r}$.
On préfère retenir que,
si de plus $r'\neq0$, $\tan(\beta)=\frac{r}{r'}$.
Les deux dernières égalités déterminent $\beta$ modulo $\pi$, ce qui
est suffisant pour construire la tangente, mais
insuffisant pour construire le vecteur $\overrightarrow{\frac{\dd M}{\dd\theta}}$.
\end{itemize}

\end{proof}


\begin{exemple}
Déterminer, au point $M(\frac{\pi}{2})$, la tangente à la courbe polaire :
$$r=1-2\cos\theta.$$

\medskip
\textbf{Solution.}

On note $\mathcal{C}$ la courbe.

\begin{enumerate}
  \item \textbf{Première méthode.}
  On détermine l'angle
  $(\overrightarrow{u_\theta},\overrightarrow{\frac{\dd M}{\dd\theta}})$
  formé par la tangente avec la droite d'angle polaire $\theta$.
  Comme $r'(\theta)=2\sin\theta$, alors $r'(\frac{\pi}{2})=2$.
  De plus, $r(\frac{\pi}{2})=1\neq0$. Donc,
  $$\tan\beta  = \frac{r(\frac{\pi}{2})}{r'(\frac{\pi}{2})} = \frac{1}{2}.$$
  Ainsi, modulo $\pi$,
  $$\beta = \arctan(\tfrac{1}{2})
  = \frac\pi2-\arctan(2).$$

  L'angle polaire de la tangente en $M(\frac{\pi}{2})$ est
donc
$$\alpha=\beta + \theta=\frac\pi2+\frac\pi2-\arctan(2) = \pi-\arctan(2).$$

  \item \textbf{Seconde méthode.}
  On calcule le vecteur dérivé, qui bien sûr dirige la tangente :
  $$\overrightarrow{\frac{\dd M}{\dd\theta}}\left(\frac{\pi}{2}\right)
=2\cdot\overrightarrow{u_{\pi/2}}+1\cdot\overrightarrow{v_{\pi/2}}
=-\overrightarrow{i}+2\overrightarrow{j}.$$
  Comme la tangente passe par le point
  $M(\frac{\pi}{2}) = \big[1:\frac{\pi}{2}\big] = (0,1)$,
une équation cartésienne de cette tangente est donc
$2\cdot(x-0)+1\cdot(y-1)=0$ ou encore $y=-2x+1$.
\end{enumerate}

\myfigure{1}{
\tikzinput{fig_courbes_part5_04}
}
\end{exemple}



%---------------------------------------------------------------
\subsection{Tangente à l'origine}

Supposons maintenant que, pour un certain réel $\theta_0$, la courbe
passe par l'origine $O$. On suppose comme d'habitude que l'arc est
localement simple, ce qui revient à dire qu'au voisinage de $\theta_0$,
la fonction $r$ ne s'annule qu'en $\theta_0$.
\begin{theoreme}[Tangente à l'origine]
\index{courbe!tangente}
Si $M(\theta_0)=O$, la tangente en $M(\theta_0)$ est la droite
d'angle polaire $\theta_0$.
\end{theoreme}

\myfigure{1}{
\tikzinput{fig_courbes_part5_05}
}
Une équation cartésienne de cette
droite dans le repère $(O,\overrightarrow{i},\overrightarrow{j})$
est donc $y=\tan(\theta_0)x$, si $\theta_0\notin\frac{\pi}{2}+\pi\Zz$
et $x=0$, si $\theta_0\in\frac{\pi}{2}+\pi\Zz$.


\begin{proof}
Pour $\theta\neq\theta_0$, le vecteur
$$\frac{1}{r(\theta)}\overrightarrow{M(\theta_0)M(\theta)}=
\frac{1}{r(\theta)}\overrightarrow{OM(\theta)}
=\overrightarrow{u_\theta},$$
dirige la droite $\big(M(\theta_0)M(\theta)\big)$. Or, quand $\theta$
tend vers $\theta_0$, $\overrightarrow{u_\theta}$ tend vers
$\overrightarrow{u_{\theta_0}}$. Ainsi $\overrightarrow{u_{\theta_0}}$
est un vecteur directeur de la tangente, comme on le souhaitait.
\end{proof}







\begin{remarque*}
En l'origine, on ne peut avoir qu'un
\evidence{point d'allure ordinaire} ou un \evidence{rebroussement de première espèce}.

\begin{itemize}
  \item Si $r$ s'annule en changeant de signe, le point $M(\theta)$
franchit l'origine en tournant dans le sens direct : c'est un
point d'allure ordinaire.

  \item Si $r$ s'annule sans changer de signe
en arrivant en $O$, on rebrousse chemin en traversant la tangente
(puisque l'on tourne toujours dans le même sens) : c'est un
rebroussement de première espèce.
\end{itemize}

\end{remarque*}





\begin{exemple}
Étudier le point $M(\frac{\pi}{2})$ dans les deux cas suivants :
$$r=(\theta+1)\cos \theta \qquad \text{ et } \qquad r=\cos^2(\theta).$$

\medskip
\textbf{Solution.}
Dans les deux cas, $M(\frac{\pi}{2})=O$ et
la tangente en $M(\frac{\pi}{2})$ est la droite passant par $O$ et
d'angle polaire $\frac{\pi}{2}$, c'est-à-dire l'axe des ordonnées.

\begin{itemize}
  \item Dans le premier cas, $r$ change de signe en franchissant
$\frac{\pi}{2}$, de positif à négatif. Ainsi, en tournant toujours dans le même
sens, on se rapproche de l'origine, on la franchit et on
s'en écarte : c'est un point d'allure ordinaire.

  \item Dans le deuxième cas, $r$ ne change pas de signe.
On ne franchit pas l'origine. On rebrousse chemin tout en tournant
toujours dans le même sens : c'est un point de rebroussement de première espèce.
\end{itemize}

\myfigure{0.95}{
\tikzinput{fig_courbes_part5_06}
\tikzinput{fig_courbes_part5_07}
}
\end{exemple}


\begin{miniexercices}
\sauteligne
\begin{enumerate}
  \item Soit la courbe d'équation polaire $r = (\cos \theta)^2$.
  Quand est-ce que la tangente en $M(\theta)$ est perpendiculaire à $\overrightarrow{u_\theta}$ ?
  Quelle est la tangente à l'origine ? En quels points la tangente est-elle
  horizontale ? Tracer la courbe.

  \item Montrer que la courbe polaire $r=\frac{1}{\cos \theta+2\sin\theta}$
  est une droite, que vous déterminerez. Même problème avec
  $r = \frac{1}{\cos \left( \theta-\frac\pi4 \right)}$.

  \item Montrer que la courbe polaire $r = \cos \theta$
  est un cercle, que vous déterminerez.
\end{enumerate}
\end{miniexercices}



%%%%%%%%%%%%%%%%%%%%%%%%%%%%%%%%%%%%%%%%%%%%%%%%%%%%%%%%%%%%%%%%
\section{Courbes en polaires : exemples}


%---------------------------------------------------------------
\subsection{Réduction du domaine d'étude}


On doit connaître l'effet de transformations géométriques
usuelles sur les coordonnées polaires d'un point.
Le plan est rapporté à un repère orthonormé direct,
$M$ étant le point de coordonnées polaires $[r:\theta]$.

\begin{itemize}
  \item Réflexion d'axe $(Ox)$. $s_{(Ox)} : [r:\theta] \mapsto [r:-\theta]$.

  \item Réflexion d'axe $(Oy)$. $s_{(Oy)} : [r:\theta] \mapsto [r:\pi-\theta]$.

  \item Symétrie centrale de centre $O$. $s_O : [r:\theta] \mapsto [r:\theta+\pi]=[-r:\theta]$.

  \item Réflexion d'axe la droite $D$ d'équation $(y=x)$.
$s_D(M) : [r:\theta] \mapsto [r:\frac{\pi}{2}-\theta]$.

  \item Réflexion d'axe la droite $D'$ d'équation $(y=-x)$.
$s_{D'}(M) : [r:\theta] \mapsto [-r:\frac{\pi}{2}-\theta]=[r:-\frac{\pi}{2}-\theta]$.

  \item Rotation d'angle $\frac{\pi}{2}$ autour de $O$.
$r_{O,\pi/2} : [r:\theta] \mapsto [r:\theta+\frac{\pi}{2}]$.

  \item Rotation d'angle $\varphi$ autour de $O$.
  $r_{O,\varphi} : [r:\theta] \mapsto [r:\theta+\varphi]$.
\end{itemize}

Voici quelques transformations :
\myfigure{0.6}{
\tikzinput{fig_courbes_part6_01a}
\tikzinput{fig_courbes_part6_01b}
}
\myfigure{0.6}{
\tikzinput{fig_courbes_part6_01c}
\tikzinput{fig_courbes_part6_01d}
}

\begin{exemple}

Déterminer un domaine d'étude le plus simple possible de la
courbe d'équation polaire
$$r=1+2\cos^2\theta.$$

\medskip
\textbf{Solution.}

\begin{itemize}
  \item La fonction $r$ est définie sur $\Rr$ et
$2\pi$-périodique. Donc, pour $\theta\in\Rr$,
$$M(\theta+2\pi)=\big[r(\theta+2\pi):\theta+2\pi\big]=\big[r(\theta):\theta\big]=M(\theta).$$
La courbe complète est donc obtenue quand $\theta$ décrit un
intervalle de longueur $2\pi$ comme $[-\pi,\pi]$ par exemple.

  \item La fonction $r$ est paire. Donc, pour $\theta\in[-\pi,\pi]$,
$$M(-\theta)=\big[r(-\theta):-\theta\big]=\big[r(\theta):-\theta\big]=s_{(Ox)}\big(M(\theta)\big).$$
On étudie et construit la courbe sur $[0,\pi]$, puis on obtient
la courbe complète par réflexion d'axe $(Ox)$.

  \item $r(\pi-\theta)=r(\theta)$. Donc, pour $\theta\in[0,\pi]$,
$$M(\pi-\theta)=\big[r(\pi-\theta):\pi-\theta\big]=\big[r(\theta):\pi-\theta\big]=s_{(Oy)}\big(M(\theta)\big).$$
On étudie et construit la courbe sur $[0,\frac{\pi}{2}]$,
puis on obtient la courbe complète par réflexion d'axe $(Oy)$ puis par
réflexion d'axe $(Ox)$.



  \item On obtiendrait les tracés suivants sur $[0,\frac{\pi}{2}]$,
sur $[0,\pi]$ puis $[0,2\pi]$.

\end{itemize}

\myfigure{0.7}{
\tikzinput{fig_courbes_part6_02a}\quad
\tikzinput{fig_courbes_part6_02b}
}
\myfigure{1}{
\tikzinput{fig_courbes_part6_02}
}

\end{exemple}


%---------------------------------------------------------------
\subsection{Plan d'étude}


\begin{enumerate}
  \item \evidence{Domaine de définition} et réduction du \evidence{domaine d'étude}
en détaillant à chaque fois les transformations géométriques permettant de reconstituer la courbe.

\item \evidence{Passages par l'origine.} On résout l'équation $r(\theta)=0$
et on précise les tangentes en les points correspondants.

\item \evidence{Variations} de la fonction $r$ ainsi que le \evidence{signe} de
la fonction $r$. Ce signe aura une influence sur le tracé
de la courbe (voir plus bas).
Ce signe permet aussi de savoir si l'origine est un point
de rebroussement ou un point ordinaire.

\item \evidence{Tangentes parallèles aux axes.} Recherche éventuelle des points en lesquels
la tangente est
parallèle à un axe de coordonnées (pour une tangente en un point distinct de $O$,
parallèle à $(Ox)$, on résout $\big(r\sin(\theta)\big)'=y'=0$).


\item \evidence{Étude des branches infinies.} Aucun résultat spécifique
ne sera fait ici. Le plus simple est alors de se ramener à l'étude des branches
infinies d'une courbe paramétrée classique :
$\left\{
\begin{array}{l}
x(\theta) = r(\theta)\cos(\theta)\\
y(\theta) = r(\theta)\sin(\theta)
\end{array}
\right.$.

\item \evidence{Construction de la courbe.}

\mybox{
\begin{tabular}{c}
\textbf{Tracé de la courbe d'équation polaire $r=f(\theta)$}\\[3mm]
Si $r$ est positif et croît,\\ on tourne dans le sens direct en s'écartant de l'origine.\\[1mm]
Si $r$ est négatif et décroît,\\ on tourne dans le sens direct en s'écartant de l'origine.\\[1mm]
Si $r$ est positif et décroît,\\ on tourne dans le sens direct en se rapprochant de l'origine.\\[1mm]
Si $r$ est négatif et croît,\\ on tourne dans le sens direct en se rapprochant de l'origine.
\end{tabular}
}

  \item \evidence{Points multiples.} Recherche éventuelle de points multiples si le tracé de la
courbe le suggère (et si les calculs sont simples).

\end{enumerate}


%---------------------------------------------------------------
\subsection{Exemples détaillés}

\begin{exemple}
Construire la \defi{cardioïde}, courbe d'équation polaire
$$r=1-\cos\theta.$$

\medskip
\textbf{Solution.}

\begin{itemize}
  \item \textbf{Domaine d'étude.} La fonction $r$ est $2\pi$-périodique, donc on l'étudie sur $[-\pi,\pi]$,
  mais comme $r$ est une fonction paire, on se limite à l'intervalle $[0,\pi]$, la courbe étant symétrique par
  rapport à l'axe des abscisses.

    \item \textbf{Localisation de la courbe.} Comme $0 \le r \le 2$ alors la courbe est bornée,
    incluse dans le disque de rayon $2$, centré à l'origine. Il n'y a pas de branches infinies.

  \item \textbf{Passage par l'origine.} $r=0 \iff \cos \theta  = 1 \iff \theta = 0$ (toujours avec
  notre restriction $\theta \in [0,\pi]$). La courbe passe par l'origine uniquement pour $\theta = 0$.

  \item \textbf{Variations de $r$.} La fonction $r$ est croissante sur $[0,\pi]$ avec
  $r(0)=0$, $r(\pi) = 2$. Conséquence : $r$ est positif et croît, on tourne dans
  le sens direct en s'écartant de l'origine.

  \item \textbf{Tangentes parallèles aux axes.}
  La représentation paramétrique de la courbe est $x(\theta) = r(\theta) \cos \theta$,
  $y(\theta) = r(\theta) \sin \theta$. La tangente est horizontale
  lorsque $y'(\theta)=0$ (et $x'(\theta)\neq 0$) et verticale lorsque $x'(\theta)=0$
  (et $y'(\theta)\neq0$).
  On calcule
  $$x(\theta) = r(\theta) \cos \theta = \cos \theta - \cos^2\theta \qquad
  x'(\theta) = \sin\theta ( 2\cos\theta -1)$$
  $$x'(\theta) = 0 \iff \theta = 0, \quad \theta = \frac\pi3, \quad \theta = \pi$$
  Puis :
  $$y(\theta) = r(\theta) \sin \theta = \sin \theta - \cos\theta \sin\theta \qquad
  y'(\theta) = -2\cos^2\theta+\cos\theta+1$$
  Or $-2X^2+X+1=0 \iff X=-\frac12 \ \text{ ou } \  X = 1$ donc 
  $$y'(\theta) = 0 \iff \theta = 0, \quad \theta = \frac{2\pi}{3}$$
  En $\theta=0$ les deux dérivées s'annulent, donc on ne peut encore rien dire.
  En $\theta = \frac{2\pi}{3}$ la tangente est horizontale,
  et en $\theta = \frac\pi3$ et $\theta = \pi$ la tangente est verticale.


  \item \textbf{Comportement à l'origine.} À l'origine
  (pour $\theta_0=0$), une équation de la tangente est $y=\tan \theta_0 x$,
  donc ici d'équation $y=0$. Comme $r(\theta)\ge0$,
  il s'agit d'un point de rebroussement.

  \item \textbf{Graphe.} Toutes ces informations permettent de tracer cette courbe polaire.

\end{itemize}



\myfigure{1}{
\tikzinput{fig_courbes_part6_05}
}

\end{exemple}



\begin{exemple}
Le plan est rapporté à un repère orthonormé direct.
Construire la courbe d'équation polaire
$$r=\frac{1+2\sin\theta}{1+2\cos\theta}.$$

\medskip
\textbf{Solution.}

\begin{itemize}
  \item \textbf{Domaine d'étude.}

La fonction $r$ est $2\pi$-périodique. De plus, pour $r\in[-\pi,\pi]$,

$$1+2\cos\theta=0 \iff \left(\theta=-\frac{2\pi}{3}\ \text{ ou }\ \theta=\frac{2\pi}{3}\right).$$

On obtient la courbe complète quand $\theta$ décrit
$D=[-\pi,-\frac{2\pi}{3}[ \cup ]-\frac{2\pi}{3},\frac{2\pi}{3}[ \cup ]\frac{2\pi}{3},\pi]$.

  \item \textbf{Passages par l'origine.}

Pour $\theta\in D$,
$$1+2\sin\theta=0\iff \left(\theta=-\frac{\pi}{6}\ \text{ ou }\ \theta=-\frac{5\pi}{6}\right).$$
En $M(-\frac{\pi}{6})=O$, la tangente est la
droite d'équation $y=\tan(-\frac{\pi}{6})x=-\frac{1}{\sqrt{3}}x$
et en $M(-\frac{5\pi}{6})=O$, la tangente est la droite d'équation
$y=\tan(-\frac{5\pi}{6})x=\frac{1}{\sqrt{3}}x$.

  \item \textbf{Signe et variations de $r$.}

  $r$ est strictement positive sur $]-\frac{5\pi}{6},-\frac{2\pi}{3}[ \  \cup  \ ]-\frac{\pi}{6},\frac{2\pi}{3}[$,
  et strictement négative sur
  $[-\pi,-\frac{5\pi}{6}[ \  \cup \  ]-\frac{2\pi}{3},-\frac{\pi}{6}[ \  \cup  \ ]\frac{2\pi}{3},\pi]$.
Ensuite, $r$ est dérivable sur $D$ et, pour $\theta\in D$,
\begin{eqnarray*}
r'(\theta)
&=&\frac{2\cos\theta(1+2\cos\theta)+2\sin\theta(1+2\sin\theta)}{(1+2\cos\theta)^2}\\
&=&\frac{2(\cos\theta+\sin\theta+2)}{(1+2\cos\theta)^2}
=\frac{2\sqrt{2}(\cos(\theta-\frac{\pi}{4})+\sqrt{2})}{(1+2\cos\theta)^2}>0.
\end{eqnarray*}
Ainsi, $r$ est strictement croissante sur $[-\pi,-\frac{2\pi}{3}[$,
sur $]-\frac{2\pi}{3},\frac{2\pi}{3}[$ et sur $]\frac{2\pi}{3},\pi]$.

  \item \textbf{Étude des branches infinies}.

  Quand $\theta$ tend vers $-\frac{2\pi}{3}$, $|r(\theta)|$ tend vers $+\infty$. Plus précisément,
  \begin{itemize}
    \item $x(\theta)=\frac{(1+2\sin\theta)\cos\theta}{1+2\cos\theta}$
tend vers $\pm\infty$,

    \item et $y(\theta)=\frac{(1+2\sin\theta)\sin\theta}{1+2\cos\theta}$
tend vers $\pm\infty$,

    \item $\frac{y(\theta)}{x(\theta)}=\tan\theta$ tend vers $\tan(-\frac{2\pi}{3})=\sqrt{3}$.
Donc la courbe admet une direction asymptotique d'équation $y=\sqrt{3}x$.
  \end{itemize}


Ensuite,
\begin{align*}
y(\theta)-\sqrt{3} x(\theta)
&=\frac{(1+2\sin\theta)(\sin\theta-\sqrt{3}\cos\theta)}{1+2\cos\theta}\\
&=\frac{(1+2\sin\theta)\big(-2\sin(\theta+\frac{2\pi}{3})\big)}{2\big(\cos\theta-\cos(\frac{2\pi}{3})\big)}\\
&=\frac{(1+2\sin\theta)\big(-4\sin(\frac{\theta}{2}+\frac{\pi}{3})\cos(\frac{\theta}{2}+\frac{\pi}{3})\big)}
 {-4\sin(\frac{\theta}{2}-\frac{\pi}{3})\sin(\frac{\theta}{2}+\frac{\pi}{3})}\\
&=\frac{(1+2\sin\theta)\cos(\frac{\theta}{2}+\frac{\pi}{3})}
 {\sin(\frac{\theta}{2}-\frac{\pi}{3})}
\end{align*}
Quand $\theta$ tend vers $-\frac{2\pi}{3}$, cette dernière
expression tend vers $2(1-\frac{1}{\sqrt{3}})$ et on en déduit que la droite d'équation
$y=\sqrt{3}x+2(1-\frac{1}{\sqrt{3}})$ est asymptote à la courbe.

On trouve de même que, quand $\theta$ tend vers $\frac{2\pi}{3}$,
la droite d'équation $y=-\sqrt{3}x+2(1+\frac{1}{\sqrt{3}})$ est asymptote à la courbe.


  \item \textbf{Graphe.}

\end{itemize}

\myfigure{1}{
\tikzinput{fig_courbes_part6_06}
}

\end{exemple}


\begin{miniexercices}
\sauteligne
\begin{enumerate}
  \item Si la fonction $\theta \mapsto r(\theta)$ est
  $\pi$-périodique, comment limiter l'étude à un intervalle
  de longueur $\pi$ ? Et si en plus la fonction $r$ est impaire ?

  \item Soit la courbe d'équation polaire $r=\cos\theta+\sin\theta$.
  Montrer que l'on peut se limiter à $[-\frac{\pi}{4},\frac{\pi}{4}]$
  comme domaine d'étude.

  \item Étudier la courbe d'équation polaire $r = \sin(2\theta)$.

  \item Étudier la courbe d'équation polaire $r = 1+\tan \frac \theta2$ et en particulier ses
  branches infinies.

\end{enumerate}
\end{miniexercices}



\auteurs{

Jean-Louis Rouget,  \texttt{\href{http://http://www.maths-france.fr/}{maths-france.fr}}

Amendé par Arnaud Bodin

Relu par Stéphanie Bodin et Vianney Combet
}

\finchapitre
\end{document}


