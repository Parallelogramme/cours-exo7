
%%%%%%%%%%%%%%%%%% PREAMBULE %%%%%%%%%%%%%%%%%%

\documentclass[aspectratio=169,utf8]{beamer}
%\documentclass[aspectratio=169,handout]{beamer}

\usetheme{Boadilla}
%\usecolortheme{seahorse}
\usecolortheme[RGB={245,66,24}]{structure}
\useoutertheme{infolines}

% packages
\usepackage{amsfonts,amsmath,amssymb,amsthm}
\usepackage[utf8]{inputenc}
\usepackage[T1]{fontenc}
\usepackage{lmodern}

\usepackage[francais]{babel}
\usepackage{fancybox}
\usepackage{graphicx}

\usepackage{float}
\usepackage{xfrac}

%\usepackage[usenames, x11names]{xcolor}
\usepackage{tikz}
\usepackage{pgfplots}
\usepackage{datetime}



%-----  Package unités -----
\usepackage{siunitx}
\sisetup{locale = FR,detect-all,per-mode = symbol}

%\usepackage{mathptmx}
%\usepackage{fouriernc}
%\usepackage{newcent}
%\usepackage[mathcal,mathbf]{euler}

%\usepackage{palatino}
%\usepackage{newcent}
% \usepackage[mathcal,mathbf]{euler}



% \usepackage{hyperref}
% \hypersetup{colorlinks=true, linkcolor=blue, urlcolor=blue,
% pdftitle={Exo7 - Exercices de mathématiques}, pdfauthor={Exo7}}


%section
% \usepackage{sectsty}
% \allsectionsfont{\bf}
%\sectionfont{\color{Tomato3}\upshape\selectfont}
%\subsectionfont{\color{Tomato4}\upshape\selectfont}

%----- Ensembles : entiers, reels, complexes -----
\newcommand{\Nn}{\mathbb{N}} \newcommand{\N}{\mathbb{N}}
\newcommand{\Zz}{\mathbb{Z}} \newcommand{\Z}{\mathbb{Z}}
\newcommand{\Qq}{\mathbb{Q}} \newcommand{\Q}{\mathbb{Q}}
\newcommand{\Rr}{\mathbb{R}} \newcommand{\R}{\mathbb{R}}
\newcommand{\Cc}{\mathbb{C}} 
\newcommand{\Kk}{\mathbb{K}} \newcommand{\K}{\mathbb{K}}

%----- Modifications de symboles -----
\renewcommand{\epsilon}{\varepsilon}
\renewcommand{\Re}{\mathop{\text{Re}}\nolimits}
\renewcommand{\Im}{\mathop{\text{Im}}\nolimits}
%\newcommand{\llbracket}{\left[\kern-0.15em\left[}
%\newcommand{\rrbracket}{\right]\kern-0.15em\right]}

\renewcommand{\ge}{\geqslant}
\renewcommand{\geq}{\geqslant}
\renewcommand{\le}{\leqslant}
\renewcommand{\leq}{\leqslant}
\renewcommand{\epsilon}{\varepsilon}

%----- Fonctions usuelles -----
\newcommand{\ch}{\mathop{\text{ch}}\nolimits}
\newcommand{\sh}{\mathop{\text{sh}}\nolimits}
\renewcommand{\tanh}{\mathop{\text{th}}\nolimits}
\newcommand{\cotan}{\mathop{\text{cotan}}\nolimits}
\newcommand{\Arcsin}{\mathop{\text{arcsin}}\nolimits}
\newcommand{\Arccos}{\mathop{\text{arccos}}\nolimits}
\newcommand{\Arctan}{\mathop{\text{arctan}}\nolimits}
\newcommand{\Argsh}{\mathop{\text{argsh}}\nolimits}
\newcommand{\Argch}{\mathop{\text{argch}}\nolimits}
\newcommand{\Argth}{\mathop{\text{argth}}\nolimits}
\newcommand{\pgcd}{\mathop{\text{pgcd}}\nolimits} 


%----- Commandes divers ------
\newcommand{\ii}{\mathrm{i}}
\newcommand{\dd}{\text{d}}
\newcommand{\id}{\mathop{\text{id}}\nolimits}
\newcommand{\Ker}{\mathop{\text{Ker}}\nolimits}
\newcommand{\Card}{\mathop{\text{Card}}\nolimits}
\newcommand{\Vect}{\mathop{\text{Vect}}\nolimits}
\newcommand{\Mat}{\mathop{\text{Mat}}\nolimits}
\newcommand{\rg}{\mathop{\text{rg}}\nolimits}
\newcommand{\tr}{\mathop{\text{tr}}\nolimits}


%----- Structure des exercices ------

\newtheoremstyle{styleexo}% name
{2ex}% Space above
{3ex}% Space below
{}% Body font
{}% Indent amount 1
{\bfseries} % Theorem head font
{}% Punctuation after theorem head
{\newline}% Space after theorem head 2
{}% Theorem head spec (can be left empty, meaning ‘normal’)

%\theoremstyle{styleexo}
\newtheorem{exo}{Exercice}
\newtheorem{ind}{Indications}
\newtheorem{cor}{Correction}


\newcommand{\exercice}[1]{} \newcommand{\finexercice}{}
%\newcommand{\exercice}[1]{{\tiny\texttt{#1}}\vspace{-2ex}} % pour afficher le numero absolu, l'auteur...
\newcommand{\enonce}{\begin{exo}} \newcommand{\finenonce}{\end{exo}}
\newcommand{\indication}{\begin{ind}} \newcommand{\finindication}{\end{ind}}
\newcommand{\correction}{\begin{cor}} \newcommand{\fincorrection}{\end{cor}}

\newcommand{\noindication}{\stepcounter{ind}}
\newcommand{\nocorrection}{\stepcounter{cor}}

\newcommand{\fiche}[1]{} \newcommand{\finfiche}{}
\newcommand{\titre}[1]{\centerline{\large \bf #1}}
\newcommand{\addcommand}[1]{}
\newcommand{\video}[1]{}

% Marge
\newcommand{\mymargin}[1]{\marginpar{{\small #1}}}

\def\noqed{\renewcommand{\qedsymbol}{}}


%----- Presentation ------
\setlength{\parindent}{0cm}

%\newcommand{\ExoSept}{\href{http://exo7.emath.fr}{\textbf{\textsf{Exo7}}}}

\definecolor{myred}{rgb}{0.93,0.26,0}
\definecolor{myorange}{rgb}{0.97,0.58,0}
\definecolor{myyellow}{rgb}{1,0.86,0}

\newcommand{\LogoExoSept}[1]{  % input : echelle
{\usefont{U}{cmss}{bx}{n}
\begin{tikzpicture}[scale=0.1*#1,transform shape]
  \fill[color=myorange] (0,0)--(4,0)--(4,-4)--(0,-4)--cycle;
  \fill[color=myred] (0,0)--(0,3)--(-3,3)--(-3,0)--cycle;
  \fill[color=myyellow] (4,0)--(7,4)--(3,7)--(0,3)--cycle;
  \node[scale=5] at (3.5,3.5) {Exo7};
\end{tikzpicture}}
}


\newcommand{\debutmontitre}{
  \author{} \date{} 
  \thispagestyle{empty}
  \hspace*{-10ex}
  \begin{minipage}{\textwidth}
    \titlepage  
  \vspace*{-2.5cm}
  \begin{center}
    \LogoExoSept{2.5}
  \end{center}
  \end{minipage}

  \vspace*{-0cm}
  
  % Astuce pour que le background ne soit pas discrétisé lors de la conversion pdf -> png
\begin{tikzpicture}
        \fill[opacity=0,green!60!black] (0,0)--++(0,0)--++(0,0)--++(0,0)--cycle; 
\end{tikzpicture}

% toc S'affiche trop tot :
% \tableofcontents[hideallsubsections, pausesections]
}

\newcommand{\finmontitre}{
  \end{frame}
  \setcounter{framenumber}{0}
} % ne marche pas pour une raison obscure

%----- Commandes supplementaires ------

% \usepackage[landscape]{geometry}
% \geometry{top=1cm, bottom=3cm, left=2cm, right=10cm, marginparsep=1cm
% }
% \usepackage[a4paper]{geometry}
% \geometry{top=2cm, bottom=2cm, left=2cm, right=2cm, marginparsep=1cm
% }

%\usepackage{standalone}


% New command Arnaud -- november 2011
\setbeamersize{text margin left=24ex}
% si vous modifier cette valeur il faut aussi
% modifier le decalage du titre pour compenser
% (ex : ici =+10ex, titre =-5ex

\theoremstyle{definition}
%\newtheorem{proposition}{Proposition}
%\newtheorem{exemple}{Exemple}
%\newtheorem{theoreme}{Théorème}
%\newtheorem{lemme}{Lemme}
%\newtheorem{corollaire}{Corollaire}
%\newtheorem*{remarque*}{Remarque}
%\newtheorem*{miniexercice}{Mini-exercices}
%\newtheorem{definition}{Définition}

% Commande tikz
\usetikzlibrary{calc}
\usetikzlibrary{patterns,arrows}
\usetikzlibrary{matrix}
\usetikzlibrary{fadings} 

%definition d'un terme
\newcommand{\defi}[1]{{\color{myorange}\textbf{\emph{#1}}}}
\newcommand{\evidence}[1]{{\color{blue}\textbf{\emph{#1}}}}
\newcommand{\assertion}[1]{\emph{\og#1\fg}}  % pour chapitre logique
%\renewcommand{\contentsname}{Sommaire}
\renewcommand{\contentsname}{}
\setcounter{tocdepth}{2}



%------ Figures ------

\def\myscale{1} % par défaut 
\newcommand{\myfigure}[2]{  % entrée : echelle, fichier figure
\def\myscale{#1}
\begin{center}
\footnotesize
{#2}
\end{center}}


%------ Encadrement ------

\usepackage{fancybox}


\newcommand{\mybox}[1]{
\setlength{\fboxsep}{7pt}
\begin{center}
\shadowbox{#1}
\end{center}}

\newcommand{\myboxinline}[1]{
\setlength{\fboxsep}{5pt}
\raisebox{-10pt}{
\shadowbox{#1}
}
}

%--------------- Commande beamer---------------
\newcommand{\beameronly}[1]{#1} % permet de mettre des pause dans beamer pas dans poly


\setbeamertemplate{navigation symbols}{}
\setbeamertemplate{footline}  % tiré du fichier beamerouterinfolines.sty
{
  \leavevmode%
  \hbox{%
  \begin{beamercolorbox}[wd=.333333\paperwidth,ht=2.25ex,dp=1ex,center]{author in head/foot}%
    % \usebeamerfont{author in head/foot}\insertshortauthor%~~(\insertshortinstitute)
    \usebeamerfont{section in head/foot}{\bf\insertshorttitle}
  \end{beamercolorbox}%
  \begin{beamercolorbox}[wd=.333333\paperwidth,ht=2.25ex,dp=1ex,center]{title in head/foot}%
    \usebeamerfont{section in head/foot}{\bf\insertsectionhead}
  \end{beamercolorbox}%
  \begin{beamercolorbox}[wd=.333333\paperwidth,ht=2.25ex,dp=1ex,right]{date in head/foot}%
    % \usebeamerfont{date in head/foot}\insertshortdate{}\hspace*{2em}
    \insertframenumber{} / \inserttotalframenumber\hspace*{2ex} 
  \end{beamercolorbox}}%
  \vskip0pt%
}


\definecolor{mygrey}{rgb}{0.5,0.5,0.5}
\setlength{\parindent}{0cm}
%\DeclareTextFontCommand{\helvetica}{\fontfamily{phv}\selectfont}

% background beamer
\definecolor{couleurhaut}{rgb}{0.85,0.9,1}  % creme
\definecolor{couleurmilieu}{rgb}{1,1,1}  % vert pale
\definecolor{couleurbas}{rgb}{0.85,0.9,1}  % blanc
\setbeamertemplate{background canvas}[vertical shading]%
[top=couleurhaut,middle=couleurmilieu,midpoint=0.4,bottom=couleurbas] 
%[top=fondtitre!05,bottom=fondtitre!60]



\makeatletter
\setbeamertemplate{theorem begin}
{%
  \begin{\inserttheoremblockenv}
  {%
    \inserttheoremheadfont
    \inserttheoremname
    \inserttheoremnumber
    \ifx\inserttheoremaddition\@empty\else\ (\inserttheoremaddition)\fi%
    \inserttheorempunctuation
  }%
}
\setbeamertemplate{theorem end}{\end{\inserttheoremblockenv}}

\newenvironment{theoreme}[1][]{%
   \setbeamercolor{block title}{fg=structure,bg=structure!40}
   \setbeamercolor{block body}{fg=black,bg=structure!10}
   \begin{block}{{\bf Th\'eor\`eme }#1}
}{%
   \end{block}%
}


\newenvironment{proposition}[1][]{%
   \setbeamercolor{block title}{fg=structure,bg=structure!40}
   \setbeamercolor{block body}{fg=black,bg=structure!10}
   \begin{block}{{\bf Proposition }#1}
}{%
   \end{block}%
}

\newenvironment{corollaire}[1][]{%
   \setbeamercolor{block title}{fg=structure,bg=structure!40}
   \setbeamercolor{block body}{fg=black,bg=structure!10}
   \begin{block}{{\bf Corollaire }#1}
}{%
   \end{block}%
}

\newenvironment{mydefinition}[1][]{%
   \setbeamercolor{block title}{fg=structure,bg=structure!40}
   \setbeamercolor{block body}{fg=black,bg=structure!10}
   \begin{block}{{\bf Définition} #1}
}{%
   \end{block}%
}

\newenvironment{lemme}[0]{%
   \setbeamercolor{block title}{fg=structure,bg=structure!40}
   \setbeamercolor{block body}{fg=black,bg=structure!10}
   \begin{block}{\bf Lemme}
}{%
   \end{block}%
}

\newenvironment{remarque}[1][]{%
   \setbeamercolor{block title}{fg=black,bg=structure!20}
   \setbeamercolor{block body}{fg=black,bg=structure!5}
   \begin{block}{Remarque #1}
}{%
   \end{block}%
}


\newenvironment{exemple}[1][]{%
   \setbeamercolor{block title}{fg=black,bg=structure!20}
   \setbeamercolor{block body}{fg=black,bg=structure!5}
   \begin{block}{{\bf Exemple }#1}
}{%
   \end{block}%
}


\newenvironment{miniexercice}[0]{%
   \setbeamercolor{block title}{fg=structure,bg=structure!20}
   \setbeamercolor{block body}{fg=black,bg=structure!5}
   \begin{block}{Mini-exercices}
}{%
   \end{block}%
}


\newenvironment{tp}[0]{%
   \setbeamercolor{block title}{fg=structure,bg=structure!40}
   \setbeamercolor{block body}{fg=black,bg=structure!10}
   \begin{block}{\bf Travaux pratiques}
}{%
   \end{block}%
}
\newenvironment{exercicecours}[1][]{%
   \setbeamercolor{block title}{fg=structure,bg=structure!40}
   \setbeamercolor{block body}{fg=black,bg=structure!10}
   \begin{block}{{\bf Exercice }#1}
}{%
   \end{block}%
}
\newenvironment{algo}[1][]{%
   \setbeamercolor{block title}{fg=structure,bg=structure!40}
   \setbeamercolor{block body}{fg=black,bg=structure!10}
   \begin{block}{{\bf Algorithme}\hfill{\color{gray}\texttt{#1}}}
}{%
   \end{block}%
}


\setbeamertemplate{proof begin}{
   \setbeamercolor{block title}{fg=black,bg=structure!20}
   \setbeamercolor{block body}{fg=black,bg=structure!5}
   \begin{block}{{\footnotesize Démonstration}}
   \footnotesize
   \smallskip}
\setbeamertemplate{proof end}{%
   \end{block}}
\setbeamertemplate{qed symbol}{\openbox}


\makeatother
\usecolortheme[RGB={102,102,0}]{structure}

   
%%%%%%%%%%%%%%%%%%%%%%%%%%%%%%%%%%%%%%%%%%%%%%%%%%%%%%%%%%%%%
%%%%%%%%%%%%%%%%%%%%%%%%%%%%%%%%%%%%%%%%%%%%%%%%%%%%%%%%%%%%%


\begin{document}


\title{{\bf Courbes paramétrées}}
\subtitle{Notions de base}

\begin{frame}
  
  \debutmontitre

  \pause

{\footnotesize
\hfill
\setbeamercovered{transparent=50}
\begin{minipage}{0.6\textwidth}
  \begin{itemize}
    \item<3-> La cycloïde
    \item<4-> Définition
    \item<5-> Réduction du domaine d'étude
    \item<6-> Points simples, points multiples   
  \end{itemize}
\end{minipage}
}

\end{frame}

\setcounter{framenumber}{0}



%%%%%%%%%%%%%%%%%%%%%%%%%%%%%%%%%%%%%%%%%%%%%%%%%%%%%%%%%%%%%%%%
\section{La cycloïde}

\begin{frame}

\uncover<6->{
$$\left\{\begin{array}{rcl}
x(t) &=& r(t-\sin t) \\           
y(t) &=& r(1-\cos t)
\end{array} \right.$$
}

\myfigure{0.8}{
\tikzinput{fig_courbes_part0_01}
} 

\end{frame}


\begin{frame}
\myfigure{1.6}{
\tikzinput{fig_courbes_part0_03}
} 

\end{frame}



%%%%%%%%%%%%%%%%%%%%%%%%%%%%%%%%%%%%%%%%%%%%%%%%%%%%%%%%%%%%%%%%
\section{Définition}

\begin{frame}

\begin{mydefinition}
Une \defi{courbe paramétrée plane} est une application 
$$\begin{array}[t]{cccc}
f :&D\subset\Rr&\rightarrow&\Rr^2\\
 &t&\mapsto&f(t)
\end{array}$$ d'un sous-ensemble $D$ de $\Rr$ dans $\Rr^2$ 
\end{mydefinition}

\pause

\myfigure{0.8}{
\tikzinput{fig_courbes_part1_00}
}

\end{frame}

\begin{frame}

\begin{exemple}
\begin{itemize}
\item $t\mapsto (\cos t,\sin t)$, $t\in[0,2\pi[$ : une paramétrisation du cercle trigonométrique

\uncover<2->{
\item $t\mapsto(2t-3,3t+1)$, $t\in\Rr$ : une paramétrisation de la droite 
passant par le point $A(-3,1)$ et de vecteur directeur $\vec{u}(2,3)$
}

\end{itemize}
\vspace*{-2ex}
\myfigure{0.7}{
\tikzinput{fig_courbes_part1_01a} \qquad\qquad
\uncover<2->{
\tikzinput{fig_courbes_part1_01b}
}
}
\vspace*{-3ex}
\end{exemple}


\end{frame}

\begin{frame}

\begin{exemple}
\begin{itemize}	
\item $\lambda\mapsto\big((1-\lambda)x_A+\lambda x_B,(1-\lambda)y_A+\lambda y_B\big)$, 
$\lambda\in[0,1]$ : une paramétrisation du segment $[AB]$

\uncover<2->{
\item Graphe d'une fonction $f : D \subset \Rr \to \Rr$ : 
$\left\{
\begin{array}{l}
x(t)=t\\
y(t)=f(t)
\end{array}
\right.$
}
\end{itemize} 
\vspace*{-2ex}
\myfigure{0.6}{
\tikzinput{fig_courbes_part1_01c} \qquad\qquad
\uncover<2->{
\tikzinput{fig_courbes_part1_01d} 
}}


\end{exemple}
\end{frame}


\begin{frame}

\begin{mydefinition}
Le \defi{support d'une courbe paramétrée} $\begin{array}[t]{cccc}
f :&D\subset\Rr&\rightarrow&\Rr^2\\
 &t&\mapsto&f(t)
\end{array}$ est l'ensemble des points $M(t)$ où $t$ décrit $D$
\end{mydefinition}

\pause	

\myfigure{0.7}{
\tikzinput{fig_courbes_part1_02} 
\uncover<3->{\tikzinput{fig_courbes_part1_03}}
}
$$\begin{array}[t]{ccc}
[0,2\pi[&\longrightarrow&\Rr^2\\
t&\longmapsto&(\cos t,\sin t)
\end{array} \qquad \qquad 
\uncover<3->{
\begin{array}[t]{ccc}
\Rr &\longrightarrow&\Rr^2\\
t&\longmapsto& \left(\frac{1-t^2}{1+t^2},\frac{2t}{1+t^2}\right)
\end{array}
}$$ 

\end{frame}

%%%%%%%%%%%%%%%%%%%%%%%%%%%%%%%%%%%%%%%%%%%%%%%%%%%%%%%%%%%%%%%%
\section{Réduction du domaine d'étude}

\begin{frame}


\begin{itemize}
  \item Translation de vecteur $\vec{u}(a,b)$ : $t_{\vec{u}}(M)=(x+a,y+b)$
\uncover<3->{  \item Réflexion d'axe $(Ox)$ : $s_{(Ox)}(M)=(x,-y)$}
\uncover<5->{  \item Réflexion d'axe $(Oy)$ : $s_{(Oy)}(M)=(-x,y)$}
\uncover<6->{  \item Réflexion d'axe la droite $(D)$ d'équation $y=x$ : $s_D(M)=(y,x)$}
\end{itemize}


\myfigure{0.65}{
\uncover<2->{\tikzinput{fig_courbes_part1_04a} }
\uncover<4->{\tikzinput{fig_courbes_part1_04b} }
}

\end{frame}


\begin{frame}

\begin{itemize}
  \item Symétrie centrale de centre $O$ : $s_O(M)=(-x,-y)$
\uncover<3->{  \item Rotation d'angle $\frac{\pi}{2}$ autour de $O$ : $\text{rot}_{O,\pi/2}(M)=(-y,x)$}
\end{itemize}


\myfigure{0.5}{
\uncover<2->{\tikzinput{fig_courbes_part1_04c} }
\uncover<4->{\tikzinput{fig_courbes_part1_04d} }
}

\end{frame}

\begin{frame}
\begin{exemple}
Déterminer un domaine d'étude simple de l'arc 
$\left\{
\begin{array}{l}
x(t)=t-\frac32\sin t\\
y(t)=1-\frac32\cos t
\end{array}
\right.$
\pause

\vspace*{-1ex}
\textbf{Solution}
\vspace*{-1ex}
{\small
\begin{itemize}
  \item $M(t+2\pi)\pause=(t-\tfrac32\sin t,1-\tfrac32\cos t)+(2\pi,0)\pause=t_{\vec{u}} \big(M(t)\big)$
où $\vec{u}=\!(2\pi,0)$
  \pause
  \item Pour $t\in[-\pi,\pi]$,\pause 
$M(-t)\pause=\big(-(t-\tfrac32\sin t),1-\tfrac32\cos t \big)\pause=s_{(Oy)}\big(M(t)\big)$
  \pause
  \item \'Etude sur $[0,\pi]$\pause
\end{itemize}
}
\vspace*{-4ex}
\myfigure{0.9}{
\tikzinput{fig_courbes_part1_05a} \pause
\tikzinput{fig_courbes_part1_05b}
} 
\vspace*{-5ex}\pause
\myfigure{0.9}{
\tikzinput{fig_courbes_part1_05}
}
\vspace*{-5ex}
\end{exemple}
\end{frame}


\begin{frame}
\begin{exemple}
Domaine d'étude de \defi{courbe de Lissajous} 
$\left\{
\begin{array}{l}
x(t)=\sin(2t)\\
y(t)=\sin(3t)
\end{array}
\right.$
\pause

\vspace*{-1ex}

%\medskip
\textbf{Solution}

\vspace*{-1ex}
\begin{itemize}
  \item Pour $t\in\Rr$, $M(t+2\pi)=M(t)$
\pause
  \item Pour $t\in[-\pi,\pi]$, \pause $M(-t)\pause= \big(-\sin(2t),-\sin(3t)\big)\pause=s_O\big(M(t)\big)$
\pause
  \item Pour $t\in[0,\pi]$, \pause 
$M(\pi-t)\pause=\big(\sin(2\pi-2t),\sin(3\pi-3t)\big)\pause=\big(\sin(-2t),\sin(\pi-3t)\big)
\pause=\big(-\sin(2t),\sin(3t)\big)\pause=s_{(Oy)}\big(M(t)\big)$
\pause 
  \item \'Etude sur $[0,\frac\pi2]$
\end{itemize}

\pause
\vspace*{-3ex}

\myfigure{0.8}{
\tikzinput{fig_courbes_part1_06a} \qquad 
\pause
\tikzinput{fig_courbes_part1_06b} \qquad 
\pause
\tikzinput{fig_courbes_part1_06}}
\end{exemple}

\end{frame}



%%%%%%%%%%%%%%%%%%%%%%%%%%%%%%%%%%%%%%%%%%%%%%%%%%%%%%%%%%%%%%%%
\section{Points simples, points multiples}

\begin{frame}

Soit $f : t\mapsto M(t)$ une courbe paramétrée et soit $A$ un point du plan
\begin{mydefinition}
La \defi{multiplicité} du point $A$ par rapport à la courbe $f$ est le nombre 
de réels $t$ pour lesquels $M(t)=A$
\end{mydefinition}

\pause

\myfigure{0.8}{
\tikzinput{fig_courbes_part1_09a} \qquad 
\pause
\tikzinput{fig_courbes_part1_09b} \qquad 
\pause
\tikzinput{fig_courbes_part1_09c}
}

\end{frame}


\begin{frame}
\begin{exemple}
Trouver les points multiples de 
$\left\{
\begin{array}{l}
x(t)=2t+t^2\\
y(t)=2t-\frac{1}{t^2}
\end{array}
\right.$ \quad $t\in\Rr^*$

\pause

\medskip
\textbf{Solution}

\vspace*{-0.5cm}

\hfill\begin{minipage}{0.3\textwidth}
\myfigure{0.5}{
\tikzinput{fig_courbes_part1_10}
}  
\end{minipage}

\pause
\vspace*{-3.5cm}

\begin{itemize}
  \item Soit $(t,u)\in(\Rr^*)^2$ tel que $t>u$
  
  \pause
  \item $
M(t)=M(u)\pause\Leftrightarrow\left\{
\begin{array}{l}
2t+t^2=2u+u^2\\
2t-\frac{1}{t^2}=2u-\frac{1}{u^2}
\end{array}
\right.$
  \pause
  \item $
M(t)=M(u)\Leftrightarrow\left\{
\begin{array}{l}
(t^2-u^2)+2(t-u)=0\\
2(t-u)-\big(\frac{1}{t^2}-\frac{1}{u^2}\big)=0
\end{array}
\right.$
  \pause
  \item \ldots
  \pause
  \item $M(t)=M(u)\Leftrightarrow t=-1+\sqrt{2}\quad\text{et}\quad u=-1-\sqrt{2}$
  \pause
  \item $M(-1+\sqrt{2})=M(-1-\sqrt{2})=(1,-5)$
\end{itemize}
\end{exemple}

\end{frame}




%%%%%%%%%%%%%%%%%%%%%%%%%%%%%%%%%%%%%%%%%%%%%%%%%%%%%%%%%%%%%%%%
\section{Mini-exercices}

\begin{frame}
\begin{miniexercice}
\begin{enumerate}
  \item Représenter graphiquement chacune des transformations du plan 
  qui servent à réduire l'intervalle d'étude. 
  
  \item Pour la courbe de Lissajous définie par $x(t)=\sin(2t)$ et
  $y(t)=\sin(3t)$, montrer que la courbe est symétrique par rapport à
  l'axe $(Ox)$. Exprimer cette symétrie en fonction de celles déjà trouvées :
  $s_O$ et $s_{(Oy)}$.
    
  \item Trouver les symétries et les points multiples de la courbe définie par 
  $x(t) = \frac{1-t^2}{1+t^2}$ et $y(t) = t \frac{1-t^2}{1+t^2}$.
  
  \item Trouver un intervalle d'étude pour l'astroïde définie par $x(t) = \cos^3 t$,
  $y(t) = \sin^3 t$.  
  
  \item Trouver un intervalle d'étude pour la cycloïde définie par $x(t) = r(t-\sin t)$,
  $y(t) = r(1-\cos t)$. Montrer que la cycloïde n'a pas de points multiples.

\end{enumerate}
\end{miniexercice}
\end{frame}

\end{document}
