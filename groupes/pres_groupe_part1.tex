
%%%%%%%%%%%%%%%%%% PREAMBULE %%%%%%%%%%%%%%%%%%

\documentclass[aspectratio=169,utf8]{beamer}
%\documentclass[aspectratio=169,handout]{beamer}

\usetheme{Boadilla}
%\usecolortheme{seahorse}
\usecolortheme[RGB={245,66,24}]{structure}
\useoutertheme{infolines}

% packages
\usepackage{amsfonts,amsmath,amssymb,amsthm}
\usepackage[utf8]{inputenc}
\usepackage[T1]{fontenc}
\usepackage{lmodern}

\usepackage[francais]{babel}
\usepackage{fancybox}
\usepackage{graphicx}

\usepackage{float}
\usepackage{xfrac}

%\usepackage[usenames, x11names]{xcolor}
\usepackage{tikz}
\usepackage{pgfplots}
\usepackage{datetime}



%-----  Package unités -----
\usepackage{siunitx}
\sisetup{locale = FR,detect-all,per-mode = symbol}

%\usepackage{mathptmx}
%\usepackage{fouriernc}
%\usepackage{newcent}
%\usepackage[mathcal,mathbf]{euler}

%\usepackage{palatino}
%\usepackage{newcent}
% \usepackage[mathcal,mathbf]{euler}



% \usepackage{hyperref}
% \hypersetup{colorlinks=true, linkcolor=blue, urlcolor=blue,
% pdftitle={Exo7 - Exercices de mathématiques}, pdfauthor={Exo7}}


%section
% \usepackage{sectsty}
% \allsectionsfont{\bf}
%\sectionfont{\color{Tomato3}\upshape\selectfont}
%\subsectionfont{\color{Tomato4}\upshape\selectfont}

%----- Ensembles : entiers, reels, complexes -----
\newcommand{\Nn}{\mathbb{N}} \newcommand{\N}{\mathbb{N}}
\newcommand{\Zz}{\mathbb{Z}} \newcommand{\Z}{\mathbb{Z}}
\newcommand{\Qq}{\mathbb{Q}} \newcommand{\Q}{\mathbb{Q}}
\newcommand{\Rr}{\mathbb{R}} \newcommand{\R}{\mathbb{R}}
\newcommand{\Cc}{\mathbb{C}} 
\newcommand{\Kk}{\mathbb{K}} \newcommand{\K}{\mathbb{K}}

%----- Modifications de symboles -----
\renewcommand{\epsilon}{\varepsilon}
\renewcommand{\Re}{\mathop{\text{Re}}\nolimits}
\renewcommand{\Im}{\mathop{\text{Im}}\nolimits}
%\newcommand{\llbracket}{\left[\kern-0.15em\left[}
%\newcommand{\rrbracket}{\right]\kern-0.15em\right]}

\renewcommand{\ge}{\geqslant}
\renewcommand{\geq}{\geqslant}
\renewcommand{\le}{\leqslant}
\renewcommand{\leq}{\leqslant}
\renewcommand{\epsilon}{\varepsilon}

%----- Fonctions usuelles -----
\newcommand{\ch}{\mathop{\text{ch}}\nolimits}
\newcommand{\sh}{\mathop{\text{sh}}\nolimits}
\renewcommand{\tanh}{\mathop{\text{th}}\nolimits}
\newcommand{\cotan}{\mathop{\text{cotan}}\nolimits}
\newcommand{\Arcsin}{\mathop{\text{arcsin}}\nolimits}
\newcommand{\Arccos}{\mathop{\text{arccos}}\nolimits}
\newcommand{\Arctan}{\mathop{\text{arctan}}\nolimits}
\newcommand{\Argsh}{\mathop{\text{argsh}}\nolimits}
\newcommand{\Argch}{\mathop{\text{argch}}\nolimits}
\newcommand{\Argth}{\mathop{\text{argth}}\nolimits}
\newcommand{\pgcd}{\mathop{\text{pgcd}}\nolimits} 


%----- Commandes divers ------
\newcommand{\ii}{\mathrm{i}}
\newcommand{\dd}{\text{d}}
\newcommand{\id}{\mathop{\text{id}}\nolimits}
\newcommand{\Ker}{\mathop{\text{Ker}}\nolimits}
\newcommand{\Card}{\mathop{\text{Card}}\nolimits}
\newcommand{\Vect}{\mathop{\text{Vect}}\nolimits}
\newcommand{\Mat}{\mathop{\text{Mat}}\nolimits}
\newcommand{\rg}{\mathop{\text{rg}}\nolimits}
\newcommand{\tr}{\mathop{\text{tr}}\nolimits}


%----- Structure des exercices ------

\newtheoremstyle{styleexo}% name
{2ex}% Space above
{3ex}% Space below
{}% Body font
{}% Indent amount 1
{\bfseries} % Theorem head font
{}% Punctuation after theorem head
{\newline}% Space after theorem head 2
{}% Theorem head spec (can be left empty, meaning ‘normal’)

%\theoremstyle{styleexo}
\newtheorem{exo}{Exercice}
\newtheorem{ind}{Indications}
\newtheorem{cor}{Correction}


\newcommand{\exercice}[1]{} \newcommand{\finexercice}{}
%\newcommand{\exercice}[1]{{\tiny\texttt{#1}}\vspace{-2ex}} % pour afficher le numero absolu, l'auteur...
\newcommand{\enonce}{\begin{exo}} \newcommand{\finenonce}{\end{exo}}
\newcommand{\indication}{\begin{ind}} \newcommand{\finindication}{\end{ind}}
\newcommand{\correction}{\begin{cor}} \newcommand{\fincorrection}{\end{cor}}

\newcommand{\noindication}{\stepcounter{ind}}
\newcommand{\nocorrection}{\stepcounter{cor}}

\newcommand{\fiche}[1]{} \newcommand{\finfiche}{}
\newcommand{\titre}[1]{\centerline{\large \bf #1}}
\newcommand{\addcommand}[1]{}
\newcommand{\video}[1]{}

% Marge
\newcommand{\mymargin}[1]{\marginpar{{\small #1}}}

\def\noqed{\renewcommand{\qedsymbol}{}}


%----- Presentation ------
\setlength{\parindent}{0cm}

%\newcommand{\ExoSept}{\href{http://exo7.emath.fr}{\textbf{\textsf{Exo7}}}}

\definecolor{myred}{rgb}{0.93,0.26,0}
\definecolor{myorange}{rgb}{0.97,0.58,0}
\definecolor{myyellow}{rgb}{1,0.86,0}

\newcommand{\LogoExoSept}[1]{  % input : echelle
{\usefont{U}{cmss}{bx}{n}
\begin{tikzpicture}[scale=0.1*#1,transform shape]
  \fill[color=myorange] (0,0)--(4,0)--(4,-4)--(0,-4)--cycle;
  \fill[color=myred] (0,0)--(0,3)--(-3,3)--(-3,0)--cycle;
  \fill[color=myyellow] (4,0)--(7,4)--(3,7)--(0,3)--cycle;
  \node[scale=5] at (3.5,3.5) {Exo7};
\end{tikzpicture}}
}


\newcommand{\debutmontitre}{
  \author{} \date{} 
  \thispagestyle{empty}
  \hspace*{-10ex}
  \begin{minipage}{\textwidth}
    \titlepage  
  \vspace*{-2.5cm}
  \begin{center}
    \LogoExoSept{2.5}
  \end{center}
  \end{minipage}

  \vspace*{-0cm}
  
  % Astuce pour que le background ne soit pas discrétisé lors de la conversion pdf -> png
\begin{tikzpicture}
        \fill[opacity=0,green!60!black] (0,0)--++(0,0)--++(0,0)--++(0,0)--cycle; 
\end{tikzpicture}

% toc S'affiche trop tot :
% \tableofcontents[hideallsubsections, pausesections]
}

\newcommand{\finmontitre}{
  \end{frame}
  \setcounter{framenumber}{0}
} % ne marche pas pour une raison obscure

%----- Commandes supplementaires ------

% \usepackage[landscape]{geometry}
% \geometry{top=1cm, bottom=3cm, left=2cm, right=10cm, marginparsep=1cm
% }
% \usepackage[a4paper]{geometry}
% \geometry{top=2cm, bottom=2cm, left=2cm, right=2cm, marginparsep=1cm
% }

%\usepackage{standalone}


% New command Arnaud -- november 2011
\setbeamersize{text margin left=24ex}
% si vous modifier cette valeur il faut aussi
% modifier le decalage du titre pour compenser
% (ex : ici =+10ex, titre =-5ex

\theoremstyle{definition}
%\newtheorem{proposition}{Proposition}
%\newtheorem{exemple}{Exemple}
%\newtheorem{theoreme}{Théorème}
%\newtheorem{lemme}{Lemme}
%\newtheorem{corollaire}{Corollaire}
%\newtheorem*{remarque*}{Remarque}
%\newtheorem*{miniexercice}{Mini-exercices}
%\newtheorem{definition}{Définition}

% Commande tikz
\usetikzlibrary{calc}
\usetikzlibrary{patterns,arrows}
\usetikzlibrary{matrix}
\usetikzlibrary{fadings} 

%definition d'un terme
\newcommand{\defi}[1]{{\color{myorange}\textbf{\emph{#1}}}}
\newcommand{\evidence}[1]{{\color{blue}\textbf{\emph{#1}}}}
\newcommand{\assertion}[1]{\emph{\og#1\fg}}  % pour chapitre logique
%\renewcommand{\contentsname}{Sommaire}
\renewcommand{\contentsname}{}
\setcounter{tocdepth}{2}



%------ Figures ------

\def\myscale{1} % par défaut 
\newcommand{\myfigure}[2]{  % entrée : echelle, fichier figure
\def\myscale{#1}
\begin{center}
\footnotesize
{#2}
\end{center}}


%------ Encadrement ------

\usepackage{fancybox}


\newcommand{\mybox}[1]{
\setlength{\fboxsep}{7pt}
\begin{center}
\shadowbox{#1}
\end{center}}

\newcommand{\myboxinline}[1]{
\setlength{\fboxsep}{5pt}
\raisebox{-10pt}{
\shadowbox{#1}
}
}

%--------------- Commande beamer---------------
\newcommand{\beameronly}[1]{#1} % permet de mettre des pause dans beamer pas dans poly


\setbeamertemplate{navigation symbols}{}
\setbeamertemplate{footline}  % tiré du fichier beamerouterinfolines.sty
{
  \leavevmode%
  \hbox{%
  \begin{beamercolorbox}[wd=.333333\paperwidth,ht=2.25ex,dp=1ex,center]{author in head/foot}%
    % \usebeamerfont{author in head/foot}\insertshortauthor%~~(\insertshortinstitute)
    \usebeamerfont{section in head/foot}{\bf\insertshorttitle}
  \end{beamercolorbox}%
  \begin{beamercolorbox}[wd=.333333\paperwidth,ht=2.25ex,dp=1ex,center]{title in head/foot}%
    \usebeamerfont{section in head/foot}{\bf\insertsectionhead}
  \end{beamercolorbox}%
  \begin{beamercolorbox}[wd=.333333\paperwidth,ht=2.25ex,dp=1ex,right]{date in head/foot}%
    % \usebeamerfont{date in head/foot}\insertshortdate{}\hspace*{2em}
    \insertframenumber{} / \inserttotalframenumber\hspace*{2ex} 
  \end{beamercolorbox}}%
  \vskip0pt%
}


\definecolor{mygrey}{rgb}{0.5,0.5,0.5}
\setlength{\parindent}{0cm}
%\DeclareTextFontCommand{\helvetica}{\fontfamily{phv}\selectfont}

% background beamer
\definecolor{couleurhaut}{rgb}{0.85,0.9,1}  % creme
\definecolor{couleurmilieu}{rgb}{1,1,1}  % vert pale
\definecolor{couleurbas}{rgb}{0.85,0.9,1}  % blanc
\setbeamertemplate{background canvas}[vertical shading]%
[top=couleurhaut,middle=couleurmilieu,midpoint=0.4,bottom=couleurbas] 
%[top=fondtitre!05,bottom=fondtitre!60]



\makeatletter
\setbeamertemplate{theorem begin}
{%
  \begin{\inserttheoremblockenv}
  {%
    \inserttheoremheadfont
    \inserttheoremname
    \inserttheoremnumber
    \ifx\inserttheoremaddition\@empty\else\ (\inserttheoremaddition)\fi%
    \inserttheorempunctuation
  }%
}
\setbeamertemplate{theorem end}{\end{\inserttheoremblockenv}}

\newenvironment{theoreme}[1][]{%
   \setbeamercolor{block title}{fg=structure,bg=structure!40}
   \setbeamercolor{block body}{fg=black,bg=structure!10}
   \begin{block}{{\bf Th\'eor\`eme }#1}
}{%
   \end{block}%
}


\newenvironment{proposition}[1][]{%
   \setbeamercolor{block title}{fg=structure,bg=structure!40}
   \setbeamercolor{block body}{fg=black,bg=structure!10}
   \begin{block}{{\bf Proposition }#1}
}{%
   \end{block}%
}

\newenvironment{corollaire}[1][]{%
   \setbeamercolor{block title}{fg=structure,bg=structure!40}
   \setbeamercolor{block body}{fg=black,bg=structure!10}
   \begin{block}{{\bf Corollaire }#1}
}{%
   \end{block}%
}

\newenvironment{mydefinition}[1][]{%
   \setbeamercolor{block title}{fg=structure,bg=structure!40}
   \setbeamercolor{block body}{fg=black,bg=structure!10}
   \begin{block}{{\bf Définition} #1}
}{%
   \end{block}%
}

\newenvironment{lemme}[0]{%
   \setbeamercolor{block title}{fg=structure,bg=structure!40}
   \setbeamercolor{block body}{fg=black,bg=structure!10}
   \begin{block}{\bf Lemme}
}{%
   \end{block}%
}

\newenvironment{remarque}[1][]{%
   \setbeamercolor{block title}{fg=black,bg=structure!20}
   \setbeamercolor{block body}{fg=black,bg=structure!5}
   \begin{block}{Remarque #1}
}{%
   \end{block}%
}


\newenvironment{exemple}[1][]{%
   \setbeamercolor{block title}{fg=black,bg=structure!20}
   \setbeamercolor{block body}{fg=black,bg=structure!5}
   \begin{block}{{\bf Exemple }#1}
}{%
   \end{block}%
}


\newenvironment{miniexercice}[0]{%
   \setbeamercolor{block title}{fg=structure,bg=structure!20}
   \setbeamercolor{block body}{fg=black,bg=structure!5}
   \begin{block}{Mini-exercices}
}{%
   \end{block}%
}


\newenvironment{tp}[0]{%
   \setbeamercolor{block title}{fg=structure,bg=structure!40}
   \setbeamercolor{block body}{fg=black,bg=structure!10}
   \begin{block}{\bf Travaux pratiques}
}{%
   \end{block}%
}
\newenvironment{exercicecours}[1][]{%
   \setbeamercolor{block title}{fg=structure,bg=structure!40}
   \setbeamercolor{block body}{fg=black,bg=structure!10}
   \begin{block}{{\bf Exercice }#1}
}{%
   \end{block}%
}
\newenvironment{algo}[1][]{%
   \setbeamercolor{block title}{fg=structure,bg=structure!40}
   \setbeamercolor{block body}{fg=black,bg=structure!10}
   \begin{block}{{\bf Algorithme}\hfill{\color{gray}\texttt{#1}}}
}{%
   \end{block}%
}


\setbeamertemplate{proof begin}{
   \setbeamercolor{block title}{fg=black,bg=structure!20}
   \setbeamercolor{block body}{fg=black,bg=structure!5}
   \begin{block}{{\footnotesize Démonstration}}
   \footnotesize
   \smallskip}
\setbeamertemplate{proof end}{%
   \end{block}}
\setbeamertemplate{qed symbol}{\openbox}


\makeatother
\usecolortheme[RGB={0,153,0}]{structure}

% Commande spécifique à ce chapitre
\newcommand{\GL}{\mathcal{G}\ell}
\newcounter{saveenumi}

%%%%%%%%%%%%%%%%%%%%%%%%%%%%%%%%%%%%%%%%%%%%%%%%%%%%%%%%%%%%%
%%%%%%%%%%%%%%%%%%%%%%%%%%%%%%%%%%%%%%%%%%%%%%%%%%%%%%%%%%%%%



\begin{document}

\title{{\bf Groupes}}
\subtitle{Définitions et exemples}

\begin{frame}
  
  \debutmontitre

  \pause

{\footnotesize
\hfill
\setbeamercovered{transparent=50}
\begin{minipage}{0.6\textwidth}
  \begin{itemize}
    \item<3-> Définition
    \item<4-> Exemples
    \item<5-> Puissance
    \item<6-> Matrices $2\times 2$
  \end{itemize}
\end{minipage}
}

\end{frame}

\setcounter{framenumber}{0}


%%%%%%%%%%%%%%%%%%%%%%%%%%%%%%%%%%%%%%%%%%%%%%%%%%%%%%%%%%%%%%%%


\section*{Motivation}


\begin{frame}


\begin{itemize}
  \item équation de degré 2 \quad $ax^2+bx+c=0$ \quad  $x_i = \frac{-b\pm\sqrt{b^2-4ac}}{2a}$

\pause

  \item équation de degré 3 \quad $x^3+3x+1=0$ \quad $x_0 = \sqrt[3]{\frac{\sqrt 5 - 1}{2}} -  \sqrt[3]{\frac{\sqrt 5 + 1}{2}}$

\pause

  \item équation de degré 4

\pause

  \item équation de degré 5 et plus ? \pause : il n'existe pas en général une telle  formule

  \item \'Evariste Galois (1811-1832) \& Niels Abel (1802-1829) 

\pause
  \item Notion de \og groupe \fg

\pause
  \item \`A la base d'autres notions : anneaux, corps, matrices, espaces vectoriels,...

\pause
  \item Applications : arithmétique, géométrie, cryptographie,...
\end{itemize}

\end{frame}




%---------------------------------------------------------------
\section*{Définition}


\begin{frame}
\begin{mydefinition}
Un \defi{groupe} $(G,\star)$ est un ensemble, associé à une opération $\star$ 
(la \defi{loi de composition}) vérifiant 

\pause

\begin{enumerate}
  \item \label{it:groupei} pour tout $x,y \in G$, \quad $x \star y \in G$ 

\hfill $\star$ est une \defi{loi de composition interne}

\pause

  \item \label{it:groupeii} pour tout $x,y,z \in G$, \quad $(x \star y) \star z = x \star (y \star z)$ 

\hfill la loi est \defi{associative}

\pause

  \item \label{it:groupeiii} il existe $e \in G$ tel que \quad $\forall x \in G$, $x \star e = x$ et $e \star x = x$ 

\hfill $e$ est l'\defi{élément neutre}

\pause

  \item \label{it:groupeiv} pour tout $x \in G$ il existe $x' \in G$ tel que \quad  $x \star x' = x' \star x = e$ 

\hfill $x'$ est l'\defi{inverse} de $x$ et est noté $x^{-1}$
\end{enumerate}
\end{mydefinition}

\bigskip
\pause

$G$ est un groupe \defi{commutatif} (ou \defi{abélien}) si en plus
$$\text{pour tout } x,y \in G  \qquad x \star y = y \star x$$


\end{frame}



%---------------------------------------------------------------
\section*{Exemples}


\begin{frame}

$(\Rr^*,\times)$ est un groupe commutatif

\pause

   \begin{enumerate}
     \item si $x,y \in \Rr^*$ alors $x \times y \in \Rr^*$
\pause
     \item $x\times (y \times z) = (x \times y) \times z$
\pause
     \item $1$ est l'élément neutre : $1 \times x = x$ et $x \times 1=x$
\pause
     \item $x' = \frac{1}{x}$ est l'inverse d'un élément $x \in \Rr^*$, car $x \times \frac{1}{x}=1$
\pause
     \item $x \times y = y \times x$
   \end{enumerate}



\pause

 $(\Zz,+)$ est un groupe commutatif

\pause

   \begin{enumerate}
     \item si $x,y \in \Zz$ alors $x + y \in \Zz$
\pause
     \item $x +  (y + z) = (x + y) + z$
\pause
     \item $0$ est l'élément neutre, en effet $0 + x = x$ et $x + 0=x$
\pause
     \item $x' = -x$ est l'inverse de $x$, car $x + (-x) = 0$  

\hfill Pour la loi $+$ l'inverse s'appelle aussi l'\defi{opposé}
\pause

     \item $x + y = y + x$
   \end{enumerate}

\end{frame}



\begin{frame}
 $(\Qq^*,\times)$, $(\Cc^*, \times)$ sont des groupes commutatifs

\pause
\bigskip


 $(\Qq,+)$, $(\Rr,+)$, $(\Cc,+)$ sont des groupes commutatifs  


\pause
\bigskip
\bigskip

Les deux exemples suivants \evidence{ne sont pas} des groupes
\begin{itemize}
  \item $(\Zz^*,\times)$ n'est pas un groupe :  $2$ n'a pas d'inverse (pour $\times$)
  \item $(\Nn,+)$ n'est pas un groupe : $3$ n'a pas d'inverse (pour $+$)
\end{itemize}  


\end{frame}



\begin{frame}


Soit $\mathcal{R}$ l'ensemble des rotations du plan dont le centre est à l'origine $O$

$(\mathcal{R},\circ)$ forme un groupe commutatif

\pause

\hfill\begin{minipage}{0.5\textwidth}
\myfigure{1.3}{
\tikzinput{fig_groupes02}
}  
\end{minipage}

\vspace*{-5ex}

\pause

   \begin{enumerate}
     \item $R_\theta \circ R_{\theta'}$ est une rotation
\pause
     \item  $R_\theta \circ (R_{\theta'}\circ R_{\theta''})=(R_\theta \circ R_{\theta'})\circ R_{\theta''}$
\pause
     \item l'élément neutre est la rotation d'angle $0$ : c'est l'identité du plan
 \pause
    \item l'inverse d'une rotation d'angle $\theta$ est la rotation d'angle $-\theta$
\pause
     \item $R_\theta \circ R_{\theta'}=R_{\theta'} \circ R_{\theta}$
   \end{enumerate}
\end{frame}


\begin{frame}
Soit $\mathcal{I}$ l'ensemble des isométries du plan (translations, rotations, réflexions et leurs composées)

$(\mathcal{I},\circ)$ est un groupe \evidence{non} commutatif

\pause

  \begin{itemize}
     \item soit $R$ la rotation de centre $O=(0,0)$ et d'angle $\frac \pi 2$
     \item soit $T$ la translation de vecteur $(1,0)$
     \item $T \circ R$ et $R \circ T$ ne sont pas égales
\pause
     \item $T \circ R (1,1) = T (-1,1) = (0,1)$, $R\circ T(1,1) = R(2,1) = (-1,2)$
  \end{itemize}


\myfigure{1}{
\footnotesize
\tikzinput{fig_groupes04}
\hspace*{-10ex}
\tikzinput{fig_groupes03}
} 


\end{frame}


%---------------------------------------------------------------
\section{Puissance}

\begin{frame}
Soit un groupe $(G,\star)$
\begin{itemize}
  \item $x^2 = x \star x$, \quad  $x^3 = x\star x \star x$
  \item $x^{n} = x\star x \star \cdots \star x$
  \item $x^0 = e$
  \item $x^{-n} = x^{-1}\star \cdots \star x^{-1}$
\end{itemize}

\pause
\bigskip

Règles de calculs 
\begin{itemize}
  \item $x^m \star x^n = x^{m+n}$
  \item $(x^m)^n = x^{mn}$
  \item $(x \star y)^{-1} = y^{-1} \star x^{-1}$ \  \evidence{attention à l'ordre !}
  \item \evidence{Si} $(G,\star)$ est \evidence{commutatif} alors $(x\star y)^n = x^n \star y^n$
\end{itemize}  
\end{frame}



%---------------------------------------------------------------
\section*{Matrices $2\times 2$}

\begin{frame}
Une \defi{matrice} $2\times 2$ est un tableau de $4$ nombres réels :
$$M=\begin{pmatrix} a & b \\ c & d \\ \end{pmatrix}$$

\pause

Le \defi{produit} de deux matrices $M$, $M'$ est noté $M\times M'$ 
$$ \left(\begin{matrix} a & b \\ c & d \\ \end{matrix}\right) \times 
\left(\begin{matrix} a' & b' \\ c' & d' \\ \end{matrix}\right)
= 
\pause
\left(\begin{matrix} aa'+bc' & ab' + bd' \\ ca'+dc' & cb'+dd' \\ \end{matrix}\right)$$

\pause
\vspace{0cm}

\only<1-3>{\vspace{6em}\ }


\only<4,5,6>{
$$
\begin{array}{cc}
 & 
\uncover<5->{\left(\begin{matrix} \quad a' \quad  & \quad b' \quad  \\ \quad c' \quad  & \quad d' \quad \\ \end{matrix}\right)} \\
\uncover<4->{\left(\begin{matrix} a & b \\ c & d \\ \end{matrix}\right)}  &  
  \uncover<6->{\left(\begin{matrix} {\scriptstyle aa'+bc'} & {\scriptstyle ab' + bd'} \\
    {\scriptstyle ca'+dc'} & {\scriptstyle cb'+dd'} \\ \end{matrix}\right)} \\
\end{array}
$$
}
\pause\pause\pause
\only<7-11>{
$$
\begin{array}{cc}
 & 
\left(\begin{matrix} \textcolor{blue}{\quad \tikz[remember picture]\coordinate(mataa); a' \quad} & \quad b' \quad  \\ \textcolor{blue}{\quad \tikz[remember picture]\coordinate(matcc); c' \quad} & \quad d' \quad  \\ \end{matrix}\right) \\
\left(\begin{matrix} \textcolor{blue}{a}\tikz[remember picture]\coordinate(mata); & \textcolor{blue}{b}\tikz[remember picture]\coordinate(matb); \\ c & d \\ \end{matrix}\right)  &  
%  \left(\begin{matrix} \uncover<9->{\textcolor{red}{{\scriptstyle aa'+bc'}}} 
%  & \uncover<12->{{\scriptstyle ab' + bd'}} \\
%    \uncover<12->{{\scriptstyle ca'+dc'}} & \uncover<12->{{\scriptstyle cb'+dd'}} \\ \end{matrix}\right) \\
  \left(\begin{matrix} {\textcolor{red}{{\scriptstyle \uncover<9->{aa'}\uncover<10->{+}\uncover<11->{bc'}}}} 
  & \uncover<12->{{\scriptstyle ab' + bd'}} \\
    \uncover<12->{{\scriptstyle ca'+dc'}} & \uncover<12->{{\scriptstyle cb'+dd'}} \\ \end{matrix}\right) \\
\end{array}
$$
\begin{tikzpicture}[x=1mm,y=1mm, remember picture, overlay]
   \coordinate (mmata) at ($(mata)+(-1,+3)$);
   \coordinate (mmatb) at ($(matb)+(-1,+3)$);
   \coordinate (mmataa) at ($(mataa)+(-1,+1)$);
   \coordinate (mmatcc) at ($(matcc)+(-1,+1)$);
\uncover<8->{\draw[<->, myred, very thick] (mmataa) to[bend right, thick]node[above, midway]{$\times$} (mmata);}
\uncover<8->{\draw[<->, myred, very thick] (mmatcc) to[bend right, thick]node[below, midway]{$\times$} (mmatb);}
\end{tikzpicture}}
\only<12->{
$$
\begin{array}{cc}
 & 
\left(\begin{matrix} \quad \only<12,14>{a'}\only<13>{\textcolor{blue}{a'}} \quad & \quad \tikz[remember picture]\coordinate(natbb); \only<13>{b'}\only<12,14>{\textcolor{blue}{b'}} \quad \\
 \quad \only<12,14>{c'}\only<13>{\textcolor{blue}{c'}} \quad & \quad \tikz[remember picture]\coordinate(natdd); \only<13>{d'}\only<12,14>{\textcolor{blue}{d'}} \quad  \\ \end{matrix}\right) \\
\left(\begin{matrix} \only<13->{a}\only<12>{\textcolor{blue}{a}}\tikz[remember picture]\coordinate(nata); & \only<13->{b}\only<12>{\textcolor{blue}{b}}\tikz[remember picture]\coordinate(natb); \\ 
\only<12>{c}\only<13->{\textcolor{blue}{c}} & \only<12>{d}\only<13->{\textcolor{blue}{d}} \\ \end{matrix}\right)  &  
  \left(\begin{matrix} \uncover<12->{{\scriptstyle aa'+bc'}} & \only<12>{\textcolor{red}{{\scriptstyle ab' + bd'}}}\only<13->{{\scriptstyle ab' + bd'}} \\
    \only<13>{\textcolor{red}{{\scriptstyle ca'+dc'}}}\only<14->{{\scriptstyle ca'+dc'}} & \uncover<14->{\textcolor{red}{{\scriptstyle cb'+dd'}}} \\ \end{matrix}\right) \\
\end{array}
$$
\begin{tikzpicture}[x=1mm,y=1mm, remember picture, overlay]
   \coordinate (nnata) at ($(nata)+(-1,+3)$);
   \coordinate (nnatb) at ($(natb)+(-1,+3)$);
   \coordinate (nnatbb) at ($(natbb)+(-1,+1)$);
   \coordinate (nnatdd) at ($(natdd)+(-1,+1)$);
\uncover<12>{\draw[<->, myred, very thick] (nnatbb) to[bend right, thick] (nnata);}
\uncover<12>{\draw[<->, myred, very thick] (nnatdd) to[bend right, thick] (nnatb);}
\end{tikzpicture}}
\pause\pause\pause\pause
%\vspace{2cm}
\end{frame}

\begin{frame}
\begin{exemple}
$$M=\left(\begin{matrix} 1 & 1 \\ 0 & -1 \\ \end{matrix}\right) \qquad 
M' = \left(\begin{matrix} 1 & 0 \\ 2 & 1 \\ \end{matrix}\right)$$

$$M\times M' = \uncover<6->{\left(\begin{matrix} 3 & 1 \\ -2 & -1 \\ \end{matrix}\right)} \qquad
M'\times M=  \uncover<6->{\left(\begin{matrix} 1 & 1 \\ 2 & 1 \\ \end{matrix}\right)}$$
\end{exemple}



\uncover<7->{\centerline{En général $M\times M' \neq M'\times M$}}

\pause
\bigskip

$\begin{array}{cc}
& \left(\begin{matrix} \ 1\  & \ 0\  \\ \ 2\  & \ 1\  \\ \end{matrix}\right)\\
\left(\begin{matrix} 1 & 1 \\ 0 & -1 \\ \end{matrix}\right)  & 
\pause 
\left(\begin{matrix} 3 & 1 \\ -2 & -1 \\ \end{matrix}\right)
 \end{array}$
\qquad 
\pause
$\begin{array}{cc}
& \left(\begin{matrix} 1 & 1 \\ 0 & -1 \\ \end{matrix}\right) \\
\left(\begin{matrix} 1 & 0 \\ 2 & 1 \\ \end{matrix}\right) & 
\pause
\left(\begin{matrix} \ 1\  & \ 1\  \\ \ 2\  & \ 1\  \\ \end{matrix}\right) \\
 \end{array}$
\pause
\end{frame}


\begin{frame}
Le \defi{déterminant} d'une matrice est 

$$\det \left(\begin{matrix} a & b \\ c & d \\ \end{matrix}\right) = ad-bc$$

\pause

\begin{lemme}
\label{lem:det}
$\det (M\times M') = \det M \cdot \det M'$
\end{lemme}


\pause

\begin{proof}
$M\times M' = \left(\begin{matrix} aa'+bc' & ab' + bd' \\ ca'+dc' & cb'+dd' \\ \end{matrix}\right)$

\pause

$\big(aa'+bc'  \big) \big( cb'+dd' \big) - 
\big( ab' + bd' \big) \big( ca'+dc' \big) = (ad-bc)(a'd'-b'c')$  

\end{proof}

\end{frame}



\begin{frame}



\begin{proposition}
L'ensemble des matrices $2\times 2$ ayant un déterminant non nul,
muni de la multiplication des matrices $\times$, forme un groupe non commutatif
\end{proposition}

\pause

Ce groupe est noté $(\GL_2,\times)$

\pause
\bigskip



\begin{proof}
\begin{enumerate}
  \item loi de composition interne 
 
$M\times M'$ est une matrice $2\times 2$

$\det (M\times M') = \det M \cdot \det M' \neq 0$

\pause

  \item  la loi est associative : $(M\times M')\times M''=M \times (M'\times M'')$

\pause

  \item  l'élément neutre est la \defi{matrice identité}  
$I = \left(\begin{smallmatrix} 1 & 0 \\ 0 & 1 \\ \end{smallmatrix}\right)$

 
$\left(\begin{smallmatrix} a & b \\ c & d \\ \end{smallmatrix}\right) \times  
\left(\begin{smallmatrix} 1 & 0 \\ 0 & 1 \\ \end{smallmatrix}\right) =
\left(\begin{smallmatrix} a & b \\ c & d \\ \end{smallmatrix}\right)$
et
$\left(\begin{smallmatrix} 1 & 0 \\ 0 & 1 \\ \end{smallmatrix}\right) \times  
\left(\begin{smallmatrix} a & b \\ c & d \\ \end{smallmatrix}\right) =
\left(\begin{smallmatrix} a & b \\ c & d \\ \end{smallmatrix}\right)$

\pause

  \item l'inverse de  $M=\left(\begin{smallmatrix} a & b \\ c & d \\ \end{smallmatrix}\right)$
est $M^{-1} = \frac{1}{ad-bc} \left(\begin{smallmatrix} d & -b \\ -c & a\\ \end{smallmatrix}\right)$

$M \times M^{-1} = I$ et $M^{-1} \times M = I$


\end{enumerate}

\end{proof}   
\end{frame}


%%%%%%%%%%%%%%%%%%%%%%%%%%%%%%%%%%%%%%%%%%%%%%%%%%%%%%%%%%%%%%%%
\section*{Mini-exercices}



\begin{frame}

\begin{miniexercice}
\begin{enumerate}
% %   \setlength{\itemsep}{1em}
% %   \vspace{1em}

 \item Montrer que $(\Rr_+^*,\times)$ est un groupe commutatif.

 \item Soit $f_{a,b} : \Rr \to \Rr$ la fonction définie par $x \mapsto ax+b$.
Montrer que l'ensemble $\mathcal{F}=\{ f_{a,b} \mid a \in \Rr^*, b\in\Rr\}$ muni 
de la composition \og$\circ$\fg\   est un groupe non commutatif.

 \item (Plus dur) Soit $G=]-1,1[$. Pour $x,y\in G$ on définit $x\star y = \frac{x+y}{1+xy}$.
Montrer que $(G,\star)$ forme un groupe: {\footnotesize (a) Montrer que $\star$ est une loi de composition interne :
$x\star y \in G$ (b) Montrer que la loi est associative (c) Montrer que l'élément neutre est $0$
(d) Trouver l'inverse de $x$.}

\setcounter{saveenumi}{\theenumi}
\end{enumerate}
\end{miniexercice}

\medskip

Soit $(G,\star)$ est un groupe quelconque, $x,y,z$ sont des éléments de $G$.
\begin{miniexercice}
\begin{enumerate}
  \setcounter{enumi}{\thesaveenumi}
  \item Montrer que si $x\star y = x \star z$ alors $y = z$.
  \item Que vaut $\big(x^{-1}\big)^{-1}$ ?
  \item Si $x^n=e$, quel est l'inverse de $x$ ?
  \setcounter{saveenumi}{\theenumi}
\end{enumerate}
\end{miniexercice}
\end{frame}


\begin{frame}

\begin{miniexercice}

\begin{enumerate}
  \setcounter{enumi}{\thesaveenumi}

  \item Soient 
$M_1= \left(\begin{smallmatrix} 0 & -1 \\ 1 & 0 \\ \end{smallmatrix}\right)$,
$M_2= \left(\begin{smallmatrix} 1 & 2 \\ 1 & 0 \\ \end{smallmatrix}\right)$,
$M_3= \left(\begin{smallmatrix} 1 & 2 \\ 3 & 4 \\ \end{smallmatrix}\right)$.
Vérifier que $M_1 \times (M_2 \times M_3)= (M_1 \times M_2) \times M_3$.

  \item Calculer $(M_1\times M_2)^2$ et $M_1^2 \times M_2^2$. (Rappel : $M^2=M\times M$)

  \item Calculer les déterminants des $M_i$ ainsi que leurs inverses.

  \item Montrer que l'ensemble des matrices $2\times 2$ muni de l'addition $+$ définie par 
$\left(\begin{smallmatrix} a & b \\ c & d \\ \end{smallmatrix}\right) 
+ \left(\begin{smallmatrix} a' & b' \\ c' & d' \\ \end{smallmatrix}\right)
= \left(\begin{smallmatrix} a+a' & b+b' \\ c+c' & d+d' \\ \end{smallmatrix}\right)$ forme 
un groupe commutatif.
\end{enumerate}


\end{miniexercice}
\end{frame}



\end{document}