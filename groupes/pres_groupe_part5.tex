
%%%%%%%%%%%%%%%%%% PREAMBULE %%%%%%%%%%%%%%%%%%

\documentclass[aspectratio=169,utf8]{beamer}
%\documentclass[aspectratio=169,handout]{beamer}

\usetheme{Boadilla}
%\usecolortheme{seahorse}
\usecolortheme[RGB={245,66,24}]{structure}
\useoutertheme{infolines}

% packages
\usepackage{amsfonts,amsmath,amssymb,amsthm}
\usepackage[utf8]{inputenc}
\usepackage[T1]{fontenc}
\usepackage{lmodern}

\usepackage[francais]{babel}
\usepackage{fancybox}
\usepackage{graphicx}

\usepackage{float}
\usepackage{xfrac}

%\usepackage[usenames, x11names]{xcolor}
\usepackage{tikz}
\usepackage{pgfplots}
\usepackage{datetime}



%-----  Package unités -----
\usepackage{siunitx}
\sisetup{locale = FR,detect-all,per-mode = symbol}

%\usepackage{mathptmx}
%\usepackage{fouriernc}
%\usepackage{newcent}
%\usepackage[mathcal,mathbf]{euler}

%\usepackage{palatino}
%\usepackage{newcent}
% \usepackage[mathcal,mathbf]{euler}



% \usepackage{hyperref}
% \hypersetup{colorlinks=true, linkcolor=blue, urlcolor=blue,
% pdftitle={Exo7 - Exercices de mathématiques}, pdfauthor={Exo7}}


%section
% \usepackage{sectsty}
% \allsectionsfont{\bf}
%\sectionfont{\color{Tomato3}\upshape\selectfont}
%\subsectionfont{\color{Tomato4}\upshape\selectfont}

%----- Ensembles : entiers, reels, complexes -----
\newcommand{\Nn}{\mathbb{N}} \newcommand{\N}{\mathbb{N}}
\newcommand{\Zz}{\mathbb{Z}} \newcommand{\Z}{\mathbb{Z}}
\newcommand{\Qq}{\mathbb{Q}} \newcommand{\Q}{\mathbb{Q}}
\newcommand{\Rr}{\mathbb{R}} \newcommand{\R}{\mathbb{R}}
\newcommand{\Cc}{\mathbb{C}} 
\newcommand{\Kk}{\mathbb{K}} \newcommand{\K}{\mathbb{K}}

%----- Modifications de symboles -----
\renewcommand{\epsilon}{\varepsilon}
\renewcommand{\Re}{\mathop{\text{Re}}\nolimits}
\renewcommand{\Im}{\mathop{\text{Im}}\nolimits}
%\newcommand{\llbracket}{\left[\kern-0.15em\left[}
%\newcommand{\rrbracket}{\right]\kern-0.15em\right]}

\renewcommand{\ge}{\geqslant}
\renewcommand{\geq}{\geqslant}
\renewcommand{\le}{\leqslant}
\renewcommand{\leq}{\leqslant}
\renewcommand{\epsilon}{\varepsilon}

%----- Fonctions usuelles -----
\newcommand{\ch}{\mathop{\text{ch}}\nolimits}
\newcommand{\sh}{\mathop{\text{sh}}\nolimits}
\renewcommand{\tanh}{\mathop{\text{th}}\nolimits}
\newcommand{\cotan}{\mathop{\text{cotan}}\nolimits}
\newcommand{\Arcsin}{\mathop{\text{arcsin}}\nolimits}
\newcommand{\Arccos}{\mathop{\text{arccos}}\nolimits}
\newcommand{\Arctan}{\mathop{\text{arctan}}\nolimits}
\newcommand{\Argsh}{\mathop{\text{argsh}}\nolimits}
\newcommand{\Argch}{\mathop{\text{argch}}\nolimits}
\newcommand{\Argth}{\mathop{\text{argth}}\nolimits}
\newcommand{\pgcd}{\mathop{\text{pgcd}}\nolimits} 


%----- Commandes divers ------
\newcommand{\ii}{\mathrm{i}}
\newcommand{\dd}{\text{d}}
\newcommand{\id}{\mathop{\text{id}}\nolimits}
\newcommand{\Ker}{\mathop{\text{Ker}}\nolimits}
\newcommand{\Card}{\mathop{\text{Card}}\nolimits}
\newcommand{\Vect}{\mathop{\text{Vect}}\nolimits}
\newcommand{\Mat}{\mathop{\text{Mat}}\nolimits}
\newcommand{\rg}{\mathop{\text{rg}}\nolimits}
\newcommand{\tr}{\mathop{\text{tr}}\nolimits}


%----- Structure des exercices ------

\newtheoremstyle{styleexo}% name
{2ex}% Space above
{3ex}% Space below
{}% Body font
{}% Indent amount 1
{\bfseries} % Theorem head font
{}% Punctuation after theorem head
{\newline}% Space after theorem head 2
{}% Theorem head spec (can be left empty, meaning ‘normal’)

%\theoremstyle{styleexo}
\newtheorem{exo}{Exercice}
\newtheorem{ind}{Indications}
\newtheorem{cor}{Correction}


\newcommand{\exercice}[1]{} \newcommand{\finexercice}{}
%\newcommand{\exercice}[1]{{\tiny\texttt{#1}}\vspace{-2ex}} % pour afficher le numero absolu, l'auteur...
\newcommand{\enonce}{\begin{exo}} \newcommand{\finenonce}{\end{exo}}
\newcommand{\indication}{\begin{ind}} \newcommand{\finindication}{\end{ind}}
\newcommand{\correction}{\begin{cor}} \newcommand{\fincorrection}{\end{cor}}

\newcommand{\noindication}{\stepcounter{ind}}
\newcommand{\nocorrection}{\stepcounter{cor}}

\newcommand{\fiche}[1]{} \newcommand{\finfiche}{}
\newcommand{\titre}[1]{\centerline{\large \bf #1}}
\newcommand{\addcommand}[1]{}
\newcommand{\video}[1]{}

% Marge
\newcommand{\mymargin}[1]{\marginpar{{\small #1}}}

\def\noqed{\renewcommand{\qedsymbol}{}}


%----- Presentation ------
\setlength{\parindent}{0cm}

%\newcommand{\ExoSept}{\href{http://exo7.emath.fr}{\textbf{\textsf{Exo7}}}}

\definecolor{myred}{rgb}{0.93,0.26,0}
\definecolor{myorange}{rgb}{0.97,0.58,0}
\definecolor{myyellow}{rgb}{1,0.86,0}

\newcommand{\LogoExoSept}[1]{  % input : echelle
{\usefont{U}{cmss}{bx}{n}
\begin{tikzpicture}[scale=0.1*#1,transform shape]
  \fill[color=myorange] (0,0)--(4,0)--(4,-4)--(0,-4)--cycle;
  \fill[color=myred] (0,0)--(0,3)--(-3,3)--(-3,0)--cycle;
  \fill[color=myyellow] (4,0)--(7,4)--(3,7)--(0,3)--cycle;
  \node[scale=5] at (3.5,3.5) {Exo7};
\end{tikzpicture}}
}


\newcommand{\debutmontitre}{
  \author{} \date{} 
  \thispagestyle{empty}
  \hspace*{-10ex}
  \begin{minipage}{\textwidth}
    \titlepage  
  \vspace*{-2.5cm}
  \begin{center}
    \LogoExoSept{2.5}
  \end{center}
  \end{minipage}

  \vspace*{-0cm}
  
  % Astuce pour que le background ne soit pas discrétisé lors de la conversion pdf -> png
\begin{tikzpicture}
        \fill[opacity=0,green!60!black] (0,0)--++(0,0)--++(0,0)--++(0,0)--cycle; 
\end{tikzpicture}

% toc S'affiche trop tot :
% \tableofcontents[hideallsubsections, pausesections]
}

\newcommand{\finmontitre}{
  \end{frame}
  \setcounter{framenumber}{0}
} % ne marche pas pour une raison obscure

%----- Commandes supplementaires ------

% \usepackage[landscape]{geometry}
% \geometry{top=1cm, bottom=3cm, left=2cm, right=10cm, marginparsep=1cm
% }
% \usepackage[a4paper]{geometry}
% \geometry{top=2cm, bottom=2cm, left=2cm, right=2cm, marginparsep=1cm
% }

%\usepackage{standalone}


% New command Arnaud -- november 2011
\setbeamersize{text margin left=24ex}
% si vous modifier cette valeur il faut aussi
% modifier le decalage du titre pour compenser
% (ex : ici =+10ex, titre =-5ex

\theoremstyle{definition}
%\newtheorem{proposition}{Proposition}
%\newtheorem{exemple}{Exemple}
%\newtheorem{theoreme}{Théorème}
%\newtheorem{lemme}{Lemme}
%\newtheorem{corollaire}{Corollaire}
%\newtheorem*{remarque*}{Remarque}
%\newtheorem*{miniexercice}{Mini-exercices}
%\newtheorem{definition}{Définition}

% Commande tikz
\usetikzlibrary{calc}
\usetikzlibrary{patterns,arrows}
\usetikzlibrary{matrix}
\usetikzlibrary{fadings} 

%definition d'un terme
\newcommand{\defi}[1]{{\color{myorange}\textbf{\emph{#1}}}}
\newcommand{\evidence}[1]{{\color{blue}\textbf{\emph{#1}}}}
\newcommand{\assertion}[1]{\emph{\og#1\fg}}  % pour chapitre logique
%\renewcommand{\contentsname}{Sommaire}
\renewcommand{\contentsname}{}
\setcounter{tocdepth}{2}



%------ Figures ------

\def\myscale{1} % par défaut 
\newcommand{\myfigure}[2]{  % entrée : echelle, fichier figure
\def\myscale{#1}
\begin{center}
\footnotesize
{#2}
\end{center}}


%------ Encadrement ------

\usepackage{fancybox}


\newcommand{\mybox}[1]{
\setlength{\fboxsep}{7pt}
\begin{center}
\shadowbox{#1}
\end{center}}

\newcommand{\myboxinline}[1]{
\setlength{\fboxsep}{5pt}
\raisebox{-10pt}{
\shadowbox{#1}
}
}

%--------------- Commande beamer---------------
\newcommand{\beameronly}[1]{#1} % permet de mettre des pause dans beamer pas dans poly


\setbeamertemplate{navigation symbols}{}
\setbeamertemplate{footline}  % tiré du fichier beamerouterinfolines.sty
{
  \leavevmode%
  \hbox{%
  \begin{beamercolorbox}[wd=.333333\paperwidth,ht=2.25ex,dp=1ex,center]{author in head/foot}%
    % \usebeamerfont{author in head/foot}\insertshortauthor%~~(\insertshortinstitute)
    \usebeamerfont{section in head/foot}{\bf\insertshorttitle}
  \end{beamercolorbox}%
  \begin{beamercolorbox}[wd=.333333\paperwidth,ht=2.25ex,dp=1ex,center]{title in head/foot}%
    \usebeamerfont{section in head/foot}{\bf\insertsectionhead}
  \end{beamercolorbox}%
  \begin{beamercolorbox}[wd=.333333\paperwidth,ht=2.25ex,dp=1ex,right]{date in head/foot}%
    % \usebeamerfont{date in head/foot}\insertshortdate{}\hspace*{2em}
    \insertframenumber{} / \inserttotalframenumber\hspace*{2ex} 
  \end{beamercolorbox}}%
  \vskip0pt%
}


\definecolor{mygrey}{rgb}{0.5,0.5,0.5}
\setlength{\parindent}{0cm}
%\DeclareTextFontCommand{\helvetica}{\fontfamily{phv}\selectfont}

% background beamer
\definecolor{couleurhaut}{rgb}{0.85,0.9,1}  % creme
\definecolor{couleurmilieu}{rgb}{1,1,1}  % vert pale
\definecolor{couleurbas}{rgb}{0.85,0.9,1}  % blanc
\setbeamertemplate{background canvas}[vertical shading]%
[top=couleurhaut,middle=couleurmilieu,midpoint=0.4,bottom=couleurbas] 
%[top=fondtitre!05,bottom=fondtitre!60]



\makeatletter
\setbeamertemplate{theorem begin}
{%
  \begin{\inserttheoremblockenv}
  {%
    \inserttheoremheadfont
    \inserttheoremname
    \inserttheoremnumber
    \ifx\inserttheoremaddition\@empty\else\ (\inserttheoremaddition)\fi%
    \inserttheorempunctuation
  }%
}
\setbeamertemplate{theorem end}{\end{\inserttheoremblockenv}}

\newenvironment{theoreme}[1][]{%
   \setbeamercolor{block title}{fg=structure,bg=structure!40}
   \setbeamercolor{block body}{fg=black,bg=structure!10}
   \begin{block}{{\bf Th\'eor\`eme }#1}
}{%
   \end{block}%
}


\newenvironment{proposition}[1][]{%
   \setbeamercolor{block title}{fg=structure,bg=structure!40}
   \setbeamercolor{block body}{fg=black,bg=structure!10}
   \begin{block}{{\bf Proposition }#1}
}{%
   \end{block}%
}

\newenvironment{corollaire}[1][]{%
   \setbeamercolor{block title}{fg=structure,bg=structure!40}
   \setbeamercolor{block body}{fg=black,bg=structure!10}
   \begin{block}{{\bf Corollaire }#1}
}{%
   \end{block}%
}

\newenvironment{mydefinition}[1][]{%
   \setbeamercolor{block title}{fg=structure,bg=structure!40}
   \setbeamercolor{block body}{fg=black,bg=structure!10}
   \begin{block}{{\bf Définition} #1}
}{%
   \end{block}%
}

\newenvironment{lemme}[0]{%
   \setbeamercolor{block title}{fg=structure,bg=structure!40}
   \setbeamercolor{block body}{fg=black,bg=structure!10}
   \begin{block}{\bf Lemme}
}{%
   \end{block}%
}

\newenvironment{remarque}[1][]{%
   \setbeamercolor{block title}{fg=black,bg=structure!20}
   \setbeamercolor{block body}{fg=black,bg=structure!5}
   \begin{block}{Remarque #1}
}{%
   \end{block}%
}


\newenvironment{exemple}[1][]{%
   \setbeamercolor{block title}{fg=black,bg=structure!20}
   \setbeamercolor{block body}{fg=black,bg=structure!5}
   \begin{block}{{\bf Exemple }#1}
}{%
   \end{block}%
}


\newenvironment{miniexercice}[0]{%
   \setbeamercolor{block title}{fg=structure,bg=structure!20}
   \setbeamercolor{block body}{fg=black,bg=structure!5}
   \begin{block}{Mini-exercices}
}{%
   \end{block}%
}


\newenvironment{tp}[0]{%
   \setbeamercolor{block title}{fg=structure,bg=structure!40}
   \setbeamercolor{block body}{fg=black,bg=structure!10}
   \begin{block}{\bf Travaux pratiques}
}{%
   \end{block}%
}
\newenvironment{exercicecours}[1][]{%
   \setbeamercolor{block title}{fg=structure,bg=structure!40}
   \setbeamercolor{block body}{fg=black,bg=structure!10}
   \begin{block}{{\bf Exercice }#1}
}{%
   \end{block}%
}
\newenvironment{algo}[1][]{%
   \setbeamercolor{block title}{fg=structure,bg=structure!40}
   \setbeamercolor{block body}{fg=black,bg=structure!10}
   \begin{block}{{\bf Algorithme}\hfill{\color{gray}\texttt{#1}}}
}{%
   \end{block}%
}


\setbeamertemplate{proof begin}{
   \setbeamercolor{block title}{fg=black,bg=structure!20}
   \setbeamercolor{block body}{fg=black,bg=structure!5}
   \begin{block}{{\footnotesize Démonstration}}
   \footnotesize
   \smallskip}
\setbeamertemplate{proof end}{%
   \end{block}}
\setbeamertemplate{qed symbol}{\openbox}


\makeatother
\usecolortheme[RGB={0,153,0}]{structure}

% Commande spécifique à ce chapitre
\newcounter{saveenumi}

%%%%%%%%%%%%%%%%%%%%%%%%%%%%%%%%%%%%%%%%%%%%%%%%%%%%%%%%%%%%%
%%%%%%%%%%%%%%%%%%%%%%%%%%%%%%%%%%%%%%%%%%%%%%%%%%%%%%%%%%%%%



\begin{document}



\title{{\bf Groupes}}
\subtitle{Le groupe des permutations $\mathcal{S}_n$}

\begin{frame}
  
  \debutmontitre

  \pause

{\footnotesize
\hfill
\setbeamercovered{transparent=50}
\begin{minipage}{0.6\textwidth}
  \begin{itemize}
    \item<3-> Groupe des permutations
    \item<4-> Le groupe $\mathcal{S}_3$
    \item<5-> Décomposition en cycles
  \end{itemize}
\end{minipage}
}

\end{frame}

\setcounter{framenumber}{0}


%%%%%%%%%%%%%%%%%%%%%%%%%%%%%%%%%%%%%%%%%%%%%%%%%%%%%%%%%%%%%%%%





%---------------------------------------------------------------
\section{Groupe des permutations}


\begin{frame}

Fixons $n\ge 2$
\begin{proposition}
L'ensemble des bijections de $\{1,2,\ldots,n\}$ dans lui-même, muni de la composition
des fonctions est un groupe, noté $(\mathcal{S}_n,\circ)$
\end{proposition}

\pause
\bigskip

Une bijection de $\{1,2,\ldots,n\}$  est une \defi{permutation}

Le groupe $(\mathcal{S}_n,\circ)$ est le \defi{groupe des permutations} 

\end{frame}


\begin{frame}

\begin{lemme}
Le cardinal de $\mathcal{S}_n$ est $n!$
\end{lemme}

\pause
\bigskip

\begin{proof}

Bijection $f : \{1,2,\ldots,n\} \longrightarrow \{1,2,\ldots,n\}$

\medskip
 
Pour $f(1)$ il y a $n$ choix

Pour $f(2)$ il y a $n-1$ choix

...


Pour l'image du dernier élément $n$ il ne reste qu'une possibilité

Au final il y a $n\times(n-1)\times \cdots \times 2 \times 1 = n!$ permutations
\end{proof}

\end{frame}


%---------------------------------------------------------------
\section{Notation et exemples}


\begin{frame}
\'Ecriture d'une permutation $f : \{1,2,\ldots,n\} \longrightarrow \{1,2,\ldots,n\}$
$$\left[\begin{matrix} 
 1    & 2    & \cdots & n \\  
 f(1) & f(2) & \cdots & f(n) \\      
        \end{matrix}
 \right]
$$

\pause
\bigskip

\begin{exemple}
La permutation de $\mathcal{S}_7$ 
$$\left[\begin{matrix} 
 1 & 2 & 3 & 4 & 5 & 6 & 7 \tikz[remember picture]\coordinate(extop); \\  
 3 & 7 & 5 & 4 & 6 & 1 & 2  \tikz[remember picture]\coordinate(exbot);\\      
        \end{matrix} \right]
$$  

\pause

\begin{tikzpicture}[x=1mm,y=1mm, remember picture, overlay]
   \coordinate (mytop) at ($(extop)+(3,3)$);
   \coordinate (mybot) at ($(exbot)+(3,-1)$);
   \draw[|->, myred, very thick] (mytop) to[bend left,thick]node[right, midway]{$f$} (mybot);
\end{tikzpicture}

\pause

est la bijection $f : \{1,2,\ldots,7\} \longrightarrow \{1,2,\ldots,7\}$
définie par 

\centerline{$f(1)= 3$ \quad  $f(2)=7$ \quad  $f(3)=5$ \quad $f(4)=4$}
\centerline{$f(5)=6$ \quad $f(6)=1$ \quad $f(7)=2$}
\end{exemple}

\end{frame}


\begin{frame}
L'élément neutre du groupe est l'\evidence{identité}

Pour $\mathcal{S}_7$ c'est \quad
$\id = \left[\begin{matrix} 
 1 & 2 & 3 & 4 & 5 & 6 & 7 \\  
 1 & 2 & 3 & 4 & 5 & 6 & 7 \\     
        \end{matrix} \right]
$


\pause
\medskip

\evidence{Composition} de deux permutations
$$f =\left[\begin{matrix} 
 1 & 2 & 3 & 4 & 5 & 6 & 7 \\  
 3 & 7 & 5 & 4 & 6 & 1 & 2 \\      
        \end{matrix} \right]
\qquad 
g = \left[\begin{matrix} 
 1 & 2 & 3 & 4 & 5 & 6 & 7 \\  
 4 & 3 & 2 & 1 & 7 & 5 & 6 \\      
        \end{matrix} \right]
$$

\pause

$$g\circ f 
= \left[\begin{matrix} 
 1 & 2 & 3 & 4 & 5 & 6 & 7 \tikz[remember picture]\coordinate(comptop); \\  
 3 & 7 & 5 & 4 & 6 & 1 & 2  \tikz[remember picture]\coordinate(compmid);\\   
\uncover<5->{2 & 6 & 7 & 1 & 5 & 4 & 3 \tikz[remember picture]\coordinate(compbot);} \\
        \end{matrix} \right]
\uncover<8->{
\qquad\qquad = \left[\begin{matrix} 
 1 & 2 & 3 & 4 & 5 & 6 & 7 \\      
 2 & 6 & 7 & 1 & 5 & 4 & 3
        \end{matrix} \right]
}
$$
\begin{tikzpicture}[x=1mm,y=1mm, remember picture, overlay]
   \coordinate (mytopone) at ($(comptop)+(3,3)$);
   \coordinate (mymidone) at ($(compmid)+(3,+1)$);
   \draw[|->, myred, very thick] (mytopone) to[thick]node[right, midway]{$f$} (mymidone);
\end{tikzpicture}
\pause
\begin{tikzpicture}[x=1mm,y=1mm, remember picture, overlay]
   \coordinate (mymidtwo) at ($(compmid)+(3,+1)$);
   \coordinate (mybottwo) at ($(compbot)+(3,-1)$);
   \draw[|->, myred, very thick] (mymidtwo) to[thick]node[right, midway]{$g$} (mybottwo);
\end{tikzpicture}
\pause\pause
\begin{tikzpicture}[x=1mm,y=1mm, remember picture, overlay]
   \coordinate (mytopthree) at ($(comptop)+(6,3)$);
   \coordinate (mybotthree) at ($(compbot)+(6,-1)$);
  \draw[|->, myred, very thick] (mytopthree) to[bend left,thick]node[right, midway]{$g\circ f$} (mybotthree);
\end{tikzpicture}
\pause
\begin{tikzpicture}[x=1mm,y=1mm, remember picture, overlay]
   \coordinate (rayemid) at ($(compmid)+(0,1)$);
   \draw[myred, very thick] (rayemid)-- +(-35,0);
\end{tikzpicture}

\pause\pause
\medskip

\evidence{L'inverse} : échanger les lignes du haut et du bas
$$f= \left[\begin{matrix} 
 1 & 2 & 3 & 4 & 5 & 6 & 7 \tikz[remember picture]\coordinate(invtop); \\  
 3 & 7 & 5 & 4 & 6 & 1 & 2 \tikz[remember picture]\coordinate(invbot); \\      
        \end{matrix} \right]$$
\pause
\begin{tikzpicture}[x=1mm,y=1mm, remember picture, overlay]
   \coordinate (mytop) at ($(invtop)+(3,3)$);
   \coordinate (mybot) at ($(invbot)+(3,-1)$);
   \draw[|->, myred, very thick] (mybot) to[bend right,thick]node[right, midway]{$f^{-1}$} (mytop);
\end{tikzpicture}
\pause
$$f^{-1}=\left[ \begin{matrix}
 3 & 7 & 5 & 4 & 6 & 1 & 2\\ 
\uncover<12->{ 1 & 2 & 3 & 4 & 5 & 6 & 7}\\  
        \end{matrix} \right]
\pause\pause
=\left[\begin{matrix} 
 1 & 2 & 3 & 4 & 5 & 6 & 7 \\ 
 6 & 7 & 1 & 4 & 3 & 5 & 2 \\ 
        \end{matrix} \right]
$$

\end{frame}

%---------------------------------------------------------------
\section{Le groupe $\mathcal{S}_3$}

\begin{frame}
\medskip 
Le groupe $\mathcal{S}_3$ des permutations de $\{1,2,3\}$ a $3!=6$ éléments

\pause

\begin{minipage}{0.39\textwidth}
\begin{itemize}
 \item $\id = \left[\begin{smallmatrix} 
 1 & 2 & 3 \\  
 1 & 2 & 3 \\     
        \end{smallmatrix} \right]
$
\pause

 \item $\tau_1 = \left[\begin{smallmatrix} 
 1 & 2 & 3 \\  
 1 & 3 & 2 \\     
        \end{smallmatrix} \right]
$ 
\pause

 \item $\tau_2 = \left[\begin{smallmatrix} 
 1 & 2 & 3 \\  
 3 & 2 & 1 \\     
        \end{smallmatrix} \right]
$ 
\pause

 \item $\tau_3 = \left[\begin{smallmatrix} 
 1 & 2 & 3 \\  
 2 & 1 & 3 \\     
        \end{smallmatrix} \right]
$
\pause

 \item $\sigma = \left[\begin{smallmatrix} 
 1 & 2 & 3 \\  
 2 & 3 & 1 \\     
        \end{smallmatrix} \right]
$
\pause


 \item $\sigma^{-1} = \left[\begin{smallmatrix} 
 1 & 2 & 3 \\  
 3 & 1 & 2 \\     
        \end{smallmatrix} \right]
$ 
\end{itemize}  
\end{minipage}
\pause
\begin{minipage}{0.39\textwidth}
$$\mathcal{S}_3 = \big\{id,\tau_1,\tau_2,\tau_3,\sigma,\sigma^{-1}\big\}$$  
\pause
\medskip 
$$\tau_1 \circ \sigma
\pause 
=\left[\begin{smallmatrix} 
 1 & 2 & 3 \\  
 2 & 3 & 1 \\  
 3 & 2 & 1 \\    
        \end{smallmatrix} \right]
\pause
= \left[\begin{matrix} 
 1 & 2 & 3 \\  
 3 & 2 & 1 \\    
        \end{matrix} \right]
\pause
= \tau_2
$$
\pause
$$
\sigma \circ \tau_1 
= \left[\begin{smallmatrix} 
 1 & 2 & 3 \\  
 1 & 3 & 2 \\ 
 2 & 1 & 3 \\     
        \end{smallmatrix} \right]
= \left[\begin{matrix} 
 1 & 2 & 3 \\  
 2 & 1 & 3 \\     
        \end{matrix} \right]
\pause
= \tau_3
 $$
\end{minipage}

\pause
\bigskip 

\begin{lemme}
Le groupe $\mathcal{S}_n$ n'est pas commutatif ($n\ge3$)
\end{lemme}
\end{frame}


\begin{frame}
\centerline{Table du groupe $\mathcal{S}_3$}

\medskip 

\begin{center}
\begin{tabular}{c||c|c|c|c|c|c}
\textcolor{green}{$g$} $\circ$ \textcolor{blue}{$f$}  & \textcolor{blue}{$\id$} & \textcolor{blue}{$\tau_1$} 
 & \textcolor{blue}{$\tau_2$} & \textcolor{blue}{$\tau_3$} 
&  \textcolor{blue}{$\sigma$} & \textcolor{blue}{$\sigma^{-1}$} \\ \hline\hline

\textcolor{green}{$\id$} & \uncover<4->{$\id$} & \uncover<4->{$\tau_1$} & \uncover<4->{$\tau_2$} & \uncover<4->{$\tau_3$} & \uncover<4->{$\sigma$} & \uncover<4->{$\sigma^{-1}$} \\ \hline

\textcolor{green}{$\tau_1$} & \uncover<5->{$\tau_1$} & \uncover<6->{$\id$} & \uncover<6->{$\sigma$} & \uncover<6->{$\sigma^{-1}$} & \uncover<2->{\textcolor{red}{$\tau_1 \circ \sigma = \tau_2$}} & \uncover<6->{$\tau_3$} \\ \hline

\textcolor{green}{$\tau_2$} & \uncover<5->{$\tau_2$}  & \uncover<6->{$\sigma^{-1}$} & \uncover<6->{$\id$} & \uncover<6->{$\sigma$} & \uncover<6->{$\tau_3$} & \uncover<6->{$\tau_1$} \\ \hline

\textcolor{green}{$\tau_3$} & \uncover<5->{$\tau_3$} & \uncover<6->{$\sigma$} & \uncover<6->{$\sigma^{-1}$} & \uncover<6->{$\id$} & \uncover<6->{$\tau_1$} & \uncover<6->{$\tau_2$} \\ \hline

\textcolor{green}{$\sigma$} & \uncover<5->{$\sigma$} & \uncover<3->{\textcolor{orange}{$\sigma \circ \tau_1=\tau_3$}} & \uncover<6->{$\tau_1$} & \uncover<6->{$\tau_2$} & \uncover<6->{$\sigma^{-1}$} & \uncover<6->{$\id$} \\ \hline

\textcolor{green}{$\sigma^{-1}$} & \uncover<5->{$\sigma^{-1}$} & \uncover<6->{$\tau_2$} & \uncover<6->{$\tau_3$} & \uncover<6->{$\tau_1$} & \uncover<6->{$\id$} & \uncover<6->{$\sigma$} \\
\end{tabular}
\end{center}

\end{frame}

%---------------------------------------------------------------
\section{Groupe des isométries du triangle}

\begin{frame}

\begin{minipage}{0.60\textwidth}
\myfigure{1}{
\tikzinput{fig_groupes01}
}
\pause\pause
Isométries $f$ telles que $f(A) \in \{A,B,C\}$,
$f(B) \in \{A,B,C\}$, $f(C) \in \{A,B,C\}$
\pause
\end{minipage}
 \begin{minipage}{0.39\textwidth}
\begin{itemize}
  \item l'identité $\id$
\pause
  \item la rotation $s$ d'angle $\frac{2\pi}{3}$ et 
la rotation $s^{-1}$ d'angle $-\frac{2\pi}{3}$
\pause
  \item les réflexions $t_1, t_2, t_3$ d'axes 
$\mathcal{D}_1, \mathcal{D}_2,\mathcal{D}_3$
\end{itemize}
\end{minipage}

\medskip
\pause

\begin{proposition}
L'ensemble des  isométries d'un triangle équilatéral, muni de la composition, forme un groupe.
\pause
Ce groupe est isomorphe à $(\mathcal{S}_3,\circ)$
\end{proposition}

\end{frame}

%---------------------------------------------------------------
\section{Décomposition en cycles}


\begin{frame}

Un \defi{cycle} est une permutation $\sigma$ 
\begin{itemize}
  \item qui fixe un certain nombre d'éléments ($\sigma(i)=i$)
  \item les autres sont obtenus par itération : $j, \sigma(j),\sigma^2(j),\ldots$
\end{itemize}


\pause
\bigskip 

\begin{exemple}
$$
\sigma = \left[\begin{matrix} 
 1 & \mathbf{2} & 3 & \mathbf{4} & \mathbf{5} & 6 & 7 & \mathbf{8} \\  
 1 & \mathbf{8} & 3 & \mathbf{5} & \mathbf{2} & 6 & 7 & \mathbf{4} \\      
        \end{matrix} \right]
$$
\pause
les éléments $1, 3, 6, 7$ sont fixes

\pause
les autres s'obtiennent comme itération de $2$ 

\pause
\centerline{$2 \mapsto \sigma(2)=8 \mapsto \sigma(8)=\sigma^2(2)=4 \mapsto \sigma(4)=\sigma^3(2) =  5$}

\centerline{$\sigma^4(2)=\sigma(5)=2$}

\medskip 
\pause

Notation : \hfil $(2\tikz[remember picture]\coordinate(cycun);\quad 8\tikz[remember picture]\coordinate(cycdeux);\quad
4\tikz[remember picture]\coordinate(cyctrois);\quad 5\tikz[remember picture]\coordinate(cycquatre);)$

\pause

\begin{tikzpicture}[x=1mm,y=1mm, remember picture, overlay]
   \coordinate (mycycun) at ($(cycun)+(-1,-1)$);
   \coordinate (mycycdeux) at ($(cycdeux)+(-1,-1)$);
   \coordinate (mycycdeuxbis) at ($(cycdeux)+(-1,-2)$);
   \coordinate (mycyctroisbis) at ($(cyctrois)+(-1,-2)$);
   \coordinate (mycyctrois) at ($(cyctrois)+(-1,-1)$);
   \coordinate (mycycquatre) at ($(cycquatre)+(-1,-1)$);
   \coordinate (mycycquatrebis) at ($(cycquatre)+(-1,+3)$);
   \coordinate (mycycunbis) at ($(cycun)+(-1,+3)$);

   \draw[->, myred, very thick] (mycycun) to[bend right, thick] (mycycdeux);
   \draw[->, myred, very thick] (mycycdeuxbis) to[bend right, thick] (mycyctroisbis);
   \draw[->, myred, very thick] (mycyctrois) to[bend right, thick] (mycycquatre);
   \draw[->, myred, very thick] (mycycquatrebis) to[bend right, thick] (mycycunbis);
\end{tikzpicture}

\pause
Ce même cycle est aussi noté $(8\ 4\ 5\ 2)$, $(4\ 5\ 2\ 8)$ ou $(5\ 2\ 8\ 4)$

\medskip 
\pause

L'inverse s'obtient en renversant l'ordre:
$\sigma^{-1}=(5\ 4\ 8\ 2)$  
\end{exemple}


\end{frame}

\begin{frame}

\begin{itemize}
  \item Le \defi{support} d'un cycle sont les éléments qui ne sont pas fixes 
\pause
  \item La \defi{longueur} (ou l'\defi{ordre}) d'un cycle est le nombre d'éléments qui ne sont pas fixes
\end{itemize}

\pause
Le support de $\sigma=(2\ 8\ 4\ 5)$ est $\{2, 4, 5, 8\}$, c'est un cycle de longueur $4$

\pause
\begin{exemple}
$
\sigma = \left[\begin{smallmatrix} 
 1 & 2 & 3 \\  
 2 & 3 & 1 \\      
        \end{smallmatrix} \right]
= (1\ 2\ 3)$ est un cycle de longueur $3$ 

\medskip
\pause

$
\tau = \left[\begin{smallmatrix} 
 1 & 2 & 3 & 4 \\  
 1 & 4 & 3 & 2 \\      
        \end{smallmatrix} \right]
= (2\ 4)$ est un cycle de longueur $2$ : une \defi{transposition}  
\end{exemple}

\pause
\begin{exemple}
$f =\left[\begin{smallmatrix} 
 1 & 2 & 3 & 4 & 5 & 6 & 7 \\  
 7 & 2 & 5 & 4 & 6 & 3 & 1 \\      
        \end{smallmatrix} \right]
$ n'est pas un cycle

\pause

Il s'écrit comme la composition de deux cycles $f = (1\ 7) \circ (3\ 5\ 6)$

\pause

Les supports de $(1\ 7)$ et $(3\ 5\ 6)$ sont disjoints donc $f = (3\ 5\ 6) \circ (1\ 7)$  
\end{exemple}

\end{frame}



\begin{frame}

\begin{theoreme}
Toute permutation de $\mathcal{S}_n$ se décompose en composition
de cycles à supports disjoints

\pause

De plus cette décomposition est unique
\end{theoreme}

\pause

\begin{itemize}
  \item Unique à l'écriture de chaque cycle près
\hfill $(3\ 5\ 6)=(5\ 6\ 3)$

\pause
  \item Unique à l'ordre près
\hfill $(1\ 7) \circ (3\ 5\ 6)= (3\ 5\ 6) \circ (1\ 7)$
\end{itemize}
\pause


\begin{exemple}
\centerline {$
f = \left[\begin{smallmatrix} 
 1 & 2 & 3 & 4 & 5 & 6 & 7 & 8 \\  
 5 & 2 & 1 & 8 & 3 & 7 & 6 & 4 \\      
        \end{smallmatrix} \right]
\pause =  (1\ 5\ 3) \circ \pause (4\ 8) \circ \pause (6\ 7)$
}
\end{exemple}

\pause
Si les supports ne sont pas disjoints alors cela ne commute plus !

\pause

\begin{exemple}
$g = (1\ 2)\circ(2\ 3\ 4) 
\pause
= 
\left[\begin{smallmatrix} 
 1 & 2 & 3 & 4 \\  
 1 & 3 & 4 & 2 \\   
 2 & 3 & 4 & 1 \\  
\end{smallmatrix} \right] 
\pause
=
\left[\begin{smallmatrix} 
 1 & 2 & 3 & 4 \\  
 2 & 3 & 4 & 1 \\      
\end{smallmatrix} \right] 
\pause =
(1\ 2\ 3\ 4)$

\pause

$h = (2\ 3\ 4)\circ(1\ 2) 
\pause = 
\left[\begin{smallmatrix} 
 1 & 2 & 3 & 4 \\  
 3 & 1 & 4 & 2 \\      
\end{smallmatrix} \right]
\pause 
= (1\ 3\ 4\ 2)$ 
\end{exemple}

\end{frame}







%---------------------------------------------------------------
\section*{Mini-exercices}


\begin{frame}

\begin{miniexercice}
{\small
\begin{enumerate}
  \item Soient $f$ définie par $f(1)=2$, $f(2)=3$, $f(3)=4$, $f(4)=5$, $f(5)=1$
et $g$ définie par $g(1)=2$, $g(2)=1$, $g(3)=4$, $g(4)=3$, $g(5)=5$. \'Ecrire les permutations
$f$, $g$, $f^{-1}$, $g^{-1}$, $g\circ f$, $f\circ g$, $f^2$, $g^2$, $(g\circ f)^2$.
  \item \'Enumérer toutes les permutations de $\mathcal{S}_4$ qui n'ont pas d'éléments fixes.
Les écrire ensuite sous le forme de compositions de cycles à supports disjoints.
  \item Trouver les isométries directes préservant le carré. Dresser la table des compositions
et montrer qu'elles forment un groupe. Montrer que ce groupe est isomorphe à $\Zz/4\Zz$.
  \item Montrer qu'il existe un sous-groupe de $\mathcal{S}_3$ isomorphe à $\Zz/2\Zz$.
Même question avec $\Zz/3\Zz$. Est-ce que $\mathcal{S}_3$ et $\Zz/6\Zz$ sont isomorphes ?
  \item Décomposer la permutation suivante en produit de cycles à supports disjoints :
$f=
\left[\begin{smallmatrix} 
 1 & 2 & 3 & 4 & 5 & 6 & 7 \\  
 5 & 7 & 2 & 6 & 1 & 4 & 3 \\      
\end{smallmatrix} \right]
$. 
Calculer $f^2$, $f^3$, $f^4$ puis $f^{20xx}$ où $20xx$ est l'année en cours. Mêmes questions avec 
$g=
\left[\begin{smallmatrix} 
 1 & 2 & 3 & 4 & 5 & 6 & 7 & 8 & 9 \\  
 3 & 8 & 9 & 6 & 5 & 2 & 4 & 7 & 1\\      
\end{smallmatrix} \right]$ et
$h = (2 5)(1 2 4 3)(1 2)$.
\end{enumerate}
}
\end{miniexercice}
\end{frame}



\end{document}