
%%%%%%%%%%%%%%%%%% PREAMBULE %%%%%%%%%%%%%%%%%%


\documentclass[12pt]{article}

\usepackage{amsfonts,amsmath,amssymb,amsthm}
\usepackage[utf8]{inputenc}
\usepackage[T1]{fontenc}
\usepackage[francais]{babel}


% packages
\usepackage{amsfonts,amsmath,amssymb,amsthm}
\usepackage[utf8]{inputenc}
\usepackage[T1]{fontenc}
%\usepackage{lmodern}

\usepackage[francais]{babel}
\usepackage{fancybox}
\usepackage{graphicx}

\usepackage{float}

%\usepackage[usenames, x11names]{xcolor}
\usepackage{tikz}
\usepackage{datetime}

\usepackage{mathptmx}
%\usepackage{fouriernc}
%\usepackage{newcent}
\usepackage[mathcal,mathbf]{euler}

%\usepackage{palatino}
%\usepackage{newcent}


% Commande spéciale prompteur

%\usepackage{mathptmx}
%\usepackage[mathcal,mathbf]{euler}
%\usepackage{mathpple,multido}

\usepackage[a4paper]{geometry}
\geometry{top=2cm, bottom=2cm, left=1cm, right=1cm, marginparsep=1cm}

\newcommand{\change}{{\color{red}\rule{\textwidth}{1mm}\\}}

\newcounter{mydiapo}

\newcommand{\diapo}{\newpage
\hfill {\normalsize  Diapo \themydiapo \quad \texttt{[\jobname]}} \\
\stepcounter{mydiapo}}


%%%%%%% COULEURS %%%%%%%%%%

% Pour blanc sur noir :
%\pagecolor[rgb]{0.5,0.5,0.5}
% \pagecolor[rgb]{0,0,0}
% \color[rgb]{1,1,1}



%\DeclareFixedFont{\myfont}{U}{cmss}{bx}{n}{18pt}
\newcommand{\debuttexte}{
%%%%%%%%%%%%% FONTES %%%%%%%%%%%%%
\renewcommand{\baselinestretch}{1.5}
\usefont{U}{cmss}{bx}{n}
\bfseries

% Taille normale : commenter le reste !
%Taille Arnaud
%\fontsize{19}{19}\selectfont

% Taille Barbara
%\fontsize{21}{22}\selectfont

%Taille François
\fontsize{25}{30}\selectfont

%Taille Pascal
%\fontsize{25}{30}\selectfont

%Taille Laura
%\fontsize{30}{35}\selectfont


%\myfont
%\usefont{U}{cmss}{bx}{n}

%\Huge
%\addtolength{\parskip}{\baselineskip}
}


% \usepackage{hyperref}
% \hypersetup{colorlinks=true, linkcolor=blue, urlcolor=blue,
% pdftitle={Exo7 - Exercices de mathématiques}, pdfauthor={Exo7}}


%section
% \usepackage{sectsty}
% \allsectionsfont{\bf}
%\sectionfont{\color{Tomato3}\upshape\selectfont}
%\subsectionfont{\color{Tomato4}\upshape\selectfont}

%----- Ensembles : entiers, reels, complexes -----
\newcommand{\Nn}{\mathbb{N}} \newcommand{\N}{\mathbb{N}}
\newcommand{\Zz}{\mathbb{Z}} \newcommand{\Z}{\mathbb{Z}}
\newcommand{\Qq}{\mathbb{Q}} \newcommand{\Q}{\mathbb{Q}}
\newcommand{\Rr}{\mathbb{R}} \newcommand{\R}{\mathbb{R}}
\newcommand{\Cc}{\mathbb{C}} 
\newcommand{\Kk}{\mathbb{K}} \newcommand{\K}{\mathbb{K}}

%----- Modifications de symboles -----
\renewcommand{\epsilon}{\varepsilon}
\renewcommand{\Re}{\mathop{\text{Re}}\nolimits}
\renewcommand{\Im}{\mathop{\text{Im}}\nolimits}
%\newcommand{\llbracket}{\left[\kern-0.15em\left[}
%\newcommand{\rrbracket}{\right]\kern-0.15em\right]}

\renewcommand{\ge}{\geqslant}
\renewcommand{\geq}{\geqslant}
\renewcommand{\le}{\leqslant}
\renewcommand{\leq}{\leqslant}

%----- Fonctions usuelles -----
\newcommand{\ch}{\mathop{\mathrm{ch}}\nolimits}
\newcommand{\sh}{\mathop{\mathrm{sh}}\nolimits}
\renewcommand{\tanh}{\mathop{\mathrm{th}}\nolimits}
\newcommand{\cotan}{\mathop{\mathrm{cotan}}\nolimits}
\newcommand{\Arcsin}{\mathop{\mathrm{Arcsin}}\nolimits}
\newcommand{\Arccos}{\mathop{\mathrm{Arccos}}\nolimits}
\newcommand{\Arctan}{\mathop{\mathrm{Arctan}}\nolimits}
\newcommand{\Argsh}{\mathop{\mathrm{Argsh}}\nolimits}
\newcommand{\Argch}{\mathop{\mathrm{Argch}}\nolimits}
\newcommand{\Argth}{\mathop{\mathrm{Argth}}\nolimits}
\newcommand{\pgcd}{\mathop{\mathrm{pgcd}}\nolimits} 

\newcommand{\Card}{\mathop{\text{Card}}\nolimits}
\newcommand{\Ker}{\mathop{\text{Ker}}\nolimits}
\newcommand{\id}{\mathop{\text{id}}\nolimits}
\newcommand{\ii}{\mathrm{i}}
\newcommand{\dd}{\mathrm{d}}
\newcommand{\Vect}{\mathop{\text{Vect}}\nolimits}
\newcommand{\Mat}{\mathop{\mathrm{Mat}}\nolimits}
\newcommand{\rg}{\mathop{\text{rg}}\nolimits}
\newcommand{\tr}{\mathop{\text{tr}}\nolimits}
\newcommand{\ppcm}{\mathop{\text{ppcm}}\nolimits}

%----- Structure des exercices ------

\newtheoremstyle{styleexo}% name
{2ex}% Space above
{3ex}% Space below
{}% Body font
{}% Indent amount 1
{\bfseries} % Theorem head font
{}% Punctuation after theorem head
{\newline}% Space after theorem head 2
{}% Theorem head spec (can be left empty, meaning ‘normal’)

%\theoremstyle{styleexo}
\newtheorem{exo}{Exercice}
\newtheorem{ind}{Indications}
\newtheorem{cor}{Correction}


\newcommand{\exercice}[1]{} \newcommand{\finexercice}{}
%\newcommand{\exercice}[1]{{\tiny\texttt{#1}}\vspace{-2ex}} % pour afficher le numero absolu, l'auteur...
\newcommand{\enonce}{\begin{exo}} \newcommand{\finenonce}{\end{exo}}
\newcommand{\indication}{\begin{ind}} \newcommand{\finindication}{\end{ind}}
\newcommand{\correction}{\begin{cor}} \newcommand{\fincorrection}{\end{cor}}

\newcommand{\noindication}{\stepcounter{ind}}
\newcommand{\nocorrection}{\stepcounter{cor}}

\newcommand{\fiche}[1]{} \newcommand{\finfiche}{}
\newcommand{\titre}[1]{\centerline{\large \bf #1}}
\newcommand{\addcommand}[1]{}
\newcommand{\video}[1]{}

% Marge
\newcommand{\mymargin}[1]{\marginpar{{\small #1}}}



%----- Presentation ------
\setlength{\parindent}{0cm}

%\newcommand{\ExoSept}{\href{http://exo7.emath.fr}{\textbf{\textsf{Exo7}}}}

\definecolor{myred}{rgb}{0.93,0.26,0}
\definecolor{myorange}{rgb}{0.97,0.58,0}
\definecolor{myyellow}{rgb}{1,0.86,0}

\newcommand{\LogoExoSept}[1]{  % input : echelle
{\usefont{U}{cmss}{bx}{n}
\begin{tikzpicture}[scale=0.1*#1,transform shape]
  \fill[color=myorange] (0,0)--(4,0)--(4,-4)--(0,-4)--cycle;
  \fill[color=myred] (0,0)--(0,3)--(-3,3)--(-3,0)--cycle;
  \fill[color=myyellow] (4,0)--(7,4)--(3,7)--(0,3)--cycle;
  \node[scale=5] at (3.5,3.5) {Exo7};
\end{tikzpicture}}
}



\theoremstyle{definition}
%\newtheorem{proposition}{Proposition}
%\newtheorem{exemple}{Exemple}
%\newtheorem{theoreme}{Théorème}
\newtheorem{lemme}{Lemme}
\newtheorem{corollaire}{Corollaire}
%\newtheorem*{remarque*}{Remarque}
%\newtheorem*{miniexercice}{Mini-exercices}
%\newtheorem{definition}{Définition}




%definition d'un terme
\newcommand{\defi}[1]{{\color{myorange}\textbf{\emph{#1}}}}
\newcommand{\evidence}[1]{{\color{blue}\textbf{\emph{#1}}}}



 %----- Commandes divers ------

\newcommand{\codeinline}[1]{\texttt{#1}}

%%%%%%%%%%%%%%%%%%%%%%%%%%%%%%%%%%%%%%%%%%%%%%%%%%%%%%%%%%%%%
%%%%%%%%%%%%%%%%%%%%%%%%%%%%%%%%%%%%%%%%%%%%%%%%%%%%%%%%%%%%%



\begin{document}

\debuttexte

%%%%%%%%%%%%%%%%%%%%%%%%%%%%%%%%%%%%%%%%%%%%%%%%%%%%%%%%%%%
\diapo

\change

\change

Nous commençons ce chapitre sur les groupes par la définition,

\change 

nous verrons ensuite plusieurs ensembles
que vous connaissez et qui ont une structure de groupe

\change

Nous ferons quelques calculs dans les groupes

\change

Et nous terminons par l'étude en détails
d'un groupe obtenu à partir des matrices $2 \times 2$.


%%%%%%%%%%%%%%%%%%%%%%%%%%%%%%%%%%%%%%%%%%%%%%%%%%%%%%%%%%%
\diapo

Racontons avant tout comment les\\ groupes sont apparus.

Vous savez résoudre les équation de degré $2$, $ax^2+bx+c=0$
les solutions sont les $x_i = \frac{-b\pm\sqrt{\Delta}}{2a}$
avec $\Delta = b^2-4ac$

\change

On sait aussi résoudre les équations de degré $3$, les solutions font
intervenir en plus des racines cubiques.

Par exemple une solution de $x^3+3x+1=0$ 
est de cette forme [[montrer $\sqrt[3]{\frac{\sqrt 5 - 1}{2}} -  \sqrt[3]{\frac{\sqrt 5 + 1}{2}}$]]


\change

Il existe également de telles formules pour les équation de degré $4$

\change

La question s'est longtemps posée si l'on pouvait faire de même
pour les équations de degré 5 et plus.

\change

La réponse fut apportée par Galois et Abel : il n'existe pas en général une telle  formule.

Galois sait même dire pour quels polynômes c'est possible
et pour lesquels ce ne l'est pas


\change

Il introduit pour cela la notion de groupe, il est alors âgé de 17 ans !

\change

Les groupes servent de fondement à beaucoup d'autres objets mathématiques :
les anneaux, les corps, les matrices, les espaces vectoriels,...

\change

Ils s'appliquent dans des domaines variés : de l'arithmétique à la cryptographie 
en passant par la géométrie.



%%%%%%%%%%%%%%%%%%%%%%%%%%%%%%%%%%%%%%%%%%%%%%%%%%%%%%%%%%%
\diapo

Un groupe est un ensemble $G$ auquel est associé une opération $\star$,

$G$ et la loi de composition étoile doivent vérifier quatre propriétés :

\change

l'opération $\star$ est une loi de composition interne

c'est à dire 

 pour tout $x,y \in G$, \quad $x \star y \in G$ 

Autrement dit si l'on compose deux éléments de $G$ le résultat est encore
un élément de $G$

\change


la loi $\star$  est associative


c'est-à-dire  $(x \star y) \star z = x \star (y \star z)$ 

on peut donc déplacer ou omettre les parenthèses.

\change

Il existe un élément petit $e$ de $G$

qui vérifie $x \star e = x$ et $e \star x = x$ pour tous les $x$ de $G$


ce $e$ s'appelle l'élément neutre,
car $e$ composé avec n'importe quel élément, conserve cet élément.


\change

Enfin tout élément $x$ admet un inverse pour la loi $\star$ :
c'est à dire qu'il existe un autre élément $x'$ de $G$
tel que $x \star x' =e$ et $x' \star x = e$ 

Ce $x'$ s'appelle l'inverse de $x$ et est noté [[$x^{-1}$]] $x$ à la puissance $-1$


Ces quatre propriétés font de $G$ avec la loi $\star$ un groupe.

\change

On dira d'un groupe $G$ qu'il est commutatif si en plus

$\text{pour tout } x,y   \qquad x \star y = y \star x$

Mais attention, cette dernière égalité n'est pas vraie pour tous les groupes !


%%%%%%%%%%%%%%%%%%%%%%%%%%%%%%%%%%%%%%%%%%%%%%%%%%%%%%%%%%%
\diapo

Nous allons voir que des ensembles que vous connaissez déjà on en fait
une structure de groupe.

Tout d'abord l'ensemble $\Rr^*$ des réels non nuls auquel on associe la loi $\times$ qui est la multiplication
habituelle est un groupe commutatif.

\change

On a bien que le produit de deux nombres réels non nuls est un nombre réel non nul,

\change

La multiplication est bien associative,

\change

il existe un élément neutre : c'est $1$. En effet $1\times x=x$
et $x\times 1 = x$.

\change


Tout réel non nul admet un inverse, c'est $x'= \frac{1}{x}$ 

en effet $x \times \frac{1}{x}$ égal l'élément neutre $1$ (et $\frac{1}{x}\times x=1$)


Cela fait de $(\Rr^*,\times)$ un groupe.

\change

En plus, pour tout $x,y$ on a $x\times y=y\times x$,
donc  $(\Rr^*,\times)$ est un groupe commutatif.


\change

On fait la même chose avec l'ensemble $\Zz$ des entiers relatifs mais cette fois
le loi est l'addition habituelle $+$.

\change

On a bien $x+y \in \Zz$

\change

La loi $+$ est associative

\change

L'élément neutre est cette fois $0$ car $0+x=x$ (et $x+0=x$)

\change

L'inverse d'un entier $x$ est $x'=-x$ car $x+(-x)$ égale l'élément neutre $0$.

Lorsque la loi est $+$ on appellera $-x$ l'opposé plutôt que l'inverse.

$(\Zz,+)$ est un groupe,

\change

il est même commutatif car $x+y=y+x$ pour tout $x,y$.

%%%%%%%%%%%%%%%%%%%%%%%%%%%%%%%%%%%%%%%%%%%%%%%%%%%%%%%%%%%
\diapo

Montrez de la même façon que l'ensemble 

 $\Qq^*$ des rationnels non nuls muni de la multiplication est un groupe commutatif

  idem avec les nombres complexes non nuls muni de la multiplication des\\ nombres complexes.

\change

Pour la loi d'addition les ensemble $\Qq$, $\Rr$, $\Cc$ sont des groupes commutatifs.

\change

Il est important de comprendre pourquoi les ensembles suivants ne sont pas 
des groupes.

Prenons l'ensemble $\Zz^*$ des entiers relatifs non nuls avec la multiplication habituelle

ce n'est pas un groupe car l'une des propriétés n'est pas vérifiée :
l'existence de l'inverse.

Par exemple si l'inverse de $2$ existait ce devrait être $\frac 12$,
mais $\frac 12$ n'est pas un entier. Donc $2$ n'a pas d'inverse.


Même chose avec $\Nn$ muni de l'addition, si $3$ admettait un inverse pour l'addition ce serait $-3$
mais $-3$ n'est pas dans $\Nn$. Ainsi $(\Nn,+)$ n'est pas un groupe.



%%%%%%%%%%%%%%%%%%%%%%%%%%%%%%%%%%%%%%%%%%%%%%%%%%%%%%%%%%%
\diapo

Etudions maintenant des exemples géométriques.

Soit $\mathcal{R}$ l'ensemble des rotations du plan dont le centre est à l'origine $O$

la loi $\circ$ est ici la composition des fonctions

Vérifions que $\mathcal{R}$ muni de cette loi est un groupe commutatif.

\change

Une rotation dont le centre est à l'origine est caractérisée par un angle $\theta$.

\change

Si on compose deux rotations  $R_\theta$ avec $R_{\theta'}$ cela reste une rotation 
dont le centre est l'origine.

\change

La loi de composition des fonctions est associative. 

\change

Il existe un élément neutre : c'est la rotation d'angle $0$
qui est aussi l'identité  : l'image de tout point $P$ est $P$ lui même.

\change

Si l'on compose une rotation d'angle $\theta$ par la rotation d'angle $-\theta$

alors un point $P$ s'envoie sur lui même, c'est donc que la rotation d'angle $-\theta$
est l'inverse de la rotation d'angle $\theta$ pour notre loi $\circ$ de composition.

\change

Enfin ce groupe est commutatif  :

$R_\theta \circ R_{\theta'}=R_{\theta'} \circ R_{\theta}$

car nos rotations sont toutes centrées à l'origine.


%%%%%%%%%%%%%%%%%%%%%%%%%%%%%%%%%%%%%%%%%%%%%%%%%%%%%%%%%%%
\diapo

Continuons avec la géométrie :

soit $\mathcal{I}$ l'ensemble de isométries du plan 

ce sont les translations, les rotations (de centre quelconque)
les réflexions par rapports à des droites et les composées de toutes ces transformations.

La loi $\circ$ est une nouvelle fois la composition des fonctions.

Alors $(\mathcal{I},\circ)$ est un groupe mais il n'est pas commutatif.

\change

Admettons que c'est un groupe et montrons la non-commutativité :

Prenons par exemple 

  \begin{itemize}
     \item $R$ la rotation centrée à l'origine,\\ d'angle $\frac \pi 2$
     \item $T$ la translation de vecteur $(1,0)$
     \item alors les isométries  $T \circ R$ et $R \circ T$ ne sont pas égales

\change

     \item il suffit de choisir un point et de calculer ses images,
on prend par exemple $A$ le point de coordonnées $(1,1)$.


$T \circ R (1,1) = T (-1,1) = (0,1)$

 $R\circ T(1,1) = R(2,1) = (-1,2)$,

  \end{itemize}

Les images sont distinctes donc les isométries ne sont pas égales.

Le groupe des isométries n'est pas commutatif.

Dans un groupe non commutatif il faut faire attention 
à l'ordre des compositions.



%%%%%%%%%%%%%%%%%%%%%%%%%%%%%%%%%%%%%%%%%%%%%%%%%%%%%%%%%%%
\diapo

Revenons au cadre général d'un groupe $G$ quelconque, avec une loi $\star$.

Nous notons $x \star x$ par $x^2$,

 $x\star x \star x$ par $x^3$

et plus généralement 
 $x^{n} = x\star x \star \cdots \star x$ avec $n$ occurrences de $x$


Par définition $x^0$ vaut $e$ l'élément\\ neutre du groupe $G$

Enfin $x^{-n} = x^{-1}\star \cdots \star x^{-1}$ ou $x^{-1}$ est l'inverse
de $x$ dans le groupe.


\change

Avec ces notations les règles de calcul sont les mêmes
qu'avec les puissances usuelles :
 
\begin{itemize}
  \item $x^m \star x^n = x^{m+n}$
  \item $x$ puissance $m$ le tout  à la puissance $n$ vaut $x$ à la puisance $m$ fois $n$
[[$(x^m)^n = x^{mn}$]]
  \item l'inverse de $x\star y$ c'est $y^{-1} \star x^{-1}$
Notez bien l'inversion de l'ordre.
[[$(x \star y)^{-1} = y^{-1} \star x^{-1}$]]
  \item Enfin si $G$ est commutatif alors $(x\star y)^n = x^n \star y^n$

mais attention si $G$ n'est pas commutatif $(x\star y)^n$ vaut seulement
$(x\star y) \star (x\star y) \star  $, $n$ fois
\end{itemize} 

%%%%%%%%%%%%%%%%%%%%%%%%%%%%%%%%%%%%%%%%%%%%%%%%%%%%%%%%%%%
\diapo

Nous allons maintenant étudier un groupe, basé sur les matrices et leur multiplication.

Une matrice $2\times 2$ est simplement un tableau de $4$ nombres réels
$a,b,c,d$ que l'on note ainsi.

\change

Si l'on a deux matrices $M$ et $M'$ alors le produit de $M$ par $M'$
est encore une matrice 

\change

qui est définie par la formule suivante.

\change

Voici comment présenter les calculs, on place $M$ à gauche,

\change

et $M'$ au dessus de ce qui va être le résultat.

\change

On calcule un par un, chacun des termes de $M \times M'$.

\change

Reprenons pas à pas.


Pour le premier terme on prend la colonne situé au dessus et la ligne
située à gauche, 

\change

\change

\change

on effectue les produits $a\times a'$ et $b\times c'$ et additionne
pour obtenir le premier terme.

\change

Même chose avec le second terme : la colonne situé au dessus, la ligne
située à gauche, les produits, on additionne.

\change

Idem pour les deux autres termes.

\change


%%%%%%%%%%%%%%%%%%%%%%%%%%%%%%%%%%%%%%%%%%%%%%%%%%%%%%%%%%%
\diapo

Voyons les calculs sur un exemples.

$M$ est la matrice $M=\left(\begin{matrix} 1 & 1 \\ 0 & -1 \\ \end{matrix}\right)$

et $M' = \left(\begin{matrix} 1 & 0 \\ 2 & 1 \\ \end{matrix}\right)$

\change

Calculons d'abord $M \times M'$

\change

$$\left(\begin{matrix} 3 & 1 \\ -2 & -1 \\ \end{matrix}\right)$$

\change

Calculons $M' \times M$

$$\left(\begin{matrix} \ 1\  & \ 1\  \\ \ 2\  & \ 1\  \\ \end{matrix}\right) $$

\change

Nous constatons que les matrices $M \times M'$ et $M' \times M$
sont distinctes. 

\change

Le produit de matrices n'est pas commutatifs.



%%%%%%%%%%%%%%%%%%%%%%%%%%%%%%%%%%%%%%%%%%%%%%%%%%%%%%%%%%%
\diapo

Introduisons le déterminant d'une matrice $2\times2$.

La déterminant de la matrice $a,b,c,d$ est le nombre réel $ad-bc$.

\change

Une propriété facile à vérifier est que que le déterminant de $M \times M'$
est égal à déterminant de $M$ fois déterminant de $M'$.

\change

Pour la preuve, on se remémore d'abord ce que vaut la matrice $M\times M'$.

\change

Il ne reste plus qu'à calculer les trois déterminants : celui de 
$M\times M'$, celui de $M$, celui de $M'$ 

et de vérifier la formule.


%%%%%%%%%%%%%%%%%%%%%%%%%%%%%%%%%%%%%%%%%%%%%%%%%%%%%%%%%%%
\diapo

Voici notre dernier exemple de groupe.

Nous considérons l'ensemble des matrices ayant un déterminant non nul, 

la loi est la multiplication des matrices.

Alors cela forme un groupe, qui n'est pas commutatif.

\change

Ce groupe se note $G\ell_2$

\change

Effectuons la preuve.

Le produit de deux matrices est encore une matrice

et si les déterminants de $M$ et $M'$ sont nuls alors
le déterminant de $M\times M'$ est non nul grâce au lemme précédent.

\change

La multiplication des matrices est associative.

Il faut vérifier que pour trois matrices on a bien : $(M\times M')\times M''=M \times (M'\times M'')$

c'est un peu fastidieux mais c'est un bon exercice.

\change

L'élément neutre est la matrice identité, noté $I$,
avec des $1$ sur la diagonale et des zéros ailleurs.

Vérifier que $M$ fois $I$ égal $M$ et aussi que $I$ fois $M$ égal $M$.

\change


Enfin il existe un inverse de $M$ pour la multiplication des matrices.

On définit $M^{-1} = \frac{1}{ad-bc} \left(\begin{smallmatrix} d & -b \\ -c & a\\ \end{smallmatrix}\right)$

(remarquez que l'on divise chaque coefficient par le déterminant qui est non nul)

Alors vérifiez que  $M$ fois $M^{-1}$ égal $I$ (l'élément neutre)

et que $M^{-1}$ fois $M$ égal $I$.




Cela fait des matrices à déterminant non nul un groupe.

Et nous avions déjà vu que le produit de matrices n'était pas commutatif.

%%%%%%%%%%%%%%%%%%%%%%%%%%%%%%%%%%%%%%%%%%%%%%%%%%%%%%%%%%%
\diapo

La notion de groupe est vraiment difficile, prenez le temps
de bien comprendre les définitions et les exemples de cette vidéo.
Avant de passer à la suite.

Pour vous aider à comprendre voici deux pages d'exercices !



%%%%%%%%%%%%%%%%%%%%%%%%%%%%%%%%%%%%%%%%%%%%%%%%%%%%%%%%%%%
\diapo


Et voici la suite des exercices...


\end{document}