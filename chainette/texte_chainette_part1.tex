
%%%%%%%%%%%%%%%%%% PREAMBULE %%%%%%%%%%%%%%%%%%


\documentclass[12pt]{article}

\usepackage{amsfonts,amsmath,amssymb,amsthm}
\usepackage[utf8]{inputenc}
\usepackage[T1]{fontenc}
\usepackage[francais]{babel}


% packages
\usepackage{amsfonts,amsmath,amssymb,amsthm}
\usepackage[utf8]{inputenc}
\usepackage[T1]{fontenc}
%\usepackage{lmodern}

\usepackage[francais]{babel}
\usepackage{fancybox}
\usepackage{graphicx}

\usepackage{float}

%\usepackage[usenames, x11names]{xcolor}
\usepackage{tikz}
\usepackage{datetime}

\usepackage{mathptmx}
%\usepackage{fouriernc}
%\usepackage{newcent}
\usepackage[mathcal,mathbf]{euler}

%\usepackage{palatino}
%\usepackage{newcent}


% Commande spéciale prompteur

%\usepackage{mathptmx}
%\usepackage[mathcal,mathbf]{euler}
%\usepackage{mathpple,multido}

\usepackage[a4paper]{geometry}
\geometry{top=2cm, bottom=2cm, left=1cm, right=1cm, marginparsep=1cm}

\newcommand{\change}{{\color{red}\rule{\textwidth}{1mm}\\}}

\newcounter{mydiapo}

\newcommand{\diapo}{\newpage
\hfill {\normalsize  Diapo \themydiapo \quad \texttt{[\jobname]}} \\
\stepcounter{mydiapo}}


%%%%%%% COULEURS %%%%%%%%%%

% Pour blanc sur noir :
%\pagecolor[rgb]{0.5,0.5,0.5}
% \pagecolor[rgb]{0,0,0}
% \color[rgb]{1,1,1}



%\DeclareFixedFont{\myfont}{U}{cmss}{bx}{n}{18pt}
\newcommand{\debuttexte}{
%%%%%%%%%%%%% FONTES %%%%%%%%%%%%%
\renewcommand{\baselinestretch}{1.5}
\usefont{U}{cmss}{bx}{n}
\bfseries

% Taille normale : commenter le reste !
%Taille Arnaud
%\fontsize{19}{19}\selectfont

% Taille Barbara
%\fontsize{21}{22}\selectfont

%Taille François
\fontsize{25}{30}\selectfont

%Taille Pascal
%\fontsize{25}{30}\selectfont

%Taille Laura
%\fontsize{30}{35}\selectfont


%\myfont
%\usefont{U}{cmss}{bx}{n}

%\Huge
%\addtolength{\parskip}{\baselineskip}
}


% \usepackage{hyperref}
% \hypersetup{colorlinks=true, linkcolor=blue, urlcolor=blue,
% pdftitle={Exo7 - Exercices de mathématiques}, pdfauthor={Exo7}}


%section
% \usepackage{sectsty}
% \allsectionsfont{\bf}
%\sectionfont{\color{Tomato3}\upshape\selectfont}
%\subsectionfont{\color{Tomato4}\upshape\selectfont}

%----- Ensembles : entiers, reels, complexes -----
\newcommand{\Nn}{\mathbb{N}} \newcommand{\N}{\mathbb{N}}
\newcommand{\Zz}{\mathbb{Z}} \newcommand{\Z}{\mathbb{Z}}
\newcommand{\Qq}{\mathbb{Q}} \newcommand{\Q}{\mathbb{Q}}
\newcommand{\Rr}{\mathbb{R}} \newcommand{\R}{\mathbb{R}}
\newcommand{\Cc}{\mathbb{C}} 
\newcommand{\Kk}{\mathbb{K}} \newcommand{\K}{\mathbb{K}}

%----- Modifications de symboles -----
\renewcommand{\epsilon}{\varepsilon}
\renewcommand{\Re}{\mathop{\text{Re}}\nolimits}
\renewcommand{\Im}{\mathop{\text{Im}}\nolimits}
%\newcommand{\llbracket}{\left[\kern-0.15em\left[}
%\newcommand{\rrbracket}{\right]\kern-0.15em\right]}

\renewcommand{\ge}{\geqslant}
\renewcommand{\geq}{\geqslant}
\renewcommand{\le}{\leqslant}
\renewcommand{\leq}{\leqslant}

%----- Fonctions usuelles -----
\newcommand{\ch}{\mathop{\mathrm{ch}}\nolimits}
\newcommand{\sh}{\mathop{\mathrm{sh}}\nolimits}
\renewcommand{\tanh}{\mathop{\mathrm{th}}\nolimits}
\newcommand{\cotan}{\mathop{\mathrm{cotan}}\nolimits}
\newcommand{\Arcsin}{\mathop{\mathrm{Arcsin}}\nolimits}
\newcommand{\Arccos}{\mathop{\mathrm{Arccos}}\nolimits}
\newcommand{\Arctan}{\mathop{\mathrm{Arctan}}\nolimits}
\newcommand{\Argsh}{\mathop{\mathrm{Argsh}}\nolimits}
\newcommand{\Argch}{\mathop{\mathrm{Argch}}\nolimits}
\newcommand{\Argth}{\mathop{\mathrm{Argth}}\nolimits}
\newcommand{\pgcd}{\mathop{\mathrm{pgcd}}\nolimits} 

\newcommand{\Card}{\mathop{\text{Card}}\nolimits}
\newcommand{\Ker}{\mathop{\text{Ker}}\nolimits}
\newcommand{\id}{\mathop{\text{id}}\nolimits}
\newcommand{\ii}{\mathrm{i}}
\newcommand{\dd}{\mathrm{d}}
\newcommand{\Vect}{\mathop{\text{Vect}}\nolimits}
\newcommand{\Mat}{\mathop{\mathrm{Mat}}\nolimits}
\newcommand{\rg}{\mathop{\text{rg}}\nolimits}
\newcommand{\tr}{\mathop{\text{tr}}\nolimits}
\newcommand{\ppcm}{\mathop{\text{ppcm}}\nolimits}

%----- Structure des exercices ------

\newtheoremstyle{styleexo}% name
{2ex}% Space above
{3ex}% Space below
{}% Body font
{}% Indent amount 1
{\bfseries} % Theorem head font
{}% Punctuation after theorem head
{\newline}% Space after theorem head 2
{}% Theorem head spec (can be left empty, meaning ‘normal’)

%\theoremstyle{styleexo}
\newtheorem{exo}{Exercice}
\newtheorem{ind}{Indications}
\newtheorem{cor}{Correction}


\newcommand{\exercice}[1]{} \newcommand{\finexercice}{}
%\newcommand{\exercice}[1]{{\tiny\texttt{#1}}\vspace{-2ex}} % pour afficher le numero absolu, l'auteur...
\newcommand{\enonce}{\begin{exo}} \newcommand{\finenonce}{\end{exo}}
\newcommand{\indication}{\begin{ind}} \newcommand{\finindication}{\end{ind}}
\newcommand{\correction}{\begin{cor}} \newcommand{\fincorrection}{\end{cor}}

\newcommand{\noindication}{\stepcounter{ind}}
\newcommand{\nocorrection}{\stepcounter{cor}}

\newcommand{\fiche}[1]{} \newcommand{\finfiche}{}
\newcommand{\titre}[1]{\centerline{\large \bf #1}}
\newcommand{\addcommand}[1]{}
\newcommand{\video}[1]{}

% Marge
\newcommand{\mymargin}[1]{\marginpar{{\small #1}}}



%----- Presentation ------
\setlength{\parindent}{0cm}

%\newcommand{\ExoSept}{\href{http://exo7.emath.fr}{\textbf{\textsf{Exo7}}}}

\definecolor{myred}{rgb}{0.93,0.26,0}
\definecolor{myorange}{rgb}{0.97,0.58,0}
\definecolor{myyellow}{rgb}{1,0.86,0}

\newcommand{\LogoExoSept}[1]{  % input : echelle
{\usefont{U}{cmss}{bx}{n}
\begin{tikzpicture}[scale=0.1*#1,transform shape]
  \fill[color=myorange] (0,0)--(4,0)--(4,-4)--(0,-4)--cycle;
  \fill[color=myred] (0,0)--(0,3)--(-3,3)--(-3,0)--cycle;
  \fill[color=myyellow] (4,0)--(7,4)--(3,7)--(0,3)--cycle;
  \node[scale=5] at (3.5,3.5) {Exo7};
\end{tikzpicture}}
}



\theoremstyle{definition}
%\newtheorem{proposition}{Proposition}
%\newtheorem{exemple}{Exemple}
%\newtheorem{theoreme}{Théorème}
\newtheorem{lemme}{Lemme}
\newtheorem{corollaire}{Corollaire}
%\newtheorem*{remarque*}{Remarque}
%\newtheorem*{miniexercice}{Mini-exercices}
%\newtheorem{definition}{Définition}




%definition d'un terme
\newcommand{\defi}[1]{{\color{myorange}\textbf{\emph{#1}}}}
\newcommand{\evidence}[1]{{\color{blue}\textbf{\emph{#1}}}}



 %----- Commandes divers ------

\newcommand{\codeinline}[1]{\texttt{#1}}

%%%%%%%%%%%%%%%%%%%%%%%%%%%%%%%%%%%%%%%%%%%%%%%%%%%%%%%%%%%%%
%%%%%%%%%%%%%%%%%%%%%%%%%%%%%%%%%%%%%%%%%%%%%%%%%%%%%%%%%%%%%



\begin{document}

\debuttexte


%%%%%%%%%%%%%%%%%%%%%%%%%%%%%%%%%%%%%%%%%%%%%%%%%%%%%%%%%%%
\diapo

\change

\change
Dans cette série nous allons étudier la chaînette, 
qui est la courbe obtenue par une corde ou une chaîne.

\change
La fonction correspondante est le cosinus hyperbolique 
que nous allons étudier dans cette première
partie.

\change
Nous en profiterons pour étudier la bijection réciproque qui s'appelle l'argument
cosinus hyperbolique.

\change
On terminera par une discussion importante pour la suite :
pourquoi en physique on note les dérivées $df/dx$
alors qu'en maths on note $f'(x)$.


%%%%%%%%%%%%%%%%%%%%%%%%%%%%%%%%%%%%%%%%%%%%%%%%%%%%%%%%%%%
\diapo


La \defi{chaînette} est le nom que porte la courbe obtenue en tenant 
une corde (ou un collier, un fil,\ldots) par deux extrémités.

\change
Sans plus tarder voici l'équation d'une chaînette :
$$y = a \ch\left( \frac x a \right)$$

Ici <<$\ch$>> désigne le cosinus hyperbolique.

\change
Le paramètre $a$ dépend de la chaînette : on peut écarter plus ou moins les mains.
Ou, ce qui revient au même, si l'on garde les mains fixes, on peut prendre des cordes de différentes longueurs.

\change

\change

\change

\change

Le cosinus hyperbolique, $\ch$, 
est défini à partir de la fonction exponentielle :
Ainsi $y =  a \frac{\left( e^{\frac x a} + e^{-\frac x a} \right)}{2},$
 nous y reviendrons.





%%%%%%%%%%%%%%%%%%%%%%%%%%%%%%%%%%%%%%%%%%%%%%%%%%%%%%%%%%%
\diapo


La chaînette est donc une courbe que vous voyez tous les jours :
la chaîne qui pend à votre cou ou le fil électrique entre deux pylônes.

\change
Voici un exemple et avec le choix du bon paramètre 
on retrouve la bonne courbe.


\change
Mais on  retrouve cette courbe de la chaînette 
dans des endroits plus surprenants : 
vous pouvez voir des chaînettes avec des films de savon. 
Trempez deux cercles métalliques parallèles dans de l'eau savonneuse.
Il en sort une surface de révolution dont le profil est...

\change
...une chaînette !

\change
Enfin, si vous souhaitez faire une arche qui s'appuie sur deux piles 
alors la forme la plus stable est une chaînette renversée.

\change
Gaudi a beaucoup utilisé cette forme dans les bâtiments
qu'il a construits.



%%%%%%%%%%%%%%%%%%%%%%%%%%%%%%%%%%%%%%%%%%%%%%%%%%%%%%%%%%%
\diapo



Le \defi{cosinus hyperbolique} est définie à l'aide de la fonction exponentielle 

\change
dont voici le graphe.


Le \defi{cosinus hyperbolique} est donc la partie paire de l'exponentielle
c'est-à-dire qu'il est défini par la formule :

$\ch x = \frac{e^x + e^{-x}}{2}$,

\change
Voici son graphe. C'est une fonction paire, qui vaut $1$ en $x=0$,
et se comporte comme l'exponentielle en $+\infty$.

\change
Le \defi{sinus hyperbolique} est la partie impaire de l'exponentielle :
$\sh x = \frac{e^x - e^{-x}}{2}.$

\change
C'est une fonction impaire
qui vaut $0$ en $x=0$.

Et ce comporte aussi comme l'exponentielle en $+\infty$.

%%%%%%%%%%%%%%%%%%%%%%%%%%%%%%%%%%%%%%%%%%%%%%%%%%%%%%%%%%%
\diapo

Voici quelques propriétés dont nous aurons besoin :

Tout d'abord $\ch^2 x - \sh^2 x = 1$, ceci pour tout $x$ réel.

\change
Ensuite la dérivée de $\ch$ est $\sh$
et inversement la dérivée de $\sh$ et $\ch$.

\change
Voyons rapidement les preuves.

\change
Pour calculer $\ch^2 x- \sh^2 x$ nous revenons à la définition 

\change
du cosinus hyperbolique et du sinus hyperbolique,

\change
on développe les carrés

\change
et les expressions se simplifient pour donner $1$.

\change
Pour calculer la dérivée du cosinus hyperbolique c'est très simple,

la dérivée de $\ch x$ 

\change
est donc la dérivée de $\frac{e^x+e^{-x}}{2}$

\change
ce qui donne $\frac{e^x-e^{-x}}{2}$

\change
et vaut donc exactement $\sh x$.

La dérivée du sinus hyperbolique s'obtient de façon similaire.


%%%%%%%%%%%%%%%%%%%%%%%%%%%%%%%%%%%%%%%%%%%%%%%%%%%%%%%%%%%
\diapo

Le nom cosinus hyperbolique et sinus hyperbolique ne sont pas un hasard :
souvenez-vous des formules d'Euler pour le cosinus et sinus classiques (dits aussi <<circulaires>>) :
$$\cos x = \frac{e^{\ii x} + e^{-\ii x}}{2}, \qquad \sin x = \frac{e^{\ii x} - e^{-\ii x}}{2\ii}.$$
L'analogie avec la définition de $\ch x$ et $\sh x$ justifie les termes <<cosinus>> et <<sinus>>. 


Reste à justifier le terme <<hyperbolique>>.

\change

\change
Si nous dessinons une courbe paramétrée par 
$(x(t) = \cos t,y(t) = \sin t)$ 

\change
alors
$x(t)^2+y(t)^2 = \cos^2 t + \sin^2 t =1$. 

\change

Donc nous avons affaire à un cercle
(d'où le terme <<circulaire>>). 

\change
Par contre si on dessine une courbe paramétrée 
par $(x(t) = \ch t, y(t) = \sh t)$.

\change
On obtient la courbe suivante.

\change
Et par la proposition précédente on a 
$x(t)^2 - y(t)^2 = \ch^2 t - \sh^2 t = 1$.

\change
C'est donc l'équation d'une branche d'hyperbole ! 


%%%%%%%%%%%%%%%%%%%%%%%%%%%%%%%%%%%%%%%%%%%%%%%%%%%%%%%%%%%
\diapo


Etudions un peu plus les fonction cosinus et sinus hyperbolique.


La fonction $x \mapsto \ch x$ est une bijection
de $[0,+\infty[$ dans $[1,+\infty[$. 
Attention il faut se restreindre au $x\ge0$ pour avoir une bijection.

\change
Cette restriction étant bijective, on note sa bijection réciproque par 
$\Argch x$.

\change
Le graphe de $\Argch$ s'obtient par symétrie du graphe de $\ch$ par rapport à la bissectrice
$(y=x)$.

\change
Par définition $\Argch$ vérifie :

$\ch\big(\Argch(x)\big) = x$

\change
et $\Argch\big(\ch(x)\big) = x$


\change 
De même, la fonction $x \mapsto \sh x$ est une bijection
de $\Rr$ dans $\Rr$. Ici pas besoin de restreindre les intervalles.
Sa bijection réciproque est 
notée $\Argsh x$.

Voici son graphe.

Et on a 
$\sh\big(\Argsh(x)\big) = x$ et 
$\Argsh\big(\sh(x)\big) = x$

%%%%%%%%%%%%%%%%%%%%%%%%%%%%%%%%%%%%%%%%%%%%%%%%%%%%%%%%%%%
\diapo

Pour résoudre une équation différentielle nous aurons besoin 
de la dérivée de $\Argsh x$.


Proposition :
Les fonctions $\Argch$ et $\Argsh$
sont dérivables et 
$$\Argch' x = \frac{1}{\sqrt{x^2-1}} \qquad\qquad \Argsh' x = \frac{1}{\sqrt{x^2+1}}.$$


Par exemple cette formule se prouve en dérivant l'égalité $\ch(\Argch x) = x$.



%%%%%%%%%%%%%%%%%%%%%%%%%%%%%%%%%%%%%%%%%%%%%%%%%%%%%%%%%%%
\diapo

En fait, les fonctions hyperboliques inverses 
peuvent s'exprimer à l'aide des fonctions usuelles :

$\Argch x  =  \ln\left( x+\sqrt{x^2-1} \right)$

$\Argsh x  =  \ln\left( x+\sqrt{x^2+1} \right)$

On vérifie ces formules en dérivant. [Montrer argsh]. On dérive des deux côtés, 
les deux dérivées sont égales, et elles prennent la même valeur en $0$ donc les fonctions sont égales.


%%%%%%%%%%%%%%%%%%%%%%%%%%%%%%%%%%%%%%%%%%%%%%%%%%%%%%%%%%%
\diapo



Deux notations pour la dérivée s'affrontent : 


celle du mathématicien $f'(x)$ et celle du physicien $\frac{df}{dx}$. 
Comparons-les.

\change
La dérivée de $f$ en $x$ est par définition la limite (si elle existe) 
du taux d'accroissement :
$$\frac{f(x+h)-f(x)}{x+h-x},$$
ce qui fait $h$ [montrer le dénominateur]

lorsque $h$ tend vers $0$.

\change
Notons $dx = h$ et $df = f(x+h)-f(x) = f(x+dx)-f(x)$ 

\change
alors le taux d'accroissement vaut
$\frac{df}{dx}$ 

\change
et comme $dx$ est un nombre aussi petit que l'on veut 
(on dit qu'il est \emph{infinitésimal}),

\change
on identifie ce quotient $\frac{df}{dx}$ 
avec sa limite lorsque $dx \to 0$.

L'avantage de la notation des physiciens est que cela peut correspondre à un raisonnement physique.
On peut raisonner sur des petits morceaux (de longueur $dx$ petite mais pas nulle) et en déduire 
une relation avec des dérivées. C'est ce que nous ferons pour le calcul de l'équation de la chaînette.

\change
Autre avantage de cette notation, 

\change
il est facile de retenir la formule :
$$\frac{df}{dx} = \frac{dy}{dx}\times\frac{df}{dy}.$$
Il s'agit juste de <<simplifier>> le numérateur avec le dénominateur.

\change
Cette opération est justifiée, car il s'agit de la dérivée de la composée 
$f \circ y \; (x) = f\big( y(x) \big)$

\change
qui est bien 
$$\left(f \circ y \right)'(x) = y'(x) \times f'\big( y(x) \big).$$




\end{document}
