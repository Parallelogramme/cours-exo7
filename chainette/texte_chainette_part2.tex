
%%%%%%%%%%%%%%%%%% PREAMBULE %%%%%%%%%%%%%%%%%%


\documentclass[12pt]{article}

\usepackage{amsfonts,amsmath,amssymb,amsthm}
\usepackage[utf8]{inputenc}
\usepackage[T1]{fontenc}
\usepackage[francais]{babel}


% packages
\usepackage{amsfonts,amsmath,amssymb,amsthm}
\usepackage[utf8]{inputenc}
\usepackage[T1]{fontenc}
%\usepackage{lmodern}

\usepackage[francais]{babel}
\usepackage{fancybox}
\usepackage{graphicx}

\usepackage{float}

%\usepackage[usenames, x11names]{xcolor}
\usepackage{tikz}
\usepackage{datetime}

\usepackage{mathptmx}
%\usepackage{fouriernc}
%\usepackage{newcent}
\usepackage[mathcal,mathbf]{euler}

%\usepackage{palatino}
%\usepackage{newcent}


% Commande spéciale prompteur

%\usepackage{mathptmx}
%\usepackage[mathcal,mathbf]{euler}
%\usepackage{mathpple,multido}

\usepackage[a4paper]{geometry}
\geometry{top=2cm, bottom=2cm, left=1cm, right=1cm, marginparsep=1cm}

\newcommand{\change}{{\color{red}\rule{\textwidth}{1mm}\\}}

\newcounter{mydiapo}

\newcommand{\diapo}{\newpage
\hfill {\normalsize  Diapo \themydiapo \quad \texttt{[\jobname]}} \\
\stepcounter{mydiapo}}


%%%%%%% COULEURS %%%%%%%%%%

% Pour blanc sur noir :
%\pagecolor[rgb]{0.5,0.5,0.5}
% \pagecolor[rgb]{0,0,0}
% \color[rgb]{1,1,1}



%\DeclareFixedFont{\myfont}{U}{cmss}{bx}{n}{18pt}
\newcommand{\debuttexte}{
%%%%%%%%%%%%% FONTES %%%%%%%%%%%%%
\renewcommand{\baselinestretch}{1.5}
\usefont{U}{cmss}{bx}{n}
\bfseries

% Taille normale : commenter le reste !
%Taille Arnaud
%\fontsize{19}{19}\selectfont

% Taille Barbara
%\fontsize{21}{22}\selectfont

%Taille François
\fontsize{25}{30}\selectfont

%Taille Pascal
%\fontsize{25}{30}\selectfont

%Taille Laura
%\fontsize{30}{35}\selectfont


%\myfont
%\usefont{U}{cmss}{bx}{n}

%\Huge
%\addtolength{\parskip}{\baselineskip}
}


% \usepackage{hyperref}
% \hypersetup{colorlinks=true, linkcolor=blue, urlcolor=blue,
% pdftitle={Exo7 - Exercices de mathématiques}, pdfauthor={Exo7}}


%section
% \usepackage{sectsty}
% \allsectionsfont{\bf}
%\sectionfont{\color{Tomato3}\upshape\selectfont}
%\subsectionfont{\color{Tomato4}\upshape\selectfont}

%----- Ensembles : entiers, reels, complexes -----
\newcommand{\Nn}{\mathbb{N}} \newcommand{\N}{\mathbb{N}}
\newcommand{\Zz}{\mathbb{Z}} \newcommand{\Z}{\mathbb{Z}}
\newcommand{\Qq}{\mathbb{Q}} \newcommand{\Q}{\mathbb{Q}}
\newcommand{\Rr}{\mathbb{R}} \newcommand{\R}{\mathbb{R}}
\newcommand{\Cc}{\mathbb{C}} 
\newcommand{\Kk}{\mathbb{K}} \newcommand{\K}{\mathbb{K}}

%----- Modifications de symboles -----
\renewcommand{\epsilon}{\varepsilon}
\renewcommand{\Re}{\mathop{\text{Re}}\nolimits}
\renewcommand{\Im}{\mathop{\text{Im}}\nolimits}
%\newcommand{\llbracket}{\left[\kern-0.15em\left[}
%\newcommand{\rrbracket}{\right]\kern-0.15em\right]}

\renewcommand{\ge}{\geqslant}
\renewcommand{\geq}{\geqslant}
\renewcommand{\le}{\leqslant}
\renewcommand{\leq}{\leqslant}

%----- Fonctions usuelles -----
\newcommand{\ch}{\mathop{\mathrm{ch}}\nolimits}
\newcommand{\sh}{\mathop{\mathrm{sh}}\nolimits}
\renewcommand{\tanh}{\mathop{\mathrm{th}}\nolimits}
\newcommand{\cotan}{\mathop{\mathrm{cotan}}\nolimits}
\newcommand{\Arcsin}{\mathop{\mathrm{Arcsin}}\nolimits}
\newcommand{\Arccos}{\mathop{\mathrm{Arccos}}\nolimits}
\newcommand{\Arctan}{\mathop{\mathrm{Arctan}}\nolimits}
\newcommand{\Argsh}{\mathop{\mathrm{Argsh}}\nolimits}
\newcommand{\Argch}{\mathop{\mathrm{Argch}}\nolimits}
\newcommand{\Argth}{\mathop{\mathrm{Argth}}\nolimits}
\newcommand{\pgcd}{\mathop{\mathrm{pgcd}}\nolimits} 

\newcommand{\Card}{\mathop{\text{Card}}\nolimits}
\newcommand{\Ker}{\mathop{\text{Ker}}\nolimits}
\newcommand{\id}{\mathop{\text{id}}\nolimits}
\newcommand{\ii}{\mathrm{i}}
\newcommand{\dd}{\mathrm{d}}
\newcommand{\Vect}{\mathop{\text{Vect}}\nolimits}
\newcommand{\Mat}{\mathop{\mathrm{Mat}}\nolimits}
\newcommand{\rg}{\mathop{\text{rg}}\nolimits}
\newcommand{\tr}{\mathop{\text{tr}}\nolimits}
\newcommand{\ppcm}{\mathop{\text{ppcm}}\nolimits}

%----- Structure des exercices ------

\newtheoremstyle{styleexo}% name
{2ex}% Space above
{3ex}% Space below
{}% Body font
{}% Indent amount 1
{\bfseries} % Theorem head font
{}% Punctuation after theorem head
{\newline}% Space after theorem head 2
{}% Theorem head spec (can be left empty, meaning ‘normal’)

%\theoremstyle{styleexo}
\newtheorem{exo}{Exercice}
\newtheorem{ind}{Indications}
\newtheorem{cor}{Correction}


\newcommand{\exercice}[1]{} \newcommand{\finexercice}{}
%\newcommand{\exercice}[1]{{\tiny\texttt{#1}}\vspace{-2ex}} % pour afficher le numero absolu, l'auteur...
\newcommand{\enonce}{\begin{exo}} \newcommand{\finenonce}{\end{exo}}
\newcommand{\indication}{\begin{ind}} \newcommand{\finindication}{\end{ind}}
\newcommand{\correction}{\begin{cor}} \newcommand{\fincorrection}{\end{cor}}

\newcommand{\noindication}{\stepcounter{ind}}
\newcommand{\nocorrection}{\stepcounter{cor}}

\newcommand{\fiche}[1]{} \newcommand{\finfiche}{}
\newcommand{\titre}[1]{\centerline{\large \bf #1}}
\newcommand{\addcommand}[1]{}
\newcommand{\video}[1]{}

% Marge
\newcommand{\mymargin}[1]{\marginpar{{\small #1}}}



%----- Presentation ------
\setlength{\parindent}{0cm}

%\newcommand{\ExoSept}{\href{http://exo7.emath.fr}{\textbf{\textsf{Exo7}}}}

\definecolor{myred}{rgb}{0.93,0.26,0}
\definecolor{myorange}{rgb}{0.97,0.58,0}
\definecolor{myyellow}{rgb}{1,0.86,0}

\newcommand{\LogoExoSept}[1]{  % input : echelle
{\usefont{U}{cmss}{bx}{n}
\begin{tikzpicture}[scale=0.1*#1,transform shape]
  \fill[color=myorange] (0,0)--(4,0)--(4,-4)--(0,-4)--cycle;
  \fill[color=myred] (0,0)--(0,3)--(-3,3)--(-3,0)--cycle;
  \fill[color=myyellow] (4,0)--(7,4)--(3,7)--(0,3)--cycle;
  \node[scale=5] at (3.5,3.5) {Exo7};
\end{tikzpicture}}
}



\theoremstyle{definition}
%\newtheorem{proposition}{Proposition}
%\newtheorem{exemple}{Exemple}
%\newtheorem{theoreme}{Théorème}
\newtheorem{lemme}{Lemme}
\newtheorem{corollaire}{Corollaire}
%\newtheorem*{remarque*}{Remarque}
%\newtheorem*{miniexercice}{Mini-exercices}
%\newtheorem{definition}{Définition}




%definition d'un terme
\newcommand{\defi}[1]{{\color{myorange}\textbf{\emph{#1}}}}
\newcommand{\evidence}[1]{{\color{blue}\textbf{\emph{#1}}}}



 %----- Commandes divers ------

\newcommand{\codeinline}[1]{\texttt{#1}}

%%%%%%%%%%%%%%%%%%%%%%%%%%%%%%%%%%%%%%%%%%%%%%%%%%%%%%%%%%%%%
%%%%%%%%%%%%%%%%%%%%%%%%%%%%%%%%%%%%%%%%%%%%%%%%%%%%%%%%%%%%%


\begin{document}

\debuttexte


%%%%%%%%%%%%%%%%%%%%%%%%%%%%%%%%%%%%%%%%%%%%%%%%%%%%%%%%%%%
\diapo


\change
Nous attaquons le coeur de ce chapitre : le calcul de l'équation de la chaînette.

\change
Pour cela nous allons découper notre chaînette en une succession de tous petits bouts.

\change
A chaque bout nous appliquerons le principe fondamental de la mécanique

\change
Cela nous permettra de montrer que la tension horizontale est constante

\change
et de relier la tension verticale au poids

\change
Enfin nous détaillerons pas à pas le calcul de l'équation
de la chaînette.


%%%%%%%%%%%%%%%%%%%%%%%%%%%%%%%%%%%%%%%%%%%%%%%%%%%%%%%%%%%
\diapo

Nous découpons la chaînette en petits morceaux, chaque morceau étant 
compris entre les abscisses $x$ et $x+dx$. Ici $dx$ désigne donc un
réel aussi petit que l'on veut.

Nous noterons $d\ell$ la longueur de ce petit morceau de chaînette.

\change
Trois forces s'appliquent à notre mini-bout de chaînette :

\change
Première force : Le poids $\vec P$. 

\change
C'est une force verticale, 

\change
proportionnelle à la masse du morceau.

\change
Si $\mu$ est la masse linéique (c'est-à-dire la masse que ferait un mètre de chaîne, exprimée en $kg/m$),

\change
la masse de notre petit bout est $\mu \cdot d\ell$.

\change
$g$ dénote la constante de gravitation (avec $g \approx 9,81 \; m/s^2$) alors le poids 

le poids est la masse fois la constante de gravitation d'où la formule :

 $\vec P = - \mu\cdot d\ell \cdot g \cdot \vec j$.


%$\vec P = - \mu\cdot d\ell \cdot g \cdot \vec j$
où $\vec j$ est un vecteur vertical dirigé vers le haut

\change
Deuxième force : La tension à gauche $\vec T(x)$.

\change
La tension à gauche, s'applique au point dont l'abscisse est $x$.

\change
 Par un principe physique, les forces de tension de notre morceau à l'équilibre sont
des forces tangentes à la chaînette. 

\change
Troisième et dernière force : La tension à droite $-\vec T(x+dx)$.

\change
La tension à droite s'applique au point d'abscisse $x+dx$.

\change
Comme notre morceau est en équilibre elle s'oppose à la tension à gauche du morceau suivant compris entre $x+dx$ et 
$x + 2dx$. 

\change
La tension à droite de notre morceau est donc l'opposée de la tension à gauche du morceau suivant, 
cette force est donc $-\vec T(x+dx)$.



%%%%%%%%%%%%%%%%%%%%%%%%%%%%%%%%%%%%%%%%%%%%%%%%%%%%%%%%%%%
\diapo



Le principe fondamental de la mécanique nous dit que, 
à l'équilibre, la somme des forces est nulle :

\change
Ce qui donne pour nos trois forces

\change
la relation 
$\vec P + \vec T(x)-\vec T(x+dx) = \vec 0. $




\change
Décomposons chaque force de tension, en une tension horizontale 
et une tension verticale 

\change
$\vec T(x) = -T_h(x)\vec i - T_v(x) \vec j.$

où $\vec i$ est un vecteur horizontale.


\change
Alors le principe fondamental de la mécanique devient dans le repère  $(\vec i,\vec j)$ :
$$-P \vec j - T_h(x)\vec i - T_v(x) \vec j - \left( - T_h(x+dx)\vec i - T_v(x+dx)  \vec j \right)= \vec 0.$$


Comme $(\vec i,\vec j)$ est une base, nous reformulons 
le principe fondamental de la mécanique en deux équations, 
correspondant aux forces horizontales et aux forces verticales :


\change
(1) $T_h(x+dx)-T_h(x) = 0$

\change
et (2) $T_v(x+dx) - T_v(x) - P = 0$

\change
Nous allons maintenant exploiter ces deux équations.


%%%%%%%%%%%%%%%%%%%%%%%%%%%%%%%%%%%%%%%%%%%%%%%%%%%%%%%%%%%
\diapo

Le premier résultat est que la tension horizontale est indépendante de $x$ :

$T_h(x) = T_h $ pour tout $x$.

\change
Pour la preuve nous allons utiliser la première équation 
issue du principe fondamental de la mécanique.

\change
En effet, fixons $x$, nous savons $T_h(x+dx)-T_h(x)=0$,

\change
Donc le rapport $\frac{T_h(x+dx)-T_h(x)}{x+dx-x}$
est nul.

\change
Ceci est vrai quelque soit l'élément infinitésimal $dx$. 
Ce taux d'accroissement étant toujours nul, 
la limite lorsque $dx$ tend vers $0$ est nulle.
Mais la limite est --par définition-- la dérivée $T'_h(x)$.

\change
Bilan : $T_h'(x)=0$. 

\change
La dérivée étant nulle, la fonction $T_h(x)$ est une fonction constante
comme nous l'avions annoncé.

Physiquement cela signifie que la composante horizontale de la tension est 
indépendante du point de la chaînette.

%%%%%%%%%%%%%%%%%%%%%%%%%%%%%%%%%%%%%%%%%%%%%%%%%%%%%%%%%%%
\diapo


On va maintenant s'attaquer à la composante verticale de la tension.

Avant cela commençons par des considération géométriques.

Nous noterons $y(x)$ l'équation de la chaînette.

\change
Nous considérons que chaque morceau infinitésimal de la chaîne 
entre $x$ et $x+dx$ est rectiligne, 

Ce qui revient à remplacer le morceaux en pointillé rouge par le segment vert.
Ici le morceaux dessiné n'est pas très petit et on perçoit l'écart commis 
par notre approximation ; mais si $dx$ est très petit l'erreur commise est très très petite.


\change
Nous pouvons
alors appliquer le théorème de Pythagore, 
dans le petit triangle rectangle vert dont l'hypoténuse est $d\ell$ :
$d \ell^2 = dx^2 + dy^2.$

\change
on divise par $(dx)^2$.

\change
Cela conduit à la relation
$\frac{d\ell}{dx}=\sqrt{1+ \left(\frac{dy}{dx}\right)^2}.$

dont nous aurons besoin dans la diapo suivante.


%%%%%%%%%%%%%%%%%%%%%%%%%%%%%%%%%%%%%%%%%%%%%%%%%%%%%%%%%%%
\diapo

Nous allons maintenant nous occuper de la tension verticale.

\change
Nous avons appliqué le principe fondamental de la mécanique
aux $3$ forces s'appliquant à notre morceaux de chaîne...

\change
...en ne considérant que les composantes verticales 
nous en avions extrait l'équation :
$T_v(x+dx)-T_v(x) -P = 0.$

où $T_v$ sont les tensions verticales aux deux extrémités.

\change
Et le poids $P$ dépend de la longueur $d\ell$ de notre petit morceaux.

\change
On obtient $T_v(x+dx)-T_v(x)= \mu g d\ell$

\change
Cela donne en divisant par $dx$ :

$\frac{T_v(x+dx)-T_v(x)}{dx} = \mu g \frac {d\ell}{dx}$

\change
Mais on a obtenu une expression pour $d\ell/dx$ dans la diapo précédente
ce qui donne : 
$\frac{T_v(x+dx)-T_v(x)}{dx} =\mu g \sqrt{1+ \left(\frac{dy}{dx}\right)^2} .$

\change
En terme de dérivée$\frac{T_v(x+dx)-T_v(x)}{dx}$ vaut à la limite $T_v'(x)$

\change
alors que $\frac{dy}{dx}$ vaut à la limite $y'(x)$

\change
Nous avons donc montré :
$T_v'(x) = \mu g \sqrt{1+ y'(x)^2}.$

Nous n'avons pas calculer la tension verticale mais nous avons obtenu une relation
entre la tension verticale et l'équation de la chaînette.
Cette relation, reliant les dérivées des fonctions, il s'agit d'une équation différentielle.



%%%%%%%%%%%%%%%%%%%%%%%%%%%%%%%%%%%%%%%%%%%%%%%%%%%%%%%%%%%
\diapo

Voici l'énoncé que je vous avais annoncé dès le début de ce chapitre.

Théorème : 
Une équation de la chaînette est 

$y(x) = a \ch \left( \frac x a\right)$.

$\ch$ est le cosinus hyperbolique.

\change
Le paramètre $a$ est  propre à chaque chaînette,

c'est une constante qui vaut $a = \frac {T_h}{\mu g}$
(souvenez-vous que la tension horizontale est constante,
$\mu$ est la masse d'un mètre de chaînette et $g$ est la constante de gravitation.



%%%%%%%%%%%%%%%%%%%%%%%%%%%%%%%%%%%%%%%%%%%%%%%%%%%%%%%%%%%
\diapo

La preuve se déroule en quatre étapes : la résolution du problème va se faire
en utilisant des équation différentielles et la fonction cosinus hyperbolique.

La première étape est de trouver un lien entre la 
tension verticale et la tension horizontale.

\change
Comme on a décomposé la tension en une composante horizontale et une composante verticale.
On note $\alpha$ l'angle que forme la chaînette avec l'horizontale au point considéré,
angle que l'on retrouve ici.


\change
Alors $T_h(x) = T(x) \cos \alpha(x)$

\change
et 
$T_v(x) = T(x) \sin \alpha(x).$

\change
Ce qui conduit à $T_v(x) = T_h(x) \tan \alpha(x)$.

\change
Maintenant, dans le triangle infinitésimal, 
de côtés $dx$, $dy$, $d\ell$ on retrouve le même angle 
$\alpha$ que forme la chaînette avec l'horizontale.

\change
De ce triangle nous extrayons la relation  
$\tan \alpha(x) = \frac{dy}{dx}$ qui est aussi $y'(x)$.


\change

En combinant ceci avec la relation précédente on obtient
notre lien entre tension verticale et tension horizontale :
$T_v(x) = T_h(x) \cdot y'(x).$




%%%%%%%%%%%%%%%%%%%%%%%%%%%%%%%%%%%%%%%%%%%%%%%%%%%%%%%%%%%
\diapo

Nous allons traduire nos relations physiques en des relations 
mathématiques sous la forme d'équations différentielles.

\change
On vient de montrer que :
$T_v(x) = T_h(x) \cdot y'(x).$

\change
Donc en dérivant cette égalité, 
et en souvenant que la tension horizontale est constante
on obtient $T_v'(x) = T_h \cdot y''(x).$

\change
Mais nous avions déjà obtenu une expression pour ce même $T_v'(x)$ :
$T_v'(x) = \mu g \sqrt{1+ y'(x)^2}$.

\change
En identifiant les relations on trouve :
$\mu g \sqrt{1+ y'(x)^2} = T_h \cdot y''(x)$

\change
Nous allons simplifier l'écriture de cette équation :

on définit $a$ la constante $a =\frac{T_h}{\mu g}$

\change
et on poses $z(x)= y'(x)$.

\change
L'équation précédente devient alors simplement  
$\sqrt{1+z(x)^2} = a z'(x)$
C'est une équation différentielle
du premier ordre

\change
que l'on peut écrire aussi
$\frac{z'(x)}{\sqrt{1+z(x)^2}} = \frac 1 a.$

%%%%%%%%%%%%%%%%%%%%%%%%%%%%%%%%%%%%%%%%%%%%%%%%%%%%%%%%%%%
\diapo


\change
On a obtenu une relation différentielle liant la dérivée de $z$
et $z$, je vous rappelle que $z(x)$ désigne la dérivée $y'(x)$

On va résoudre cette équation différentielle.


\change
Une primitive de $\frac{z'(x)}{\sqrt{1+z(x)^2}}$ est $\Argsh z(x)$.

Je vous rappelle $\Argsh$ est la bijection réciproque du sinus hyperbolique $\sh$.

\change
On peut donc intégrer cette équation différentielle pour obtenir
$$\Argsh z(x) = \frac x a + \alpha$$

où $\alpha$ est une constante.

\change
En composant des deux côtés par le sinus hyperbolique :
on obtient 

$z(x) = \sh\left(\frac x a + \alpha\right).$

Mais je vous rappelle que $z(x) = y'(x)$.

On a donc $y'(x)= \sh\left(\frac x a + \alpha\right).$

\change
Une primitive de $\sh x$ étant $\ch x$,
il ne reste plus qu'à intégrer une dernière fois 

$y(x) = a \ch \left(\frac x a + \alpha\right) + \beta.$

où $\beta$ est une autre constante.




%%%%%%%%%%%%%%%%%%%%%%%%%%%%%%%%%%%%%%%%%%%%%%%%%%%%%%%%%%%
\diapo

On a presque terminé !

\change
Il ne nous reste plus qu'a obtenir la forme annoncée
à partir de cette formule.

\change

\change
On choisit le repère de sorte que le point le plus 
bas de la chaînette 
a pour coordonnées $(0,a)$ 

\change
Cela signifie que $y(0)=a$ et comme la tangente 
est horizontale en ce point $y'(0)=0$. 

\change
Ce que l'on vient de faire c'est de fixer
$\alpha=0$ et $\beta=0$ pour les deux constantes.

\change
Et voilà, on bien obtenu $y(x) = a \ch \left(\frac x a\right)$.




\end{document}
