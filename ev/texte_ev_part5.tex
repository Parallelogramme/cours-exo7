
%%%%%%%%%%%%%%%%%% PREAMBULE %%%%%%%%%%%%%%%%%%


\documentclass[12pt]{article}

\usepackage{amsfonts,amsmath,amssymb,amsthm}
\usepackage[utf8]{inputenc}
\usepackage[T1]{fontenc}
\usepackage[francais]{babel}


% packages
\usepackage{amsfonts,amsmath,amssymb,amsthm}
\usepackage[utf8]{inputenc}
\usepackage[T1]{fontenc}
%\usepackage{lmodern}

\usepackage[francais]{babel}
\usepackage{fancybox}
\usepackage{graphicx}

\usepackage{float}

%\usepackage[usenames, x11names]{xcolor}
\usepackage{tikz}
\usepackage{datetime}

\usepackage{mathptmx}
%\usepackage{fouriernc}
%\usepackage{newcent}
\usepackage[mathcal,mathbf]{euler}

%\usepackage{palatino}
%\usepackage{newcent}


% Commande spéciale prompteur

%\usepackage{mathptmx}
%\usepackage[mathcal,mathbf]{euler}
%\usepackage{mathpple,multido}

\usepackage[a4paper]{geometry}
\geometry{top=2cm, bottom=2cm, left=1cm, right=1cm, marginparsep=1cm}

\newcommand{\change}{{\color{red}\rule{\textwidth}{1mm}\\}}

\newcounter{mydiapo}

\newcommand{\diapo}{\newpage
\hfill {\normalsize  Diapo \themydiapo \quad \texttt{[\jobname]}} \\
\stepcounter{mydiapo}}


%%%%%%% COULEURS %%%%%%%%%%

% Pour blanc sur noir :
%\pagecolor[rgb]{0.5,0.5,0.5}
% \pagecolor[rgb]{0,0,0}
% \color[rgb]{1,1,1}



%\DeclareFixedFont{\myfont}{U}{cmss}{bx}{n}{18pt}
\newcommand{\debuttexte}{
%%%%%%%%%%%%% FONTES %%%%%%%%%%%%%
\renewcommand{\baselinestretch}{1.5}
\usefont{U}{cmss}{bx}{n}
\bfseries

% Taille normale : commenter le reste !
%Taille Arnaud
%\fontsize{19}{19}\selectfont

% Taille Barbara
%\fontsize{21}{22}\selectfont

%Taille François
\fontsize{25}{30}\selectfont

%Taille Pascal
%\fontsize{25}{30}\selectfont

%Taille Laura
%\fontsize{30}{35}\selectfont


%\myfont
%\usefont{U}{cmss}{bx}{n}

%\Huge
%\addtolength{\parskip}{\baselineskip}
}


% \usepackage{hyperref}
% \hypersetup{colorlinks=true, linkcolor=blue, urlcolor=blue,
% pdftitle={Exo7 - Exercices de mathématiques}, pdfauthor={Exo7}}


%section
% \usepackage{sectsty}
% \allsectionsfont{\bf}
%\sectionfont{\color{Tomato3}\upshape\selectfont}
%\subsectionfont{\color{Tomato4}\upshape\selectfont}

%----- Ensembles : entiers, reels, complexes -----
\newcommand{\Nn}{\mathbb{N}} \newcommand{\N}{\mathbb{N}}
\newcommand{\Zz}{\mathbb{Z}} \newcommand{\Z}{\mathbb{Z}}
\newcommand{\Qq}{\mathbb{Q}} \newcommand{\Q}{\mathbb{Q}}
\newcommand{\Rr}{\mathbb{R}} \newcommand{\R}{\mathbb{R}}
\newcommand{\Cc}{\mathbb{C}} 
\newcommand{\Kk}{\mathbb{K}} \newcommand{\K}{\mathbb{K}}

%----- Modifications de symboles -----
\renewcommand{\epsilon}{\varepsilon}
\renewcommand{\Re}{\mathop{\text{Re}}\nolimits}
\renewcommand{\Im}{\mathop{\text{Im}}\nolimits}
%\newcommand{\llbracket}{\left[\kern-0.15em\left[}
%\newcommand{\rrbracket}{\right]\kern-0.15em\right]}

\renewcommand{\ge}{\geqslant}
\renewcommand{\geq}{\geqslant}
\renewcommand{\le}{\leqslant}
\renewcommand{\leq}{\leqslant}

%----- Fonctions usuelles -----
\newcommand{\ch}{\mathop{\mathrm{ch}}\nolimits}
\newcommand{\sh}{\mathop{\mathrm{sh}}\nolimits}
\renewcommand{\tanh}{\mathop{\mathrm{th}}\nolimits}
\newcommand{\cotan}{\mathop{\mathrm{cotan}}\nolimits}
\newcommand{\Arcsin}{\mathop{\mathrm{Arcsin}}\nolimits}
\newcommand{\Arccos}{\mathop{\mathrm{Arccos}}\nolimits}
\newcommand{\Arctan}{\mathop{\mathrm{Arctan}}\nolimits}
\newcommand{\Argsh}{\mathop{\mathrm{Argsh}}\nolimits}
\newcommand{\Argch}{\mathop{\mathrm{Argch}}\nolimits}
\newcommand{\Argth}{\mathop{\mathrm{Argth}}\nolimits}
\newcommand{\pgcd}{\mathop{\mathrm{pgcd}}\nolimits} 

\newcommand{\Card}{\mathop{\text{Card}}\nolimits}
\newcommand{\Ker}{\mathop{\text{Ker}}\nolimits}
\newcommand{\id}{\mathop{\text{id}}\nolimits}
\newcommand{\ii}{\mathrm{i}}
\newcommand{\dd}{\mathrm{d}}
\newcommand{\Vect}{\mathop{\text{Vect}}\nolimits}
\newcommand{\Mat}{\mathop{\mathrm{Mat}}\nolimits}
\newcommand{\rg}{\mathop{\text{rg}}\nolimits}
\newcommand{\tr}{\mathop{\text{tr}}\nolimits}
\newcommand{\ppcm}{\mathop{\text{ppcm}}\nolimits}

%----- Structure des exercices ------

\newtheoremstyle{styleexo}% name
{2ex}% Space above
{3ex}% Space below
{}% Body font
{}% Indent amount 1
{\bfseries} % Theorem head font
{}% Punctuation after theorem head
{\newline}% Space after theorem head 2
{}% Theorem head spec (can be left empty, meaning ‘normal’)

%\theoremstyle{styleexo}
\newtheorem{exo}{Exercice}
\newtheorem{ind}{Indications}
\newtheorem{cor}{Correction}


\newcommand{\exercice}[1]{} \newcommand{\finexercice}{}
%\newcommand{\exercice}[1]{{\tiny\texttt{#1}}\vspace{-2ex}} % pour afficher le numero absolu, l'auteur...
\newcommand{\enonce}{\begin{exo}} \newcommand{\finenonce}{\end{exo}}
\newcommand{\indication}{\begin{ind}} \newcommand{\finindication}{\end{ind}}
\newcommand{\correction}{\begin{cor}} \newcommand{\fincorrection}{\end{cor}}

\newcommand{\noindication}{\stepcounter{ind}}
\newcommand{\nocorrection}{\stepcounter{cor}}

\newcommand{\fiche}[1]{} \newcommand{\finfiche}{}
\newcommand{\titre}[1]{\centerline{\large \bf #1}}
\newcommand{\addcommand}[1]{}
\newcommand{\video}[1]{}

% Marge
\newcommand{\mymargin}[1]{\marginpar{{\small #1}}}



%----- Presentation ------
\setlength{\parindent}{0cm}

%\newcommand{\ExoSept}{\href{http://exo7.emath.fr}{\textbf{\textsf{Exo7}}}}

\definecolor{myred}{rgb}{0.93,0.26,0}
\definecolor{myorange}{rgb}{0.97,0.58,0}
\definecolor{myyellow}{rgb}{1,0.86,0}

\newcommand{\LogoExoSept}[1]{  % input : echelle
{\usefont{U}{cmss}{bx}{n}
\begin{tikzpicture}[scale=0.1*#1,transform shape]
  \fill[color=myorange] (0,0)--(4,0)--(4,-4)--(0,-4)--cycle;
  \fill[color=myred] (0,0)--(0,3)--(-3,3)--(-3,0)--cycle;
  \fill[color=myyellow] (4,0)--(7,4)--(3,7)--(0,3)--cycle;
  \node[scale=5] at (3.5,3.5) {Exo7};
\end{tikzpicture}}
}



\theoremstyle{definition}
%\newtheorem{proposition}{Proposition}
%\newtheorem{exemple}{Exemple}
%\newtheorem{theoreme}{Théorème}
\newtheorem{lemme}{Lemme}
\newtheorem{corollaire}{Corollaire}
%\newtheorem*{remarque*}{Remarque}
%\newtheorem*{miniexercice}{Mini-exercices}
%\newtheorem{definition}{Définition}




%definition d'un terme
\newcommand{\defi}[1]{{\color{myorange}\textbf{\emph{#1}}}}
\newcommand{\evidence}[1]{{\color{blue}\textbf{\emph{#1}}}}



 %----- Commandes divers ------

\newcommand{\codeinline}[1]{\texttt{#1}}

%%%%%%%%%%%%%%%%%%%%%%%%%%%%%%%%%%%%%%%%%%%%%%%%%%%%%%%%%%%%%
%%%%%%%%%%%%%%%%%%%%%%%%%%%%%%%%%%%%%%%%%%%%%%%%%%%%%%%%%%%%%



\begin{document}

\debuttexte


%%%%%%%%%%%%%%%%%%%%%%%%%%%%%%%%%%%%%%%%%%%%%%%%%%%%%%%%%%%
\diapo

\change

On termine notre étude des sous-espaces vectoriels avec des notions plus délicates :

\change

Nous allons définir la somme de deux sous-espaces vectoriels,

\change

ce qui permettra de définir ce que sont des sous-espaces vectoriels supplémentaires.

\change

On terminera avec la définition de sous-espace vectoriel engendré. 



%%%%%%%%%%%%%%%%%%%%%%%%%%%%%%%%%%%%%%%%%%%%%%%%%%%%%%%%%%
\diapo

Comme la réunion des deux sous-espaces vectoriels $F$ et $G$ 
n'est pas en général un sous-espace vectoriel, il est utile de connaître 
les sous-espaces vectoriels qui contiennent à la fois les deux sous-espaces 
vectoriels $F$ et $G$, et en particulier le plus petit d'entre eux .



Soient donc $F$ et $G$ deux sous-espaces vectoriels d'un $\Kk$-espace vectoriel $E$. 

L'ensemble de tous les éléments $u+v$, où $u$ est un élément de 
$F$ et $v$ un élément de $G$, est appelé \defi{somme} des sous-espaces vectoriels 
$F$ et $G$. 

\change


Cette somme est notée  $F+G$. On a donc par définition :
$$F+G=\big\{u+v \mid u \in F, v \in G \big\}.$$

\change
Voici un dessin avec $F$, $G$ et la somme $F+G$.

\change

Deux propriétés  font tout l’intérêt de la somme :

Tout d'abord : $F+G$ est un sous-espace vectoriel de $E$. (contrairement à l'union qui n'était pas un sev.)

Deuxièmement : $F+G$ est le plus petit sous-espace vectoriel qui contient à la fois $F$ et $G$.

Il faut comprendre "plus petit" au sens de l'inclusion.



%%%%%%%%%%%%%%%%%%%%%%%%%%%%%%%%%%%%%%%%%%%%%%%%%%%%%%%%%%%
\diapo


Déterminons $F+G$  dans le cas où $F$ et $G$ sont les sous-espaces vectoriels de 
$\Rr^3$ suivants : 
$$F=\big\{(x,y,z) \in \Rr^3\mid y=z=0\big\} 
\qquad\text{ et }\qquad 
G=\big\{(x,y,z) \in \Rr^3 \mid x=z=0\big\}.$$

\change
Voici une figure

\change

Un élément $w$ de $F+G$ s'écrit comme la somme $w=u+v$ où $u \in F$  et $v \in G$. 

\change

Comme $u\in F$ alors par définition de $F$ il existe $x \in \Rr$ tel que $u=(x,0,0)$, 

\change

et comme $v \in G$ alors par définition de $G$ il existe $y \in \Rr$ tel que $v=(0,y,0)$. 

\change

Ce qui veut dire que $w=u+v=(x,y,0)$. 


Réciproquement, un tel élément $w=(x,y,0)$ est la somme de $(x,0,0)$ 
et de $(0,y,0)$. 

\change

Donc $F+G=\big\{(x,y,z) \in \Rr^3\mid z=0\big\}$, c'est un plan passant par l'origine.


On voit même que, pour cet exemple, tout élément de $F+G$ s'écrit 
de façon \emph{unique} comme la somme d'un élément de $F$ et d'un élément de $G$.



%%%%%%%%%%%%%%%%%%%%%%%%%%%%%%%%%%%%%%%%%%%%%%%%%%%%%%%%%%
\diapo

Soient $F$ et $G$ les deux sous-espaces vectoriels de $\Rr^3$ : 
$$F=\big\{(x,y,z) \in \Rr^3\mid x=0\big\}
\qquad \text{ et } \qquad 
G=\big\{(x,y,z) \in \Rr^3 \mid y=0\big\}.$$

\change


Dans cet exemple, montrons que $F+G=\Rr^3$.

Par définition de $F+G$, tout élément de  $F+G$ est dans $\Rr^3$.

\change

Mais réciproquement, si  $w=(x,y,z)$ est un élément quelconque
de $\Rr^3$ : 

\change

$w=(x,y,z)=(0,y,z)+(x,0,0)$, avec
$(0,y,z) \in F$ et $(x,0,0) \in G$, 

\change

donc $w$ appartient à $F+G$.

\change

Remarquons que, dans cet exemple, un élément de $\Rr^3$ ne s'écrit pas forcément de façon unique 
comme la somme d'un élément de $F$ et d'un élément de $G$. Par exemple  
$(1,2,3)=(0,2,3)+ (1,0,0)= (0,2,0)+(1,0,3).$


%%%%%%%%%%%%%%%%%%%%%%%%%%%%%%%%%%%%%%%%%%%%%%%%%%%%%%%%%%%
\diapo


Soient $F$ et $G$ deux sous-espaces vectoriels de $E$.

On dit que $F$ et $G$ sont en \defi{somme directe} dans $E$ si 
les deux conditions suivantes sont vérifiées 

(1) $F \cap G = \{ 0_E \}$ : l'intersection ne contient que le vecteur nul ;

(2) $F+G=E$ ; la somme vaut tout l'espace $E$.

\change

Si c'est le cas alors on note "$F$ somme directe $G$ égal $E$" ainsi [$F \oplus G=E$] pour dire que 
$F$ et $G$ sont en \defi{somme directe} dans $E$

\change


on dit aussi que $F$ et $G$ sont des sous-espaces vectoriels \defi{supplémentaires} dans $E$.


\change

L'intérêt des sous-espaces supplémentaires, c'est qu'ils fournissent une décomposition *unique* :


$F$ et $G$ sont supplémentaires dans $E$ si et seulement si tout 
élément de $E$ s'écrit d'une manière \evidence{unique} 
comme la somme d'un élément de $F$ et d'un élément de $G$. 

\change

Dire qu'un élément $w$ de $E$ s'écrit d'une manière unique comme la somme d'un élément de $F$ et d'un élément de $G$
  signifie que si $w=u+v$ avec $u\in F$, $v\in G$ et $w=u'+v'$ avec $u'\in F$, $v'\in G$
alors en fait $u=u'$ et $v=v'$. 



%%%%%%%%%%%%%%%%%%%%%%%%%%%%%%%%%%%%%%%%%%%%%%%%%%%%%%%%%%
\diapo

Soient $F = \big\{ (x,0) \in \Rr^2 \mid x \in \Rr \big\}$
et $G = \big\{ (0,y) \in \Rr^2 \mid y \in \Rr \big\}$.

\change

Est-ce que $F \oplus G = \Rr^2$.

\change

La première façon de le voir est de vérifier les deux axiomes :

(1) on a clairement $F\cap G = \{ (0,0) \}$

\change

et (2) comme $(x,y)=(x,0)+(0,y)$, alors $F+G = \Rr^2$.

\change

Donc $F$ et $G$ sont supplémentaires dans $\Rr^2$ [$F \oplus G = \Rr^2$.]

\change


Une autre façon de le voir est d'utiliser la proposition précédente,
car la décomposition $(x,y)=(x,0)+(0,y)$, un élément de $F$ plus un élément de $G$, est unique.
 


\change

Gardons $F$ et notons $G' = \big\{ (x,x) \in \Rr^2 \mid x \in \Rr \big\}$.
Montrons que l'on a aussi $F$ et $G' $ sont supplémentaires dans $\Rr^2$ [$F\oplus G'=\Rr^2$] :

\change

Il n'est pas dur de voir que $F \cap G' =\{(0,0)\}$. 

\change

Pour montrer que la somme $F+G' = \Rr^2$ il faut chercher un peu, et on trouve que 
la décomposition un élément de $F$ plus un élément de $G'$ qui convient est 
$(x,y) = (x-y,0) + (y,y),$
    
    
    Conclusion : $F$ et $G' $ sont supplémentaires dans $\Rr^2$ [$F\oplus G'=\Rr^2$.]
    
    
\change


En fait de façon générale, deux droites distinctes du plan 
qui passent par l'origine sont des sous-espaces supplémentaires.


Remarquons qu'il n'y a pas unicité du supplémentaire d'un sous-espace vectoriel donné.
  
Pour notre sous-espace $F$ on lui a trouvé deux espaces supplémentaires possibles : $G$ ou $G'$.
  

%%%%%%%%%%%%%%%%%%%%%%%%%%%%%%%%%%%%%%%%%%%%%%%%%%%%%%%%%%%
\diapo

Passons à un exemple dans l'espace.

Est-ce que les sous-espaces vectoriels $F$ et $G$ suivants sont supplémentaires dans $\Rr^3$ ?


$$F=\big\{ (x,y,z) \in \Rr^3\mid x-y-z=0\big\} \qquad \text { et } \qquad
G=\big\{(x,y,z) \in \Rr^3 \mid y=z=0\big\}$$



\change

Il est facile de vérifier que  $F\cap G=\{0\}$. 

\change

En effet si l'élément  $u=(x,y,z)$ appartient à l'intersection de $F$ et de $G$, 
alors les coordonnées de $u$ vérifient :  $x-y-z=0$ (car $u$ appartient à $F$), 
et  $y=z=0$ (car $u$ appartient à $G$), donc  $u=(0,0,0)$.
  
\change

Il reste à démontrer que $F+G=\Rr^3$.

\change


  Soit donc  $u=(x,y,z)$ un élément quelconque de $\Rr^3$ ;
  
  \change
  
  il faut déterminer un vecteur $v$ de $F$ et un vecteur $w$ de $G$ tels que $u=v+w$.
  
  \change
  
L'élément $v$ doit être de la forme $v=(y_1+z_1, y_1,z_1)$  et l'élément $w$ de la forme
  $w=(x_2,0,0)$. 
  
  \change
  
Donc la somme $(x,y,z)$ s'écrit $(y_1+z_1+x_2,y_1,z_1)$.

Un petit calcul donne  $y_1=y$, $z_1=z$, $x_2=x-y-z$


\change

Ce qui conduit à la décomposition $(x,y,z)=(y+z,y,z)+ (x-y-z, 0,0)$ : un élément de $F$, plus un élément de $G$.

\change

On vient donc de prouver que : $F$ et $G$ sont supplémentaires dans $\Rr^3$ [$F \oplus G=\Rr^3$].


%%%%%%%%%%%%%%%%%%%%%%%%%%%%%%%%%%%%%%%%%%%%%%%%%%%%%%%%%%%
\diapo


Dans le $\Rr$-espace vectoriel 
des fonctions de $\Rr$ dans $\Rr$, on considère le sous-espace 
vectoriel $\mathcal{P}$ des fonctions paires et le sous-espace 
vectoriel $\mathcal{I}$ des fonctions impaires. 

On va montrer que 
$\mathcal{P}\oplus\mathcal{I}=\mathcal{F}(\Rr,\Rr)$. 

\change

Premièrement calculons l'intersection.

\change

  Soit $f \in \mathcal{P} \cap \mathcal{I}$, c'est-à-dire que $f$ est à la 
  fois une fonction paire et impaire.
  Il s'agit de montrer que $f$ est la fonction identiquement nulle.
  
\change

  Soit $x \in \Rr$. Comme $f(-x)=f(x)$ (car $f$ est paire) et $f(-x)=-f(x)$ 
  (car $f$ est impaire), alors $f(x)=-f(x)$, ce qui implique $f(x)=0$. 
  
 \change
 
  Ceci est vrai quel que soit $x$; donc $f$ est la fonction nulle.
  Ainsi $\mathcal{P} \cap \mathcal{I} = \{ 0_{\mathcal{F}(\Rr,\Rr)} \}$.
 
 \change
 
 Deuxièmement calculons la somme.
  
  \change
  
  Soit $f$ une fonction quelconque. Il s'agit de montrer que $f$ peut s'écrire comme la somme 
  d'une fonction paire et d'une fonction impaire. On pourrait procéder par analyse/synthèse mais ici on donne directement le résultat.
  
  \change
  
  Posons $g(x)=\frac{f(x)+f(-x)}2$ alors $g(-x)=g(x)$ donc $g$ est une fonction paire.
  
  \change
  
  Posons aussi $h(x)=\frac{f(x)-f(-x)}2$ alors $h(-x)=-h(x)$ donc $h$ est une fonction impaire.
  
  \change
  
  En plus on a $f(x)=g(x)+h(x)$
  
  Bilan : la somme $\mathcal{P}+\mathcal{I}$ vaut $\mathcal{F}(\Rr,\Rr)$
  
  et ainsi $\mathcal{P}$ et $\mathcal{I}$ sont en somme directe.
  
  Notez que, dans la décomposition $f=g+h$ obtenue, les $g$ et $h$ sont uniques.



%%%%%%%%%%%%%%%%%%%%%%%%%%%%%%%%%%%%%%%%%%%%%%%%%%%%%%%%%%%
\diapo

[$v_1$ à $v_n$] [$v_1$ etc $v_2$] [$v_1$, $v_2$ etc]

Soit  $\{v_1, \dots , v_n\}$ un ensemble fini quelconque de vecteurs d'un 
$\Kk$-espace vectoriel $E$.

[pause]

Alors : L'ensemble des combinaisons linéaires des vecteurs 
  $\{v_1, \dots , v_n\}$ est un sous-espace vectoriel de $E$.
  
  
\change

En plus c'est le plus petit sous-espace vectoriel de $E$ 
  (au sens de l'inclusion) qui contient tous ces vecteurs. 


\change


Ce sous-espace vectoriel est appelé \defi{sous-espace engendré par les vecteurs $v_1, \dots , v_n$} et est 
 noté $Vect (v_1, \dots , v_n )$. 
 
 
\change

On a donc 

$u \in Vect( v_1, \dots , v_n ) \quad \Longleftrightarrow \quad
 \text{il existe} \ \lambda_1, \dots , \lambda_n \in \Kk \quad \text{tels que} \quad 
 u=\lambda_1v_1+ \dots+\lambda_nv_n$
 


Remarque :  dire que $Vect (v_1, \dots , v_n )$ est le plus petit sous-espace vectoriel 
contenant les vecteurs  $v_1, \ldots , v_n$ signifie que si 
$F$ est un autre sous-espace vectoriel contenant lui aussi les vecteurs $v_1, \ldots , v_n$
alors $Vect (v_1, \dots , v_n ) \subset F$.
 
 

%%%%%%%%%%%%%%%%%%%%%%%%%%%%%%%%%%%%%%%%%%%%%%%%%%%%%%%%%%%
\diapo

Grâce à la notion de sous-espaces vectoriel engendré nous obtenons plein d'exemples d'espaces vectoriels.


$E$ étant un $\Kk$-espace vectoriel, et $u$ un élément quelconque de $E$,
l'ensemble $Vect (u) =\{ \lambda u \mid \lambda \in \Kk \}$ est 
le sous-espace vectoriel de $E$ engendré par $u$.
Il est souvent noté $\Kk u$. Si $u$ n'est pas le vecteur nul, 
on parle d'une \defi{droite vectorielle}.
 
 
 \change
 
Prenons maintenant deux vecteurs $u$ et $v$ de $E$, alors
  $Vect (u,v) = \big\{ \lambda u + \mu v \mid \lambda, \mu \in \Kk \big\}$.
  
  
  Si $u$ et $v$ ne sont pas colinéaires, alors $Vect (u,v)$ est un \defi{plan vectoriel}.
  
  
\change

Voyons un exemple explicite :

$u = \left(\begin{smallmatrix}1 \\ 1 \\ 1 \end{smallmatrix}\right)$, $v = \left(\begin{smallmatrix}1 \\ 2 \\ 3 \end{smallmatrix}\right)$ 
  dans $\Rr^3$.
  Ils ne sont pas colinéaires.
  
  Déterminons le plan vectoriel $\mathcal{P} = Vect (u,v)$.

\change

Dire que $ \left(\begin{smallmatrix}x \\ y \\ z \end{smallmatrix}\right) \in Vect (u,v)$

\change

 équivaut par définition de sous-espace engendré à 
$\left(\begin{smallmatrix}x \\ y \\ z \end{smallmatrix}\right) = \lambda u + \mu v$ \text{pour certains réels $\lambda,\mu$}

\change

cela s'écrit aussi 
$ \left(\begin{smallmatrix}x \\ y \\ z \end{smallmatrix}\right) = 
 \lambda\left(\begin{smallmatrix}1 \\ 1 \\ 1 \end{smallmatrix}\right) + \mu \left(\begin{smallmatrix}1 \\ 2 \\ 3 \end{smallmatrix}\right)
$

\change

ce qui conduit au système 
$
 \left\{
 \begin{array}{rcl}
   x & = & \lambda + \mu \\
   y & = & \lambda + 2 \mu \\
   z & = & \lambda + 3 \mu \\
 \end{array}\right. $
 
 
Nous obtenons bien une équation paramétrique d'un plan $\mathcal{P}$
passant par l'origine et contenant les vecteurs $u$ et $v$.

\change

On saurait en trouver une équation cartésienne : $(x-2y+z=0)$.

\change

Terminons avec un espace de fonctions.

Soient $E$ l'espace vectoriel des fonctions de $\Rr$ dans $\Rr$ et  
$f_0, f_1, f_2$ les fonctions définies par :
$$f_0(x)=1, \;\; f_1(x)=x\;\;\text{ et } \;\; f_2(x)=x^2.$$

\change

Le sous-espace vectoriel de $E$ engendré par $\{f_0, f_1, f_2\}$ 
est l'espace vectoriel des fonctions polynômes $f$ de degré inférieur 
ou égal à $2$, c'est-à-dire de la forme $f(x) = ax^2+bx+c$.


%%%%%%%%%%%%%%%%%%%%%%%%%%%%%%%%%%%%%%%%%%%%%%%%%%%%%%%%%%%
\diapo

La notion de somme de sous-espaces vectoriels est vraiment importante
et mérite que vous y consacriez du temps.


\end{document}
