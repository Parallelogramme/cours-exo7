\documentclass[class=report,crop=false]{standalone}
\usepackage[screen]{../exo7book}


% Commandes locales 
\usepackage{siunitx}
\sisetup{locale = FR, detect-all, per-mode = symbol}


\begin{document}

\newcommand{\vect}{\overrightarrow}

% Pour latitude/longitude 3d
% Stereographic and cylindrical map projections
% Author: Tomasz M. Trzeciak
% Source: LaTeX-Community.org
%         <http://www.latex-community.org/viewtopic.php?f=4&t=2111>

\newcommand\pgfmathsinandcos[3]{%
  \pgfmathsetmacro#1{sin(#3)}%
  \pgfmathsetmacro#2{cos(#3)}%
}
\newcommand\LongitudePlane[3][current plane]{%
  \pgfmathsinandcos\sinEl\cosEl{#2} % elevation
  \pgfmathsinandcos\sint\cost{#3} % azimuth
  \tikzset{#1/.style={cm={\cost,\sint*\sinEl,0,\cosEl,(0,0)}}}
}
\newcommand\LatitudePlane[3][current plane]{%
  \pgfmathsinandcos\sinEl\cosEl{#2} % elevation
  \pgfmathsinandcos\sint\cost{#3} % latitude
  \pgfmathsetmacro\yshift{\cosEl*\sint}
  \tikzset{#1/.style={cm={\cost,0,0,\cost*\sinEl,(0,\yshift)}}} %
}
\newcommand\DrawLongitudeCircle[2][1]{
  \LongitudePlane{\angEl}{#2}
  \tikzset{current plane/.prefix style={scale=#1}}
   % angle of "visibility"
  \pgfmathsetmacro\angVis{atan(sin(#2)*cos(\angEl)/sin(\angEl))} %
  \draw[current plane] (\angVis:1) arc (\angVis:\angVis+180:1);
  \draw[current plane,dashed] (\angVis-180:1) arc (\angVis-180:\angVis:1);
}
\newcommand\DrawLatitudeCircle[2][1]{
  \LatitudePlane{\angEl}{#2}
  \tikzset{current plane/.prefix style={scale=#1}}
  \pgfmathsetmacro\sinVis{sin(#2)/cos(#2)*sin(\angEl)/cos(\angEl)}
  % angle of "visibility"
  \pgfmathsetmacro\angVis{asin(min(1,max(\sinVis,-1)))}
  \draw[current plane] (\angVis:1) arc (\angVis:-\angVis-180:1);
  \draw[current plane,dashed] (180-\angVis:1) arc (180-\angVis:\angVis:1);
}


% \begin{tikzpicture} % "THE GLOBE" showcase
% \def\R{2.5} % sphere radius
% \def\angEl{35} % elevation angle
% \filldraw[ball color=white] (0,0) circle (\R);
% \foreach \t in {-80,-60,...,80} { \DrawLatitudeCircle[\R]{\t} }
% \foreach \t in {-5,-35,...,-175} { \DrawLongitudeCircle[\R]{\t} }
% \end{tikzpicture}


%====================================================================
\chapitre{Mathématiques du GPS}
%====================================================================


%%%%%%%%%%%%%%%%%%%%%%%%%%%%%%%%%%%%%%%%%%%%%%%%%%%%%%%%%%%%%%%%
\section{L'île aux 7 phares}



%---------------------------------------------------------------
\subsection{Perdu !}

Au large des côtes bretonnes existe une île ayant la particularité étrange 
de posséder 7 phares. Cette île, maintenant inhabitée, appartenait à un 
savant passionné de navigation et de technologie. J'étais venu visiter 
chacun des 7 phares. Bien sûr la nuit chaque phare s'illumine et émet un signal
particulier afin que les marins puissent contourner l'île. Chaque phare porte 
le nom d'une note de musique. En effet, lorsque
le brouillard tombe sur l'île, les cornes de brume résonnent, chacune émettant 
un son différent, \emph{do}, \emph{ré}, \emph{mi}, \emph{fa},...
En plus le savant avait construit un mécanisme horloger très précis de sorte que toutes les 
10 minutes, à la seconde près, les 7 cornes de brume sonnaient un bref coup.


\myfigure{0.5}{
  \tikzinput{fig_gps_ile_01}
} 

En arrivant sur l'île, j'avais réglé ma montre sur l'horloge du phare \emph{Do} après avoir grimpé tout en haut,
puis j'étais parti explorer l'île. Après une belle matinée, le brouillard apparût et, 
un peu avant midi, je voyais à peine le bout de mes pieds. J'avais bien une carte de l'île
et une boussole, mais comme je ne savais pas où je me trouvais, j'étais complètement perdu !
Commençant à m’inquiéter, j'entends tout à coup le son d'une corne de brume, puis quelques instants
après une deuxième et bientôt les 7 cornes résonnent chacune à son tour. Rassuré par ces signaux et 
assis sur un rocher je me fis quand même la remarque que les horloges n'étaient 
pas si précises qu'on le disait car elles avaient sonné à plusieurs secondes d'écart.

Plongé dans mes réflexions et surtout dans le brouillard, je n'avais rien d'autre à faire
que de patienter 10 minutes pour les prochains coups de corne. Je me demandais si les 
cornes de brume pouvaient m'aider à m'orienter, peut-être qu'en me dirigeant vers le phare 
émettant le son le plus fort j'arriverais  à le rejoindre. Mais malheureusement, le brouillard
atténuait les sons et je ne pouvais pas déterminer d'où ils venaient.

\newpage
\thispagestyle{empty}
\def\myscale{0.9}
\vfill
\begin{center}
\rotatebox{90}
{
  \tikzinput{fig_gps_ile_02}
}   
\end{center}




\newpage

%---------------------------------------------------------------
\subsection{À toute vitesse}

Soudain une illumination : et si les horloges des phares étaient parfaitement réglées, mais que 
les sons ne me parvenaient pas tous en même temps. Les nuit d'orage, 
lorsque que j'étais petit, je comptais dans mon lit le temps qui s'écoulait entre
l'éclair et le tonnerre, pour savoir si l'orage était loin ou proche.
En effet, la lumière de l'éclair arrive quasi-instantanément alors que le son du tonnerre 
voyage moins vite. On disait qu'un décalage de 3 secondes correspondait à 1 kilomètre.
Me souvenant que la vitesse du son dans l'air est de \SI{1200}{\kilo \meter \per \hour},
soit \SI{333}{\meter\per\second}, il faut bien 3 secondes au son pour parcourir 
1 kilomètre.

Aussitôt je sors ma trousse de survie : papier, crayon, règle, compas
et je me prépare pour les prochains signaux de 12h00.
\`A 12h00 et 0 seconde rien. J'attends 1 seconde, 2 secondes, toujours rien !
Enfin à 12h00 et 9 secondes j'entends un \emph{do}, 
suivi à 12h00 et 12 secondes de \emph{ré}, puis les sons s’enchaînent sans que j'ai le temps
de les discerner, sauf un \emph{la} feutré qui conclut à 12h00 et 24 secondes.

Je me concentre sur le son \emph{do}. Le son \emph{do} est parti du phare à 12h00 tapante
et arrive à mes oreilles 9 secondes plus tard. Il a donc mis 9 secondes pour voyager,
et comme en 3 secondes il parcourt 1 kilomètre, il a voyagé sur 3 kilomètres.

Je me trouve donc à \SI{3}{\kilo \meter} du phare \emph{do}. Géométriquement cela signifie que je me trouve
sur un cercle de rayon \SI{3}{\kilo \meter}, centré sur le phare \emph{do}.
Sur la carte de l'île, je m'empresse de tracer ce cercle.

\myfigure{0.5}{
  \tikzinput{fig_gps_ile_03}
}

Bien sûr cela ne m'aide qu'à moitié car je ne sais pas où je suis exactement situé sur ce cercle.
Voyons ce que m'indique le phare \emph{ré}, le son a mis 12 secondes à parvenir, donc je suis 
à \SI{4}{\kilo \meter} du phare \emph{ré}. Je dessine le cercle centré sur ce phare et de rayon 
\SI{4}{\kilo \meter}. 

Fantastique ! Ces deux cercles se coupent en seulement deux points (normal pour deux cercles).
Comme l'un des points est au milieu de l'eau, je sais exactement où je suis.
Je ne suis plus perdu !




%---------------------------------------------------------------
\subsection{Pour aller plus loin...}

Avant d'aller plus loin je préfère vérifier mes déductions.
Le son \emph{la} du dernier phare est parvenu après 24 secondes, donc je suis
à \SI{8}{\kilo \meter} du phare \emph{la}. Je trace le cercle correspondant
et effectivement ce cercle passe presque par le point où je pensais me trouver.
Voilà qui est rassurant.
Ce qui l'est moins c'est que depuis mon arrivée au phare du \emph{do}, je me suis dirigé 
plein Nord et que si je continue je vais me diriger droit vers les marais.
Je décide donc de m'orienter plutôt vers le Nord-Est afin de rejoindre les phares 
jumeaux \emph{fa} et \emph{sol}.

Au bout d'une heure de marche plus ou moins laborieuse dans le brouillard, je refais le point
avec les cornes de brume de 13h00.
Cette fois à 13h00 et 9 secondes j'entends en même temps le phare \emph{fa} et le phare \emph{sol}.
Puis à 13h00 et 12 secondes le phare \emph{la}.

\myfigure{0.5}{
  \tikzinput{fig_gps_ile_04}
}


Je reprends ma carte, je dessine un cercle de rayon \SI{3}{\kilo \meter}
centré sur le phare \emph{fa}, puis un autre cercle de même rayon mais centré sur le phare 
\emph{sol}. Malheureusement ces deux cercles sont trop proches l'un de l'autre, ce qui fait
qu'avec l'épaisseur du trait, j'ai un grosse zone d'intersection.
Pas de problème, je trace le cercle centré sur le phare \emph{la} de rayon \SI{4}{\kilo \meter}
qui lui recoupe correctement les deux premiers cercles
en un seul point de l'île (et pas dans l'eau).

Je sais encore une fois précisément où je suis. En plus 
par rapport à ma position d'il y a une heure, je mesure sur la carte
que j'ai parcouru un peu plus de \SI{4}{\kilo \meter}
(environ \SI{4.3}{\kilo \meter}) donc ma vitesse moyenne (en ligne droite)
est de plus de \SI{4}{\kilo \meter \per \hour}.

Enfin, je vois clairement que ma direction depuis 12h00
n'est pas Nord-Est mais plutôt Est-Nord-Est, en fait avec mon rapporteur, je mesure précisément que mon cap est de
\ang{63} par rapport au Nord. Je n'ai même plus besoin de boussole.

C'est alors que le soleil revient, et je me trouve non loin des phares jumeaux.



%---------------------------------------------------------------
\subsection{\`A vous de jouer}

Si vous avez bien compris voici une liste de petits problèmes de difficulté croissante.
Armez-vous de votre matériel de géométrie, d'un papier et d'un crayon.


\begin{enumerate}
  \item \`A la suite de ma balade je me trouve à un point $P_3$
  où j'entends le phare \emph{si} à 6 secondes et le phare \emph{la}
  à 15 secondes. Où suis-je ?
  
  \item Deux heures plus tard, je me trouve au point $P_4$ où j'entends le phare
  \emph{ré} à 8 secondes et le phare \emph{mi} à 11 secondes. Où suis-je ?
  Quelle a été ma vitesse moyenne ? Et ma direction par rapport au Nord ?
  
  \item Je me trouve dans une zone située entre 9 et 10 secondes du phare \emph{ré}
  et entre 11 et 12 secondes du phare \emph{do}. Dessiner cette zone. Mesurer graphiquement
  l'erreur maximale commise. 
  
  \item Partant de la côte ouest, je me promène en prenant bien soin 
  d'entendre la corne de brume des phares \emph{ré} et \emph{mi} en même temps.
  Vers quel phare je me dirige ?
  
  \item Je me promène maintenant de sorte que le son venant du phare \emph{si}
  arrive avec le double de temps que le son venant du phare \emph{do}.
  Où puis-je être ? Si je ne suis pas sur l'île principale, où suis-je ?
  
  \item Ma montre fonctionne toujours, mais elle n'est plus à l'heure ! 
  J'entends le phare \emph{si}, puis 5 secondes plus tard le phare \emph{la}
  et encore 3 secondes après le phare \emph{ré}. Où suis-je ?
  
  
\end{enumerate}

%---------------------------------------------------------------
\subsection{Bilan}

Voici quelques conclusions de notre visite sur l'île.

\begin{itemize}
  \item  Avec deux signaux, je détermine deux positions possibles. L'une des deux peut souvent
  être exclue car aberrante.
  
  \item En répétant les mesures des signaux au fil du temps, je peux en déduire
  mon parcours, mais aussi ma vitesse ainsi que la direction suivie.
  
  \item Si mes mesures sont imprécises alors, au lieu d'un point, je me situe dans une zone.
  
  \item Deux signaux qui proviennent de points très proches (comme le cas des phares jumeaux) 
  fournissent une information redondante, qui ne permet pas de déterminer sa position avec une précision raisonnable.
    
  \item La mesure d'un troisième signal permet de valider le choix d'un point (par exemple s'il reste deux points possibles)
  ou bien de réduire la taille d'une zone.
  
  \item Enfin, avec trois signaux je peux me passer d'une montre parfaitement à l'heure, 
  en utilisant seulement la fonction chronomètre.

  
\end{itemize}



%%%%%%%%%%%%%%%%%%%%%%%%%%%%%%%%%%%%%%%%%%%%%%%%%%%%%%%%%%%%%%%%
\section{Se repérer grâce au GPS}

%---------------------------------------------------------------
\subsection{Deux cercles}

Chaque satellite du système GPS, ou chaque phare de l'île, émet un signal à un instant précis.
Ce signal voyage  au cours du temps dans toutes les directions, 
sous la forme d'un cercle qui s'agrandit (ou d'une sphère dans l'espace)
comme lorsque l'on lance un caillou dans l'eau.

\myfigure{1}{
  \tikzinput{fig_gps_reperer_01}
} 

On se place pour l'instant dans le plan. Si le signal est émis au point $S_1$
à l'instant $t_1=0$ et que le signal se déplace à une vitesse $c$, alors à un instant $t>0$ quelconque,
le signal est perçu exactement en tout point du cercle $\mathcal{C}_1$ centré en $S_1$ et de rayon $c \cdot t$.
Si le centre $S_1$ a pour coordonnées $(x_1,y_1)$ alors l'équation de ce cercle est
$$\sqrt{(x-x_1)^2 + (y-y_1)^2} = c \cdot t.$$
Où encore en élevant au carré :
$$(x-x_1)^2 + (y-y_1)^2 = c^2 \cdot t^2.$$

Rappelons que je cherche à déterminer ma position $P$, et donc si je reçois le signal 
à l'instant~$t$, j'en déduis que je suis situé sur le cercle $\mathcal{C}_1$ (de rayon $c\cdot t$).
Le fait que deux signaux dans le plan déterminent seulement deux positions possibles 
est la traduction mathématique du résultat suivant :
\begin{proposition}
Deux cercles du plan, ayant des centres distincts, se coupent en $0$, $1$, ou $2$ points.  
\end{proposition}

\myfigure{0.8}{
  \tikzinput{fig_gps_reperer_02a}\qquad\qquad\qquad\qquad
  \tikzinput{fig_gps_reperer_02b}\qquad\qquad
  \tikzinput{fig_gps_reperer_02c} 
} 

Dans notre situation l'intersection n'est pas vide, puisque nous sommes
à un point d'intersection des deux cercles.
La preuve va même nous fournir les coordonnées des points d'intersection.
Lorsque qu'il n'y a qu'un point d'intersection, c'est que les deux cercles sont tangents.
C'est une situation exceptionnelle qui ne peut pas nous être utile dans la pratique.


\begin{proof}
On considère deux cercles $\mathcal{C}_1$ et $\mathcal{C}_2$.
Pour simplifier les calculs, et sans perte de généralité, on choisit le repère de sorte que le centre du premier cercle
soit $(x_1,y_1) = (0,0)$, et on choisit l'axe des abscisses de sorte que le centre du second cercle soit dessus :
$(x_2,y_2)=(x_2,0)$. 

\myfigure{1.2}{
  \tikzinput{fig_gps_reperer_03}
} 

Les équations des cercles sont alors
$$x^2 + y^2 = r_1^2 \qquad \text{ et } \qquad (x-x_2)^2 + y^2 = r_2^2$$
Un point $P = (x,y)$ est dans l'intersection si ses coordonnées sont solutions de 
$$\left\{\begin{array}{rcl}
x^2 + y^2 &=& r_1^2 \\
(x-x_2)^2 + y^2 &=& r_2^2
\end{array}\right.
\quad \text{ ou encore }\quad
\left\{\begin{array}{rcl}
x^2 + y^2 &=& r_1^2 \\
x^2-2xx_2 +x_2^2 + y^2 &=& r_2^2
\end{array}\right.$$
Attention, c'est un système à deux équations et deux inconnues, mais les équations ne sont pas linéaires.
En retranchant la première ligne à la seconde, ce système équivaut à :
$$\left\{\begin{array}{rcl}
x^2 + y^2 &=& r_1^2 \\
-2xx_2 + x_2^2 &=& r_2^2-r_1^2
\end{array}\right.
\quad \text{ c'est-à-dire }\quad
\left\{\begin{array}{rcl}
x^2 + y^2 &=& r_1^2 \\
x &=& \frac{r_1^2-r_2^2+x_2^2}{2x_2}
\end{array}\right.$$
On a donc trouvé l'abscisse $x$ de nos solutions. En reportant la valeur de $x$ dans la première équation
on trouve :
$$y^2 = r_1^2 - \left(\frac{r_1^2-r_2^2+x_2^2}{2x_2}\right)^2$$
Notons $\rho$ cette quantité, $\rho = r_1^2 - \left(\frac{r_1^2-r_2^2+x_2^2}{2x_2}\right)^2$.
Trois cas sont possibles :
\begin{itemize}
  \item Si $\rho < 0$ alors l'équation $y^2 = \rho$ n'admet pas de solutions et notre système non plus.
  Les deux cercles ne se coupent pas.
  
  \item Si $\rho = 0$ alors $y=0$. Le système admet une unique solution
  donnée par $x= \frac{r_1^2-r_2^2+x_2^2}{2x_2}$ et $y=0$. Les deux cercles se coupent en un unique point.
  
  \item Si $\rho > 0$ alors $y = +\sqrt \rho$ ou $y=-\sqrt \rho$.
  Le système admet deux solutions données par $\left(\frac{r_1^2-r_2^2+x_2^2}{2x_2},+\sqrt \rho\right)$
  et  $\left(\frac{r_1^2-r_2^2+x_2^2}{2x_2},-\sqrt \rho\right)$. Les deux cercles se coupent en deux points.
\end{itemize}
  
\end{proof}


%---------------------------------------------------------------
\subsection{Trois sphères}

Dans l'espace chaque satellite émet un signal qui se propage en une famille de sphères
centrées sur le satellite, dont le rayon grandit avec le temps.
Lorsque l'on reçoit un signal d'un satellite, 
nous savons que nous sommes situés sur une sphère centrée en ce satellite.
Avec deux satellites, nous savons que nous sommes sur l'intersection de deux sphères,
ce qui laisse une infinité de possibilités. Il faut trois sphères pour
n'avoir que deux possibilités pour notre position.
C'est exactement ce qui dit le théorème :
\begin{theoreme}
Trois sphères, dont les centres ne sont pas alignés, 
ont une intersection commune de $0$, $1$ ou $2$ points.
\end{theoreme}

Encore une fois dans une situation normale, il y aura deux points d'intersections.
On exclut un des points, qui correspond souvent à une solution aberrante 
(par exemple sous terre ou dans l'espace) ou alors on valide une solution par une quatrième signal.
Nous allons voir deux méthodes qui permettent de calculer les solutions de façon exacte. 
Nous verrons une troisième méthode, par calcul approché,
lorsque nous aborderons les problèmes d'erreurs.

Mais avant cela commençons par bien comprendre la situation géométrique.
Tout d'abord pour deux sphères qui se rencontrent, 
leur intersection est un cercle de l'espace. Ceci n'est pas si évident à visualiser, 
mais nous le montrerons par le calcul. Lorsque l'on a trois
sphères qui se coupent, alors les deux premières s'intersectent en un cercle (figure de gauche), 
et ce cercle va recouper la troisième sphère en deux points (figure du milieu).
L'intersection des trois sphères est alors formée de deux points (figure de droite).


\myfigure{1}{  
  \tikzinput{fig_gps_3d_10} \quad
  \tikzinput{fig_gps_3d_11} \quad
  \tikzinput{fig_gps_3d_12} \quad
}


%---------------------------------------------------------------
\subsection{Preuve analytique}


Pour simplifier les calculs de l'intersection de trois sphères, nous choisissons
le repère de sorte que :
\begin{itemize}
  \item la sphère $\mathcal{S}_1$ soit centrée à l'origine, c'est-à-dire son centre
  vérifie $(x_1,y_1,z_1) = (0,0,0)$ ;
  \item la sphère $\mathcal{S}_2$ soit centrée sur l'axe des abscisses $(Ox)$, c'est-à-dire son centre
  vérifie $(x_2,y_2,z_2) = (x_2,0,0)$ ;
  \item la sphère $\mathcal{S}_3$ soit centrée sur le plan $(Oxy)$, c'est-à-dire son centre
  vérifie $(x_3,y_3,z_3) = (x_3,y_3,0)$.
\end{itemize}

Les équations des sphères sont alors :

$$\left\{\begin{array}{rcl}
x^2 + y^2 + z^2 &=& r_1^2 \\
(x-x_2)^2 + y^2 + z^2 &=& r_2^2 \\
(x-x_3)^2 + (y-y_3)^2 + z^2 &=& r_3^2 \\
\end{array}\right.$$
On soustrait la première ligne aux deux suivantes :
$$\left\{\begin{array}{rcl}
x^2 + y^2 + z^2 &=& r_1^2 \\
(x-x_2)^2 -x^2 &=& r_2^2-r_1^2 \\
(x-x_3)^2 -x^2 + (y-y_3)^2 - y^2 &=& r_3^2-r_1^2 \\
\end{array}\right.$$
En développant les carrés, le système se simplifie, et on trouve $x$, puis $y$ (en fonction de ce $x$) :
$$\left\{\begin{array}{rcl}
x^2 + y^2 + z^2 &=& r_1^2 \\
x &=& \frac{r_1^2-r_2^2+x_2^2}{2x_2}\\
y &=& \frac{r_1^2-r_3^2+x_3^2+y_3^2-2xx_3}{2y_3}\\
\end{array}\right.$$

Remarquons que si on ne regarde que l'intersection des deux premières 
sphères (donc les deux premières équations) 
alors on trouve $x^2 + y^2 + z^2 = r_1^2$
avec $x=x_0$ fixé ($x_0= \frac{r_1^2-r_2^2+x_2^2}{2x_2}$).
Cette équation $y^2 + z^2 = r_1^2 - x_0^2$ est l'équation d'un cercle
dans le plan d'équation $(x=x_0)$ orthogonal à la droite 
joignant les deux centres.


Revenons à nos trois sphères ;
$x$ et $y$ sont donc entièrement déterminés par $x=x_0$ et $y=y_0$ avec
$$x_0 = \frac{r_1^2-r_2^2+x_2^2}{2x_2}
\quad \text{ et } \quad
y_0 = \frac{r_1^2-r_3^2+x_3^2+y_3^2-2x_0x_3}{2y_3}$$

Il nous reste à trouver $z$, à l'aide de la première équation devenue
$x_0^2 + y_0^2 +z^2 = r_1^2$.
On pose $\rho = r_1^2 - x_0^2 - y_0^2$, de sorte que 
$z^2 = \rho$ et ainsi :
\begin{itemize}
  \item si $\rho < 0$, pas de solutions,
  \item si $\rho = 0$, une unique solution $(x_0,y_0,0)$,
  \item si $\rho > 0$, deux solutions $(x_0,y_0,+\sqrt\rho)$,
  $(x_0,y_0,-\sqrt\rho)$.
\end{itemize}

\begin{exemple}
Quelle est l'intersection  des sphères suivantes, dont on donne le centre et le rayon ?
$$\mathcal{S}_1  : \quad (0,0,0), \quad r_1 = 4 \quad ; \quad 
\mathcal{S}_2, \quad (4,0,0), \quad r_2 = 3 \quad ; \quad 
\mathcal{S}_3, \quad (2,1,0), \quad r_3 = 2$$

Les équations sont
$$\left\{\begin{array}{rcl}
x^2 + y^2 + z^2 &=& 16 \\
(x-4)^2 + y^2 + z^2 &=& 9 \\
(x-2)^2 + (y-1)^2 + z^2 &=& 4 \\
\end{array}\right.$$
On reprend pas à pas la démonstration, ou bien on applique directement les formules obtenues, pour trouver
$$x_0 = \frac{23}{8}, \quad \quad y_0 = \frac{11}{4}.$$
On pose $\rho = r_1^2-x_0^2-y_0^2 = 
16 - \left(\frac{23}{8}\right)^2 - \left(\frac{11}{4}\right)^2
= \frac{11}{64}$.
Comme $\rho>0$ alors en posant
$z_0 = \sqrt\rho = \frac{\sqrt{11}}{8}$, 
les trois sphères se coupent en deux points :
$$(x_0,y_0,+z_0) = \left(\frac{23}{8},\frac{11}{4},\frac{\sqrt{11}}{8}\right) 
\quad \text{ et } \quad (x_0,y_0,-z_0)= \left(\frac{23}{8},\frac{11}{4},-\frac{\sqrt{11}}{8}\right) .$$
\end{exemple}



%---------------------------------------------------------------
\subsection{Preuve vectorielle}

Les calculs précédents sont tout à fait valides, mais nous aurons besoin
d'une autre méthode pour plusieurs raisons : 
(a) le problème de synchronisation de l'horloge du récepteur GPS ;
(b) la gestion de plus de trois satellites.

Pour plus de clarté, nous allons d'abord expliquer cette méthode dans le plan seulement
(avec deux cercles, c'est-à-dire avec deux satellites). 
Nous allons transformer les deux équations non linéaires correspondant aux deux signaux,
en deux systèmes linéaires. Chaque système fournira une solution.

Rappelons que dans le plan les deux équations sont
$$(x-x_1)^2 + (y-y_1)^2 = r_1^2 
\quad\text{ et }\quad
(x-x_2)^2 + (y-y_2)^2 = r_2^2$$
On développe chacune des équations ($i=1,2$) :
$$x^2 -2xx_i + x_i^2 + y^2-2yy_i + y_i^2 = r_i^2$$
ce qui donne le système des deux équations :
\begin{equation}
\label{eq :gps2cercles}
\left\{\begin{array}{rcl}
2xx_1 + 2yy_1 &=& x^2+y^2 + x_1^2 + y_1^2 - r_1^2 \\
2xx_2 + 2yy_2 &=& x^2+y^2 + x_2^2 + y_2^2 - r_2^2 
\end{array}
\right.  
\end{equation}

Passons à une notation vectorielle et matricielle, on pose :
$$
A = \begin{pmatrix}2x_1&2y_1\\2x_2&2y_2\end{pmatrix}\qquad
X = \begin{pmatrix}x\\y\end{pmatrix}$$

Alors 
$$A X = \begin{pmatrix}2xx_1&2yy_1\\2xx_2&2yy_2\end{pmatrix}$$

Posons 
$$U = \begin{pmatrix}1\\1\end{pmatrix} \qquad
V = \begin{pmatrix}x_1^2 + y_1^2 - r_1^2\\x_2^2 + y_2^2 - r_2^2\end{pmatrix}$$
de sorte que 
$$B = (x^2+y^2) U + V = 
\begin{pmatrix}x^2+y^2 + x_1^2 + y_1^2 - r_1^2\\
x^2+y^2 + x_2^2 + y_2^2 - r_2^2\end{pmatrix}$$
Les équations (\ref{eq :gps2cercles}) s'écrivent simplement
\begin{equation}
\label{eq :gps2cerclesmat}
AX=B
\end{equation}

Supposons pour l'instant que la matrice $A$ soit inversible, 
alors 
$$X = A^{-1} B$$
où $X = \begin{pmatrix}x\\y\end{pmatrix}$ est la position à déterminer.

Il y a cependant un problème ! Contrairement à ce que je vous laisse croire,
l'écriture $AX=B$ ne correspond pas à un système linéaire, car
$B$ dépend de $x$ et $y$ (donc de $X$). Nous allons profiter du fait que $B$ ne dépende que de 
$x^2+y^2$ afin de montrer qu'il y a seulement deux choix possibles pour $B$.

Pour $X = \begin{pmatrix}x\\y\end{pmatrix}$, 
$X' = \begin{pmatrix}x'\\y'\end{pmatrix}$, 
on définit le produit scalaire :
$$\langle X \mid X' \rangle =  xx'+yy'$$
et la norme
$$\| X \| = \sqrt{\langle X \mid X \rangle} = \sqrt{x^2+y^2}$$

Posons $d = x^2+y^2$, alors
\begin{align*}
 d = x^2+y^2 = \| X \|^2 
 &= \langle  X \mid X \rangle \\
 &= \langle  A^{-1} B \mid A^{-1} B \rangle \quad \text{ car } AX=B \\
 &= \langle  A^{-1} ( dU+V)\mid A^{-1}( dU+V)\rangle \quad \text{ car } B = dU+V \\  
 &= \langle  A^{-1} ( dU+V)\mid dA^{-1} U\rangle
 + \langle  A^{-1} ( dU+V)\mid A^{-1} V \rangle \\
 &= \langle  dA^{-1} U\mid dA^{-1} U\rangle
 + \langle A^{-1} V\mid dA^{-1} U\rangle \\
 &\quad + \langle  dA^{-1}  U \mid A^{-1} V \rangle
 + \langle  A^{-1} V\mid A^{-1} V \rangle \\ 
 &= d^2 \langle  A^{-1} U\mid A^{-1} U\rangle
 + 2d\langle  A^{-1}  U \mid A^{-1} V \rangle
 +  \langle  A^{-1} V\mid A^{-1} V \rangle \\ 
 &= d^2 \|  A^{-1}  U  \|^2 + 2d\langle  A^{-1}  U \mid A^{-1} V \rangle
 +\| A^{-1} V \|^2 \\
 \end{align*}

Ainsi $d=x^2+y^2$ vérifie une équation du second degré :
\begin{equation}
\label{eq :gpsseconddegre}
\|  A^{-1}  U  \|^2 d^2 
+ \left(2\langle  A^{-1}  U \mid A^{-1} V \rangle - 1\right) d  +\| A^{-1} V \|^2 = 0  
\end{equation}


Cette équation admet (au plus) deux solutions positives $d_1$ et $d_2$, donc deux possibilités pour
le second membre de (\ref{eq :gps2cerclesmat}) :
$$B_1 = d_1U+V \quad \text{ ou } \quad B_2 = d_2U+V.$$

$B_1$ et $B_2$ sont maintenant clairement déterminés (ils ne dépendent plus de $x$ et de $y$).
Pour le second membre $B_1$, on résout le système linéaire
$AX=B_1$ en $X = A^{-1}B_1$, puis
on résout le système linéaire $AX=B_2$, en $X = A^{-1}B_2$.

Ce qui donne bien deux solutions possibles pour notre position $X$.



\bigskip

Terminons par quelques commentaires.


\begin{itemize}
  \item Imaginons que l'origine du repère soit le centre de la Terre, 
  alors la matrice $A$ est inversible si et seulement si les vecteurs
  $\begin{pmatrix}x_1\\y_1\end{pmatrix}$, $\begin{pmatrix}x_2\\y_2\end{pmatrix}$ 
  ne sont pas colinéaires, c'est-à-dire, si et seulement si,
  les deux satellites ne sont pas alignés avec le centre de la Terre.
  
\myfigure{1.3}{
  \tikzinput{fig_gps_reperer_06a}\qquad
  \tikzinput{fig_gps_reperer_06b}
}
  
  \item  Dans ce repère $\sqrt d$ représente le rayon du cercle, centré au centre
  de la Terre sur lequel on se trouve. Le fait de trouver deux possibilités pour $d$,
  correspond à deux rayons possibles. Si l'on se déplace à la surface de la Terre, 
  il est facile de choisir le «bon» rayon.
  
\myfigure{1.1}{  
  \tikzinput{fig_gps_reperer_07}
}   
\end{itemize}


\begin{exemple}
On considère un cercle $\mathcal{C}_1$ de centre $S_1 = (1,1)$ et de rayon $r_1=2$
et un cercle de $\mathcal{C}_2$ de centre $S_2 = (-2,-1)$ et de rayon $r_2=3$.
On pose comme ci-dessus :
$$
A = \begin{pmatrix}2x_1&2y_1\\2x_2&2y_2\end{pmatrix}
= \begin{pmatrix}2&2\\-4&-2\end{pmatrix}\quad
X = \begin{pmatrix}x\\y\end{pmatrix}\quad
U = \begin{pmatrix}1\\1\end{pmatrix} \quad
V = \begin{pmatrix}x_1^2 + y_1^2  - r_1^2\\x_2^2 + y_2^2  - r_2^2\end{pmatrix}
= \begin{pmatrix}-2\\-4\end{pmatrix}\quad
B = (x^2+y^2) U + V$$


On calcule 
$$A^{-1} = \frac12\begin{pmatrix}-1&-1\\2&1\end{pmatrix} \qquad
A^{-1}U = \begin{pmatrix}-1\\\frac32\end{pmatrix}\qquad
A^{-1}V = \begin{pmatrix}3\\-4\end{pmatrix}$$
et donc 
$$\|  A^{-1}  U  \|^2 = \frac{13}{4}\qquad
\langle  A^{-1}  U \mid A^{-1} V \rangle = -9\qquad
\| A^{-1} V \|^2= 25$$

Alors l'équation (\ref{eq :gpsseconddegre}) que vérifie $d=x^2+y^2$ s'écrit ici :
$$\frac{13}{4} d^2 
-19 d  + 25 = 0$$
Les deux solutions sont :
$$d_1 = \frac{50}{13} \qquad\text{ et } \qquad
d_2 = 2$$


\myfigure{1.2}{  
  \tikzinput{fig_gps_reperer_08}
}  

On obtient deux seconds membres possibles :
$$B_1 = d_1 U + V = \begin{pmatrix}\frac{24}{13}\\\frac{-2}{13}\end{pmatrix}\qquad
B_2 = d_2 U + V = \begin{pmatrix}0\\-2\end{pmatrix}$$                      
Et donc deux points d'intersection dont les coordonnées sont données 
par $X= A^{-1}B$ :
$$P_1 = A^{-1}B_1 = \begin{pmatrix}\frac{-11}{13}\\\frac{23}{13}\end{pmatrix}
\qquad \text{ et } \qquad
P_2 = A^{-1}B_2 = \begin{pmatrix}1\\-1\end{pmatrix}$$ 
\end{exemple}

%---------------------------------------------------------------
\subsection{Dimension $3$}

Le cas de la dimension trois est vraiment très similaire, c'est l'avantage de la méthode vectorielle.
Traitons-le succinctement,
à partir de trois équations ($i=1,2,3$), 
$$(x-x_i)^2 + (y-y_i)^2 + (z-z_i)^2 = r_i^2$$
fournies par le signal de trois satellites, on développe, pour obtenir :
$$2xx_i + 2yy_i + 2zz_i = x^2+y^2+z^2 + x_i^2 + y_i^2 +z_i^2 - r_i^2$$
On pose alors
$$
A = \begin{pmatrix}2x_1&2y_1&2z_1\\2x_2&2y_2&2z_2\\2x_3&2y_3&2z_3\end{pmatrix}\quad
X = \begin{pmatrix}x\\y\\z\end{pmatrix}\quad
U = \begin{pmatrix}1\\1\\1\end{pmatrix} \quad
V = \begin{pmatrix}
x_1^2 + y_1^2 +z_1^2 - r_1^2\\
x_2^2 + y_2^2 +z_2^2 - r_2^2\\
x_3^2 + y_3^2 +z_3^2 - r_3^2
\end{pmatrix}\quad
B = (x^2+y^2+z^2) U + V$$
de sorte que l'on se ramène au système $AX=B$.
On détermine deux solutions possibles pour $d=x^2+y^2+z^2$, 
qui vérifie exactement la même équation (\ref{eq :gpsseconddegre}).
Il reste alors à résoudre deux systèmes linéaires $AX=B_1$ et $AX=B_2$.


%%%%%%%%%%%%%%%%%%%%%%%%%%%%%%%%%%%%%%%%%%%%%%%%%%%%%%%%%%%%%%%%
\section{Temps}

Avant d'être un instrument de mesure de position, le GPS 
est un instrument de mesure du temps !


%---------------------------------------------------------------
\subsection{Le vieux problème de la longitude}

La navigation des bateaux s'est longtemps heurtée à un problème crucial : 
où est le bateau ? Pour se diriger les marins disposent d'une boussole
qui permet de suivre un cap, mais le préalable est 
de savoir où l'on est ! Pour se repérer sur Terre (et sur mer) on utilise
les coordonnées latitude et longitude (on y reviendra).

\myfigure{0.8}{  
  \tikzinput{fig_gps_3d_08} 
}

Déterminer la latitude est facile, imaginons que nous sommes le 21 mars, le jour
de l'équinoxe. Ce jour là, lorsqu'il est au plus haut, 
le Soleil est vertical pour les gens situés à l'équateur.

Pour un marin situé n'importe où en mer, il détermine à l'aide d'un sextant 
(une sorte de rapporteur) l'angle entre l'horizon et le Soleil, 
lorsque celui-ci est au plus haut dans le ciel.
Cet angle est exactement $\ang{90} - \varphi$, où $\varphi$ est la latitude du lieu.
Il connaît donc sa latitude.

\myfigure{1}{  
  \tikzinput{fig_gps_temps_04}
} 

Pour d'autres jours il faut utiliser des tables astronomiques qui donne
la hauteur attendue du Soleil. La nuit on peut faire un calcul 
similaire avec l'étoile polaire, dans l'hémisphère nord.

\bigskip

Le problème de la longitude est beaucoup plus délicat, des tentatives ont été faites avec 
des cartes de la position des étoiles mais cela restait très imprécis. La solution
vient du temps.
En 1735, John Harrison horloger en Angleterre invente un chronomètre maritime 
qui allait révolutionner la navigation.
Ce système permet à chaque bateau de disposer d'une horloge qui fournit 
l'heure avec une bonne précision, chose qui n'était pas possible avant, les précédentes horloges 
avançaient ou retardaient à cause des mouvements du bateau ou des changements de température et de pression.

Comment l'heure exacte permet de trouver la longitude ?
Imaginons un bateau quittant le port de Londres à midi pile, le 1\up{er} juillet
avec à son bord un horloge précise. Au bout de dix jours en mer à naviguer 
vers le Sud-Ouest, le capitaine souhaite savoir où il est. Lorsqu'il est midi pour lui, c'est-à-dire 
lorsque le Soleil est au plus haut dans le ciel, il mesure la latitude 
comme on vient de l'expliquer. Puis il retourne à sa cabine pour voir l'heure indiquée par l'horloge,
sachant que l'horloge indique toujours l'heure de Londres.
Par exemple l'horloge indique 14h30. Ce qui signifie que le Soleil était au plus haut 2h30 plus tôt à Londres.
Ce décalage est simplement dû à la rotation de la Terre. La Terre fait un tour
complet ($\ang{360}$) sur elle-même en 24h, donc $\ang{15}$ en 1 heure.
Donc ici le bateau est à $\ang{37,5} (= \num{2.5} \times \ang{15}$) à l'ouest du méridien de Greenwich situé
dans la banlieue de Londres.

\myfigure{1.5}{  
  \tikzinput{fig_gps_3d_09} 
}

Si l'horloge (et les mesures) sont précises à $\SI{30}{\second}$ près alors l'erreur commise 
sur la longitude est inférieure à $\SI{15}{\kilo\meter}$. Ce qui est tout fait raisonnable 
après avoir traversé l'océan !


\bigskip

\textbf{Conversion temps -- angle -- distance.}

Voyons différentes façons de mesurer des déplacements si l'on voyage le long de l'équateur.
On note $R$ le rayon de la Terre (on prendra $R = \SI{6300}{\kilo\meter}$). La Terre fait un tour complet en $\SI{24}{\hour}$ et la longueur de l'équateur est $2\pi R$, alors on a la correspondance temps/angle/distance:
$$\SI{24}{\hour}\longleftrightarrow\ang{360}\longleftrightarrow 2\pi R$$

On applique la règle de trois, pour obtenir exemple 
$$\SI{1}{\minute}\longleftrightarrow\ang{0.25}\longleftrightarrow \SI{27.5}{\kilo\meter}$$

Ces calculs sont aussi valables si on voyage le long d'un méridien (n'importe lequel).

Par contre si on voyage sur un parallèle autre que l'équateur alors il faut tenir compte de la latitude $\varphi$. Le parallèle est un cercle de longueur $2\pi R \cos \varphi$.
On a alors la correspondance :
$$\SI{24}{\hour}\longleftrightarrow\ang{360}\longleftrightarrow 2\pi R\cos \varphi$$

Par exemple pour une latitude de $\varphi = \ang{45}$ alors la correspondance devient :
$$\SI{1}{\minute}\longleftrightarrow\ang{0.25}\longleftrightarrow \SI{19.4}{\kilo\meter}$$

%---------------------------------------------------------------
\subsection{Vision géométrique (une dimension)}

Commençons par un modèle spatial en une seule dimension !
Pour se repérer sur une droite, un seul satellite suffit à déterminer
deux abscisses possibles, si nous disposons d'une mesure absolue du temps.
En effet pour un signal émis au temps $t_1$ (cette information est fournie 
dans le signal, de même que la position du satellite) et reçu au temps $t$, 
la distance parcourue par le signal est $r_1= c(t-t_1)$ : c'est la distance entre notre position $x$ et 
la position $x_1$ du satellite. On doit donc résoudre l'équation :
$$|x-x_1 | = r_1$$
et on trouve deux solutions $x=x_1+r_1$ et $x=x_1-r_1$.
On préfère écrire l'équation :
$$(x-x_1)^2 =  c^2(t-t_1)^2.$$ 
Géométriquement on trouve les solutions $x$ en traçant le cercle de rayon $r_1$ centré en $x_1$.

\myfigure{1}{  
  \tikzinput{fig_gps_temps_05}
}  


Le problème c'est que l'horloge du satellite et l'horloge du récepteur GPS doivent 
être parfaitement synchronisées, afin de mesurer le décalage de temps. La satellite dispose 
d'une horloge atomique de très grande précision mais pas le récepteur GPS.
Il faut donc considérer que l'instant $t$ de réception n'est pas connu 
(du moins pas avec une précision suffisante). Il nous faut un autre satellite et un autre signal.

Avec deux satellites nous avons deux équations :
$$\left\{\begin{array}{rcl} 
         |x-x_1| &=& c |t-t_1| \\  
         |x-x_2| &=& c |t-t_2| \\ 
         \end{array}
 \right.$$
 
Si on représente le déplacement du signal en fonction de l'abscisse $x$ au cours du temps $t$ et 
si le signal est émis au point $x_1$ à l'instant $t_1$, alors le signal
part du point $(x_1,t_1)$ et «voyage» le long d'un cône.
Ce cône correspond à l'équation $|x-x_1| = c |t-t_1|$, dont les solutions sont
les points des deux droites d'équations $x-x_1 = + c(t-t_1)$ et
$x-x_1 = - c(t-t_1)$.
Donc avec le signal d'un satellite, nous savons que notre position $(x,t)$ est sur ce cône.

\myfigure{1}{  
  \tikzinput{fig_gps_temps_06}\qquad
  \tikzinput{fig_gps_temps_07}
}  

Avec deux satellites, nous disposons de deux cônes.
Ces deux cônes s'intersectent en un seul point. Ce point est notre position
$(x,t)$. On vient de déterminer à la fois notre position $x$, mais aussi 
l'instant $t$ auquel on reçoit les deux signaux.

Noter que les deux cônes sont composés de droites parallèles, car pour les deux cônes
les pentes de droite sont $+\frac1c$ et $-\frac1c$ (et la vitesse $c$ du signal est la même pour tous les satellites).

\bigskip

Remarque : dans l'introduction les signaux (les cornes de brume) étaient émis tous en même temps mais
perçus à des instants différents. Ici la situation est différente : on perçoit deux signaux au même instant
$t$, mais l'un des signaux en émis au temps $t_1$ et l'autre au temps $t_2$. 
Le signal contient comme information l'instant où il a été émis. En fait, chaque satellite émet en continu un signal et le récepteur capte en continu des signaux.


\bigskip

\begin{exemple}
Je reçois le signal d'un premier satellite, 
en position $x_1=4$, signal émis au temps $t_1=0$.
Au même instant je reçois le signal d'un second satellite, 
en position $x_2=2$, signal émis au temps $t_2=2$.
Les signaux se déplacent à une vitesse $c=\frac14$. 
Où suis-je ?

\bigskip

Le problème algébrique est de trouver un couple $(x,t)$ qui vérifie les deux équations,
$$|x-x_1| = c |t-t_1| \qquad \text{ et } \qquad  |x-x_2| = c |t-t_2|$$

Cherchons graphiquement la solution. On trace 
le cône de sommet $(x_1,t_1)$ et de pente $\frac1c=4$
et le cône de sommet $(x_2,t_2)$ et de même pente.
Ces deux cônes s'intersectent en deux points, l'un est exclu car il correspond
à une solution <<dans le passé>>. Par lecture graphique
on trouve notre position $x \simeq \num{2.7}$ et l'heure de réception des messages $t\simeq 5$

\myfigure{1}{  
  \tikzinput{fig_gps_temps_08}
} 

Par calcul algébrique on cherche l'intersection des droites d'équations
$+(x-2)=\frac14(t-2)$ et $-(x-4)=\frac14 t$. On trouve $(x,t) = (\frac{11}{4},5)$,
donc $x=\num{2.75}$ et $t=5$.


\end{exemple}




%---------------------------------------------------------------
\subsection{Vision géométrique (dimension 2 et 3)}
Si maintenant on passe en dimension $2$, alors le signal reçu par un satellite correspond à l'équation
$$(x-x_i)^2+(y-y_i)^2 = c^2 (t-t_i)^2$$
où $(x_i,y_i)$ est la position du signal,
et $t_i$ l'instant où le signal est émis ;
$(x,y)$ est notre position et $t$ l'instant où le signal est reçu.
Géométriquement le signal voyage le long d'un cône (en dimension $3$ cette fois : $2$ dimension pour $x$ et $y$,
$1$ dimension pour $t$).
Ce cône correspond à un cercle qui s'agrandit au fil du temps.


\myfigure{0.4}{  
  \tikzinput{fig_gps_3d_01}
  \tikzinput{fig_gps_3d_02}  
}


Deux cônes ne suffisent pas à déterminer notre position. 
Il faut $3$ satellites, et donc $3$ cônes, pour déterminer à la fois $(x,y)$ et $t$.

\bigskip

Il est difficile de se représenter la situation en dimension $3$, qui est la dimension d'usage :
il faudrait \og voir \fg{} avec $4$ dimensions (pour $x,y,z$ et $t$). 
Le signal part du satellite, et à chaque instant 
atteint tous les points d'une sphère, dont le rayon grandit au cours du temps.
Les inconnues sont $(x,y,z)$ et $t$. 
Il faut $4$ satellites pour déterminer à la fois $(x,y,z)$ et $t$.




%---------------------------------------------------------------
\subsection{Solution vectorielle}


Nous venons de voir qu'une solution géométrique n'a rien d'évident ! C'est là où le
calcul vectoriel entre en jeu. Nous allons reprendre la solution vectorielle détaillée auparavant 
en dimension $2$ (les inconnues étaient $x$ et $y$) et voir que la démarche fonctionne avec $4$ inconnues
$(x,y,z,t)$.

Nous voulons déterminer notre position dans l'espace et le temps
à partir de $4$ signaux reçus  :
\begin{itemize}
  \item notre position est $(x,y,z,t)$,
  \item les satellites sont situés en $(x_i,y_i,z_i)$,
  \item chaque signal que nous recevons a été émis à l'instant $t_i$,
  \item $i=1,2,3,4$.
\end{itemize}

Chaque signal correspond à l'équation  :
$$(x-x_i)^2 + (y-y_i)^2 + (z-z_i)^2 = c^2(t-t_i)^2$$
que l'on développe  :
$$2xx_i + 2yy_i + 2zz_i -2c^2 tt_i 
= x^2+y^2+z^2 - c^2t^2 + x_i^2 + y_i^2 + z_i^2 - c^2t_i^2$$
On pose alors
$$
A = \begin{pmatrix}
2x_1&2y_1&2z_1&-2c^2t_1\\
2x_2&2y_2&2z_2&-2c^2t_2\\
2x_3&2y_3&2z_3&-2c^2t_3\\
2x_4&2y_4&2z_4&-2c^2t_4
\end{pmatrix}\quad
X = \begin{pmatrix}x\\y\\z\\t\end{pmatrix}\quad
U = \begin{pmatrix}1\\1\\1\\1\end{pmatrix} \quad
V = \begin{pmatrix}
x_1^2 + y_1^2 +z_1^2 - c^2t_1^2\\
x_2^2 + y_2^2 +z_2^2 - c^2t_2^2\\
x_3^2 + y_3^2 +z_3^2 - c^2t_3^2\\
x_4^2 + y_4^2 +z_4^2 - c^2t_4^2\\
\end{pmatrix}\quad
B = (x^2+y^2+z^2 -c^2t^2) U + V$$


Nos équations s'écrivent simplement
\begin{equation}
\label{eq:gps4dmat}
AX=B
\end{equation}



Si $A$ est inversible et si on connaît $B$, alors $X = A^{-1} B$
où $X = \left(\begin{smallmatrix}x\\y\\z\\t\end{smallmatrix}\right)$ est la position à déterminer
dans l'espace-temps.
Le problème est toujours le même, $B$ dépend de $(x,y,z,t)$,
où plus précisément de $d=x^2+y^2+z^2-c^2t^2$, 
On calcule comme auparavant que : 
$$ d = \langle  X \mid X \rangle_\ell = \langle  A^{-1} B \mid A^{-1} B \rangle_\ell
= d^2 \|  A^{-1}  U  \|_\ell^2 + 2d\langle  A^{-1}  U \mid A^{-1} V \rangle_\ell
 +\| A^{-1} V \|_\ell^2 $$
Ainsi, $d$ vérifie l'équation du second degré  :
\begin{equation}
\label{eq :gpsseconddegrebis}
\|  A^{-1}  U  \|_\ell^2 d^2 
+ \left(2\langle  A^{-1}  U \mid A^{-1} V \rangle_\ell - 1\right) d  +\| A^{-1} V \|_\ell^2 = 0  
\end{equation}
Attention ! Ici le produit scalaire est \og tordu \fg{} : $\langle  X \mid X' \rangle_\ell
= xx'+yy'+zz' -c^2tt'$ et la norme au carré $\|X\|_\ell^2 = \langle  X \mid X \rangle_\ell
= x^2+y^2+z^2-c^2t^2$ n'est plus nécessairement positive.


Il y a au plus deux solutions, notées $d_1$ et $d_2$,
donc deux seconds membres possibles pour
le second membre de (\ref{eq:gps4dmat})  :
$$B_1 = d_1U+V \quad \text{ ou } \quad B_2 = d_2U+V.$$
Ainsi on résout deux systèmes linéaires
$AX=B_1$ et $AX=B_2$, ce qui donne 
deux solutions $X = A^{-1}B_1$ et $X = A^{-1}B_2$.
On obtient deux solutions possibles pour notre position 
$X=\left(\begin{smallmatrix}x\\y\\z\\t\end{smallmatrix}\right)$.
Pour choisir quelle solution est la bonne, on peut exclure une solution
qui serait non réaliste, ou refaire les calculs avec un cinquième satellite,
ou recommencer les mesures plus tard.

% \begin{exemple} 
% 
% Où se trouve-t-on dans la situation 
% $S_1 = (1,0,0)$,  $t_1=-\sqrt{13}$ ;
% $S_2 = (0,1,0)$,  $t_2=-\sqrt{11}$ ;
% $S_3 = (0,0,1)$,  $t_3=-3$ ;
% $S_4 = (0,0,0)$,  $t_4=-\sqrt{14}$ 
% et $c=1$ ?
% La matrice $A$ et son inverse sont :
% $$A = \begin{pmatrix}
% % 0  & 0  & 2 & -2 \\
% % 0  & 2  & 0 & 0 \\  
% % 6  & -4 & 2 & -6 \\
% % -2 & 0  & 2 & -4 \\
% \end{pmatrix}$$
% 
%   
% \end{exemple}
% 
% [[rajouter un exemple]]


%%%%%%%%%%%%%%%%%%%%%%%%%%%%%%%%%%%%%%%%%%%%%%%%%%%%%%%%%%%%%%%%
\section{Vitesse}

%---------------------------------------------------------------
\subsection{Formule par dérivation}

L'idée naturelle est de déterminer la vitesse 
par un calcul à partir des positions.
On note $P$ le point où l'on se trouve, on a vu comment déterminer
les coordonnées $(x,y,z)$ de $P$ à l'aide des satellites.
La position $P$ varie,
on note $P(t) = \big(x(t),y(t),z(t)\big)$ les coordonnées de $P$ en fonction du temps.
Comment calculer la vitesse ? Si $P$ décrit un trajet rectiligne
de longueur,  disons \SI{5}{\kilo\meter}, en un temps d'une heure alors la vitesse moyenne est de 
$$v = \frac{\text{distance}}{\text{durée}} = \SI{5}{\kilo\meter\per\hour}.$$
Plus généralement si on va du point $P_1$ au point $P_2$ entre les instants $t_1$ et $t_2$ alors
la vitesse moyenne et le vecteur vitesse moyenne sont :
$$v = \frac{\Delta P}{\Delta t} = \frac{P_1P_2}{t_2-t_1} \qquad \text{ et } \qquad
\vect v = \frac{\Delta \vect{P}}{\Delta t} = \frac{\vect{P_1P_2}}{t_2-t_1}$$
En fait $v$ est la norme $\| \vect v\|$ (si $t_2>t_1$).

\myfigure{1.2}{  
  \tikzinput{fig_gps_vitesse_01}\qquad
  \tikzinput{fig_gps_vitesse_02} 
} 

Pour obtenir la vitesse instantanée il faut réduire l'intervalle de temps : 
passer d'une grand intervalle de temps $\Delta t$, à un petit intervalle 
$\dd t$.
Physiquement, la vitesse (instantanée) se calcule par 
$$\vect v = \frac{\dd \vect P}{\dd t} = \left(\frac{\dd x}{\dd t},\frac{\dd y}{\dd t},\frac{\dd z}{\dd t} \right)$$
Mathématiquement, le $\dd t$ est infinitésimal, et diviser par $\dd t$ correspond 
à la dérivation :
$$\vect v = P'(t) = \left( x'(t), y'(t), z'(t) \right)$$
On calcule comme auparavant $v = \| \vect v\|$.

Dans la pratique que l'on soit à pied, en bateau ou en voiture on avance
quasiment en ligne droite au moins sur des périodes de quelques secondes.
Ce qui est largement suffisant pour approcher la vitesse instantanée par
la vitesse moyenne sur ce court laps de temps.
Cependant nous allons voir une méthode beaucoup plus précise.



%---------------------------------------------------------------
\subsection{Effet Doppler}

L'effet Doppler est le fameux effet de la sirène des pompiers :
lorsque la sirène des pompiers s'approche droit sur vous, le son est plus aigu, 
puis lorsque la sirène s'éloigne le son est plus grave. Ce phénomène n'est pas lié à la distance
de la sirène, mais bien à la vitesse de déplacement de cette sirène.
(Si le camion s'arrête vous entendrez le même son que vous soyez à
$10$ ou $100$ mètres.)

\myfigure{0.3}{  
  \tikzinput{fig_gps_vitesse_03}\qquad
  \tikzinput{fig_gps_vitesse_04} 
} 

\bigskip

On rappelle que la \defi{période} $T$ d'une onde est le temps qui s'écoule entre 
deux crêtes du signal. Pour simplifier l'exposé on suppose que les signaux sont
des <<bips>> émis à chaque crête, avec un intervalle de temps $T$ entre deux signaux.
La \defi{longueur d'onde} $\lambda$ est la distance entre deux crêtes, 
si le signal se déplace à une vitesse $c$ alors : $\lambda = c T$.
Enfin, la \defi{fréquence} $f$ d'une onde est l'inverse de la période $T$ : $f = \frac{1}{T}$.
La fréquence est donc le nombre de bips par seconde. 

\myfigure{1}{  
  \tikzinput{fig_gps_vitesse_05}
} 


\bigskip


La formule de Doppler relie la fréquence émise par le satellite 
(qui remplace la sirène des pompiers), et celle reçue en notre position $P$.
On note $f_S$ la fréquence d'émission du satellite (qui est une constante connue),
on note $f_P$ la fréquence du signal mesurée au point $P$.

\begin{theoreme}[Effet Doppler]
Dans le cas où $P$ est fixe, l'\defi{effet Doppler} se traduit par la formule suivante :
\mybox{$\displaystyle f_P = \frac{1}{1+\frac{v_S}{c}} f_S $}  
\end{theoreme}
On a noté :
\begin{itemize}
  \item $c$ la vitesse de déplacement du signal (la vitesse de la lumière dans le cas du signal
  émis par le satellite, la vitesse du son pour la sirène des pompiers) ;
  
  \item $v_S$ la vitesse d'éloignement de $S$ dans la direction de $P$ 
(vitesse positive si le satellite s'éloigne, vitesse négative s'il se rapproche).
\end{itemize}

\begin{exemple}
\label{ex:freqpomp}
Un camion de pompier se rapproche à un vitesse de $\SI{33}{\meter\per\second}$
(environ \SI{120}{\kilo\meter\per\hour}), sa sirène émet à la fréquence
$f_S = \SI{440}{\hertz}$. Sachant que la vitesse de déplacement du son dans l'air
est $c = \SI{333}{\meter\per\second}$, quelle est la fréquence 
perçue par un piéton ?

\bigskip

Réponse. Il s'agit juste d'appliquer la formule de l'effet Doppler, en prenant
garde que puisque le camion se rapproche la vitesse d'éloignement est négative :
$v_S = -\SI{33}{\meter\per\second}$
$$f_P = \frac{1}{1+\frac{v_S}{c}} f_S = \frac{1}{1-\frac{33}{333}} 440
= \SI{488,4}{hertz}$$
Le son perçu est plus aigu que le son émis.  
\end{exemple}


\begin{proof}
Commençons par bien comprendre la situation dans le cas où rien ne bouge.
$S$ émet un signal de fréquence $f_S$. Notons $T_S = \frac{1}{f_S}$ la période du signal


\myfigure{1}{  
  \tikzinput{fig_gps_vitesse_07}
} 

Au repos la situation est simple, au temps $t=0$,
$S$ émet son premier bip, ce signal voyage à la vitesse $c$ jusqu'à $P$.
Si $d$ est la distance entre $P$ et $S$ alors le signal arrivera au temps
$T_0=\frac{d}{c}$.
Mais entre temps, $S$ a émis un deuxième bip à l'instant $T_S$ (un bip est émis en 
$t=0$, en $t=T_S$, $t=2T_S$...).
Ce deuxième bip a la même distance à parcourir, donc il met le même temps pour arriver en $P$.
Il arrive donc au temps $T_1 = T_S + T_0$ (n'oubliez pas qu'il est parti plus tard).
Le décalage entre deux bips au départ est de $T_S$ et à l'arrivée c'est
$T_P = T_1-T_0 = T_S$. C'est bien le même décalage. Autrement dit, la fréquence reçue 
est égale à la fréquence émise.


\begin{center}
\setlength{\arrayrulewidth}{0.05mm}
%\begin{tabular}{|l|l|l|} \hline
\begin{tabular}[t]{ccc@{\vrule depth 2.5ex height 3.5ex width 0mm \ }} 
  \quad  \qquad         & \quad $S$ \quad     & \quad $P$ \quad          \\ \hline\hline
   $1^{\text{er}}$ bip  & $0$      & $T_0$         \\ \hline
   $2^{\text{nd}}$ bip  & $T_S$    & $T_1=T_0+T_S$ \\ \hline
   Décalage             & $T_S$    & $T_1-T_0 = T_S$  \\    
\end{tabular} 
\end{center}

\bigskip

Passons au cas intéressant : $S$ s'éloigne avec une vitesse $v_S$.
Le premier bip est émis par $S$ au temps $t=0$, le second au temps $t=T_S$.
Le premier bip est reçu en $P$ à un certain instant $T_0$ après avoir parcouru une distance $d$
(et alors $T_0 = \frac{d}{c}$).

\myfigure{1}{  
  \tikzinput{fig_gps_vitesse_08}
} 

Le second bip n'a pas la même distance à parcourir, en effet entre $t=0$ 
et $t=T_S$, la source s'est éloignée.
La distance à parcourir par le second bip est donc de $d' = d + v_ST_S$.
Il arrive en $P$ au temps $T_1 = T_S  + \frac{d'}{c}$ :
$$T_1 = T_S + \frac{d'}{c} 
= T_S + \frac{d + v_ST_S}{c} 
= T_S + T_0 + \frac{v_S}{c} T_S
= T_0 + T_S\left(1+\frac{v_S}{c}\right)$$
L'écart à la réception des bips est donc
$T_P = T_1-T_0 = T_S \left(1+\frac{v_S}{c}\right)$.

\begin{center}
\setlength{\arrayrulewidth}{0.05mm}
%\begin{tabular}{|l|l|l|} \hline
\begin{tabular}[t]{ccc@{\vrule depth 2.5ex height 3.5ex width 0mm \ }} 
  \quad  \qquad         & \quad $S$ \quad     & \quad $P$ \quad          \\ \hline\hline
   $1^{\text{er}}$ bip  & $0$      & $T_0$         \\ \hline
   $2^{\text{nd}}$ bip  & $T_S$    & $T_1=T_0+\left(1+\frac{v_S}{c}\right)T_S$ \\ \hline
   Décalage             & $T_S$    & $T_1-T_0 = \left(1+\frac{v_S}{c}\right)T_S$  \\    
\end{tabular} 
\end{center}

Pour les fréquences $f_P = \frac{1}{T_P}$ et $f_S = \frac{1}{T_S}$ 
on trouve la relation de Doppler :
$$f_P = \frac{1}{ 1+\frac{v_S}{c}} f_S $$
\end{proof}


\begin{proposition}
Dans le cas où $P$ se déplace avec une vitesse $v_P$ et 
$S$ avec une vitesse $v_S$, la formule de Doppler est :
$$f_P = \frac{1+\frac{v_P}{c}}{1+\frac{v_S}{c}}f_S $$  
\end{proposition}



Les deux vitesses sont comptées positivement
selon le sens $\vect{PS}$.
\myfigure{1}{  
  \tikzinput{fig_gps_vitesse_09}
} 
La preuve est similaire mais un peu plus subtile. Entre le premier et le second bip \emph{émis}, la source $S$
s'est éloignée de $v_ST_S$. 
Entre le premier et le second bip \emph{reçu}, le récepteur $P$ s'est approché
de $v_PT_P$ (donc compte pour $-v_PT_P$).
On trouve donc $d' = d + v_ST_S - v_PT_P$ et
$T_1 = T_S + \frac{d'}{c}$. 
On veut calculer $T_P = T_1-T_0$
On trouve ainsi la relation :
$$T_P = T_S  + \frac{v_S}{c} T_S - \frac{v_P}{c} T_P$$
Donc 
$$\left(1+\frac{v_P}{c}\right) T_P = \left(1+\frac{v_S}{c}\right) T_S .$$
Ce qui donne le lien entre $f_P$ et $f_S$.


%---------------------------------------------------------------
\subsection{Approximation de l'effet Doppler}

La constante $c$ est la vitesse de déplacement du signal, ce sera 
pour nous la vitesse du son (dans le cas de la sirène) ou la vitesse de la lumière
(dans le cas du signal d'un satellite).
Dans tous les cas les vitesses de déplacement $v_P$ et $v_S$ sont négligeables devant $c$,
ainsi $\frac{v_S}{c}$ et $\frac{v_P}{c}$ sont très petits.
On utilise le développement limité à l'ordre $1$, autour de $x=0$ :
$$\frac{1}{1+x} = 1 - x + o(x)$$

Ce qui donne 
$$f_P = f_S \frac{1+\frac{v_P}{c}}{1+\frac{v_S}{c}}
\simeq f_S \left(1+\frac{v_P}{c}\right)\left(1-\frac{v_S}{c}\right)
\simeq f_S \left(1+\frac{v_P}{c}-\frac{v_S}{c}\right)$$

Donc si on note $v=v_S-v_P$, la vitesse relative de $P$ par rapport à $S$, alors
l'effet Doppler approché se traduit en :
\mybox{$f_P \simeq f_S \left(1-\frac{v}{c}\right)$}


\begin{exemple}
Reprenons l'exemple \ref{ex:freqpomp} du camion de pompier
émettant un son $f_S = \SI{440}{\hertz}$ en approche à une vitesse
$v_S = -\SI{33}{\meter\per\second}$.

Alors la formule d'approximation de l'effet Doppler est :
$$f_P \simeq f_S \left(1-\frac{v}{c}\right)
= 440 \left(1-\frac{-33}{333}\right)
\simeq \SI{483,6}{hertz}$$

Avec la formule exacte on avait trouvé \SI{488,4}{hertz}.
L'approximation est assez moyenne, car la vitesse $v_S$ n'est 
pas assez petite par rapport à la vitesse du son $c$.
Lorsque $c$ sera la vitesse de la lumière, toutes les vitesses
de déplacement des objets seront négligeables.
\end{exemple}

%---------------------------------------------------------------
\subsection{Le décalage Doppler}

Voyons comment récupérer la vitesse par la mesure des fréquences.
Le signal émis et sa fréquence $f_S$ sont connus.
La fréquence mesurée $f_P$, nous permet de déduire la vitesse.
On note $D = f_S-f_P$ le \defi{décalage Doppler}.
Dans le cas où $P$ est fixe et $S$ se déplace à une vitesse $v_S$, alors
la formule de Doppler 
$f_P = \frac{1}{1+\frac{v_S}{c}} f_S$
implique 
$$v_S = \frac{c D}{f_P}$$


Dans le cas plus général où $P$ et $S$ sont en mouvement et que
$v = v_S-v_P$ est la vitesse de $S$ par rapport à $P$, alors l'approximation 
$f_P \simeq f_S \left(1-\frac{v}{c}\right)$
conduit à :
\mybox{$\displaystyle v \simeq \frac{c D}{f_S}$}

C'est donc une approximation raisonnable de dire que la 
vitesse est proportionnelle au décalage Doppler.


%---------------------------------------------------------------
\subsection{Vitesse d'éloignement}

Il faut faire bien attention que l'effet Doppler ne prend en compte que la vitesse
d'éloignement (ou de rapprochement).
Rappelons que $P$ est le point où l'on est et $S$ la position du satellite.
 On note $\vect u$ un vecteur unité de $P$ vers $S$ : 
 $\vect u = \frac{\vect{PS}}{\|\vect{PS}\|}$.
Supposons pour l'instant que $P$ soit fixe 
et $S$ se déplace dans n'importe quelle direction, son vecteur vitesse 
étant noté $\vect v$.

\begin{definition}
La \defi{vitesse d'éloignement} est la composante de la vitesse 
le long de droite $(PS)$ reliant notre position au satellite.
Mathématiquement $\vect w$ est la projection de $\vect v$ sur la droite 
$\Rr \vect u$, c'est-à-dire :
$$\vect w = \langle \vect v | \vect u \rangle \vect u = \| \vect v \| \cos \theta \cdot \vect u$$
où $\theta$ est l'angle formé par $\vect u$ et $\vect v$.  
\end{definition}

\myfigure{1}{  
  \tikzinput{fig_gps_vitesse_10}
} 

On aura besoin de la mesure de la vitesse :
$$w = \langle \vect v | \vect u \rangle = \| \vect v \| \cos \theta $$
Attention ! $w$ a un signe : positif si le satellite s'éloigne, 
négatif s'il se rapproche.
Si $P$ et $S$ se déplacent tous les deux, ayant des vecteurs vitesses
$\vect{v_P}$ et $\vect{v_S}$ quelconques. Notons $\vect v = \vect{v_S}-\vect{v_P}$ le vecteur vitesse relative de $S$ 
par rapport à $P$. Dans ce cas, la vitesse d'éloignement est définie par rapport à ce vecteur $\vect v$ :
$\vect w = \langle \vect v | \vect u \rangle \vect u$.
L'effet Doppler ne permet pas de déterminer $\vect v$, mais seulement $\vect w$.

\begin{exemple}
Nous sommes immobiles au point $P = (0,0)$. Le point 
$S$ se déplace sur la droite 
d'équation $x+2y-2=0$ à une vitesse constante
$\vect v = (2,-1)$.
On commence par déterminer une équation paramétrique de la droite : 
$S_t = (2t+2,-t)$. 

\myfigure{1}{  
  \tikzinput{fig_gps_vitesse_11}
} 
Le vecteur unité pointant vers $S_t$ est :
$$\vect u = \frac{\vect{PS_t}}{\|\vect{PS_t}\|}
= \frac{(2t+2,-t)}{\| (2t+2,-t) \|} = \frac{1}{\sqrt{5t^2+4t+4}} (2t+2,-t)$$

Calculons la vitesse d'éloignement :
$$\vect w = \langle \vect v | \vect u \rangle \vect u
= \bigg\langle (2,-1) | \frac{1}{\sqrt{5t^2+4t+4}} (2t+2,-t) \bigg\rangle \vect u
= \frac{5t+4}{\sqrt{5t^2+4t+4}} \vect u$$

Voici le graphe de la fonction $w = w(t) = \frac{5t+4}{\sqrt{5t^2+4t+4}}$.
\myfigure{1}{  
  \tikzinput{fig_gps_vitesse_12}
} 
Lorsque $t$ varie de $-\infty$ à $t_0$, la vitesse est négative donc
$S_t$ se rapproche ; puis de $t_0$ à $+\infty$, la vitesse d'éloignement devient positive,
$S_t$ s'éloigne. Le satellite est au plus proche de nous en $t_0 = -\frac{4}{5}$.
Le décalage Doppler $D$, qui est proportionnel à $w$, permet bien de calculer
$w$, mais pas $v$. C'est clair sur cet exemple où la vitesse de $S$ est constante,
mais pas sa vitesse d'éloignement.
\end{exemple}

%---------------------------------------------------------------
\subsection{Vitesse par l'effet Doppler (2D)}

On a vu que l'effet Doppler avec un seul satellite 
ne permet de calculer que la composante 
de la vitesse dans une seule direction. Pour obtenir complètement 
la vitesse on utilise plusieurs satellites.
Commençons par le cas d'un modèle en deux dimensions.
On note toujours $P$ notre position et $S_1$, $S_2$ les positions 
des $2$ satellites. Cette fois 
on considère que ce sont les satellites qui sont fixes, alors
que le point $P$ se déplace selon une vitesse $\vect v$ à déterminer.

Les vecteurs unités pointant vers chaque satellite sont $\vect{u_1}$, $\vect{u_2}$.
On suppose que les points $P$, $S_1$, $S_2$ ne sont pas alignés.
On connaît chacune des vitesses d'éloignement 
$w_i = \langle \vect v | \vect{u_i} \rangle$.
Mais attention le vecteur $\vect v$ n'est pas égal à $w_1 \vect{u_1} + w_2 \vect{u_2}$,
car il n'y a aucune raison que les vecteurs $\vect{u_1}$, $\vect{u_2}$ soient orthogonaux.
Notons $\alpha$ l'angle entre les vecteurs $\vect{u_1}$ et $\vect{u_2}$.
\begin{proposition}
$$\vect v = \frac{w_1 - w_2 \cos\alpha}{\sin^2\alpha} \vect{u_1} + 
 \frac{w_2 -w_1 \cos \alpha}{\sin^2\alpha} \vect{u_2}$$  
\end{proposition}

\myfigure{1}{  
  \tikzinput{fig_gps_vitesse_13}
} 

Rappelons que les vecteurs $\vect{u_1}$ et $\vect{u_2}$ sont connus par la positions
de $P$ et des satellites, donc $\cos\alpha = \langle \vect{u_1} | \vect{u_2} \rangle$ est aussi connu.
Par contre $w_1$ et $w_2$ sont déterminés par les décalages Doppler par rapport à chacun des satellites :
$w_1 \simeq c \frac{D_1}{f_{S_1}}$ et 
$w_2 \simeq c \frac{D_2}{f_{S_2}}$.

\begin{proof}
On cherche donc $x_1, x_2$ tels que $\vect v  = x_1 \vect{u_1} + x_2 \vect{u_2}$.
On a les équations :
$$
\left\{
\begin{array}{rcl}
\langle \vect v | \vect{u_1} \rangle &=& \langle x_1 \vect{u_1} + x_2 \vect{u_2} | \vect{u_1} \rangle  \\
\langle \vect v | \vect{u_2} \rangle &=& \langle x_1 \vect{u_1} + y_2 \vect{u_2} | \vect{u_2} \rangle  \\
\end{array}
\right.
\quad\text{ donc }\quad
\left\{
\begin{array}{rcl}
w_1 &=&  x_1+ x_2 \langle \vect{u_1} | \vect{u_2} \rangle  \\
w_2 &=& x_1 \langle \vect{u_1} | \vect{u_2} \rangle+ x_2  \\
\end{array}
\right.
$$
Où l'on s'est souvenu que $\langle \vect{u_i} | \vect{u_i} \rangle = \| \vect{u_i} \|^2 = 1$.
C'est un système linéaire où les données sont $w_1$, $w_2$, $\vect{u_1}$, $\vect{u_2}$, 
les inconnues sont $x_1$, $x_2$. On le réécrit sous la forme matricielle :
$$\begin{pmatrix}w_1\\w_2\end{pmatrix}
= \begin{pmatrix}1&\langle \vect{u_1} | \vect{u_2} \rangle\\ \langle \vect{u_1} | \vect{u_2} \rangle&1\end{pmatrix}
\begin{pmatrix}x_1\\x_2\end{pmatrix}$$
En notant 
$$A = \begin{pmatrix}1&\langle \vect{u_1} | \vect{u_2} \rangle\\ 
\langle \vect{u_1} | \vect{u_2} \rangle&1\end{pmatrix}
\quad \text{ on a donc } \quad 
\begin{pmatrix}x_1\\x_2\end{pmatrix} =  A^{-1} \begin{pmatrix}w_1\\w_2\end{pmatrix}$$
Et si on note $\alpha$ l'angle entre les vecteurs  $\vect{u_1}$ et $\vect{u_2}$ alors
$\langle \vect{u_1} | \vect{u_2} \rangle = \cos \alpha$, d'où :
$$A = \begin{pmatrix}1&\cos \alpha \\ \cos \alpha & 1\end{pmatrix}
\qquad\text{ et }\qquad 
A^{-1} = \frac{1}{1-\cos^2 \alpha}\begin{pmatrix}1&-\cos \alpha \\ -\cos \alpha & 1\end{pmatrix}
= \frac{1}{\sin^2 \alpha}\begin{pmatrix}1&-\cos \alpha \\ -\cos \alpha & 1\end{pmatrix} $$
On trouve donc
$$\begin{pmatrix}x_1\\x_2\end{pmatrix}
=\frac{1}{\sin^2 \alpha} 
\begin{pmatrix} w_1 - w_2 \cos\alpha  \\ w_2 -w_1 \cos \alpha \end{pmatrix}$$
Ce qui donne
$$\vect v = \frac{w_1 - w_2 \cos\alpha}{\sin^2\alpha} \vect{u_1} + 
 \frac{w_2 -w_1 \cos \alpha}{\sin^2\alpha} \vect{u_2}$$

\end{proof}


Si on souhaite calculer la norme de $\vect v = x_1 \vect{u_1} + x_2 \vect{u_2}$,
il faut encore faire attention que l'on n'est pas dans une base orthonormée :
$$ \| \vect v \|^2 = \langle \vect{v} | \vect{v} \rangle
= \langle x_1 \vect{u_1} + x_2 \vect{u_2} | x_1 \vect{u_1} + x_2 \vect{u_2} \rangle
= x_1^2 + x_2^2 + 2 x_1x_2 \langle \vect{u_1} | \vect{u_2} \rangle$$
Ce qui donne pour nous 
%$$v =  \| \vect v \|
%= \frac{1}{\sin^2\alpha}\sqrt{(w_1 - w_2 \cos\alpha)^2 + (w_2 -w_1 \cos \alpha)^2  + 2\cos\alpha(w_1 - w_2 \cos\alpha)(w_2 -w_1 \cos \alpha) }
%$$
 après développement et simplifications :
$$v =  \| \vect v \|
= \frac{1}{|\sin\alpha|}\sqrt{w_1^2 + w_2^2 -2w_1w_2\cos\alpha}
$$
 
\begin{exemple}
Voici des données :
$$P(0,0) \qquad S_1 (6,0) \qquad S_2(7,7) \qquad w_1 = 3 \qquad w_2 = \frac32$$
\`A quelle vitesse se déplace $P$ et dans quelle direction ?

\myfigure{0.8}{  
  \tikzinput{fig_gps_vitesse_14}
  \tikzinput{fig_gps_vitesse_15}
} 

Les vecteurs unités pointant vers les satellites sont
$\vect{u_1} = (1,0)$ et $\vect{u_2}=(\frac{\sqrt2}{2},\frac{\sqrt2}{2})$.
Ce qui donne $\langle \vect{u_1} | \vect{u_2} \rangle = \frac{\sqrt2}{2}$.
Si la vitesse cherchée s'écrit $\vect v  = x_1 \vect{u_1} + x_2 \vect{u_2}$,
alors
$$
\left\{
\begin{array}{rcl}
w_1 = \langle \vect v | \vect{u_1} \rangle &=& \langle x_1 \vect{u_1} + x_2 \vect{u_2} | \vect{u_1} \rangle  \\
w_2 = \langle \vect v | \vect{u_2} \rangle &=& \langle x_1 \vect{u_1} + y_2 \vect{u_2} | \vect{u_2} \rangle  \\
\end{array}
\right.
\quad\text{ donc }\quad
\begin{pmatrix}w_1\\w_2\end{pmatrix}
= \begin{pmatrix}1&\frac{\sqrt2}{2}\\ 
\frac{\sqrt2}{2}&1\end{pmatrix}
\begin{pmatrix}x_1\\x_2\end{pmatrix}
$$
Ainsi 
$$\begin{pmatrix}x_1\\x_2\end{pmatrix}
= \begin{pmatrix}1&\frac{\sqrt2}{2}\\ 
\frac{\sqrt2}{2}&1\end{pmatrix}^{-1}
\begin{pmatrix}w_1\\w_2\end{pmatrix}
= \begin{pmatrix}2&-\sqrt2\\ 
-\sqrt2&2\end{pmatrix}
\begin{pmatrix}3\\\frac32\end{pmatrix}
= \begin{pmatrix}6-\frac32\sqrt2\\3-3\sqrt2\end{pmatrix}$$
Mais attention, on vient de calculer les coordonnées 
de $\vect v$ dans la base $(\vect{u_1},\vect{u_2})$.
Dans la base canonique $\big( (1,0), (0,1) \big)$, on trouve 
$$\vect v = (6-\frac32\sqrt2) \vect{u_1} + (3-3\sqrt2) \vect{u_2}
= (6-\frac32\sqrt2) (1,0)  + (3-3\sqrt2)(\frac{\sqrt2}{2},\frac{\sqrt2}{2})  
=  3(1,0) + (-3+\frac32\sqrt2)(0,1)$$
Les coordonnées de $\vect v$ dans la base canonique sont donc
$(3,-3+2\sqrt2)$.
\end{exemple}

%---------------------------------------------------------------
\subsection{Vitesse par l'effet Doppler (3D)}

Revenons en dimension $3$, l'idée est exactement la même avec une coordonnée de plus, ce qui va juste compliquer
un peu les expressions. Par contre, on se place dans une situation plus réaliste : tout le monde bouge !
On note toujours $P$ notre position et $\vect v$ notre vitesse 
et on considère $3$ satellites de position $S_1$, $S_2$, $S_3$ et de vitesse respective 
$\vect{v_1}$, $\vect{v_2}$, $\vect{v_3}$.


Les vecteurs unités pointant de $P$ vers chaque satellite sont 
$\vect {u_1}$, $\vect {u_2}$, $\vect {u_3}$.
Supposons que les $4$ points $P, S_1, S_2, S_3$ ne soient pas dans un même plan, 
alors les vecteurs $(\vect {u_1}, \vect {u_2}, \vect {u_3})$ forment une base de $\Rr^3$. 
On cherche l'expression de $\vect v$ dans cette base :
$$\vect v = x_1 \vect{u_1}+x_2 \vect{u_2}+x_3 \vect{u_3}.$$

Cette fois le décalage Doppler avec chaque satellite calcule la vitesse relative :
$$w_i = \langle \vect{v}-\vect{v_i} | \vect{u_i} \rangle$$
Ce qui donne :
$$
\langle \vect{v} | \vect{u_i} \rangle
= w_i + \langle \vect{v_i} | \vect{u_i} \rangle$$
On remplace $\vect v$ par son expression 
$\vect v = x_1 \vect{u_1}+x_2 \vect{u_2}+x_3 \vect{u_3}$ et en 
écrivant $y_i = w_i + \langle \vect{v_i} | \vect{u_i} \rangle$, on obtient le système linéaire :
$$\left\{
\begin{array}{rcl}
x_1+ x_2 \langle \vect{u_1} | \vect{u_2}\rangle + x_3 \langle\vect{u_1} | \vect{u_3} \rangle & = & y_1\\
x_1\langle \vect{u_1} | \vect{u_2}\rangle+ x_2  + x_3 \langle\vect{u_2} | \vect{u_3} \rangle & = & y_2\\ 
x_1\langle\vect{u_1} | \vect{u_3} \rangle+ x_2 \langle \vect{u_2} | \vect{u_3}\rangle + x_3  & = & y_3\\ 
\end{array}
\right.$$
Qui s'écrit 
$$A \begin{pmatrix}x_1\\x_2\\x_3\end{pmatrix} = \begin{pmatrix}y_1\\y_2\\y_3\end{pmatrix}
\quad \text{ avec } \quad
A = 
\begin{pmatrix}
  1 & \langle \vect{u_1} | \vect{u_2} \rangle & \langle\vect{u_1} | \vect{u_3}\rangle\\ 
\langle \vect{u_1} | \vect{u_2} \rangle & 1 & \langle \vect{u_2} | \vect{u_3}\rangle \\
  \langle\vect{u_1} | \vect{u_3}\rangle & \langle \vect{u_2} | \vect{u_3}\rangle & 1 \\
\end{pmatrix}$$
La matrice $A$ s'inverse sans problème 
(son expression n'a rien de spéciale mais est un peu longuette à écrire),
on en déduirait donc
$$\begin{pmatrix}x_1\\x_2\\x_3\end{pmatrix} 
= A^{-1} \begin{pmatrix}y_1\\y_2\\y_3\end{pmatrix}$$


Revenons à notre but calculer notre vitesse. Les signaux des satellites permettent de :
\begin{itemize}
  \item calculer notre position $P$,
  \item savoir la position de chaque satellite et sa vitesse (elle est transmise dans le signal) 
  donc on connaît les $\vect{u_i}$ et les $\vect{v_i}$,
  \item mesurer le décalage des fréquences, ce qui  nous donne les vitesses relatives $w_i$.
\end{itemize}

Ainsi on connaît la matrice $A$ et les $y_i = w_i + \langle \vect{v_i} | \vect{u_i} \rangle$. Ce qui permet d'en déduire
les $x_i$ et 
$\vect v = x_1 \vect{u_1}+x_2 \vect{u_2}+x_3 \vect{u_3}$.
On peut aussi en déduire la norme de la vitesse comme précédemment
$v = \| \vect v \|$ (en faisant toujours attention que la base
$(\vect {u_1}, \vect {u_2}, \vect {u_3})$ n'est pas orthogonale).


%---------------------------------------------------------------
\subsection{Pratique et précision}

Voici des données utiles pour avoir un ordre de grandeur des calculs et de la précision.

\begin{itemize}
  \item Les satellites sont sur des orbites quasi-circulaire d'un rayon de 
  \SI{26500}{\kilo\meter} (donc à \SI{20000}{\kilo\meter} d'altitude).
  La période de rotation est d'environ \SI{12}{\hour}.
  En calculant la circonférence on trouve que la vitesse du satellite est d'environ
  \SI{13868}{\kilo\meter\per\hour}, soit environ \SI{3850}{\meter\per\second}.
  En fait cette vitesse varie au cours du temps, mais est connue avec une grande précision 
  car l'éphéméride du satellite est transmise dans le signal. Cela permet de calculer
  précisément la position et la vitesse du satellite.
 
  
  \item Pour nous $c$ sera la vitesse de la lumière :
  $c = \SI{300000}{\kilo\meter\per\second}$.
  
  \item Notre vitesse $v_P$ est située entre $0$ et \SI{1000}{\kilo\meter\per\hour}
  (donc de $0$ à \SI{280}{\meter\per\second}).
  
  \item Chaque satellite émet en fait deux signaux simultanément. On considère celui
  d'une fréquence de $f_S = \SI{1575.42}{\mega\hertz}$.
  
\end{itemize}


\begin{exemple}
Le récepteur GPS d'un avion enregistre un signal de fréquence
$f_P = \SI{1575.40}{\mega\hertz}$ pour un satellite dont
la vitesse d'éloignement est $v_S = \SI{4000.00}{\meter\per\second}$.
À quelle vitesse se déplace l'avion ?

Le décalage Doppler est $D=f_S-f_P
= \num{1575.4200} - \num{1575.4000} = \SI{0.0200}{\mega\hertz} = \SI{20 000}{\hertz}$.

Par la formule reliant le décalage Doppler à la vitesse relative
$v$ entre $P$ et $S$ on a :
$$v \simeq \frac{c D}{f_S} = \frac{\num{300 000 000} \cdot \num{20000}}{\num{1 575 420 000}}
\simeq \SI{3808.51}{\meter\per\second}$$

On sait que le satellite s'éloigne de nous avec la vitesse $v_S$,
donc notre vitesse $v_P$ est
$$v_P = v_S - v = \num{4000.00}-\num{3808.51} = \SI{191.49}{\meter\per\second}
\simeq \SI{690}{\kilo\meter\per\hour}$$


Bien sûr, nous avons seulement calculer notre vitesse 
de déplacement le long de l'axe $(PS)$. Pour calculer le vecteur vitesse
il nous faudrait $3$ satellites.
\end{exemple}





%---------------------------------------------------------------
\subsection{Boussole}

Comment votre GPS peut-il savoir où est le Nord, 
alors qu'il ne contient pas de boussole ?
Tout simplement en connaissant votre vecteur vitesse !

Pour bien comprendre les explications, il vaut mieux d'abord avoir lu la section suivante
sur les différentes façons de se repérer dans l'espace.
Le GPS nous fournit notre position $P$ et, on vient de le voir, notre vitesse $\vect v$.
Ce point et ce vecteur s'expriment dans un repère $(O,\vect i,\vect j,\vect k)$
dont l'origine est $O$ le centre de la Terre ; $\vect k$ est porté par l'axe de rotation.


\begin{proposition}
Le vecteur 
$$\vect n = R^2 \vect{k} - z_0 \vect{OP}$$
est un vecteur pointant vers le Nord depuis notre position
$P = (x_0,y_0,z_0)$, où $R$ est le rayon de la Terre.

Le cap suivi, c'est-à-dire l'angle entre le Nord et notre direction, est 
l'angle entre $\vect n$ et notre vecteur vitesse $\vect v$.
\end{proposition}

 

On retrouve alors le cap $\alpha$ (ou plus exactement $\cos \alpha$ et $\sin \alpha$) 
à l'aide du produit scalaire et du produit vectoriel :
$$\cos \alpha  = \frac{\langle \vect n | \vect v\rangle}{\| \vect n \| \cdot \| \vect v \|}
\qquad \text{ et } \qquad
\sin \alpha \cdot \frac{\vect{OP}}{\|\vect{OP} \|} = \frac{ \vect n \wedge \vect v}{\| \vect n \| \cdot \| \vect v \|}
$$

Comme on connaît le vecteur $\vect v$ (c'est notre direction) on en déduit la direction du Nord.
Remarquez que si nous restons immobile, alors notre vitesse est nulle et il n'est pas possible
d'en déduire la direction du Nord, contrairement à une boussole !




\myfigure{1}{  
  \tikzinput{fig_gps_vitesse_16} \qquad
  \tikzinput{fig_gps_vitesse_17}
}
\begin{proof}
Au point $P$, situé à la surface de la Terre, on se place dans le plan tangent 
à la sphère. On suppose %(pour simplifier les calculs) 
que l'on se déplace sans changer d'altitude,
c'est-à-dire que le vecteur $\vect v$ est dans ce plan.
Et c'est aussi dans ce plan que l'on cherche à s'orienter.



Un vecteur $\vect n$ porté par l'axe Nord-Sud vérifie les conditions suivantes :
\begin{enumerate}
  \item Ce vecteur $\vect n$ est dans le plan tangent, 
  c'est-à-dire orthogonal au vecteur $\vect{OP}$.
  
  \item Ce vecteur est dans le plan défini par $O$, $P$ et $\vect k$.
  C'est-à-dire le plan méridien correspondant à la longitude de $P$.
\end{enumerate}

Mathématiquement cela donne :
\begin{enumerate}
  \item $\langle \vect n | \vect{OP} \rangle = 0$
  \item $n = \alpha \vect k + \beta\vect{OP}  $ pour des $\alpha,\beta \in \Rr$.
\end{enumerate}


En substituant l'expression pour $\vect n$ dans l'égalité $\langle \vect n | \vect{OP} \rangle = 0$,
on obtient
\begin{eqnarray*}
&&          \langle \vect n | \vect{OP} \rangle = 0   \\
&\iff& \langle \alpha \vect k + \beta\vect{OP} | \vect{OP} \rangle = 0 \\
&\iff&  \alpha \langle \vect k | \vect{OP} \rangle
+ \beta \langle \vect{OP} | \vect{OP} \rangle = 0 \\
&\iff&  \alpha\langle (0,0,1) | (x_0,y_0,z_0) \rangle  + \beta \|\vect{OP} \|^2 = 0 
\quad \text{ où } \vect k = (0,0,1) \text{ et } \vect{OP} = (x_0,y_0,z_0)\\
&\iff&  \alpha z_0 + \beta R^2  = 0 
\quad \text{ où } R \text{ est le rayon de la Terre} \\
\end{eqnarray*}
Une solution est donnée par $\alpha = R^2$ et $\beta = -z_0$, ce qui correspond au vecteur
$$\vect n = R^2 \vect{k} - z_0 \vect{OP}$$

Nous avons tout fait pour que ce vecteur soit dans le plan tangent et soit sur l'axe Nord-Sud.
En plus il pointe vraiment vers le Nord (et pas le Sud) car
$\langle \vect n | \vect k \rangle = R^2 > 0$.
\end{proof}

 



%%%%%%%%%%%%%%%%%%%%%%%%%%%%%%%%%%%%%%%%%%%%%%%%%%%%%%%%%%%%%%%%
\section{Systèmes de coordonnées}

%---------------------------------------------------------------
\subsection{Coordonnées cartésiennes et polaires du plan}

Dans le plan, soit $(O,\vect{i},\vect{j})$ un repère cartésien.
Pour un point $P$ quelconque du plan on peut décomposer le vecteur $\vect{OP}$
dans la base $(\vect{i},\vect{j})$, c'est-à-dire qu'il existe des réels $x$ et $y$ tels que :
$$\vect{OP} = x \vect{i}+y\vect{j}.$$

De plus de tels $x$ et $y$ sont uniques. Les \defi{coordonnées} de $P$ par rapport au
repère $(O,\vect{i},\vect{j})$ est le couple 
$\left(\begin{smallmatrix}x\\y\end{smallmatrix}\right)$, noté aussi
$(x,y)$.

\myfigure{1.3}{  
  \tikzinput{fig_gps_coord_01}
}


\bigskip

Passons aux coordonnées polaires. 
Soit $(O,\vect{i},\vect{j})$ un repère orthonormé direct
Pour $\theta$ réel, on pose 
$$\vect{u_\theta}=\cos\theta\vect{i}+\sin\theta\vect{j}.$$

Le point $P$ étant un point du plan, on dit que $[r :\theta]$ est un 
couple de \defi{coordonnées polaires} si $\vect{OP}=r\vect{u_\theta}$.

\myfigure{1}{  
  \tikzinput{fig_gps_coord_02}\qquad 
  \tikzinput{fig_gps_coord_03}  
}


\bigskip

On passe des coordonnées polaires aux coordonnées cartésiennes par les formules suivantes :
\mybox{
$\left\{\begin{array}{rcl}
x & = & r \cos \theta \\
y & = & r \sin \theta
\end{array}\right.$
}
Ces formules s'obtiennent simplement en identifiant les coordonnées du vecteur 
$\vect{OP} = x \vect{i}+y\vect{j} = r\vect{u_\theta}$.
Dans l'autre sens, on retrouve les coordonnées polaires à partir des coordonnées cartésiennes
par 
$$r = \sqrt{x^2+y^2}$$
qui représente la distance du point $P$ à l'origine $O$,
et si $r\neq 0$ alors on détermine $\theta$ (modulo $2\pi$) par
les identités :
$$\cos \theta = \frac{x}{r} \qquad\text{ et }\qquad \sin \theta = \frac{y}{r}.$$

\bigskip

\textbf{Modélisation de la Terre.}

Si on souhaite modéliser la Terre alors on fixe le rayon $r$ à $R
= \SI{6300}{\kilo\meter}$. Sur ce modèle du plan, 
un point $P$ à la surface de la Terre est alors
déterminé par un seul paramètre, l'angle $\theta$.

\myfigure{0.8}{  
  \tikzinput{fig_gps_coord_04}
}



%---------------------------------------------------------------
\subsection{Coordonnées cartésiennes et sphériques de l'espace}
\label{ssec:coord}

On fait le même travail dans l'espace. Soit $(O,\vect{i},\vect{j},\vect{k})$ un repère orthonormé direct.
Pour un point $P$ de l'espace, on décompose le vecteur $\vect{OP}$
dans la base $(\vect{i},\vect{j},\vect{k})$ :
$$\vect{OP} = x \vect{i}+y\vect{j}+z\vect{k}$$
et on appelle les \defi{coordonnées} de $P$, par rapport au
repère $(O,\vect{i},\vect{j},\vect{k})$, le triplet 
$\left(\begin{smallmatrix}x\\y\\z\end{smallmatrix}\right)$, ou encore $(x,y,z)$.






Les coordonnées sphériques sont définies par $3$ paramètres 
$[r :\varphi :\lambda]$.
On note 
$$\vect{u_{\varphi,\lambda}}= \cos \varphi \cos \lambda \vect{i} + \cos \varphi \sin \lambda\vect{j} + \sin \varphi\vect{k}$$
le vecteur qui forme un angle $\lambda$ avec l'axe $(O,\vect{i})$ et
un angle $\varphi$ avec le plan horizontal $(O,\vect{i},\vect{j})$.




Le point $P$ de l'espace a pour \defi{coordonnées sphériques} $[r :\varphi :\lambda]$ 
si $\vect{OP}=r\vect{u_{\varphi,\lambda}}$.

\bigskip

Le passage des coordonnées sphériques vers les coordonnées cartésiennes s'exprime par
\mybox{
$\left\{\begin{array}{rcl}
x & = & r \cos \varphi \cos \lambda \\
y & = & r \cos \varphi \sin \lambda \\
z & = & r \sin \varphi
\end{array}\right.$
}
Dans l'autre sens on obtient $r = \sqrt{x^2+y^2+z^2}$.

\bigskip

\textbf{Modélisation de la Terre.}
On modélise la Terre par une sphère (ou plutôt une boule) de rayon
$R = \SI{6300}{\kilo\meter}$.
Un point à la surface de la Terre (donc avec $r=R$) est déterminé
par :
\begin{itemize}
  \item l'angle $\varphi$, qui est la \defi{latitude}, c'est un angle 
entre $\ang{-90}$ et $\ang{+90}$.  Un point de l'équateur vérifie $\varphi = \ang{0}$ ;
  
  \item l'angle $\lambda$, qui est la \defi{longitude}, c'est un angle
  entre $\ang{-180}$ et $\ang{+180}$. Le méridien de Greenwich correspond à $\lambda = \ang{0}$.
\end{itemize}

\myfigure{1}{  
  \tikzinput{fig_gps_3d_07} 
}

\begin{exemple}
Le sommet de la montagne du Chimborazo en Équateur est 
le point sur Terre le plus loin du centre de la Terre.
Ces coordonnées $(x,y,z)$ en kilomètres sont :
$$x= \num{1238.0} \qquad y =  \num{-6261.0} \qquad  z = \num{-163.1}$$
Calculer la distance $r$ du sommet à l'origine ainsi que la latitude $\varphi$
et la longitude $\lambda$.


Il s'agit d'inverser les relations :
$\left\{\begin{array}{rcl}
x & = & r \cos \varphi \cos \lambda \\
y & = & r \cos \varphi \sin \lambda \\
z & = & r \sin \varphi 
\end{array}\right.$

On commence par trouver $r$ :
$$r = \sqrt{x^ 2+y^ 2+z^2} \simeq \SI{6 384.4}{\kilo\meter}$$
Puis comme $\frac{z}{r} =\sin \varphi$, alors
$$\varphi = \arcsin \left( \frac{z}{r}\right) \simeq \ang{-1.464}$$
On en déduit $\cos\varphi$, 
et de l'égalité $\frac{y}{r}  =  \cos \varphi \sin \lambda$, on obtient:
$$\lambda = \arcsin\left( \frac{1}{\cos\varphi}\frac{y}{r}\right) 
 \simeq \ang{-78.815}$$

Note : ce n'est pas le mont Everest qui est le point le plus éloigné du centre 
de la Terre, ceci à cause du phénomène d'aplatissement aux pôles que l'on verra juste après.
Le sommet de l'Everest est à distance $r= \SI{6 382.3}{\kilo\meter}$ du centre de la Terre.
\end{exemple}


%---------------------------------------------------------------
\subsection{La Terre n'est pas ronde}

La Terre n'est pas ronde ! Bien sûr il y a des montagnes et des vallées, mais ce sont 
des variations assez minimes.
Par contre la Terre est aplatie aux pôles.
Pour la modélisation dans le plan, on remplace le cercle de rayon $R$ 
par une ellipse ayant un demi-grand axe $a$ et un demi-petit axe $b$.

\myfigure{1}{  
  \tikzinput{fig_gps_coord_06}
}

On adapte nos coordonnées polaires à cette ellipse. Un point sur l'ellipse est déterminé
par un paramètre $\theta$ :
\mybox{
$\left\{\begin{array}{rcl}
x & = & a \cos \theta \\
y & = & b \sin \theta
\end{array}\right.$
}
\myfigure{1}{  
  \tikzinput{fig_gps_coord_07}
}

Bien sûr cette ellipse n'est pas très aplatie, les valeurs sont :
$$a = \SI{6378.137}{\kilo\meter} \qquad b = \SI{6356.752}{\kilo\meter}$$
On défini l'\defi{aplatissement} par 
$$f = \frac{a-b}{a}$$
L'aplatissement est nul pour un cercle et vaut $f \simeq \num{0.00335}$ pour la Terre.
Par comparaison Saturne a un aplatissement de $\num{0.1}$
($a= \SI{60268}{\kilo\meter}$ $b=\SI{54364}{\kilo\meter}$) ; le Soleil lui a un aplatissement quasiment nul.


\myfigure{1}{  
  \tikzinput{fig_gps_coord_08} 
}

Attention $\theta$ n'est pas l'angle que l'on croit. Ce n'est pas l'angle entre l'horizontale
et le vecteur $\vect{OP}$, mais l'angle entre l'horizontale et le vecteur 
$\vect{OP_1}$, où le point $P_1$ est le point du cercle principal, de même abscisse que $P$
(et avec l'ordonnée de même signe que celle de $P$).


La Terre en $3$ dimensions se modélise en un ellipsoïde obtenu 
par la rotation d'une ellipse autour de l'axe reliant le pôle Nord au pôle Sud.
L'équateur est donc un cercle de rayon $a$ alors que le méridien de Greenwich
est une ellipse de demi-axes $a$ et $b$.

\myfigure{1.3}{  
  \tikzinput{fig_gps_3d_04}
}

%---------------------------------------------------------------
\subsection{Changement de repère}

Si l'on a deux repères cartésiens du plan ou de l'espace, alors on peut passer
de l'un à l'autre par un changement de repère. 
Fixons deux repères $\mathcal{R}= (O,\vect{i},\vect{j})$ et 
$\mathcal{R}'= (O',\vect{i'},\vect{j'})$ du plan. Fixons aussi un point $P$ du plan
qui a pour coordonnées $X = \left(\begin{smallmatrix}x\\y\end{smallmatrix}\right)$
dans le repère $\mathcal{R}$, et a pour coordonnées
$X' = \left(\begin{smallmatrix}x'\\y'\end{smallmatrix}\right)$
dans le repère $\mathcal{R}'$.

Alors il existe une matrice inversible $A$ telle que :
$$X = A X' + \vect{OO'}$$
où $\vect{OO'}$ est le vecteur joignant les deux origines des repères
(dans cette formule $\vect{OO'}$ est considéré comme un vecteur colonne).

Sur la figure suivante les deux repères sont orthonormés directs.
\myfigure{1}{  
  \tikzinput{fig_gps_coord_09}
}

Par exemple $A$ peut être :
\begin{itemize}
  \item la matrice d'une rotation d'angle $\theta$ (et de centre $O$) :
$\begin{pmatrix}\cos\theta&-\sin\theta\\\sin\theta&\cos\theta\end{pmatrix}$ 
(c'est le cas si les deux repères sont orthonormés directs) ;
  \item la matrice d'une homothétie de rapport $k$ (et de centre $O$) :
$\begin{pmatrix}k&0\\0&k\end{pmatrix}$ ;
  \item la matrice d'une symétrie par rapport à l'axe des
  abscisses $(O\vect{i})$ : $\begin{pmatrix}1&0\\0&-1\end{pmatrix}$ ;
  ou à l'axe des ordonnées $(O\vect{j})$ : $\begin{pmatrix}-1&0\\0&1\end{pmatrix}$ ;
  \item ou le produit de plusieurs de ces matrices ;
  \item ou plus généralement une matrice $\begin{pmatrix}a&b\\c&d\end{pmatrix}$
  avec $ad-bc\neq 0$.
\end{itemize}


\bigskip

Pour un point $P$ de l'espace dont les coordonnées sont 
$X= \left(\begin{smallmatrix}x\\y\\z\end{smallmatrix}\right)$
dans un repère $\mathcal{R} = (O,\vect{i},\vect{j},\vect{k})$ et 
de coordonnées $X' = \left(\begin{smallmatrix}x'\\y'\\z'\end{smallmatrix}\right)$
dans le repère $\mathcal{R}'= (O',\vect{i'},\vect{j'},\vect{k'})$. 
Alors il existe une matrice inversible $A$ telle que :
$$X = A X' + \vect{OO'}.$$

La matrice $A$ peut être :
\begin{itemize}
  \item la matrice d'une rotation d'angle $\lambda$ d'axe $(O\vect{k})$ :
$\begin{pmatrix}
\cos \lambda & -\sin \lambda & 0 \\
\sin \lambda & \sin \lambda  & 0 \\
0            & 0             & 1 \\
\end{pmatrix}$ ;
  \item la matrice d'une rotation d'angle $\varphi$ d'axe $(O\vect{j})$ :
$\begin{pmatrix}
\cos \varphi & 0 & \sin \varphi \\
0            & 1 & 0 \\
-\sin \varphi & 0 & \sin \varphi \\
\end{pmatrix}$ ;
  \item la matrice d'une homothétie de rapport $k$  (et de centre $O$) :
$\begin{pmatrix}k&0&0\\0&k&0\\0&0&k\end{pmatrix}$ ;
  \item la matrice d'une symétrie par rapport au plan $(O,\vect{i},\vect{j})$ :
  $\begin{pmatrix}1&0&0\\0&1&0\\0&0&-1\end{pmatrix}$ ; ...
  \item ou le produit de plusieurs de ces matrices ;
  \item ou plus généralement une matrice 
  $\begin{pmatrix}
  a_{11}&a_{12}&a_{13}\\
  a_{21}&a_{22}&a_{23}\\
  a_{31}&a_{32}&a_{33}\\
  \end{pmatrix}$ de déterminant non nul.
\end{itemize}

\bigskip

Attention, ces formules de changement de repères ne concernent que 
le passage entre deux repères cartésiens, et pas le passage des coordonnées 
cartésiennes aux coordonnées polaires. 


%---------------------------------------------------------------
\subsection{Coordonnées locales}

Pour quelqu'un situé à un point $P$ de la surface de la Terre, il est utile
de définir un repère dont l'origine est le point $P$. On choisit un repère
$\mathcal{R}_{\text{loc}} = (P,\vect{h},\vect{e},\vect{n})$ (pour \emph{haut}, \emph{est}, \emph{nord}).
Le premier vecteur est perpendiculaire (et sortant) à la sphère.
Les deux autres vecteurs sont tangents à la sphère, 
le second vecteur pointe vers l'Est, le troisième vers le Nord.



\myfigure{1.5}{  
  \tikzinput{fig_gps_3d_05} 
}

Soit le point $P$ de coordonnées sphériques $[r :\varphi :\lambda]$.
On passe du repère $\mathcal{R}_O = (O,\vect{i},\vect{j},\vect{k})$ au repère
$\mathcal{R}_{\text{loc}} = (P,\vect{h},\vect{e},\vect{n})$ par 
la séquence de transformations suivantes (dans cet ordre) :
\begin{itemize}
  \item la rotation d'axe $(Oy)$ et d'angle $\varphi$ ;
  \item la rotation d'axe $(Oz)$ et d'angle $\lambda$ ;
  \item la translation de vecteur $\vect{OP}$.
\end{itemize}

Voir la figure du paragraphe \ref{ssec:coord}.


Les matrices des rotations sont respectivement :
$$R_\varphi = 
\begin{pmatrix}
\cos \varphi & 0 & \sin \varphi \\
0            & 1 & 0 \\
-\sin \varphi & 0 & \cos \varphi \\
\end{pmatrix}
\qquad R_\lambda = 
\begin{pmatrix}
\cos \lambda & -\sin \lambda & 0 \\
\sin \lambda & \cos \lambda  & 0 \\
0            & 0             & 1 \\
\end{pmatrix}$$
et la translation a pour vecteur 
$$\vect{OP} = 
\begin{pmatrix}
r \cos \varphi \cos \lambda \\
r \cos \varphi \sin \lambda \\
r \sin \varphi  
\end{pmatrix}$$

Pour un point $M$ de l'espace, on note
$X_O$ les coordonnées de $M$ dans le repère $\mathcal{R}_O$
et $X_{\text{loc}}$ les coordonnées de ce même point $M$ 
mais cette fois dans le repère $\mathcal{R}_{\text{loc}}$.
Alors on passe de l'un à l'autre des systèmes de coordonnées par :
$$X_O =  \big(R_\lambda R_\varphi\big) X_{\text{loc}} + \vect{OP}$$
où $R_\lambda R_\varphi$ est le produit des deux matrices.

\begin{exemple}
Nous sommes au point $P$ de latitude $\varphi = \ang{45}$, de longitude 
$\lambda = \ang{60}$ à la surface de la Terre (on prend $r= \SI{6300}{\kilo\meter}$).
Vous apercevez un avion $M$ situé à $\SI{20}{\kilo\meter}$ au Sud-Est et à 
$\SI{5}{\kilo\meter}$ d'altitude. Calculer les coordonnées $(x_M,y_M,z_M)$ du point $M$.

\begin{enumerate}
  \item Nous aurons besoin des coordonnées $(x_P,y_P,z_P)$ du point $P$ données
par les formules :
$\left\{\begin{array}{rcl}
x_P & = & r \cos \varphi \cos \lambda \simeq \num{2227.4}\\
y_P & = & r \cos \varphi \sin \lambda \simeq \num{3857.9}\\
z_P & = & r \sin \varphi \simeq \num{4454.8}
\end{array}\right.$

Ce qui donne bien sûr les coordonnées du vecteur 
$\vect{OP}=\left(\begin{smallmatrix}x_P\\y_P\\z_P\end{smallmatrix}\right)$.
  
  \item Nous devons trouver les coordonnées locales $X_{\text{loc}}$ de $M$ dans le repère
  $\mathcal{R}_{\text{loc}}$ : $\SI{20}{\kilo\meter}$ au Sud-Est et $\SI{5}{\kilo\meter}$ de hauteur
  correspond à $5 \vect{h} + \frac{\sqrt{2}}{2} 20 \vect{e}-\frac{\sqrt{2}}{2} 20 \vect{n}$, donc les coordonnées de $M$ sont 
  $$X_{\text{loc}} = \begin{pmatrix} 5\\10\sqrt{2}\\-10\sqrt{2}\end{pmatrix}$$
  
  
  \item Les matrices de rotations sont ici :
 $$R_\varphi = 
\begin{pmatrix}
\cos \varphi & 0 & \sin \varphi \\
0            & 1 & 0 \\
-\sin \varphi & 0 & \cos \varphi \\
\end{pmatrix}
=
\begin{pmatrix}
\frac{\sqrt2}{2} & 0 & \frac{\sqrt2}{2} \\
0                & 1 & 0 \\
-\frac{\sqrt2}{2} & 0 & \frac{\sqrt2}{2} \\
\end{pmatrix}
\qquad R_\lambda = 
\begin{pmatrix}
\cos \lambda & -\sin \lambda & 0 \\
\sin \lambda & \cos \lambda  & 0 \\
0            & 0             & 1 \\
\end{pmatrix} = 
\begin{pmatrix}
\frac12          & -\frac{\sqrt3}{2} & 0 \\
\frac{\sqrt3}{2} & \frac12           & 0 \\
0                & 0                 & 1 \\
\end{pmatrix}$$ 
  
  
  \item On applique la formule de changement de base pour obtenir les coordonnées
  $X_O =\left(\begin{smallmatrix}x_M\\y_M\\z_M\end{smallmatrix}\right)$ 
  de $M$ dans $\mathcal{R}_O$ :
  $$X_O =\big(R_\lambda R_\varphi\big) X_{\text{loc}} + \vect{OP} 
  \simeq \begin{pmatrix}\num{2211.4}\\\num{3859.4}\\\num{4441.2}\end{pmatrix}$$
\end{enumerate}
\end{exemple}


%---------------------------------------------------------------
\subsection{Différentes latitudes}

Comme la Terre n'est pas ronde, les choses se compliquent.
Nous allons devoir définir trois latitudes ! On modélise la Terre 
par un ellipsoïde. Nous ferons les calculs dans le plan pour une ellipse, 
les formules pour l'ellipsoïde s'en déduiront par rotation.
On modélise donc la surface de la Terre par une ellipse $\mathcal{E}$, d'équation :
$$\frac{x^2}{a^2}+\frac{y^2}{b^2}=1.$$
Le demi-grand axe est de longueur $a$, le demi-petit axe est de longueur $b$.
On trace aussi son cercle principal $\mathcal{C}$ de rayon $a$.
Soit $P_0=(x_0,y_0)$ un point de l'ellipse $\mathcal{E}$. 
Soit $P_1=(x_1,y_1)$ le point de même abscisse que $P_0$ 
mais situé sur le cercle $\mathcal{C}$.

\myfigure{1.3}{  
  \tikzinput{fig_gps_coord_10}
}

\begin{itemize}
  \item La \defi{latitude géocentrique} $\bar \varphi$ est l'angle entre l'axe des abscisses
  et la droite $(OP_0)$.
  
  \item La \defi{latitude géographique} $\varphi$ est l'angle entre 
  l'axe des abscisses et la verticale en $P_0$. La verticale est la droite verticale du lieu :
  c'est la droite orthogonale à l'ellipse en $P_0$ (c'est-à-dire perpendiculaire à la tangente en $P_0$ à l'ellipse).   La verticale en $P_0$ ne passe (en général) pas par l'origine $O$.
  
  \item La \defi{latitude réduite} $\theta$ est l'angle entre l'axe des abscisses
  et la droite $(OP_1)$.
\end{itemize}


Ce que l'observateur mesure en $P_0$ c'est la latitude géographique. 
C'est donc $\varphi$ l'angle le plus important avec lequel on souhaite se repérer.
Pour l'instant nous allons trouver la relation entre ces trois angles.

\begin{proposition}
\label{prop :gpstangente}
$$
\tan \bar\varphi = \frac{y_0}{x_0} \qquad\qquad 
\tan \theta = \frac{a}{b} \tan \bar\varphi \qquad\qquad  
\tan \varphi = \frac{a^2}{b^2} \tan \bar\varphi
$$  
\end{proposition}

\begin{proof}
\begin{enumerate}
  \item Tout d'abord l'ellipse est l'image du cercle par la transformation affine 
  $(x,y) \mapsto (x,\frac{b}{a}y)$. Ainsi
  $(x_0,y_0) = (x_1,\frac{b}{a}y_1)$. En particulier 
  $\frac{y_1}{x_1}= \frac{a}{b}\frac{y_0}{x_0}$.


  \item Ensuite si on note $r$ la distance $OP_0$ alors
  $x_0 = r \cos \bar\varphi$ et $y_0 = r \sin \bar\varphi$ et ainsi :
  $$\tan \bar\varphi 
  = \frac{\sin \bar\varphi}{\cos \bar\varphi} 
  = \frac{y_0}{x_0}
  .$$
  
  \item Comme $x_1 = a \cos \theta$, $y_1 = a \sin \theta$ alors
  $$\tan \theta 
  = \frac{y_1}{x_1} 
  = \frac{a}{b}\frac{y_0}{x_0}
  = \frac{a}{b} \tan \bar\varphi.$$
  
  \item Obtenir une relation avec la latitude géographique $\varphi$ demande
  de connaître la verticale du lieu. Par définition de l'angle $\varphi$, 
  un vecteur (unitaire) vertical en $P_0$ est 
  $$ \vect{u} = \begin{pmatrix}\cos \varphi \\ \sin \varphi \end{pmatrix}.$$
  
  Nous avons besoin d'une autre façon de calculer la direction verticale.
  Pour cela on utilise le fait que si une courbe a pour équation
  $f(x,y)=0$ et passe par le point $(x_0,y_0)$ alors 
  une équation de la tangente au point $(x_0,y_0)$ est :
  $$\frac{\partial f}{\partial x}(x_0,y_0) (x-x_0) + \frac{\partial f}{\partial y}(x_0,y_0)(y-y_0) = 0.$$
  
\myfigure{1}{  
  \tikzinput{fig_gps_coord_11}
}
  
  Autrement dit le gradient 
  $$\begin{pmatrix} \frac{\partial f}{\partial x}(x_0,y_0) \\ \frac{\partial f}{\partial y}(x_0,y_0) \end{pmatrix}$$
  est orthogonal à la tangente à la courbe en ce point.

  On applique ceci à l'ellipse définie par l'équation $f(x,y)=0$ où 
  $f(x,y) = \frac{x^2}{a^2}+\frac{y^2}{b^2} - 1$.
  Comme 
  $$\frac{\partial f}{\partial x}(x_0,y_0) = \frac{2x_0}{a^2}
  \quad \text{ et } \quad \frac{\partial f}{\partial y}(x_0,y_0) = \frac{2y_0}{b^2}.$$
  Alors le vecteur gradient est 
  $$\vect v = \begin{pmatrix}\frac{2x_0}{a^2} \\ \frac{2y_0}{b^2} \end{pmatrix}.$$
  

  Ainsi le vecteur 
  $\vect u = \left(\begin{smallmatrix}\cos \varphi \\ \sin \varphi \end{smallmatrix}\right)$ et le vecteur 
  gradient $\vect v$ sont deux vecteurs colinéaires, ce qui implique que
  $$\begin{pmatrix}\cos \varphi \\ \sin \varphi \end{pmatrix} = k \begin{pmatrix}\frac{2x_0}{a^2} \\ \frac{2y_0}{b^2} \end{pmatrix}$$
  pour un certain $k\in\Rr$. En particulier 
  $$\tan \varphi = \frac{\sin\varphi}{\cos\varphi} 
  = \frac{\frac{2y_0}{b^2}}{\frac{2x_0}{a^2}} = \frac{a^2}{b^2}\frac{y_0}{x_0}= \frac{a^2}{b^2} \tan \bar\varphi.$$
\end{enumerate}

\end{proof}


%---------------------------------------------------------------
\subsection{Coordonnées géographiques}


\begin{theoreme}
Pour un point $P = (x,y,z)$ ayant une latitude géographique $\varphi$ et une altitude $h$
alors
$$\left\{ 
\begin{array}{rcl}
x &=& (N+h) \cos \varphi \\
y &=& \left(\frac{b^2}{a^2}N+h\right)\sin\varphi 
\end{array}
\right.$$
où 
$$N = \frac{a}{\sqrt{1-\left(1-\frac{b^2}{a^2}\right)\sin^2 \varphi}}.$$
\end{theoreme}

Si on note $P_0$ le point de $(OP)$ situé à la surface de la Terre, alors 
la valeur $N$ (qui dépend de la latitude $\varphi$) est la distance $P_0$, c'est-à-dire la distance entre 
$P_0$ et l'axe $(Oy)$, mesurée le long de la verticale en $P_0$.



\myfigure{1}{  
  \tikzinput{fig_gps_coord_12}
}

Si $e$ est l'excentricité de l'ellipse alors $e^2 = 1-\frac{b^2}{a^2}$, donc 
$N= \frac{a}{\sqrt{1-e^2\sin^2 \varphi}}$.
Cette valeur $N$ est aussi le rayon de courbure de l'ellipse au point $P_0$. En effet
l'ellipse étant symétrique par rapport à l'axe $(Oy)$, le centre de courbure doit appartenir à l'axe $(Oy)$.
Ainsi le rayon de courbure est bien cette longueur $N$.

Les formules en dimension $3$ pour un ellipsoïde sont similaires, elles découlent du théorème
en prenant une section de la Terre passant les pôles et le point $P$ :
\begin{corollaire}
Pour un point $P = (x,y,z)$ de latitude géographique $\varphi$, 
de longitude $\lambda$ et d'altitude $h$ alors 
$$\left\{ 
\begin{array}{rcl}
x &=& (N+h) \cos \varphi \cos\lambda \\
y &=& (N+h) \cos \varphi \sin\lambda \\
z &=& \left(\frac{b^2}{a^2}N+h\right)\sin\varphi 
\end{array}
\right.$$
où 
$\displaystyle N = \frac{a}{\sqrt{1-\left(1-\frac{b^2}{a^2}\right)\sin^2 \varphi}}.$
\end{corollaire}

\begin{exemple}
Les satellites GPS sont lancés depuis la base de Cap Canaveral en Floride.
Sa latitude est $\varphi = \ang{28.475630}$, 
sa longitude $\lambda = \ang{-80.565834}$
et se situe au niveau de la mer.
Calculer les coordonnées $(x,y,z)$ de la base, tout d'abord pour le modèle sphérique, en prenant
$R = \SI{6 371,0}{\kilo\meter}$ comme rayon moyen de la Terre.
Faire le même calcul pour le modèle ellipsoïdal avec
$a = \SI{6378.137}{\kilo\meter}$, $b = \SI{6356.752}{\kilo\meter}$.

\begin{enumerate}
  \item Il s'agit d'appliquer les formules sphériques :
$$\left\{\begin{array}{rcl}
x_1 & = & R \cos \varphi \cos \lambda \simeq \num{918.0}\\
y_1 & = & R \cos \varphi \sin \lambda \simeq \num{-5524.5}\\
z_1 & = & R \sin \varphi \simeq \num{3037.6}
\end{array}\right.$$  

  \item Pour le modèle ellipsoïdal on commence par calculer le rayon de 
  courbure $N$ (qui dépend de la latitude $\varphi$):
  $$N = \frac{a}{\sqrt{1-\left(1-\frac{b^2}{a^2}\right)\sin^2 \varphi}}\simeq \num{6383.0}$$
  Puis on applique les formules ellipsoïdales (avec ici $h=0$):
$$\left\{ 
\begin{array}{rcl}
x_2 &=& N \cos \varphi \cos\lambda  \simeq \num{919.7}\\
y_2 &=& N \cos \varphi \sin\lambda  \simeq \num{-5534.9}\\
z_2 &=& \frac{b^2}{a^2}N \sin\varphi  \simeq \num{3022.9}
\end{array}
\right.$$  
\end{enumerate}
On constate des écarts de plusieurs kilomètres entre les deux modèles.




\end{exemple}



\begin{proof}[Preuve du théorème]~

\textbf{Coordonnées de $P_0$.}  

Notons $N$ la longueur $P_0R$.
Alors dans la triangle rectangle d'hypoténuse $P_0R$ on a 
$$x_0 = N \cos \varphi.$$

D'après la proposition \ref{prop :gpstangente} on a
$$y_0 = x_0 \tan \bar\varphi = x_0 \frac{b^2}{a^2} \tan \varphi 
= N \cos \varphi\frac{b^2}{a^2} \frac{\sin\varphi}{\cos \varphi} = N\frac{b^2}{a^2}\sin\varphi.$$


\textbf{Calcul de $N$.}   

Le point $P_0 = (x_0,y_0)$ est sur l'ellipse d'équation 
$\frac{x^2}{a^2}+\frac{y^2}{b^2} = 1$.
En substituant les valeur de $x_0$ et $y_0$ obtenues juste au-dessus on obtient

\begin{align*}
         & \quad \frac{x_0^2}{a^2}+\frac{y_0^2}{b^2} = 1 \\
\implies & \quad \frac{N^2 \cos^2 \varphi}{a^2}+\frac{b^4}{a^4}\frac{N^2 \sin^2 \varphi}{b^2} = 1 \\
\implies & \quad N^2 \left( (1-\sin^2\varphi) + \frac{b^2}{a^2}\sin^2 \varphi \right) = a^2 \\
\implies & \quad N = \frac{a}{\sqrt{1-\left(1-\frac{b^2}{a^2}\right)\sin^2 \varphi}}
\end{align*}

\textbf{Coordonnées de $P$.} 

Rappelons que $\vect u$ est un vecteur unitaire dirigé suivant la verticale du lieu :
$$ \vect{u} = \begin{pmatrix}\cos \varphi \\ \sin \varphi \end{pmatrix}.$$

De l'identité vectorielle 
$$\vect{OP} = \vect{OP_0} + h \vect{u}$$
on déduit 
$$\begin{pmatrix}x\\y\end{pmatrix}
= \begin{pmatrix}x_0\\y_0\end{pmatrix} + h \begin{pmatrix}\cos \varphi \\ \sin \varphi \end{pmatrix}.$$
Ce qui donne
$$\begin{pmatrix}x\\y\end{pmatrix}
= \begin{pmatrix}(N+h) \cos \varphi \\ 
  \left(\frac{b^2}{a^2}N+h\right)\sin\varphi  
  \end{pmatrix}$$

\end{proof}




%%%%%%%%%%%%%%%%%%%%%%%%%%%%%%%%%%%%%%%%%%%%%%%%%%%%%%%%%%%%%%%%
\section{Position approchée}

%---------------------------------------------------------------
\subsection{Les erreurs}

Einstein disait : \og En théorie, la théorie et la pratique sont identiques.
En pratique, elles sont différentes !\fg{} 
En théorie, $4$ satellites permettent de déterminer la position exacte (parmi deux possibles).
En pratique, il est illusoire d'espérer une précision absolue, car 
il existe une multitude de sources possibles d'erreurs, les principales sont  :
\begin{itemize}
  \item des perturbations du signal lorsqu'il traverse l'atmosphère
  (l'ionosphère et la troposphère) ; 
  
  \item des erreurs de mesures des instruments, principalement 
  des erreurs de mesure du temps ; 
  
  \item des erreurs dues aux effets de la physique relativiste (effets qu'il est possible de modéliser) ;  
  
  \item des erreurs sur la position des satellites ;
  
  \item des erreurs dues à une mauvaise configuration des satellites.
\end{itemize}

Il existe d'autres sources d'erreurs  :
\begin{itemize}
  \item des erreurs dans les calculs numériques ;
  \item une dégradation volontaire possible du signal, car le GPS reste avant tout
  un service de l'armée des \'Etats-Unis.
\end{itemize}

Une première conséquence de toutes ces erreurs, c'est qu'il est illusoire de
chercher une solution exacte pour notre position. Aussi dans la pratique, on
se contente de chercher une solution approchée, et d'appliquer des méthodes
d'approximation qui sont beaucoup plus rapides : nous étudierons 
la méthode de linéarisation.
La deuxième conséquence est qu'il faut pallier à toutes ces erreurs pour 
obtenir une position assez précise. La façon la plus efficace est d'utiliser plus  
que les $4$ satellites indispensables. Nous verrons comment déterminer
une position optimale à partir d'un nombre quelconque de satellites.



%---------------------------------------------------------------
\subsection{Calcul par linéarisation (dimension 1)}

On commence par un modèle simple, en une dimension.
L'équation à résoudre en dimension $1$, avec un seul satellite est
$$(x-x_1)^2=r_1^2.$$
On note $f(x) = (x-x_1)^2-r_1^2$.
Il s'agit donc de trouver $x$ tel que $f(x)=0$ (on oublie qu'ici on pourrait trouver une solution exacte).
Une méthode classique et efficace est la méthode de Newton.
Expliquons-la d'abord géométriquement, on part d'une valeur 
$a_0$ et on construit une suite $(a_n)$ qui va tendre
(très vite) vers un zéro de $f$. On construit
d'abord la tangente à la courbe de $f$ au point d'abscisse $a_0$,
cette tangente recoupe l'axe des abscisse en une abscisse que l'on note
$a_1$. On recommence le processus  : on construit la tangente au point d'abscisse $a_1$\ldots
On remplace localement la courbe par une droite,
d'où le nom du processus  : la linéarisation. 


\myfigure{1.2}{  
  \tikzinput{fig_gps_erreur_01}
}

Passons à la partie calculs. Pour un point $a$, la tangente en 
$a$ admet pour équation :
$$y = f(a) + f'(a)(x-a).$$
Trouver l'intersection avec l'axe des abscisses 
d'équation $(y=0)$ revient 
à résoudre l'équation $0 = f(a) + f'(a)(x-a)$.  
Ce qui conduit à $x = a - \frac{f(a)}{f'(a)}$.
Ainsi on définit par récurrence, la suite
\begin{equation}
\label{eq :gpsnewt1}
a_{n+1} = a_n - \frac{f(a_n)}{f'(a_n)}
\end{equation}
Comme $f'(x) = 2(x-x_1)$, on trouve explicitement  :
$$a_{n+1} = a_n - \frac12 \left( (a_n-x_1) - 
\frac{r_1^2}{a_n-x_1} \right).$$

Une étude de fonction et de suite récurrente prouve le résultat suivant
pour notre fonction $f(x) = (x-x_1)^2-r_1^2$.
\begin{proposition}
Fixons $a_0 \in \Rr$. La suite définie par la relation de récurrence
\mybox{$\displaystyle a_{n+1} = a_n - \frac{f(a_n)}{f'(a_n)}$}
converge vers une solution de $f(x)=0$.
\end{proposition}


Une remarque importante  : une équation de la tangente est $(g(x)=0)$ où
$g(x) = f(a) + f'(a)(x-a)$ ce qui correspond au développement limité à l'ordre $1$ de $f$.
On remplace donc la recherche d'un zéro de $f$ par celle, beaucoup plus facile, 
d'un zéro de $g$. Le zéro de $g$ obtenu ne répond qu'approximativement au problème, mais on
peut recommencer le processus jusqu'à obtenir la précision souhaitée. 


%---------------------------------------------------------------
\subsection{Calcul par linéarisation (dimension 2)}

En dimension $2$, chacun des deux satellites correspond à une fonction  :
$$f_1(x,y) = (x-x_1)^2+(y-y_1)^2-r_1^2 \qquad f_2(x,y) = (x-x_2)^2+(y-y_2)^2-r_2^2$$
qui donne deux équations $(f_1(x,y)=0)$ et $(f_2(x,y)=0)$.
Il s'agit donc de linéariser la fonction $f_1$ (et la fonction $f_2$)
en prenant leur développement limité à l'ordre $1$ en un point $(a,b)$  :
$$g_1(x,y) = f_1(a,b) + (x-a) \frac{\partial f_1}{\partial x}(a,b)+ (y-b) \frac{\partial f_1}{\partial y}(a,b)$$
$$g_2(x,y) = f_2(a,b) + (x-a) \frac{\partial f_2}{\partial x}(a,b)+ (y-b) \frac{\partial f_2}{\partial y}(a,b)$$
Au lieu de chercher l'intersection $\big(f_1(x,y)=0\big) \cap\big(f_2(x,y)=0\big)$
on cherche l'intersection $\big(g_1(x,y)=0\big) \cap \big(g_2(x,y)=0\big)$, ce qui en notation vectorielle conduit à :
$$\begin{pmatrix}0\\0\end{pmatrix}
= \begin{pmatrix}f_1(a,b)\\f_2(a,b)\end{pmatrix}
+ \begin{pmatrix}
  \frac{\partial f_1}{\partial x}(a,b) & \frac{\partial f_1}{\partial y}(a,b) \\
  \frac{\partial f_2}{\partial x}(a,b) & \frac{\partial f_2}{\partial y}(a,b)
  \end{pmatrix}
\begin{pmatrix}x-a\\y-b\end{pmatrix}$$
Notons $J(a,b)$ la matrice jacobienne  :
$$J(a,b) = \begin{pmatrix}
  \frac{\partial f_1}{\partial x}(a,b) & \frac{\partial f_1}{\partial y}(a,b) \\
  \frac{\partial f_2}{\partial x}(a,b) & \frac{\partial f_2}{\partial y}(a,b)
  \end{pmatrix}$$
alors, on doit résoudre 
$$J(a,b) \begin{pmatrix}x-a\\y-b\end{pmatrix} = - \begin{pmatrix}f_1(a,b)\\f_2(a,b)\end{pmatrix}$$
Si $J(a,b)$ est inversible alors on trouve
$$ \begin{pmatrix}x-a\\y-b\end{pmatrix} = -J(a,b)^{-1}\begin{pmatrix}f_1(a,b)\\f_2(a,b)\end{pmatrix}$$
donc 
$$\begin{pmatrix}x\\y\end{pmatrix} = \begin{pmatrix}a\\b\end{pmatrix}-J(a,b)^{-1}\begin{pmatrix}f_1(a,b)\\f_2(a,b)\end{pmatrix}$$

\begin{proposition}
Partant d'un point quelconque 
$\left(\begin{smallmatrix}a_0\\b_0\end{smallmatrix}\right)$ on définit par récurrence la suite de points 
\mybox{$\displaystyle
\begin{pmatrix}a_{n+1}\\b_{n+1}\end{pmatrix} = \begin{pmatrix}a_n\\b_n\end{pmatrix}-J(a_n,b_n)^{-1}\begin{pmatrix}f_1(a_n,b_n)\\f_2(a_n,b_n)\end{pmatrix}
$}
Cette suite converge vers une solution $(x,y)$ vérifiant $f_1(x,y)=0$ et $f_2(x,y)=0$.
\end{proposition}

Nous admettons cette proposition.

La formule de récurrence 
\begin{equation}
\label{eq :gpsnewt2}
\begin{pmatrix}a_{n+1}\\b_{n+1}\end{pmatrix} = \begin{pmatrix}a_n\\b_n\end{pmatrix}-J(a_n,b_n)^{-1}\begin{pmatrix}f_1(a_n,b_n)\\f_2(a_n,b_n)\end{pmatrix}
\end{equation}
est comparable à 
la formule (\ref{eq :gpsnewt1}) si l'on considère que $J(a,b)$ joue le rôle de la dérivée.
Connaissant $f_1(x,y) = (x-x_1)^2+(y-y_1)^2-r_1^2$
et $f_2(x,y) = (x-x_2)^2+(y-y_2)^2-r_2^2$, on calcule
$$J(a,b) = \begin{pmatrix}
  \frac{\partial f_1}{\partial x}(a,b) & \frac{\partial f_1}{\partial y}(a,b) \\
  \frac{\partial f_2}{\partial x}(a,b) & \frac{\partial f_2}{\partial y}(a,b)
  \end{pmatrix}
  = 2 \begin{pmatrix}a-x_1&b-y_1\\a-x_2&b-y_2\end{pmatrix}$$

La matrice $J(a,b)$ est donc inversible si les vecteurs
$(a-x_1,b-y_1)$ et $(a-x_2,b-y_2)$ ne sont pas colinéaires.
Autrement dit, si les points
$(a,b)$, $(x_1,y_1)$, $(x_2,y_2)$ ne sont pas alignés.



Voici une interprétation géométrique de la linéarisation  :
on cherche une solution qui doit appartenir au cercle d'équation $(f_1(x,y)=0)$.
Pour cela on part d'un point $(a,b)$ hors du cercle, on note
$\delta = f_1(a,b) \neq 0$ qui mesure l'écart entre le point $(a,b)$ 
et le cercle.
Par définition de $g_1$ on a aussi $g_1(a,b)=\delta$.
Les courbes d'équation $g_1(x,y)= \text{cst}$ sont des droites
perpendiculaires à la droite passant par $(x_1,y_1)$ et $(a,b)$.

Nous ce qui nous intéresse ce sont les points du cercle $(f_1(x,y)=0)$,
la méthode de linéarisation consiste à remplacer ce cercle par 
les points de la droite $(g_1(x,y)=0)$. Cette droite 
est une droite proche (mais pas égale) à la tangente au cercle.

\myfigure{0.5}{  
  \tikzinput{fig_gps_erreur_02}  
}


Dans la situation de deux satellites, on souhaite trouver une des intersections
de deux cercles $(f_1(x,y)=0)$ et $(f_2(x,y)=0)$. Partant d'un $(a_0,b_0)$
quelconque, on calcule les fonctions $g_1$ et $g_2$ associées à ce point $(a_0,b_0)$.
On trace les deux droites d'équation $(g_1(x,y)=0)$ et $(g_2(x,y)=0)$.
Elles se coupent en un point $(a_1,b_1)$ assez proche de l'intersection des cercles.
On recommence en partant du point $(a_1,b_1)$ pour lequel on calcule de
nouvelles fonctions $g_1$, $g_2$, qui fournissent deux nouvelles droites, dont l'intersection
est encore plus proche de notre solution.

\myfigure{0.7}{  
  \tikzinput{fig_gps_erreur_03}  
}


\begin{exemple}
On reçoit le signal d'un satellite avec $(x_1,y_1)=(0,0)$, $r_1=4$
et d'un second avec $(x_2,y_2)=(5,2)$, $r_2=3$.
La dernière position connue est $(a_0,b_0)=(1,6)$. Où sommes nous ?

Nous allons chercher notre position approchée par la méthode de linéarisation.
On pose 
$$f_1(x,y) = x^2+y^2-4^2 \qquad f_2(x,y) = (x-5)^2+(y-2)^2-3^2$$
Donc
$$J(a,b) = 2 \begin{pmatrix}a&b\\a-5&b-2\end{pmatrix}$$

\begin{itemize}
  \item \textbf{\'Etape 0.} $\begin{pmatrix}a_0\\b_0\end{pmatrix}=\begin{pmatrix}1\\6\end{pmatrix}$
  \item \textbf{\'Etape 1.} Nous avons :
  $$J(a_0,b_0) = \begin{pmatrix}1&6\\-4&4\end{pmatrix}\qquad
  \begin{pmatrix}f_1(a_0,b_0)\\f_2(a_0,b_0)\end{pmatrix}=\begin{pmatrix}21\\23\end{pmatrix}$$
  Donc
  $$\begin{pmatrix}a_{1}\\b_{1}\end{pmatrix} = \begin{pmatrix}a_0\\b_0\end{pmatrix}-J(a_0,b_0)^{-1}\begin{pmatrix}f_1(a_0,b_0)\\f_2(a_0,b_0)\end{pmatrix}
  = \begin{pmatrix}1\\6\end{pmatrix} - \begin{pmatrix}2&12\\-8&8\end{pmatrix}^{-1} \begin{pmatrix}21\\23\end{pmatrix}
  = \begin{pmatrix}55/28\\229/56\end{pmatrix}
  \simeq \begin{pmatrix} \num{1.9643}\\\num{4.0893}\end{pmatrix}$$
  \item \textbf{\'Etape 2.} 
  $$J(a_1,b_1) = \begin{pmatrix}55/14&229/28\\-85/14& 117/28\end{pmatrix}\qquad
  \begin{pmatrix}f_1(a_1,b_1)\\f_2(a_1,b_1)\end{pmatrix}
  =\begin{pmatrix}14365/3136\\14365/3136\end{pmatrix}$$
  Donc
  $$\begin{pmatrix}a_{2}\\b_{2}\end{pmatrix} = \begin{pmatrix}a_1\\b_1\end{pmatrix}-J(a_1,b_1)^{-1}\begin{pmatrix}f_1(a_1,b_1)\\f_2(a_1,b_1)\end{pmatrix}
  = \begin{pmatrix}23223/10360\\14073/4144\end{pmatrix}
  \simeq \begin{pmatrix} \num{2.2416}\\\num{3.3960}\end{pmatrix}$$
  \item \textbf{\'Etape 2.} Après calculs
  $$\begin{pmatrix}a_{3}\\b_{3}\end{pmatrix} \simeq \begin{pmatrix} \num{2.2862}\\\num{3.2845}\end{pmatrix}$$
\end{itemize}
Les solutions exactes s'obtiennent par calculs algébriques :
$$\begin{pmatrix}\frac{90}{29}-\frac{4}{29}\sqrt{35}\\ \frac{36}{29}+\frac{10}{29} \sqrt{35} \end{pmatrix} \simeq \begin{pmatrix}\num{2.2874} \\ \num{3.2814}\end{pmatrix} \qquad
\begin{pmatrix}\frac{90}{29}+\frac{4}{29}\sqrt{35}\\ \frac{36}{29}-\frac{10}{29} \sqrt{35} \end{pmatrix} \simeq \begin{pmatrix}\num{3.9194} \\ \num{-0.7986}\end{pmatrix}$$

\`A la fin de l'étape 2, sur le dessin, on ne distingue plus 
la solution approchée $(a_2,b_2)$ de la solution exacte.
Sur le dessin, seule la construction de $(a_0,b_0)$ vers $(a_1,b_1)$ est dessinée. 
\myfigure{0.8}{  
  \tikzinput{fig_gps_erreur_04}
}
  
\end{exemple}

%---------------------------------------------------------------
\subsection{Passage à un système surdimensionné}


Quelque soit la méthode de résolution, par linéarisation ou par 
calcul exact, on est amené à résoudre un (ou plusieurs) système linéaire
de type $AX=B$.
Jusqu'ici nous avons toujours supposé qu'il y a avait autant d'inconnues
(dans le cas le plus général $(x,y,z,t)$)
que d'équations (une équation par satellite). Ce qui fait que la matrice $A$ est une matrice 
carrée, et le plus souvent inversible, ce qui permet de résoudre le système linéaire
en : $X = A^{-1} B$.

Comment profiter de la réception de plus de signaux ? Parmi les $30$ satellites
du système GPS, tous ne sont pas \og visibles \fg{} par l'utilisateur.
Cependant il y a bien souvent entre $7$ et $10$ signaux reçus.
Nous allons voir comment en profiter. 


Il faut tout d'abord bien comprendre qu'en théorie rajouter plus d'équations
ne change rien : les solutions restent les mêmes.
En pratique c'est différent, du fait des erreurs inévitables il n'y a aucun point
qui satisfasse toutes les équations ; cependant on souhaite trouver un point
qui répond au mieux au problème.
En théorie (figure de gauche) un point est solution de toutes les équations,
en pratique même si aucun point n'est solution, une zone de points est à privilégier
(zoom de la figure de droite).
\myfigure{0.9}{  
  \tikzinput{fig_gps_erreur_05}\quad
  \tikzinput{fig_gps_erreur_06}
}


Dans le cadre plus général nous avions $p=4$ inconnues $(x,y,z,t)$
(mais souvent pour les exemples on se contentait de $(x,y)$).
On note $X$ le vecteur colonne des inconnues, c'est un vecteur de taille $p$.
Chaque satellite fournit une équation linéaire, donc avec $n$ satellites
on aboutit à un système linéaire 
$$AX=B$$
où $A \in M_{n,p}$ est une matrice avec $n$ lignes et $p$ colonnes,
le second membre $B$ est un vecteur colonne de taille $n$.

Dans le cas qui nous intéresse ici, $n\ge p$, 
$A\in M_{n,p}$ n'est plus une matrice carrée,
parler de son inverse n'a plus de sens.

%---------------------------------------------------------------
\subsection{Méthode des moindres carrés}


Soit $A \in M_{n,p}$ une matrice, $X$ un vecteur inconnu de taille $p$ et
$B$ un vecteur de taille $n$.
Comme il n'y a en général pas de solution $X$,
au système $AX=B$, il faut se mettre d'accord sur le $X$ qui 
convient, comme il faut que $AX-B$ soit le plus proche du vecteur nul,
on définit  :
\begin{definition}
Une \defi{solution des moindres carrés} $X$ est un vecteur
tel que $\| AX -B \|$ soit le plus petit possible.
\end{definition}

Si on note le vecteur 
$C = AX-B = \left(\begin{smallmatrix}c_1\\c_2\\\vdots\\c_n\end{smallmatrix}\right)$ alors 
$$\| C \|^2 = \langle C \mid C \rangle = C^T \cdot C = c_1^2 + c_2^2+\cdots+c_n^2$$
Il s'agit donc de minimiser cette somme des carrés, d'où le nom \og moindres carrés \fg.

\begin{proposition}
\sauteligne
\begin{enumerate}
  \item Une solution au problème des moindres carrés $AX=B$ existe et est unique.
  \item La solution des moindres carrés est donnée par la formule 
  $$X = (A^TA)^{-1} A^T B$$ 
  \item Géométriquement, si $Y$ est le projeté orthogonal de $B$
  sur le sous-espace vectoriel engendré par les vecteurs colonnes de la matrice 
  $A$, alors $X$ est la solution du système $AX=Y$.
\end{enumerate} 
\end{proposition}


Remarque : nous admettons (et nous supposons) que la matrice $A^TA$ est inversible 
lorsque $A$ est de rang maximal (donc de rang $p$, car pour nous $n \ge p$).
Expliquons la caractérisation géométrique.

\myfigure{1.2}{  
  \tikzinput{fig_gps_erreur_07}
}

Pour une matrice $A \in M_{n,p}$, voici des définitions équivalentes de l'\defi{image} de $A$ :
Notons $v_1,v_2,\ldots,v_p$ les $p$ colonnes de la matrice $A$.

\begin{itemize}
  \item $\Im A = \Vect(v_1,v_2,\ldots,v_p)$ est le sous-espace vectoriel de $\Rr^n$ engendré par 
  les $p$ vecteurs colonnes $v_1,v_2,\ldots,v_p$ de la matrice $A$. Autrement dit:
  $$\Im A = \Vect(v_1,v_2,\ldots,v_p) = \big\lbrace x_1v_1+\cdots +x_pv_p \in\Rr^n\mid x_1,\ldots,x_p \in \Rr\big\rbrace$$
  
  \item $\Im A$ est l'ensemble des vecteurs images par $A$ :
  $$\Im A = \big\lbrace AX \mid X \in \Rr^p \big\rbrace$$
\end{itemize}

Dire que le problème $AX=B$ n'a pas de solution exacte, c'est dire 
que le vecteur $B$ n'appartient pas à $\Im A$.
Soit $Y = p(B)$ le projeté orthogonal de $B$ sur $\Im A$. 
\begin{itemize}
  \item D'une part $Y \in \Im A$ et $Y$ minimise la distance entre $B$ et les points de $\Im A$,
autrement dit $\| Y-B\| \le \| Y'-B\|$ quelque soit $Y'\in\Im A$. C'est une des 
caractérisations géométriques de la projection orthogonale.
  
  \item D'autre part comme $Y \in \Im A$ alors il existe $X \in \Rr^p$ tel que 
  $AX = Y$.
\end{itemize}


\begin{exemple}
On reçoit $4$ signaux sous la forme de $4$ droites du plan.
On cherche quel est le point du plan \og le plus proche \fg{}  de toutes les droites.
Les équations sont les suivantes :
$$D_1 : \quad (x=4) \qquad D_2 :\quad (y=3) \qquad
D_3 : \quad (x-2y=0) \qquad D_4 : \quad (x+y=4)$$

Bien sûr dans le cas de deux droites le point \og le plus proche \fg{} 
serait le point d'intersection. Dans l'idéal on aimerait donc trouver un point commun 
à toutes les droites, c'est-à-dire que l'on aimerait résoudre le système linéaire
$$\left\lbrace\begin{array}{cccl}
x&&=&4\\
&y&=&3\\
x&-2y&=&0\\
x&+y&=&4\\
\end{array}\right.$$
Autrement dit on voudrait $X$ avec: 
$$AX = B \quad \text{ où } \quad 
X=\begin{pmatrix}x\\y\end{pmatrix}\quad
A=\begin{pmatrix}1&0\\0&1\\1&-2\\1&1\end{pmatrix}\quad
B=\begin{pmatrix}4\\3\\0\\4\end{pmatrix}$$

Il n'y a aucune solution $X$ à ce problème, les droites
n'étant pas concourantes. On décide d'appeler point \og le plus proche \fg{}
la solution $X$ du problème des moindres carrés.
Ainsi 
$$X = (A^TA)^{-1} A^T B$$
On calcul
$$A^TA  
= \begin{pmatrix}1&0&1&1\\0&1&-2&1\end{pmatrix}\begin{pmatrix}1&0\\0&1\\1&-2\\1&1\end{pmatrix}
= \begin{pmatrix}3&-1\\-1&6\end{pmatrix} \quad  \text{ donc }
(A^TA)^{-1} = \frac{1}{17} \begin{pmatrix}6&1\\1&3\end{pmatrix}$$
Donc 
$$(A^TA)^{-1} A^T = 
\frac{1}{17} \begin{pmatrix}6&1&4&7\\1&3&-5&4\end{pmatrix}$$
et ainsi :
$$X = (A^TA)^{-1} A^T B = \begin{pmatrix}55/17\\29/18\end{pmatrix}
\simeq \begin{pmatrix}\num{3.235}\\\num{1.706}\end{pmatrix}$$

\myfigure{1}{  
  \tikzinput{fig_gps_erreur_08}
}
  
\end{exemple}


\begin{proof}~
\begin{enumerate}
  \item L'existence découle du troisième point, 
  l'unicité du deuxième.
  
  \item On a déjà dit que l'on aimerait résoudre l'équation $AX-B=0$, 
  mais il n'y a pas de solution en général. On cherche donc $AX-B$ le plus petit possible.
  Il s'agit donc de minimiser $\|AX-B\|$, où ce qui revient au même $\|AX-B\|^2$.
  Comme dans le cas d'une variable, lorsqu'il y a un minimum  la dérivée s'annule. 
  On cherche donc un point où la dérivée s'annule 
  ($\dd$ dénote l'application de dérivation) :
  \begin{eqnarray*}
    && \dd  \big(  c_1^2 + c_2^2+\cdots+c_n^2 \big) = 0 \\
    & \iff & \dd \|AX-B\|^2 = 0 \\
    & \iff & \dd \big((AX-B)^T(AX-B)\big) = 0 \\  
    & \iff & \big(\dd (AX-B)^T\big)(AX-B) + (AX-B)^T\big(\dd (AX-B)\big)  = 0
    \quad \text{par la formule } (uv)'=u'v+uv' \\
    & \iff & (A\dd X)^T(AX-B) + (AX-B)^T (A\dd X)  = 0
    \quad \text{car $A$ et $B$ sont des constantes et $X$ est la variable}  \\  
    & \iff & (2A\dd X)^T(AX-B) = 0 
    \quad \text{car pour des vecteurs } u^T \cdot v = \langle u \mid v \rangle = v^T \cdot u \\
    & \iff & \dd X^T A^T(AX-B) = 0 \\
    & \iff & \dd X^T (A^TAX-A^TB) = 0  
    \quad \text{(quel que soit $\dd X$)}\\
    & \iff & A^TAX-A^TB = 0 \\
    & \iff & A^TAX = A^TB \quad \text{(attention $A^T$ n'est pas inversible, mais $A^TA$ oui)}\\
    & \iff & X = (A^TA)^{-1} A^TB \\
  \end{eqnarray*}
  Conclusion : si un $X$ existe alors il est unique et c'est  $X = (A^TA)^{-1} A^TB$.
  
  \item C'est l'une des caractérisation du projeté orthogonal : $p(B)$ est le point de $\Im A$
  qui minimise la distance entre $B$ et un point de $\Im A$. Cette propriété entraîne en particulier 
  l'existence de $X$. Vérifions que le calcul précédent coïncide avec le calcul de la projection orthogonale.
  
  Soit $p : \Rr^n \to \Rr^n$ la projection orthogonale sur $\Im A$.
  
  Notons $Y=p(B)$. Que $Y$ soit la projection orthogonale de $B$ sur $\Im A$ signifie
  que le vecteur $B-Y$ est orthogonal à tous vecteurs de $\Im A$:
  \begin{eqnarray*}
    && \langle AV \mid B-Y \rangle = 0 \quad \forall V \in \Rr^p \\
    & \iff & \langle AV \mid B-AX \rangle = 0 \quad \forall V \in \Rr^p \quad \text{ pour } Y=AX\\
    & \iff & (AV)^T \cdot (B-AX)= 0 \quad \forall V \in \Rr^p \quad \text{ car } \langle u \mid v\rangle = u^T\cdot v\\
    & \iff & V^T  A^T (B-AX)= 0 \quad \forall V \in \Rr^p \\    
    & \iff & A^T (B-AX)= 0 \\
    & \iff & A^TAX= A^T B \\
    & \iff & X = (A^TA)^{-1} A^TB \\    
  \end{eqnarray*}    
\end{enumerate}

\end{proof}



%---------------------------------------------------------------
\subsection*{Bibliographie}

\begin{itemize}
  \item Stephen Bancroft, 
\emph{An algebraic solution of the GPS equations.}
IEEE Transactions, 1985.
  
  \item Pierre-André Chevalier, 
\emph{Comment le GPS calcule-t-il sa position ?}
\texttt{\href{https://staff.hti.bfh.ch/cip1/gps/}{Une introduction didactique à la géométrie du GPS.}}
2002.
  
  \item Wolfgang Torge, 
\emph{Geodesy.}
de Gruyter, 2001.
\end{itemize}




\auteurs{
Arnaud Bodin

Relu par Bernard Callenaere
}

\finchapitre
\end{document}

