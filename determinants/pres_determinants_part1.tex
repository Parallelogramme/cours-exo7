
%%%%%%%%%%%%%%%%%% PREAMBULE %%%%%%%%%%%%%%%%%%

\documentclass[aspectratio=169,utf8]{beamer}
%\documentclass[aspectratio=169,handout]{beamer}

\usetheme{Boadilla}
%\usecolortheme{seahorse}
\usecolortheme[RGB={245,66,24}]{structure}
\useoutertheme{infolines}

% packages
\usepackage{amsfonts,amsmath,amssymb,amsthm}
\usepackage[utf8]{inputenc}
\usepackage[T1]{fontenc}
\usepackage{lmodern}

\usepackage[francais]{babel}
\usepackage{fancybox}
\usepackage{graphicx}

\usepackage{float}
\usepackage{xfrac}

%\usepackage[usenames, x11names]{xcolor}
\usepackage{tikz}
\usepackage{pgfplots}
\usepackage{datetime}



%-----  Package unités -----
\usepackage{siunitx}
\sisetup{locale = FR,detect-all,per-mode = symbol}

%\usepackage{mathptmx}
%\usepackage{fouriernc}
%\usepackage{newcent}
%\usepackage[mathcal,mathbf]{euler}

%\usepackage{palatino}
%\usepackage{newcent}
% \usepackage[mathcal,mathbf]{euler}



% \usepackage{hyperref}
% \hypersetup{colorlinks=true, linkcolor=blue, urlcolor=blue,
% pdftitle={Exo7 - Exercices de mathématiques}, pdfauthor={Exo7}}


%section
% \usepackage{sectsty}
% \allsectionsfont{\bf}
%\sectionfont{\color{Tomato3}\upshape\selectfont}
%\subsectionfont{\color{Tomato4}\upshape\selectfont}

%----- Ensembles : entiers, reels, complexes -----
\newcommand{\Nn}{\mathbb{N}} \newcommand{\N}{\mathbb{N}}
\newcommand{\Zz}{\mathbb{Z}} \newcommand{\Z}{\mathbb{Z}}
\newcommand{\Qq}{\mathbb{Q}} \newcommand{\Q}{\mathbb{Q}}
\newcommand{\Rr}{\mathbb{R}} \newcommand{\R}{\mathbb{R}}
\newcommand{\Cc}{\mathbb{C}} 
\newcommand{\Kk}{\mathbb{K}} \newcommand{\K}{\mathbb{K}}

%----- Modifications de symboles -----
\renewcommand{\epsilon}{\varepsilon}
\renewcommand{\Re}{\mathop{\text{Re}}\nolimits}
\renewcommand{\Im}{\mathop{\text{Im}}\nolimits}
%\newcommand{\llbracket}{\left[\kern-0.15em\left[}
%\newcommand{\rrbracket}{\right]\kern-0.15em\right]}

\renewcommand{\ge}{\geqslant}
\renewcommand{\geq}{\geqslant}
\renewcommand{\le}{\leqslant}
\renewcommand{\leq}{\leqslant}
\renewcommand{\epsilon}{\varepsilon}

%----- Fonctions usuelles -----
\newcommand{\ch}{\mathop{\text{ch}}\nolimits}
\newcommand{\sh}{\mathop{\text{sh}}\nolimits}
\renewcommand{\tanh}{\mathop{\text{th}}\nolimits}
\newcommand{\cotan}{\mathop{\text{cotan}}\nolimits}
\newcommand{\Arcsin}{\mathop{\text{arcsin}}\nolimits}
\newcommand{\Arccos}{\mathop{\text{arccos}}\nolimits}
\newcommand{\Arctan}{\mathop{\text{arctan}}\nolimits}
\newcommand{\Argsh}{\mathop{\text{argsh}}\nolimits}
\newcommand{\Argch}{\mathop{\text{argch}}\nolimits}
\newcommand{\Argth}{\mathop{\text{argth}}\nolimits}
\newcommand{\pgcd}{\mathop{\text{pgcd}}\nolimits} 


%----- Commandes divers ------
\newcommand{\ii}{\mathrm{i}}
\newcommand{\dd}{\text{d}}
\newcommand{\id}{\mathop{\text{id}}\nolimits}
\newcommand{\Ker}{\mathop{\text{Ker}}\nolimits}
\newcommand{\Card}{\mathop{\text{Card}}\nolimits}
\newcommand{\Vect}{\mathop{\text{Vect}}\nolimits}
\newcommand{\Mat}{\mathop{\text{Mat}}\nolimits}
\newcommand{\rg}{\mathop{\text{rg}}\nolimits}
\newcommand{\tr}{\mathop{\text{tr}}\nolimits}


%----- Structure des exercices ------

\newtheoremstyle{styleexo}% name
{2ex}% Space above
{3ex}% Space below
{}% Body font
{}% Indent amount 1
{\bfseries} % Theorem head font
{}% Punctuation after theorem head
{\newline}% Space after theorem head 2
{}% Theorem head spec (can be left empty, meaning ‘normal’)

%\theoremstyle{styleexo}
\newtheorem{exo}{Exercice}
\newtheorem{ind}{Indications}
\newtheorem{cor}{Correction}


\newcommand{\exercice}[1]{} \newcommand{\finexercice}{}
%\newcommand{\exercice}[1]{{\tiny\texttt{#1}}\vspace{-2ex}} % pour afficher le numero absolu, l'auteur...
\newcommand{\enonce}{\begin{exo}} \newcommand{\finenonce}{\end{exo}}
\newcommand{\indication}{\begin{ind}} \newcommand{\finindication}{\end{ind}}
\newcommand{\correction}{\begin{cor}} \newcommand{\fincorrection}{\end{cor}}

\newcommand{\noindication}{\stepcounter{ind}}
\newcommand{\nocorrection}{\stepcounter{cor}}

\newcommand{\fiche}[1]{} \newcommand{\finfiche}{}
\newcommand{\titre}[1]{\centerline{\large \bf #1}}
\newcommand{\addcommand}[1]{}
\newcommand{\video}[1]{}

% Marge
\newcommand{\mymargin}[1]{\marginpar{{\small #1}}}

\def\noqed{\renewcommand{\qedsymbol}{}}


%----- Presentation ------
\setlength{\parindent}{0cm}

%\newcommand{\ExoSept}{\href{http://exo7.emath.fr}{\textbf{\textsf{Exo7}}}}

\definecolor{myred}{rgb}{0.93,0.26,0}
\definecolor{myorange}{rgb}{0.97,0.58,0}
\definecolor{myyellow}{rgb}{1,0.86,0}

\newcommand{\LogoExoSept}[1]{  % input : echelle
{\usefont{U}{cmss}{bx}{n}
\begin{tikzpicture}[scale=0.1*#1,transform shape]
  \fill[color=myorange] (0,0)--(4,0)--(4,-4)--(0,-4)--cycle;
  \fill[color=myred] (0,0)--(0,3)--(-3,3)--(-3,0)--cycle;
  \fill[color=myyellow] (4,0)--(7,4)--(3,7)--(0,3)--cycle;
  \node[scale=5] at (3.5,3.5) {Exo7};
\end{tikzpicture}}
}


\newcommand{\debutmontitre}{
  \author{} \date{} 
  \thispagestyle{empty}
  \hspace*{-10ex}
  \begin{minipage}{\textwidth}
    \titlepage  
  \vspace*{-2.5cm}
  \begin{center}
    \LogoExoSept{2.5}
  \end{center}
  \end{minipage}

  \vspace*{-0cm}
  
  % Astuce pour que le background ne soit pas discrétisé lors de la conversion pdf -> png
\begin{tikzpicture}
        \fill[opacity=0,green!60!black] (0,0)--++(0,0)--++(0,0)--++(0,0)--cycle; 
\end{tikzpicture}

% toc S'affiche trop tot :
% \tableofcontents[hideallsubsections, pausesections]
}

\newcommand{\finmontitre}{
  \end{frame}
  \setcounter{framenumber}{0}
} % ne marche pas pour une raison obscure

%----- Commandes supplementaires ------

% \usepackage[landscape]{geometry}
% \geometry{top=1cm, bottom=3cm, left=2cm, right=10cm, marginparsep=1cm
% }
% \usepackage[a4paper]{geometry}
% \geometry{top=2cm, bottom=2cm, left=2cm, right=2cm, marginparsep=1cm
% }

%\usepackage{standalone}


% New command Arnaud -- november 2011
\setbeamersize{text margin left=24ex}
% si vous modifier cette valeur il faut aussi
% modifier le decalage du titre pour compenser
% (ex : ici =+10ex, titre =-5ex

\theoremstyle{definition}
%\newtheorem{proposition}{Proposition}
%\newtheorem{exemple}{Exemple}
%\newtheorem{theoreme}{Théorème}
%\newtheorem{lemme}{Lemme}
%\newtheorem{corollaire}{Corollaire}
%\newtheorem*{remarque*}{Remarque}
%\newtheorem*{miniexercice}{Mini-exercices}
%\newtheorem{definition}{Définition}

% Commande tikz
\usetikzlibrary{calc}
\usetikzlibrary{patterns,arrows}
\usetikzlibrary{matrix}
\usetikzlibrary{fadings} 

%definition d'un terme
\newcommand{\defi}[1]{{\color{myorange}\textbf{\emph{#1}}}}
\newcommand{\evidence}[1]{{\color{blue}\textbf{\emph{#1}}}}
\newcommand{\assertion}[1]{\emph{\og#1\fg}}  % pour chapitre logique
%\renewcommand{\contentsname}{Sommaire}
\renewcommand{\contentsname}{}
\setcounter{tocdepth}{2}



%------ Figures ------

\def\myscale{1} % par défaut 
\newcommand{\myfigure}[2]{  % entrée : echelle, fichier figure
\def\myscale{#1}
\begin{center}
\footnotesize
{#2}
\end{center}}


%------ Encadrement ------

\usepackage{fancybox}


\newcommand{\mybox}[1]{
\setlength{\fboxsep}{7pt}
\begin{center}
\shadowbox{#1}
\end{center}}

\newcommand{\myboxinline}[1]{
\setlength{\fboxsep}{5pt}
\raisebox{-10pt}{
\shadowbox{#1}
}
}

%--------------- Commande beamer---------------
\newcommand{\beameronly}[1]{#1} % permet de mettre des pause dans beamer pas dans poly


\setbeamertemplate{navigation symbols}{}
\setbeamertemplate{footline}  % tiré du fichier beamerouterinfolines.sty
{
  \leavevmode%
  \hbox{%
  \begin{beamercolorbox}[wd=.333333\paperwidth,ht=2.25ex,dp=1ex,center]{author in head/foot}%
    % \usebeamerfont{author in head/foot}\insertshortauthor%~~(\insertshortinstitute)
    \usebeamerfont{section in head/foot}{\bf\insertshorttitle}
  \end{beamercolorbox}%
  \begin{beamercolorbox}[wd=.333333\paperwidth,ht=2.25ex,dp=1ex,center]{title in head/foot}%
    \usebeamerfont{section in head/foot}{\bf\insertsectionhead}
  \end{beamercolorbox}%
  \begin{beamercolorbox}[wd=.333333\paperwidth,ht=2.25ex,dp=1ex,right]{date in head/foot}%
    % \usebeamerfont{date in head/foot}\insertshortdate{}\hspace*{2em}
    \insertframenumber{} / \inserttotalframenumber\hspace*{2ex} 
  \end{beamercolorbox}}%
  \vskip0pt%
}


\definecolor{mygrey}{rgb}{0.5,0.5,0.5}
\setlength{\parindent}{0cm}
%\DeclareTextFontCommand{\helvetica}{\fontfamily{phv}\selectfont}

% background beamer
\definecolor{couleurhaut}{rgb}{0.85,0.9,1}  % creme
\definecolor{couleurmilieu}{rgb}{1,1,1}  % vert pale
\definecolor{couleurbas}{rgb}{0.85,0.9,1}  % blanc
\setbeamertemplate{background canvas}[vertical shading]%
[top=couleurhaut,middle=couleurmilieu,midpoint=0.4,bottom=couleurbas] 
%[top=fondtitre!05,bottom=fondtitre!60]



\makeatletter
\setbeamertemplate{theorem begin}
{%
  \begin{\inserttheoremblockenv}
  {%
    \inserttheoremheadfont
    \inserttheoremname
    \inserttheoremnumber
    \ifx\inserttheoremaddition\@empty\else\ (\inserttheoremaddition)\fi%
    \inserttheorempunctuation
  }%
}
\setbeamertemplate{theorem end}{\end{\inserttheoremblockenv}}

\newenvironment{theoreme}[1][]{%
   \setbeamercolor{block title}{fg=structure,bg=structure!40}
   \setbeamercolor{block body}{fg=black,bg=structure!10}
   \begin{block}{{\bf Th\'eor\`eme }#1}
}{%
   \end{block}%
}


\newenvironment{proposition}[1][]{%
   \setbeamercolor{block title}{fg=structure,bg=structure!40}
   \setbeamercolor{block body}{fg=black,bg=structure!10}
   \begin{block}{{\bf Proposition }#1}
}{%
   \end{block}%
}

\newenvironment{corollaire}[1][]{%
   \setbeamercolor{block title}{fg=structure,bg=structure!40}
   \setbeamercolor{block body}{fg=black,bg=structure!10}
   \begin{block}{{\bf Corollaire }#1}
}{%
   \end{block}%
}

\newenvironment{mydefinition}[1][]{%
   \setbeamercolor{block title}{fg=structure,bg=structure!40}
   \setbeamercolor{block body}{fg=black,bg=structure!10}
   \begin{block}{{\bf Définition} #1}
}{%
   \end{block}%
}

\newenvironment{lemme}[0]{%
   \setbeamercolor{block title}{fg=structure,bg=structure!40}
   \setbeamercolor{block body}{fg=black,bg=structure!10}
   \begin{block}{\bf Lemme}
}{%
   \end{block}%
}

\newenvironment{remarque}[1][]{%
   \setbeamercolor{block title}{fg=black,bg=structure!20}
   \setbeamercolor{block body}{fg=black,bg=structure!5}
   \begin{block}{Remarque #1}
}{%
   \end{block}%
}


\newenvironment{exemple}[1][]{%
   \setbeamercolor{block title}{fg=black,bg=structure!20}
   \setbeamercolor{block body}{fg=black,bg=structure!5}
   \begin{block}{{\bf Exemple }#1}
}{%
   \end{block}%
}


\newenvironment{miniexercice}[0]{%
   \setbeamercolor{block title}{fg=structure,bg=structure!20}
   \setbeamercolor{block body}{fg=black,bg=structure!5}
   \begin{block}{Mini-exercices}
}{%
   \end{block}%
}


\newenvironment{tp}[0]{%
   \setbeamercolor{block title}{fg=structure,bg=structure!40}
   \setbeamercolor{block body}{fg=black,bg=structure!10}
   \begin{block}{\bf Travaux pratiques}
}{%
   \end{block}%
}
\newenvironment{exercicecours}[1][]{%
   \setbeamercolor{block title}{fg=structure,bg=structure!40}
   \setbeamercolor{block body}{fg=black,bg=structure!10}
   \begin{block}{{\bf Exercice }#1}
}{%
   \end{block}%
}
\newenvironment{algo}[1][]{%
   \setbeamercolor{block title}{fg=structure,bg=structure!40}
   \setbeamercolor{block body}{fg=black,bg=structure!10}
   \begin{block}{{\bf Algorithme}\hfill{\color{gray}\texttt{#1}}}
}{%
   \end{block}%
}


\setbeamertemplate{proof begin}{
   \setbeamercolor{block title}{fg=black,bg=structure!20}
   \setbeamercolor{block body}{fg=black,bg=structure!5}
   \begin{block}{{\footnotesize Démonstration}}
   \footnotesize
   \smallskip}
\setbeamertemplate{proof end}{%
   \end{block}}
\setbeamertemplate{qed symbol}{\openbox}


\makeatother
\usecolortheme[RGB={204,0,0}]{structure}
   
%%%%%%%%%%%%%%%%%%%%%%%%%%%%%%%%%%%%%%%%%%%%%%%%%%%%%%%%%%%%%
%%%%%%%%%%%%%%%%%%%%%%%%%%%%%%%%%%%%%%%%%%%%%%%%%%%%%%%%%%%%%


\begin{document}


\title{{\bf Déterminants}}
\subtitle{Déterminants en dimension 2 et 3}

\begin{frame}
  
  \debutmontitre

  \pause

{\footnotesize
\hfill
\setbeamercovered{transparent=50}
\begin{minipage}{0.6\textwidth}
  \begin{itemize}
   \item<3-> Matrice $2\times2$ 
   \item<4-> Matrice $3\times3$
   \item<5-> Interprétation géométrique
  \end{itemize}
\end{minipage}
}

\end{frame}
\setcounter{framenumber}{0}


%%%%%%%%%%%%%%%%%%%%%%%%%%%%%%%%%%%%%%%%%%%%%%%%%%%%%%%%%%%%%%%%
\section{Matrice $2\times2$}

\begin{frame}

$\Kk=\Rr$ ou $\Kk=\Cc$

\pause

\bigskip


\evidence{Matrice $2\times2$}


$$\det \begin{pmatrix}a&b\\c&d\end{pmatrix} = ad-bc$$


\pause 

\myfigure{2}{
\tikzinput{fig_determinants01} 
}


\end{frame}


%%%%%%%%%%%%%%%%%%%%%%%%%%%%%%%%%%%%%%%%%%%%%%%%%%%%%%%%%%%%%%%%
\section{Matrice $3\times3$}

\begin{frame}

\evidence{Matrice $3\times3$}

\begin{itemize}
  \item Soit $A = \begin{pmatrix}
      a_{11} & a_{12} & a_{13} \\
      a_{21} & a_{22} & a_{23} \\
      a_{31} & a_{32} & a_{33} \\      
      \end{pmatrix}$
  \pause 
  
  \item $\displaystyle 
\begin{array}{rcl}
\text{ }
\\
\det A & = &
\color<4,6->{blue!100}{a_{11} a_{22} a_{33} 
+ a_{12} a_{23} a_{31} 
+ a_{13} a_{21} a_{32} }\\
&& \quad \color<5,6->{orange!100}{- a_{31} a_{22} a_{13} 
- a_{32} a_{23} a_{11} 
- a_{33} a_{21} a_{12}}
\end{array}$

  \item \pause La \defi{règle de Sarrus}
\myfigure{2}{
\tikzinput{fig_determinants02beamer} 
}
\pause\pause\pause
\item\pause Uniquement pour les matrices $3\times 3$
\end{itemize}

\end{frame}


\begin{frame}
\begin{exemple}

\begin{itemize}
  \item Calculons le déterminant de
$A = 
\begin{pmatrix}
 2 & 1  & 0\\
 1 & -1 & 3\\
 3 & 2  & 1  
\end{pmatrix}
$

  \item\pause Par la règle de Sarrus: \pause
\[
\begin{aligned}
\det A & = 
 {\color<3->{blue!100} 2\times (-1) \times 1 }
\pause+{\color<4->{blue!70} 1\times 3 \times  3 }
\pause+{\color<5->{blue!50} 0\times 1 \times 2 }\\
&\pause\quad  -{\color<6->{orange!100} 3\times (-1) \times 0 } 
-{\color<6->{orange!70} 2\times 3 \times 2 }
-{\color<6->{orange!50} 1\times 1 \times 1 }\\
\pause &= -6
\end{aligned}
\]
\onslide<2->
\myfigure{2}{
\tikzinput{fig_determinants03bis} 
} 
\end{itemize}

\end{exemple}

\end{frame}



%%%%%%%%%%%%%%%%%%%%%%%%%%%%%%%%%%%%%%%%%%%%%%%%%%%%%%%%%%%%%%%%
\section{Interprétation géométrique}

\begin{frame}

\begin{itemize}
  \item<2-> Donnons nous deux vecteurs 
$v_1= \left(\begin{smallmatrix}a\\c\end{smallmatrix}\right)$ et 
$v_2= \left(\begin{smallmatrix}b\\d\end{smallmatrix}\right)$             
du plan $\Rr^2$ 

  \item<3->
$v_1,v_2$ déterminent un parallélogramme
\end{itemize}
\myfigure{1}{
\tikzinput{fig_determinants04} 
}
\onslide<4->{
\begin{proposition}
L'aire du parallélogramme est égale à la valeur absolue du déterminant
$$\mathcal{A} 
= \Big|\det(v_1,v_2)\Big| 
= \Big|\det
\begin{pmatrix}
a & b \\
c & d
\end{pmatrix}\Big|
 = | a d - b c |
$$
\end{proposition}
}
\end{frame}

%--------------------------------------------------------------

\begin{frame}

%\begin{itemize}
  %\item 
  Trois vecteurs
$
v_1=\begin{pmatrix}a_{11}\\a_{21}\\a_{31}\end{pmatrix} \
v_2=\begin{pmatrix}a_{12}\\a_{22}\\a_{32}\end{pmatrix} \
v_3=\begin{pmatrix}a_{13}\\a_{23}\\a_{33}\end{pmatrix}$ \
 de $\Rr^3$
\pause
\medskip

définissent un parallélépipède

\myfigure{1}{
\tikzinput{fig_determinants05} 
}

\pause
  %\item $v_1,v_2,v_3$ définissent 
  et une matrice %et un déterminant
\[ 
%\det(v_1,v_2,v_3)=\det
\begin{pmatrix}
a_{11}&a_{12}&a_{13}\\
a_{21}&a_{22}&a_{23}\\
a_{31}&a_{32}&a_{33}
\end{pmatrix}
\]
%\end{itemize}

\end{frame}

%%%%%%%

\begin{frame}

\begin{minipage}{0.68\textwidth}
\begin{proposition}
Le volume du parallélépipède est égal à la valeur absolue du déterminant
\[
\mathcal{V} = \Big|\det(v_1,v_2,v_3)\Big|
 = \Big| \det \begin{pmatrix}
a_{11}&a_{12}&a_{13}\\
a_{21}&a_{22}&a_{23}\\
a_{31}&a_{32}&a_{33}
\end{pmatrix}
\Big|
\]
\end{proposition}  
\end{minipage}\hspace*{0.5em}
\begin{minipage}{0.29\textwidth}
\myfigure{0.8}{
\tikzinput{fig_determinants05} 
}  
\end{minipage}


\bigskip

\begin{itemize}
  \item\pause unité d'aire dans $\Rr^2$ = aire du carré unité de côtés $\left(\left(\begin{smallmatrix} 1 \\ 0 \end{smallmatrix} \right),
\left(\begin{smallmatrix} 0 \\ 1 \end{smallmatrix} \right)\right)$
  \item\pause unité de volume dans $\Rr^3$ = volume du cube unité
\end{itemize}

\end{frame}

%%%%%%%


\begin{frame}

\begin{proof}

\begin{itemize}
  \item On traite le cas de la dimension $2$
  \item\pause Le résultat est vrai si 
$v_1=\left(\begin{smallmatrix}a\\0\end{smallmatrix}\right)$ et 
$v_2=\left(\begin{smallmatrix}0\\d\end{smallmatrix}\right)$\pause.  En effet
\begin{itemize}
  \item on a un rectangle de côtés $|a|$ et $|d|$ donc d'aire $|ad|$
  \item\pause $\det \begin{pmatrix}
	a&0\\0&d
\end{pmatrix} =ad$
\end{itemize}
\vspace{-.6cm} 
\onslide<2->{
\hfill
\begin{minipage}{0.7\textwidth}
\myfigure{1}{
\tikzinput{fig_determinants06} 
}  
\end{minipage}
}


  \item \pause Si $v_1$ et $v_2$ sont colinéaires alors
le parallélogramme est d'aire nulle

 \pause
Or le déterminant de deux vecteurs colinéaires est nul

  \item \pause Dans la suite on suppose que les vecteurs ne sont pas colinéaires
  
\end{itemize}
\noqed\vspace*{-4ex}
\end{proof}
\end{frame}

%%%%%%%%

\begin{frame}
\begin{proof}

\begin{itemize}
  \item Notons $v_1= \left(\begin{smallmatrix}a\\c\end{smallmatrix}\right)$ et 
$v_2= \left(\begin{smallmatrix}b\\d\end{smallmatrix}\right)$
  \item \pause Si $a\neq0$ alors $v'_2=v_2-\frac{b}{a}v_1 = \left(\begin{smallmatrix}0\\d-\frac{b}{a}c\end{smallmatrix}\right)$ est vertical

 \item \pause En remplaçant $v_2$ par $v_2'$ 
\begin{itemize}   
  \item \pause on ne change pas l'aire du parallélogramme
\myfigure{1}{
\tikzinput{fig_determinants06b} 
}
  \item \pause on ne change pas le déterminant
$$\det (v_1,v_2')= 
\det \begin{pmatrix}
a & 0 \\ b & d-\frac{b}{a}c
\end{pmatrix}
=ad-bc=\det(v_1,v_2)$$
\end{itemize}
\end{itemize}
\noqed\vspace*{-3ex}
\end{proof}
\end{frame}

\begin{frame}
\begin{proof}

\begin{itemize}
  \item $v'_1= \left(\begin{smallmatrix}a\\0\end{smallmatrix}\right)$ vecteur horizontal
  
  \item \pause En remplaçant $v_1$ par $v_1'$ 
\begin{itemize}   
  \item \pause on ne change pas l'aire du parallélogramme
  \myfigure{1}{
\tikzinput{fig_determinants06c} 
}
\vspace*{-1ex}
  \item \pause ni le déterminant $$\det(v_1',v_2')=
\det \begin{pmatrix}
a&0\\
0&d-\frac{b}{a}c
\end{pmatrix}
=ad-bc=\det(v_1,v_2)$$
\end{itemize}
  \item \pause On s'est donc ramené au  cas d'un rectangle aux côtés parallèles aux axes \qedhere
\end{itemize}
\vspace*{-2ex}
\end{proof}
\end{frame}



%%%%%%%%%%%%%%%%%%%%%%%%%%%%%%%%%%%%%%%%%%%%%%%%%%%%%%%%%%%%%%%%
\section{Mini-exercices}

\begin{frame}
\begin{miniexercice}
\begin{enumerate}
  \item Pour $A = \begin{pmatrix}1&2\\5&3\end{pmatrix}$
  et $B = \begin{pmatrix}-7&8\\-9&5\end{pmatrix}$
  calculer les déterminants de $A$, $B$, $A \times B$, $A+B$, $A^{-1}$,
  $\lambda A$, $A^T$.

  \item Mêmes questions pour $A = \begin{pmatrix}a&b\\c&d\end{pmatrix}$
  et $B = \begin{pmatrix}a'&0\\c'&d'\end{pmatrix}$.
  
  \item Mêmes questions pour 
  $A = \begin{pmatrix}2&0&1\\2&-1&2\\3&1&0\end{pmatrix}$
  et $B = \begin{pmatrix}1&2&3\\0&2&2\\0&0&3\end{pmatrix}$.
  
  \item Calculer l'aire du parallélogramme défini par les vecteurs
  
  $\left(\begin{smallmatrix}7\\3\end{smallmatrix}\right)$
  et $\left(\begin{smallmatrix}1\\4\end{smallmatrix}\right)$.
  
  \item Calculer le volume du parallélépipède défini 
  par les vecteurs 
  
  $\left(\begin{smallmatrix}2\\1\\1\end{smallmatrix}\right)$,
  $\left(\begin{smallmatrix}1\\1\\4\end{smallmatrix}\right)$ et
  $\left(\begin{smallmatrix}1\\3\\1\end{smallmatrix}\right)$.
\end{enumerate}
\end{miniexercice}
\end{frame}

\end{document}