
%%%%%%%%%%%%%%%%%% PREAMBULE %%%%%%%%%%%%%%%%%%


\documentclass[12pt]{article}

\usepackage{amsfonts,amsmath,amssymb,amsthm}
\usepackage[utf8]{inputenc}
\usepackage[T1]{fontenc}
\usepackage[francais]{babel}


% packages
\usepackage{amsfonts,amsmath,amssymb,amsthm}
\usepackage[utf8]{inputenc}
\usepackage[T1]{fontenc}
%\usepackage{lmodern}

\usepackage[francais]{babel}
\usepackage{fancybox}
\usepackage{graphicx}

\usepackage{float}

%\usepackage[usenames, x11names]{xcolor}
\usepackage{tikz}
\usepackage{datetime}

\usepackage{mathptmx}
%\usepackage{fouriernc}
%\usepackage{newcent}
\usepackage[mathcal,mathbf]{euler}

%\usepackage{palatino}
%\usepackage{newcent}


% Commande spéciale prompteur

%\usepackage{mathptmx}
%\usepackage[mathcal,mathbf]{euler}
%\usepackage{mathpple,multido}

\usepackage[a4paper]{geometry}
\geometry{top=2cm, bottom=2cm, left=1cm, right=1cm, marginparsep=1cm}

\newcommand{\change}{{\color{red}\rule{\textwidth}{1mm}\\}}

\newcounter{mydiapo}

\newcommand{\diapo}{\newpage
\hfill {\normalsize  Diapo \themydiapo \quad \texttt{[\jobname]}} \\
\stepcounter{mydiapo}}


%%%%%%% COULEURS %%%%%%%%%%

% Pour blanc sur noir :
%\pagecolor[rgb]{0.5,0.5,0.5}
% \pagecolor[rgb]{0,0,0}
% \color[rgb]{1,1,1}



%\DeclareFixedFont{\myfont}{U}{cmss}{bx}{n}{18pt}
\newcommand{\debuttexte}{
%%%%%%%%%%%%% FONTES %%%%%%%%%%%%%
\renewcommand{\baselinestretch}{1.5}
\usefont{U}{cmss}{bx}{n}
\bfseries

% Taille normale : commenter le reste !
%Taille Arnaud
%\fontsize{19}{19}\selectfont

% Taille Barbara
%\fontsize{21}{22}\selectfont

%Taille François
\fontsize{25}{30}\selectfont

%Taille Pascal
%\fontsize{25}{30}\selectfont

%Taille Laura
%\fontsize{30}{35}\selectfont


%\myfont
%\usefont{U}{cmss}{bx}{n}

%\Huge
%\addtolength{\parskip}{\baselineskip}
}


% \usepackage{hyperref}
% \hypersetup{colorlinks=true, linkcolor=blue, urlcolor=blue,
% pdftitle={Exo7 - Exercices de mathématiques}, pdfauthor={Exo7}}


%section
% \usepackage{sectsty}
% \allsectionsfont{\bf}
%\sectionfont{\color{Tomato3}\upshape\selectfont}
%\subsectionfont{\color{Tomato4}\upshape\selectfont}

%----- Ensembles : entiers, reels, complexes -----
\newcommand{\Nn}{\mathbb{N}} \newcommand{\N}{\mathbb{N}}
\newcommand{\Zz}{\mathbb{Z}} \newcommand{\Z}{\mathbb{Z}}
\newcommand{\Qq}{\mathbb{Q}} \newcommand{\Q}{\mathbb{Q}}
\newcommand{\Rr}{\mathbb{R}} \newcommand{\R}{\mathbb{R}}
\newcommand{\Cc}{\mathbb{C}} 
\newcommand{\Kk}{\mathbb{K}} \newcommand{\K}{\mathbb{K}}

%----- Modifications de symboles -----
\renewcommand{\epsilon}{\varepsilon}
\renewcommand{\Re}{\mathop{\text{Re}}\nolimits}
\renewcommand{\Im}{\mathop{\text{Im}}\nolimits}
%\newcommand{\llbracket}{\left[\kern-0.15em\left[}
%\newcommand{\rrbracket}{\right]\kern-0.15em\right]}

\renewcommand{\ge}{\geqslant}
\renewcommand{\geq}{\geqslant}
\renewcommand{\le}{\leqslant}
\renewcommand{\leq}{\leqslant}

%----- Fonctions usuelles -----
\newcommand{\ch}{\mathop{\mathrm{ch}}\nolimits}
\newcommand{\sh}{\mathop{\mathrm{sh}}\nolimits}
\renewcommand{\tanh}{\mathop{\mathrm{th}}\nolimits}
\newcommand{\cotan}{\mathop{\mathrm{cotan}}\nolimits}
\newcommand{\Arcsin}{\mathop{\mathrm{Arcsin}}\nolimits}
\newcommand{\Arccos}{\mathop{\mathrm{Arccos}}\nolimits}
\newcommand{\Arctan}{\mathop{\mathrm{Arctan}}\nolimits}
\newcommand{\Argsh}{\mathop{\mathrm{Argsh}}\nolimits}
\newcommand{\Argch}{\mathop{\mathrm{Argch}}\nolimits}
\newcommand{\Argth}{\mathop{\mathrm{Argth}}\nolimits}
\newcommand{\pgcd}{\mathop{\mathrm{pgcd}}\nolimits} 

\newcommand{\Card}{\mathop{\text{Card}}\nolimits}
\newcommand{\Ker}{\mathop{\text{Ker}}\nolimits}
\newcommand{\id}{\mathop{\text{id}}\nolimits}
\newcommand{\ii}{\mathrm{i}}
\newcommand{\dd}{\mathrm{d}}
\newcommand{\Vect}{\mathop{\text{Vect}}\nolimits}
\newcommand{\Mat}{\mathop{\mathrm{Mat}}\nolimits}
\newcommand{\rg}{\mathop{\text{rg}}\nolimits}
\newcommand{\tr}{\mathop{\text{tr}}\nolimits}
\newcommand{\ppcm}{\mathop{\text{ppcm}}\nolimits}

%----- Structure des exercices ------

\newtheoremstyle{styleexo}% name
{2ex}% Space above
{3ex}% Space below
{}% Body font
{}% Indent amount 1
{\bfseries} % Theorem head font
{}% Punctuation after theorem head
{\newline}% Space after theorem head 2
{}% Theorem head spec (can be left empty, meaning ‘normal’)

%\theoremstyle{styleexo}
\newtheorem{exo}{Exercice}
\newtheorem{ind}{Indications}
\newtheorem{cor}{Correction}


\newcommand{\exercice}[1]{} \newcommand{\finexercice}{}
%\newcommand{\exercice}[1]{{\tiny\texttt{#1}}\vspace{-2ex}} % pour afficher le numero absolu, l'auteur...
\newcommand{\enonce}{\begin{exo}} \newcommand{\finenonce}{\end{exo}}
\newcommand{\indication}{\begin{ind}} \newcommand{\finindication}{\end{ind}}
\newcommand{\correction}{\begin{cor}} \newcommand{\fincorrection}{\end{cor}}

\newcommand{\noindication}{\stepcounter{ind}}
\newcommand{\nocorrection}{\stepcounter{cor}}

\newcommand{\fiche}[1]{} \newcommand{\finfiche}{}
\newcommand{\titre}[1]{\centerline{\large \bf #1}}
\newcommand{\addcommand}[1]{}
\newcommand{\video}[1]{}

% Marge
\newcommand{\mymargin}[1]{\marginpar{{\small #1}}}



%----- Presentation ------
\setlength{\parindent}{0cm}

%\newcommand{\ExoSept}{\href{http://exo7.emath.fr}{\textbf{\textsf{Exo7}}}}

\definecolor{myred}{rgb}{0.93,0.26,0}
\definecolor{myorange}{rgb}{0.97,0.58,0}
\definecolor{myyellow}{rgb}{1,0.86,0}

\newcommand{\LogoExoSept}[1]{  % input : echelle
{\usefont{U}{cmss}{bx}{n}
\begin{tikzpicture}[scale=0.1*#1,transform shape]
  \fill[color=myorange] (0,0)--(4,0)--(4,-4)--(0,-4)--cycle;
  \fill[color=myred] (0,0)--(0,3)--(-3,3)--(-3,0)--cycle;
  \fill[color=myyellow] (4,0)--(7,4)--(3,7)--(0,3)--cycle;
  \node[scale=5] at (3.5,3.5) {Exo7};
\end{tikzpicture}}
}



\theoremstyle{definition}
%\newtheorem{proposition}{Proposition}
%\newtheorem{exemple}{Exemple}
%\newtheorem{theoreme}{Théorème}
\newtheorem{lemme}{Lemme}
\newtheorem{corollaire}{Corollaire}
%\newtheorem*{remarque*}{Remarque}
%\newtheorem*{miniexercice}{Mini-exercices}
%\newtheorem{definition}{Définition}




%definition d'un terme
\newcommand{\defi}[1]{{\color{myorange}\textbf{\emph{#1}}}}
\newcommand{\evidence}[1]{{\color{blue}\textbf{\emph{#1}}}}



 %----- Commandes divers ------

\newcommand{\codeinline}[1]{\texttt{#1}}

%%%%%%%%%%%%%%%%%%%%%%%%%%%%%%%%%%%%%%%%%%%%%%%%%%%%%%%%%%%%%
%%%%%%%%%%%%%%%%%%%%%%%%%%%%%%%%%%%%%%%%%%%%%%%%%%%%%%%%%%%%%



\begin{document}

\debuttexte


%%%%%%%%%%%%%%%%%%%%%%%%%%%%%%%%%%%%%%%%%%%%%%%%%%%%%%%%%%%
\diapo
\change
Nous allons voir plusieurs applications des déterminants.

\change
Nous commencerons par la méthode de Cramer pour résou\-dre des systèmes linéaires.

\change
Puis nous verrons comment les déterminants permettent de savoir si une famille de vecteurs forme une base.

\change
Grâce aux déterminants, nous définirons les mineurs d'une matrice, 

\change
et nous les utiliserons pour déterminer son rang.

\change
Enfin, nous en déduirons le rang d'une matrice transposée.

%%%%%%%%%%%%%%%%%%%%%%%%%%%%%%%%%%%%%%%%%%%%%%%%%%%%%%%%%%%
\diapo
Nous commençons par énoncer un théorème, appelé \defi{règle de Cramer}, qui donne une 
formule explicite pour la solution de systèmes d'équations linéaires 
ayant autant d'équa\-tions que d'inconnues. 

Considérons le système d'équations linéaires à $n$ équations et $n$ inconnues suivant.
% geste

\change

[petit x, grand X]

Ce système peut aussi s'écrire sous forme matricielle $AX=B$ où 

\change
$A$ est la matrice carrée des coefficients $a_{ij}$ du système linéaire, 

\change
$X$ est le vecteur colonne de taille $n$ dont les composantes $x_1$, \ldots, $x_n$ sont les inconnues du système, 

\change
et $B$ est le vecteur colonne de taille $n$ également, dont les composantes sont les seconds membres du système.

\change
Définissons la matrice $A_j \in M_{n}(\Kk)$ ainsi : $A_j$ est la matrice obtenue en remplaçant la $j$-ème colonne de $A$ par le second membre $B$. 

%%%%%%%%%%%%%%%%%%%%%%%%%%%%%%%%%%%%%%%%%%%%%%%%%%%%%%%%%%%
\diapo

La règle de Cramer va nous permettre de calculer la solution $X$ du système précédent dans le cas où le déterminant de $A $ est non nul, en fonction des déterminants des matrices $A$ et $A_j$, que l'on vient de définir. \\

Théorème : Soit $AX = B$ un système de $n$ équations  à $n$ inconnues. 

\change
Supposons que $\det A \neq 0$.

\change
Alors l'unique solution $(x_1,x_2,\ldots,x_n)$ du système est donnée par:\\

$
x_1 = \frac{\det A_1}{\det A} \qquad x_2 = \frac{\det A_2}{\det A} \qquad \ldots \qquad x_n = \frac{\det A_n}{\det A}.
$

% 
% %%%%%%%%%%%%%%%%%%%%%%%%%%%%%%%%%%%%%%%%%%%%%%%%%%%%%%%%%%%
% \diapo
% Démontrons le théorème précédent, c'est-à-dire que l'unique solution du système est donnée par $x_i = \frac{\det A_i}{\det A} $.
% 
% \change
% Nous avons supposé que $\det A \neq 0$. 
% 
% \change
% Donc $A$ est inversible. 
% 
% \change
% Alors $X = A^{-1} B$ est l'unique solution du système. 
% 
% \change
% D'autre part, nous avons vu que l'inverse de $A$ est donnée par $\frac{1}{\det A} C^T$ où $C$ est la comatrice de $A$.
% 
% \change
% Donc $X = \frac{1}{\det A} C^T B$.
% 
% \change
% En développant, on a que le vecteur $X$, de composantes $x_1, \ldots, x_n $ est égal à $ \frac{1}{\det A}$ fois la transposée de la comatrice, fois le second membre $b_1, \ldots, b_n $,
% 
% \change
% ce qui vaut $\frac{1}{\det A} $ multiplié par le vecteur colonne
% $
% \begin{pmatrix}
% C_{11} b_1 + \dots + C_{n1} b_n\\
% \vdots\\
% C_{1n}b_1  + \dots + C_{nn} b_n  
% \end{pmatrix}
% $
% 
% \change
% C'est-à-dire si on l'écrit composante par composante :
% \[
% x_1  =  \frac{C_{11} b_1 + \dots + C_{n1} b_n}{\det A}
% \quad\ldots\quad
% x_i  =  \frac{C_{1i}b_1 + \dots + C_{ni} b_n}{\det A} \quad\ldots\ \text{etc}
% \]
% 
% \change
% Mais $b_1 C_{1i} + \dots + b_n C_{ni}$ est exactement le développement en cofacteurs de 
% $\det A_i$ par rapport à sa $i$-ème colonne.  
% 
% \change
% Donc
% \[
% x_i = \frac{\det A_i}{\det A}\, \cdotp
% \]
% 

%%%%%%%%%%%%%%%%%%%%%%%%%%%%%%%%%%%%%%%%%%%%%%%%%%%%%%%%%%%
\diapo

Appliquons la règle de Cramer sur un exemple et résolvons le système suivant de 3 équations à 3 inconnues.
%geste

\change
On écrit ce système sous forme matricielle avec $A$ la matrice $3\times3$ des coefficients du système, et
$
B = \left(
\begin{array}{c} 6 \\ 30 \\ 8 \end{array}\right)
$
le second membre. Les matrices $A_j$ sont données par

\change
$A_1$ où on a remplacé la première colonne de $A$ par $6, 30, 8$, c'est-à-dire par $B$, les deux dernières colonnes étant inchangées

\change
de même pour $A_2$ en remplaçant la deuxième colonne de $A$ par $B$

\change
et pour $A_3$ en remplaçant la troisième colonne.

\change
On calcule  $\det A  =  44$,

\change
$\det A_1  =  -40$,

\change
$\det A_2  =  72$, (...)

\change
et enfin $\det A_3  =  152.$

\change
La solution de notre système est alors 
$
x_1 = \frac{\det A_1}{\det A} $ c'est-à-dire $= -\frac{40}{44} $

\change
ce qui après simplification donne $ -\frac{10}{11}$.

\change
De même $x_2 = \frac{\det A_2}{\det A} = \frac{72}{44} $

\change
ce qui donne $\frac{18}{11}$.

\change
Et enfin $x_3 = \frac{\det A_3}{\det A} = \frac{152}{44} $

\change
c-à-d $ \frac{38}{11}$. \\


Remarquons pour conclure que la méthode de Cramer n'est pas la méthode la plus efficace 
pour résoudre un système, mais elle est utile si le système contient des paramètres.


%%%%%%%%%%%%%%%%%%%%%%%%%%%%%%%%%%%%%%%%%%%%%%%%%%%%%%%%%%%
\diapo

Soit maintenant $E$ un $\Kk$-espace vectoriel de dimension $n$. Fixons une base $\mathcal{B}$ de $E$.

\change
On considère $n$ vecteurs $v_1,v_2,\ldots,v_n$ de $E$

\change
et on  veut décider s'ils forment une base de $E$.

\change
Pour cela, on écrit la matrice $A \in M_n(\Kk)$ dont la $j$-ème colonne est formée
des coordonnées du vecteur $v_j$ dans la base $\mathcal{B}$ 
(comme pour la matrice de passage). \\

La matrice $A$ est donc obtenue en juxtaposant les coordonnées des vecteurs $v_j$ dans la base $\mathcal{B}$.

\change
Le calcul de déterminant apporte la réponse à notre problème. \\

Théorème. Les vecteurs $(v_1,v_2,\ldots,v_n)$ forment une base de $E$ si et seulement si $\det A \neq 0.$


%%%%%%%%%%%%%%%%%%%%%%%%%%%%%%%%%%%%%%%%%%%%%%%%%%%%%%%%%%%
\diapo

La démonstration du théorème fait appel à des résultats du chapitre «Matrices et applications linéaires»,
 section «Rang d'une famille de vecteurs». Elle est immédiate.

\change
Les vecteurs $(v_1,v_2,\ldots,v_n)$ forment une base de $E$ si et seulement si leur rang est maximal, 
c'est-à-dire égal à $n$.

\change
Ce qui est équivalent à ce que la matrice $A$ soit elle-même de rang $n$.

\change
Mais cela signifie que $A$ est inversible,

\change
c-à-d que son déterminant est non nul.

\change
On a un corollaire du théorème précédent pour le cas où l'espace vectoriel $E$ est $\Rr^n$.\\

Considérons une famille de $n$ vecteurs de $\Rr^n$ .

\change
Elle forme une base si et seulement si le déterminant de la matrice dont les coefficients sont 
$a_{ij}$ n'est pas nul.


%%%%%%%%%%%%%%%%%%%%%%%%%%%%%%%%%%%%%%%%%%%%%%%%%%%%%%%%%%%
\diapo

Appliquons le corollaire précédent sur un exemple et déter\-minons pour quelles valeurs réelles de $a,b$ les trois vecteurs  de $\Rr^3$
$$
\begin{pmatrix}0\\a\\b\end{pmatrix} \quad
\begin{pmatrix}a\\b\\0\end{pmatrix} \quad
\begin{pmatrix}b\\0\\a\end{pmatrix}$$
forment une base de $\Rr^3$. 

\change
Pour répondre, il suffit de calculer le déterminant formé par ces trois vecteurs

\change
Il vaut $ - a^3 - b^3$.

\change
En conclusion: 

\change
si $a^3 \neq - b^3$ alors les trois vecteurs forment une base de $\Rr^3$.

\change
Si $a^3 = - b^3$ alors les trois vecteurs sont liés.

\change
Pour finir, vous pouvez montrer en exercice que $a^3=-b^3$ si et seulement si $a=-b$.


%%%%%%%%%%%%%%%%%%%%%%%%%%%%%%%%%%%%%%%%%%%%%%%%%%%%%%%%%%%
\diapo
Soit $A=(a_{ij}) \in M_{n,p}(\Kk)$ une matrice à $n$ lignes 
et $p$ colonnes à coefficients dans $\Kk$. 

\change
On se donne un entier  $k$, inférieur à la fois à $n$ et à $p$.

\change
On appelle \defi{mineur d'ordre $k$} le déterminant de toute matrice carrée de taille $k$ extraite 
de $A$

\change
 Une matrice carrée de taille $k$ extraite de $A$, est obtenue en supprimant $n-k$ lignes 
 et $p-k$ colonnes de $A$.

\change
Noter que $A$ n'a pas besoin d'être une matrice carrée, pas contre, les déterminant que l'on calcule 
sont bien sûr ceux de sous-matrices carrées.

%%%%%%%%%%%%%%%%%%%%%%%%%%%%%%%%%%%%%%%%%%%%%%%%%%%%%%%%%%%
\diapo
Calculons quelques mineurs de la matrice $A$ suivante.

\change
Un mineur d'ordre $1$ est simplement un coefficient de la matrice $A$.

\change 
Un mineur d'ordre $2$ est le déterminant d'une matrice $2\times 2$ extraite de $A$.

\change
Par exemple en supprimant la ligne 2 et les colonnes 1~et~3, 

\change
on obtient la matrice extraite 
$\begin{pmatrix}2&4\\1&5\end{pmatrix}$.
  
\change  
Donc un des mineurs d'ordre $2$ de $A$ est le déterminant de cette matrice, qui vaut $ 6$.

%%%%%%%%%%%%%%%%%%%%%%%%%%%%%%%%%%%%%%%%%%%%%%%%%%%%%%%%%%%
\diapo
Poursuivons nos calculs de mineurs sur la même matrice~$A$.
  
\change
Un mineur d'ordre $3$ est le déterminant d'une matrice $3\times 3$ extraite de $A$.

\change
Par exemple, en supprimant la deuxième colonne de $A$, 

\change
on obtient ce mineur, qui vaut $-28$.
  
\change
Il n'y a pas de mineur d'ordre $4$ (car la matrice n'a que $3$ lignes).



%%%%%%%%%%%%%%%%%%%%%%%%%%%%%%%%%%%%%%%%%%%%%%%%%%%%%%%%%%%
\diapo
Rappelons la définition du rang d'une matrice. \\

Le rang d'une matrice est la dimension de l'espace vectoriel  engendré par les vecteurs colonnes. C'est donc le maximum de vecteurs colonnes linéairement indépendants. \\

Les mineurs nous fournissent un moyen de déterminer le rang d'une matrice.

\change
On a en effet le théorème suivant. Le rang d'une matrice $A$ est le plus grand entier $r$ 
tel qu'il existe un mineur d'ordre~$r$ extrait de $A$ qui soit non nul.


%%%%%%%%%%%%%%%%%%%%%%%%%%%%%%%%%%%%%%%%%%%%%%%%%%%%%%%%%%%
\diapo
Soit $\alpha$ un paramètre réel. Calculons le rang
de cette matrice $A$ qui dépend de $\alpha$.

\change
Clairement, le rang ne peut pas être égal à $4$, puisque $4$ vecteurs de $\Rr^3$ ne sauraient être indépendants.\\

On obtient les mineurs d'ordre $3$ de $A$ en supprimant une colonne. 

\change
Calculons le mineur d'ordre $3$ obtenu en supprimant la première colonne, 

\change
c'est-à-dire ce déterminant

\change
On le développe par rapport à \emph{sa} première colonne,
  
\change 
et on obtient qu'il vaut  $\alpha - 2$.

\change
Par conséquent, si $\alpha \neq 2$, le mineur précédent est non nul et le rang de la matrice $A$ est $3$.\\

Il nous reste encore à étudier le cas où $\alpha$ vaut $2$.


%%%%%%%%%%%%%%%%%%%%%%%%%%%%%%%%%%%%%%%%%%%%%%%%%%%%%%%%%%%
\diapo

Si $\alpha =2$, la matrice $A$ prend cette forme
% geste

\change
on vérifie que les $4$ mineurs d'ordre $3$ de $A$ sont nuls. Ils s'obtiennent chacun en supprimant une colonne de $A$ :

\change
la première

\change
la deuxième

\change
la troisième

\change
et enfin la quatrième.

\change
Par le calcul, on trouve à chaque fois $0$.

\change
Donc dans ce cas,  $A$ est de rang inférieur ou égal à $2$.

\change
Cherchons si $A$ admet un mineur d'ordre $2$ non nul. En supprimant $L_3$, $C_3$, $C_4$ dans $A$, 

\change
on obtient  ce mineur d'ordre $2$

\change
qui vaut $1$ : il est donc non nul.

\change
Donc si $\alpha = 2$ , le rang de $A$ est $2$.


%%%%%%%%%%%%%%%%%%%%%%%%%%%%%%%%%%%%%%%%%%%%%%%%%%%%%%%%%%%
\diapo

On termine à la fois cette leçon et le chapitre sur les déterminants par la proposition suivante:\\

Le rang de $A$ est égal au rang de sa transposée $A^T$.


%%%%%%%%%%%%%%%%%%%%%%%%%%%%%%%%%%%%%%%%%%%%%%%%%%%%%%%%%%%
\diapo

Voici la dernière série d'exercices sur les déterminants : entraînez-vous bien, pour assimiler cette notion qui est fondamentale en mathématiques.

\end{document}
