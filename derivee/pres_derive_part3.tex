
%%%%%%%%%%%%%%%%%% PREAMBULE %%%%%%%%%%%%%%%%%%

\documentclass[aspectratio=169,utf8]{beamer}
%\documentclass[aspectratio=169,handout]{beamer}

\usetheme{Boadilla}
%\usecolortheme{seahorse}
\usecolortheme[RGB={245,66,24}]{structure}
\useoutertheme{infolines}

% packages
\usepackage{amsfonts,amsmath,amssymb,amsthm}
\usepackage[utf8]{inputenc}
\usepackage[T1]{fontenc}
\usepackage{lmodern}

\usepackage[francais]{babel}
\usepackage{fancybox}
\usepackage{graphicx}

\usepackage{float}
\usepackage{xfrac}

%\usepackage[usenames, x11names]{xcolor}
\usepackage{tikz}
\usepackage{pgfplots}
\usepackage{datetime}



%-----  Package unités -----
\usepackage{siunitx}
\sisetup{locale = FR,detect-all,per-mode = symbol}

%\usepackage{mathptmx}
%\usepackage{fouriernc}
%\usepackage{newcent}
%\usepackage[mathcal,mathbf]{euler}

%\usepackage{palatino}
%\usepackage{newcent}
% \usepackage[mathcal,mathbf]{euler}



% \usepackage{hyperref}
% \hypersetup{colorlinks=true, linkcolor=blue, urlcolor=blue,
% pdftitle={Exo7 - Exercices de mathématiques}, pdfauthor={Exo7}}


%section
% \usepackage{sectsty}
% \allsectionsfont{\bf}
%\sectionfont{\color{Tomato3}\upshape\selectfont}
%\subsectionfont{\color{Tomato4}\upshape\selectfont}

%----- Ensembles : entiers, reels, complexes -----
\newcommand{\Nn}{\mathbb{N}} \newcommand{\N}{\mathbb{N}}
\newcommand{\Zz}{\mathbb{Z}} \newcommand{\Z}{\mathbb{Z}}
\newcommand{\Qq}{\mathbb{Q}} \newcommand{\Q}{\mathbb{Q}}
\newcommand{\Rr}{\mathbb{R}} \newcommand{\R}{\mathbb{R}}
\newcommand{\Cc}{\mathbb{C}} 
\newcommand{\Kk}{\mathbb{K}} \newcommand{\K}{\mathbb{K}}

%----- Modifications de symboles -----
\renewcommand{\epsilon}{\varepsilon}
\renewcommand{\Re}{\mathop{\text{Re}}\nolimits}
\renewcommand{\Im}{\mathop{\text{Im}}\nolimits}
%\newcommand{\llbracket}{\left[\kern-0.15em\left[}
%\newcommand{\rrbracket}{\right]\kern-0.15em\right]}

\renewcommand{\ge}{\geqslant}
\renewcommand{\geq}{\geqslant}
\renewcommand{\le}{\leqslant}
\renewcommand{\leq}{\leqslant}
\renewcommand{\epsilon}{\varepsilon}

%----- Fonctions usuelles -----
\newcommand{\ch}{\mathop{\text{ch}}\nolimits}
\newcommand{\sh}{\mathop{\text{sh}}\nolimits}
\renewcommand{\tanh}{\mathop{\text{th}}\nolimits}
\newcommand{\cotan}{\mathop{\text{cotan}}\nolimits}
\newcommand{\Arcsin}{\mathop{\text{arcsin}}\nolimits}
\newcommand{\Arccos}{\mathop{\text{arccos}}\nolimits}
\newcommand{\Arctan}{\mathop{\text{arctan}}\nolimits}
\newcommand{\Argsh}{\mathop{\text{argsh}}\nolimits}
\newcommand{\Argch}{\mathop{\text{argch}}\nolimits}
\newcommand{\Argth}{\mathop{\text{argth}}\nolimits}
\newcommand{\pgcd}{\mathop{\text{pgcd}}\nolimits} 


%----- Commandes divers ------
\newcommand{\ii}{\mathrm{i}}
\newcommand{\dd}{\text{d}}
\newcommand{\id}{\mathop{\text{id}}\nolimits}
\newcommand{\Ker}{\mathop{\text{Ker}}\nolimits}
\newcommand{\Card}{\mathop{\text{Card}}\nolimits}
\newcommand{\Vect}{\mathop{\text{Vect}}\nolimits}
\newcommand{\Mat}{\mathop{\text{Mat}}\nolimits}
\newcommand{\rg}{\mathop{\text{rg}}\nolimits}
\newcommand{\tr}{\mathop{\text{tr}}\nolimits}


%----- Structure des exercices ------

\newtheoremstyle{styleexo}% name
{2ex}% Space above
{3ex}% Space below
{}% Body font
{}% Indent amount 1
{\bfseries} % Theorem head font
{}% Punctuation after theorem head
{\newline}% Space after theorem head 2
{}% Theorem head spec (can be left empty, meaning ‘normal’)

%\theoremstyle{styleexo}
\newtheorem{exo}{Exercice}
\newtheorem{ind}{Indications}
\newtheorem{cor}{Correction}


\newcommand{\exercice}[1]{} \newcommand{\finexercice}{}
%\newcommand{\exercice}[1]{{\tiny\texttt{#1}}\vspace{-2ex}} % pour afficher le numero absolu, l'auteur...
\newcommand{\enonce}{\begin{exo}} \newcommand{\finenonce}{\end{exo}}
\newcommand{\indication}{\begin{ind}} \newcommand{\finindication}{\end{ind}}
\newcommand{\correction}{\begin{cor}} \newcommand{\fincorrection}{\end{cor}}

\newcommand{\noindication}{\stepcounter{ind}}
\newcommand{\nocorrection}{\stepcounter{cor}}

\newcommand{\fiche}[1]{} \newcommand{\finfiche}{}
\newcommand{\titre}[1]{\centerline{\large \bf #1}}
\newcommand{\addcommand}[1]{}
\newcommand{\video}[1]{}

% Marge
\newcommand{\mymargin}[1]{\marginpar{{\small #1}}}

\def\noqed{\renewcommand{\qedsymbol}{}}


%----- Presentation ------
\setlength{\parindent}{0cm}

%\newcommand{\ExoSept}{\href{http://exo7.emath.fr}{\textbf{\textsf{Exo7}}}}

\definecolor{myred}{rgb}{0.93,0.26,0}
\definecolor{myorange}{rgb}{0.97,0.58,0}
\definecolor{myyellow}{rgb}{1,0.86,0}

\newcommand{\LogoExoSept}[1]{  % input : echelle
{\usefont{U}{cmss}{bx}{n}
\begin{tikzpicture}[scale=0.1*#1,transform shape]
  \fill[color=myorange] (0,0)--(4,0)--(4,-4)--(0,-4)--cycle;
  \fill[color=myred] (0,0)--(0,3)--(-3,3)--(-3,0)--cycle;
  \fill[color=myyellow] (4,0)--(7,4)--(3,7)--(0,3)--cycle;
  \node[scale=5] at (3.5,3.5) {Exo7};
\end{tikzpicture}}
}


\newcommand{\debutmontitre}{
  \author{} \date{} 
  \thispagestyle{empty}
  \hspace*{-10ex}
  \begin{minipage}{\textwidth}
    \titlepage  
  \vspace*{-2.5cm}
  \begin{center}
    \LogoExoSept{2.5}
  \end{center}
  \end{minipage}

  \vspace*{-0cm}
  
  % Astuce pour que le background ne soit pas discrétisé lors de la conversion pdf -> png
\begin{tikzpicture}
        \fill[opacity=0,green!60!black] (0,0)--++(0,0)--++(0,0)--++(0,0)--cycle; 
\end{tikzpicture}

% toc S'affiche trop tot :
% \tableofcontents[hideallsubsections, pausesections]
}

\newcommand{\finmontitre}{
  \end{frame}
  \setcounter{framenumber}{0}
} % ne marche pas pour une raison obscure

%----- Commandes supplementaires ------

% \usepackage[landscape]{geometry}
% \geometry{top=1cm, bottom=3cm, left=2cm, right=10cm, marginparsep=1cm
% }
% \usepackage[a4paper]{geometry}
% \geometry{top=2cm, bottom=2cm, left=2cm, right=2cm, marginparsep=1cm
% }

%\usepackage{standalone}


% New command Arnaud -- november 2011
\setbeamersize{text margin left=24ex}
% si vous modifier cette valeur il faut aussi
% modifier le decalage du titre pour compenser
% (ex : ici =+10ex, titre =-5ex

\theoremstyle{definition}
%\newtheorem{proposition}{Proposition}
%\newtheorem{exemple}{Exemple}
%\newtheorem{theoreme}{Théorème}
%\newtheorem{lemme}{Lemme}
%\newtheorem{corollaire}{Corollaire}
%\newtheorem*{remarque*}{Remarque}
%\newtheorem*{miniexercice}{Mini-exercices}
%\newtheorem{definition}{Définition}

% Commande tikz
\usetikzlibrary{calc}
\usetikzlibrary{patterns,arrows}
\usetikzlibrary{matrix}
\usetikzlibrary{fadings} 

%definition d'un terme
\newcommand{\defi}[1]{{\color{myorange}\textbf{\emph{#1}}}}
\newcommand{\evidence}[1]{{\color{blue}\textbf{\emph{#1}}}}
\newcommand{\assertion}[1]{\emph{\og#1\fg}}  % pour chapitre logique
%\renewcommand{\contentsname}{Sommaire}
\renewcommand{\contentsname}{}
\setcounter{tocdepth}{2}



%------ Figures ------

\def\myscale{1} % par défaut 
\newcommand{\myfigure}[2]{  % entrée : echelle, fichier figure
\def\myscale{#1}
\begin{center}
\footnotesize
{#2}
\end{center}}


%------ Encadrement ------

\usepackage{fancybox}


\newcommand{\mybox}[1]{
\setlength{\fboxsep}{7pt}
\begin{center}
\shadowbox{#1}
\end{center}}

\newcommand{\myboxinline}[1]{
\setlength{\fboxsep}{5pt}
\raisebox{-10pt}{
\shadowbox{#1}
}
}

%--------------- Commande beamer---------------
\newcommand{\beameronly}[1]{#1} % permet de mettre des pause dans beamer pas dans poly


\setbeamertemplate{navigation symbols}{}
\setbeamertemplate{footline}  % tiré du fichier beamerouterinfolines.sty
{
  \leavevmode%
  \hbox{%
  \begin{beamercolorbox}[wd=.333333\paperwidth,ht=2.25ex,dp=1ex,center]{author in head/foot}%
    % \usebeamerfont{author in head/foot}\insertshortauthor%~~(\insertshortinstitute)
    \usebeamerfont{section in head/foot}{\bf\insertshorttitle}
  \end{beamercolorbox}%
  \begin{beamercolorbox}[wd=.333333\paperwidth,ht=2.25ex,dp=1ex,center]{title in head/foot}%
    \usebeamerfont{section in head/foot}{\bf\insertsectionhead}
  \end{beamercolorbox}%
  \begin{beamercolorbox}[wd=.333333\paperwidth,ht=2.25ex,dp=1ex,right]{date in head/foot}%
    % \usebeamerfont{date in head/foot}\insertshortdate{}\hspace*{2em}
    \insertframenumber{} / \inserttotalframenumber\hspace*{2ex} 
  \end{beamercolorbox}}%
  \vskip0pt%
}


\definecolor{mygrey}{rgb}{0.5,0.5,0.5}
\setlength{\parindent}{0cm}
%\DeclareTextFontCommand{\helvetica}{\fontfamily{phv}\selectfont}

% background beamer
\definecolor{couleurhaut}{rgb}{0.85,0.9,1}  % creme
\definecolor{couleurmilieu}{rgb}{1,1,1}  % vert pale
\definecolor{couleurbas}{rgb}{0.85,0.9,1}  % blanc
\setbeamertemplate{background canvas}[vertical shading]%
[top=couleurhaut,middle=couleurmilieu,midpoint=0.4,bottom=couleurbas] 
%[top=fondtitre!05,bottom=fondtitre!60]



\makeatletter
\setbeamertemplate{theorem begin}
{%
  \begin{\inserttheoremblockenv}
  {%
    \inserttheoremheadfont
    \inserttheoremname
    \inserttheoremnumber
    \ifx\inserttheoremaddition\@empty\else\ (\inserttheoremaddition)\fi%
    \inserttheorempunctuation
  }%
}
\setbeamertemplate{theorem end}{\end{\inserttheoremblockenv}}

\newenvironment{theoreme}[1][]{%
   \setbeamercolor{block title}{fg=structure,bg=structure!40}
   \setbeamercolor{block body}{fg=black,bg=structure!10}
   \begin{block}{{\bf Th\'eor\`eme }#1}
}{%
   \end{block}%
}


\newenvironment{proposition}[1][]{%
   \setbeamercolor{block title}{fg=structure,bg=structure!40}
   \setbeamercolor{block body}{fg=black,bg=structure!10}
   \begin{block}{{\bf Proposition }#1}
}{%
   \end{block}%
}

\newenvironment{corollaire}[1][]{%
   \setbeamercolor{block title}{fg=structure,bg=structure!40}
   \setbeamercolor{block body}{fg=black,bg=structure!10}
   \begin{block}{{\bf Corollaire }#1}
}{%
   \end{block}%
}

\newenvironment{mydefinition}[1][]{%
   \setbeamercolor{block title}{fg=structure,bg=structure!40}
   \setbeamercolor{block body}{fg=black,bg=structure!10}
   \begin{block}{{\bf Définition} #1}
}{%
   \end{block}%
}

\newenvironment{lemme}[0]{%
   \setbeamercolor{block title}{fg=structure,bg=structure!40}
   \setbeamercolor{block body}{fg=black,bg=structure!10}
   \begin{block}{\bf Lemme}
}{%
   \end{block}%
}

\newenvironment{remarque}[1][]{%
   \setbeamercolor{block title}{fg=black,bg=structure!20}
   \setbeamercolor{block body}{fg=black,bg=structure!5}
   \begin{block}{Remarque #1}
}{%
   \end{block}%
}


\newenvironment{exemple}[1][]{%
   \setbeamercolor{block title}{fg=black,bg=structure!20}
   \setbeamercolor{block body}{fg=black,bg=structure!5}
   \begin{block}{{\bf Exemple }#1}
}{%
   \end{block}%
}


\newenvironment{miniexercice}[0]{%
   \setbeamercolor{block title}{fg=structure,bg=structure!20}
   \setbeamercolor{block body}{fg=black,bg=structure!5}
   \begin{block}{Mini-exercices}
}{%
   \end{block}%
}


\newenvironment{tp}[0]{%
   \setbeamercolor{block title}{fg=structure,bg=structure!40}
   \setbeamercolor{block body}{fg=black,bg=structure!10}
   \begin{block}{\bf Travaux pratiques}
}{%
   \end{block}%
}
\newenvironment{exercicecours}[1][]{%
   \setbeamercolor{block title}{fg=structure,bg=structure!40}
   \setbeamercolor{block body}{fg=black,bg=structure!10}
   \begin{block}{{\bf Exercice }#1}
}{%
   \end{block}%
}
\newenvironment{algo}[1][]{%
   \setbeamercolor{block title}{fg=structure,bg=structure!40}
   \setbeamercolor{block body}{fg=black,bg=structure!10}
   \begin{block}{{\bf Algorithme}\hfill{\color{gray}\texttt{#1}}}
}{%
   \end{block}%
}


\setbeamertemplate{proof begin}{
   \setbeamercolor{block title}{fg=black,bg=structure!20}
   \setbeamercolor{block body}{fg=black,bg=structure!5}
   \begin{block}{{\footnotesize Démonstration}}
   \footnotesize
   \smallskip}
\setbeamertemplate{proof end}{%
   \end{block}}
\setbeamertemplate{qed symbol}{\openbox}


\makeatother
\usecolortheme[RGB={0,199,174}]{structure}

%%%%%%%%%%%%%%%%%%%%%%%%%%%%%%%%%%%%%%%%%%%%%%%%%%%%%%%%%%%%%
%%%%%%%%%%%%%%%%%%%%%%%%%%%%%%%%%%%%%%%%%%%%%%%%%%%%%%%%%%%%%

\begin{document}

\title{{\bf Dérivée d'une fonction}}
\subtitle{Extremum local, théorème de Rolle}

\begin{frame}
  
  \debutmontitre

  \pause

{\footnotesize
\hfill
\setbeamercovered{transparent=50}
\begin{minipage}{0.6\textwidth}
  \begin{itemize}
    \item<3-> Extremum local
    \item<4-> Théorème de Rolle
  \end{itemize}
\end{minipage}
}

\end{frame}

\setcounter{framenumber}{0}


%%%%%%%%%%%%%%%%%%%%%%%%%%%%%%%%%%%%%%%%%%%%%%%%%%%%%%%%%%%%%%%%


\section*{Extremum local}


\begin{frame}
Soit $f : I \to \Rr$
\begin{mydefinition}
\begin{itemize}
  \item $x_0$ est un \defi{point critique} de $f$ si $f'(x_0)=0$

\uncover<2->{  
  \item $f$ admet un \defi{maximum local en $x_0$} 
s'il existe un intervalle ouvert $J$ contenant $x_0$  tel que 
$\text{pour tout } x\in I \cap J \quad f(x) \le f(x_0)$
}

\uncover<4->{ 
  \item ... un \defi{minimum local en $x_0$} ... $f(x) \ge f(x_0)$
}

\uncover<6->{ 
  \item $f$ admet un \defi{extremum local en $x_0$} si $f$ admet un maximum 
local ou un minimum local en $x_0$
}

\uncover<7->{ 
  \item $x_0$ est un \defi{maximum global} si pour tout $x\in I \quad f(x) \le f(x_0)$
}

\end{itemize}
\end{mydefinition}
\vspace*{-2ex}
\myfigure{0.8}{
\tikzinput{fig_derive04pres} 
} 
\end{frame}

\begin{frame}
\begin{theoreme}
\label{th:extremum}
Soit $I$ un intervalle ouvert et $f : I \to \Rr$ une fonction
dérivable


 Si $f$ admet un maximum local (ou un minimum local)
en $x_0$ alors 
\centerline{$f'(x_0)=0$}
\end{theoreme}
\pause
\myfigure{1.5}{
\tikzinput{fig_derive05} 
}  

\end{frame}

\begin{frame}
\begin{exemple}[Extremums de $f_\lambda(x)= x^3+\lambda x$, $\lambda \in \Rr$]

\pause

$f_\lambda'(x) = 3x^2+\lambda$ \pause \qquad Si $x_0$ est un extremum local alors $f'_\lambda(x_0)=0$

 \pause
\begin{itemize}
  \item
  \begin{itemize}
     \item Si $\lambda>0$ alors $f'_\lambda(x)>0$
\pause
     \item Pas de points critiques donc pas non plus d'extremums
  \end{itemize}
\uncover<7->{
  \item
  \begin{itemize}
     \item Si $\lambda = 0$ alors $f_\lambda'(x)=3x^2$
}
\uncover<8->{
     \item Le seul point critique est $x_0=0$
}
\uncover<9->{
     \item Mais ce n'est ni un maximum local, ni un minimum local
}
\uncover<10->{
     \item En effet si $x<0$, $f_0(x)<0=f_0(0)$ et si $x>0$, $f_0(x)>0=f_0(0)$
}
  \end{itemize}
\end{itemize}
\pause
\vspace*{-4ex}

\myfigure{1.1}{
\uncover<6->{
\tikzinput{fig_derive12a}
}
\pause
\qquad \qquad 
\uncover<11->{
\tikzinput{fig_derive12b} 
}
}  
\vspace*{-2ex}
\end{exemple}
\end{frame}

\begin{frame}
\begin{exemple}[Extremums de $f_\lambda(x)= x^3+\lambda x$, $\lambda \in \Rr$]

\pause

\begin{itemize}
  \item
  \begin{itemize}
     \item Si $\lambda <0$ \pause alors $f'_\lambda(x)= 3x^2-|\lambda| \pause
= 3\big(x+\sqrt{\frac{|\lambda|}{3}}\big)\big(x-\sqrt{\frac{|\lambda|}{3}}\big)$
\pause
     \item Deux points critiques $x_1= -\sqrt{\frac{|\lambda|}{3}}$ et $x_2=+\sqrt{\frac{|\lambda|}{3}}$
\pause
     \item $f_\lambda'(x) > 0$ sur $]-\infty,x_1[$ et $]x_2,+\infty[$ et $f_\lambda'(x) < 0$ sur $]x_1,x_2[$
\pause
     \item $f_\lambda$ est croissante sur $]-\infty,x_1[$, puis décroissante sur $]x_1,x_2[$
\pause
     \item donc $x_1$ est un maximum local
\pause
     \item $f_\lambda$ est décroissante sur $]x_1,x_2[$ 
puis croissante sur $]x_2,+\infty[$ 
\pause
     \item donc $x_2$ est un minimum local
  \end{itemize}
\end{itemize}
\pause
\vspace*{-6ex}
\myfigure{1.1}{
\qquad \qquad \qquad \qquad \tikzinput{fig_derive12c} 
}  
\vspace*{-2ex}
\end{exemple}
\end{frame}

\begin{frame}
\begin{proof}
\centerline{\emph{Un extremum est un point critique}}

\pause

Supposons que $x_0$ soit un maximum local de $f$

\pause

soit $J$ l'intervalle ouvert contenant $x_0$  tel que 
$\text{pour tout } x\in I \cap J \quad f(x) \le f(x_0)$
\pause
\begin{itemize}
  \item Pour $x \in I\cap J$ tel que $x < x_0$ 
\pause
  \begin{itemize}
    \item $f(x)-f(x_0) \le 0$ et $x-x_0<0$
\pause
    \item donc $\frac{f(x)-f(x_0)}{x-x_0} \ge 0$
\pause
    \item donc $\lim_{x \to x_0^-} \frac{f(x)-f(x_0)}{x-x_0} \ge 0$
  \end{itemize}
\pause
  \item $x \in I\cap J$ tel que $x > x_0$
\pause
  \begin{itemize}
    \item $f(x)-f(x_0) \le 0$ et $x-x_0>0$
\pause
    \item donc $\frac{f(x)-f(x_0)}{x-x_0} \le 0$
\pause
    \item donc $\lim_{x \to x_0^+} \frac{f(x)-f(x_0)}{x-x_0} \le 0$
  \end{itemize}
\pause
  \item $\lim_{x \to x_0^-} \frac{f(x)-f(x_0)}{x-x_0} = f'(x_0)$ donc $f'(x_0) \ge 0$ 
\pause
  \item $\lim_{x \to x_0^+} \frac{f(x)-f(x_0)}{x-x_0} = f'(x_0)$ donc $f'(x_0) \le 0$ 
\pause
  \item Bilan $f'(x_0)=0$
\end{itemize}
\vspace*{-2ex}
\end{proof}
\end{frame}



\begin{frame}
\begin{enumerate}
  \item La réciproque du théorème est fausse
\pause

Exemple : $f : \Rr \to \Rr$, $f(x)= x^3$, $f'(0)=0$ 
\pause

mais $x_0=0$ n'est ni un maximum local ni un minimum local

\pause

  \item Cas d'un intervalle fermé : attention aux extrémités

\pause

Exemple : $f : [a,b] \to \Rr$  dérivable avec un extremum en $x_0$
\begin{itemize}
  \item $x_0= a$
  \item $x_0 =b$
  \item $x_0 \in ]a,b[$, dans ce cas $f'(x_0)=0$
\end{itemize}

\pause

  \item Pour déterminer $\max_{[a,b]} f$ et $\min_{[a,b]} f$ 
 il faut comparer les valeurs de $f$ aux différents points critiques et en $a$ et en $b$

\pause
\myfigure{0.8}{
\tikzinput{fig_derive06} \quad
\pause
\tikzinput{fig_derive07} \quad
\pause
\tikzinput{fig_derive08} 
}  

\end{enumerate}
\end{frame}




%%%%%%%%%%%%%%%%%%%%%%%%%%%%%%%%%%%%%%%%%%%%%%%%%%%%%%%%%%%%%%%%


\section*{Théorème de Rolle}


\begin{frame}
\begin{theoreme}[de Rolle]
\label{th:rolle}
Soit $f:[a,b] \to \Rr$ telle que 

\begin{itemize}
  \item $f$ est continue sur $[a,b]$

  \item $f$ est dérivable sur $]a,b[$

  \item $f(a)=f(b)$
\end{itemize}

Alors il existe $c \in ]a,b[$  tel que $f'(c)=0$
\end{theoreme}
\pause
\myfigure{1.4}{
\tikzinput{fig_derive09} 
}  

\end{frame}

\begin{frame}
\begin{exemple}
Soit $P(X) = (X-\alpha_1)(X-\alpha_2)\cdots(X-\alpha_n)$
avec $\alpha_1< \alpha_2 < \cdots < \alpha_n$
\pause
\begin{enumerate}
  \item \emph{Montrons que $P'$ a $n-1$ racines distinctes}
\pause
  \begin{itemize}
    \item $P$ est une fonction continue et dérivable
\pause
    \item $P(\alpha_1)=0=P(\alpha_2)$ alors par le théorème de Rolle 
il existe $c_1 \in ]\alpha_1,\alpha_2[$ tel que $P'(c_1)=0$
\pause
    \item  pour $1 \le k \le n-1$, comme $P(\alpha_k)=0=P(\alpha_{k+1})$ alors
il existe $c_k \in ]\alpha_k,\alpha_{k+1}[$ tel que $P'(c_k)=0$
\pause
    \item $n-1$ racines de $P'$ : $c_1< c_2 < \cdots < c_{n-1}$
  \end{itemize}
\pause
  \item \emph{Montrons que $P+P'$ a $n-1$ racines distinctes}
  \begin{itemize}
\pause
    \item fonction auxiliaire $f(x)=P(x)\exp x$
\pause
    \item $f$ s'annule comme $P$ en $\alpha_1,\ldots,\alpha_n$
\pause
    \item $f'(x)=\big(P(x)+P'(x)\big) \exp x$
\pause
    \item Théorème de Rolle :  comme $f(\alpha_k)=0=f(\alpha_{k+1})$ alors
il existe $\gamma_k \in ]\alpha_k,\alpha_{k+1}[$ tel que $f'(\gamma_k)=0$ ($1 \le k \le n-1$)
\pause
    \item $n-1$ racines distinctes de $P+P'$ : $\gamma_1 < \gamma_2 < \cdots < \gamma_{n-1}$
  \end{itemize}
\end{enumerate}
\end{exemple}

\end{frame}

\begin{frame}

\begin{theoreme}[de Rolle]
\label{th:rolle}

\begin{minipage}{0.47\textwidth}
\begin{itemize}
  \item $f$ est continue sur $[a,b]$

  \item $f$ est dérivable sur $]a,b[$

  \item $f(a)=f(b)$
\end{itemize}  
\end{minipage}
\begin{minipage}{0.49\textwidth}
Il existe $c \in ]a,b[$  tel que $f'(c)=0$  
\end{minipage}
\end{theoreme}


\begin{proof}
\pause
\begin{itemize}
  \item Si $f$ est constante sur $[a,b]$ alors n'importe quel $c\in]a,b[$ convient
\pause
  \item Sinon il existe $x_0 \in [a,b]$ tel que $f(x_0) \neq f(a)$
\pause
  \item Supposons par exemple $f(x_0) > f(a)$
\pause
  \item $f$ est continue sur $[a,b]$, donc elle admet un maximum en un point $c\in[a,b]$
\pause
  \item  $f(c) \ge f(x_0) > f(a)$ donc $c \neq a$

\pause 
De même comme $f(a)=f(b)$ alors  $c\neq b$

\pause 
Ainsi $c\in ]a,b[$

\pause 
  \item En $c$, $f$ est dérivable et admet un maximum (local) donc $f'(c)=0$
\end{itemize}
\end{proof}
\end{frame}






%%%%%%%%%%%%%%%%%%%%%%%%%%%%%%%%%%%%%%%%%%%%%%%%%%%%%%%%%%%%%%%%
\section*{Mini-exercices}


\begin{frame}
\begin{miniexercice}
\begin{enumerate}
  \item Dessiner le graphe de fonctions vérifiant :  $f_1$ admet deux minimums locaux et un maximum local ;
$f_2$ admet un minimum local qui n'est pas global et un maximum local qui est global ; 
$f_3$ admet une infinité d'extremum locaux ; $f_4$ n'admet aucun extremum local.
  \item Calculer en quel point la fonction $f(x)=ax^2+bx+c$ admet un extremum local.
  \item Soit $f : [0,2] \to \Rr$ une fonction deux fois dérivable telle que $f(0)=f(1)=f(2)=0$.
Montrer qu'il existe $c_1,c_2$ tels que $f'(c_1)=0$ et $f'(c_2)=0$. Montrer qu'il existe
$c_3$ tel que $f''(c_3)=0$.
  \item Montrer que chacune des trois hypothèses du théorème de Rolle est nécessaire.
\end{enumerate}
\end{miniexercice}
\end{frame}


\end{document}