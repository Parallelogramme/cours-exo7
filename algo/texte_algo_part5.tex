
%%%%%%%%%%%%%%%%%% PREAMBULE %%%%%%%%%%%%%%%%%%


\documentclass[12pt]{article}

\usepackage{amsfonts,amsmath,amssymb,amsthm}
\usepackage[utf8]{inputenc}
\usepackage[T1]{fontenc}
\usepackage[francais]{babel}


% packages
\usepackage{amsfonts,amsmath,amssymb,amsthm}
\usepackage[utf8]{inputenc}
\usepackage[T1]{fontenc}
%\usepackage{lmodern}

\usepackage[francais]{babel}
\usepackage{fancybox}
\usepackage{graphicx}

\usepackage{float}

%\usepackage[usenames, x11names]{xcolor}
\usepackage{tikz}
\usepackage{datetime}

\usepackage{mathptmx}
%\usepackage{fouriernc}
%\usepackage{newcent}
\usepackage[mathcal,mathbf]{euler}

%\usepackage{palatino}
%\usepackage{newcent}


% Commande spéciale prompteur

%\usepackage{mathptmx}
%\usepackage[mathcal,mathbf]{euler}
%\usepackage{mathpple,multido}

\usepackage[a4paper]{geometry}
\geometry{top=2cm, bottom=2cm, left=1cm, right=1cm, marginparsep=1cm}

\newcommand{\change}{{\color{red}\rule{\textwidth}{1mm}\\}}

\newcounter{mydiapo}

\newcommand{\diapo}{\newpage
\hfill {\normalsize  Diapo \themydiapo \quad \texttt{[\jobname]}} \\
\stepcounter{mydiapo}}


%%%%%%% COULEURS %%%%%%%%%%

% Pour blanc sur noir :
%\pagecolor[rgb]{0.5,0.5,0.5}
% \pagecolor[rgb]{0,0,0}
% \color[rgb]{1,1,1}



%\DeclareFixedFont{\myfont}{U}{cmss}{bx}{n}{18pt}
\newcommand{\debuttexte}{
%%%%%%%%%%%%% FONTES %%%%%%%%%%%%%
\renewcommand{\baselinestretch}{1.5}
\usefont{U}{cmss}{bx}{n}
\bfseries

% Taille normale : commenter le reste !
%Taille Arnaud
%\fontsize{19}{19}\selectfont

% Taille Barbara
%\fontsize{21}{22}\selectfont

%Taille François
\fontsize{25}{30}\selectfont

%Taille Pascal
%\fontsize{25}{30}\selectfont

%Taille Laura
%\fontsize{30}{35}\selectfont


%\myfont
%\usefont{U}{cmss}{bx}{n}

%\Huge
%\addtolength{\parskip}{\baselineskip}
}


% \usepackage{hyperref}
% \hypersetup{colorlinks=true, linkcolor=blue, urlcolor=blue,
% pdftitle={Exo7 - Exercices de mathématiques}, pdfauthor={Exo7}}


%section
% \usepackage{sectsty}
% \allsectionsfont{\bf}
%\sectionfont{\color{Tomato3}\upshape\selectfont}
%\subsectionfont{\color{Tomato4}\upshape\selectfont}

%----- Ensembles : entiers, reels, complexes -----
\newcommand{\Nn}{\mathbb{N}} \newcommand{\N}{\mathbb{N}}
\newcommand{\Zz}{\mathbb{Z}} \newcommand{\Z}{\mathbb{Z}}
\newcommand{\Qq}{\mathbb{Q}} \newcommand{\Q}{\mathbb{Q}}
\newcommand{\Rr}{\mathbb{R}} \newcommand{\R}{\mathbb{R}}
\newcommand{\Cc}{\mathbb{C}} 
\newcommand{\Kk}{\mathbb{K}} \newcommand{\K}{\mathbb{K}}

%----- Modifications de symboles -----
\renewcommand{\epsilon}{\varepsilon}
\renewcommand{\Re}{\mathop{\text{Re}}\nolimits}
\renewcommand{\Im}{\mathop{\text{Im}}\nolimits}
%\newcommand{\llbracket}{\left[\kern-0.15em\left[}
%\newcommand{\rrbracket}{\right]\kern-0.15em\right]}

\renewcommand{\ge}{\geqslant}
\renewcommand{\geq}{\geqslant}
\renewcommand{\le}{\leqslant}
\renewcommand{\leq}{\leqslant}

%----- Fonctions usuelles -----
\newcommand{\ch}{\mathop{\mathrm{ch}}\nolimits}
\newcommand{\sh}{\mathop{\mathrm{sh}}\nolimits}
\renewcommand{\tanh}{\mathop{\mathrm{th}}\nolimits}
\newcommand{\cotan}{\mathop{\mathrm{cotan}}\nolimits}
\newcommand{\Arcsin}{\mathop{\mathrm{Arcsin}}\nolimits}
\newcommand{\Arccos}{\mathop{\mathrm{Arccos}}\nolimits}
\newcommand{\Arctan}{\mathop{\mathrm{Arctan}}\nolimits}
\newcommand{\Argsh}{\mathop{\mathrm{Argsh}}\nolimits}
\newcommand{\Argch}{\mathop{\mathrm{Argch}}\nolimits}
\newcommand{\Argth}{\mathop{\mathrm{Argth}}\nolimits}
\newcommand{\pgcd}{\mathop{\mathrm{pgcd}}\nolimits} 

\newcommand{\Card}{\mathop{\text{Card}}\nolimits}
\newcommand{\Ker}{\mathop{\text{Ker}}\nolimits}
\newcommand{\id}{\mathop{\text{id}}\nolimits}
\newcommand{\ii}{\mathrm{i}}
\newcommand{\dd}{\mathrm{d}}
\newcommand{\Vect}{\mathop{\text{Vect}}\nolimits}
\newcommand{\Mat}{\mathop{\mathrm{Mat}}\nolimits}
\newcommand{\rg}{\mathop{\text{rg}}\nolimits}
\newcommand{\tr}{\mathop{\text{tr}}\nolimits}
\newcommand{\ppcm}{\mathop{\text{ppcm}}\nolimits}

%----- Structure des exercices ------

\newtheoremstyle{styleexo}% name
{2ex}% Space above
{3ex}% Space below
{}% Body font
{}% Indent amount 1
{\bfseries} % Theorem head font
{}% Punctuation after theorem head
{\newline}% Space after theorem head 2
{}% Theorem head spec (can be left empty, meaning ‘normal’)

%\theoremstyle{styleexo}
\newtheorem{exo}{Exercice}
\newtheorem{ind}{Indications}
\newtheorem{cor}{Correction}


\newcommand{\exercice}[1]{} \newcommand{\finexercice}{}
%\newcommand{\exercice}[1]{{\tiny\texttt{#1}}\vspace{-2ex}} % pour afficher le numero absolu, l'auteur...
\newcommand{\enonce}{\begin{exo}} \newcommand{\finenonce}{\end{exo}}
\newcommand{\indication}{\begin{ind}} \newcommand{\finindication}{\end{ind}}
\newcommand{\correction}{\begin{cor}} \newcommand{\fincorrection}{\end{cor}}

\newcommand{\noindication}{\stepcounter{ind}}
\newcommand{\nocorrection}{\stepcounter{cor}}

\newcommand{\fiche}[1]{} \newcommand{\finfiche}{}
\newcommand{\titre}[1]{\centerline{\large \bf #1}}
\newcommand{\addcommand}[1]{}
\newcommand{\video}[1]{}

% Marge
\newcommand{\mymargin}[1]{\marginpar{{\small #1}}}



%----- Presentation ------
\setlength{\parindent}{0cm}

%\newcommand{\ExoSept}{\href{http://exo7.emath.fr}{\textbf{\textsf{Exo7}}}}

\definecolor{myred}{rgb}{0.93,0.26,0}
\definecolor{myorange}{rgb}{0.97,0.58,0}
\definecolor{myyellow}{rgb}{1,0.86,0}

\newcommand{\LogoExoSept}[1]{  % input : echelle
{\usefont{U}{cmss}{bx}{n}
\begin{tikzpicture}[scale=0.1*#1,transform shape]
  \fill[color=myorange] (0,0)--(4,0)--(4,-4)--(0,-4)--cycle;
  \fill[color=myred] (0,0)--(0,3)--(-3,3)--(-3,0)--cycle;
  \fill[color=myyellow] (4,0)--(7,4)--(3,7)--(0,3)--cycle;
  \node[scale=5] at (3.5,3.5) {Exo7};
\end{tikzpicture}}
}



\theoremstyle{definition}
%\newtheorem{proposition}{Proposition}
%\newtheorem{exemple}{Exemple}
%\newtheorem{theoreme}{Théorème}
\newtheorem{lemme}{Lemme}
\newtheorem{corollaire}{Corollaire}
%\newtheorem*{remarque*}{Remarque}
%\newtheorem*{miniexercice}{Mini-exercices}
%\newtheorem{definition}{Définition}




%definition d'un terme
\newcommand{\defi}[1]{{\color{myorange}\textbf{\emph{#1}}}}
\newcommand{\evidence}[1]{{\color{blue}\textbf{\emph{#1}}}}



 %----- Commandes divers ------

\newcommand{\codeinline}[1]{\texttt{#1}}

\newcommand{\codeinline}[1]{\texttt{#1}}

%%%%%%%%%%%%%%%%%%%%%%%%%%%%%%%%%%%%%%%%%%%%%%%%%%%%%%%%%%%%%
%%%%%%%%%%%%%%%%%%%%%%%%%%%%%%%%%%%%%%%%%%%%%%%%%%%%%%%%%%%%%


\begin{document}

\debuttexte

%%%%%%%%%%%%%%%%%%%%%%%%%%%%%%%%%%%%%%%%%%%%%%%%%%%%%%%%%%%
\diapo

\change

Nous allons présenter quelques algorithmes élémentaires en lien avec l'arithmétique.

\change

Nous en profitons pour présenter une façon complètement différente d'écrire des algorithmes :
les fonctions récursives.

\change

Nous écrirons l'algorithme d'Euclide

\change

Et différentes façons de trouver des nombres premiers.



%%%%%%%%%%%%%%%%%%%%%%%%%%%%%%%%%%%%%%%%%%%%%%%%%%%%%%%%%%%
\diapo

Voici un algorithme très classique :

\change

Voyons comment fonctionne cette boucle avec l'exemple de $n=5$.

\change

On initialise la variable \codeinline{produit} à $1$,


Pour notre exemple $n=5$ alors $i$ varie de $1$ à $5$.

\`A chaque étape on multiplie \codeinline{produit}
par $i$ et on affecte le résultat dans \codeinline{produit}. 

\change

Pour $i=1$ la variable produit devient
$1 \times 1 = 1$, 

\change

Puis produit devient $2 \times 1 = 2$, 

\change


$3 \times 2 = 6$, 

\change

$4\times 6 = 24$, 

\change

Pour le dernier indice $i=5$ on trouve
$5 \times 24 = 120$.

Et la boucle se termine.


\change

Vous avez bien sûr reconnu le calcul de $5!$




%%%%%%%%%%%%%%%%%%%%%%%%%%%%%%%%%%%%%%%%%%%%%%%%%%%%%%%%%%%
\diapo

\'Etudions un autre algorithme. Nous définissons une fonction factorielle qui dépend d'une variable $n$
ainsi :

Tout d'abord si $n$ égal $1$ alors la fonction renvoie $1$.

Sinon, $n$ est plus grand que $1$ et la fonction renvoie $n$ * notre fonction factorielle
évaluée en $n-1$.

Cela semble un peu bizarre au premier abord car factorielle (n-1) n'est pas connue,

l'ordinateur doit donc recommencer tout le processus avec cette fois $n-1$ à la place de $n$.

\change

Voyons cela pour $n=5$.

Pour $n=5$ la condition du "si"  n'est pas vérifiée donc on passe directement au "sinon"

\change

Donc \codeinline{factorielle(5)} renvoie comme résultat : \codeinline{5 * factorielle(4)}

On a plus ou moins progressé : le calcul n'est pas fini car on ne connaît pas encore \codeinline{factorielle(4)}
mais on s'est ramené à un calcul au rang précédent.

\change

Pour $4$ on a \codeinline{factorielle(4)}=\codeinline{4 * factorielle(3)}
Donc \codeinline{factorielle(5)= 5 * 4 * factorielle(3)}

\change


on itère :

\codeinline{factorielle(5) == 5 * 4 * 3 * factorielle(2)}

\change

\codeinline{factorielle(5) = 5 * 4 * 3 * 2 *  factorielle(1)}.

Pour \codeinline{factorielle(1)} la condition du \codeinline{if (n==1)} est vérifiée 

\change

et alors  \codeinline{factorielle(1)=1}.

\change

Le bilan est donc que \codeinline{factorielle(5) = 5 * 4 * 3 * 2 * 1}
c'est bien $5!$


%%%%%%%%%%%%%%%%%%%%%%%%%%%%%%%%%%%%%%%%%%%%%%%%%%%%%%%%%%%
\diapo

Une fonction qui lorsque elle s’exécute s'appelle elle-même est une \defi{fonction récursive}.

\change

Il y a une analogie très forte avec la récurrence.

\change

Par exemple on peut définir la suite des factorielles ainsi :

\change

$u_1 = 1 \quad \text{ et } \text{ si } n \ge 1 \quad u_{n} = n \times u_{n-1} .$

\change


Comme pour la récurrence une fonction récursive comporte une étape d'\evidence{initialisation}

\change

ici \codeinline{if (n==1): return 1} correspondant à $u_1=1$


\change

et une étape d'\evidence{hérédité} 

\change

ici \codeinline{return n * factorielle(n-1)} correspondant à $u_{n} = n \times u_{n-1}$). 





%%%%%%%%%%%%%%%%%%%%%%%%%%%%%%%%%%%%%%%%%%%%%%%%%%%%%%%%%%%
\diapo

Voyons un autre exemple.

La suite de Fibonacci est définie ainsi

on part de $1$ et encore $1$

le nombre suivant est la somme de deux précédents : c'est $2$.

Le suivant est la somme $2 + 3$ c'est $5$, puis $3+5=8$,...

\change

On peut définir cette suite par récurrence.

On initialise les formules avec deux conditions initiales $F_0 = 1, \ F_1 = 1$

la formule de récurrence est $F_{n} = F_{n-1} + F_{n-2}$ : le nombre $F_n$ est bien la somme des deux précédents.

\change

Cela se programme tel quel avec une fonction récursive.

On commence par les conditions initiales au rang $n=0$ et $n=1$ on
renvoie $F_n = 1$.

\change

Pour un rang plus grand on renvoie la somme des deux précédents :
c'est à dire que l'algorithme fait appelle à la fonction aux deux rangs précédents et fait la somme.

Bien sûr pour le calculs des rangs précédents l'algorithme va faire appel aux rang d'encore avant. Etc.


%%%%%%%%%%%%%%%%%%%%%%%%%%%%%%%%%%%%%%%%%%%%%%%%%%%%%%%%%%%
\diapo

L’algorithme d'Euclide est basé sur le principe suivant

$\text{si } b | a \text{ alors } \pgcd(a,b)=b$


sinon le $\pgcd(a,b)$ vaut le $\pgcd$ de $b$ avec 
le reste de la division euclidienne de $a$ par $b$.


\change

A vous de mettre en oeuvre cette formule pour calculer 
le pgcd à l'aide d'un algorithme récursif.

En déduire une estimation de la probabilité que deux entiers tirés
au hasard soit premiers entre eux.



%%%%%%%%%%%%%%%%%%%%%%%%%%%%%%%%%%%%%%%%%%%%%%%%%%%%%%%%%%%
\diapo

Voici le code pour l'algorithme d'Euclide récursif. 

On commence par la condition initiale :

si $b$ divise $a$ alors on s’arrête et le pgcd est $b$.

La condition $b$ divise $a$ s'écrit ici que $a$ modulo $b$ vaut $0$.

\change

Si $b$ ne divise pas $a$
alors on applique le critère de l'algorithme d'Euclide : 
le pgcd de $a$ et de $b$ vaut
le pgcd de $b$ avec le reste de la division euclidienne de $a$ par $b$.


Notez à quel point le code est succinct et épuré !


%%%%%%%%%%%%%%%%%%%%%%%%%%%%%%%%%%%%%%%%%%%%%%%%%%%%%%%%%%%
\diapo


Par définition deux entiers $a,b$ sont premiers entre eux ssi 
$\pgcd(a,b)=1$ :

Donc une fois que l'on sait calculer le pgcd on peut décider 
si deux entiers sont premiers entre eux ou pas.


\change

Voici le calcul pour la probabilité 
de deux entiers d'être premiers entre eux.


Je vous laisse lire les détails :

\change

Par exemple on souhaite calculer la probabilité que deux entiers compris entre $1$ et $1000$
soient premiers entre eux. On pose $n=1000$ et on fait faire $10 \, 000$ tirages.

Chaque itération de la boucle "tant que" correspond à un tirage.

On tire au hasard deux entiers $a$ et $b$ entre $1$ et $1000$.

Si $pgcd(a,b)=1$ alors $a$ et $b$ sont premiers entre eux

et on incrémente alors un compteur qui compte le nombre de cas favorable. 

\change

A la fin on estime la probabilité
mesurée par le quotient 

\codeinline{nbpremiers/nbtirages} qui vaut ici environ  $0,60\ldots$ 

(les décimales d'après dépendent des tirages).


\change

Lorsque $n$ tend vers $+\infty$ alors $p_n \to \frac{6}{\pi^2} = 0,607927\ldots$ et on dit souvent que :


\change


<<la probabilité que deux entiers tirés au hasard soient premiers entre eux est $\frac{6}{\pi^2}$.>>



%%%%%%%%%%%%%%%%%%%%%%%%%%%%%%%%%%%%%%%%%%%%%%%%%%%%%%%%%%%
\diapo


Pour conclure avec les algorithmes récursifs avant de passer à autre chose voici quelques remarques :

\change

Un des avantages des algorithmes récursifs est d'abord
qu'ils ont souvent un code très court,

\change

en plus ils sont particulièrement faciles à programmer lorsque l'on a une relation de récurrence.

\change

Ils ont cependant des inconvénient selon le langage ou la fonction programmée.

Il peut y avoir des problèmes de mémoire.

Par exemple pour calculer $5!$ l'ordinateur a besoin de stocker $4!$ pour lequel il a besoin de stocker $3!$,...



\change

Il est aussi important de bien réfléchir à la condition initiale (qui est en fait celle qui termine l'algorithme)
  et à la récurrence sous peine d'avoir une fonction qui boucle indéfiniment !
  
\change

Il n'existe pas des algorithmes récursifs pour tout (nous allons le voir tout de suite pour les nombres premiers) 
mais ils apparaissent   beaucoup dans les algorithmes de tris. 
Autre exemple : la dichotomie se programme très bien par une fonction récursive.



%%%%%%%%%%%%%%%%%%%%%%%%%%%%%%%%%%%%%%%%%%%%%%%%%%%%%%%%%%%
\diapo

On oublie maintenant les algorithme récursifs et on s'intéresse aux nombres premiers.

Voici votre travail à faire.

On commence par une façon élémentaire pour décider si un nombre donné est premier  ou pas, on teste
tous les diviseurs possibles.


Si l'on souhaite obtenir la liste de tous les nombres premiers jusq'à $1000$ par exemple, le crible d'Eratosthène
est simple et efficace.



Enfin voici une représentation fascinante des nombres premiers.
On place tous les entiers en spirale comme ceci : $1$, $2$, $3$, $4$, ...

A vous de colorez en rouge les entiers qui sont premiers ! 





%%%%%%%%%%%%%%%%%%%%%%%%%%%%%%%%%%%%%%%%%%%%%%%%%%%%%%%%%%%
\diapo

Voici l'algorithme pour tester si un entier $n$ est premier.

Déjà on décrète que $0$ et $1$ ne sont pas des nombres premiers.

\change

Ensuite commence l'itération, on commence par tester 
si $k=2$ divise notre entier $n$. Notre test est ici de calculer
$n $ modulo $k$, cela vaut $0$ ssi $k$ divise $n$.

Alors si c'est le cas inutile d'aller plus loin on a trouvé un diviseur de $n$.

$n$ n'est donc pas un nombre premier on s’arrête là et on renvoie la valeur ``Faux''.

\change

Si $k$ ne divise $n$ alors on passe au test du diviseur suivant $k+1$.

\change

Si à la fin on ne trouve aucun diviseur alors c'est que $n$ est premier et on renvoie la valeur ``Vrai''.


\change

Cet algorithme est basé sur la définition même d'un nombre premier.
Plus exactement un entier n'est *pas* un nombre premier s'il s'écrit $a\times b$
avec $a$ *et* $b$ $\ge 2$.

\change

Mais il est clair qu'alors $a \le \sqrt n$ ou $b \le \sqrt n$ 

(car si ce n'est pas le cas $ab$ serait $> n$)



\change

Donc il suffit de tester les diviseurs $2 \le k \le \sqrt n$.

\change

D'où la condition \codeinline{k*k <= n}

\change

Qui est préférable à  \codeinline{k <= sqrt(n)} 

\change


Nous avons utilisé un nouveau type de variable : un \defi{booléen} est une variable qui ne peut prendre que deux
états Vrai ou Faux. 

Ainsi \codeinline{est-premier(13)} renvoie Vrai, 
alors que \codeinline{est-premier(14)} renvoie Faux.



%%%%%%%%%%%%%%%%%%%%%%%%%%%%%%%%%%%%%%%%%%%%%%%%%%%%%%%%%%%
\diapo

Pour programmer le crible d'Eratosthène le plus dur est de trouver le bon codage de l'information.

Voici le code.


Ici on commence par faire un tableau contenant les entiers \codeinline{[0,1,2,3,4,5,6,7,8,9,10,11,12,13,...]}. 


\change

Pour signifier qu'un nombre n'est pas premier on remplace l'entier par \codeinline{0}.
Comme $1$ n'est pas un nombre premier : on le remplace par $0$.

\change

Puis on fait une boucle, on part de $k=2$ et on remplace tous les autres multiples de $2$ par $0$ :

On incrémente $k$, jusqu'à obtenir un entier $k$ non barré. 

Ici le premier nombre non barré après $2$ est $3$ c'est donc un nombre premier
(il n'a aucun diviseur autre que $1$ et lui-même car sinon il aurait été rayé).

On garde $3$ et remplace tous les autres multiples de $3$ par $0$.


\change

A la fin, cette procédure ne garde que les nombres premiers.

On reformate la liste pour se débarrasser des zéros.

\change

Voyons en exemple comme cela fonctionne :

Ici on commence par faire un tableau contenant les entiers 

On raye déjà $1$ qui n'est pas premier.

\change

Puis on part de $2$ et on raye tous les multiples de $2$.

\change

le premier nombre non barré après $2$ est $3$, on le conserve et on remplace tous les autres multiples de $3$
par $0$ (ici seul $9$ disparait),

le prochain nombre non nul est $5$, on conserve $5$ et on raye tous les multiples de $5$. etc.

\change

Par principe à la fin il ne reste que des nombres sans diviseurs stricts, ce sont les nombres premiers.



%%%%%%%%%%%%%%%%%%%%%%%%%%%%%%%%%%%%%%%%%%%%%%%%%%%%%%%%%%%
\diapo

Pour la spirale d'Ulam la seule difficulté est de placer les entiers sur une spirale, voici juste
le résultat.

On part du centre qui correspond à l'entier $1$ puis les entiers 
$2$, $3$, $4$, $5$ le long de la spirale.
Si l'entier est premier on le colore en rouge. 

Voici ici les entier de $1$ à $37$.


\change

Lorsque l'on va jusqu'à de grandes valeurs 
on obtient des motifs intrigants.

$1$ est ici,

les nombres premiers sont toujours en rouge,
et en blanc les nombres non premiers.




%%%%%%%%%%%%%%%%%%%%%%%%%%%%%%%%%%%%%%%%%%%%%%%%%%%%%%%%%%%
\diapo

Voici une liste des petits programmes à écrire

où les fonctions récursives sont mis en avant.



\end{document}